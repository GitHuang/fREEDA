%\documentclass[10pt,titlepage]{article}
% \usepackage{epsf} - Already called in top-level "updated_freeda_programmers.tex"
% Got rid of lgrind; used 'listings' package instead.  5/6/2003 FPH
%%\usepackage[nolineno]{lgrind}
%\usepackage{listings}
    % Set up a set of common parameters for listings package to format C++ code...
    % These settings are "too wide" for the standalone doc, but should be about
    % right once integrated into the overall fREEDA doc again.
%    \lstloadlanguages{[GNU]C++}
%        \lstset{language={[GNU]C++}, % p 12 of listings.dvi
%        numbers=left,           % p 29 of listings.dvi
%        numberstyle=\tiny,      % p 29 of listings.dvi
%        stepnumber=1,           % p 29 of listings.dvi
%        numbersep=5pt,          % p 29 of listings.dvi
%        showstringspaces=false  % p 29 of listings.dvi
%        xrightmargin=10mm,      % p 31 of listings.dvi
%        frame=lines,            % p 32 of listings.dvi
%        framexleftmargin=5mm,   % p 32 of listings.dvi
%        framexrightmargin=0mm,  % p 32 of listings.dvi change this to 15 when reintegrated?
%        texcl=false,            % p 35 of listings.dvi
%        mathescape=true,        % p 35 of listings.dvi
%        escapeinside={(*@}{@*)},  % p 49 of listings.dvi
%        basicstyle=\footnotesize}   % p 27 of listings.dvi
    % End of adds for listings package

% Remove this later when integrated in book. 5/6/2003 FPH
%% SUMMARY OF USEFUL MACROS
%
% 1. \marginpar[LeftText]{RightText} Standard Latex \marginpar
%
% 2. \mymarginpar{MarginText}{BodyText} Places MarginText in margin and
%     BodyText in main text opposite margin. Places lineas above and below
%     text.  Used in element and model catalogs and elsewhere.
%
% 3. \marginlabel{text} Places large text in margin and underlines. Used
%    to put element name in margin at top of page for continued
%    descriptions.  Does not automaticly put it at the top of a page
%    and so a \clearpage is required.
%
% 4. \marginid{text} Used to place a short identifier in the margin. Just one
%    word and justified to the outside edge.
%
% 5. \offset a standard indent.
%
% 6. \offsetparbox{} Places text in an offset parbox.  Use
%    \hspace*{\fill} \offsetparbox{text}        to insert an offset text.
%
% 7. \boxed{} similar to above but starts a new line and right justifies
%    offsetparbox.
%
% 8. \form
%
% 9. \example
%
%10. \begin{widelist}  .... \end{widelist} 
%    A widelist that has 1 inch wide labels that has additional left hand
%    margin of 0.6 inch.
%
%11. \spicethreeonly{} Text is inserted only if output is for spice3.
%
%12. \pspiceonly{} 
%    Text is inserted only if output is for pspice (superspice for the most
%    part is upwards compatible to pspice).
%
%13. \sspiceonly{} 
%    Text is inserted only if output is for superspice but not for pspice.

%14. \sym{}
%    Right justifies a symbol in a keyword table.
%
%15. Use the following wherever as then we can have a standard way to
%    report them.  Note that "\dc\ " is required to get a space after dc.
%    \dc
%    \ac
%    \SPICE
%
%16. \kwnote{}  This is a convenient way to include notes in a keyword table.
%    It right justifies a note in the description column and starts it on a
%    newline.
\newcommand{\kwnote}[1]{\newline\hspace*{\fill} #1}

%
%17. \kwversion{} This is a convenient way to indicate versions in a keyword
%    table. It does a \kwnote .  It should not printed when outputing for
%     specific versions.
% typical usage is \kwversion{\sspice ; \hspice} to produce the output
%            VERSIONS: SUPERSPICE; HSPICE
% typical usage is \kwversionNote{\sspice ; \hspice} to produce the output
%                 |                    VERSIONS: SUPERSPICE; HSPICE |
%for unifomity the recommended version order is:
%    \hspice \pspice \sspice \spicetwo \spicethree
\newcommand{\version}[1]{({\sc version: #1})}
\newcommand{\kwversion}[1]{\newline({\sc version: #1})}
\newcommand{\kwversionNote}[1]{\kwnote{({\sc version: #1})}}
\newcommand{\versions}[1]{({\sc versions: #1})}
\newcommand{\kwversions}[1]{\newline({\sc versions: #1})}
\newcommand{\kwversionsNote}[1]{\kwnote{({\sc versions: #1})}}

%


\newcommand{\mymargin}{1in}   % Use to set width of margin notes
\newcommand{\mymarginparsep}{0.1in}
\newcommand{\mymarginparsepplustext}{5.6in}
\newcommand{\mymarginplus}{1.1in}
\newcommand{\mymarginplustext}{6.6in}
\newcommand{\underlinehead}{\\[-0.1in] \rule{\mymarginplustext}{0.01in}}
\newcommand{\overlinefoot}{\rule{\mymarginplustext}{0.01in}\\}
% WIDEPARBOX uses margin space as well.
\newcommand{\wideparbox}[1]{\parbox[t]{\mymarginplustext}{#1}}
%
% HEADER AND FOOT
% Standard header and foot, but includes copyright notice.a
% To omit copyrite notice do not include copyrite.sty
%
\newcommand{\mycopyrite}{}
\def\ps@headings{\let\@mkboth\markboth
\def\@oddfoot{\wideparbox{\overlinefoot\rm\today \hfill \mycopyrite}}
\def\@evenfoot{\hspace*{-\mymarginplus}\wideparbox{\overlinefoot
\mycopyrite\hfill \today}}
\def\@evenhead{\hspace*{-\mymarginplus}\wideparbox{\rm
\thepage\hfill \sl \leftmark \underlinehead }}
\def\@oddhead{\wideparbox{\hbox{}\sl \rightmark \hfill
\rm\thepage \underlinehead}}\def\chaptermark##1{\markboth {\uppercase{\ifnum
\c@secnumdepth
>\m@ne
 \@chapapp\ \thechapter. \ \fi ##1}}{}}\def\sectionmark##1{\markright
 {\uppercase{\ifnum \c@secnumdepth >\z@
  \thesection. \ \fi ##1}}}}

\oddsidemargin 0in \topmargin 0.0in \textwidth 6in \textheight 9in
\evensidemargin 0.0in \headheight 0.18in \footskip 0.16in

%   EPSF.TEX macro file:
%   Written by Tomas Rokicki of Radical Eye Software, 29 Mar 1989.
%   Revised by Don Knuth, 3 Jan 1990.
%   Revised by Tomas Rokicki to accept bounding boxes with no
%      space after the colon, 18 Jul 1990.
%
%   TeX macros to include an Encapsulated PostScript graphic.
%   Works by finding the bounding box comment,
%   calculating the correct scale values, and inserting a vbox
%   of the appropriate size at the current position in the TeX document.
%
%   To use with the center environment of LaTeX, preface the \epsffile
%   call with a \leavevmode.  (LaTeX should probably supply this itself
%   for the center environment.)
%
%   To use, simply say
%   \input epsf           % somewhere early on in your TeX file
%   \epsfbox{filename.ps} % where you want to insert a vbox for a figure
%
%   Alternatively, you can type
%
%   \epsfbox[0 0 30 50]{filename.ps} % to supply your own BB
%
%   which will not read in the file, and will instead use the bounding
%   box you specify.
%
%   The effect will be to typeset the figure as a TeX box, at the
%   point of your \epsfbox command. By default, the graphic will have its
%   `natural' width (namely the width of its bounding box, as described
%   in filename.ps). The TeX box will have depth zero.
%
%   You can enlarge or reduce the figure by saying
%     \epsfxsize=<dimen> \epsfbox{filename.ps}
%   (or
%     \epsfysize=<dimen> \epsfbox{filename.ps})
%   instead. Then the width of the TeX box will be \epsfxsize and its
%   height will be scaled proportionately (or the height will be
%   \epsfysize and its width will be scaled proportiontally).  The
%   width (and height) is restored to zero after each use.
%
%   A more general facility for sizing is available by defining the
%   \epsfsize macro.    Normally you can redefine this macro
%   to do almost anything.  The first parameter is the natural x size of
%   the PostScript graphic, the second parameter is the natural y size
%   of the PostScript graphic.  It must return the xsize to use, or 0 if
%   natural scaling is to be used.  Common uses include:
%
%      \epsfxsize  % just leave the old value alone
%      0pt         % use the natural sizes
%      #1          % use the natural sizes
%      \hsize      % scale to full width
%      0.5#1       % scale to 50% of natural size
%      \ifnum#1>\hsize\hsize\else#1\fi  % smaller of natural, hsize
%
%   If you want TeX to report the size of the figure (as a message
%   on your terminal when it processes each figure), say `\epsfverbosetrue'.
%
\newread\epsffilein    % file to \read
\newif\ifepsffileok    % continue looking for the bounding box?
\newif\ifepsfbbfound   % success?
\newif\ifepsfverbose   % report what you're making?
\newdimen\epsfxsize    % horizontal size after scaling
\newdimen\epsfysize    % vertical size after scaling
\newdimen\epsftsize    % horizontal size before scaling
\newdimen\epsfrsize    % vertical size before scaling
\newdimen\epsftmp      % register for arithmetic manipulation
\newdimen\pspoints     % conversion factor
%
\pspoints=1bp          % Adobe points are `big'
\epsfxsize=0pt         % Default value, means `use natural size'
\epsfysize=0pt         % ditto
%
\def\epsfbox#1{\global\def\epsfllx{72}\global\def\epsflly{72}%
   \global\def\epsfurx{540}\global\def\epsfury{720}%
   \def\lbracket{[}\def\testit{#1}\ifx\testit\lbracket
   \let\next=\epsfgetlitbb\else\let\next=\epsfnormal\fi\next{#1}}%
%
\def\epsfgetlitbb#1#2 #3 #4 #5]#6{\epsfgrab #2 #3 #4 #5 .\\%
   \epsfsetgraph{#6}}%
%
\def\epsfnormal#1{\epsfgetbb{#1}\epsfsetgraph{#1}}%
%
\def\epsfgetbb#1{%
%
%   The first thing we need to do is to open the
%   PostScript file, if possible.
%
\openin\epsffilein=#1
\ifeof\epsffilein\errmessage{I couldn't open #1, will ignore it}\else
%
%   Okay, we got it. Now we'll scan lines until we find one that doesn't
%   start with %. We're looking for the bounding box comment.
%
   {\epsffileoktrue \chardef\other=12
    \def\do##1{\catcode`##1=\other}\dospecials \catcode`\ =10
    \loop
       \read\epsffilein to \epsffileline
       \ifeof\epsffilein\epsffileokfalse\else
%
%   We check to see if the first character is a % sign;
%   if not, we stop reading (unless the line was entirely blank);
%   if so, we look further and stop only if the line begins with
%   `%%BoundingBox:'.
%
          \expandafter\epsfaux\epsffileline:. \\%
       \fi
   \ifepsffileok\repeat
   \ifepsfbbfound\else
    \ifepsfverbose\message{No bounding box comment in #1; using defaults}\fi\fi
   }\closein\epsffilein\fi}%
%
%   Now we have to calculate the scale and offset values to use.
%   First we compute the natural sizes.
%
\def\epsfclipstring{}% do we clip or not?  If so,
\def\epsfclipon{\def\epsfclipstring{ clip}}%
\def\epsfclipoff{\def\epsfclipstring{}}%
%
\def\epsfsetgraph#1{%
   \epsfrsize=\epsfury\pspoints
   \advance\epsfrsize by-\epsflly\pspoints
   \epsftsize=\epsfurx\pspoints
   \advance\epsftsize by-\epsfllx\pspoints
%
%   If `epsfxsize' is 0, we default to the natural size of the picture.
%   Otherwise we scale the graph to be \epsfxsize wide.
%
   \epsfxsize\epsfsize\epsftsize\epsfrsize
   \ifnum\epsfxsize=0 \ifnum\epsfysize=0
      \epsfxsize=\epsftsize \epsfysize=\epsfrsize
      \epsfrsize=0pt
%
%   We have a sticky problem here:  TeX doesn't do floating point arithmetic!
%   Our goal is to compute y = rx/t. The following loop does this reasonably
%   fast, with an error of at most about 16 sp (about 1/4000 pt).
% 
     \else\epsftmp=\epsftsize \divide\epsftmp\epsfrsize
       \epsfxsize=\epsfysize \multiply\epsfxsize\epsftmp
       \multiply\epsftmp\epsfrsize \advance\epsftsize-\epsftmp
       \epsftmp=\epsfysize
       \loop \advance\epsftsize\epsftsize \divide\epsftmp 2
       \ifnum\epsftmp>0
          \ifnum\epsftsize<\epsfrsize\else
             \advance\epsftsize-\epsfrsize \advance\epsfxsize\epsftmp \fi
       \repeat
       \epsfrsize=0pt
     \fi
   \else \ifnum\epsfysize=0
     \epsftmp=\epsfrsize \divide\epsftmp\epsftsize
     \epsfysize=\epsfxsize \multiply\epsfysize\epsftmp   
     \multiply\epsftmp\epsftsize \advance\epsfrsize-\epsftmp
     \epsftmp=\epsfxsize
     \loop \advance\epsfrsize\epsfrsize \divide\epsftmp 2
     \ifnum\epsftmp>0
        \ifnum\epsfrsize<\epsftsize\else
           \advance\epsfrsize-\epsftsize \advance\epsfysize\epsftmp \fi
     \repeat
     \epsfrsize=0pt
    \else
     \epsfrsize=\epsfysize
    \fi
   \fi
%
%  Finally, we make the vbox and stick in a \special that dvips can parse.
%
   \ifepsfverbose\message{#1: width=\the\epsfxsize, height=\the\epsfysize}\fi
   \epsftmp=10\epsfxsize \divide\epsftmp\pspoints
   \vbox to\epsfysize{\vfil\hbox to\epsfxsize{%
      \ifnum\epsfrsize=0\relax
        \special{PSfile=#1 llx=\epsfllx\space lly=\epsflly\space
            urx=\epsfurx\space ury=\epsfury\space rwi=\number\epsftmp
            \epsfclipstring}%
      \else
        \epsfrsize=10\epsfysize \divide\epsfrsize\pspoints
        \special{PSfile=#1 llx=\epsfllx\space lly=\epsflly\space
            urx=\epsfurx\space ury=\epsfury\space rwi=\number\epsftmp\space
            rhi=\number\epsfrsize \epsfclipstring}%
      \fi
      \hfil}}%
\global\epsfxsize=0pt\global\epsfysize=0pt}%
%
%   We still need to define the tricky \epsfaux macro. This requires
%   a couple of magic constants for comparison purposes.
%
{\catcode`\%=12 \global\let\epsfpercent=%\global\def\epsfbblit{%BoundingBox}}%
%
%   So we're ready to check for `%BoundingBox:' and to grab the
%   values if they are found.
%
\long\def\epsfaux#1#2:#3\\{\ifx#1\epsfpercent
   \def\testit{#2}\ifx\testit\epsfbblit
      \epsfgrab #3 . . . \\%
      \epsffileokfalse
      \global\epsfbbfoundtrue
   \fi\else\ifx#1\par\else\epsffileokfalse\fi\fi}%
%
%   Here we grab the values and stuff them in the appropriate definitions.
%
\def\epsfempty{}%
\def\epsfgrab #1 #2 #3 #4 #5\\{%
\global\def\epsfllx{#1}\ifx\epsfllx\epsfempty
      \epsfgrab #2 #3 #4 #5 .\\\else
   \global\def\epsflly{#2}%
   \global\def\epsfurx{#3}\global\def\epsfury{#4}\fi}%
%
%   We default the epsfsize macro.
%
\def\epsfsize#1#2{\epsfxsize}
%
%   Finally, another definition for compatibility with older macros.
%
\let\epsffile=\epsfbox


%\newcommand{\fig}[1]{figures/#1}
%\newcommand{\pfig}[1]{\epsfbox{\fig{#1}}}
\newcommand{\newfig}[1]{\epsffile{\fig{#1}}}

\newcommand{\fdaelement}[1]{freeda_elements/#1}
\newcommand{\spiceelement}[1]{equivalent_spice_elements/#1}

% 5/11/03 Reordered $ to get rid of warning messages... FPH
% First two lines below are the 'originals.'  DO NOT MODIFY.
%\newcommand{\FREEDA}{{\Huge{\textsl{\textsf{f}}}${\mathsf{REEDA}}^{{\tiny{\textsf{TM}}}}$}}
%\newcommand{\FDA}{{\textsl{\textsf{f}}}${\mathsf{REEDA}}^{{\tiny{\textsf{TM}}}}$}
% Next two lines below are the mod experiments.  Okay to modify.
\newcommand{\FREEDA}{\Huge{\textsl{\textsf{f}}{\textup{\textsf{REEDA}}}{\textsuperscript{\tiny{\texttrademark}}}}}
\newcommand{\FDA}{{\textsl{\textsf{f}}}{\textup{\textsf{REEDA}}}{\textsuperscript{\tiny{\texttrademark}}}}

\newcommand{\notforsspice}[1]{#1}
\newcommand{\spicetwoonly}[1]{#1}
\newcommand{\spicethreeonly}[1]{#1}
\newcommand{\pspiceonly}[1]{#1}
\newcommand{\pspiceninetytwoonly}[1]{#1}
\newcommand{\sspiceonly}[1]{}
\newcommand{\fornutmeg}[1]{}
\newcommand{\sspicetwoonly}[1]{}

%%%%%%%%%%%%%%%%%%%%%%%%%%%%%%%%%%%%%%%%%%%%%%%%%%%%%%%%%%%%%%%%%%%%%%%%%%%%%%%%
\marginparwidth=\mymargin
\marginparsep=\mymarginparsep
\newcommand{\textplusmarginparwidth}{\textwidth+\marginparsep+\mymargin}
\newcommand{\X}{\\ \hline}    % line terminataion in keyword environment
\newcommand{\B}{{ \rm [}}     % begin optional parameter in \form{}
\newcommand{\E}{{\ \rm\hspace{-0.04in}] }}   % end optional parameter in \form{}
\newcommand{\expr}{{\sc Expressions supported}}
\newcommand{\reqd}{{\scriptsize REQUIRED}}
\newcommand{\omitted}{{\scriptsize OMITTED}}
\newcommand{\inferred}{{\scriptsize INFERRED}}
\newcommand{\para}{\newline{\scriptsize (PARASITIC)}}
\newcommand{\opt}{{\tiny  OPTIONAL}}

\newcommand{\Spice}{{\sc Spice}}
\newcommand{\spice}{{\sc Spice}}
\newcommand{\justspice}{{\sc Spice}}
\newcommand{\nutmeg}{{\sc NUTMEG}}
\newcommand{\probe}{{\sc Probe}}

\newcommand{\accusim}{{\sc AccuSim}}
\newcommand{\contec}{{\sc ContecSpice}}
\newcommand{\cdsspice}{{\sc CDS Spice}}
\newcommand{\hpimpulse}{{\sc HP Impulse}}
\newcommand{\hspice}{{\sc HSpice}}
\newcommand{\igspice}{{\sc IG\_SPICE}}
\newcommand{\isspice}{{\sc IsSpice}}
\newcommand{\mspice}{{\sc Microwave Spice}}
\newcommand{\pspice}{{\sc PSpice}}
\newcommand{\justpspice}{{\sc PSpice}}
\newcommand{\pspiceninetytwo}{{\sc PSpice92}}
\newcommand{\spectre}{{\sc Spectre}}
\newcommand{\spicetwo}{{\sc Spice2g6}}
\newcommand{\spicethree}{{\sc Spice3}}
\newcommand{\spiceplus}{{\sc SpicePlus}}
\newcommand{\sspice}{{\sc SuperSpice}}

\newcommand{\modelversion}[1]{& #1}
%%%%%%%%%%%%%%%%%%%%%%%%%%%%%%%%%%%%%%%%%%%%%%%%%%%%%%%%%%%%%%%%%%%%%%%%%%%%%%%%
% set up a counter for all occasions
%
\newcounter{count}

% set up new commands
%
\newcommand{\vshift}{\vspace{0.2in}}
% OFFSET
\newcommand{\offset}{\hspace*{0.45in} }
% OFFSETPARBOXWIDTH argument should be \textwith - offset
\newcommand{\offsetparbox}[1]{\parbox[t]{5in}{#1}}


\newcommand{\mymarginpar}[2]{
   \mbox{ }
   \marginpar[
   \parbox{\mymarginplustext}{
   \rule{\mymarginplustext}{.5mm} \\ \vspace*{-0.070in} \\
   {\huge #1} \hspace*{\fill} {\Large #2} \\
   \rule{\mymargin}{0.1in} \hfill
   \rule{\textwidth}{0.5mm}}
   ]{
   \hspace*{-\mymarginparsepplustext} \hspace*{-.1in}
   \parbox{\mymarginplustext}{
   \rule{\mymarginplustext}{.5mm} \\ \vspace*{-0.070in} \\
   {\Large #2} \hspace*{\fill} {\huge #1}\\
   \rule{\textwidth}{0.5mm} \hfill
   \rule{\mymargin}{0.1in}}
   } \\ \vspace{0.6in}\\
   }

\newcommand{\mymarginparx}[3]{
   \mbox{ }
   \marginpar[
   \parbox{\mymarginplustext}{
   \rule{\mymarginplustext}{.5mm} \\ \vspace*{-0.070in} \\
   {\huge #1} \hspace*{\fill}
   {\large #2} \hspace*{\fill}
 {\Large #3} \\
   \rule{\mymargin}{0.1in} \hfill
   \rule{\textwidth}{0.5mm}}
   ]{
   \hspace*{-\mymarginparsepplustext} \hspace*{-.1in}
   \parbox{\mymarginplustext}{
   \rule{\mymarginplustext}{.5mm} \\ \vspace*{-0.070in} \\
   {\Large #3} \hspace*{\fill}
   {\Large #2} \hspace*{\fill} {\huge #1}\\
   \rule{\textwidth}{0.5mm} \hfill
   \rule{\mymargin}{0.1in}}
   } \\ \vspace{0.6in}\\
   }

\newcommand{\marginid}[1]{
   \marginpar[ #1]{\hfill #1}}

\newcommand{\marginlabel}[1]{
   \marginpar[ {\huge #1}\\ \rule{\mymargin}{0.1in} ]{
              \hfill  {\huge #1}\\ \rule{\mymargin}{0.1in}}
              \vspace{-0.2in}}

%marginlabelfull is very wide
\newcommand{\marginlabelfull}[1]{
   \marginpar[ {\huge #1}\\ \rule{\mymargin}{0.1in} ]{
              \hspace*{-3in}\parbox{4in}{\hspace*{\fill}\huge #1}
              \\[0.05in] \rule{\mymargin}{0.1in}}
             \vspace*{0.2in}
          }

\newcommand{\element}[2]{\clearpage\rm\markright{#2}
\addcontentsline{toc}{section}{#1, #2}
\mymarginpar{#1}{#2}\label{#1element}\index{#1}\index{#2}}

% note: no labeling
\newcommand{\elementx}[3]{\clearpage\rm\markright{#3:#2}
\addcontentsline{toc}{section}{#1, #2: #3}
\mymarginparx{#1}{#2}{#3}\index{#1:#2}\index{#3:#2}}

% macros for sub elements
\newcommand{\subelement}[2]{\clearpage 
   \noindent\rule{\textwidth /2}{.5mm} \\[0.1in]
   {\large \bf #1} \hspace{0.2in} {\bf #2} \\
   \noindent\rule{\textwidth /2}{.5mm} \index{#1} \index{#2}}
% macros for models
\newcommand{\model}[2]{{
   \noindent\vspace{0.2in}\parbox{\textwidth}{
   \noindent\rule{\textwidth}{.5mm} \\[0.1in]
   {\large \bf #1 Model} \label{#1model} \hfill {\large #2} \\
   \noindent\rule{\textwidth}{.5mm} \index{#1} \index{#2}}}}
\newcommand{\modelx}[3]{{
   \noindent\vspace{0.2in}\parbox{\textwidth}{
   \noindent\rule{\textwidth}{.5mm} \\[0.1in]
   {\large \bf #1 Model} \hfill {\large #2} \hfill {\large #3} \\
   \noindent\rule{\textwidth}{.5mm} \index{#1} \index{#2}}}}


% macros for statements
\newcommand{\statement}[2]{\clearpage\rm\markright{#2}
   \addcontentsline{toc}{section}{#1, #2}
   \mymarginpar{#1}{#2} \label{#1statement}\index{#1}\index{#2}}

\newcommand{\statementx}[3]{\clearpage\rm\rm\markright{#3:#2}
   \addcontentsline{toc}{section}{#1, #2: #3}
   \mymarginparx{#1}{#2}{#3} \index{#1:#2} \index{#3:#2} \label{#1statement}}

% BOXED
\newcommand{\boxed}[1]{\noindent
\newline \vshift \hspace*{\fill} {\tt \offsetparbox{\tt #1}}
 \vshift}


%
% KEYWORD
%
\newcommand{\keywordtable}[1]{
        \sloppy
        \hyphenation{ca-pac-i-t-an-ce} 
        \begin{center}
    \sf
        \begin{tabular}[t]
        {|p{0.58in}|p{3.07in}|p{0.55in}|p{0.60in}|}
        \hline
        \multicolumn{1}{|c}{\bf Name} &
        \multicolumn{1}{|c}{\parbox{2.77in}{\bf Description}}  &
        \multicolumn{1}{|c}{\bf Units} &
        \multicolumn{1}{|c|}{\bf Default} \X
        #1
        \end{tabular}
        \end{center}
    }

\newcommand{\keywordtwotable}[2]{
        \sloppy
        \hyphenation{ca-pac-i-t-an-ce} 
        \begin{center}
    \sf
        \begin{tabular}[t]
        {|p{0.58in}|p{2.38in}|p{0.55in}|p{0.60in}|p{0.53in}|}
        \hline
        \multicolumn{1}{|c}{\bf Name} &
        \multicolumn{1}{|c}{\parbox{2.20in}{\bf Description}}  &
        \multicolumn{1}{|c}{\bf Units} &
        \multicolumn{1}{|c}{\bf Default} &
        \multicolumn{1}{|c|}{\bf #1} \X
        #2
        \end{tabular}
        \end{center}
    }

\newcommand{\kw}[2]{
     \samepage{
     \noindent {\sl #1} \vspace{-0.5in} \\
     \keywordtable{#2} }}

\newcommand{\kwtwo}[3]{
     \samepage{
     \noindent {\sl #1} \vspace{-0.4in} \\
     \keywordtwotable{#2}{#3} }}

\newcommand{\keyword}[1]{\kw{Keywords:}{#1}}
\newcommand{\keywordtwo}[2]{\kwtwo{Keywords:}{#1}{#2}}
\newcommand{\modelkeyword}[1]{\kw{Model Keywords}{#1}}
\newcommand{\modelkeywordtwo}[2]{\kwtwo{Model Keywords}{#1}{#2}}

\newcommand{\myline}{\\[-0.1in]
\noindent \rule{\textwidth}{0.01in} \newline}

\newcommand{\myThickLine}{\\[-0.1in]
\noindent \rule{\textwidth}{0.02in} \newline}


% SPICE 2G6 FORM
\newcommand{\spicetwoform}[1]{
\spicetwoonly{\samepage{\noindent{\sl\spicetwo\form{#1}}}}}

% PSPICE88 FORM
\newcommand{\pspiceform}[1]{
\pspiceonly{\samepage{\noindent{\sl\pspice}\form{#1}}}}

% PSPICE92 FORM
\newcommand{\pspiceninetytwoform}[1]{
\pspiceninetytwoonly{\samepage{\noindent{\sl\pspice92\form{#1}}}}}


% SPICE3E2 FORM
\newcommand{\spicethreeform}[1]{
\spicethreeonly{\samepage{\noindent{\sl\spicethree\form{#1}}}}}

% FORM
\newcommand{\form}[1]{\samepage{\noindent
 {\sl Form} \myline
% \hspace*{\fill} % For some reason \fill = 0 when \pspiceform{} is used?
\offset
\it  \offsetparbox{#1}}
\\[0.1in]}

% ELEMENT FORM
\newcommand{\elementform}[1]{\samepage{\noindent
 {\sl Element Form} \myline
% \hspace*{\fill} % For some reason \fill = 0 when \pspiceform{} is used?
\offset
\it  \offsetparbox{#1}}
\\[0.1in]}

% MODEL FORM
\newcommand{\modelform}[1]{\samepage{\noindent
 {\sl Model Form} \myline
% \hspace*{\fill} % For some reason \fill = 0 when \pspiceform{} is used?
\offset
\it  \offsetparbox{#1}}
\\[0.1in]}

% LIMITS
\newcommand{\mylimits}[1]{\samepage{\noindent
 {\sl Limits} \myline
 \hspace*{\fill} \it  \offsetparbox{#1}}
 \vshift}

% EXAMPLE
\newcommand{\example}[1]{\samepage{\noindent
{\sl Example} \myline
\offset \tt  \offsetparbox{#1}}
 \vshift}

% PSPICE88 EXAMPLE
\newcommand{\pspiceexample}[1]{\samepage{\noindent
{\sl \pspice\ Example} \myline
\offset \tt  \offsetparbox{#1}}
 \vshift}

% MODEL TYPES
\newcommand{\modeltype}[1]{\samepage{\noindent
{\sl Model Type} \myline
 \hspace*{\fill} \tt \offsetparbox{#1}}
 \\[0.1in]}

% MODEL TYPES
\newcommand{\modeltypes}[1]{\samepage{\noindent
{\sl Model Types:} \myline
 \hspace*{\fill} \tt \offsetparbox{\tt #1}}
 \vshift}

% OFFSET ENUMERATE
\newcommand{\offsetenumerate}[1]{
     \offset \hspace*{-0.1in} {\begin{enumerate} #1 \end{enumerate}}}

% NOTE
\newcommand{\note}[1]{
\vshift\samepage{\noindent {\sl Note}\myline\vspace{-0.24in}}
 \offsetenumerate{#1} }

% SPECIAL NOTE
\newcommand{\specialnote}[2]{
\vshift\samepage{\noindent {\sl #1}\myline\vspace{-0.24in}}\\#2}

\newcommand{\dc}{\mbox{\tt DC}}
\newcommand{\ac}{\mbox{\tt AC}}
\newcommand{\SPICE}{\mbox{\tt SPICE}}
\newcommand{\m}[1]{{\bf #1}}                           % matrix command  \m{}

%
% print nodes in \tt and enclose in a circle use outside
\newcommand{\node}[1]{\: \mbox{\tt #1} \!\!\!\! \bigcirc }
%
% set up environment for example
%
\newcounter{excount}
\newcounter{dummy}
\newenvironment{eg}{\vspace{0.1in}\noindent\rule{\textwidth}{.5mm}
   \begin{list}
   {{\addtocounter{excount}{1}
   \em Example\/ \arabic{chapter}.\arabic{excount}\/}:}
   {\usecounter{dummy}
   \setlength{\rightmargin}{\leftmargin}}
   }{\end{list} \rule{\textwidth}{.5mm}\vspace{0.1in}}
%
% set up environment for block
% currently this draws a horizontal line at the start of block and another
% at the end of block.
%
\newenvironment{block}{\vspace{0.1in}\noindent\rule{\textwidth}{.5mm}
   }{\rule{\textwidth}{.5mm}\vspace{0.1in}}
%

%
% Macros for element summaries
%
% macros for element summary
\newcommand{\summaryelement}[2]{
   \vspace{0.4in}
   \mymarginpar{#1}{#2} 
   \addcontentsline{toc}{section}{#1, #2}
   \vspace{-0.6in} \\
   \noindent Full description on page \pageref{#1element}. \vshift\\
   }

% Summary MODEL TYPE
\newcommand{\summarymodeltype}[1]{\samepage{\noindent
{\sl Model Type} \myline
 \hspace*{\fill} \tt \offsetparbox{\tt #1
 \hfill Summary on \pageref{#1summary} \index{#1}}}
 }

% Summary MODEL TYPES
\newcommand{\summarymodeltypes}[1]{\samepage{\noindent
{\sl Model Types} \myline
 \hspace*{\fill} \tt \offsetparbox{\tt #1
 \hfill Summary on \pageref{#1summary} \index{#1}}}
 }

%
% macros for model summary
\newcommand{\summarymodel}[2]{\clearpage
   \addcontentsline{toc}{section}{#1, #2}
   \mymarginpar{#1}{#2} \label{#1summary}
   \index{#1} \index{#2}
   \noindent Full description on page \pageref{#1model} \\[0.1in]
   }



%
% set up wide descriptive list
%
\newenvironment{widelist}
    {\begin{list}{}{\setlength{\rightmargin}{0in} \setlength{\itemsep}{0.1in}
    \setlength{\labelwidth}{0.95in} \setlength{\labelsep}{0.1in}
\setlength{\listparindent}{0in} \setlength{\parsep}{0in}
    \setlength{\leftmargin}{1.0in}}
    }{\end{list}}

\newcommand{\STAR}{\hspace*{\fill} * \hspace*{\fill}}

\newcommand{\sym}[1]{\hspace*{\fill} ($#1$)}

%
% The thebibliography environment is redefined so the the word References is
% not output
%\def\thebibliography#1{\list
% {[\arabic{enumi}]}{\settowidth\labelwidth{[#1]}\leftmargin\labelwidth
% \advance\leftmargin\labelsep
% \usecounter{enumi}}
% \def\newblock{\hskip .11em plus .33em minus -.07em}
% \sloppy\clubpenalty4000\widowpenalty4000
% \sfcode`\.=1000\relax}
%\let\endthebibliography=\endlist
% END thebibliography environment redefinition



\newcommand{\eskipv}[1]{\clearpage\marginlabel{#1}}
\newcommand{\eskip}[1]{\vspace*{\fill}\clearpage\marginlabel{#1}}
\newcommand{\eskipnv}[1]{\newpage\marginlabel{#1}}
\newcommand{\eskipn}[1]{\vspace*{\fill}\newpage\marginlabel{#1}}

%marginlabel is very wide
\newcommand{\eskipfullv}[1]{\clearpage\marginlabelfull{#1}}
\newcommand{\eskipfull}[1]{\vspace*{\fill}\clearpage\marginlabelfull{#1}}
\newcommand{\eskipfullnv}[1]{\newpage\marginlabelfull{#1}}
\newcommand{\eskipfulln}[1]{\vspace*{\fill}\newpage\marginlabelfull{#1}}

\newcommand{\mycontentsline}[5]{\parbox{#1}{#2}#3
\hspace{0.1in}\dotfill\parbox{0.3in}{\hfill\pageref{#4#5}}\\[0.1in]}
\newcommand{\mysline}[2]{\mycontentsline{1.2in}{#1}{#2}{#1}{statement}}
\newcommand{\mymline}[2]{\mycontentsline{0.7in}{#1}{#2}{#1}{model}}
\newcommand{\myeline}[2]{\mycontentsline{0.7in}{#1}{#2}{#1}{element}}

\newcommand{\myincontentsline}[5]{\vspace{0.05in}\noindent\parbox{#1}{#2}#3
\hspace{0.1in}
\dotfill\parbox{0.7in}{\hfill Page \pageref{#4#5}}\\[0.05in]\noindent}
\newcommand{\myinsline}[2]{\myincontentsline{1.2in}{#1}{#2}{#1}{statement}}
\newcommand{\myinmline}[2]{\myincontentsline{0.5in}{#1}{#2}{#1}{model}}
\newcommand{\myineline}[2]{\myincontentsline{0.5in}{#1}{#2}{#1}{element}}

%
%
% The following a symbols that could used alot.
\newcommand{\ms}[1]{\mbox{\scriptsize #1}}
\newcommand{\AF}{A_F}
\newcommand{\CBD}{C'_{BD}}
\newcommand{\CBS}{C'_{BS}}
\newcommand{\CGBO}{C_{GBO}}
\newcommand{\CGDO}{C_{GDO}}
\newcommand{\CGSO}{C_{GSO}}
\newcommand{\CJ}{C_J}
\newcommand{\CJSW}{C_{J,\ms{SW}}}
\newcommand{\DELTA}{\delta}
\newcommand{\ETA}{\eta}
\newcommand{\FC}{F_C}
\newcommand{\GAMMA}{\gamma}
\newcommand{\IS}{I_S}
\newcommand{\JS}{J_S}
\newcommand{\KAPPA}{\kappa}
\newcommand{\KF}{K_F}
\newcommand{\KP}{K_P}
\newcommand{\LAMBDA}{\lambda}
\newcommand{\LD}{X_{JL}}
\newcommand{\LEVEL}{M_J}
\newcommand{\MJ}{M_J}
\newcommand{\MJSW}{M_{J,\ms{SW}}}
\newcommand{\NSUB}{N_B}
\newcommand{\NSS}{N_{\ms{SS}}}
\newcommand{\NFS}{N_{\ms{FS}}}
\newcommand{\NEFF}{N_{\ms{EFF}}}
\newcommand{\PB}{\phi_J}
\newcommand{\PHI}{2\phi_B}
\newcommand{\RD}{R_D}
\newcommand{\RS}{R_S}
\newcommand{\RSH}{R_{\ms{SH}}}
\newcommand{\THETA}{\theta}
\newcommand{\TOX}{T_{OX}}
\newcommand{\TPG}{T_{\ms{PG}}}
\newcommand{\UCRIT}{U_C}
\newcommand{\UEXP}{U_{\ms{EXP}}}
\newcommand{\UO}{\mu_0}
\newcommand{\UTRA}{U_{\ms{TRA}}}
\newcommand{\VMAX}{V_{\ms{MAX}}}
\newcommand{\VTZERO}{V_{T0}}
\newcommand{\VTO}{V_{T0}}
\newcommand{\XJ}{X_J}
\newcommand{\Length}{L} %  \L already used
\newcommand{\N}{N}
\newcommand{\PBSW}{\phi_{J,{\ms{SW}}}}
\newcommand{\RB}{R_B}
\newcommand{\RG}{R_B}
\newcommand{\RDS}{R_{DS}}
\newcommand{\TT}{\tau_T}
\newcommand{\W}{W}
\newcommand{\WD}{W_D}
\newcommand{\XQC}{X_{QC}}
\newcommand{\JSSW}{J_{S,{\ms{SW}}}}
\newcommand{\DL}{\Delta_L}
\newcommand{\DW}{\Delta_W}
\newcommand{\DELL}{\Delta_{L,\ms{SW}}}
\newcommand{\KONE}{K_1}
\newcommand{\KTWO}{K_2}
\newcommand{\MUS}{\mu_S}
\newcommand{\MUZ}{\mu_Z}
\newcommand{\NZERO}{N_0}
\newcommand{\NB}{N_B}
\newcommand{\ND}{N_D}
\newcommand{\TEMP}{T}
\newcommand{\VDD}{V_{DD}}
\newcommand{\WDF}{W_{\ms{DF}}}
\newcommand{\VFB}{V_{\ms{FB}}}
\newcommand{\UZERO}{U_0}
\newcommand{\UONE}{U_1}
\newcommand{\XTWOE}{X_{2E}}
\newcommand{\XTWOMS}{X_{2\ms{MS}}}
\newcommand{\XTWOMZ}{X_{2\ms{MZ}}}
\newcommand{\XTWOUZERO}{X_{2\ms{U0}}}
\newcommand{\XTWOUONE}{X_{2\ms{U1}}}
\newcommand{\XTHREEE}{X_{3E}}
\newcommand{\XTHREEMS}{X_{3\ms{MS}}}
\newcommand{\XTHREEMZ}{X_{3\ms{MZ}}}
\newcommand{\XTHREEUZERO}{X_{3\ms{U0}}}
\newcommand{\XTHREEUONE}{X_{3\ms{U1}}}
\newcommand{\XPART}{X_{\ms{PART}}}
\newcommand{\PS}{P_S}
\newcommand{\PD}{P_D}
\newcommand{\NRS}{N_{RS}}
\newcommand{\NRG}{N_{RG}}
\newcommand{\NRB}{N_{RB}}
\newcommand{\NRD}{N_{RD}}


\newcommand{\Net}{{${\cal N}$}}                          % network \N
\newcommand{\Nprime}{{${\cal N}^{\prime}$}}            % another network \Nprime
\newcommand{\Nold}{{${\cal N}^{\mbox{old}}$}}          % old network  \Nold
\newcommand{\Nnew}{{${\cal N}^{\mbox{new}}$}}          % new network  \Nnew

\newcommand{\GMIN}{{G_{\ms{MIN}}}}

\newcommand{\optionitem}[2]{
\item[{\tt #1}{#2}]\label{.OPTION#1}\index{.OPTIONS, #1}\index{#1}}

\newcommand{\error}[1]{\vspace{0.1in}\noindent{\tt #1}\\}


%For numbering an equation which is incoorporated
%with text.
\newcommand{\inlineeq}{\hspace*{\fill}\refstepcounter{equation}{\rm
(\theequation)}\\}


%\usepackage{graphicx}

% figures in the current directory
%\newcommand{\fig}[1]{#1}
% figures in the erl space
%% SUMMARY OF USEFUL MACROS
%
% 1. \marginpar[LeftText]{RightText} Standard Latex \marginpar
%
% 2. \mymarginpar{MarginText}{BodyText} Places MarginText in margin and
%     BodyText in main text opposite margin. Places lineas above and below
%     text.  Used in element and model catalogs and elsewhere.
%
% 3. \marginlabel{text} Places large text in margin and underlines. Used
%    to put element name in margin at top of page for continued
%    descriptions.  Does not automaticly put it at the top of a page
%    and so a \clearpage is required.
%
% 4. \marginid{text} Used to place a short identifier in the margin. Just one
%    word and justified to the outside edge.
%
% 5. \offset a standard indent.
%
% 6. \offsetparbox{} Places text in an offset parbox.  Use
%    \hspace*{\fill} \offsetparbox{text}        to insert an offset text.
%
% 7. \boxed{} similar to above but starts a new line and right justifies
%    offsetparbox.
%
% 8. \form
%
% 9. \example
%
%10. \begin{widelist}  .... \end{widelist} 
%    A widelist that has 1 inch wide labels that has additional left hand
%    margin of 0.6 inch.
%
%11. \spicethreeonly{} Text is inserted only if output is for spice3.
%
%12. \pspiceonly{} 
%    Text is inserted only if output is for pspice (superspice for the most
%    part is upwards compatible to pspice).
%
%13. \sspiceonly{} 
%    Text is inserted only if output is for superspice but not for pspice.

%14. \sym{}
%    Right justifies a symbol in a keyword table.
%
%15. Use the following wherever as then we can have a standard way to
%    report them.  Note that "\dc\ " is required to get a space after dc.
%    \dc
%    \ac
%    \SPICE
%
%16. \kwnote{}  This is a convenient way to include notes in a keyword table.
%    It right justifies a note in the description column and starts it on a
%    newline.
\newcommand{\kwnote}[1]{\newline\hspace*{\fill} #1}

%
%17. \kwversion{} This is a convenient way to indicate versions in a keyword
%    table. It does a \kwnote .  It should not printed when outputing for
%     specific versions.
% typical usage is \kwversion{\sspice ; \hspice} to produce the output
%            VERSIONS: SUPERSPICE; HSPICE
% typical usage is \kwversionNote{\sspice ; \hspice} to produce the output
%                 |                    VERSIONS: SUPERSPICE; HSPICE |
%for unifomity the recommended version order is:
%    \hspice \pspice \sspice \spicetwo \spicethree
\newcommand{\version}[1]{({\sc version: #1})}
\newcommand{\kwversion}[1]{\newline({\sc version: #1})}
\newcommand{\kwversionNote}[1]{\kwnote{({\sc version: #1})}}
\newcommand{\versions}[1]{({\sc versions: #1})}
\newcommand{\kwversions}[1]{\newline({\sc versions: #1})}
\newcommand{\kwversionsNote}[1]{\kwnote{({\sc versions: #1})}}

%


\newcommand{\mymargin}{1in}   % Use to set width of margin notes
\newcommand{\mymarginparsep}{0.1in}
\newcommand{\mymarginparsepplustext}{5.6in}
\newcommand{\mymarginplus}{1.1in}
\newcommand{\mymarginplustext}{6.6in}
\newcommand{\underlinehead}{\\[-0.1in] \rule{\mymarginplustext}{0.01in}}
\newcommand{\overlinefoot}{\rule{\mymarginplustext}{0.01in}\\}
% WIDEPARBOX uses margin space as well.
\newcommand{\wideparbox}[1]{\parbox[t]{\mymarginplustext}{#1}}
%
% HEADER AND FOOT
% Standard header and foot, but includes copyright notice.a
% To omit copyrite notice do not include copyrite.sty
%
\newcommand{\mycopyrite}{}
\def\ps@headings{\let\@mkboth\markboth
\def\@oddfoot{\wideparbox{\overlinefoot\rm\today \hfill \mycopyrite}}
\def\@evenfoot{\hspace*{-\mymarginplus}\wideparbox{\overlinefoot
\mycopyrite\hfill \today}}
\def\@evenhead{\hspace*{-\mymarginplus}\wideparbox{\rm
\thepage\hfill \sl \leftmark \underlinehead }}
\def\@oddhead{\wideparbox{\hbox{}\sl \rightmark \hfill
\rm\thepage \underlinehead}}\def\chaptermark##1{\markboth {\uppercase{\ifnum
\c@secnumdepth
>\m@ne
 \@chapapp\ \thechapter. \ \fi ##1}}{}}\def\sectionmark##1{\markright
 {\uppercase{\ifnum \c@secnumdepth >\z@
  \thesection. \ \fi ##1}}}}

\oddsidemargin 0in \topmargin 0.0in \textwidth 6in \textheight 9in
\evensidemargin 0.0in \headheight 0.18in \footskip 0.16in

%   EPSF.TEX macro file:
%   Written by Tomas Rokicki of Radical Eye Software, 29 Mar 1989.
%   Revised by Don Knuth, 3 Jan 1990.
%   Revised by Tomas Rokicki to accept bounding boxes with no
%      space after the colon, 18 Jul 1990.
%
%   TeX macros to include an Encapsulated PostScript graphic.
%   Works by finding the bounding box comment,
%   calculating the correct scale values, and inserting a vbox
%   of the appropriate size at the current position in the TeX document.
%
%   To use with the center environment of LaTeX, preface the \epsffile
%   call with a \leavevmode.  (LaTeX should probably supply this itself
%   for the center environment.)
%
%   To use, simply say
%   \input epsf           % somewhere early on in your TeX file
%   \epsfbox{filename.ps} % where you want to insert a vbox for a figure
%
%   Alternatively, you can type
%
%   \epsfbox[0 0 30 50]{filename.ps} % to supply your own BB
%
%   which will not read in the file, and will instead use the bounding
%   box you specify.
%
%   The effect will be to typeset the figure as a TeX box, at the
%   point of your \epsfbox command. By default, the graphic will have its
%   `natural' width (namely the width of its bounding box, as described
%   in filename.ps). The TeX box will have depth zero.
%
%   You can enlarge or reduce the figure by saying
%     \epsfxsize=<dimen> \epsfbox{filename.ps}
%   (or
%     \epsfysize=<dimen> \epsfbox{filename.ps})
%   instead. Then the width of the TeX box will be \epsfxsize and its
%   height will be scaled proportionately (or the height will be
%   \epsfysize and its width will be scaled proportiontally).  The
%   width (and height) is restored to zero after each use.
%
%   A more general facility for sizing is available by defining the
%   \epsfsize macro.    Normally you can redefine this macro
%   to do almost anything.  The first parameter is the natural x size of
%   the PostScript graphic, the second parameter is the natural y size
%   of the PostScript graphic.  It must return the xsize to use, or 0 if
%   natural scaling is to be used.  Common uses include:
%
%      \epsfxsize  % just leave the old value alone
%      0pt         % use the natural sizes
%      #1          % use the natural sizes
%      \hsize      % scale to full width
%      0.5#1       % scale to 50% of natural size
%      \ifnum#1>\hsize\hsize\else#1\fi  % smaller of natural, hsize
%
%   If you want TeX to report the size of the figure (as a message
%   on your terminal when it processes each figure), say `\epsfverbosetrue'.
%
\newread\epsffilein    % file to \read
\newif\ifepsffileok    % continue looking for the bounding box?
\newif\ifepsfbbfound   % success?
\newif\ifepsfverbose   % report what you're making?
\newdimen\epsfxsize    % horizontal size after scaling
\newdimen\epsfysize    % vertical size after scaling
\newdimen\epsftsize    % horizontal size before scaling
\newdimen\epsfrsize    % vertical size before scaling
\newdimen\epsftmp      % register for arithmetic manipulation
\newdimen\pspoints     % conversion factor
%
\pspoints=1bp          % Adobe points are `big'
\epsfxsize=0pt         % Default value, means `use natural size'
\epsfysize=0pt         % ditto
%
\def\epsfbox#1{\global\def\epsfllx{72}\global\def\epsflly{72}%
   \global\def\epsfurx{540}\global\def\epsfury{720}%
   \def\lbracket{[}\def\testit{#1}\ifx\testit\lbracket
   \let\next=\epsfgetlitbb\else\let\next=\epsfnormal\fi\next{#1}}%
%
\def\epsfgetlitbb#1#2 #3 #4 #5]#6{\epsfgrab #2 #3 #4 #5 .\\%
   \epsfsetgraph{#6}}%
%
\def\epsfnormal#1{\epsfgetbb{#1}\epsfsetgraph{#1}}%
%
\def\epsfgetbb#1{%
%
%   The first thing we need to do is to open the
%   PostScript file, if possible.
%
\openin\epsffilein=#1
\ifeof\epsffilein\errmessage{I couldn't open #1, will ignore it}\else
%
%   Okay, we got it. Now we'll scan lines until we find one that doesn't
%   start with %. We're looking for the bounding box comment.
%
   {\epsffileoktrue \chardef\other=12
    \def\do##1{\catcode`##1=\other}\dospecials \catcode`\ =10
    \loop
       \read\epsffilein to \epsffileline
       \ifeof\epsffilein\epsffileokfalse\else
%
%   We check to see if the first character is a % sign;
%   if not, we stop reading (unless the line was entirely blank);
%   if so, we look further and stop only if the line begins with
%   `%%BoundingBox:'.
%
          \expandafter\epsfaux\epsffileline:. \\%
       \fi
   \ifepsffileok\repeat
   \ifepsfbbfound\else
    \ifepsfverbose\message{No bounding box comment in #1; using defaults}\fi\fi
   }\closein\epsffilein\fi}%
%
%   Now we have to calculate the scale and offset values to use.
%   First we compute the natural sizes.
%
\def\epsfclipstring{}% do we clip or not?  If so,
\def\epsfclipon{\def\epsfclipstring{ clip}}%
\def\epsfclipoff{\def\epsfclipstring{}}%
%
\def\epsfsetgraph#1{%
   \epsfrsize=\epsfury\pspoints
   \advance\epsfrsize by-\epsflly\pspoints
   \epsftsize=\epsfurx\pspoints
   \advance\epsftsize by-\epsfllx\pspoints
%
%   If `epsfxsize' is 0, we default to the natural size of the picture.
%   Otherwise we scale the graph to be \epsfxsize wide.
%
   \epsfxsize\epsfsize\epsftsize\epsfrsize
   \ifnum\epsfxsize=0 \ifnum\epsfysize=0
      \epsfxsize=\epsftsize \epsfysize=\epsfrsize
      \epsfrsize=0pt
%
%   We have a sticky problem here:  TeX doesn't do floating point arithmetic!
%   Our goal is to compute y = rx/t. The following loop does this reasonably
%   fast, with an error of at most about 16 sp (about 1/4000 pt).
% 
     \else\epsftmp=\epsftsize \divide\epsftmp\epsfrsize
       \epsfxsize=\epsfysize \multiply\epsfxsize\epsftmp
       \multiply\epsftmp\epsfrsize \advance\epsftsize-\epsftmp
       \epsftmp=\epsfysize
       \loop \advance\epsftsize\epsftsize \divide\epsftmp 2
       \ifnum\epsftmp>0
          \ifnum\epsftsize<\epsfrsize\else
             \advance\epsftsize-\epsfrsize \advance\epsfxsize\epsftmp \fi
       \repeat
       \epsfrsize=0pt
     \fi
   \else \ifnum\epsfysize=0
     \epsftmp=\epsfrsize \divide\epsftmp\epsftsize
     \epsfysize=\epsfxsize \multiply\epsfysize\epsftmp   
     \multiply\epsftmp\epsftsize \advance\epsfrsize-\epsftmp
     \epsftmp=\epsfxsize
     \loop \advance\epsfrsize\epsfrsize \divide\epsftmp 2
     \ifnum\epsftmp>0
        \ifnum\epsfrsize<\epsftsize\else
           \advance\epsfrsize-\epsftsize \advance\epsfysize\epsftmp \fi
     \repeat
     \epsfrsize=0pt
    \else
     \epsfrsize=\epsfysize
    \fi
   \fi
%
%  Finally, we make the vbox and stick in a \special that dvips can parse.
%
   \ifepsfverbose\message{#1: width=\the\epsfxsize, height=\the\epsfysize}\fi
   \epsftmp=10\epsfxsize \divide\epsftmp\pspoints
   \vbox to\epsfysize{\vfil\hbox to\epsfxsize{%
      \ifnum\epsfrsize=0\relax
        \special{PSfile=#1 llx=\epsfllx\space lly=\epsflly\space
            urx=\epsfurx\space ury=\epsfury\space rwi=\number\epsftmp
            \epsfclipstring}%
      \else
        \epsfrsize=10\epsfysize \divide\epsfrsize\pspoints
        \special{PSfile=#1 llx=\epsfllx\space lly=\epsflly\space
            urx=\epsfurx\space ury=\epsfury\space rwi=\number\epsftmp\space
            rhi=\number\epsfrsize \epsfclipstring}%
      \fi
      \hfil}}%
\global\epsfxsize=0pt\global\epsfysize=0pt}%
%
%   We still need to define the tricky \epsfaux macro. This requires
%   a couple of magic constants for comparison purposes.
%
{\catcode`\%=12 \global\let\epsfpercent=%\global\def\epsfbblit{%BoundingBox}}%
%
%   So we're ready to check for `%BoundingBox:' and to grab the
%   values if they are found.
%
\long\def\epsfaux#1#2:#3\\{\ifx#1\epsfpercent
   \def\testit{#2}\ifx\testit\epsfbblit
      \epsfgrab #3 . . . \\%
      \epsffileokfalse
      \global\epsfbbfoundtrue
   \fi\else\ifx#1\par\else\epsffileokfalse\fi\fi}%
%
%   Here we grab the values and stuff them in the appropriate definitions.
%
\def\epsfempty{}%
\def\epsfgrab #1 #2 #3 #4 #5\\{%
\global\def\epsfllx{#1}\ifx\epsfllx\epsfempty
      \epsfgrab #2 #3 #4 #5 .\\\else
   \global\def\epsflly{#2}%
   \global\def\epsfurx{#3}\global\def\epsfury{#4}\fi}%
%
%   We default the epsfsize macro.
%
\def\epsfsize#1#2{\epsfxsize}
%
%   Finally, another definition for compatibility with older macros.
%
\let\epsffile=\epsfbox


%\newcommand{\fig}[1]{figures/#1}
%\newcommand{\pfig}[1]{\epsfbox{\fig{#1}}}
\newcommand{\newfig}[1]{\epsffile{\fig{#1}}}

\newcommand{\fdaelement}[1]{freeda_elements/#1}
\newcommand{\spiceelement}[1]{equivalent_spice_elements/#1}

% 5/11/03 Reordered $ to get rid of warning messages... FPH
% First two lines below are the 'originals.'  DO NOT MODIFY.
%\newcommand{\FREEDA}{{\Huge{\textsl{\textsf{f}}}${\mathsf{REEDA}}^{{\tiny{\textsf{TM}}}}$}}
%\newcommand{\FDA}{{\textsl{\textsf{f}}}${\mathsf{REEDA}}^{{\tiny{\textsf{TM}}}}$}
% Next two lines below are the mod experiments.  Okay to modify.
\newcommand{\FREEDA}{\Huge{\textsl{\textsf{f}}{\textup{\textsf{REEDA}}}{\textsuperscript{\tiny{\texttrademark}}}}}
\newcommand{\FDA}{{\textsl{\textsf{f}}}{\textup{\textsf{REEDA}}}{\textsuperscript{\tiny{\texttrademark}}}}

\newcommand{\notforsspice}[1]{#1}
\newcommand{\spicetwoonly}[1]{#1}
\newcommand{\spicethreeonly}[1]{#1}
\newcommand{\pspiceonly}[1]{#1}
\newcommand{\pspiceninetytwoonly}[1]{#1}
\newcommand{\sspiceonly}[1]{}
\newcommand{\fornutmeg}[1]{}
\newcommand{\sspicetwoonly}[1]{}

%%%%%%%%%%%%%%%%%%%%%%%%%%%%%%%%%%%%%%%%%%%%%%%%%%%%%%%%%%%%%%%%%%%%%%%%%%%%%%%%
\marginparwidth=\mymargin
\marginparsep=\mymarginparsep
\newcommand{\textplusmarginparwidth}{\textwidth+\marginparsep+\mymargin}
\newcommand{\X}{\\ \hline}    % line terminataion in keyword environment
\newcommand{\B}{{ \rm [}}     % begin optional parameter in \form{}
\newcommand{\E}{{\ \rm\hspace{-0.04in}] }}   % end optional parameter in \form{}
\newcommand{\expr}{{\sc Expressions supported}}
\newcommand{\reqd}{{\scriptsize REQUIRED}}
\newcommand{\omitted}{{\scriptsize OMITTED}}
\newcommand{\inferred}{{\scriptsize INFERRED}}
\newcommand{\para}{\newline{\scriptsize (PARASITIC)}}
\newcommand{\opt}{{\tiny  OPTIONAL}}

\newcommand{\Spice}{{\sc Spice}}
\newcommand{\spice}{{\sc Spice}}
\newcommand{\justspice}{{\sc Spice}}
\newcommand{\nutmeg}{{\sc NUTMEG}}
\newcommand{\probe}{{\sc Probe}}

\newcommand{\accusim}{{\sc AccuSim}}
\newcommand{\contec}{{\sc ContecSpice}}
\newcommand{\cdsspice}{{\sc CDS Spice}}
\newcommand{\hpimpulse}{{\sc HP Impulse}}
\newcommand{\hspice}{{\sc HSpice}}
\newcommand{\igspice}{{\sc IG\_SPICE}}
\newcommand{\isspice}{{\sc IsSpice}}
\newcommand{\mspice}{{\sc Microwave Spice}}
\newcommand{\pspice}{{\sc PSpice}}
\newcommand{\justpspice}{{\sc PSpice}}
\newcommand{\pspiceninetytwo}{{\sc PSpice92}}
\newcommand{\spectre}{{\sc Spectre}}
\newcommand{\spicetwo}{{\sc Spice2g6}}
\newcommand{\spicethree}{{\sc Spice3}}
\newcommand{\spiceplus}{{\sc SpicePlus}}
\newcommand{\sspice}{{\sc SuperSpice}}

\newcommand{\modelversion}[1]{& #1}
%%%%%%%%%%%%%%%%%%%%%%%%%%%%%%%%%%%%%%%%%%%%%%%%%%%%%%%%%%%%%%%%%%%%%%%%%%%%%%%%
% set up a counter for all occasions
%
\newcounter{count}

% set up new commands
%
\newcommand{\vshift}{\vspace{0.2in}}
% OFFSET
\newcommand{\offset}{\hspace*{0.45in} }
% OFFSETPARBOXWIDTH argument should be \textwith - offset
\newcommand{\offsetparbox}[1]{\parbox[t]{5in}{#1}}


\newcommand{\mymarginpar}[2]{
   \mbox{ }
   \marginpar[
   \parbox{\mymarginplustext}{
   \rule{\mymarginplustext}{.5mm} \\ \vspace*{-0.070in} \\
   {\huge #1} \hspace*{\fill} {\Large #2} \\
   \rule{\mymargin}{0.1in} \hfill
   \rule{\textwidth}{0.5mm}}
   ]{
   \hspace*{-\mymarginparsepplustext} \hspace*{-.1in}
   \parbox{\mymarginplustext}{
   \rule{\mymarginplustext}{.5mm} \\ \vspace*{-0.070in} \\
   {\Large #2} \hspace*{\fill} {\huge #1}\\
   \rule{\textwidth}{0.5mm} \hfill
   \rule{\mymargin}{0.1in}}
   } \\ \vspace{0.6in}\\
   }

\newcommand{\mymarginparx}[3]{
   \mbox{ }
   \marginpar[
   \parbox{\mymarginplustext}{
   \rule{\mymarginplustext}{.5mm} \\ \vspace*{-0.070in} \\
   {\huge #1} \hspace*{\fill}
   {\large #2} \hspace*{\fill}
 {\Large #3} \\
   \rule{\mymargin}{0.1in} \hfill
   \rule{\textwidth}{0.5mm}}
   ]{
   \hspace*{-\mymarginparsepplustext} \hspace*{-.1in}
   \parbox{\mymarginplustext}{
   \rule{\mymarginplustext}{.5mm} \\ \vspace*{-0.070in} \\
   {\Large #3} \hspace*{\fill}
   {\Large #2} \hspace*{\fill} {\huge #1}\\
   \rule{\textwidth}{0.5mm} \hfill
   \rule{\mymargin}{0.1in}}
   } \\ \vspace{0.6in}\\
   }

\newcommand{\marginid}[1]{
   \marginpar[ #1]{\hfill #1}}

\newcommand{\marginlabel}[1]{
   \marginpar[ {\huge #1}\\ \rule{\mymargin}{0.1in} ]{
              \hfill  {\huge #1}\\ \rule{\mymargin}{0.1in}}
              \vspace{-0.2in}}

%marginlabelfull is very wide
\newcommand{\marginlabelfull}[1]{
   \marginpar[ {\huge #1}\\ \rule{\mymargin}{0.1in} ]{
              \hspace*{-3in}\parbox{4in}{\hspace*{\fill}\huge #1}
              \\[0.05in] \rule{\mymargin}{0.1in}}
             \vspace*{0.2in}
          }

\newcommand{\element}[2]{\clearpage\rm\markright{#2}
\addcontentsline{toc}{section}{#1, #2}
\mymarginpar{#1}{#2}\label{#1element}\index{#1}\index{#2}}

% note: no labeling
\newcommand{\elementx}[3]{\clearpage\rm\markright{#3:#2}
\addcontentsline{toc}{section}{#1, #2: #3}
\mymarginparx{#1}{#2}{#3}\index{#1:#2}\index{#3:#2}}

% macros for sub elements
\newcommand{\subelement}[2]{\clearpage 
   \noindent\rule{\textwidth /2}{.5mm} \\[0.1in]
   {\large \bf #1} \hspace{0.2in} {\bf #2} \\
   \noindent\rule{\textwidth /2}{.5mm} \index{#1} \index{#2}}
% macros for models
\newcommand{\model}[2]{{
   \noindent\vspace{0.2in}\parbox{\textwidth}{
   \noindent\rule{\textwidth}{.5mm} \\[0.1in]
   {\large \bf #1 Model} \label{#1model} \hfill {\large #2} \\
   \noindent\rule{\textwidth}{.5mm} \index{#1} \index{#2}}}}
\newcommand{\modelx}[3]{{
   \noindent\vspace{0.2in}\parbox{\textwidth}{
   \noindent\rule{\textwidth}{.5mm} \\[0.1in]
   {\large \bf #1 Model} \hfill {\large #2} \hfill {\large #3} \\
   \noindent\rule{\textwidth}{.5mm} \index{#1} \index{#2}}}}


% macros for statements
\newcommand{\statement}[2]{\clearpage\rm\markright{#2}
   \addcontentsline{toc}{section}{#1, #2}
   \mymarginpar{#1}{#2} \label{#1statement}\index{#1}\index{#2}}

\newcommand{\statementx}[3]{\clearpage\rm\rm\markright{#3:#2}
   \addcontentsline{toc}{section}{#1, #2: #3}
   \mymarginparx{#1}{#2}{#3} \index{#1:#2} \index{#3:#2} \label{#1statement}}

% BOXED
\newcommand{\boxed}[1]{\noindent
\newline \vshift \hspace*{\fill} {\tt \offsetparbox{\tt #1}}
 \vshift}


%
% KEYWORD
%
\newcommand{\keywordtable}[1]{
        \sloppy
        \hyphenation{ca-pac-i-t-an-ce} 
        \begin{center}
    \sf
        \begin{tabular}[t]
        {|p{0.58in}|p{3.07in}|p{0.55in}|p{0.60in}|}
        \hline
        \multicolumn{1}{|c}{\bf Name} &
        \multicolumn{1}{|c}{\parbox{2.77in}{\bf Description}}  &
        \multicolumn{1}{|c}{\bf Units} &
        \multicolumn{1}{|c|}{\bf Default} \X
        #1
        \end{tabular}
        \end{center}
    }

\newcommand{\keywordtwotable}[2]{
        \sloppy
        \hyphenation{ca-pac-i-t-an-ce} 
        \begin{center}
    \sf
        \begin{tabular}[t]
        {|p{0.58in}|p{2.38in}|p{0.55in}|p{0.60in}|p{0.53in}|}
        \hline
        \multicolumn{1}{|c}{\bf Name} &
        \multicolumn{1}{|c}{\parbox{2.20in}{\bf Description}}  &
        \multicolumn{1}{|c}{\bf Units} &
        \multicolumn{1}{|c}{\bf Default} &
        \multicolumn{1}{|c|}{\bf #1} \X
        #2
        \end{tabular}
        \end{center}
    }

\newcommand{\kw}[2]{
     \samepage{
     \noindent {\sl #1} \vspace{-0.5in} \\
     \keywordtable{#2} }}

\newcommand{\kwtwo}[3]{
     \samepage{
     \noindent {\sl #1} \vspace{-0.4in} \\
     \keywordtwotable{#2}{#3} }}

\newcommand{\keyword}[1]{\kw{Keywords:}{#1}}
\newcommand{\keywordtwo}[2]{\kwtwo{Keywords:}{#1}{#2}}
\newcommand{\modelkeyword}[1]{\kw{Model Keywords}{#1}}
\newcommand{\modelkeywordtwo}[2]{\kwtwo{Model Keywords}{#1}{#2}}

\newcommand{\myline}{\\[-0.1in]
\noindent \rule{\textwidth}{0.01in} \newline}

\newcommand{\myThickLine}{\\[-0.1in]
\noindent \rule{\textwidth}{0.02in} \newline}


% SPICE 2G6 FORM
\newcommand{\spicetwoform}[1]{
\spicetwoonly{\samepage{\noindent{\sl\spicetwo\form{#1}}}}}

% PSPICE88 FORM
\newcommand{\pspiceform}[1]{
\pspiceonly{\samepage{\noindent{\sl\pspice}\form{#1}}}}

% PSPICE92 FORM
\newcommand{\pspiceninetytwoform}[1]{
\pspiceninetytwoonly{\samepage{\noindent{\sl\pspice92\form{#1}}}}}


% SPICE3E2 FORM
\newcommand{\spicethreeform}[1]{
\spicethreeonly{\samepage{\noindent{\sl\spicethree\form{#1}}}}}

% FORM
\newcommand{\form}[1]{\samepage{\noindent
 {\sl Form} \myline
% \hspace*{\fill} % For some reason \fill = 0 when \pspiceform{} is used?
\offset
\it  \offsetparbox{#1}}
\\[0.1in]}

% ELEMENT FORM
\newcommand{\elementform}[1]{\samepage{\noindent
 {\sl Element Form} \myline
% \hspace*{\fill} % For some reason \fill = 0 when \pspiceform{} is used?
\offset
\it  \offsetparbox{#1}}
\\[0.1in]}

% MODEL FORM
\newcommand{\modelform}[1]{\samepage{\noindent
 {\sl Model Form} \myline
% \hspace*{\fill} % For some reason \fill = 0 when \pspiceform{} is used?
\offset
\it  \offsetparbox{#1}}
\\[0.1in]}

% LIMITS
\newcommand{\mylimits}[1]{\samepage{\noindent
 {\sl Limits} \myline
 \hspace*{\fill} \it  \offsetparbox{#1}}
 \vshift}

% EXAMPLE
\newcommand{\example}[1]{\samepage{\noindent
{\sl Example} \myline
\offset \tt  \offsetparbox{#1}}
 \vshift}

% PSPICE88 EXAMPLE
\newcommand{\pspiceexample}[1]{\samepage{\noindent
{\sl \pspice\ Example} \myline
\offset \tt  \offsetparbox{#1}}
 \vshift}

% MODEL TYPES
\newcommand{\modeltype}[1]{\samepage{\noindent
{\sl Model Type} \myline
 \hspace*{\fill} \tt \offsetparbox{#1}}
 \\[0.1in]}

% MODEL TYPES
\newcommand{\modeltypes}[1]{\samepage{\noindent
{\sl Model Types:} \myline
 \hspace*{\fill} \tt \offsetparbox{\tt #1}}
 \vshift}

% OFFSET ENUMERATE
\newcommand{\offsetenumerate}[1]{
     \offset \hspace*{-0.1in} {\begin{enumerate} #1 \end{enumerate}}}

% NOTE
\newcommand{\note}[1]{
\vshift\samepage{\noindent {\sl Note}\myline\vspace{-0.24in}}
 \offsetenumerate{#1} }

% SPECIAL NOTE
\newcommand{\specialnote}[2]{
\vshift\samepage{\noindent {\sl #1}\myline\vspace{-0.24in}}\\#2}

\newcommand{\dc}{\mbox{\tt DC}}
\newcommand{\ac}{\mbox{\tt AC}}
\newcommand{\SPICE}{\mbox{\tt SPICE}}
\newcommand{\m}[1]{{\bf #1}}                           % matrix command  \m{}

%
% print nodes in \tt and enclose in a circle use outside
\newcommand{\node}[1]{\: \mbox{\tt #1} \!\!\!\! \bigcirc }
%
% set up environment for example
%
\newcounter{excount}
\newcounter{dummy}
\newenvironment{eg}{\vspace{0.1in}\noindent\rule{\textwidth}{.5mm}
   \begin{list}
   {{\addtocounter{excount}{1}
   \em Example\/ \arabic{chapter}.\arabic{excount}\/}:}
   {\usecounter{dummy}
   \setlength{\rightmargin}{\leftmargin}}
   }{\end{list} \rule{\textwidth}{.5mm}\vspace{0.1in}}
%
% set up environment for block
% currently this draws a horizontal line at the start of block and another
% at the end of block.
%
\newenvironment{block}{\vspace{0.1in}\noindent\rule{\textwidth}{.5mm}
   }{\rule{\textwidth}{.5mm}\vspace{0.1in}}
%

%
% Macros for element summaries
%
% macros for element summary
\newcommand{\summaryelement}[2]{
   \vspace{0.4in}
   \mymarginpar{#1}{#2} 
   \addcontentsline{toc}{section}{#1, #2}
   \vspace{-0.6in} \\
   \noindent Full description on page \pageref{#1element}. \vshift\\
   }

% Summary MODEL TYPE
\newcommand{\summarymodeltype}[1]{\samepage{\noindent
{\sl Model Type} \myline
 \hspace*{\fill} \tt \offsetparbox{\tt #1
 \hfill Summary on \pageref{#1summary} \index{#1}}}
 }

% Summary MODEL TYPES
\newcommand{\summarymodeltypes}[1]{\samepage{\noindent
{\sl Model Types} \myline
 \hspace*{\fill} \tt \offsetparbox{\tt #1
 \hfill Summary on \pageref{#1summary} \index{#1}}}
 }

%
% macros for model summary
\newcommand{\summarymodel}[2]{\clearpage
   \addcontentsline{toc}{section}{#1, #2}
   \mymarginpar{#1}{#2} \label{#1summary}
   \index{#1} \index{#2}
   \noindent Full description on page \pageref{#1model} \\[0.1in]
   }



%
% set up wide descriptive list
%
\newenvironment{widelist}
    {\begin{list}{}{\setlength{\rightmargin}{0in} \setlength{\itemsep}{0.1in}
    \setlength{\labelwidth}{0.95in} \setlength{\labelsep}{0.1in}
\setlength{\listparindent}{0in} \setlength{\parsep}{0in}
    \setlength{\leftmargin}{1.0in}}
    }{\end{list}}

\newcommand{\STAR}{\hspace*{\fill} * \hspace*{\fill}}

\newcommand{\sym}[1]{\hspace*{\fill} ($#1$)}

%
% The thebibliography environment is redefined so the the word References is
% not output
%\def\thebibliography#1{\list
% {[\arabic{enumi}]}{\settowidth\labelwidth{[#1]}\leftmargin\labelwidth
% \advance\leftmargin\labelsep
% \usecounter{enumi}}
% \def\newblock{\hskip .11em plus .33em minus -.07em}
% \sloppy\clubpenalty4000\widowpenalty4000
% \sfcode`\.=1000\relax}
%\let\endthebibliography=\endlist
% END thebibliography environment redefinition



\newcommand{\eskipv}[1]{\clearpage\marginlabel{#1}}
\newcommand{\eskip}[1]{\vspace*{\fill}\clearpage\marginlabel{#1}}
\newcommand{\eskipnv}[1]{\newpage\marginlabel{#1}}
\newcommand{\eskipn}[1]{\vspace*{\fill}\newpage\marginlabel{#1}}

%marginlabel is very wide
\newcommand{\eskipfullv}[1]{\clearpage\marginlabelfull{#1}}
\newcommand{\eskipfull}[1]{\vspace*{\fill}\clearpage\marginlabelfull{#1}}
\newcommand{\eskipfullnv}[1]{\newpage\marginlabelfull{#1}}
\newcommand{\eskipfulln}[1]{\vspace*{\fill}\newpage\marginlabelfull{#1}}

\newcommand{\mycontentsline}[5]{\parbox{#1}{#2}#3
\hspace{0.1in}\dotfill\parbox{0.3in}{\hfill\pageref{#4#5}}\\[0.1in]}
\newcommand{\mysline}[2]{\mycontentsline{1.2in}{#1}{#2}{#1}{statement}}
\newcommand{\mymline}[2]{\mycontentsline{0.7in}{#1}{#2}{#1}{model}}
\newcommand{\myeline}[2]{\mycontentsline{0.7in}{#1}{#2}{#1}{element}}

\newcommand{\myincontentsline}[5]{\vspace{0.05in}\noindent\parbox{#1}{#2}#3
\hspace{0.1in}
\dotfill\parbox{0.7in}{\hfill Page \pageref{#4#5}}\\[0.05in]\noindent}
\newcommand{\myinsline}[2]{\myincontentsline{1.2in}{#1}{#2}{#1}{statement}}
\newcommand{\myinmline}[2]{\myincontentsline{0.5in}{#1}{#2}{#1}{model}}
\newcommand{\myineline}[2]{\myincontentsline{0.5in}{#1}{#2}{#1}{element}}

%
%
% The following a symbols that could used alot.
\newcommand{\ms}[1]{\mbox{\scriptsize #1}}
\newcommand{\AF}{A_F}
\newcommand{\CBD}{C'_{BD}}
\newcommand{\CBS}{C'_{BS}}
\newcommand{\CGBO}{C_{GBO}}
\newcommand{\CGDO}{C_{GDO}}
\newcommand{\CGSO}{C_{GSO}}
\newcommand{\CJ}{C_J}
\newcommand{\CJSW}{C_{J,\ms{SW}}}
\newcommand{\DELTA}{\delta}
\newcommand{\ETA}{\eta}
\newcommand{\FC}{F_C}
\newcommand{\GAMMA}{\gamma}
\newcommand{\IS}{I_S}
\newcommand{\JS}{J_S}
\newcommand{\KAPPA}{\kappa}
\newcommand{\KF}{K_F}
\newcommand{\KP}{K_P}
\newcommand{\LAMBDA}{\lambda}
\newcommand{\LD}{X_{JL}}
\newcommand{\LEVEL}{M_J}
\newcommand{\MJ}{M_J}
\newcommand{\MJSW}{M_{J,\ms{SW}}}
\newcommand{\NSUB}{N_B}
\newcommand{\NSS}{N_{\ms{SS}}}
\newcommand{\NFS}{N_{\ms{FS}}}
\newcommand{\NEFF}{N_{\ms{EFF}}}
\newcommand{\PB}{\phi_J}
\newcommand{\PHI}{2\phi_B}
\newcommand{\RD}{R_D}
\newcommand{\RS}{R_S}
\newcommand{\RSH}{R_{\ms{SH}}}
\newcommand{\THETA}{\theta}
\newcommand{\TOX}{T_{OX}}
\newcommand{\TPG}{T_{\ms{PG}}}
\newcommand{\UCRIT}{U_C}
\newcommand{\UEXP}{U_{\ms{EXP}}}
\newcommand{\UO}{\mu_0}
\newcommand{\UTRA}{U_{\ms{TRA}}}
\newcommand{\VMAX}{V_{\ms{MAX}}}
\newcommand{\VTZERO}{V_{T0}}
\newcommand{\VTO}{V_{T0}}
\newcommand{\XJ}{X_J}
\newcommand{\Length}{L} %  \L already used
\newcommand{\N}{N}
\newcommand{\PBSW}{\phi_{J,{\ms{SW}}}}
\newcommand{\RB}{R_B}
\newcommand{\RG}{R_B}
\newcommand{\RDS}{R_{DS}}
\newcommand{\TT}{\tau_T}
\newcommand{\W}{W}
\newcommand{\WD}{W_D}
\newcommand{\XQC}{X_{QC}}
\newcommand{\JSSW}{J_{S,{\ms{SW}}}}
\newcommand{\DL}{\Delta_L}
\newcommand{\DW}{\Delta_W}
\newcommand{\DELL}{\Delta_{L,\ms{SW}}}
\newcommand{\KONE}{K_1}
\newcommand{\KTWO}{K_2}
\newcommand{\MUS}{\mu_S}
\newcommand{\MUZ}{\mu_Z}
\newcommand{\NZERO}{N_0}
\newcommand{\NB}{N_B}
\newcommand{\ND}{N_D}
\newcommand{\TEMP}{T}
\newcommand{\VDD}{V_{DD}}
\newcommand{\WDF}{W_{\ms{DF}}}
\newcommand{\VFB}{V_{\ms{FB}}}
\newcommand{\UZERO}{U_0}
\newcommand{\UONE}{U_1}
\newcommand{\XTWOE}{X_{2E}}
\newcommand{\XTWOMS}{X_{2\ms{MS}}}
\newcommand{\XTWOMZ}{X_{2\ms{MZ}}}
\newcommand{\XTWOUZERO}{X_{2\ms{U0}}}
\newcommand{\XTWOUONE}{X_{2\ms{U1}}}
\newcommand{\XTHREEE}{X_{3E}}
\newcommand{\XTHREEMS}{X_{3\ms{MS}}}
\newcommand{\XTHREEMZ}{X_{3\ms{MZ}}}
\newcommand{\XTHREEUZERO}{X_{3\ms{U0}}}
\newcommand{\XTHREEUONE}{X_{3\ms{U1}}}
\newcommand{\XPART}{X_{\ms{PART}}}
\newcommand{\PS}{P_S}
\newcommand{\PD}{P_D}
\newcommand{\NRS}{N_{RS}}
\newcommand{\NRG}{N_{RG}}
\newcommand{\NRB}{N_{RB}}
\newcommand{\NRD}{N_{RD}}


\newcommand{\Net}{{${\cal N}$}}                          % network \N
\newcommand{\Nprime}{{${\cal N}^{\prime}$}}            % another network \Nprime
\newcommand{\Nold}{{${\cal N}^{\mbox{old}}$}}          % old network  \Nold
\newcommand{\Nnew}{{${\cal N}^{\mbox{new}}$}}          % new network  \Nnew

\newcommand{\GMIN}{{G_{\ms{MIN}}}}

\newcommand{\optionitem}[2]{
\item[{\tt #1}{#2}]\label{.OPTION#1}\index{.OPTIONS, #1}\index{#1}}

\newcommand{\error}[1]{\vspace{0.1in}\noindent{\tt #1}\\}


%For numbering an equation which is incoorporated
%with text.
\newcommand{\inlineeq}{\hspace*{\fill}\refstepcounter{equation}{\rm
(\theequation)}\\}

%\topmargin 0mm \oddsidemargin -2mm \evensidemargin -2mm \textwidth
%173mm \textheight 210mm

\chapter{Adding Nonlinear Elements to \FDA} \label{nonlinear:addition}
%\title{Adding Nonlinear Elements to \FDA} \label{nonlinear:addition}
%\author{Frank P. Hart}
%\begin{document}
%\maketitle
%\section{}
%----------------------------------------------------------------------
\section{The \FDA Circuit Simulator}

This tutorial describes the addition of nonlinear device models to
\FDA \footnote{Further details may be found at
http://www.freeda.org}. We assume the reader is familiar with C++
syntax and basic concepts about object-oriented programming.

Presently, there are two ways to write nonlinear element models,
using two different base classes.  One class, AdolcElement, has
been extensively used and is the base class of the case study
to follow.  Unfortunately, a limitation recently discovered in
the Adol-C library is reflected directly into a limitation on
the elements that can be defined by the AdolcElement class.
This class has three function prototypes for the \texttt{createTape()}
function.  One of these prototype functions is designed to
allow the declaration of a second derivative state variable
intended to track one of the declared state variables.  \texttt{This
second derivative capability has been found to be non-functional.}
Usage of the function prototype is still permitted, as it is the
only way to obtained delayed instances of state variables, but
the argument for the second derivative IntVector to createTape() in
this case should be 'novar' and this is noted herein.

The \FDA team is engaged in the on-going development of a new
nonlinear base class called NAdolcElement which is intended to
alleviate the limitation in Adol-C and allow evaluation of
second and third derivatives in models.  While NAdolcElement
is subject to change, model writers requiring these higher
order derivatives are invited to inspect the header files and
source codes for NMOSx and PMOSx elements (where x is 1,2,3,
or 14) and the CompactDiode element and use them as templates
for writing a model based on the NAdolcElement class.  Model
writers are encouraged to use NAdolcElement for writing all
new elements, since over a period of time, AdolcElement will take
on full legacy status and its continued use will be discouraged.

A future revision of this document will also contain a case
study of an element based on the NAdolcElement base class.


%----------------------------------------------------------------------
\section{Example: Adding a Nonlinear Electro-Thermal Resistor}

We illustrate the addition of nonlinear device models using a
step-by-step example with a simple model: a nonlinear electro-thermal
resistor\footnote{Note that the electro-thermal resistor
behaves as a linear resistor when the parameter ``pdr'' is set to
\emph{false}}.

\subsection{Netlist syntax}
A brief description of the \FDA netlist syntax will assist in
understanding of the details of model construction.  The standard
\FDA netlist syntax is common to all elements. This differs from
the \texttt{SPICE} syntax.  \texttt{SPICE} syntax is also supported
but its grammar is less general than the standard \FDA netlist syntax.
This is because the \FDA grammar is common to all elements while the
grammar of the \texttt{SPICE} syntax is not consistent and
each \texttt{SPICE} element must be handled separately in the
netlist parser.

The standard \FDA netlist syntax is\\

$\langle \tt{netlist\ Name} \rangle$:$\langle \tt{element\
ID}\rangle$ $\langle \tt{terminal1}\rangle \cdots \langle
\tt{terminaln}\rangle$ $[\langle \tt{parameter} \rangle = \langle
\tt{value} \rangle \cdots]$
\newline
\newline
For example, the \FDA netlist syntax for the thermal resistor element
(the {\bf tr} element), which will be shown here for tutorial
purposes, is:
\newline
\newline
\indent {\bf tr:tr1\ n1\ n2\ n3\ n4\ r0=100 l=1 w=1 t=35 rsh=100 pdr=true} \\
\newline
{\bf tr1} is the {\bf elementID} and {\bf n1} through {\bf n4} are the
names of four terminals.  {\bf r0, l, w, t,} etc. are parameter syntax
names, and the values on the right side of the $=$ symbol are parameter
values specific to this instance of {\bf tr:1}.  Some optional parameters
have been omitted in this example.

\subsection{Class Name, Required Files, etc.}

A good class name for the element is \textbf{Tresistor}. By
convention (in \FDA), class names should begin with capital
letters and contain no underscores. A netlist name is needed for
the element. Assume we call it ``tr''.  \emph{The netlist name
must always be given in lowercase letters because the \FDA netlist
parser converts a netlist to lower case before attempting to
identify elements or parameters.}

The files containing the model description are located in the
\texttt{elements/} directory. There is a header file
(\texttt{Tresistor.h}) containing the declaration of the class that
defines the element and a file containing the definition of the class
with the actual model (\texttt{Tresistor.cc}). Those are the only files
needed to define this element. They are described next.

\newpage
\subsection{The Header File}

The header file starts with comment lines describing the element
type and sometimes a simple ASCII drawing of the element schematic
as shown in Listing \ref{nonlinear:hhdrcmts}. The figure can be used to
describe terminal numbering.

\begin{lstlisting}[firstnumber=1, label={nonlinear:hhdrcmts},
    caption={Header Comments of Tresistor.h}]

     // This may look like C code, but it is really -*- C++ -*-
     //
     // This is an electro-thermal resistor model
     //
     //           ++++++++++++++
     //           +     tr     +
     //     $t_0$  o------/\/\/\-------o  $t_1$
     //           +            +
     //           +  *      *  +
     //           +  |      |  +
     //           ++ | ++++ | ++
     //              |      |
     //           $t_3$   o      o  $t_2$
     //             tref   tout
     //
     // by Houssam S. Kanj
\end{lstlisting}
%\linebreak 2

Header files may be included more than once in C++ programs. To
avoid multiple declarations of the classes defined in the body of
the header file, the definitions in the header file are enclosed
by following preprocessor directives as in Listing
\ref{nonlinear:hpdirect}.

\begin{lstlisting}[firstnumber=last, label={nonlinear:hpdirect},
    caption={Preprocessor Directives of Tresistor.h}]

#ifndef Tresistor_h
#define Tresistor_h 1

#include "../network/AdolcElement.h"


class Tresistor : public AdolcElement

\end{lstlisting}
%\linebreak 2

The \textbf{Tresistor} class is derived from the base class
\textbf{AdolcElement}. This is so for all nonlinear elements in
\FDA that rely upon ADOL-C for evaluating the first and higher
derivatives of vector functions. For simple elements, the only
public functions that must be declared are shown in Listing
\ref{nonlinear:hclasspub}.

\begin{lstlisting}[firstnumber=24, label={nonlinear:hclasspub},
    caption={Class declaration and public functions of Tresistor.h}]

class Tresistor : public AdolcElement
{ public:

  Tresistor(const string& iname);

  ~Tresistor() {}

  static const char* getNetlistName()
  {
    return einfo.name;
  }

  // Do some local initialization
  virtual void init() throw(string&);

  // Get a vector with the indexes of the local reference nodes.
  // Get terminal pointers in term_list
  // ordered by local reference node:
  //
  // $t_0$ $t_1$ $t_2$ $t_3$
  //
  //   ^    ^
  //  LRN1 LRN2
  //
  // local_ref_vec contains: {1, 3}
  //
  virtual void getLocalRefIdx(UnsignedVector& local_ref_vec,//
                  TerminalVector& term_list);//

  // fill MNAM
  virtual void fillMNAM(FreqMNAM* mnam);

  virtual void fillMNAM(TimeMNAM* mnam);

\end{lstlisting}
%\linebreak 2

The constructor always takes the same argument, \texttt{iname}. This
is the name of the particular instance of the thermal resistor being created
(\emph{e.g.} ``tr1'').  It is passed from the input netlist at the time
\FDA is run.

The destructor in this case is trivial. The next function,
\texttt{getNetlistName()} must be defined in every \FDA element.
It returns the netlist name of the element (``tres'' in our
example) and it is used during the netlist parsing. It is declared
static because this function is called before the actual element
instance is created.

The function \texttt{init()} must be defined for every nonlinear
element based on Adol-C, since it is here where we create the
Adol-C tape. It is also here where we perform some initializations
that cannot be done by the constructor, \textit{\textbf{i.e.}}
when some parameters are not known at construction time.

The function \texttt{getLocalRefIdx()} must be defined whenever we
want to implement an element with multiple reference nodes
\textbf{\textit{e.g.}} spatially distributed elements or
electro-thermal ones. It is important to know that this function
has nothing to do with linearity. So we may have a linear element
with one reference node, or with multiple reference nodes. We may
also have a nonlinear element with one reference node, or with
multiple reference nodes.  Local reference nodes are covered in
further detail in Section \ref{nonlinear:locrefidx}.

The last two functions are the declarations of the functions
to fill the entries corresponding to this element in the
\emph{modified nodal admittance matrix} (MNAM) of the circuit
being simulated. Please refer to
\cite{l_element} for a detailed
explanation on how to implement a linear element.
\newline
\newline
The private members of this class are shown in Listing
\ref{nonlinear:hclasspri}.

\begin{lstlisting}[firstnumber=last, label={nonlinear:hclasspri},
    caption={Private functions of Tresistor.h}]

private:


  virtual void eval(adoublev& x, adoublev& vp, adoublev& ip);



  // Element information
  static ItemInfo einfo;

  // Number of parameters of this element
  static const unsigned n_par;

  // Parameter variables
  double r0, rsh, l, w, narrow, defw, t, tnom, tc1, tc2;
  bool pdr;

  // Parameter information
  static ParmInfo pinfo[];

  // The calculated resistor value
     double r;
};

#endif
\end{lstlisting}
%\linebreak 2

The netlist parameters of this element are declared as normal
member variables. Here we have 11 parameters, and again we
emphasize that \emph{parameter names must always be given in
lowercase letters because the \FDA netlist parser converts a netlist
to lower case before attempting to identify elements or parameters.}
Please refer to reference \cite{etrcat} for a description of the
parameters and the device equations. The variable ``r'' is a temporary
variable used to compute the resistance value when the resistor is linear
i.e. when pdr == \emph{false}. The last three variables can be left
unchanged for any other element in \FDA. We will describe their use later.

The only private function that must be defined is the \texttt{eval}
function.
It takes 3 arguments of type adoublev\&, the first one is an input argument,
and the second and the third are output arguments. Things will be more clear
when we discuss the \texttt{createTape} function.

\subsection{The Class Source File}

\subsubsection{Preprocessor Directives}

The include preprocessor directives go first; see Listing
\ref{nonlinear:spdirect}. The \texttt{ElementManager.h} file
must be always included first. The next file to be included is
\texttt{AdolcElement.h}. This file must be included in all nonlinear
elements that are based on Adol-C.  Since our element is also a linear
one, we also need to include the declarations for the MNAM classes.
There are currently two implementations of the MNAM in \FDA.  One is
the frequency domain (\texttt{FreqMNAM}) and the other is the
time domain (\texttt{TimeMNAM}). The last included file is the
declaration of the class for our element, \texttt{Tresistor.h}.

\begin{lstlisting}[firstnumber=1, label={nonlinear:spdirect},
    caption={Preprocessor directives of Tresistor.cc}]

#include "../network/ElementManager.h"
#include "../analysis/FreqMNAM.h"
#include "../analysis/TimeMNAM.h"
#include "Tresistor.h"
\end{lstlisting}
%\linebreak 2

\subsubsection{Static Variables and Model Parameters}
Now we must define the static member variables as in Listing
\ref{nonlinear:ssparm}.

\begin{lstlisting}[firstnumber=7, label={nonlinear:ssparm},
    caption={Static members and Parameter info of Tresistor.cc}]

// Static members
const unsigned Tresistor::n_par = 11;

// Element information
ItemInfo Tresistor::einfo = {
  "tr",
  "Tresistor",
  "Houssam S. Kanj",
  DEFAULT_ADDRESS"elements/Tresistor.h.html"
};

// Parameter information
ParmInfo Tresistor::pinfo[] = {
  {"r0", "Resistance value (Ohms)", TR_DOUBLE, false},
  {"l", "length (meters)", TR_DOUBLE, false},
  {"w", "width (meters)", TR_DOUBLE, false},
  {"t", "system Temperature (Celsius)", TR_DOUBLE, false},
  {"rsh", "sheat resistance (Ohms/sq)", TR_DOUBLE, false},
  {"defw", "default device width (meters)", TR_DOUBLE, false},
  {"narrow", "narrowing due to side etching (meters)", TR_DOUBLE, false},
  {"tnom", "initial Temperature (Celsius)", TR_DOUBLE, false},
  {"tc1", "Temperature Coefficient (1/Celsius)", TR_DOUBLE, false},
  {"tc2", "Temperature Coefficient (1/Celsius)", TR_DOUBLE, false},
  {"pdr", "Power Depenent Resistor", TR_BOOLEAN, false}

};
\end{lstlisting}
%\linebreak 2

The \texttt{n\_par} variable must be equal to the number of
parameters defined in our element. In our example, 11 parameters
are defined. Note the use of the scope (::) operator. It is used to
indicate that the \texttt{n\_par} variable is a member of the
\textbf{Tresistor} class.  The \texttt{einfo} structure contains
strings describing the element. The first string is the netlist
name of the element (``tr''). \emph{This name must always be given in
lowercase letters because the \FDA netlist parser converts a netlist
to lower case before attempting to identify elements or parameters.}
The second string is a more verbose description of the element. The third
string should list the authors of the element code and the last
string is used to store documentation information, see Section \ref{sec:doc:string}

The \texttt{pinfo} vector contains the description of each parameter
defined in the element. The first field (\emph{again, always in lowercase})
is the netlist name of the parameter followed by a more verbose
description. We recommend to include the units of the parameter (if
any) in this description. The third field is a flag denoting the type
of the parameter. The possible flags are defined in the
\texttt{NetListItem.h} file.  The corresponding types for each flag are
given in Table \ref{nltable1}.
%
\begin{table}[htpb]
\centerline {
\begin{tabular}{|l|l|}
\hline
Flag & Corresponding type \\
\hline
TR\_INT  & \texttt{int} \\
TR\_LONG & \texttt{long int} \\
TR\_FLOAT & \texttt{float} \\
TR\_DOUBLE & \texttt{double} \\
TR\_CHAR & \texttt{char} \\
TR\_STRING & \texttt{string} \\
TR\_COMPLEX & \texttt{double\_complex} \\
TR\_BOOLEAN & \texttt{bool} \\
TR\_INT\_VECTOR & \texttt{IntVector} \\
TR\_DOUBLE\_VECTOR & \texttt{DoubleVector} \\
TR\_INT\_MATRIX & \texttt{IntMatrix} \\
TR\_DOUBLE\_MATRIX & \texttt{DoubleMatrix} \\
\hline
\end{tabular}}
\caption{Possible parameter types. \label{nltable1}}
\end{table}
%
The last field in the parameter description is a flag that is
\emph{true} if the parameter is required in the netlist (\emph{i.e.}
an error occurs if the parameter is not specified) and \emph{false}
otherwise.

\subsubsection{Constructor for a Nonlinear Instance of Tresistor}
The constructor definition for a nonlinear instance of
\texttt{Tresistor} (where pdr == \emph{true}) follows in
Listing \ref{nonlinear:sconstr}.

\begin{lstlisting}[firstnumber=35, label={nonlinear:sconstr},
    caption={Constructor definition of Tresistor.cc}]

Tresistor::Tresistor(const string& iname) : AdolcElement(&einfo,
pinfo, n_par, iname) {

  // Set parameters
  paramvalue[0] = &(r0);
  paramvalue[1] = &(l);
  paramvalue[2] = &(w);
  paramvalue[3] = &(t);
  paramvalue[4] = &(rsh);
  paramvalue[5] = &(defw = 1e-6);
  paramvalue[6] = &(narrow = zero);
  paramvalue[7] = &(tnom = 27);
  paramvalue[8] = &(tc1= zero);
  paramvalue[9] = &(tc2= zero);
  paramvalue[10] = &(pdr=false);


}
\end{lstlisting}
%\linebreak 2

The constructor for the \textbf{AdolcElement} class takes the static
member variables we defined before and the instance name
(\texttt{iname}) as arguments. In the body of the \textbf{Tresistor}
constructor we must point each element of the \texttt{paramvalue}
vector to the address of the member variables corresponding to the
netlist parameters. The order of the elements in the
\texttt{paramvalue} vector must be the same order that we used in the
description in \texttt{pinfo}.

Then the \texttt{init()} function definition follows in Listing
\ref{nonlinear:sinitpdrt}.

\begin{lstlisting}[firstnumber=54, label={nonlinear:sinitpdrt},
    caption={Initialization function of Tresistor.cc}]

void Tresistor::init() throw(string&)
{
  if (!isSet(&w))
    w = defw;
  if (!isSet(&t))
    t = tnom;

  if (pdr){
    // Set the number of terminals
    setNumTerms(4);
    // Set flags
    setFlags(NONLINEAR | MULTI_REF | TR_TIME_DOMAIN);
    // Set number of states
    setNumberOfStates(2);

    if (!isSet(&r0)) {
      if (!isSet(&l))
    throw("r0 or l must be specified for the resistance");
      if (!isSet(&rsh))
    throw("rsh must be specified for the resistance");
    }
    // create tape
    IntVector var(2,0);
    var[1] = 1;
    createTape(var);
  }
\end{lstlisting}
%\linebreak 2

The \texttt{init()} function starts by doing some conditional assignments.
The condition test cannot be carried out in the constructor since the
parameters value are not known at construction time. Then if pdr == \emph{true},
\texttt{i.e.}
if it is an electro-thermal resistor, we use the member function
\texttt{setnumTerms()},
which is derived from the \textbf{Element} class, to set how many terminals
our element
has.

The next function is used to sets some flags that are
useful to classify the different elements in a circuit. Table
\ref{nltable2} summarizes the meaning of each possible flag.
%
\begin{table}[htpb]
\centerline {
\begin{tabular}{|l|l|}
\hline
Flag & Meaning \\
\hline
LINEAR    &  Linear in HB and tran2 analyses. \\
NONLINEAR &  Nonlinear in HB and tran2 analyses. \\
\hline
ONE\_REF   &  One internal reference node (\emph{e.g.} a normal element). \\
MULTI\_REF &  Two or more internal reference nodes. \\
\hline
TR\_TIME\_DOMAIN & The element is treated in time domain in convolution \\
               & transient analysis. This includes nonlinear elements). \\
TR\_FREQ\_DOMAIN & The element is treated in frequency domain in convolution
\\
               & transient analysis. This includes linear elements). \\
\hline
SOURCE    & The element is a source in tran2 analysis. \\
\hline
\end{tabular}}
\caption{Currently defined flags. \label{nltable2}}
\end{table}
%
Function \texttt{setNumberOfStates()} is used to set the number of
state variables used to implement the model.  It is important to
know that the the functions \texttt{setnumTerms(), setFlags(), \&
setNumberOfStates()} could be called from the constructor, but in
our special case, we needed to put them in the \texttt{init()}
function since we had to take some conditional decisions based on
the parameters which are not known at construction time.  Note,
for example, that it is vitally important to set the NONLINEAR
flag when an element requires the Adolc library facilities to
implement a nonlinear function.

Next, the code makes some tests on the input parameters, and then
throws an exception if some parameters are missing.

The final thing that you need to do is to create the Adolc Tape.
This is done by calling the \texttt{createTape()} function with
the proper arguments. All nonlinear elements that are derived from
the \texttt{AdolcElement} class must do this.

First the statement \textit{IntVector var(2,0);} create a vector
called \texttt{var} of two elements and initialize them to 0. Then
the statement \textit{var[1] = 1;} initialize the second element
to 1. Last, the statement \texttt{createTape(var)} creates the
tape.

\subsubsection{Building Arguments for \texttt{createTape(~)}}
We will digress from the discussion of the \texttt{Tresistor}
model for a moment in order to discuss the process of creating
state variables in some detail.  The \texttt{createTape()}
function is a member function of the \texttt{AdolcElement}
class. There are three versions of it corresponding to three
different valid function prototypes.  The choice of function
prototype depends upon the function that the model is to implement.

\begin{description}
    \item[1:]\texttt{{createTape(IntVector\& var)}}
    \begin{itemize}
        \item Takes 1 argument, a vector of state variables.
        \item This is the form used in the \textbf{Tresistor} model.
    \end{itemize}
    \item[2:]\texttt{{createTape(IntVector\& var, IntVector\& dvar)}}
    \begin{itemize}
        \item Takes 2 arguments -- a vector of state variables
        and a vector corresponding to 1st derivatives of state
        variables.
        \item This is the most widely used form, since many physical
        models use physical variables and their first derivatives.
    \end{itemize}
    \item[3:]\texttt{{createTape(IntVector\& var, IntVector\& dvar, IntVector\& d2var, \\
        IntVector\& t\_var, DoubleVector\& delay)}}
    \begin{itemize}
        \item Takes 5 arguments -- a vector of state variables,
        first derivatives, second derivatives, delayed state
        variables, and a vector of state variable delays.
        \item This form must be used 2nd derivatives or time delays
        are required in models.
    \end{itemize}
\end{description}

For example, suppose that we need to create a new \FDA element
with the following description (\textbf{f}(~) can be \textbf{v}(~)
or \textbf{i}(~)):
\newline
\newline
\[\textbf{f}(x_0, x_1, x_2, \frac{dx_0}{dt}, \frac{dx_1}{dt}, \frac{dx_2}{dt},
    \frac{d^2x_2}{dt^2}, x_0(t-\tau_0), x_2(t-\tau_2))\]
\newline
Then the arguments of the \texttt{createTape(~)} function would be as
follows:
\newline
\[\textbf{createTape(i\_vec1, i\_vec2, i\_vec3, i\_vec4, d\_vec)}\]
\newline
In this case 4 vectors of type \texttt{IntVector\&} and one vector of type
\texttt{DoubleVector\&} are needed. The first one, \texttt{i\_vec1}, consists of the
indexes of the state variables. The second one, \texttt{i\_vec2}, consists
of the indexes of the derivatives of the state variables. The
third one, \texttt{i\_vec3}, consists of the indexes of the second derivatives
of the state variables. The fourth one, \texttt{i\_vec4}, consists of the
indexes of the state variables that are delayed.  The last vector,
\texttt{d\_vec}, contains the value of the delays applied to the state
variables created by passing \texttt{i\_vec4} to \texttt{createTape(~)}.
For this example, let $\tau_0$ be 1 picosecond and $\tau_2$ be 0.1 picosecond.
To prepare this set of vectors for passing to \texttt{createTape(~)},
the portion of C++ code in Listing \ref{nonlinear:sinitctex}
would appear in the \texttt{init(~)} function prior to the call
to \texttt{createTape(~)}.

\begin{lstlisting}[firstnumber=1, label={nonlinear:sinitctex},
    caption={Example creating input vectors passed to createTape(~)}]
// Set number of states
setNumberOfStates(3);   // Three main state vars $x_0$, $x_1$, and $x_2$

// Argument vector for creating main state variables
IntVector i_vec1(3,0);  // Initialize 3-element integer array to 0
i_vec1(1) = 1;          // Set elements for additional state variables
i_vec1(2) = 2;

// Argument vector for 1st derivatives of state variables
IntVector i_vec2(3,0);  // Initialize 3-element integer array to 0
i_vec2(1) = 1;          // Set elements for additional 1st derivatives
i_vec2(2) = 2;

// Argument vector for 2nd derivatives of state variables  (*@\label{nlsecderb}@*)
IntVector i_vec3(1,0);  // Initialize 1-element integer array to 0
i_vec3(0) = 2;          // Set element to get 2nd derivative of state var $x_2$! (*@\label{nlsecdere}@*)

// Argument vector for delayed state variables
IntVector i_vec4(2,0);  // Initialize 2-element integer array to 0
i_vec4(1) = 2;          // Set elemement to get delayed copy of state var $x_2$!

// Argument vector for time delays
DoubleVector d_vec(2,0);    // Init. 2-element double array to 0
d_vec(0) = 1e-12;           // Delay state variable $x_0$ by 1 picosecond
d_vec(1) = 1e-13;           // Delay state variable $x_2$ by 0.1 picosecond
\end{lstlisting}
%\newline

Notice that \texttt{i\_vec1} and \texttt{i\_vec2} are equivalent, so in this
case it was not necessary to create \texttt{i\_vec2}, and one could instead
have simply passed \texttt{i\_vec1} to \texttt{createTape(~)} twice, in succession:
\newline
\[\textbf{createTape(i\_vec1, i\_vec1, i\_vec3, i\_vec4, d\_vec)}\]
\newline
This emphasizes the point that the variables of type \texttt{IntVector\&}
and \texttt{DoubleVector\&} are used only to create the collection of
state variables and their derivatives and delayed versions.  A survey of
the source code library for existing \FDA elements will reveal many instances
of this sort of variable re-use in calls to \texttt{createTape(~)}.

After the completion of the call to \texttt{createTape(~)}, a vector
variable \texttt{x[]} of type \texttt{adoublev\&} will have been created.
\texttt{x[]} is limited in scope to the instance of the model object.  The
number of elements in \texttt{x[]} will be equal to the number of unique
mathematical objects requested in the call to \texttt{createTape(~)}.
For the example here, there are 3 state variables, 3 first derivatives
of state variables, 1 second derivative of a state variable, and 2
delayed state variables, so the total number of elements in \texttt{x[]}
will be 9, with indices running from 0 to 8.  State variables and their
derivatives or delays are assigned indices in \texttt{x[]} consistent with
the order of argument passage to \texttt{createTape(~)}.  Table \ref{tab:state:var}
summarizes the relationship between vector input variables to \texttt{createTape(~)}
and the resulting state variables created by \texttt{createTape(~)} for the
example call described here:
\begin{table}[h]
    \caption{State Variable representation in \FDA.}
    \label{tab:state:var}
    \begin{center}
    \begin{tabular}{|c|c|}
    \hline \multicolumn{2}{|c|}{\bf Physical to State Variable Mapping}\\
%
    \hline \multicolumn{1}{|c|}{Physical Variable} & \multicolumn{1}{|c|}{~$X$~Variable} \\
%
    \hline
    \hline $x_0$ & x[$0$] \\
    \hline $x_1$ & x[$1$] \\
    \hline $x_2$ & x[$2$] \\
    \hline ${dx_0}/{dt}$ & x[$3$] \\
    \hline ${dx_1}/{dt}$ & x[$4$] \\
    \hline ${dx_2}/{dt}$ & x[$5$] \\
    \hline ${d^2x_2}/{dt^2}$ & x[$6$] \\
    \hline $x_0(t-\tau_0)$ & x[$7$] \\
    \hline $x_2(t-\tau_2)$ & x[$8$] \\
    \hline
    \end{tabular}
    \end{center}
\end{table}

% Insert stuff about novar here.
One final note on constructing the arguments to \texttt{createTape(~)}:
There are times when the implementation of a complex function may not
require the creation of state variables corresponding to each argument
in the most complex function prototype for \texttt{createTape(~) }.
For example, suppose in the previous example, that we did not need
the second derivative of the second state variable, but we still needed
the time-delayed variables.  Then instead of declaring an IntVector
called \texttt{ivec\_3} of non-zero length, we would declare vector
without a declared length, or a null vector, and pass this as an
argument to \texttt{createTape(~)} instead of \texttt{ivec\_3}.  It
is common in many \FDA elements for this null vector to be called
\texttt{novar}.  For example, if indeed we wished to ignore second
derivatives in the example in Listing \ref{nonlinear:sinitctex},
then we would alter lines \ref{nlsecderb} through
\ref{nlsecdere} as follows in Listing \ref{nonlinear:secderb} and
then call \texttt{createTape(~)} as follows:
\newline
\[\textbf{createTape(i\_vec1, i\_vec2, novar, i\_vec4, d\_vec)}\]
\newline
\begin{lstlisting}[firstnumber=14, label={nonlinear:secderb},
    caption={Declaring a null vector argument for createTape(~)}]
// Argument vector for 2nd derivatives of state variables
IntVector novar         // Initialize a null vector (no elements)
                        // Nothing to set!
\end{lstlisting}
\emph{Notice: A limitation in the functionality of the Adol-C
Library has recently been uncovered:  The second derivative
of state variables, normally invoked by using the most complex
form of the createTape prototype, has been determined to be
non-functional.  Thus, the use of novar as noted in listing
\ref{nonlinear:secderb} is mandatory.  The development of the
NAdolcElement class is an attempt to work around this limitation.
See section \ref{nonlinear:evalnote} for further information.}

Returning our attention to the \texttt{Tresistor} element, the function being
implemented was \texttt{f($x_0\,$,$\,x_1$)}, so that only one vector
filled with 0, and 1 was needed.  To better understand
the process of creating the Adolc Tape, model writers are advised to
read the implementation of the \texttt{createTape()} function which is a
member function of the \texttt{AdolcElement} Class.  Model writers
are also advised to read the ADOL-C documentation \cite{adol-c}.

\subsubsection{Constructor for Linear instance of Tresistor}
The rest of the code in the \texttt{init()} function implements
the linear part of the Constructor (invoked when pdr == \emph{false} in the
netlist instantiation) and is shown in Listing \ref{nonlinear:sinitpdrf}.
Please refer to \cite{l_element} for more details.

\begin{lstlisting}[firstnumber=last, label={nonlinear:sinitpdrf},
    caption={Init function of Tresistor.cc (continued)}]

  else{  // pdr==false
    // Set the number of terminals
    setNumTerms(2);
    // Set flags
    setFlags(LINEAR | ONE_REF | TR_FREQ_DOMAIN);
    // calculate the resistor value
    if (isSet(&r0)){
      r = r0*(1+tc1*(t-tnom)+tc2*(t-tnom)*(t-tnom));
    }
    else {
      if (!isSet(&l))
    throw("r0 or l must be specified for the resistance");
      if (!isSet(&rsh))
    throw("rsh must be specified for the resistance");
      r = rsh*((l-narrow)/(w-narrow))*(1+tc1*(t-tnom)+tc2*(t-tnom)*(t-tnom));
    }
  }
}
\end{lstlisting}
%\linebreak 2

\subsubsection{Local Reference Nodes and the getLocalRefIdx(~) function} \label{nonlinear:locrefidx}
We will digress again from the \texttt{Tresistor} model for a moment to
discuss the local reference node concept and some related
modelling conventions inherent in \FDA.

A spatially distributed circuit is a circuit which, by definition,
is much larger than the wavelengths of the signals it processes.
In such circuits, the notion of a single reference node or ground
terminal is inapt, since currents passing through such a reference
node from one piece of spatially distributed circuitry to another
cannot physically do so instantaneously.  To overcome this limitation
and permit modelling of circuits which may be spatially distributed,
the local reference node concept was developed.  Put adroitly (but
in a way that should be widely understood), local reference nodes
allow the definition of multiple "ground" terminals so that spatially
distinct circuitry can reference their local circuits to these physically
isolated reference nodes.  Readers interested in understanding more
about the theory and the implementation of the local reference node
concept should refer to \cite{lref1} and \cite{lref2}.

The availability of local reference nodes in \FDA enables the
definition of spatially distributed circuit models.  Before
discussing the coding details, which are straightforward, some
modelling conventions used in \FDA will first be described.  These
conventions are familiar to researchers in the circuit simulation
community, but may not be familiar to new \FDA users who wish to
write \FDA models.

Consider the ''Blob'' circuit shown in Figure
\ref{nonlinear:blobfig}.  Blob is a circuit that has 5 terminals
and is spatially distributed.  It has three circuit terminals
with terminal numbers 0, 1, and 3, and two reference nodes which
have terminal numbers 2 and 4, as shown in the figure.  Circuit
terminals 0 and 1 use terminal 2 as a local reference node and
circuit terminal 3 uses terminal 4 as a local reference node.
Although details of the internal operation of ''Blob'' may not
be known, it is clear that the circuits associated with
terminals 0 and 1 are spatially distinct from the circuits
associated with terminal 3.

Note that in all \FDA model \emph{source code definitions},
terminal number 0 is \emph{always} a circuit terminal.  This
may seem incorrect to SPICE users familiar with using node 0 as
the ground terminal in netlist definitions.  However, note that
a \FDA netlist still permits the use of node 0 in the netlist
for the ground terminal. Bear in mind the distinction
between coding a model definition, which we are doing here,
and instantiating a model, which is done in a netlist.

\begin{figure}[h]
\centerline{\epsfxsize=13cm \epsfbox{fREEDA_pgmman_blob.eps}}
\caption{Spatially-distributed Blob circuit.
\label{nonlinear:blobfig}}
\end{figure}

Figure \ref{nonlinear:blobfig} contains other information worth
noting.  First, note that each state variable is assigned, in
order, to each \emph{non-reference node} in the model.  Second,
notice the mention of the voltage variables \texttt{vp[]}
and the current variables \texttt{ip[]}.  These variables are
also initially created in the call to \texttt{createTape(~)},
and are made available to each model instance's \texttt{eval()}
function.  (The \texttt{eval()} function will be discussed in detail
in section \ref{nonlinear:eval} .)  For each state variable created
by the first argument in the call to \texttt{createTape()}, a
corresponding \texttt{vp[]} and \texttt{ip[]} variable is created.
Third, note that by convention, the direction of the current
variable \texttt{ip[]} is into the terminal, and the
voltage \texttt{vp[]} is defined positively at the circuit terminal
and negatively at the corresponding reference node.  Users wishing
to define currents or voltages contrary to this convention may do
so by simply negating the numerical expressions for \texttt{vp[]}
and \texttt{ip[]} that appear in the model's \texttt{eval()}
function.  Finally, note that there are no state variables associated
with the reference nodes, since voltages are defined relative to
them and Kirchoff's current law requires that the sum of the
non-reference terminal currents associated with a particular reference
node must flow into that reference node.

Those accustomed to having a "port" perspective for models will
notice that, in general, the number of ports is equal to the number
of required state variables, even if local reference terminals
are shared among one or more ports.

Returning now to the model code, implementation of the
\texttt{getLocalRefIdx()} function is very simple and straight
forward.  Refer to Listing \ref{nonlinear:sglri}.  The first
two lines must always be included. Then, we push back each
terminal in the terminal list by writing
\texttt{term\_list.push\_back(getTerminal(n));} where ''n'' is
the index of the terminal.  When we push back a local reference
node we must also push the node back into the local reference
vector by writing
\texttt{local\_ref\_vec.push\_back(m);} where ''m'' is the index
of the terminal that we want it to be a local reference.

The \texttt{getLocalRefIdx()} function definition follows in Listing
\ref{nonlinear:sglri}.

\begin{lstlisting}[firstnumber=122, label={nonlinear:sglri},
    caption={getLocalRefIdx function of Tresistor.cc}]

void Tresistor::getLocalRefIdx(UnsignedVector& local_ref_vec,
                TerminalVector& term_list)
{
  // Make sure the vectors are empty
  term_list.erase(term_list.begin(), term_list.end());
  local_ref_vec.erase(local_ref_vec.begin(), local_ref_vec.end());

  // Insert vector elements
  term_list.push_back(getTerminal(0));
  term_list.push_back(getTerminal(1)); // Local reference terminal
  local_ref_vec.push_back(1); // Local reference index

  if (pdr){
    term_list.push_back(getTerminal(2));
    term_list.push_back(getTerminal(3)); // Local reference terminal
    local_ref_vec.push_back(3); // Local reference index
  }

}
\end{lstlisting}
Finally, note that if a model has only 1 local reference node, then it is not
necessary to declare the getLocalRefIdx function at all in the header file and
it may be omitted from the source code file.  Note that when getLocalRefIdx is
not used, then the highest numbered terminal is defined as the local reference
terminal.  Referring to listings \ref{nonlinear:sinitpdrt} and \ref{nonlinear:sinitpdrf}
where the call to \texttt{setNumTerms()} appears, if \texttt{n} is the argument to
\texttt{setNumTerms()}, then there will be \texttt{n} terminals numbered \texttt{0,1,..,n-1},
and when getLocalRefIdx is not used, then \texttt{n-1} will be the terminal number
of the reference node.

\subsubsection{The Eval function and using \texttt{condassign()}} \label{nonlinear:eval}

The \texttt{eval()} function definition for \texttt{Tresistor} is
shown in Listing \ref{nonlinear:seval}.

\begin{lstlisting}[firstnumber=100, label={nonlinear:seval},
    caption={Eval function of Tresistor.cc}]

void Tresistor::eval(adoublev& x,
             adoublev& vp, adoublev& ip)
{
  // x[0]: resistor voltage
  // x[1]: deltatemp in deg. Celsius

  vp[0]=x[0];
  vp[1]=x[1]+tnom+273; //vp[1]==tp[1] in Kelvin

  adouble res;
  if (isSet(&r0)){
    res = r0 * (one + tc1 * x[1] + tc2 * x[1] * x[1]);
  }
  else {
    res = rsh * ((l-narrow) / (w-narrow))
      * (one + tc1*x[1] + tc2 * x[1] * x[1]);
  }

  ip[0] = x[0] / res;
  ip[1] = - x[0] * ip[0]; //ip[1]==pp[0];
}
\end{lstlisting}
%\linebreak 2

The \texttt{eval()} function is the implementation of the device
equations. It is in here where the voltages and currents are
related to the state variables. The \texttt{eval()} function takes
three vectors of type \texttt{adoublev\&}, the first one is the
state variable vector, and it is considered as the input, the
second and the third one are the voltages and the currents, and
these two vectors are considered as output. The \texttt{eval()}
function is  called from the \texttt{createTape()} function. In
this way, the Adol-C library keeps track of all operations that
are performed on the \texttt{adouble} variables so that it can
evaluate the derivatives whenever needed. \newline
%
%subsubsection{Conditional Branching within eval() functions} \label{nonlinear:cassign}

Due to encapsulation of most of the low-level handling of
the state variables within AdolcElement, manipulation of the
state variables within the \texttt{eval()} function for an element can
be done in a straightforward fashion using the same operators
as that for floating-point arithmetic.  These operators are all
overloaded, but this is transparent in most cases.  However,
when conditional branches affecting the assignment of active
variables are to be made within an \texttt{eval()}
function, it is mandatory to use a routine with the Adol-C
library designed to facilitate conditional branching.  This
routine is called \texttt{condassign(~)}.

The \texttt{condassign(~)} routine normally takes 4 arguments
which can be of type adouble or scalars.  For users familiar
with C and C++, a function call such as \texttt{condassign(a,b,c,d)}
corresponds with the syntax of the following conditional
assignment:
\newline
\[\textbf{a = (b \textgreater$\,$ 0)? c : d}\]
\newline
%\newline
%\[\textbf{createTape(i\_vec1, i\_vec2, novar, i\_vec4, d\_vec)}\]
%\newline
For those familiar with Matlab, Listing \ref{nonlinear:cond}
shows how the same conditional branching statements might
be implemented in that language.
\lstset{language=Matlab} % Change to Matlab formatting...
\begin{lstlisting}[firstnumber=1, label={nonlinear:cond},
    caption={Matlab equivalent of \texttt{condassign()}}]
if (b > 0)
    a = c
else
    a = d
end
\end{lstlisting}
\lstset{language={[GNU]C++}} % Back to C++ formatting.

As an illustrative example, consider a rudimentary model
for a semiconductor junction diode.  Define the current
as being 0 whenever the voltage across the junction is
less than, say, 0.7 volts, and let the current be
non-zero above the threshold.  Let voltage be
the parametric state variable, so $v(t)=x(t)$, and
let the current be defined as follows:
\newline
\begin{displaymath}
i(t) = \left\{ \begin{array}{ll}
    I_s\,[exp(\frac{x(t)-0.7}{v_T})-1] & \textrm{if $x(t) > 0.7$ or $x(t) - 0.7 > 0$}\\
    0 & \textrm{if $x(t) \le 0.7$}\\
    \end{array} \right.
\end{displaymath}
\newline
In this diode equation, $I_s$ is the reverse saturation
current and $v_T$ is the "temperature voltage."  Both
are constants.  Letting \texttt{x[$0$]} be the state
variable parameter $x(t)$, \texttt{vp[$0$]} be the junction
voltage $v(t)$, and \texttt{ip[$0$]} be the current $i(t)$
across the junction, Listing \ref{nonlinear:condcall}
shows how the conditional branch for the simple
junction diode model would be implemented.
\begin{lstlisting}[firstnumber=1, label={nonlinear:condcall},
    caption={Example of branch for simple junction diode.}]
vp[0] = x[0];                           \\ voltage equals parameter
condassign(ip[0], x[0]-0.7,             \\ a, b with b's condition shifted to 0.7
            Is*(exp((x[0]-0.7)/vt)-1),  \\ c == exponential characteristic
            0);                         \\ d == 0
\end{lstlisting}
\subsubsection{The fillMNAM function}
The last two methods in the source file, in Listing
\ref{nonlinear:sfmnam}, are the routines used to
fill the MNAM of the circuit. Please refer to \cite{l_element} for
a detailed explanation on how to fill the MNAM.

\begin{lstlisting}[firstnumber=144, label={nonlinear:sfmnam},
    caption={fillMNAM function of Tresistor.cc}]

void Tresistor::fillMNAM(FreqMNAM* mnam)
{
  // Ask my terminals the row numbers
  mnam->setAdmittance(getTerminal(0)->getRC(), getTerminal(1)->getRC(),
              one/r);
}


void Tresistor::fillMNAM(TimeMNAM* mnam)
{
  // Ask my terminals the row numbers
  mnam->setMAdmittance(getTerminal(0)->getRC(), getTerminal(1)->getRC(),
               one/r);
}
\end{lstlisting}
%\linebreak 2

\subsection{Using the \texttt{cout} routine for debugging}
In the course of developing models, debugging will
inevitably be necessary.  In this section we provide a few
suggestions on using the C++ standard output routine
\texttt{cout} to facilitate debugging.  Often in model
development, modelling equations are valid over restricted
portions of the input domain.  Considering the rudimentary
p-n junction diode again, suppose the current is undefined
for voltages below the threshold voltage, rather than set
to zero.  We dispense entirely with the \texttt{condassign()}
statement, but now it would be useful to have some sort of
error notification to the user if voltages below the threshold
are present.  Listing \ref{nonlinear:coutuse} shows an
example of how the code for the rudimentary diode might
be modified to use \texttt{cout} to print such a message.
\begin{lstlisting}[firstnumber=1, label={nonlinear:coutuse},
    caption={Using \texttt{cout} to facilitate model debugging.}]
vp[0] = x[0];                           \\ voltage equals parameter
if (x[0] < 0.7)
    {
    cout << "x =" << x[0] << ".. Invalid input voltage." << endl;
    }
ip[0] = Is*(exp((x[0]-0.7)/vt)-1);      \\ exponential characteristic
\end{lstlisting}
Another use for \texttt{cout} is to validate that model parameters
are being passed as expected from the netlist into model instances.
In these cases, \texttt{cout} statements would appear in the \texttt{init()}
function call for the element and echo the parameter values of
the model instance.

Note that when an active variable is streamed to the standard
output, the real value of the variable is followed by an "(a)"
to indicate that the variable is active.  This is a result of
the overloading of the "\textless\textless"  operator by the Adol-C library.

\subsection{Modifications to the rest of the \FDA Source Files}

To be completed.

\subsection{Header File}

The header file template is shown in Listing \ref{nonlinear:hex}.

\begin{lstlisting}[firstnumber=1, label={nonlinear:hex},
    caption={Element1.h}]

// This may look like C code, but it is really -*- C++ -*-
//
// This is a generic element template
//
//              +-----+
//         o----+     +----o
//              +-----+
//

#ifndef Element1_h
#define Element1_h 1

class Element1 : public AdolcElement
{
public:

  Element1(const string& iname);

  ~Element1() {}

  static const char* getNetlistName()
  {
    return einfo.name;
  }

  // Do some local initialization
  virtual void init() throw(string&); // to perform initialization that
                                      // cannot be done by the constructor
                                      // and to create the Adolc Tape

  // Implement the getLocalRefIdx function if you are creating
  // an element with multiple reference nodes.
  // Get a vector with the indexes of the local reference nodes.
  // Get terminal pointers in term_list
  // ordered by local reference node:
  //
  // $t_0$ $t_1$ $t_2$ $t_3$
  //
  //   ^    ^
  //  LRN1 LRN2
  //
  // local_ref_vec contains: {1, 3}
  //
  virtual void getLocalRefIdx(UnsignedVector& local_ref_vec, // to implement a spatially
                                TerminalVector& term_list);  // distributed element, or an
                                                             // electro-thermal one with
                                                             // multiple reference nodes.

private:

  // Implement the eval function. It is here where you write
  // your device equations.
  virtual void eval(adoublev& x, adoublev& vp, adoublev& ip);

  // Parameter variables (replace by your parameter variables here)
  double par1, par2, ...,parn;

  // Element information
  static ItemInfo einfo;

  // Number of parameters of this element
  static const unsigned n_par;

  // Parameter information
  static ParmInfo pinfo[];

};

#endif
\end{lstlisting}
%\linebreak 2

\subsection{Class Source File}

The class source file template is shown in Listing
\ref{nonlinear:srcex}.

\begin{lstlisting}[firstnumber=1, label={nonlinear:srcex},
    caption={Element1.cc}]

#include "../network/ElementManager.h"
#include "../network/AdolcElement.h"
#include "../analysis/FreqMNAM.h"
#include "../analysis/TimeMNAM.h"
#include "Element1.h"

// Static members (set to the number of parameters of your element)
const unsigned Element1::n_par = ;

// Element information
ItemInfo Element1::einfo = {
  "tres",
  "Element1",
  "Your Name",
  "Reserved for documentation use"
};

// Parameter information
ParmInfo Element1::pinfo[] = {
  {"par1", "Parameter 1 (Ohms)", TR_DOUBLE, true},
  {"par2", "Parameter 2 (The Unit)", TR_DOUBLE, false}
     .                                                 // Note that the sequence of
     .                                                 // parameters here should be
     .                                                 // the same as in the constructor
  {"parn", "Parameter n (The Unit)", TR_BOOLEAN, false}
};

Element1::Element1(const string& iname) : AdolcElement(&einfo, pinfo, n_par, iname)
{
  // Set parameters
  paramvalue[0] = &par1;
  paramvalue[1] = &(par2 = 1e-12); // Example of how to set a default value.
        .
        .                           // Note that the sequence of parameters here
        .                           // should be the same as in the struct pinfo[]
  paramvalue[n] = &(parn = false);
}

void Element1::init() throw(string&)
{
    // Set the number of terminals (Set to the number of terminals of
    // your element)
    setNumTerms(4);

    // Set flags. If your element has multiple reference nodes,
    // set the second flag to MULTI_REF
    setFlags(NONLINEAR | ONE_REF | TR_TIME_DOMAIN);

    // Set number of states
    setNumberOfStates(2);

    // create the tape
    IntVector var(2,0);
    var[1] = 1;
    createTape(var);
}

void Element1::eval(adoublev& x,
                     adoublev& vp, adoublev& ip)
{
  // Implement the eval function
  // write the device equations
}

void Element1::getLocalRefIdx(UnsignedVector& local_ref_vec,
                            TerminalVector& term_list)
{
    // Make sure the vectors are empty
    // always write these functions as they are
    term_list.erase(term_list.begin(), term_list.end());
    local_ref_vec.erase(local_ref_vec.begin(), local_ref_vec.end());

    // Insert vector elements
    // push_back all the terminals in the term_list
    // push_back all the local reference terminal in the local_ref_vec
    term_list.push_back(getTerminal(0));
    term_list.push_back(getTerminal(1)); // Local reference terminal
    local_ref_vec.push_back(1); // Local reference index
    term_list.push_back(getTerminal(2));
    term_list.push_back(getTerminal(3)); // Local reference terminal
    local_ref_vec.push_back(3); // Local reference index
}
\end{lstlisting}
%\linebreak 2

\section{A Note on the Eval Routines in \FDA} \label{nonlinear:evalnote}

\noindent Most nonlinear computations require the evaluation of first and higher
derivatives of vector functions with $m$ components in $n$ real or complex
variables. \FDA uses the package ADOL-C which automatically
computes the derivatives of non-linear functions by successive implementation
of the chain rule. These calculation occur at a fraction of the time and is
relatively free of truncation errors. The interface details to the ADOL-C
library from within \FDA can be found in
path\_to\_freeda/simulator/network/NAdolcElement.cc.

Every nonlinear element in \FDA has an eval function that describes the
constitutive equations for the particular element at any instant. This
function is called from within NAdolcElement.cc during evaluation of the tape.
For more details on the tape, refer to ADOL-C documentation.

\subsection{Parameterized Device Models}

A non-linear device model can be described with the following set of equations
\begin{equation}
  \bf{v} ( t ) = \bf{v} ( x ( t ), dx / dt, \ldots,
  d^m x / dt^{^m}, x_D ( t ))
\end{equation}
\begin{equation}
  \bf{i} ( t ) = \bf{i} ( x ( t ), dx / dt, \ldots,
  d^m x / dt^{^m}, x_D ( t ))
\end{equation}
where \textbf{v}(t) and \textbf{i}(t) are vectors of voltages and currents at
the ports of the non-linear device, \textbf{x}(t) is a vector of parameters or
state variables and $\bf{x_D}$(t) is a vector of time delayed state
variables. There are several models like the microwave diode, the Gummel-Poon
BJT and Berkeley's BSIM4 which for accurate modeling, require charge to be
defined as a state variable. In some cases it may not be possible to find the
voltage as a function of charge and the only alternative is to treat charge as
a state variable.

Consider a simplified model for the microwave diode shown in Figure~\ref{diode_schematic_eval}.
\begin{figure}[h]
\centerline{\epsfbox{diode.ps}}
\caption{\label{diode_schematic_eval}Simplified schematic for a diode model}
\end{figure}
The corresponding equations for this model are
\begin{equation}
i(v) = I_s (\exp( \alpha v ) - 1 )
\end{equation}
and
\begin{eqnarray}
c_j(v) = & C_{t0}(1 - {v / \phi})^{-\gamma} + C_{d0} \exp{(\alpha^{'}v)} & if\; v \leq .8\phi\\
         & C_{t0}(.2) ^{-\gamma} + C_{d0} \exp{(\alpha^{'}v)} & if\; v \geq .8\phi
\end{eqnarray}
where $v$ is the junction voltage. The capacitor voltage $q_j$ is given by
\begin{equation}
q_j ( v ) = \int_0^v c_j ( u ) du
\end{equation}
Accurate transient analysis requires $q_j$ to be chosen as a state variable.
We need $v$ to calculate $i_1( v )$ . Since it is
not possible to solve analytically
for $v ( q )$, the alternative is to model the diode with two state
variables, namely $v$ and $q$ . The diode equations
can now be formulated in two stages. Hence for this case, we
will have two \emph{eval} functions; \emph{eval}1 and \emph{eval}2 .
In \emph{eval}1, we will calculate $i(v)$ and
$q(v)$, both are which are functions of voltage $v$. In this case, the
values of $q(v)$ are passed to \emph{eval}2, which automatically calculates $d q /
d t$. Hence we now have the current through the capacitor as
\begin{equation}
i_c = \frac{d q}{d t}
\end{equation}
All the code that performs the actual derivatives is outside the nonlinear
model and is handled automatically. Currently in \FDA, only three levels of
eval are supported. This means that the highest order of derivatives possible
is two. Most nonlinear elements do not require more than two levels of
derivatives. The entire formulation can be generalized as follows:
\begin{eqnarray}
  \rm{stage} \, 1 & : & f_1 ( x, x_D ) \nonumber \\
  &  & g_1 ( x, x_D )
\end{eqnarray}
\begin{eqnarray}
  \rm{stage} \, 2 & : & f_2 ( f_1, d g_1 / d t ) \nonumber \\
  &  & g_2 ( f_1, d g_1 / d t )
\end{eqnarray}
\begin{eqnarray}
  \vdots &  & \nonumber
\end{eqnarray}
\begin{eqnarray}
  \rm{stage} \, n & : & v_{} ( f_{n - 1}, d g_{n - 1} / d t ) \nonumber \\
  &  & i ( f_{n - 1}, d g_{n - 1} / d t )
\end{eqnarray}

\section{Contributors}
The following contributed to this chapter:
\begin{itemize}
\item[] Frank P. Hart
\item[] Houssam S. Kanj
\item[] Nikhil Kriplani
\item[] Carlos Christoffersen
\item[] Michael Steer.
\end{itemize}

%\end{document} % Comment this out for re-integration into larger doc.
