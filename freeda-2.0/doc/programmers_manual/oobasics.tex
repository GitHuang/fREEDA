%--------------------------------------------------------------------------
\chapter{Object Oriented Programming Basics} \label{app:oo}

Object oriented programming (OOP) \cite{eliens} provides a means for
abstraction in both programming and design. OOP does not deal with
programming in the sense of developing algorithms or data structures
but it must be studied as a means for the organization of programs
and, more generally, techniques for designing programs.

As the primary means for structuring a program or design, OOP provides
\emph{objects}. Objects may model real life entities, may function to
capture abstractions of arbitrary complex phenomena, or may represent
system artifacts such as stacks or graphics.  Operationally, objects
control the computation. From the perspective of program development,
however, the most important characteristic of objects is not their
behavior as such, but the fact that the behavior of an object may be
described by an abstract characterization of its interface. Having
such a characterization suffices for the design. The actual behavior
of the object may be implemented later and refined according to the
need. A \emph{class} specifies the behavior of the objects which are
its instances. Also, classes act as templates from which actual
objects may be created.  An \emph{instance} of a class is an object
belonging to that class. A procedure (or function) inside an object is
called a \emph{method}. A \emph{message} to an object is a request to
execute a method.

\emph{Inheritance} is the mechanism which allows the reuse of class
specifications. The use of inheritance results in a class hierarchy
that, from an operational point of view, decides what is the method
that will be selected in response to a message.

Finally, an important feature of OO languages is their support for
\emph{polymorphism}. This makes it possible to hide different
implementations behind a common interface.

The notion of flow diagram in procedural programming is replaced in
OOP by a set of objects which interact by sending messages to each
other.

We will briefly review what are traditionally considered to be
features and benefits of OOP. Both \emph{information hiding} (also
known as \emph{encapsulation}) and \emph{data abstraction} relieve the
task of the programmer using existing OO code, since with these
mechanisms the programmer's attention is no longer distracted by
irrelevant implementation details. The flexible dispatching behavior
of objects that lends them their polymorphic behavior is due to the
dynamic binding of methods to messages. For the C++ language,
polymorphic object behavior is effected by using \emph{virtual
functions}, for which in contrast to ordinary functions, the binding
to an actual function takes place at run time and not at compile time.

Encapsulation promotes \emph{modularity}, meaning that objects may be
regarded as the building blocks of a complex system. Another advantage
often attributed to the OOP is code reuse. Inheritance is an invaluable
mechanism in this respect, since it enables the programmer to modify
the behavior of a class of objects without requiring access to the
source code.

Although an object oriented approach to program development indeed
offers great flexibility, some of the problems it addresses are
intrinsically difficult and cannot really be solved by mechanisms
alone. For example, it is more likely to achieve a stable
modularization when shifting focus from programming to design.

C++ virtual functions \cite{todd} can have big performance penalties
such as extra memory accesses or the possibility of an unpredictable
branch so pipelines can grind to a halt (note however, that some
architectures have branch caches which can avoid this problem). There
are several research projects which have demonstrated success at
replacing virtual function calls with direct dispatches. Note however,
that virtual functions are not always bad when it comes to
performance. The important questions are: How much code is inside the
virtual function? How often is it used? If there is a lot of code
(\emph{i.e.} more than 25 flops), then the overhead of the virtual
function will be insignificant. But if there is a small amount of code
and the function is called very often (\emph{e.g.}  inside a loop),
then the overhead can be critical.


%--------------------------------------------------------------------------
\section{UML diagrams}

The Unified Modeling Language (UML)~\cite{rational,uml_tut} is a
language for specifying, visualizing, and constructing the artifacts
of software systems as well as for business modeling. The goal of UML
is to become a common language for creating models of object oriented
computer software.  It is used here to graphically illustrate the
relationship of classes using what is called a class diagram.  A class
diagram is a graph of Classifier elements with connections indicating
by their various static relationships (Figure~\ref{fig:network}). (Note
that a ``class'' diagram may also contain interfaces, packages,
relationships, and even instances, such as objects and links. Perhaps
a better name would be ``static structural diagram'' but ``class
diagram'' is shorter and its use is well established.)  A class is
drawn as a solid-outline rectangle with 3 compartments separated by
horizontal lines. The top name compartment holds the class name and
other general properties of the class (including stereotype); the
middle list compartment holds a list of attributes; the bottom list
compartment holds a list of operations.  Either or both of the
attribute and operation compartments may be suppressed. A separator
line is not drawn for a missing compartment. If a compartment is
suppressed, no inference can be drawn about the presence or absence of
elements in it.

Each instance of type {\bf Element}, for example, seems to contain an
instance of type {\bf ElementData}.  This class relationship is
indicated by the joining line.  The relationship is composition ---
indicated by the solid diamond symbol.  The arrowhead denotes that the
relationship is navigable in only one direction, \emph{i.e.}, {\bf
ElementData} does not know anything about {\bf Element}.  The
inheritance relationship in UML is depicted by the triangular
arrowhead and points to the base class.  A line from the base of the
arrowhead connects it to the derived classes, e.g. {\bf Element} is
derived from {\bf NetListItem}.

Other forms of containment do not have whole/part implications and are
called association relationships indicated by a line drawn between the
participating classes. (This relationship will almost certainly be
implemented using pointers unless it is very weak.)  If the
relationship between two classes is very weak (\emph{i.e.} very
little data is shared) then a dashed line is used.  For example, in
Figure~\ref{fig:network}, {\bf ElementManager} somehow depends upon {\bf
Diode}.  (In C++ the weak relationship is almost always implemented
using an {\tt \#include}.)

An illustration showing examples for the notation is given in
Figure \ref{fig:classdiag_ex}.
%
\begin{figure}[t]
\centerline{\epsfxsize=13cm \epsfbox{classdiagram_example.eps}}
\caption{Notation for a class diagram.}
\label{fig:classdiag_ex}
\end{figure}
%

\subsection{Collaboration Diagrams}

The collaboration diagram \cite{rational} is part of the Unified
Modeling Language (UML), a language for specifying, visualizing, and
constructing the artifacts of software systems as well as for business
modeling. The UML represents a collection of `best engineering
practices' that have proven successful in the modeling of large and
complex systems. Behavior is implemented by sets of objects that
exchange messages within an overall interaction to accomplish a
purpose. To understand the mechanisms used in a design, it is
important to see only the objects and the messages involved in
accomplishing a purpose or a related set of purposes, projected from
the larger system of which they are part for other purposes. Such a
static construct is called a collaboration. A collaboration is a set
of participants and relationships that are meaningful for a given set
of purposes. The identification of participants and their
relationships does not have global meaning. In the collaboration
diagram, each box represents an object (in this case, either a
terminal or an element). The lines between boxes represent
associations and the arrows are messages. The order in which messages
are sent is shown by the numbers.

\section{Contributors}
The following contributed to this chapter
\begin{itemize}
\item[] Carlos Christoffersen
\item[] Michael Steer.
\end{itemize}
