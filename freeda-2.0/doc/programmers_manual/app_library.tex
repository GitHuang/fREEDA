\chapter{Support Software Libraries \label{app_library_ref}}

A large number of support libraries are available (many of them
freely) and some of these are used in TRANSIM. The various libraries,
which should be of general interest to the microwave modeling
community, are described below.

\section{Solution of Sparse Linear Systems}

\emph{Sparse 1.3}\footnote{http://www.netlib.org/sparse/}
\cite{Sparse} is a flexible package of subroutines written in C used
to numerically solve large sparse systems of linear equations.  The
package is able to handle arbitrary real and complex square matrix
equations.  Besides being able to solve linear systems, it is also
able to quickly solve transposed systems, find determinants, and
estimate errors due to ill-conditioning in the system of equations and
instability in the computations. Sparse also provides a test program
that is able to read matrix equation from a file, solve it, and print
useful information (such as condition number of the matrix) about the
equation and its solution. Sparse was originally written for use in
circuit simulators and is well adapted to handling nodal- and
modified-nodal admittance matrices.

\emph{SuperLU}\footnote{http://www.nersc.gov/~xiaoye/SuperLU/} is used
in the wavelet and time marching transient analyses. It contains a set
of subroutines to numerically solve a sparse linear system ${\bf A}
{\bf x} = {\bf b}$. It uses Gaussian elimination with partial pivoting
(GEPP). The columns of ${\bf A}$ may be preordered before
factorization; the preordering for sparsity is completely separate
from the factorization. SuperLU is implemented in ANSI C. It provides
support for both real and complex matrices, in both single and double
precision.

\section{Vectors and matrices}

Most of the vector and matrix handling in Transim uses
\emph{MV++}\footnote{http://math.nist.gov/mv++/} \cite{mv++}.  This is
a small set of vector and simple matrix classes for numerical
computing written in C++. It is not intended as a general vector
container class but rather designed specifically for optimized
numerical computations on RISC and pipelined architectures which are
used in most new computer architectures. The various MV++ classes form
the building blocks of larger user-level libraries.  The MV++ package
includes interfaces to the computational kernels of the Basic Linear
Algebra Subprograms package (BLAS) which includes scalar updates,
vector sums, and dot products. The idea is to utilize vendor-supplied,
or optimized BLAS routines that are fine-tuned for particular
platforms.

The \emph{Matrix Template Library}
(MTL)\footnote{http://www.lsc.nd.edu/research/mtl/} is a
high-performance generic component library that provides comprehensive
linear algebra functionality for a wide variety of matrix formats. It
is used in the wavelet and time marching transient analyses.

As with the STL, MTL uses a five-fold approach, consisting of generic
functions, containers, iterators, adaptors, and function objects, all
developed specifically for high performance numerical linear
algebra. Within this framework, MTL provides generic algorithms
corresponding to the mathematical operations that define linear
algebra. Similarly, the containers, adaptors, and iterators are used
to represent and to manipulate matrices and vectors.

\section{Solution of nonlinear systems}

Nonlinear systems of equations in Transim are solved using the
\emph{NNES}\footnote{http://www.netlib.org/opt/} \cite{NNES}
library. This package is written in Fortran and provides Newton and
quasi-Newton methods with many options including the use of analytic
Jacobian or forward, backwards or central differences to approximate
it, different quasi-Newton Jacobian updates, or two globally
convergent methods, etc. This library is used through an interface
class ({\bf NLSInterface}), so it is possible to install a different
routine to solve nonlinear systems if desired by just replacing the
interface (four different nonlinear solvers have already been
used). The Fortran routine \emph{NLEQ1} (Numerical solution of
nonlinear (NL) equations (EQ)\footnote{Konrad-Zuse-Zentrum f\"ur
Informationstechnik Berlin (ZIB). Contact: Lutz Weimann, ZIB, Division
Scientific Computing, Department Scientific Software, e-mail:
weimann@zib.de}) can also be used as a compile option. 

\section{Fourier transform}

Fourier transformation is implemented in Transim using the
\emph{FFTW}\footnote{http://www.fftw.org} library~\cite{FFTW}. FFTW is
a C subroutine library for computing the Discrete Fourier Transform
(DFT) in one or more dimensions, of both real and complex data, and of
arbitrary input size. Benchmarks, performed on a variety of platforms
show that FFTW's performance is typically superior to that of other
publicly available FFT software. Moreover, FFTW's performance is
portable: the program performs well on most computer architectures
without modification.

\section{Automatic differentiation} \label{sec:adolc}

Most nonlinear computations require the evaluation of first and higher
derivatives of vector functions with $m$ components in $n$ real or
complex variables \cite{adol-c:96}.  Often these functions are defined
by sequential evaluation procedures involving many intermediate
variables. By eliminating the intermediate variables symbolically, it
is theoretically always possible to express the $m$ dependent
variables directly in terms of the $n$ independent
variables. Typically, however, the attempt results in unwieldy
algebraic formulae, if it can be completed at all. Symbolic
differentiation of the resulting formulae will usually exacerbate
this problem of \emph{expression swell} and often entails the repeated
evaluation of common expressions.

An obvious way to avoid such redundant calculations is to apply an
optimizing compiler to the source code that can be generated from the
symbolic representation of the derivatives in question. Exactly this
approach was investigated by Speelpenning during his Ph.D. research
\cite{speelpenning} at the University of Illinois from 1977 to
1980. Eventually he realized that at least in the cases $n = 1$ and $m
= 1$, the most efficient code for the evaluation of derivatives can be
obtained directly from the evaluation of the underlying vector
function. In other words, he advocated the differentiation of
evaluation algorithms rather than formulae. In his thesis he made the
particularly striking observation that the gradient of a scalar-valued
function (\emph{i.e.} $m = 1$) can always be obtained for no more than
five times the operations count of evaluating the function
itself. This bound is completely independent of $n$, the number of
independent variables, and allows the row-wise computation of
Jacobians for at most $5 m$ times the effort of evaluating the
underlying vector function.

Given a code for a function $F : \Re^n \rightarrow \Re^m$, automatic
differentiation (AD) uses the chain rule successively to compute the
derivative matrix. AD has two basic modes, forward mode and reverse
mode \cite{coleman}. The difference between these two is the way the
chain rule is used to propagate the derivatives. 

%
\begin{figure}[htpb]
\leavevmode
\centerline{\epsfxsize=12cm \epsfbox{newadolc.eps}}
\caption{Implementation of automatic differentiation.} \label{fig:adolc}
\end{figure}
%
A versatile implementation of the AD technique is
\emph{Adol-C}\footnote{http://www.math.tu-dresden.de/~adol-c/}
\cite{adol-c:96}, a software package written in C and C++.  The
numerical values of derivative vectors (required to fill a Jacobian in
Harmonic Balance analysis~\cite{svhb}, see Figure~\ref{fig:adolc}) are
obtained free of truncation errors at a small multiple of the run time
required to evaluate the original function with little additional
memory required.  It is important to note that AD is not numerical
differentiation and the same accuracy achieved by evaluating
analytically developed derivatives is obtained.

The {\tt eval()} method of the nonlinear element class is executed at
initialization time and so the operations to calculate the currents
and voltages of each element are recorded by Adol-C in a \emph{tape}
which is actually an internal buffer. After that, each time that the
values or the derivatives of the nonlinear elements are required, an
Adol-C function is called and the values are calculated using the
tapes.  This implementation is efficient because the taping process is
done only once (this almost doubles the speed of the calculation
compared to the case where the functions are taped each time they are
needed).  When the Jacobian is needed, the corresponding Adol-C
function is called using the same tape. We have tested the program
with large circuits with many tones, and the function or Jacobian
evaluation times are always very small compared with the time required
to solve the matrix equation (typically some form of Newton's method)
that uses the Jacobian. The conclusion is that there is little
detriment to the performance of the program introduced by using
automatic differentiation.  However the advantage in terms of rapid
model development is significant.  The majority of the development
time in implementing models in simulators, particularly harmonic
balance simulators, is in the manual development of the derivative
equations. Unfortunately the determination of derivatives using
numerical differences is not sufficiently accurate for any but the
simplest circuits. With Adol-C full `analytic' accuracy is obtained
and the implementation of nonlinear device models is dramatically
simplified. From experience the average time to develop and implement
a transistor model is an order of magnitude less than deriving and
coding the derivatives manually. Note that time differentiation, time
delay and transformations are left outside the automatic
differentiation block. The calculation speed achieved is approximately
ten times faster than the speed achieved by including time
differentiation, time delay and transformations inside the block.

\begin{table}
\leavevmode
   \caption{\label{nodal:table:compare}Comparison of techniques for
solving the nodal admittance formulation of the network
equations. $N$ is the number of nodes, $M_N$ is the number
of floating point multiplication and division operations
required for an $N$ node circuit.}\vspace*{0.1in}
\centering
\begin{tabular}{|lllp{3in}|}
\hline
METHOD  &  $M_5$ & $M_{100}$ & NOTES \\
\hline
\hline
\multicolumn{4}{|c|}{MATRIX REDUCTION METHODS}\\
\hline
Pivoted Reduction &  58 & 80,848&
  Very efficient. \\
  &&& $M_N =  (2N^3 - 3N^2 + N - 6)/3$\\
  \hline
Partitioned Analysis &  81  &124,416&
  Variant of pivoted reduction method advantageous for repeated analysis\\
  &&& $M_N =  N^3 - 12N + 16$\\
  \hline
  \hline
\multicolumn{4}{|c|}{METHODS INVOLVING THE SOLUTION OF SIMULTANEOUS EQUATIONS}\\
\hline
Direct Inversion    & 125 & 125,000 & e.g. using Gauss-Jordan Techniques.\\
  &&& $M_N =  N^3$\\
  \hline
Sum of Products     &  505 & 1.49 $10^{66}$&
  Inefficient.
  Evaluates ${\bf Y}^{-1} = {1 \over \Delta} (\mbox{adj} {\bf Y})$\\
                    &   &&
  \mbox{adj} {\bf Y} is composed of cofactors of {\bf Y}. \\
  &&& $M_N \approx N!(N-1) + N^2$\\
  \hline
Laplacian expansion & 232   & 5.22 $10^{64}$&
  As above. Uses cofactors of {\bf Y} to evaluate determinant $\Delta$\\
  &&& $M_N \approx 1.71828! + N^2$\\
  \hline
Triangularization   & 65    & 41,650&
  As above, This is a quick way to evaluate $\Delta$\\
  &&& $M_N \approx (N^3 - N)/3 + N^2$\\
  \hline
LU factorization    & 65    & 44,150&
  Finds lower (L) and upper (U) triangular matrices so that {\bf Y} = LU.
  Factorization can be reused for different node current sources (an additional
  $N^2$ operations required). \\
  &&& $M_N = (N^3+3N^2-N)/3$\\
  \hline
Gaussian Elimination & 65   & 44,150&
  Efficient. Yields node voltages and node current sources.\\
  &&& $M_N = (N^3+3N^2-N)/3$\\
  \hline
Gauss-Jordan        &  85   & 64,975&
  Efficient. Simpler than Gaussian elimination.\\
  &&& $M_N = (N^3+3N^2-N)/2$\\
\hline
\hline
\multicolumn{4}{|c|}{CASCADED TWO-PORT ANALYSIS}\\
\hline
Two-Port Analysis   &  40  & 400 &
  Very efficient, applicable to a restricted class of problems. Yields two-port
  parameters of network. \\
  &&& $M_N = 8N$\\
\hline
\hline
\end{tabular}
\end{table}
