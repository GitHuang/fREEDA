%
% .NOISE
%
\statement{.NOISE}{Small-Signal Noise Analysis}

In noise  analysis the noise generated by active devices and resistors
is evaluated.

\form{{\tt .NOISE } OutputVoltageSpecification InputSourceName OutputInterval}

\spicethreeform{{\tt .NOISE}
OutputVoltageSpecification InputSourceName {\tt DEC}\\{\tt +}
FrequenciesPerDecade FStart FStop \B OutputInterval \E \\[0.1in]
{\tt .NOISE} OutputVoltageSpecification InputSourceName {\tt OCT}\\{\tt +}
FrequenciesPerOctave FStart FStop \B OutputInterval \E \\[0.1in]
{\tt .NOISE} OutputVoltageSpecification InputSourceName {\tt LIN}\\{\tt +}
NumberPoints FStart FStop OutputInterval}

\begin{widelist}
\item[{\it OutputVoltageSpecification}]
     is a specification of an output voltage which is to be the output
     of the noise analysis. It acts as a summing point for the
     noise contributions of the individual noise current generators.
     The noise voltage appearing at the output for each noise generator is summed
     in the RMS sense.
     Any voltage specification may be used including
     the voltage at a node, e.g. using V(5), or the voltage between
     nodes, e.g. using V(5,3).
     The noise is reported in units of V/$\sqrt{\mbox{Hz}}$.

\item[{\it InputSourceName}]
     is the name of the independent voltage ({\tt V}) or current ({\tt I})
     source that is to be the input reference source to which equivalent
     input noise is referred. The input source does not produce noise itself.
     If the input source is an independent voltage source then the equivalent
     input noise is reported in units of V/$\sqrt{\mbox{Hz}}$.
     If the input source is an independent current source then the equivalent
     input noise is reported in units of A/$\sqrt{\mbox{Hz}}$.

\item[{\it OutputInterval}]
     is the optional output reporting interval at which the values of the noise
     current generators internal to the elements of the circuit are reported.
     The report is produced every {\it OutputInterval}~th frequency.
     If zero or omitted no detailed output is produced.

\item[{\tt DEC}] is the decade sweep keyword specifying that the
     noise analysis is to be evaluated at a number of frequency points.
     The frequency is swept logarithmically by decades.

\item[{\it FrequenciesPerDecade}]
     specifies the number of frequencies per decade.

\end{widelist}

\begin{widelist}
\item[{\tt OCT}] is the octave sweep keyword specifying that the
     noise analysis is to be evaluated at a number of frequency points.
     The frequency is swept logarithmically by octaves.

\item[{\it FrequenciesPerOctave}]
     specifies the number of frequencies per octave.

\item[{\tt LIN}] is the linear frequency sweep keyword. At each frequency a
     noise analysis is performed.

\item[{\it NumberPoints}]
     specifies the total number of frequencies in a linear sweep.

\item[{\it FStart}]
     is the starting frequency of the sweep.

\item[{\it FStop}]
     is the stopping frequency of the sweep.

\end{widelist}

\example{.NOISE V(5) VIN DEC 10 1kHZ 100Mhz\\
         .NOISE V(5,3) V1 OCT 8 1.0 1.0e6 1}

\example{.NOISE V(5) VIN \\
         .NOISE V(5,3) I1}



\note{
\item
In noise  analysis the noise generated by active devices and resistors
is evaluated as a noise spectral density.  The densities are integrated over the 
the range of frequencies to obtain a gross noise measure for the circuit
(for the specified frequency range).  The finer the frequency spacing the
more accurate will be the noise analysis.  The noise contributions of
individual noise generators are summed at the node or branch specified by
{\it OutputVoltageSpecification}.
The noise at this output port is reported as well as the equivalent input
noise (the output noise referred to the input) at the input source identified
by {\it InputSourceName}.

%The  calculated noise  value
%corresponds to the variance of the circuit variable (voltage or current)
%viewed as a stationary Gaussian process.

\item
Two types of output are produced by noise analysis:
\begin{enumerate}
\item noise spectral  density  versus frequency, and
{
\item total integrated noise  over  the  specified  frequency  range.
}
\end{enumerate}

{\item
In \spicethree\ the \ac\ frequency sweep for noise analysis must be specified in
the {\tt .NOISE} statement. In \pspice\ the frequency sweep specified in the
{\tt .AC statement} is used.}

\item
{
The noise table is produced while analysis is being performed.}
Noise reporting
is produced using the {\tt .PRINT} and {\tt .PLOT} statements.

\item
Noise is generated by resistors and by semiconductor devices. Resistors generate
thermal noise while the noise model of semiconductor devices includes thermal
noise, shot noise and flicker noise.  Noise models of individual elements are
discussed in the element catalog beginning on page \pageref{chapter:element}.
}

\noindent
\underline{\it Two-Port Noise and Gain Calculations}

This is supported in a few versions of Spice.

For all circuits the output {\tt ONOISE} and effective input noise
{\tt INOISE} are calculated. Also the voltage gain {\tt GAIN} is calculated
as the output voltage devided by the voltage across the source.
Extended gain and noise parameters are available if the circuit is defined
as a two port.  The two-port parameters that are calculated are defined in
terms of the signal and noise powers shown in Fig. 
\ref{fig:signal:noise:two:port}
\begin{figure}
\centerline{\epsfxsize=5.5in\pfig{signal_noise_two_port.id}}
\caption{Signal and noise definitions for a two-port.
\label{fig:signal:noise:two:port}}
\end{figure}
The actual noise and signal powers that are delivered to the circuit are
$P_i$ and $N_i$.
Also the actual noise and signal powers that are delivered to the load
resistance are $P_o$ an $N_o$.
The available noise and signal powers are the powers that would be
delivered to the circuit with ideal lossless matching networks.
That is, when the input and output impedances equal the source ($R_S$) and
load ($R_L$) resistances --- so that
$P_i = P_{Ai}$ and $P_o = P_{Ao}$; and
$N_I = N_A$ and $N_O = N_A$.

Two-port noise analysis yields the following quantities which can be
specified in the {\tt .PRINT} and {\tt .PLOT} statements:\\
      \offset\begin{tabular}{lclp{3.5in}}
      {\tt ONOISE} &-&$V_{NO}$&RMS output noise voltage in
                          V/$\sqrt{\mbox{Hz}}$\\
      {\tt INOISE} &-&$V_{NO}$&RMS equivalent input noise voltage in
                          V/$\sqrt{\mbox{Hz}}$\\
      {\tt GAIN} &-&$G$&voltage gain\\
      {\tt GT} &-&$G_T$&transducer gain\\
      {\tt NF} &-&$NF'$&spot noise factor\\
      {\tt SNR} &-&$\mbox{SNR}_i$&output voltage signal-to-noise ratio\\
      {\tt TNOISE} &-&$T_{\mbox{NOISE}}$&output noise temperature in celsius.
      \end{tabular}

All of the quantities can be output in dB with the exception of
$T_{\mbox{NOISE}}$.
When {\tt DB(NF)} is used the spot noise figure is obtained.

The transducer gain, $G_T$, for the two port is
defined as
\begin{equation}
G_T = P_o / P_{Ai}
\end{equation}

The most common measure of the noise performance of a two port is the noise
factor or noise figure.  The spot noise factor {NF'}
of a linear two-port network
is defined as the ratio of the noise power delivered by the network to the 
load impedance to the fraction of the noise power due to the input
termination alone.   This noise is calculated with the input termination,
 $R_S$, at the standard temperature $T_0$ = 290~K.  Note that this differs
 from the default analysis temperature of spice which is 300~K or 16.85~c.
The noise power delivered to the output is the total noise power indicated
by {\tt ONOISE} less the noise power contributed to the output by $R_L$ since it
is not part of the two-port.  These subtractions must be done using squared
voltage quantities because the noises are uncorrelated.  The noise power at the output
due to $R_S$ is the voltage gain squared multiplied by the square of the noise
voltage in series with $R_S$.  The noise factor calculated by \sspice\ is
the spot noise factor as the noise powers are not averaged over
frequency.  The spot noise factor is
\begin{equation}
NF' = { { _0V_{NO}^2 - V_{NO,RL}^2 } \over {_0V_{NO,RS}^2} }
\end{equation}
Here the leading zero subscript indicates that the noise is calculated with
$R_S$ at $T_0$.
$_0V_{NO}$ is the output noise voltage with $R_S$ at $T_0$,
$V_{NO,RL}$ is the component due to the noise generated by $R_L$ and
$_0V_{NO,RS}$ is the component due to the noise generated by $R_S$ at
$T_0$.

The noise temperature in Kelvin is
\begin{equation}
T_{NOISE} = T_K (NF' -1)
\end{equation}
where $T_K$ is the analysis temperature in Kelvin.

The output voltage signal-to-noise ratio is the ratio of the signal voltage to
the noise voltage:
\begin{equation}
\mbox{SNR}_o = {{V''_O }\over{ V''_{NO} \sqrt{2}}}
\end{equation}
where the $\sqrt{2}$ is required since noise voltages are specified in RMS
terms but the signal voltages are specified as a peak voltage.
$V_O$ is the signal voltage at the output taking into account the
signal-to-noise ratio, $\mbox{SNR}_i$, at the input of the circuit.
$\mbox{SNR}_i$ can either be specified on the voltage source line or, if
not, calculated usnig the thermal voltage of $R_S$.


Three parameters affect the results of the noise analysis.  These are the
source resistance $R_S$, the load resistance, $R_L$ and the input signal to
noise ratio, $\mbox{SNR}_i$.  The values used for the parameters are as follows:
\\\offset
\begin{tabular}{rcp{4in}}
$R_S$ &=& The resistance of port 1 ({\tt ZL}, or if Port 1 is not defined,\\
      &=& The {\tt RS} resistance specified on the source line, or if not
          specified.\\
      &=& 50~$\Omega$.\\
\\
$R_L$ &=& The resistance of port 2 ({\tt ZL}, or if Port 2  is not defined.\\
      &=& The {\tt RL} resistance specified on the source line, or if not
          specified,\\
      &=& 50~$\Omega$.\\
      \\
$\mbox{SNR}_i$ &=& The {\tt SNR} specified on the source line, or if not
             specified.\\
      &=& it is calculated as the signal voltage specified on the input line
          devided by the thermal noise voltage of $R_S$ with appropriate
          correction for the difference between RMS and peak quantities.\\
\end{tabular}
\\\offset

\noindent
{\it Example of Two-Port Noise and Gain Analysis}

The netlist for performing a two-port noise and gain analysis of the circuit 
in Fig. \ref{fig:noiser} is as follows.
\begin{figure}
\centerline{\epsfxsize=3in\pfig{noiser_cir.id}}
\caption{Circuit used as an example for specifying noise analysis.
\label{fig:noiser}}
\end{figure}

{\tt
\noindent
Gain and noise analysis of resistive attenuator\\
vin 1 0 AC 1u RS=50 SNR=100\\
RS 1 2 50\\
R1 2 0 55\\
R2 2 3 500\\
R3 3 0 55\\
RL 3 0 50\\
*The following sets the analysis temperature to the standard temperature\\
.TEMP 16.85\\
.AC DEC 1 1MEG 2G\\
.NOISE V(3,0) VIN 1\\
.PRINT NOISE nf db(nf) gt db(gt) gain snr inoise onoise\\
.END}

\medskip

The example below performs the same analysis using ports and also prints
the scattering parameters of the circuit.

{\tt 
\noindent
Gain and noise analysis of resistive attenuator using ports.\\
vin 1 0 AC 1u SNR=100\\
PIN PNR=1 ZL=50\\
R1 2 0 55\\
R2 2 3 500\\
R3 3 0 55\\
POUT PNR=2 ZL=50\\
*The following sets the analysis temperature to the standard temperature\\
.TEMP 16.85\\
.AC DEC 1 1MEG 2G\\
.NOISE V(3) VIN 1\\
.PRINT NOISE nf db(nf) gt db(gt) gain snr inoise onoise\\
.PRINT AS S(1,1) S(1,2), S(2,1), S(2,2)\\
.END}

