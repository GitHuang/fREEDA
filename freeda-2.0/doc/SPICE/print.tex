%
% .PRINT
%
\statement{.PRINT}{Print Specification}

The print specification controls the information that is reported as the result
of various analyses.

\form{{\tt .PRINT TRAN } OutputSpecification \B OutputSpecification $\ldots$
      \E\\[0.1in]
      {\tt .PRINT AC   } OutputSpecification \B OutputSpecification $\ldots$
      \E\\[0.1in]
      {\tt .PRINT DC   } OutputSpecification \B OutputSpecification $\ldots$
      \E\\[0.1in]
    {
      {\tt .PRINT DISTO} DistortionOutputSpecification
              {\tt ( }DistortionReportType{ \tt )}\\{\tt +}
              \B DistortionOutputSpecification
              {\tt ( }DistortionReportType{ \tt )} $\ldots$ \E}
     }

\begin{widelist}

\item[{\tt TRAN}] is the keyword specifying that this {\tt .PRINT} statement
     controls the reporting of results of a transient analysis initiated by the
     {\tt .TRAN} statement.

\item[{\tt AC}] is the keyword specifying that this {\tt .PRINT} statement
     controls the reporting of results of a small-signal \ac\ analysis initiated by
     the {\tt .AC} statement.

\item[{\tt DC}] is the keyword specifying that this {\tt .PRINT} statement
     controls the reporting of results of a \dc\ analysis initiated by the
     {\tt .DC} statement.

\item[{\tt NOISE}] is the keyword specifying that this {\tt .PRINT} statement
     controls the reporting of results of a noise analysis initiated by the
     {\tt .NOISE} statement.

{
\item[{\tt DISTO}] is the keyword specifying that this {\tt .PRINT} statement
     controls the reporting of results of a small-signal \ac\ distortion analysis
     initiated by the {\tt .DISTO} statement.
     }

\end{widelist}

\begin{widelist}
\item[{\it OutputSpecification}] specifies the voltage or current to be reported
     in a tabular format of up to 8 columns plus an initial column with the
     sweep variable.

     Many forms of {\tt OutputSpecification} are supported by
     \pspice\ .
     {Below is a description of the basic forms that are
     supported both by \spicetwo\ and \pspice\ .  }
     A comprehensive
     description of {\tt OutputSpecification} supported by
     \pspice\ is given in the section on output specification
     on page \pageref{.PRINToutputspecification}.
     %Up to 8 {\it OutputSpecification}s
     \\[0.1in]

{
     \underline{Voltages} may be specified as an absolute voltage at a node:
     {\tt V({\it NodeName})} or
     the voltage at one node with respect to that at another node,
     e.g. {\tt V(Node1Name,Node2Name)}.

     For the reporting of the results of an \ac\ analysis the following outputs
     can be specified by replacing the {\tt V} as follows:\\
      \offset\begin{tabular}{lcp{3.5in}}
      {\tt VR} &-&real part\\
      {\tt VI} &-&imaginary part\\
      {\tt VM} &-&magnitude\\
      {\tt VP} &-&phase\\
      {\tt VDB} &-&$10\,\mbox{log}(10\,magnitude)$
      \end{tabular}\\
      In \ac\  analysis the default is {\tt VM} for magnitude.\\[0.1in]

     \underline{Currents} are specified by referencing the name of the voltage
     source through which the current is measured, e.g.
     {\tt I(V{\it VoltageSourceName})}.

     For the reporting of the results of an \ac\ analysis the following outputs
     can be specified by replacing the {\tt I} as follows:\\
      \offset\begin{tabular}{lcp{3.5in}}
      {\tt IR} &-&real part\\
      {\tt II} &-&imaginary part\\
      {\tt IM} &-&magnitude\\
      {\tt IP} &-&phase\\
      {\tt IDB} &-&$10\,\mbox{log}(10\,magnitude)$
      \end{tabular}
      In \ac\  analysis the default is {\tt IM} for magnitude.
      }

\item[{\it NoiseOutputSpecification}]
     specifies the noise measure to be reported. The two options are
     {\tt ONOISE} which reports the output noise and
     {\tt INOISE} which reports the equivalent input noise.
     See the {\tt .NOISE} statement on page \pageref{.NOISEstatement}
     for a detailed explanation.

     It  must be one of the following:\\
      \offset\begin{tabular}{lcp{3.5in}}
      {\tt ONOISE} &-&RMS output noise voltage\\
      {\tt DB(ONOISE)} &-&output noise voltage in dB (= 20 log({\tt ONOISE})\\
      {\tt INOISE} &-&RMS equivalent input noise voltage\\
      {\tt DB(INOISE)} &-&equivalent input noise voltage in dB (= 20 log({\tt INOISE})\\
      {\tt GAIN} &-&voltage gain\\
      {\tt DB(GAIN)} &-&voltage gain in dB (= 20 log({\tt GAIN})\\
      {\tt GT} &-&transducer gain\\
      {\tt DB(GT)} &-&transducer gain in dB (= 10 log({\tt GT})\\
      {\tt NF} &-&spot noise factor\\
      {\tt DB(NF)} &-& spot noise figure (= 10 log({\tt NF})\\
      {\tt SNR} &-&output signal-to-noise ratio\\
      {\tt DB(SNR)} &-&output signal-to-noise ratio in dB (= 20 log({\tt SNR})\\
      {\tt TNOISE} &-&output noise temperature.
      \end{tabular}

\item[{\it SParameterOutputSpecification}]
      specifies the S-parameter output variables that are to be printed.
      Each variable must have one of the following forms:\\
      \offset\begin{tabular}{lcp{3.5in}}
      {\tt S(i,j)} &-&Magnitude of $S_{ij}$\\
      {\tt SR(i,j)} &-&Real part of $S_{ij}$\\
      {\tt SI(i,j)} &-&Imaginary part of $S_{ij}$\\
      {\tt SP(i,j)} &-&Phase of $S_{ij}$ in degrees\\
      {\tt SDB(i,j)} &-&Magnitude of $S_{ij}$ in dB
                         (= 20 log({\tt {\tt S(i,j)}}))\\
      {\tt SG(i,j)} &-&Group delay of $S_{ij}$
      \end{tabular}\\[0.1in]
      The port numbers are $i,j$  which are specified using the {\tt PNR} keyword
      when the port element is specified.

\end{widelist}

{
\begin{widelist}
\item[{\it DistortionOutputSpecification}]
     specifies the distortion component to be reported
     in a tabular format of up to 8 columns plus an initial column with the sweep
     variable.  The {\it DistortionOutputSpecification} is one of the
     following:\\
      \offset\begin{tabular}{lcp{3.5in}}
      {\tt HD2} &-&the second harmonic distortion\\
      {\tt HD3} &-&the second harmonic distortion\\
      {\tt SIM2} &-&the sum frequency intermodulation component\\
      {\tt DIM2} &-&the difference frequency intermodulation component\\
      {\tt DIM3} &-&the third order intermodulation component
      \end{tabular}\\[0.1in]
      See the {\tt .DISTO} statement on page \pageref{.DISTOstatement} for
      a description of these distortion components.

\item[{\it DistortionReportType }]
     specifies the format for reporting the distortion components.
     It  must be one of the
     following:\\
      \offset\begin{tabular}{lcp{3.5in}}
      R &-&real part\\
      I &-&imaginary part\\
      M &-&magnitude\\
      P &-&phase\\
      DB &-&$10\,\mbox{log}(10\,magnitude)$
      \end{tabular}\\[0.1in]
      The default is {\tt M} for magnitude.
\end{widelist}
      }

\example{
.PRINT TRAN V(10) V(5,3) I(VIN)\\
.PRINT AC VM(10) VR(5,3) IP(VLOAD)\\
.PRINT DC V(10) V(5,3) I(VIN)\\
.PRINT NOISE ONOISE INOISE DB(ONOISE) DB(INOISE)\\
.PRINT NOISE GAIN DB(GT) DB(NF) SNR TNOISE\\
.PRINT AS SDB(1,1) SP(1,1) SDB(1,2) SP(1,2)\\
{.PRINT DISTO HD2 HD3 SIM2(DB)}
}

\note{
\item There can be any number of {\tt .PRINT} statements.

\item The number of significant digits of the results reported is
{\tt NUMDGT} which is set in a {\tt .OPTIONS} statement (see page
\pageref{.OPTIONNUMDGT}).

\item The current through any element can be found by inserting independent
voltage sources in series with the elements.
This is generally what is required in \spicetwo and \spicethree.
However \pspice\ supports direct specification of the voltage and currents
of most elements. See the section on page
\pageref{.PRINToutputspecification}.
}


\specialnote{Output Specification{ for \pspice}}{}
\vspace{0.1in}
\label{.PRINToutputspecification}
\pspice\ supports a relatively large variety of output specifications compared
to that available with \spicetwo\ and \spicethree . The output
specifications described in the following can be used {\tt .PRINT}
and {\tt .PLOT} statements.  The various forms of output specifications
enable the current and voltages of virtually all devices to be
examined>\\[0.1in]
\underline{{\tt DC} and {\tt TRAN} Reporting}\\
The output specifications available for the \dc\ sweep and transient
analyses are

\begin{widelist}

\item[{\tt I(}{\it DeviceName}{\tt )}]
Current through a two terminal device (such as a resistor {\tt R} element) or
the output of a controlled voltage or current source. e.g.
{\tt I(R22)} is the current flowing through resistor {\tt R22}
from node $N_1$ to $N_2$ of {\tt R22}.

\item[{\tt I}{$\,$TerminalName}{\tt (}{\it DeviceName}{\tt )}]
Current flowing into terminal named {\it TerminalName} (such as {\tt B} for
gate) from the device named {\it DeviceName} (such as {\tt Q12}).
e.g. {\tt IB(Q12)}

\item[{\tt I}{$\,$PortName}{\tt (}{\it TransmissionLineName}{\tt )}]
Current at port named {\it PortName} (either {\tt A} or {\tt B}) of the
transmission line device named   {\it TransmissionLineName}

\item[{\tt V(}{\it NodeName}{\tt )}]
Voltage at a node of name {\it NodeName}.

\item[{\tt V(}{$n_1,n_2$}{\tt )}]
Voltage at node $n_1$ with respect to the voltage
 at node $n_2$.

\item[{\tt V(}{\it DeviceName}{\tt )}]
Voltage across a two terminal device (such
as a resistor {\tt R} element) or
at the output of a controlled voltage or current source.

\item[{\tt V}{$\,$TerminalName}{\tt (}{\it DeviceName}{\tt )}]
Voltage at terminal named {\it TerminalName} (such as {\tt G} for gate) of the
device named   {\it DeviceName} (such as {\tt M12}). e.g. {\tt VG(M12)}

\item[{\tt V}{$\,$TerminalName1$\,$TerminalName2}{\tt (}{\it DeviceName}{\tt )}]
Voltage at terminal named {\it TerminalName1} (such as {\tt G} for gate) th respect to the terminal name {\it TerminalName2} (such as {\tt S} for source) of the
device named   {\it DeviceName} (such as {\tt M12}). e.g. {\tt VGS(M12)}

\item[{\tt V}{$\,$PortName}{\tt (}{\it TransmissionLineName}{\tt )}]
Voltage at port named {\it PortName} (either {\tt A} or {\tt B}) of the
transmission line device named   {\it TransmissionLineName} (such as
{\tt T5}). e.g. {\tt VA(M5)}

\end{widelist}

\hspace*{\fill}\\[0.1in]

\noindent\underline{Two Terminal Device Types Supported for \dc and
Transient Analysis Reporting}\\[0.1in]
The single character identifier  for the following elements
as well as the rest of the device name can be used as the {\it DeviceName}
in the {\tt I(}{\it DeviceName}{\tt )}  and {\tt I(}{\it DeviceName}{\tt )}
output specifications.\\
\hspace*{\fill}
\begin{tabular}{|p{1in}|p{3in}|}
\hline
Element Type & Description\\
\hline
{\tt C} & capacitor\\
{\tt D} & diode\\
{\tt E} & voltage-controlled voltage source\\
{\tt F} & current-controlled current source\\
{\tt G} & voltage-controlled current source\\
{\tt H} & current-controlled voltage source\\
{\tt I} & independent current source\\
{\tt L} & inductor\\
{\tt R} & resistor\\
{\tt V} & independent voltage source\\
\hline
\end{tabular}\\[0.1in]

\noindent
\underline{Multi-Terminal Device Types Supported for \dc and Transient Analysis
Reporting}\\[0.1in]
The single character identifier for the following elements as well as the rest
of the device name can be used as the {\it DeviceName}
in the
{\tt I}{$\,$TerminalName}{\tt (}{\it DeviceName}{\tt )},
{\tt V}{$\,$TerminalName}{\tt (}{\it DeviceName}{\tt )} and\newline
{\tt V}{$\,$TerminalName1$\,$TerminalName2}{\tt (}{\it DeviceName}{\tt )}
output specifications.\\[0.1in]
\hspace*{\fill}
\begin{tabular}{|p{1in}|p{3in}|}
\hline
Element Type & Description\\
\hline
{\tt B} & GaAs MESFET\ \ \ Terminals:\newline
\hspace*{1in}D --- drain\newline
\hspace*{1in}G --- gate\newline
\hspace*{1in}S --- source\\
{\tt J} & JFET\ \ \ Terminals:\newline
\hspace*{1in}D --- drain\newline
\hspace*{1in}G --- gate\newline
\hspace*{1in}S --- source\\
{\tt M} & MOSFET\ \ \ Terminals:\newline
\hspace*{1in}B --- bulk or substrate\newline
\hspace*{1in}D --- drain\newline
\hspace*{1in}G --- gate\newline
\hspace*{1in}S --- source\\
{\tt Q} & BJT\ \ \ Terminals\newline
\hspace*{1in}C --- collector\newline
\hspace*{1in}B --- base\newline
\hspace*{1in}E --- emitter\newline
\hspace*{1in}S --- source\\
\hline
\end{tabular}\\[0.1in]

\noindent
\underline{{\tt AC} Reporting}\\
The output specifications available for reporting the results of an
\ac\ frequency sweep analysis includes all of the specification formats discussed
above for \dc\ and transient analysis together with a number of possible
suffixes:\\
      \offset\begin{tabular}{lcp{3.5in}}
      {\tt DB} &-&$10\,\mbox{log}(10\,magnitude)$\\
      {\tt M} &-&magnitude\\
      {\tt P} &-&phase\\
      {\tt R} &-&real part\\
      {\tt I} &-&imaginary part\\
      {\tt G} &-&group delay = ${\partial \phi/ \partial f}$\newline
                 where $\phi$ is the phase of the quantity being reported
         and $f$ is the analysis frequency.
      \end{tabular}\\
In \ac\  analysis the default suffix is {\tt M} for magnitude.\\[0.1in]

\noindent\underline{Two-Terminal Device Types Supported for
\ac\ Reporting}\\[0.1in]

The single character identifier  for the following elements
as well as the rest of the device name can be used as the {\it DeviceName}
in the {\tt I(}{\it DeviceName}{\tt )}  and {\tt I(}{\it DeviceName}{\tt )}
output specifications.\\
\hspace*{\fill}
\begin{tabular}{|p{1in}|p{3in}|}
\hline
Element Type & Description\\
\hline
{\tt C} & capacitor\\
{\tt D} & diode\\
{\tt I} & independent current source\\
{\tt L} & inductor\\
{\tt R} & resistor\\
{\tt V} & independent voltage source\\
\hline
\end{tabular}\\[0.1in]

\noindent
\underline{Multi-Terminal Device Types Supported for \dc and Transient Analysis
Reporting}\\[0.1in]
The single character identifier for the following elements as well as the rest
of the device name can be used as the {\it DeviceName}
in the
{\tt I}{$\,$TerminalName}{\tt (}{\it DeviceName}{\tt )},
{\tt V}{$\,$TerminalName}{\tt (}{\it DeviceName}{\tt )} and\newline
{\tt V}{$\,$TerminalName1$\,$TerminalName2}{\tt (}{\it DeviceName}{\tt )}
output specifications.\\[0.1in]
\hspace*{\fill}
\begin{tabular}{|p{1in}|p{3in}|}
\hline
Element Type & Description\\
\hline
{\tt B} & GaAs MESFET\ \ \ Terminals:\newline
\hspace*{1in}D --- drain\newline
\hspace*{1in}G --- gate\newline
\hspace*{1in}S --- source\\
\hline
{\tt J} & JFET\ \ \ Terminals:\newline
\hspace*{1in}D --- drain\newline
\hspace*{1in}G --- gate\newline
\hspace*{1in}S --- source\\
\hline
{\tt M} & MOSFET\ \ \ Terminals:\newline
\hspace*{1in}B --- bulk or substrate\newline
\hspace*{1in}D --- drain\newline
\hspace*{1in}G --- gate\newline
\hspace*{1in}S --- source\\
\hline
{\tt Q} & BJT\ \ \ Terminals\newline
\hspace*{1in}C --- collector\newline
\hspace*{1in}B --- base\newline
\hspace*{1in}E --- emitter\newline
\hspace*{1in}S --- source\\
\hline
\end{tabular}\\[0.1in]
