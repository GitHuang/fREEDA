\element{H}{Current-Controlled Voltage Source}

\begin{figure}[h]
\centering
\ \pfig{h_spice.ps}
\caption{H --- current-controlled voltage source element.}
\end{figure}

\form{
     {\tt H}name $N_{+}$ $N_{-}$ VoltageSourceName Transresistance\\
     {\tt H}name $N_{+}$ $N_{-}$ {\tt POLY(} D {\tt )}
     VoltageSourceName$_1$ ...  VoltageSourceName$_D$ PolynomialCoefficients
     }

\begin{widelist}
\item[$N_{+}$] is the positive voltage source node.
\item[$N_{-}$] is the negative voltage source node.
\item[{\it VoltageSourceName}] is the name of the voltage source the current
through which is the controlling current. The voltage source must be a {\tt V}
element.
\item[{\it Transresistance}] is the Transresistance of the element.
\item[{\tt POLY}] is the identifier for the polynomial form of the element
\item[{\it D}] is the degree of the poynomial. The number of pairs of
           controlling nodes must be equal to {\it Degree}.
\item[{\it VoltageSourceName$_i$}] is the name of the voltage source the current
through which is the $i$th controlling current.
The voltage source must be a {\tt V} element.
\item[{\it PolynomialCoefficients}] is the set of polynomial coefficients
which must be specified in the standard polynomial coefficient format
discussed on page \pageref{section:poly}.
\end{widelist}

\example{E1 2 3 14 1 2.0}

\noindent\underline{Linear Transresistance Instance}
\\[0.1in]\hspace*{\fill}\offsetparbox{\it
     {\tt H}name $N_{+}$ $N_{-}$ $N_{C+}$ $N_{C-}$ Transresistance }\\[0.1in]
The value of the voltage generator is linearly proportional to the controlling
current:
\begin{equation}
v_o = Transresistance\,v_c
\end{equation}
\\[0.2in]\noindent\underline{{\tt POLY}nomial Instance}
\\[0.1in]\hspace*{\fill}\offsetparbox{\it
     {\tt H}name $N_{+}$ $N_{-}$ {\tt POLY(} D {\tt )}
     {\tt (}$N_{C1+}$ $N_{C1-}{\tt)}$
     ... {\tt (}$N_{CD+}$ $N_{CD-}{\tt )}$ PolynomialCoefficients }\\[0.1in]
The value of the voltage generator is a polynomial function of the controlling
voltages:
\begin{equation}
v_o = f(i_{c1}, ...,  i_{ci}, ...  i_{cD}
\end{equation}
where the number of controlling currents is $D$ --- the degree of the polynomial
specified on the element line.
$i_{ci}$ is the $i$th controlling current and is the current flowing from the
$+$ terminal to the $-$ terminal in the
$i$th voltage source of name {\it VoltageSourceName}.
