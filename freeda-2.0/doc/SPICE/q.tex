\element{Q}{Bipolar Junction Transistor}
\begin{figure}[h]
\centering
\ \pfig{q_spice.ps}
\caption[Q --- bipolar junction transistor element]{Q --- bipolar junction
transistor element: (a) NPN transistor; (b) PNP transistor.}
\end{figure}

\form{ {\tt Q}name  NCollector NBase NEmitter  \B NSubstrate\E  ModelName
 \B Area\E  \B {\tt OFF}\E\\{\tt +} \B {\tt IC=}Vbe,Vce\E }

\begin{widelist}
\item[{\it NCollector}] is the collector node.
\item[{\it NBase}] is the base node.
\item[{\it NEmitter}] is the emitter node.
\item[{\it NSubstrate}] is the optional substrate node.
If not specified, ground is used as the substrate node.
If {\it NSubstrate} is a name {as allowed in \pspice )} it must
be enclosed in square brackets, e.g. {\tt [{\it NSubstrate}]}, to
distinguish it from {\it ModelName}.
\item[{\it ModelName}] is  the  model  name.
\item[{\it Area}]  is  the  area  factor\\
If the area  factor  is  omitted,  a  value of 1.0 is assumed.
(Units: none; Optional; Default: 1; Symbol: $Area$)
\item[{\tt OFF}] indicates an (optional)
initial condition on the device for the \dc\ analysis.
If specified the \dc\ operating point is calculated with the terminal voltages
set to zero.  Once convergence is obtained, the
program continues to iterate to obtain the exact  value of
the  terminal  voltages.  The OFF option is used to enforce the solution
to  correspond  to  a  desired  state if the circuit has more than one stable
state.
\item[{\tt IC}] is the
optional initial condition specification using  {\tt IC=}$V_{BE},V_{CE}$
is  intended  for use with the {\tt UIC} option on the {\tt .TRAN} line,
when a transient analysis is  desired  starting  from  other
than  the  quiescent  operating  point.   See  the  {\it .IC} line
description for a better way to set transient initial conditions.
\end{widelist}

\example{ Q20 10 50 0 QFAST IC=0.65,15.0 \\
          Q5PUSH 10 29 14 200 MODEL1 }

%
% NPN
%
\model{NPN}{NPN Si Bipolar Transistor Model} \model{PNP}{PNP Si
Bipolar Transistor Model} \model{LPNP}{{\pspice\
Only\hfill}Lateral PNP Si Bipolar Transistor Model}
\begin{figure}[h]
\centering
\ \pfig{bjt.ps}
\caption[Schematic of bipolar junction transistor model]{Schematic of the NPN
bipolar junction transistor model.
In the NPN and PNP models node $N_X$ is connected to node C
In the LPNP model node $N_X$ is connected to node B.
\label{bjt:model}}
\end{figure}

The NPN and PNP BJT models are identical but with the positive sense of
currents and voltages opposite so that
the model parameters are always positive.
The LPNP model is used for a lateral PNP IC transistor structure.
In the NPN and PNP models the node $N_X$ in figure \ref{bjt:model}
is connected to node C --- the internal collector node.
In the LPNP model node $N_X$
is connected to node B --- the internal base node.
Only the model type designated on the element line distinguishes which
schematic is used.

The bipolar junction transistor model in \spice\  is based on the  charge
control model of Gummel and Poon. Extensions in the \spice\ implementation
deal with effects at high bias levels.  The model reduces to the simpler
Ebers-Moll model  with the ommission of appropriate model parameters.

\begin{longtable}[h]{|p{0.6in}|p{3.5in}|p{0.6in}|p{0.6in}|p{0.6in}|}
\caption[BJT model parameters.]{BJT model parameters.}\\

\hline
\multicolumn{1}{|c}{\bf Name} &
\multicolumn{1}{|c}{\parbox{2.77in}{\bf Description}}  &
\multicolumn{1}{|c}{\bf Units} &
\multicolumn{1}{|c|}{\bf Default}&
\multicolumn{1}{|c|}{\bf Are}\\ \hline
\endhead

\hline \multicolumn{5}{|r|}{{Continued on next page}} \\ \hline
\endfoot

\hline \hline
\endlastfoot

{\tt AF}      &flicker noise exponent       \sym{\AF}& -     &   1&      \X
{\tt BF} & ideal maximum forward beta\sym{\beta_F}& - & 100 & \X
{\tt BR} & ideal maximum reverse beta\sym{\beta_R}& - & 1 & \X
{\tt C2} & alternative keyword for {\tt ISE}
           {\newline\pspice\ only.} &&&\X
{\tt C4} & alternative keyword for {\tt ISC}
           {\newline\pspice\ only.} &&&\X
{\tt CCS}& alternative keyword for {\tt CJS}
           {\newline\pspice\ only.} &&&\X
{\tt CJC} & base-collector zero-bias depletion\newline capacitance\sym{C_{JC}}
      & F & 0 & \STAR \X
{\tt CJE} & base-emitter zero-bias depletion\newline capacitance\sym{C_{JE}}
      & F & 0 & \STAR \X
{\tt CJS} & zero-bias collector-substrate\newline capacitance\sym{C_{JS}}
        & F & 0 & \STAR \X
{\tt EG} & energy gap voltage (barrier height)\sym{E_G}
         & eV & 1.11 & \X
{\tt FC} & coefficient for forward-bias depletion\newline capacitance formula
          \sym{F_C}&-&0.5 &\X
{\tt IK} & alternative keyword for {\tt IKF}
           {\newline\pspice\ only.} &&&\X
{\tt IKF} & corner of forward beta high current\newline roll-off\sym{I_{KF}}
         &A&$\infty$ &\STAR \X
{\tt IKR} & corner of reverse beta high current\newline roll-off\sym{I_KF}&
    A & $\infty$  & \STAR \X
{\tt IRB} & current where base resistance falls halfway to its minimum value
            \sym{I_{RB}} &-& $\infty$  & \STAR \X
{\tt IS} & transport saturation current\sym{I_S}& A & 1.0E-16 & \STAR \X
{\tt ISC} & base-collector leakage saturation\newline current\sym{I_{SC}}
            If {\tt ISC} is greater than 1 it is treated as a multiplier.
            In this case $I_{SC} = {\tt ISC}\,I_S$
           & A & 0 & \STAR \X
{\tt ISE} & base-emitter leakage saturation\newline current\sym{I_{SE}}
            If {\tt ISE} is greater than 1 it is treated as a multiplier.
            In this case $I_{SE} = {\tt ISE}\,I_S$
           & A & 0 & \STAR \X
{\tt ISS} & substrate p-n junction saturation current\sym{I_{SS}}
           {\newline\pspice\ only.} & A & 0 & \STAR \X
{\tt ITF} & high-current parameter for effect on TF\sym{I_{\tau F}} & A &
      0 & \STAR \X
{\tt KF} & flicker-noise coefficient\sym{K_F}& - & 0 & \X
{\tt MC} & alternative keyword for {\tt MJC}
           {\newline\pspice\ only.} &&&\X
{\tt ME} & alternative keyword for {\tt MJE}
           {\newline\pspice\ only.} &&&\X
{\tt MJC} & base-collector junction exponential\newline factor\sym{M_{JC}}&
           - & 0.33 & \X
{\tt MJE} & base-emitter junction exponential\newline factor\sym{M_{JE}}
& - &0.33 & \X
{\tt MJS} & substrate junction exponential\newline factor\sym{M_{JS}}
& - & 0 &\X
{\tt MS} & alternative keyword for {\tt MJS}
           {\newline\pspice\ only.} &&&\X
{\tt NC} & base-collector leakage emission\newline coefficient\sym{N_C}
& - & 2 & \X
{\tt NE} & base-emitter leakage emission coefficient\sym{N_E}& - & 1.5 & \X
{\tt NF} & forward current emission coefficient\sym{N_F}& - & 1.0 & \X
{\tt NR} & reverse current emission coefficient\sym{N_R}& - & 1 & \X
{\tt NS} & substrate p-n emission coefficient\sym{N_S}
           {\newline\pspice\ only.} & - & 1 & \STAR \X
{\tt PC} & alternative keyword for {\tt VJC}
           {\newline\pspice\ only.} &&&\X
{\tt PE} & alternative keyword for {\tt VJE}
           {\newline\pspice\ only.} &&&\X
{\tt PS} & alternative keyword for {\tt VJS}
           {\newline\pspice\ only.} &&&\X
{\tt PT} & alternative keyword for {\tt XTI}
           {\newline\pspice\ only.} &&&\X
{\tt PTF}& excess phase at frequency=1.0/({\tt TF}$\,2\pi$) Hz
           \sym{P_{\tau F}} & degree & 0 &\X
{\tt RB} & zero bias base resistance\sym{R_B}
         & $\Omega$ & 0 & \STAR \X
{\tt RBM}& minimum base resistance at high currents\sym{R_{BM}}
         & $\Omega$ & RB & \STAR \X
{\tt RC} & collector resistance\sym{R_C}
         & $\Omega$ & 0 &\STAR \X
{\tt RE} & emitter resistance\sym{R_E}
         & $\Omega$ & 0 & \STAR\X
{\tt TF} & ideal forward transit time\sym{\tau_F}& s & 0 & \X
{\tt TR} & ideal reverse trarsit time\sym{\tau_R}& s & 0 & \X
{\tt TRB1} & {\tt RB} linear temperature coefficient\sym{T_{RB1}}
           {\newline\pspice\ only.} & $^{\circ}$C$^{-1}$ & 1 & \STAR \X
{\tt TRB2} & {\tt RB} quadratic temperature coefficient\sym{T_{RB2}}
           {\newline\pspice\ only.} & $^{\circ}$C$^{-2}$ & 1 & \STAR \X
{\tt TRC1} & {\tt RC} linear temperature coefficient\sym{T_{RC1}}
           {\newline\pspice\ only.} & $^{\circ}$C$^{-1}$ & 1 & \STAR \X
{\tt TRC2} & {\tt RC} quadratic temperature coefficient\sym{T_{RC2}}
           {\newline\pspice\ only.} & $^{\circ}$C$^{-2}$ & 1 & \STAR \X
{\tt TRE1} & {\tt RE} linear temperature coefficient\sym{T_{RE1}}
           {\newline\pspice\ only.} & $^{\circ}$C$^{-1}$ & 1 & \STAR \X
{\tt TRE2} & {\tt RE} quadratic temperature coefficient\sym{T_{RE2}}
           {\newline\pspice\ only.} & $^{\circ}$C$^{-2}$ & 1 & \STAR \X
{\tt TRM1} & {\tt RBM} linear temperature coefficient\sym{T_{RM1}}
           {\newline\pspice\ only.} & $^{\circ}$C$^{-1}$ & 1 & \STAR \X
{\tt TRM2} & {\tt RBM} quadratic temperature coefficient\sym{T_{RM2}}
           {\newline\pspice\ only.} & $^{\circ}$C$^{-2}$ & 1 & \STAR \X
{\tt VA} & alternative keyword for {\tt VAF}
           {\newline\pspice\ only.} &&&\X
{\tt VB} & alternative keyword for {\tt VAR}
           {\newline\pspice\ only.} &&&\X
{\tt VAF} & forward Early voltage\sym{V_{AF}}& V & $\infty$  & \X
{\tt VAR} & reverse Early voltage\sym{V_{AR}}& V & $\infty$  & \X
{\tt VJC} & base-collector built-in potential\sym{V_{JC}}& V & 0.75 & \X
{\tt VJE} & base-emitter built-in potential\sym{V_{JE}}& V & 0.75 & \X
{\tt VJS} & substrate junction built-in potential\sym{V_{JS}}& V && \X
{\tt VTF} & voltage describing $V_{BC}$ dependence of TF\sym{V_{\tau F}}
      & V & $\infty$  & \X
{\tt XCJC} & fraction of B-C depletion capacitance connected to internal base
            node\sym{X_{CJC}} & - & 1 & \X
{\tt XTB} & forward and reverse beta temperature\newline exponent\sym{X_{TB}}
& - && \X
{\tt XTI} & temperature exponent for effect on\newline {\tt IS}\sym{X_{TI}}
& - & 3 & \X
{\tt XTF} & coefficient for bias dependence of\newline TF\sym{X_{\tau F}}
& - & & \X
\end{longtable}

\noindent\underline{\sl \large Standard Calculations}\\[0.1in]
The physical constants used in the model evaluation are
\begin{center}
\begin{tabular}{|l|l|l|}
\hline
$k$ & Boltzman's constant           &  $1.3806226\,10^{-23}$~J/K\\
$q$ & electronic charge             & $1.6021918\,10^{-19}$~C\\
\hline
\end{tabular}
\end{center}
Absolute temperatures (in kelvins, K) are used.
The thermal voltage
\begin{equation}
V_{\ms{TH}}(T_{\ms{NOM}}) = {{kT_{\ms{NOM}}} \over q} .
\end{equation}

\clearpage
\noindent\underline{\sl \large Temperature Dependence}
\index{BJT, Temperature Dependence}
\index{Temperature Dependence, see BJT}
\\[0.1in]
Temperature effects are incorporated as follows where $T$ and $T_{\ms{NOM}}$
are absolute temperatures in Kelvins (K).
\begin{eqnarray}
V_{\ms{TH}} & = & {{kT} \over q}\\
I_S (T) & = & I_S e^{\left( \textstyle E_g(T) {T \over {T_{\ms{NOM}}}}
      - E_G(T) \right) /V_{\ms{TH}}}+
      \left( {T \over {T_{\ms{NOM}}}} \right) ^{X_{TI}/N_F}\\
I_{SE}(T)&=&I_{SE} e^{\left( \textstyle E_g(T) {T \over {T_{\ms{NOM}}}}
      - E_G(T) \right) /V_{\ms{TH}}}+
      \left( {T \over {T_{\ms{NOM}}}} \right) ^{X_{TI}/N_E}\\
I_{SC}(T)&=&I_{SC} e^{\left( \textstyle E_g(T) {T \over {T_{\ms{NOM}}}}
      - E_G(T) \right) /V_{\ms{TH}}}+
      \left( {T \over {T_{\ms{NOM}}}} \right) ^{X_{TI}/N_C}\\
I_{SS}(T)&=&I_{SS} e^{\left( \textstyle E_g(T) {T \over {T_{\ms{NOM}}}}
      - E_G(T) \right) /V_{\ms{TH}}}+
      \left( {T \over {T_{\ms{NOM}}}} \right) ^{X_{TI}/N_S}\\
V_{JE}(T) =&&\hspace{-0.35in}
V_{JE}(T_{\ms{NOM}})(T-T_{\ms{NOM}})
        -3V_{\ms{TH}}\mbox{ln}\left( {T \over {T_{\ms{NOM}}}} \right)
      E_G(T_{\ms{NOM}}){T \over {T_{\ms{NOM}}}} -E_G(T)
      \ \ \ \ \ \ \ \ \ \ \ \ \ \  \\
V_{JC}(T)=&&\hspace{-0.35in}V_{JC}(T_{\ms{NOM}})(T-T_{\ms{NOM}})
        -3V_{\ms{TH}}\mbox{ln}\left( {T \over {T_{\ms{NOM}}}} \right)
      E_G(T_{\ms{NOM}}){T \over {T_{\ms{NOM}}}} -E_G(T) \\
V_{JS}(T)=&&\hspace{-0.35in}V_{JS}(T_{\ms{NOM}})(T-T_{\ms{NOM}})
        -3V_{\ms{TH}}\mbox{ln}\left( {T \over {T_{\ms{NOM}}}} \right)
      E_G(T_{\ms{NOM}}){T \over {T_{\ms{NOM}}}} -E_G(T) \\
C_{JC} (T) & = & C_{JC} \{1 +
     M_{JC} [0.0004(T-T_{\ms{NOM}})+(1-V_{JC}(T)/V_{JC}(T_{\ms{NOM}}))]\}\\
C_{JE} (T) & = & C_{JE} \{1 +
     M_{JE} [0.0004(T-T_{\ms{NOM}})+(1-V_{JE}(T)/V_{JE}(T_{\ms{NOM}}))]\}\\
C_{JS} (T) & = & C_{JS} \{1 +
     M_{JS} [0.0004(T-T_{\ms{NOM}})+(1-V_{JS}(T)/V_{JS}(T_{\ms{NOM}}))]\}\\
\beta_F(T)&=&\beta_F(T_{\ms{NOM}})^{X_{TB}}\\
\beta_R(T)&=&\beta_R(T_{\ms{NOM}})^{X_{TB}}\\
E_G(T) & = & E_G(T_{\ms{NOM}}) - 0.000702{{T^2} \over {T+1108}}\\
R_B(T) & = & R_B(T_{\ms{NOM}})\left[ 1 + T_{RB1} (T -  T{\ms{NOM}})
         + T_{RB2} (T -  T{\ms{NOM}})^2\right]\\
R_{BM}(T) & = & R_{BM}(T_{\ms{NOM}})\left[ 1 + T_{RM1} (T -  T{\ms{NOM}})
         + T_{RM2} (T -  T{\ms{NOM}})^2\right]\\
R_C(T) & = & R_C(T_{\ms{NOM}})\left[ 1 + T_{RC1} (T -  T{\ms{NOM}})
         + T_{RC2} (T -  T{\ms{NOM}})^2\right]\\
R_E(T) & = & R_E(T_{\ms{NOM}})\left[ 1 + T_{RE1} (T -  T{\ms{NOM}})
         + T_{RE2} (T -  T{\ms{NOM}})^2\right]
\end{eqnarray}\\[0.1in]

\clearpage
\noindent\underline{\sl \large Capacitances}\\[0.1in]
\index{Depletion capacitance, see BJT}
\index{BJT, Depletion capacitance}
\noindent The base-emitter capacitance,
$C_{BE} = Area( C_{BE\tau} + C_{BEJ})$\inlineeq
where the base-emitter transit time or diffusion capacitance
\begin{equation}
C_{BE\tau} = \tau_{F,\ms{EFF}} {{\textstyle\partial I_{BF}} \over
       {\textstyle\partial V_{BE}}}
\end{equation}
the effective base transit time is empirically modified to account for base
puchout, space-charge limited current flow, quasi-saturation and lateral
spreading which tend to increase $\tau_F$
\begin{equation}
\tau_{F,\ms{EFF}} =\tau_F\left[ 1+X_{TF}(3x^2-2x^3)
     e^{\textstyle (V_{BC}/(1.44V_{TF})}\right]
\end{equation}
and $x = {I_{BF}}/(I_{BF} + Area I_{TF})$.
The base-emitter junction (depletion) capacitance
\begin{equation}
C_{BEJ} = \left\{ \!\! \begin{array}{ll}
   C_{JE} \left(1-{{\textstyle V_{BE}}\over
   {\textstyle V_{JE}}}\right)^{\textstyle -M_{JE}}
   & \! V_{BE} \le F_C V_{JE}\\
   C_{JE} \left(1-F_C\right)^{\textstyle -(1+M_{JE})}
          \left(1-F_C(1+M_{JE})+M_{JE}{{\textstyle V_{BE}}\over
          {\textstyle V_{JE}}} \right)
   & \! V_{BE} > F_C V_{JE}
   \end{array} \right. %}
\end{equation}
The base-collector capacitance,
$C_{BC} = Area(C_{BC\tau} +X_{CJC} C_{BCJ})$\inlineeq
where the base-collector transit time or diffusion capacitance
\begin{equation}
C_{BC\tau} = \tau_R {{\textstyle\partial I_{BR}} \over
       {\textstyle\partial V_{BC}}}
\end{equation}
The base-collector junction (depletion) capacitance
\begin{equation}
C_{BCJ} = \left\{ \! \! \begin{array}{ll}
   C_{JC} \left(1-{{\textstyle V_{BC}}\over
          {\textstyle V_{JC}}}\right)^{\textstyle -M_{JC}}
   & \! V_{BC} \le F_C V_{JC}\\
   C_{JC} \left(1-F_C\right)^{\textstyle -(1+M_{JC})}
          \left(1-F_C(1+M_{JC})+M_{JC}{{\textstyle V_{BC}}\over
          {\textstyle V_{JC}}} \right)
   & \! V_{BC} > F_C V_{JC}
   \end{array} \right. %}
\end{equation}
The capacitance between the extrinsic base and the intrinsic collector
\begin{equation}
C_{BX} = \left\{ \begin{array}{ll}
 Area(1-X_{CJC}) C_{JC} \left(1-{{\textstyle V_{BX}}\over
   {\textstyle V_{JC}}}\right)^{\textstyle -M_{JC}}
   & V_{BX} \le F_C V_{JC}\\ \\
 (1-X_{CJC})
   C_{JC} \left(1-F_C\right)^{\textstyle -(1+M_{JC})}
   & V_{BX} > F_C V_{JC}\\
   \;\;\;\;\;\times\; \left(1-F_C(1+M_{JC})+M_{JC}{{\textstyle V_{BX}}\over
   {\textstyle V_{JC}}} \right)
   &
   \end{array} \right. %}
\end{equation}
The substrate junction capacitance
\begin{equation}
C_{JS} = \left\{ \begin{array}{ll}
 Area C_{JS} \left(1-{{\textstyle V_{CJS}}\over
   {\textstyle V_{JS}}}\right)^{\textstyle -M_{JS}}
   & V_{CJS} \le 0\\
   Area C_{JS}
   \left(1+M_{JS}{{\textstyle V_{CJS}}\over
   {\textstyle V_{JS}}} \right)
   & V_{CJS} > 0
   \end{array} \right. %}
\end{equation}

%\begin{figure}[b]
%\parbox[t]{1.3in}{
%\begin{tabular}[t]{|p{1in}|}
%\hline
%\multicolumn{1}{|c|}{PROCESS} \\
%\multicolumn{1}{|c|}{PARAMETERS} \\
%\hline
%\hline
%{\tt CJ} \hfill $\CJ$\\
%\hline
%\end{tabular}
%}
%\hfill
%\parbox{0.1in}{\ \vspace*{0.2in}\newline +}
%\hfill
%\begin{tabular}[t]{|p{1in}|}
%\hline
%\multicolumn{1}{|c|}{GEOMETRY} \\
%\multicolumn{1}{|c|}{PARAMETERS} \\
%\hline
%{\tt PS} \hfill $A_S$\\
%\hline
%\end{tabular}
%\hfill
%\parbox{0.1in}{\ \vspace*{0.2in}\newline $\rightarrow$}
%\hfill
%\begin{tabular}[t]{|p{1.8in}|}
%\hline
%\multicolumn{1}{|c|}{DEVICE} \\
%\multicolumn{1}{|c|}{PARAMETERS} \\
%\hline
%{\tt CBD} \hfill $\CBD = f(\CJ, A_D)$\\
%{\tt CBS} \hfill $\CBS = f(\CJ, A_S)$\\
%\{$C_{BS} = f( \PS, \CJ, \CJSW,$\hspace*{\fill}\\ \hspace*{\fill}$ \MJ, \MJSW,
%\PB, \PBSW, \FC)$\}\\
%\{$C_{BD} = f( \PD, \CJ, \CJSW,$\hspace*{\fill}\\ \hspace*{\fill}$ \MJ, \MJSW,
%\PB, \PBSW, \FC)$\}\\
%\hline
%\end{tabular}
%\caption{BJT {\tt LEVEL} 1, 2 and 3 junction depletion capacitance parameter
%relationships.  \label{level123depletionc} }
%\end{figure}
%{The parameter dependencies of the parameters describing the
%leakage current are summarized in
%figure \ref{mlevel123leakage}.\\[0.2in]
%\begin{figure}[b]
%\begin{tabular}[t]{|p{1in}|}
%\hline
%\multicolumn{1}{|c|}{PROCESS} \\
%\multicolumn{1}{|c|}{PARAMETERS} \\
%\hline
%\hline
%{\tt JS} \hfill $\JS$\\
%\hline
%\end{tabular}
%\hfill
%\parbox{0.1in}{\ \vspace*{0.2in}\newline +}
%\hfill
%\begin{tabular}[t]{|p{1in}|}
%\hline
%\multicolumn{1}{|c|}{GEOMETRY} \\
%\multicolumn{1}{|c|}{PARAMETERS} \\
%\hline
%{\tt AD} \hfill $A_D$\\
%{\tt AS} \hfill $A_S$\\
%\hline
%\end{tabular}
%\hfill
%\parbox{0.1in}{\ \vspace*{0.2in}\newline $\rightarrow$}
%\hfill
%\begin{tabular}[t]{|p{1.8in}|}
%\hline
%\multicolumn{1}{|c|}{DEVICE} \\
%\multicolumn{1}{|c|}{PARAMETERS} \\
%\hline
%{\tt IS} \hfill $\IS = f(\JS, A_D, A_S)$\\
%\hline
%\end{tabular}
%\caption{BJT leakage current parameter dependecies. \label{qleakage}}
%\end{figure}
%}

\clearpage
\noindent\underline{\sl I/V Characteristics}\\[0.1in]
\index{I/V Characteristics, see BJT}
\index{BJT, I/V Characteristics}
\index{NPN, I/V Characteristics}
\index{PNP, I/V Characteristics}
\index{I-V characteristics, see BJT}

\noindent{The base-emitter current, }
$I_{BE} = {{\textstyle I_{BF} } \over {\textstyle \beta_F }} + I_{LE}$\inlineeq
\noindent{the base-collector current, }
$I_{BC} = {{\textstyle I_{BR} } \over {\textstyle \beta_R }} + I_{LC}$\inlineeq
and the collector-emitter current,
$I_{CE} = {{\textstyle I_{BF} - I_{BR} } \over {\textstyle K_{QB} }}$\inlineeq
where the forward diffusion current,
$I_{BF} = I_S\left(e^{\textstyle V_{BE}/(N_F V_{\ms{TH}}) - 1} \right)$\inlineeq
the nonideal base-emitter current,
$I_{LE}=I_{SE}\left(e^{\textstyle V_{BE}/(N_E V_{\ms{TH}}) - 1} \right)$\inlineeq
the reverse diffusion current,
$I_{BR} = I_S\left(e^{\textstyle V_{BC}/(N_R V_{\ms{TH}}) - 1} \right)$\inlineeq
the nonideal base-collector current,
$I_{LC}=I_{SC}\left(e^{\textstyle V_{BC}/(N_C V_{\ms{TH}}) - 1} \right)$\inlineeq
and the base charge factor,
$K_{QB} = {{\textstyle 1}\over{2}} \left[1 -
    {{\textstyle V_{BC}}\over{\textstyle V_{AF}}}-
         {{\textstyle V_{BE}}\over{\textstyle V_{AB}}}
    \right]^{-1} \left(1 + \sqrt{1 + 4\left(
        {{\textstyle I_{BF}}\over{\textstyle I_{KF}}}+
         {{\textstyle I_{BR}}\over{\textstyle I_{KR}}}
        \right)}\right)$\\\inlineeq
Thus the conductive current flowing into the base,
$I_B = I_{BE}+I_{BC}$\inlineeq
the conductive current flowing into the collector,
$I_C = I_{CE}-I_{BC}$\inlineeq
and the conductive current flowing into the emitter,
$I_C = I_{BE}+I_{CE}$\inlineeq
\vspace{0.1in}
\noindent\underline{\sl \large Parasitic Resistances}\\[0.1in]
\index{Parasitic Resistances, see BJT}
\index{BJT, Parasitic Resistance}
\index{BJT, $R_B$}
\index{BJT, $R_E$}
\index{BJT, $R_C$}
The resistive parasitics $R_B$, $R_E$, are $R_C$ are scaled by the area
factor, $Area$, specified on the element line.  This enables the model
parameters {\tt RB}, {\tt RE} and {\tt RC} to be absolute quantities
if $Area$ is omitted as it defaults to 1, or as sheet resistivities.
\begin{eqnarray}
R'_B & = & R_B/Area\\
R'_C & = & R_C/Area\\
R'_E & = & R_E/Area
\end{eqnarray}
%{The parasitic resistance parameter dependencies are summarized in
%figure \ref{qpara}.
%\begin{figure}[b]
%\parbox[t]{1.3in}{
%\begin{tabular}[t]{|p{1in}|}
%\hline
%\multicolumn{1}{|c|}{PROCESS} \\
%\multicolumn{1}{|c|}{PARAMETERS} \\
%\hline
%\hline
%{\tt RB} \hfill $R_B$\\
%{\tt RC} \hfill $R_C$\\
%{\tt RE} \hfill $R_E$\\
%\hline
%\end{tabular}
%}
%\hfill
%\parbox{0.1in}{\ \vspace*{0.2in}\newline +}
%\hfill
%\begin{tabular}[t]{|p{1in}|}
%\hline
%\multicolumn{1}{|c|}{GEOMETRY} \\
%\multicolumn{1}{|c|}{PARAMETERS} \\
%\hline
%\hspace*{\fill} $Area$\\
%\hline
%\end{tabular}
%\hfill
%\parbox{0.1in}{\ \vspace*{0.2in}\newline $\rightarrow$}
%\hfill
%\begin{tabular}[t]{|p{1.8in}|}
%\hline
%\multicolumn{1}{|c|}{DEVICE} \\
%\multicolumn{1}{|c|}{PARAMETERS} \\
%\hline
%{\tt RB} \hfill $\R'_B = f(Area, R_B)$ \\
%{\tt RC} \hfill $\R'_C = f(Area, R_C)$ \\
%{\tt RE} \hfill $\R'_E = f(Area, R_E)$ \\
%\hline
%\end{tabular}
%\caption{BJT parasitic resistance parameter relationships. \label{qpara}}
%\end{figure}}
\begin{equation}
R'_B = \left\{ \begin{array}{ll}
         R_{BM} + {{\textstyle R_B - R_{BM}} \over { \textstyle K_{QB}}}
         & I_{RB} \mbox{ omitted}\\
         R_{BM} + 3(R_B-R_{BM}) {{\textstyle \tan{x} - x} \over
         { \textstyle x\mbox{tan}^2(x)}}
         & I_{RB} \mbox{ defined}
     \end{array} \right. %}
\end{equation}
where
$x = {\left( {\textstyle \sqrt{1 + {{\textstyle 144 I_B}\over{\textstyle I_{RB}\pi^2}}}
    -1 } \right)
    \left( {\textstyle {{\textstyle 24}\over{\textstyle\pi^2}}
    \sqrt{{{\textstyle I_B}\over{\textstyle I_{RB}}}}} \right)^{-1} }$\inlineeq

%{The relationships of the parameters describing the I/V
%characteristics for the {\tt LEVEL} 1 model are summarized in figure
%\ref{mlevel1i/v}.\\[0.1in]
%\begin{figure}[b]
%\begin{tabular}[t]{|p{1in}|}
%\hline
%\multicolumn{1}{|c|}{PROCESS} \\
%\multicolumn{1}{|c|}{PARAMETERS} \\
%\hline
%\hline
%{\tt NSUB} \hfill $\NSUB$\\
%{\tt TOX} \hfill $\TOX$\\
%{\tt NSS} \hfill $\NSS$\\
%{\tt UO} \hfill $\UO$\\
%\hline
%\end{tabular}
%\hfill
%\parbox{0.1in}{\ \vspace*{0.2in}\newline +}
%\hfill
%\begin{tabular}[t]{|p{1in}|}
%\hline
%\multicolumn{1}{|c|}{GEOMETRY} \\
%\multicolumn{1}{|c|}{PARAMETERS} \\
%\hline
%\hspace*{\fill} -- \hspace*{\fill} \\
%\hline
%\hline
%\multicolumn{1}{|c|}{Required} \\
%\hline
%{\tt L} \hfill $L$\\
%{\tt W} \hfill $W$ \\
%\hline
%\hline
%\multicolumn{1}{|c|}{Optional} \\
%\hline
%{\tt LD} \hfill $\LD$\\
%{\tt WD} \hfill $\WD$ \\
%\hline
%\end{tabular}
%\hfill
%\parbox{0.1in}{\ \vspace*{0.2in}\newline $\rightarrow$}
%\hfill
%\begin{tabular}[t]{|p{1.8in}|}
%\hline
%\multicolumn{1}{|c|}{DEVICE} \\
%\multicolumn{1}{|c|}{PARAMETERS} \\
%\hline
%{\tt VTO} \newline \hspace*{\fill} $\VTZERO = f(\PHI, \NSS, \TOX, \GAMMA)$\\
%\{$I_{D} = f(\W, \Length, \WD, \LD$\hspace*{\fill}\newline\hspace*{\fill}
%$\VTO, \KP, \LAMBDA, \PHI, \GAMMA)$\}\\
%\hline
%\end{tabular}
%\caption{BJT I/V dependencies. \label{qi/v}}
%\end{figure}
%}

\noindent\underline{\sl \large AC Analysis}\\[0.1in]
\index{MOSFET, AC Analysis}
The AC analysis uses the model of figure  \ref{m.ps} with the capacitor values
evaluated at the \dc\ operating point with
\begin{equation}
g_m = {{\textstyle\partial I_{CE}} \over {\textstyle\partial V_{BE}}}
\end{equation}
and
\begin{equation}
R_{O} = {{\textstyle\partial I_{CE}} \over {\textstyle\partial V_{CE}}}
\end{equation}

\clearpage
\noindent\underline{\sl \large Noise Analysis}\\[0.1in]
\index{MOSFET, Noise Model}
\index{MOSFET, Noise Analysis}
The BJT noise model accounts for thermal noise generated in the
parasitic resistances and shot and flicker noise generated in the
base-emitter and base-collector junction regions.
The rms (root-mean-square) values of
thermal noise current generators shunting the three parasitic resistance
$R_B$, $R_C$, and $R_E$ are
\begin{eqnarray}
I_{n,B} &=& \sqrt{4kT/R_B}~\mbox{A/}\sqrt{\mbox{Hz}}\\
I_{n,C} &=& \sqrt{4kT/R_C}~\mbox{A/}\sqrt{\mbox{Hz}}\\
I_{n,E} &=& \sqrt{4kT/R_E}~\mbox{A/}\sqrt{\mbox{Hz}}
\end{eqnarray}
The rms value of the base noise current generator is
\begin{equation}
I_{n,B} = \left( I_{\ms{SHOT},B}^2 + I_{\ms{FLICKER},B}^2
                \right)^{1/2}
\end{equation}
where
\begin{eqnarray}
I_{\ms{SHOT},B} &=& \sqrt{2qI_B} ~~~~\mbox{A/}\sqrt{\mbox{Hz}}\\
I_{\ms{FLICKER},B} &=& \sqrt{K_F I_{B}^{\AF} /f } ~~~~\mbox{A/}\sqrt{\mbox{Hz}}
\end{eqnarray}
and $f$ is frequency.
The rms value of the collector noise current generator is
\begin{equation}
I_{n,C} = \left( I_{\ms{SHOT},C}^2 + I_{\ms{FLICKER},C}^2
                \right)^{1/2}
\end{equation}
where
\begin{eqnarray}
I_{\ms{SHOT},C} &=& \sqrt{2qI_C} ~~~~\mbox{A/}\sqrt{\mbox{Hz}}\\
I_{\ms{FLICKER},C} &=& \sqrt{K_F I_C^{\AF} /f } ~~~~\mbox{A/}\sqrt{\mbox{Hz}}
\end{eqnarray}
