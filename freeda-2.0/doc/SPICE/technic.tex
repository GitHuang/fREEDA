\chapter{How Spice Works}

\section{Introduction}

\justspice\ has three basic analyses:
\dc\ analysis --- initiated by the {\tt .DC} statement;
\ac\ analysis --- initiated by the {\tt .AC} statement; and
transient analysis --- initiated by the {\tt .TRAN} statement.
\ac\ analysis involves incorporating the relations relating
terminal voltages and device currents into a matrix equation
This matrix equation is called the network equation and embodies the topology of
the network and the
constitutive relations \index{constitutive relations}
(that is, device equation) describing individual devices.
It derives from a mathematical statement of Kirchoff's current law with device
currents replaced by node voltages with the substitution achieved by using the
constitutive relations of individual devices.
Extensions to this basic form of the network equation (called the modal
formulation of the network equation) is required for elements that can not be
mathematicly modeled as a current or currents in terms of voltages.
\ac\ analysis is the appropriate place to begin a technical exposition of the
analysis algorithms in \spice\ as it illustrates the method of development of
network equation is used by all other analyses.

Transient analysis requires, as well, that a time-stepping numerical integration
algorithm be incorporated into the network equations.
A straight forward approach to transient analysis, which was used prior to the
development of the ``\spice '' approach, is to derive the
state equations \index{state equations}
in state space (that is in differential form using the {\it s} operator)
and apply the time discretization of a numerical integration method to the
complete network equation.
However, in \spice\ the time discretization is incorporated in the constitutive
relations before the network equation is developed.
The time discretized network equation must be solved iteratively and the Newton
iteration procedure is applied to the device equations along
with the time-discretization step.
This has proved to be a particularly robust approach and is the main reason
\spice\ is so widely accepted.

\dc\ analysis is a special case of transient analysis without the time
discretization step and so is discussed after transient analysis.

\section{AC Small Signal Analysis \label{.ACsection}}
\index{ac Analysis}\index{small signal analysis}\index{analysis, AC}%
\index{analysis, small signal}\marginid{{\tt .AC}}

\ac\ small signal analysis is inititated by the {\tt .AC} statement (see page
\pageref{.ACstatement}).  The aim in \ac\ analysis is to determine the \ac\
voltage at every node in the circuit which is now linear because of the
small-signal approximation.
First of all a matrix equation is
developed that relates node voltages to external current sources.
Through Thevenin's theorem external voltage sources are converted to
external current sources and node voltages are easily related to
voltages across elements.
This formulation  of the network equation is called the nodal admittance
formulation.
Unfortunately there are several elements, such as the current controlled
voltage source element, which do not have an admittance description
as required in this approach.
A method of circumventing this problem is discussed in section \ref{section:mna}
and is a direct extension of the method discussed below.

Consider the general network \Net\ in Figure \ref{fig:gen:net}(a) with $N$
internal 
\begin{figure}
\hspace*{\fill}
\ \epsfxsize=2.5in\pfig{gnet.ps}
\hspace*{\fill}
\hspace*{\fill}
\ \epsfxsize=2.5in\pfig{np.ps}
\hspace*{\fill}\\
\hspace*{\fill}(a)\hspace*{\fill}\hspace*{\fill}(b)\hspace*{\fill}\\
\caption{\label{fig:gen:net}Definition of networks: (a) \Net; and (b) \Nprime.}
\end{figure}
nodes in addition to the reference node.
In general the reference node is not part of the circuit and
the development of the network equations is simplified by treating all of the
nodes (except the external reference node) in the same way.

All of the nodes of the network have external current
sources $J$ with $J_{n}$
being the external current source between the $n$~th node and the reference node.
In practice most of the external current sources, the $J$'s, are zero and those
that are not are due to independent sources.
The network equations used in \spice\ relate the node voltages, the $v$'s, and
the $J$'s. In matrix form
\begin{equation}
\m{Y} \m{v}_{n} = \m{J}_{n}
            \label{nodal:yvj}
\end{equation}
where $\m{v}_{n}$ is the vector of node voltages, and $\m{J}_n$ is the vector of
external currents.
Solution of equation (\ref{nodal:vyj}) enablEs 
the node voltages to be evaluated given the external current sources.

In \spice\ the grounded reference node, node ``{\tt 0}'', is part of the network
and so there is a linear dependence of the rows of $\m{Y}$ ($| \m{Y} | = 0$).
Hence $\m{Y}$ is called the indefinite nodal admittance matrix.\index{indefinite
nodal admittance matrix} One row and one column can be deleted from
$\m{Y}$ without the lossing any
information to yield what is called the definite nodal admittance matrix.
For now consider that the reference node is an arbitrary node.
The correction required because the reference node is
actually part of the circuit is discussed in section \ref{section:soln:ne}.


\section{DC Analysis \label{.DCsection}}
\marginid{{\tt .DC}}

The analysis of nonlinear resistive circuits, or equivalently the analysis of
circuits at \dc \  is an important first step in \ac
and transient analysis.
In both cases nonlinear resistive analysis determines the initial starting
point for further analysis incorporating energy storage elements such as
capacitors and inductors.

\dc\ analysis is \justspice\ is identical to transient analysis discussed in
the previous section except that the contributions of capacitors and
inductors are ignored.
\dc\ analysis has better convergence properties ithan
transient analysis since energy
storage elements, and thus resonant responses, are eliminated.
As well the analysis is numericly efficient since only a steady-state response
is required and calculated.

\section{Discussion}

\spice\ supports several analyses other than those discussed above.
Essentially these are extensions of the \ac, \dc\ and transient analyses.

The transfer function analysis, initiated by the {\tt .TF} statement, 
calculates a transfer function as the ratio of the \dc\ value of an output
quantity to the \dc\ value of an input quantity over a range of values of the
input quantity.
A \dc\ analysis at each value of the input quantity is performed.

With the {\tt .DISTO} statement a distortion  analysis is performed by
dteremining the steady-state harmonic and intermodulation
products for small input signal. 
Evaluation of the small-signal  distortion
is based on a third-order multi-dimensional Volterra series expansion of
nonlinearities around their operating  point.
The method for calculating of the distortion products parallels \ac\ analysis
algorithm.

The sensitivity analysis calculates the \dc\ small-signal sensitivities of each
output quantity with respect to every circuit parameter. It is initiated by the
{\tt .SENS} statement.
The transfer function computed is the sensitivity (or partial derivative) of the
\dc\ value of the output quantity
with respect to the each and every circuit parameter.

If the {\tt .NOISE} statement is included in the input file
the noise generated by active devices and resistors is evaluated.
All active devices and some passive devices have noise models consisting of
uncorrelated noise current sources.  These noise current sources are used as
the external current sources in \ac analysis.  One noise source at a time is
considered and the response at the output terminals and sources specified are
calculated.  The contributions from each source are added in a root-mean-squared
sense as they are uncorrelated.  The noise analysis utilizes the network
equation formulated and solved in \ac\ analysis.

A Monte Carlo analysis is performed when the {\tt .MC} statement is
specified.  In the Monte Carlo analysis either a \dc, an \ac, or a transient
analysis is performed multiple times.

An operating point analysis initiated by the {\tt .OP} statement is just a
single \dc analysis.

The Fourier analysis performed when the {\tt .FOUR} statement is used
is not really a separate analysis at all.  Really it is a way of examiing the
results of a transient analysis by taking the Fourier transform of a
voltage or current response.

\section{To Explore Further}

The essential aspects of \spice\ are the models of devices and the algorithms
for formulating and solving the network equations.
The derivation of the \ac\ model of devices is straightforward
requiring the $y$ parameters of the device.  These are obtained 
from the analytic derivatives of the device equations calculated at the
operating point of the circuit as determined from a \dc\ analysis.
For transient and \dc\ analysis the associated discrete circuit model
must be calculated from the device equations.  The device equations of
semiconductor devices are calculated from knowledge of the device physics.
Derivation of the models used for semiconductor devices in \spicetwo\ and
\spicethree\ are described in \cite{antognetti:massobrio:88}
({\it Semiconductor Device Modeling with SPICE}
edited by P. Antognetti and G. Massobrio).
Without fail the models in \spicetwo\ are available in all 
commercial versions of \spice.  \spicethree\ and most commercial versions
of \spice\ provide additional or enhanced models.
Derivations of more advanced models are described in
\cite{lee:shur:93} ({\it Semiconductor Device Modeling for VLSI}
by K. Lee, M. Shur, T. Fjeldly and T. Ytterdal).

In \justspice\ the network equations are stored in sparse matrices
to conserve memory usage.  This also results in much faster
solution of the network equations than if regular matrices were used.
Greater detail than that provided in this chapter of the numerical algorithms
used in \spice\ can be found in
\cite{mccalla}
({\it Fundamentals of Computer-Aided Circuit Simulation,''}
by W.~J. McCalla).
