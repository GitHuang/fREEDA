\element{N}{Digital Input Interface}

\begin{figure}[h]
\centering
\ \pfig{n_spice.ps}
\caption{N --- Digital input interface element. Converts from a digital (state)
signal to an analog signal.}
\end{figure}

\form{{\tt N}name InterfaceNode LowLevelNode HighLevelNode ModelName
  \B{\tt SIGNAME = } DigitalSignalName \E
  \B{\tt IS = } InitialState \E}

\begin{widelist}
\item[{\it InterfaceNode}]
     Identifier of node interfacing between digital signal and
     continuous time circuit.
\item[{\it LowLevelNode}]
     Identifier of low level reference node. Normally this is the logic ``zero''
     voltage.
\item[{\it HighLevelNode}]
     Identifier of high level reference node. Normally this is the logic ``one''
     voltage.
\item[{\it ModelName}]
     Name of the model specifying transitions times and resistances and
     capacitances of each logic state
\item[{\tt SIGNAME}]
     Keyword for digital signal name. (optional)
\item[{\it DigitalSignalName}]
     Digital signal name.
     {\it DigitalSignalName} is the name of the signal specified in the
     input file specified in the element model.  If it is omitted then
     {\it DigitalSignalName} defaults element name {\it {\tt N}name}
     stripped of the prefix {\tt N} (i.e. {\it name}).
\item[{\tt IS }]
     Keyword for initial state. (optional)
\item[{\it InitialState}]
     Integer speciftying the initial state. If specified, it must be
     0, 1, ..., or 19. This over rides the state specified at {\tt TIME}=0
     in the digital input file (see the model specification).
     The state of the digital interface input ({\tt N}) element
     remains as the {\it InitialState} state until a state (other than the
     state at {\tt TIME}=0 is input from the specified file.
\end{widelist}

\example{
    N100 1 0 2 INTERFACE\_FROM\_REGISTER SIGNAME=REG1 IS=0 \\
    NCONTROL 1 0 2 CONTROL
    }

\modeltype{DINPUT}

The digital input interface is modeled by time variable resistances between
the {\it Interface Node} and the {\it Low Level Node} and between the
the {\it Interface Node} and the {\it High Level Node}.
The variable resistances are shunted by fixed capacitances.
The resistance are controlled by parameters specified in the model
{\it ModelName}.
The resistance varies exponentially from the old state to the new state over
the time period indicated for the new state.
This approximates the output of a digital gate. \\[0.2in]

\model{DINPUT}{Digital Input Interface Model}

\begin{figure}[h]
\centering
\ \pfig{din.ps}
\caption{Digital input interface model.}
\end{figure}


The digital input interface is modeled by time variable resistances between
the {\it Interface Node} and the {\it Low Level Node} and between the
the {\it Interface Node} and the {\it High Level Node}.
The variable resistances are shunted by fixed capacitances.
The parameters are controlled by parameters specified in the model.
Upon a state transition the two resistances vary exponentially from the old
state to the new state over the time period indicated for the new state.
This approximates the output of a digital gate.  The sequence of states and
the state change times are specified in the file specified by the {\tt FILE}
= ({\it InputFileName})
keyword in the model.  The initial state at {\tt TIME} 0 is taken from this
file unless the {\tt IS} keyword is specified on the
element line. In the $\IS$ (= {\it InitialState}) keyword is specified
then the state of the digital input interface is {\it InitialState}
until a state transition at {\tt TIME} $>$ 0 is specified in the file
{\it InputFileName}.

\keyword{
{\tt FILE}  & digital input filename.
          If more than one model refers to the same file then the
                  filenames specified must be identical and not logicly
                  equivalent.  This ensures that the file is opened
                  only once.
        & - & \reqd \X
{\tt FORMAT}    & digital input file format & - & 1 \X
{\tt TIMESTEP}& digital input file time step    & s & 1NS   \X
{\tt CLO}   & capacitance to low level node & F & 0 \X
{\tt CHI}   & capacitance to high level node& F & 0 \X
{\tt S{\it n}NAME}  & state ``n'' character abreviation\newline
      {\it n} = 0, 2, ..., or 19
    & - & \reqd \X
{\tt S{\it n}TSW}   & state ``n'' switching time\newline
      {\it n} = 0, 2, ..., or 19
    & s & \reqd \X
{\tt S{\it n}RLO}   & state ``n'' resistance to low level node\newline
      {\it n} = 0, 2, ..., or 19
    & $\Omega$& \reqd   \X
{\tt S{\it n}RHI}   & state ``n'' resistance to high level node\newline
      {\it n} = 0, 2, ..., or 19
    & $\Omega$& \reqd   \X
}
