\element{U}{Universal Element}

The {\tt U} element is a universal element for extending the number of
elements that \justspice\ can
handle. The element is determined by the model type. In some cases
model parameters indicate further refinement.

Since the \justspice\ {\tt U} element is used to specify many different
kinds of physical elements there is no common form for it.  Each physical
element is distinguished by the type of the model it refers to.  The format
and description of each element is given in the model description as
indicated in the following table.
The index is used in this manual to uniquely identify the various elements.


\modeltype{}
\noindent
\begin{tabular}{|p{0.6in}|p{0.6in}|p{4in}|r|}
\hline
Model & Index & Description                        & Page \\
\hline \hline STRUC&STRUC&Geometric coupled, lossy planar
transmission line
     & \pageref{STRUCmodel} \\
\hline
U&U311    &Geometric coupled planar transmission line (up to 5 lines)
      \newline Identifying Model Parameters:
      {\tt LEVEL=3} {\tt EVEL=1} {\tt PVEL=1}
     & \pageref{U3.1.1model} \\
\hline
U&U312    &Geometric coaxial cable
      \newline Identifying Model Parameters:
      {\tt LEVEL=3} {\tt EVEL=1} {\tt PVEL=2}
     & \pageref{U3.1.2model} \\
\hline
U&32    &General transmission line (up to 5 lines) defined by precomputed
      parameters.
      \newline Identifying Model Parameters:
      {\tt LEVEL=3} {\tt EVEL=2}
     & \pageref{U3.2model} \\
\hline
U&U34    &General transmission line (up to 5 lines) defined by measurements.
      \newline Identifying Model Parameters:
      {\tt LEVEL=3} {\tt EVEL=3}
     & \pageref{U3.3model} \\
\hline
U&U4    &Digital output element
      \newline Identifying Model Parameter: {\tt LEVEL=4}
     & \pageref{U4model} \\
\hline
U&U5    &Digital input element
      \newline Identifying Model Parameter: {\tt LEVEL=5}
     & \pageref{U5model} \\
\hline
\end{tabular}\\[0.1in]


\noindent
\begin{tabular}{|p{0.6in}|p{0.6in}|p{4in}|r|}
\hline
Model & INdex & Description                        & Page \\
\hline
\hline
UDLY&UDLY & Delay Line
     & \pageref{UDLYmodel} \\
\hline
UEFF&UEFF & Edge-Triggered Flip-Flop
     & \pageref{UEFFmodel} \\
\hline
UGATE&UGATE& Standard Gate Model Form
     & \pageref{UGATEmodel} \\
\hline
UGFF&UGFF & Gated Flip-Flop
     & \pageref{UGFFmodel} \\
\hline
UIO&UIO  & IO Model
     & \pageref{UIOmodel} \\
\hline
URC&URC  &Lossy RC transmission line
     & \pageref{URCmodel} \\
\hline
USUHD&USUHD& Setup and Hold Checker
     & \pageref{USUHDmodel} \\
\hline
UTGATE&UTGATE& Tri-State Gate
     & \pageref{UTGATEmodel} \\
\hline
UWDTH&UWDTH& Pulse-Width Checker
     & \pageref{UWDTHmodel} \\
\hline
\end{tabular}

\note{
\item
One of the problems with \justspice\ is that the first letter of an
element's name is used to determine the type of an element.  One consequence
of this is that there can only be 26 elements --- one for each letter of the
alphabet.  \spicetwo The original version of \justspice\  from which all
subsequent versions of \justspice\ have been developed had less than 26 elements
and so this was not a problem.
With the addition of new element types several of the originally unised
letters were used and a universal element, the {\tt U} element, introduced
to handle even more.  The {\tt U} element is used to represent many
different type of elements such as lossy transmission lines and digital
devices.  All of the {\tt U} elements refer to models and the model name,
and sometimes model parameters, used to indicate the actual element referred
to.}
