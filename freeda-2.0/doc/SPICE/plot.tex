%
% .PLOT
%
\statement{.PLOT}{Plot Specification}

The plot specification controls the information that is plotted as the result
of various analyses.

{
\form{{\tt .PLOT TRAN } OutputSpecification \B PlotLimits\E
     \\{\tt +} \B OutputSpecification \B PlotLimits\E $\ldots$ \E\\[0.1in]
      {\tt .PLOT AC   } OutputSpecification \B PlotLimits\E
	   \\{\tt +} \B OutputSpecification \B PlotLimits\E $\ldots$ \E\\[0.1in]
      {\tt .PLOT DC   } OutputSpecification  \B PlotLimits\E
	   \\{\tt +} \B OutputSpecification \B PlotLimits\E $\ldots$ \E\\[0.1in]
      {\tt .PLOT NOISE} NoiseOutputSpecification 
		    \B {\tt (}DistortionReportType{\tt )}\E \B PlotLimits\E
       \\{\tt +} \B NoiseOutputSpecification
	    \B {\tt (}DistortionReportType{\tt )}\E \B PlotLimits\E\\[0.1in]
      {\tt .PLOT DISTO} DistortionOutputSpecification
		   \B {\tt (}DistortionReportType{\tt )}\E \B PlotLimits\E
		   \\{\tt +} \B DistortionOutputSpecification
		   \B {\tt (}DistortionReportType{\tt )}\E \B PlotLimits\E
		      $\ldots$ \E
	    \B {\tt (}DistortionReportType{\tt )}\E \B PlotLimits\E\\[0.1in]
     }}

\pspiceform{{\tt .PLOT TRAN } OutputSpecification \B PlotLimits\E
     \\{\tt +} \B OutputSpecification \B PlotLimits\E $\ldots$ \E\\[0.1in]
      {\tt .PLOT AC   } OutputSpecification \B PlotLimits\E
	   \\{\tt +} \B OutputSpecification \B PlotLimits\E $\ldots$ \E\\[0.1in]
      {\tt .PLOT DC   } OutputSpecification  \B PlotLimits\E
	   \\{\tt +} \B OutputSpecification \B PlotLimits\E $\ldots$ \E\\[0.1in]
      {\tt .PLOT NOISE} NoiseOutputSpecification 
		    \B {\tt (}DistortionReportType{\tt )}\E \B PlotLimits\E
       \\{\tt +} \B NoiseOutputSpecification
	    \B {\tt (}DistortionReportType{\tt )}\E \B PlotLimits\E\\[0.1in]
     }


\begin{widelist}

\item[{\tt TRAN}] is the keyword specifying that this {\tt .PLOT} statement
     controls the reporting of results of a transient analysis initiated by the
     {\tt .TRAN} statement.

\item[{\tt AC}] is the keyword specifying that this {\tt .PLOT} statement
     controls the reporting of results of a small-signal \ac\ analysis initiated by
     the {\tt .AC} statement.

\item[{\tt DC}] is the keyword specifying that this {\tt .PLOT} statement
     controls the reporting of results of a \dc\ analysis initiated by the
     {\tt .DC} statement.

\item[{\tt NOISE}] is the keyword specifying that this {\tt .PLOT} statement
     controls the reporting of results of a noise analysis initiated by the
     {\tt .NOISE} statement.

{
\item[{\tt DISTO}] is the keyword specifying that this {\tt .PLOT} statement
     controls the reporting of results of a small-signal \ac\ distortion analysis
     initiated by the {\tt .DISTO} statement.
     }

\end{widelist}

\begin{widelist}
\item[{\it OutputSpecification}] specifies the voltage or current to be plotted
     against the sweep variable. The sweep variable is dependent on the type
     of analysis.

     Many forms of {\tt OutputSpecification} are supported by
     \pspice\ .
     {Below is a description of the basic forms that are
     supported both by \spicetwo\ and \pspice\ .
     }
     A comprehensive
     description of {\tt OutputSpecification} supported by
     \pspice\ is given in the section on output specification
     on page \pageref{.PRINToutputspecification}.
     %Up to 8 {\it OutputSpecification}s 
     \\[0.1in]

     {
     \underline{Voltages} may be specified as an absolute voltage at a node:
     {\tt V({\it NodeName})} or
     the voltage at one node with respect to that at another node,
     e.g. {\tt V(Node1Name,Node2Name)}.

     For the reporting of the results of an \ac\ analysis the following outputs
     can be specified by replacing the {\tt V} as follows:\\
      \offset\begin{tabular}{lcp{3.5in}}
      {\tt VR} &-&real part\\
      {\tt VI} &-&imaginary part\\
      {\tt VM} &-&magnitude\\
      {\tt VP} &-&phase\\
      {\tt VDB} &-&$10\,\mbox{log}(10\,magnitude)$
      \end{tabular}\\
      In \ac\  analysis the default is {\tt VM} for magnitude.\\[0.1in]

     \underline{Currents} are specified by referencing the name of the voltage
     source through which the current is measured, e.g.
     {\tt I(V{\it VoltageSourceName})}.

     For the reporting of the results of an \ac\ analysis the following outputs
     can be specified by replacing the {\tt I} as follows:\\
      \offset\begin{tabular}{lcp{3.5in}}
      {\tt IR} &-&real part\\
      {\tt II} &-&imaginary part\\
      {\tt IM} &-&magnitude\\
      {\tt IP} &-&phase\\
      {\tt IDB} &-&$10\,\mbox{log}(10\,magnitude)$
      \end{tabular}
      In \ac\  analysis the default is {\tt IM} for magnitude.
      }

\item[{\it PlotLimits}] are optional and can be placed after any output
     specification. {\it PlotLimits} has the form
     {\tt (}{\it LowerLimit}{\tt ,}{\it UpperLimit}{\tt )} .
     All quantities will be plotted using the same
     {\it PlotLimits}.  The default is to automatically scale the plot and
     perhaps use different scales for each of the quantities to be plotted.
      
\end{widelist}

\begin{widelist}
\item[{\it NoiseOutputSpecification}]
     specifies the noise measure to be reported. The two options are
     {\tt ONOISE} which reports the output noise and
     {\tt INOISE} which reports the equivalent input noise.
     See the {\tt .NOISE} statement on page \pageref{.NOISEstatement}
     for a detailed explanation.

     It  must be one of the following:\\
      \offset\begin{tabular}{lcp{3.5in}}
      {\tt ONOISE} &-&magnitude of the output noise\\
      {\tt DB(ONOISE)} &-&output noise in dB\\
      {\tt INOISE} &-&magnitude of the equivalent input noise\\
      {\tt DB(INOISE)} &-&equivalent input noise in dB\\
      {\tt GAIN} &-&voltage gain\\
      {\tt DB(GAIN)} &-&voltage gain in dB (= 20 log({\tt GAIN})\\
      {\tt GT} &-&transducer gain\\
      {\tt DB(GT)} &-&transducer gain in dB (= 10 log({\tt GT})\\
      {\tt NF} &-&spot noise factor\\
      {\tt DB(NF)} &-& spot noise figure (= 10 log({\tt NF})\\
      {\tt SNR} &-&output signal-to-noise ratio\\
      {\tt DB(SNR)} &-&output signal-to-noise ratio in dB (= 20 log({\tt SNR})\\
      {\tt TNOISE} &-&output noise temperature.
      \end{tabular}

\item[{\it SParameterOutputSpecification}]
      specifies the S-parameter output variables that are to be printed.
      Each variable must have one of the following forms:\\
      \offset\begin{tabular}{lcp{3.5in}}
      {\tt S(i,j)} &-&Magnitude of $S_{ij}$\\
      {\tt SR(i,j)} &-&Real part of $S_{ij}$\\
      {\tt SI(i,j)} &-&Imaginary part of $S_{ij}$\\
      {\tt SP(i,j)} &-&Phase of $S_{ij}$ in degrees\\
      {\tt SDB(i,j)} &-&Magnitude of $S_{ij}$ in dB
                         (= 20 log({\tt {\tt S(i,j)}}))\\ 
      {\tt SG(i,j)} &-&Group delay of $S_{ij}$
      \end{tabular}
      The port numbers are $i,j$  which are specified using the {\tt PNR} keywor
      when the port (`P') element is specified.


{
\item[{\it DistortionOutputSpecification}]
     specifies the distortion component to be reported
     in a tabular format of up to 8 columns plus an initial column with the sweep
     variable.  The {\it DistortionOutputSpecification} is one of the
     following:\\
      \offset\begin{tabular}{lcp{3.5in}}
      {\tt HD2} &-&the second harmonic distortion\\
      {\tt HD3} &-&the second harmonic distortion\\
      {\tt SIM2} &-&the sum frequency intermodulation component\\
      {\tt DIM2} &-&the difference frequency intermodulation component\\
      {\tt DIM3} &-&the third order intermodulation component
      \end{tabular}
      See the {\tt .DISTO} statement on page \pageref{.DISTOstatement} for
      a description of these distortion components.

\item[{\it DistortionReportType }]
     specifies the format for reporting the distortion components.
     It  must be one of the
     following:\\
      \offset\begin{tabular}{lcp{3.5in}}
      R &-&real part\\
      I &-&imaginary part\\
      M &-&magnitude\\
      P &-&phase\\
      DB &-&$10\,\mbox{log}(10\,magnitude)$
      \end{tabular}
      The default is {\tt M} for magnitude.
}
\end{widelist}

\example{
.PLOT TRAN V(10) V(5,3) I(VIN)\\
.PLOT AC VM(10) VR(5,3) IP(VLOAD)\\
.PLOT DC V(10) V(5,3) I(VIN)\\
.PLOT NOISE ONOISE INOISE DB(ONOISE) DB(INOISE)\\
.PLOT NOISE GAIN DB(GT) DB(NF) SNR TNOISE\\
.PLOT AS SDB(1,1) SP(1,1) SDB(1,2) SP(1,2)\\
{.PLOT DISTO HD2 HD3 SIM2(DB)}
}


\note{
\item There can be any number of {\tt .PLOT} statements.

\item All of the output quantities specified on a single {\tt .PLOT} statement
will be plotted on the same graph using ASCII characters.
An overlap will be indicated by the letter
{\tt X}.  The plot produced by the {\tt .PLOT} statement is a line printer plot.
While plotting is primitive it can be plotted on any printer and is
incorporated in the output log file.

\item The plot output of the results of an \ac\ analysis always have a
logarithmic vertical scale.

\item The current through any element can be found by inserting independent
voltage sources in series with the elements.
This is generally what is required in \spicetwo and \spicethree.
However \pspice\ supports direct specification of the voltage and currents
of most elements. See the section on page
\pageref{.PRINToutputspecification}.

\item
More elaborate plotting is available
{with \spicethree\ using the \nutmeg\ plotting program described on
the \nutmeg\ chapter beginning on page \pageref{ch:nutmeg}; and}
with \pspice\ using the {\tt .PROBE} statement described on page \pageref{.PROBEstatement}.

}
