\element{D}{Diode}
\begin{figure}[h]
\centering
\ \epsfxsize=1.5in\pfig{d_spice.ps}
\caption{D --- diode element.}
\end{figure}

\form{{\tt D}name $n_1$ $n_2$ ModelName \B Area\E  \B {\tt OFF}\E
\B {\tt IC}=$V_D$\E }

\pspiceform{{\tt D}name $n_1$ $n_2$ ModelName \B Area\E  \B {\tt OFF}\E }

\begin{widelist}
\item[{\it $n_1$}] is the positive (anode) diode node.
\item[{\it $n_1$}] is the negative (cathode) diode node.
\item[{\it ModelName}]  is  the  model name.
\item[{\it Area}] is an optional relative area factor.\\
               (Units: none; Optional; Default: 1, Symbol: $Area$)
\item[{\tt OFF}] indicates an (optional) starting  condition  on  the
device  for  \dc\ operating point analysis.
If specified the \dc\ operating point is calculated with the terminal voltages
set to zero.  Once convergence is obtained, the
program continues to iterate to obtain the exact  value of
the  terminal  voltages.  The OFF option is used to enforce the solution
to  correspond  to  a  desired  state if the circuit has more than one stable
state.
\item[{\tt IC}]  is the (optional)  initial  condition
specification.  Using  {\tt IC} = $V_D$ is intended for use with the {\tt UIC}
option on the other than the quiescent operating point.
Specification of the transient initial conditions using the {\tt .IC}
statement (see page \pageref{.ICstatement}) is preferred and is more
convenient.
\end{widelist}

\example{DBRIDGE 2 10 DIODE1 \\
         DCLMP 3 7 DMOD 3.0 IC=0.2}


\modeltype{DIODE}
\model{DIODE}{Diode Model}
\begin{figure}[h]
\centering
\ \epsfxsize=1in\pfig{diode.ps}
\caption{Schematic of diode element model.}
\end{figure}

\begin{table}[h]
\caption{DIODE model parameters. \label{diodetable}}
\keywordtwotable{Area}{
{\tt AF}  &  flicker noise exponent\sym{A_F}& -     & 1        & \X
{\tt BV}  &  magnitude of reverse breakdown voltage (positive)\sym{V_B}&
             V     & $\infty$ & \X
{\tt CJO} &  zero-bias {\it p-n} junction capacitance per unit area
             (CJ-oh) \sym{C_{J0}} & F  & 0      & \STAR \X
{\tt EG} & bandgap voltage (barrier height) at 0~K\sym{E_G(0)}
       \newline\hspace*{\fill} Schottky Barrier Diode: 0.69
       \newline\hspace*{\fill} Silicon: 1.16
       \newline\hspace*{\fill} Gallium Arsenide: 1.52
       \newline\hspace*{\fill} Germanium: 0.67
     & eV   & 1.11  &\X
{\tt FC} & forward-bias depletion capacitance\sym{F_C}& -   & 0.5  & \X
{\tt FC}  &  coefficient for forward-bias depletion capacitance formula
\sym{F_C}& -    &  & \X
{\tt IBV} &  magnitude of current at breakdown voltage per unit area
             (positive)\sym{I_{BV}}
          &    A     & $1\,10^{-10}$&\STAR\X
{\tt IS} & saturation current per unit area\sym{I_S}& A&$1\,10^{-14}$& \STAR \X
{\tt KF}  &  flicker noise coefficient\sym{K_F}& -     & 0        & \X
{\tt M}   &  {\it p-n} junction grading coefficient\sym{M_J}& -& 0.5 &   \X
{\tt N} &  emission coefficient\sym{n}& -     & 1        &   \X
{\tt RS}  &  ohmic resistance per unit area\sym{R_S}&$\Omega$& 0 & \STAR \X
{\tt TT}  &  transit time\sym{\tau_T}& s     & 0        &   \X
{\tt VJ}  &  junction potential\sym{\phi_J}& V     & 1        &   \X
{\tt XTI} &  saturation current ($I_S$) temperature exponent
                            \sym{X_{TI}}& -     & 3.0      &    \X
}
\end{table}
\note{
\item In some \spice\ implementations {\tt XTI} is called {\tt PT}
and {\tt VJ} is called {\tt PB}.
\item It is assumed that the model parameters were determined or
measured at the nominal temperature $T_{\ms{NOM}}$ (default
$27^{\circ}C$) specified in the most recent {\tt .OPTIONS} statement
preceeding the {\tt .MODEL} statement.
}
The physical constants used in the model evaluation are
\begin{center}
\begin{tabular}{|l|l|l|}
\hline
$k$ & Boltzman's constant           &  $1.3806226\,10^{-23}$~J/K\\
$q$ & electronic charge             & $1.6021918\,10^{-19}$~C\\
\hline
\end{tabular}
\end{center}
\noindent\underline{\sl \large Temperature Dependence}
\index{DIODE, Temperature Dependence}
\index{Temperature Dependence, see DIODE}
\\[0.1in]
Temperature effects are incorporated as follows where $T$ and $T_{\ms{NOM}}$
are absolute temperatures in Kelvins (K).
The thermal voltage
\begin{equation}
V_{\ms{TH}} = {{kT} \over q}\\
\end{equation}
and
\begin{eqnarray}
\IS (T) & = & \IS e^{\left( \textstyle E_g(T) {T \over {T_{\ms{NOM}}}}
     - E_G(T) \right) /(nV_{\ms{TH}})}
     \left({{\textstyle T} \over { \textstyle T_{\ms{NOM}}}} \right)\\
\phi_J (T) & = & \phi_J {T \over {T_{\ms{NOM}}}} - 3V_{\ms{TH}} \ln{
                {T \over {T_{\ms{NOM}}}} } + E_G (T_{\ms{NOM}})
                {T \over {T_{\ms{NOM}}}} -E_G(T)\\
C_{J0} (T) & = & C_{J0} \{1+M[0.0004(T-T_{\ms{NOM}})+(1-\phi_J(T)/\phi_J)]\} \\
%E_g(T) & = & 1.16 - 0.000702{{T^2} \over {T+1108}}\\
\end{eqnarray}\\[0.1in]
\noindent\underline{\sl \large Parasitic Resistance}\\[0.1in]
\index{DIODE, Parasitic Resistance}
\index{Parasitic Resistance, see DIODE}
\index{DIODE, $R_S$}
The parasitic diode series resistance $R_S$,
is calculated by scaling the sheet resistivity $R'_S$ (= {\tt RS})
by the $Area$ parameter on the element line
\begin{equation}
R_S = R'_S/Area
\end{equation}
\noindent\underline{\sl \large Current Characteristics}\\[0.1in]
\index{DIODE, Current Characteristic}
\begin{equation}
I_D = \left\{ \begin{array}{ll}
      Area I_S
      \left( e^{{{\textstyle V_D}\over{\textstyle nV_{\ms{TH}}}}}
         -1 \right)
      +V_D\GMIN
      &  V_D \ge -10nV_{\ms{TH}} \\
      -Area\,I_S
      +V_D\GMIN
      & -30V_{\ms{TH}}-V_B \le V_D < -10nV_{\ms{TH}} \\
      -Area
      \left(I_{BV} e^{\textstyle -{{\textstyle V_B+V_D}\over{\textstyle V_{\ms{TH}}}}}
         -I_S \right)
      +V_D\GMIN
      &  V_D < -30V_{\ms{TH}}-V_B \\
      \end{array} \right. %}
\end{equation}
where $\GMIN$ is {\tt GMIN}, the minimum conductance between
nodes.  {\tt GMIN} is set by a {\tt .OPTIONS} statement (see page
\pageref{.OPTIONSstatement:gmin}.)
\noindent\underline{\sl \large Capacitance}\\[0.1in]
\begin{equation}
C = Area(C_J+C_D)
\end{equation}
and the depletion capacitance at the junction per unit area is
\begin{equation}
C_J = \left\{ \begin{array}{ll}
      C_{J0} \left(1 -
      {{\textstyle V_D}\over{\textstyle \phi_J}}\right)^{\textstyle - M}
      & V_D < F_C\phi_J \\
      {{\textstyle C_{J0}} \over {\textstyle F_2}}
      \left( F_3 + {{\textstyle m V_D}\over{\textstyle \phi_J}}\right)
      & V_D \ge F_C\phi_J \\
      \end{array} \right. %}
\label{diode:cj:eqn}
\end{equation}
where
$\phi_J$, $F_C$, $M$ and $C_{J0}$ are the model parameters
{\tt VJ}, {\tt FC}, {\tt M} and {\tt CJ0},
and
\begin{eqnarray}
F_2 & = & ( 1 - F_C ) ^{(1+m)}\\
F_3 & = & 1 - F_C(1+m)
\end{eqnarray}
The diffusion capacitance per unit area
due to charges in transit in the depleted region is
\begin{equation}
C_D = \tau
      {{\textstyle\partial I_D }\over{\textstyle\partial V_D}}
\end{equation}
where $\tau$ is the transit time model parameter {\tt TT}.\\[0.2in]
\noindent\underline{\sl \large AC Analysis}\\[0.1in]
\index{DIODE, AC Analysis}
The AC analysis uses the model of figure  \ref{m.ps} with the capacitor values
evaluated at the \dc\ operating point with
the current source replaced by a linear resistor
\begin{equation}
R_D = {{\textstyle\partial I_D} \over {\textstyle\partial V_D}}
\label{diode:rd:eqn}
\end{equation}\\[0.1in]
\noindent\underline{\sl \large Noise Analysis}\\[0.1in]
\index{DIODE, Noise Model}
\index{DIODE, Noise Analysis}
The DIODE noise model accounts for thermal noise generated in the
parasitic series resistance and shot and flicker noise generated in the
the junction.  The rms (root-mean-square) value of
the thermal noise current generators shunting $R_S$ is
\begin{equation}
I_{n,S} = \sqrt{4kT/R_S}~\mbox{A/}\sqrt{\mbox{Hz}}
\end{equation}
The RMS value of
noise current generators shunting $R_D$ in the \ac\ model is
\begin{equation}
I_{n,J} = \left(I_{\ms{SHOT}}^2 + I_{\ms{FLICKER}}^2
                \right)
\end{equation}
where the RMS shot noise current is
\begin{equation}
I_{\ms{SHOT}} = \sqrt{2qI_D} ~~~~\mbox{A/}\sqrt{\mbox{Hz}}
~\mbox{A/}\sqrt{\mbox{Hz}}\\
\end{equation}
and the RMS flicker noise current is
\begin{equation}
I_{\ms{FLICKER}} = \sqrt{{{\textstyle\KF I_{D}^{\AF}}
                         \over {\textstyle f K_{\ms{CHANNEL}}}}}
~~~~\mbox{A/}\sqrt{\mbox{Hz}}
\end{equation}
where $f$ is the analysis frequency and $R_D$ is the \ac junction resistance
in (\ref{diode:rd:eqn})
