\element{O}{Digital Output Interface}

\begin{figure}[h]
\centering
\ \pfig{o_spice.ps}
\caption{O --- Digital output interface element.}
\end{figure}

\form{{\tt O}name InterfaceNode ReferenceNode ModelName
  \B{\tt SIGNAME = } DigitalSignalName \E}

\begin{widelist}
\item[{\it InterfaceNode}]
     Identifier of node interfacing between digital signal and
     continuous time circuit.
\item[{\it ReferenceNode}]
     Identifier of reference node. Normally this is ground
\item[{\it ModelName}]
     Name of the model specifying transistions times and resistances and
     capacitances of each logic state.
\item[{\tt SIGNAME}]
     Keyword for digital signal name. (optional)
\item[{\it DigitalSignalName}]
     Digital signal name.
\end{widelist}

\example{
    O100 1 0 INTERFACE\_TO\_MEMORY SIGNAME=MEM1 \\
    OADD1 1 0 2 ADD1
    }

\modeltype{DOUTPUT}

The digital output interface is modeled by time variable resistances between
the {\it Interface Node} and the {\it Low Level Node} and between the
the {\it Interface Node} and the {\it High Level Node}.
The variable resistances are shunted by fixed capacitances.
The parameters are controlled by parameters specified in the model.
The resistance varies exponentially from the old state to the new state over
the time period indicated for the new state.
This approximates the output of a digital gate.


\model{DOUTPUT}{Digital Output Interface Model}

\keyword{
{\tt FILE}  & digital output filename.
          If more than one model refers to the same file then the
                  filenames specified must be identical and not logicly
                  equivalent.  This ensures that the file is opened
                  only once.
        & - & \reqd \X
{\tt FORMAT}    & digital output file format    & - & 1 \X
{\tt TIMESTEP}& digital output file time step   & s & 1NS   \X
{\tt TIMESCALE}& digital output file time scale & s & 1     \X
{\tt CHGONLY}   & Output type flag: \newline
                  = 0 $\rightarrow$ output at each {\tt TIMESTEP} \newline
                  = 1 $\rightarrow$ output only on state change
        & - & 0 \X
{\tt CLOAD} & capacitance           & F & 0 \X
{\tt RLOAD} & resistance                    & $\Omega$& 1000\X
{\tt S{\it n}NAME}  & state ``n'' character abreviation\newline
      {\it n} = 0, 2, ..., or 19
    & - & \reqd \X
{\tt S{\it n}VLO}   & state ``n'' low level voltage\newline
      {\it n} = 0, 2, ..., or 19
    & s & \reqd \X
{\tt S{\it n}VHI}   & state ``n'' high level voltage\newline
      {\it n} = 0, 2, ..., or 19
    & s & \reqd \X
}

The digital output interface is modeled by a resistance {\it RLoad}
and capacitance {\it CLoad} between
the {\it InterfaceNode} and the {\it Reference Node }.
The values of {\it Rload} and {\it CLoad} are specified in the model
{\it ModelName}.

A state transistion from state $n$ ($n$ = one of 0, 1, 2, ... 19) is indicated
if the interface voltage
$V_{\mbox{\it InterfaceNode}}$ $-$ $V_{\mbox{\it ReferenceNode}}$
between the {\it InterfaceNode} and the
{\it ReferenceNode} node is outside the range
{\tt S{\it n}VHI} $-$ {\tt S{\it n}VLO}.
If there is a state transistion then the valid voltage range of each state
$k$ is considered in order from state k =  0 to state 19 to determine
which voltage range
{\tt S{\it k}VHI} $-$ {\tt S{\it k}VLO} brackets the current interface
voltage
$V_{\mbox{\it InterfaceNode}}$ $-$ $V_{\mbox{\it ReferenceNode}}$.
The first valid state becomes the new state.  If there is no valid state
then the new state is indeterminate and designated by ``{\tt ?}''.
At each {\tt TIME} being a multiple integer of {\tt TIMESTEP}
a line is written to the
digital output file {\it OutputFileName}.
If the new state at the time $t_i$ = $i \cdot {\tt TIMESTEP}$ is $n$ then
the $i$~th line is:
\begin{equation*}
{{\textrm int}(i \cdot {\tt TIMESCALE})  n}
\end{equation*}
where int() is the integer operation.
An example of the first few lines of {\it OutputFileName} with a
{\tt TIMESTEP} of 1~ns and {\tt TIMESCALE} of 2 is:
\boxed{0.0 1\\
       2 0\\
       4 2\\
       6 3\\
       8 ?\\
      10 1\\
      12 0\\
      14 1
      }
\rm
