\element{A}{Convolution}
\begin{figure}[h]
\centering
\ \pfig{a_spice.ps}
\caption{A --- convolution element.}
\end{figure}
\form{ {\tt A}name  $n_1$ $n_2$ \B $n_3$ ... $n_N$ \E ModelName
}
\begin{widelist}
\item[$n_1$] is the first node (required),
\item[$n_2$] is the second node (required),
\item[$n_3$] is the third node (optional),
\item[$n_N$] is the $N$th node (optional),
\item[{\it ModelName}] is the model name which defines this convolution element.
\end{widelist}
\example{ ANET 1 2 3 4 NETWORK1
        }
\modeltype{CONV}
\model{CONV}{Convolution Model}
\form{ {\tt .MODEL} ModelName {\tt CONV(} \B  \B keyword = value\E  ... \E
}
\example{
.MODEL CONV1 CONV(  FILE = "coupledline.y" NFREQS = 500\\
+         NPORTS = 4 ZM = 50 THRESHOLD = 0.01)}

