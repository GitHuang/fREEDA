\addcontentsline{toc}{part}{PART III CATALOG}
\chapter{Statement Catalog\label{chapter:statement}}

\mysline{.AC}{AC Analysis}
\mysline{COMMENT}{Comment Line}
\mysline{.DC}{DC Analysis}
\mysline{.DISTO}{Small-Signal Distortion Analysis
\spicetwo\ (?) and \spicethree\ Only}
\mysline{.DISTRIBUTION}{Distribution Specification
         {(\pspice\ only)}}
\mysline{.END}{End Statement}
\mysline{.ENDS}{End Subcircuit Statement}
\mysline{.FOUR}{Fourier Analysis}
\mysline{.FUNC}{Function Definition \pspiceninetytwo Only}
\mysline{.IC}{Initial Conditions}
\mysline{.INC}{Include Statement}
\mysline{.LIB}{Library Statement}
\mysline{.MC}{Monte Carlo Analysis {(\pspice\ only)}}
\mysline{.MODEL}{Model Statement}
\mysline{.NODESET}{Node Voltage Initialization}
\mysline{.NOISE}{Small-Signal Noise Analysis}
\mysline{.OP}{Operating Point Analysis}
\mysline{.OPTIONS}{Option Specification}
\mysline{.PARAM}{Parameter Definition {\pspiceninetytwo\ Only}}
\mysline{.PLOT}{Plot Specification}
\mysline{.PRINT}{Print Specification}
\mysline{.PROBE}{Data Output Specification {\pspice\ Only}}
\mysline{.PZ}{Pole-Zero Analysis \spicethree\ Only}
\mysline{.SAVEBIAS}{Save Bias Conditions}
\mysline{.SENS}{Sensitivity Analysis}
\mysline{.STEP}{Parameteric Analysis \pspiceninetytwo\ Only}
\mysline{.SUBCKT}{Subcircuit Statement}
\mysline{.TEMP}{Temperature Specification}
\mysline{.TEXT}{Text Parameter Definition \pspiceninetytwo\ Only}
\mysline{.TF}{Transfer Function Specification {\pspice\ Only}}
\mysline{TITLE}{Title Line}
\mysline{.TRAN}{Transient Analysis}
\mysline{.WATCH}{Watch Analysis Statement \pspiceninetytwo\ Only}
\mysline{.WCASE}{Sensitivity and Worst Case Analysis
\pspiceninetytwo\ Only}
\mysline{.WIDTH}{Width Specification}
\raggedbottom
\section*{Conventions}

Square brackets ``[ $\ldots$ ] ''indicate an optional quantity.

Italics indicate a quantity that is replaced by a specific value.


%
% .AC
%
\statement{.AC}{AC Analysis}

The {\tt .AC} statement initializes an \ac\ small-signal analysis sweeping the
frequency of the independent voltage and current sources.

\form{{\tt .AC DEC} FrequenciesPerDecade FStart FStop\\
      {\tt .AC OCT} FrequenciesPerOctave FStart FStop\\
      {\tt .AC LIN} NumberPoints FStart FStop}


\begin{widelist}
\item[{\tt DEC}] is the decade sweep keyword specifying that the
     frequency $F_1$ is to be swept logarithmically by decades.

\item[{\it FrequenciesPerDecade}]
      specifies the number of frequencies per decade.

\item[{\tt OCT}] is the octave sweep keyword specifying that the
     frequency is to be swept logarithmically by octaves.

\item[{\it FrequenciesPerOctave}]
      specifies the number of frequencies per octave.

\item[{\tt LIN}] is the linear sweep keyword specifying that the frequency
     is to be swept linearly.

\item[{\it NumberPoints}]
      specifies the total number of frequencies in a linear sweep.

\item[{\it FStart}]
      is the starting frequency of the frequency sweep.\\
      (Units: Hz; Required; {\it FStart} $>$ 0)

\item[{\it FStop}]
      is the stopping frequency of the frequency sweep.\\
      (Units: Hz; Required; {\it FStop} $\ge$ {\it FStart})
\end{widelist}


\example{.AC DEC 10 1kHz 100Mhz\\
         .AC DEC 10 1kHz 100Mhz}

\note{
\item A \dc\ analysis is automatically performed
      prior  to  an \ac\ small-signal analysis to find the operating point.
      Using the \dc\ values of voltage and current at the operating point
      the  linearized, small-signal models of nonlinear devices are determined.

\item In \ac\ analysis voltage sources without \ac\ specifications are shorted
and current sources without \ac\ specifications are opened.
}



%
\statement{COMMENT}{Comment Card}

Used to insert a comment in the circuit NETLIST.

\form{{\tt *} a comment string}

\note{
\item
A comment line begins with an asterisk ``*'' in the first position of the line.
There can be no leading white space.
\item
Comment lines can appear anywhere in the input file and are ignored
except that they are echoed in the output log file.
\item {In \pspice comments} can appear anywhere
on following a semicolon ``;''.
The remainder of the line (following the ``;'') is ignored except for purposes of
echoing the input NETLIST in the log file.
}

%
% .DC
%
\statement{.DC}{DC Analysis}

In \dc\ analysis the \dc\ operating  point  of a circuit is determined for
a range of values of up to two independent voltage or current sources.\\[0.1in]

\form{{\tt .DC}
  SourceName1 StartValue1 StopValue1 ValueIncrement1\\
  {\tt +} \B SourceName2 StartValue2 StopValue2 ValueIncrement2\E }

\pspiceform{{\tt .DC} \B {\tt LIN} \E
  SweepVariableName1 StartValue1 StopValue1 ValueIncrement1\\
  {\tt +} \B SourceName2 StartValue2 StopValue2 ValueIncrement2\E \\[0.1in]
  {\tt .DC} {\tt OCT}
  SweepVariableName1 StartValue1 StopValue1 PointsPerOctave1\\
  {\tt +} SweepVariableName2 StartValue2 StopValue2 PointsPerOctave2 \E \\[0.1in]
  {\tt .DC} {\tt DEC}
  SweepVariableName1 StartValue1 StopValue1 PointsPerDecade1\\
  {\tt +} SweepVariableName2 StartValue2 StopValue2 PointsPerDecade2 \E \\[0.1in]
  {\tt .DC} SweepVariableName1 {\tt LIST} Value$_{1,1}$\B Value$_{1,2}$ $\ldots$
  Value$_{1,N}$\E\\
  {\tt +} \B SweepVariableName2 {\tt LIST} Value$_{2,1}$\E Value$_{2,2}$ $\ldots$
  Value$_{2,N}$\E\E\\
  }

\begin{widelist}
\item[{\it SourceName1}] is the name of the first independent voltage
({\tt V} element) or current ({\tt I} element) source the value of which will be
swept.

\item[{\it StartValue1}] is the starting value of the sweep of the first
voltage or current source.

\item[{\it StopValue1}] is the final value of the sweep of the first
voltage or current source.

\item[{\it ValueIncrement1}] is the increment by which the value of the first
voltage or current source is incremented.

\item[{\it SourceName2}] is the name of the first independent voltage
({\tt V} element) or current ({\tt I} element) source the value of which will be
swept.

\item[{\it StartValue2}] is the starting value of the sweep of the second
voltage or current source.

\item[{\it StopValue2}] is the final value of the sweep of the second
voltage or current source.
\end{widelist}

\begin{widelist}
\item[{\it ValueIncrement2}] is the increment by which the value of the second
voltage or current source is incremented.

\item[{\tt LIN}] is the linear sweep keyword. This is the default sweep type.

\item[{\tt OCT}] is the octave sweep keyword specifying that the sweep
variable or variables is to be swept logarithmically by octaves.

\item[{\tt DEC}] is the decade sweep keyword specifying that the sweep
variable or variables is to be swept logarithmically by decades.

\item[{\it PointsPerOctave1}] is the number of points per octave in a {\tt OCT}
     sweep type for the first sweep.

\item[{\it PointsPerDecade1}] is the number of points per decade in a {\tt DEC}
     sweep type for the first sweep.

\item[{\it PointsPerOctave2}] is the number of points per octave in a {\tt OCT}
     sweep type for the second sweep.

\item[{\it PointsPerDecade2}] is the number of points per decade in a {\tt DEC}
     sweep type for the second sweep.

\item[{\it SweepVariableName1}] is the name of the first sweep variable.
The sweep variable can be:
\begin{enumerate}
\item the name of an independent voltage or current source. The {\it DCvalue}
      of the source is swept.
\item the name of a parameter of a specific model specified in the form
     {\it ModelName}{\tt (}{\it ParameterKeyword}{\tt )}. For example to sweep
     the {\tt IS} parameter of an {\tt NPN} model of name {\tt MYNPN} the
     {\it SweepVariableName} would be {\tt MYNPN(IS)}.
\item the keyword {\tt TEMP} which indicates that the analysis temperature
     in $^{\circ}$C is swept. The model parameters are updated for each
     sweep value.
\end{enumerate}

\item[{\it SweepVariableName2}] is the name of the second sweep variable.
     The properties are as for {\it SweepVariableName1} as described above.

\item[{\tt LIST}] indicates that the value of the {\it SweepVariable} will take the
     values, in order, in the following list rather than be swept.

\item[{\it Value$_{\,i,j}$}] the $j$th value to be assigned to the $i$th sweep
     variable.
\end{widelist}

\example{.DC VIN 0.25 5.0 0.25\\
         .DC VDS 0 10 .5 VGS 0 5 1\\
         .DC VCE 0 10 .25 IB 0 10U 1U\\
        }

\pspiceexample{.DC VIN 0.25 5.0 0.25\\
         .DC VDS 0 10 .5 VGS 0 5 1\\
         .DC VCE 0 10 .25 IB 0 10U 1U\\
         .DC LIN VCE 0 10 .25 IB 0 10U 1U\\
         .DC DEC MYNPN(IS) 1.E-15 1.e-17 3\\
         .DC TEMP -25 0 25 50 75 100 VIN 0.25 5.0 0.25
        }

\note{
\item The {\tt .DC} statement initiates a \dc\ operating point analysis.

\item In the \dc\ analysis inductors are shorted and capacitors are open
      circuited.

\item A \dc\ analysis over a range of source conditions can be used to produce
      the transfer characteristic of a circuit or current-voltage
      characteristics of a semiconductor device such as a transistor.
      An example of determining the transfer characteristic of an operational
      amplifier is given in section \ref{ex:opamp:DC}.
      An example of determining the current-voltage
      characteristics of a transistor
      is in section \ref{ex:opamp:DC}.

\item[3] A \dc\ analysis over a range of source conditions can also be used
      to provide biasing information in circuit design.

\item[4] The specified independent voltage
or current source or sources are stepped over a user-specified range and
the  \dc\  output  variables indicated by the {\tt .PRINT} statement are stored
for each source value.

\item[5] When two sources are specified the first
source is swept over its range for each  value  of  the
second source. For example, consider\\
\centerline{\tt .DC VDS 0 10 5 VGS 0 5 2.5}\\
{\tt VGS} is swept from 0 to 5V in 2.5V increments.  For each value of {\tt VGS}
{\tt VDS} is swept from 0 to 10V in 5V increments so that that 9 \dc\ analyses
performed are\\[0.1in]
\hspace*{\fill}
\begin{tabular}{|c|r|r|}
\hline
\multicolumn{1}{|c}{RUN}&
\multicolumn{1}{|c}{{\tt VDS}}&
\multicolumn{1}{|c|}{{\tt VGS}}\\
\hline
1 & 0 & 0\\
2 & 5 & 0\\
3 &10 & 0\\
4 & 0 &2.5\\
5 & 5 &2.5\\
6 &10 &2.5\\
7 & 0 & 5\\
8 & 5 & 5\\
9 &10 & 5\\
\hline
\end{tabular}
\hspace*{\fill}\\
\item[6] When the sweep is completed the original values are restored.
\item[7] The sweep can go in either direction. That is,
{\it StartValue} can be less than or greater than {\it StopValue} but
{\it ValueIncrement} must be positive.
}

%
% .DISTO
%
{
\statement{.DISTO}{Small-Signal Distortion Analysis}

In distortion  analysis the steady-state harmonic and intermodulation
products for small input signal are computed as a part of an \ac\ analysis.
One or two excitation frequencies, $F_1$ and $F_2$ may be specified.
If only one excitation frequency, $F_1$, is specified the program
evaluates the second and third harmonic distortions. If a second excitation
frequency, $F_2$, is specified the three lowest order intermodulation
distortion components are evaluated as well.

\form{{\tt .DISTO} ResistorName \B OutputInterval
     \B $F_2$Over$F_1$ \B $F_1$ReferencePower\\
     {\tt +} \B $F_2$ReferencePower\E\E\E\E
     }

\spicethreeform{{\tt .DISTO DEC} FrequenciesPerDecade $F_1$Start $F_1$Stop
                       \B $F_2$Over$F_1$\E\\
      {\tt .DISTO OCT} FrequenciesPerOctave $F_1$Start $F_1$Stop
                       \B $F_2$Over$F_1$\E\\
      {\tt .DISTO LIN} NumberPoints $F_1$Start $F_1$Stop \B $F_2$Over$F1$\E
     }


\begin{widelist}
\item[{\it ResistorName}]
     the name of the output resistor. The power dissipated in this resistor
     is reported as the distortion measures.
     \spicetwo\ only.

\item[{\it OutputInterval}]
     is the optional output reporting interval at which distortion components
     produced by all nonlinear components is reported.  By default, or if
     omitted no detailed output is produced.
     \spicetwo\ only.

\item[{\it $F_1$ReferencePower}]
     power level of $F_1$.
     \spicetwo\ only.\\
     (Units: W; Optional; Default: 1.0E-3 (i.e. 1~mW or 1~dBm))

\item[{\it $F_2$ReferencePower}]
     power level of $F_2$.
     \spicetwo\ only.\\
     (Units: W; Optional; Default: 1.0E-3 (i.e. 1~mW or 1~dBm))

\item[{\tt DEC}] is the decade sweep keyword specifying that the
     frequency $F_1$ is to be swept logarithmically by decades.

\item[{\it FrequenciesPerDecade}]
      specifies the number of frequencies per decade.

\item[{\tt OCT}] is the octave sweep keyword specifying that the
     frequency $F_1$ is to be swept logarithmically by octaves.

\item[{\it FrequenciesPerOctave}]
      specifies the number of frequencies per octave.
\end{widelist}

\begin{widelist}

\item[{\tt LIN}] is the linear sweep keyword specifying that the frequency
     $F_1$ is to be swept linearly.

\item[{\it NumberPoints}]
      specifies the total number of frequencies in a linear sweep.

\item[{\it $F_1$Start}]
      is the starting frequency of the $F_1$ sweep.

\item[{\it $F_1$Stop}]
      is the stopping frequency of the $F_1$ sweep.

\item[{\it $F_2$Over$F_1$}]
      is the ratio of $F_2$ to $F_1$.

      In \spicethree\ , if  {\it $F_2$Over$F_1$} is omitted an harmonic
      analysis only is reported. Otherwise
      $F_2 = ${\it $F_2$Over$F_1\,F_1$Start}

      In \spicetwo\ , if {\it $F_2$Over$F_1$} is omitted it defaults to 0.9.
      $F_2 = ${\it $F_2$Over$F_1\,F_1$}

      {\it $F_2$Over$F_1$} should be an irrational number between 0.0 and
      1.0. If it is a rational number the signals at $F_1$ and $F_2$
      are harmonically related and the spectral analysis may in error.
      Since an irrational number can not actually be specified
      care should be excercized in choosing
      {\it $F_2$Over$F_1$} so that $F_1$ and $F_2$ are not simple multiples of
      each other.  For example, use 0.498 instead, if you want to set
      {\it $F_2$Over$F_1$} to 0.5.  The rule to follow is to keep
      the denominator in the fractional representation of {\it $F_2$Over$F_1$}
      as large as possible with at least three digits for accurate  results.
\end{widelist}


\example{.DISTO DEC 10 1kHz 100Mhz\\
         .DISTO DEC 10 1kHz 100Mhz 0.9}


\note{
\item
In \spicetwo\ the distortion analysis must be performed in conjunction
with an \ac\ analysis.  The {\tt .AC} statement (see page
\pageref{.ACstatement}) specifies the sweep parameters for $F_1$.

\item
In \spicethree\ the distortion analysis is also performed in conjunction
with the \ac\ analysis specified by the {\tt .AC} statement.
The amplitudes and relative phases of the input distortion components are
specified in the {\tt .AC statement} as the arguments of the {\tt DISTOF1}
keyword for $F_1$ and of the {\tt DISTOF2} keyword for $F_2$.
This enables several sources to generate components at $F_1$ and/or $F_2$.
If the DISTOF1 or DISTOF2 keywords are missing from a source then this source
is assumed
to have no input at $F_1$ or $F_2$ respectively.

When a spectral analysis is performed (both $F_1$ and $F_2$ specified)
the  circuit is treated as having sinusoidal inputs at two
different frequencies $F_1$ and $F_2$.
$F_1$ is swept but $F_2$ is fixed at
{\it $F_2$Over$F_1\,$}{\it FStart}. Each independent
source in the circuit can have two
inputs at $F_1$ and $F_2$ for distortion analysis.
}

\note{
\item[3]
Distortion analysis is not valid if switches (if present)
change state under the small excitations used for distortion calculations.

\item[4]
Evaluation of the small-signal  distortion of
a  circuit is based on a multi-dimensional Volterra series analysis.
The  nonlinearities are expanded in
a third order multi-dimensional Taylor series around
the  operating  point determined from an operating point (\dc ) analysis.
Using this analysis the distortion components are evaluated symbolically

\item[5] In \spicetwo\ the following distortion components are
       evaluated.\\[0.1in]
      \offset\begin{tabular}{lcp{3.5in}}
      HD2 &-&the magnitude of the second harmonic at frequency
             $2F_1$ assuming that $F_2$ is not present.\\
      HD3 &-&the magnitude of the third harmonic at frequency
             $3F_1$ assuming that $F_2$ is not present.\\
      SIM2 &-&the magnitude of the sum frequency $F_1+F_2$\\
      DIM2 &-&the magnitude of the difference frequency $F_1-F_2$\\
      DIM3 &-&the magnitude of the third order intermodulation frequency
              $2F_1-F_2$
      \end{tabular}

\item
      In \spicethree\ the complex values of the above distortion components are
      computed at all nodes in the circuit. The distortion components at any
      node can be reported using the {\tt .PRINT} statement (discussed on
      page \pageref{.PRINTstatement}) or the {\tt .PLOT} statement (discussed on
      page \pageref{.PLOTstatement}). The running variable in the output is the
      frequency $F_1$.

\item
      The quantities reported are the actual \ac\ voltages
      and currents and must be normalized by the user to the sources at $F_1$ and
      $F_2$ to obtain true distortion measures.
}

\fornutmeg{
The distortion component  desired  (2*F1  or  3*F1)  can  be
selected using commands in nutmeg, and then printed or plotted.
(Normally, one is interested primarily in the magnitude
of  the  harmonic  components,  so the magnitude of the a.c.
distortion value is looked at.)  It  should  be  noted  that
these are the a.c. values of the actual harmonic components,
and are not equal to HD2 and HD3. To obtain HD2 and HD3, one
must divide by the corresponding a.c. values at F1, obtained
from an .AC card. This division can  be  done  using  nutmeg
commands.}



\fornutmeg{
The analysis produces plots of all node voltages/branch currents
at the intermodulation product frequencies F1+F2, F1-F2, and
(2*F1)-F2,  versus  the  swept  frequency  F1. The intermodulation product of
interest may be selected  using  the  setplot  command,  and
displayed  with  the  print  and  plot commands. It is to be
noted as in the harmonic analysis case, the results are  the
actual  a.c.  voltages  and  currents at the intermodulation
frequencies, and need to be normalized w.r.t .AC  values  to
obtain the intermodulation parameters.}

} % end
%
% .DISTRIBUTION
%
\statement{.DISTRIBUTION}{Distribution Specification}

The {\tt .DISTRIBUTION} statement specifies the statistical tolerance
distribution used in Monte Carlo analysis (see the {\tt .MC}
statement on page \pageref{.MCstatement}).

\pspiceform{{\tt .DISTRIBUTION} DistributionName
{\tt (}{\it Deviation1, Probability1}{\tt )}\\{\tt +}
{\tt (}{\it Deviation2, Probability2}{\tt )}
\B {\tt (}{\it Deviation3, Probability3}{\tt )} ... \E}


\begin{widelist}
\item[{\it DistributionName}] is the name the user  assigns to the distribution defined by the succeeding values.

\item[{\it Deviation}] Relative deviation from nominal value.\\
($-1 \le$ {\it Deviation} $ \le 1$).

\item[{\it Probability}] Probability of preceding deviation.\\
({\it Probability} $\ge$ 0)

\end{widelist}

\note{
\item Parentheses, `(' and `)', and commas, `,', in the NETLIST are ignored and
      are generally used to make the NETLIST more readable.

\item The pairs of values ({\it Deviation}, {\it Probability}) define a
      piecewise linear
      probability distribution curve used in calculating the random numbers
      used in Monte Carlo analysis.

\item The {\it Deviation}s must be in ascending order:\\
      {\it Deviation1} $<$ {\it Deviation2} $<$ {\it Deviation3} $<$ $\ldots$

\item 100 ({\it Deviation}, {\it Probability}) pairs can be specified.

%      For example the truncated
%      distribution in figure \ref{tgdist} is specified by\\
%      {\tt .DISTRIBUTION TRUNCATED (hg,kg) (hg,kg) (hg,kg) (hg,kg)}
%\begin{figure}[hbp]
%\centering
%\ %\pfig{tgdist.ps}
%\caption{User specified truncated distribution.\label{tgdist}}
%\end{figure}
}

%
% .END
%
\statement{.END}{End Statement}

This statement indicates the end of the NETLIST of a circuit.
{
While it is often inserted automatically by most \spice\ simulators when the
last statement of the NETLIST is read and it is not {\tt .END}
}
its usage is
recommended as several circuits in one file are supported by
{some versions of \spice\ including \pspice .}

\form{{\tt .END}}


%
% .ENDS
%
\statement{.ENDS}{End Subcircuit Statement}

The end subcircuit statement indicates the end of a subcircuit definition.

\form{{\tt .ENDS} \B SubcircuitName\E }

\begin{widelist}
\item[{\it SubcircuitName}] Indicates which subcircuit is being terminated.
If  omitted, all subcircuits being defined are terminated.
Its use is required when nested subcircuit definitions are being made.
It is good practise to use the full form of {\tt .ENDS}.
\end{widelist}

\example{.ENDS NAND\_GATE}

\note{
\item See the definition of {\tt .SUBCKT} on page \pageref{.SUBCKTstatement}.
}

%
% .FOUR
%
\statement{.FOUR}{Fourier Analysis}

The Fourier analysis statement initiates a Fourier analysis of the results of
a transient analysis.

\form{{\tt .FOUR} Frequency OutputSpecification \B OutputSpecification $\ldots$ \E}

\begin{widelist}
\item[{\it Frequency}] specifies the fundamental frequency of the transient
waveform to be analyzed.  It is used to determine the period of the waveform.

\item[{\it OutputSpecification}]  specifies the quantity to be reported
     as the result of the Fourier Analysis.  It has the same format as the
     {\it OutputSpecification} in a {\tt .PRINT} statement (see page
     \pageref{.PRINTstatement}).
\end{widelist}

\example{.FOUR 1M V(10,2) V(5) I(VLOAD)}

\note{
\item
Fourier analysis uses the transient output waveform in the time interval from
${\tt TSTOP} -T$ to {\tt TSTOP}.  $T$ is the period of the {\it Frequency}
parameter ($T=1/{\it Frequency}$) specified
in the {\tt .FOUR} statement.  {\tt TSTOP} is the final transient analysis time
specified in the {\tt .TRAN} statement (which is described on page
\pageref{.TRANstatement}.

\item Unlike most other analyses, a {\tt .PRINT}, {\tt .PROBE}
or {\tt .PLOT} statement is not required for the data to be reported.
}

%
% .FUNC
%
{
\statement{.FUNC}{Function Definition}

In this statement commonly used expressions can be defined as more convenient
functions.

\pspiceninetytwoform{{\tt .FUNC}
FunctionName{\tt (}\B Argument1, Argument2, $\ldots$ Argument10\E\E\E{\tt )}
= FunctionDeclaration
}

\begin{widelist}
\item[{\it FunctionName}] is the name of the function being defined.  It
     must begin with an alphabetic character (A-Z).

\item[{\it Argument1}] is a function argument.
     There can be from 0 to 10 arguments.

\item[{\it FunctionDeclaration}] can be any regular algebraic expression
     (see section \ref{section:algebraic:expression}
     on page \pageref{section:algebraic:expression}) and
     can use previously defined functions and the Laplace variable $s$.
     The expression delimiters
     {\tt \{} and {\tt \}} need not be used.
     The {\it FunctionDeclaration} is automatically enclosed within the expression
     delimiters {\tt \{} and {\tt \}}. The function declaration plus the
     two delimiters must be no more than 80 characters (one line) long.
\end{widelist}

\example{.FUNC}

\note{
\item The names of predefined functions must be avoided.  The predefined
      functions are listed in section \ref{section:algebraic:expression}
      on page \pageref{section:algebraic:expression}.

\item Functions are treated as macros in the C programming language.
      when user defined functions are invoked a textual expansion is
      performed and the resultant expansion is evaluated as a regular
      expression.  The FunctionDeclaration before and after expansion is
      enclosed within expression delimiters {\tt \{} and {\tt \}}.  This
      defines how nested functions are treated.

\item It is faster to use predefined functions if available. Predefined
functions also test the validity of the arguments and evaluate the correct
as asymptotic behavior.
}

} % end

%
% .IC
%
\statement{.IC}{Initial Conditions}

The {\tt .IC} statement is used to set initial conditions for transient
analysis. It has no effect on other types of analyses.


\form{{\tt .IC V(}NodeName1{\tt )}=Voltage1 \B V(NodeName2{\tt )}
     =Voltage2 $\ldots$ \E}

\begin{widelist}
\item[{\tt V}] is the keyword specifying a node voltage

\item[{\it NodeName}]
     is the name of a node.
     {Note that in \spicetwo\ and \spicethree\ {\it NodeName}
     must be an integer.}

\item[{\it Voltage}] is a numeric value.
\end{widelist}

\example{ .IC V(11)=4.9 V(2)=2.5}

\note{
\item Initial conditions can be specified for the following elements:
      {\tt B} (MESFET),
      {\tt C} (capacitor),
      {\tt D} (diode),
      {\tt Q} (BJT),
      {\tt M} (MOSFET) and
      {\tt J} (JFET)
\item The {\tt .IC} statement has two different effects depending on whether
      the {\tt UIC} (use initial conditions) keyword is present on the
      {\tt .TRAN} statement.
      \begin{enumerate}
      \item If the {\tt UIC} keyword is \underline{specified} in the {\tt .TRAN}
        statement:\\
        The initial conditions specified in the {\tt .IC} statement are used
        to establish the initial conditions.  Initial conditions specified
        for individual elements using the {\tt IC} parameter on the element
        line will always have precedence over those specified in a {\tt .IC}
        statement.\\
        No \dc\ analysis is performed prior to a transient analysis. Thus it
        is important to establish the initial conditions at all nodes using
        the {\tt .IC} statement or using the {\tt IC} element parameter.
      \item If the {\tt UIC} keyword is \underline{not specified} in the {\tt .TRAN}
        statement:\\
        A \dc\ analysis is performed prior to a transient analysis. During
        the \dc\ analysis the node voltages indicated in
        the {\tt .IC} statement are held constant at the initial
        condition values. During transient analysis the nodes are not
        constrained to the initial condition values.
      \end{enumerate}
}

%
% .INC
%
\statement{.INC}{Include Statement}

\form{ {\tt .INC} FileName}

\begin{widelist}
\item[{\it Filename}] is the name of the file which is to be included.
\end{widelist}

\note{
\item The contents of {\it Filename} are read as if it were part of the original
      file.

\item Libraries could be included using the {\tt .INC} statement or by
      the {\tt .LIB} statement discussed on page \pageref{.LIBstatement}.
      The difference is that the {\tt .INC}
      statement includes the contents of the library file (except for
      comments) in internal data structures.  However a {\tt .LIB}
      statement causes the library file to be scanned and an index
      constructed for model ({\tt .MODEL}) and subcircuit ({\tt .SUBCKT})
      statements.  The models subcircuits in the library file are included in
      internal data structures if they are referred to.  Thus
      the use of {\tt .LIB} statements is a much more efficient way of using
      library files leading to both a smaller program and faster library access,
      especially for large libraries.  Library files can only contain
      models and subcircuits so that the type of files that can be
      incorporated using a {\tt .LIB} statement is more restricted than the
      type of file that can be incorporated using a {\tt .INC} statement.
}
%
% .LIB
%
\statement{.LIB}{Library Statement}

The {\tt .LIB} statement is an efficient way to include {\tt .MODEL} statements
and subcircuits.

\form{ {\tt .LIB} \B FileName \E}

\begin{widelist}
\item[{\it Filename}] is the name of the library file.
     (Optional; Default: NOM.LIB)
\end{widelist}

\note{
\item The library file can only contain a restricted set of \spice\ statements.
      It must contain only {\tt .MODEL} statements, subcircuit definitions
      (between {\tt .SUBCKT} and {\tt .ENDS} statements), and {\tt .LIB}
      statements.

\item The library file {\it Filename} is searched in the current directory
      and then in a list of directories specified by the environment variable
      {{\tt PSPICELIB}}
      (for compatibility purposes the environment
      variable {\tt PSPICELIB} is also supported).
\item If the DOS operating system is being used the library environment
      specification has the form
      \begin{itemize}
      \item[] {\tt SET PSPICELIB =} {\it Directory1} \B {\tt ;}
              {\it Directory2} $\ldots$ \E
      \end{itemize}
      The environment variable may be set in the {\tt AUTOEXEC.BAT} file
      in the root DOS directory or before \pspice\ is evoked.
      For example:
      \begin{itemize}
      \item[] {\tt SET PSPICELIB =
      C:$\backslash$SPICE$\backslash$MY\_LIB;D:$\backslash$SPICE$\backslash$TI\_LIB}
      \end{itemize}
}

\offsetenumerate{
\item[4.] If the UNIX operating system is being used the library environment
      specification has the form
      \begin{itemize}
      \item[] {\tt setenv PSPICELIB} = {\it Directory1} \B {\tt ;}
              {\it Directory2} $\ldots$ \E
      \end{itemize}
      For example:
      \begin{itemize}
      \item[] {\tt setenv PSPICELIB = ~/SPICE/MY\_LIB;~/SPICE/TI\_LIB}
      \end{itemize}
\item[5.] Libraries could be included using the {\tt .INC} statement or by
      the {\tt .LIB} statement.  The difference is that the {\tt .INC}
      statement includes the contents of the library file (except for
      comments) in internal data structures.  However a {\tt .LIB}
      statement causes the library file to be scanned and an index
      constructed of model ({\tt .MODEL}) and subcircuit ({\tt .SUBCKT})
      statements.  The models subcircuits in the library file are included in
      internal data structures only if they are referred to.  Thus
      the use of {\tt .LIB} statements is a much more efficient way of using
      library files leading to both a smaller program and faster library access,
      especially for large libraries.
}



%
% .MC
%
\statement{.MC}{Monte Carlo Analysis}

The Monte Carlo analysis is a statistical analysis of the circuit causing
the circuit to be analyzed many times with a random change of model parameters
(parameters in a {\tt .MODEL} statement).

\pspiceform{{\tt .MC} NumberOfRuns AnalysisType OutputSpecification
                OutputFunction \B {\tt LIST}\E\\
                {\tt +} \B {\tt OUTPUT(} OutputSampleType {\tt )}\E
     }

\pspiceninetytwoform{{\tt .MC} NumberOfRuns AnalysisType OutputSpecification
                OutputFunction \B {\tt LIST}\E\\
                {\tt +} \B {\tt OUTPUT(} OutputSampleType {\tt )}\E
                \B {\tt RANGE(}LowValue{\tt ,} HighValue{\tt )}\E\\
        {\tt +} \B {\tt SEED=}SeedValue\E
     }

\begin{widelist}
\item[{\it NumberOfRuns}] is the total number of runs to do. This number
includes the initial nominal run.

\item[{\it AnalysisType}] is the type of analysis to be performed in the Monte
     Carlo runs after the initial nominal run (using the nominal values of
     model parameters).  All analyses
     specified in the {\tt NETLIST} are performed in the nominal run.
     The {\it AnalysisType} must be one of the following:
\begin{widelist}
\item[{\tt DC}] is a keyword indicating that the \dc\ analysis as
     specified by the {\tt .DC} statement is repeated.
     The sweep variable used in analyzing the output
     {\it OutputSpecification} is the value of the independent voltage
     or current source specified in the {\tt .DC} statement (as discussed
     on page \pageref{.DCstatement}).

\item[{\tt AC}] is a keyword indicating that the \ac\ small-signal analysis as
     specified by the {\tt .AC} statement is repeated.
     The sweep variable used in analyzing the output
     {\it OutputSpecification} is frequency.

\item[{\tt TRAN}] is a keyword indicating that the transient analysis as
     specified by the {\tt .TRAN} statement is repeated.
     The sweep variable used in analyzing the output
     {\it OutputSpecification} is time.
\end{widelist}

\item[{\it OutputSpecification}]  specifies the quantity to be reported
     as the result of the Monte Carlo Analysis.  It has the same format as the
     {\it OutputSpecification} in a {\tt .PRINT} statement (see page
     \pageref{.PRINTstatement}). The result is the value of the
     {\it OutputSpecification} with respect to a sweep for {\tt DC} and {\tt AC}
     analysis, and as a waveform for {\tt TRAN analysis}.

\end{widelist}
\clearpage

\begin{widelist}
\item[{\it OutputFunction}] indicates the function to be performed on the
     output indicated by {\it OutputSpecification} to reduce the sweep
     or waveform at each run to a single numeric value.
     {
     The {\it OutputSpecification} must be one of the following keywords:}
     \begin{widelist}
     \item[{\tt YMAX}] which produces the greatest deviation of the sweep or
                       waveform from the nominal run.

     {
     \item[{\tt MAX}] which results in the maximum value in each sweep or
                      waveform.

     \item[{\tt MIN}] which results in the minimum value in each sweep or
                      waveform.

     \item[{\tt RISE\_EDGE({\it Value})}] which reports as the result the
                      first run when the waveform crosses above the threshold
              {\it Value}.
              The algorithm used requires that
              one point in the waveform be below {\it Value}
              and the succeeding point be above {\it Value}.

     \item[{\tt FALL\_EDGE({\it Value})}] which reports as the result the
                      first run when the waveform crosses below the threshold
              {\it Value}.
              The algorithm used requires that
              one point in the waveform be above {\it Value}
              and the succeeding point be below {\it Value}.
     }
     \end{widelist}


\item[{\tt LIST}] is an optional keyword that results in the model parameter
                  values that are statistically varied
                  being printed out prior to each run. If it is omitted
          then the statistically generated model parameter
          values are not produced prior to each run.


\item[{\tt OUTPUT}] is an optional keyword indicating the type of output to
                  be produced by runs after the initial nominal run. The
          output produced for each run sampled is determined by the
          {\tt .PLOT}, {\tt .PRINT} and {\tt .PROBE} statements in
          the NETLIST. If this keyword is missing output is produced
          only for the nominal run.

\end{widelist}

\begin{widelist}

\item[{\it OutputSampleType}] indicates the method by which runs are selected
                  for output reporting.
          The output produced for each run selected is determined by the
          {\tt .PLOT}, {\tt .PRINT} and {\tt .PROBE} statements in
          the NETLIST.
          {\it OutputSampleType} must be one of the following:
          \begin{widelist}
          \item[{\tt ALL}]
          indicates that  the output is to be produced for all runs.

          \item[{\tt FIRST {\it Nruns}}]
          indicates that  the output is to be produced only for the
          first {\it Nruns} runs.

          \item[{\tt EVERY {\it NthRun}}]
          indicates that  the output is to be produced for
          {\it NthRun}th run.

          \item[{\tt RUNS} {\it Run1 \B Run2 ... \B Run25 \E\E}]
          indicates that  the output is to be produced for the
          indicated runs. Up to 25 runs can be indicated.

          \end{widelist}
{
\item[{\tt RANGE}] is an optional range indicating the range of the sweep
                  variable over which {\it OutputFunction} is to be performed.
          If this keyword is missing, output is produced but
          the range is not restricted.  The range of the sweep
          variable to be considered is from {\it LowValue}
          to {\it HighValue} inclusive.

\item[{\it LowValue}] is the low end of the sweep variable to be considered
                  in evaluating {\it OutputFunction}.


\item[{\it HighValue}] is the low end of the sweep variable to be considered
                  in evaluating {\it OutputFunction}.


\item[{\tt SEED}] is the keyword for the seed of the random number generator
                  used in Monte Carlo Analysis

\item[{\it SeedValue}] is the value of the seed used in the random number
generator used to select sample runs at random.
    (Optional; Default: 17,533; $1 \le$ {\it SeedValue}$ \le 32,767$)
}

\end{widelist}


\note{
\item A typical use of Monte Carlo analysis is to predict yield of a circuit
      by examining the effect of process variations such as length and width
      of transistors.

\item A {\tt .TEMP} statement also can result in multiple circuit
      simulations as the temperature is varied. In conjunction with a {\tt .MC}
      statement a Monte Carlo analysis is performed for each temperature before
      the temperature is updated in a temperature sweep.

\item Only model parameters (parameters in a {\tt .MODEL} statement) are
      varied.

\item If the {\it AnalysisType} is {\tt DC} only one independent voltage or
      current source can be specified in the {\tt .DC} statement (discussed on
      page \pageref{.DCstatement}).

\item The random number generator is the subtractive method generator described
      by Knuth \cite[p. 171]{knuth:subtractive}.

\item The initial run uses the nominal parameter values given in the NETLIST.
      Subsequent runs statistically vary model parameters indicated as having
      either lot {\tt LOT} or device {\tt DEV} tolerances. These tolerances
      are specified in a {\tt .MODEL} statement (see page
      \pageref{.MODELstatement}).
}

\specialnote{Method}{}
The initial Run of a Monte Carlo Analysis uses the nominal values of the
model parameters possibly updated to their new values if the analysis
temperature is different from the the nominal temperature in effect when the
{\tt .MODEL} statement was read.  The nominal values of the model parameters
as well as their tolerances and type of statistical distribution are
specified in a {\tt .MODEL} statement (see page \pageref{.MODELstatement}).
If $X(T)$ is the value of the model parameter at temperature $T$, $r$ is a
random number between $-1$ and $+1$ inclusive, and $X_T$ is the tolerance of
$X(T_{\ms{NOM}})$ is the nominal temperature, the value of X to be used in
Monte Carlo analysis is
\begin{equation}
X'(T) = X(T)(1+rX_T)
\end{equation}

The random number $r$ has a statistical distribution which has a type indicated
in the {\tt .MODEL} statement. The distribution type must be either one of
the built-in distributions or specified by the user in a {\tt .DISTRIBUTION}
statement (see page \pageref{.DISTRIBUTIONstatement}).  The tolerance  is
specified immediately following the distribution in a {.MODEL} statement.
Specification of the distribution type and tolerance assignment are
described on page \pageref{.MODELstatementTOL}.


%
% .MODEL
%
\statement{.MODEL}{Model Statement}

Model statements specify the parameters of elements that either are
too numerous to put on the element line or are common to many elements.\\

\form{ {\tt .MODEL} ModelName ModelType {\tt (}Keyword=Value ... {\tt )}}

\pspiceform{ {\tt .MODEL} ModelName ModelType
{\tt (}Keyword=Value \B ToleranceSpecification\E $\ldots$ {\tt )}}


\begin{widelist}
\item[{\it ModelName}] is the name of the model specified by the user.
\item[{\it ModelType}] is the model type which is specific to particular elements.
                       The model types are given in the table below.
\item[{\it ToleranceSpecification}] is the specification of the statistical
                     distribution and tolerance of a parameter.  This is used
             in a Monte Carlo simulation to assign random variations to
             the model parameter.
                     {\it ToleranceSpecification} is discussed on page
             \pageref{.MODELstatementTOL}.
\item[{\it Keyword}] is the name of the model parameter which is discussed on
                     the pages referred to below.
\item[{\it Value}] is the numeric value of the model parameter.
\end{widelist}

\example{.MODEL MOSFET1 MOS (LEVEL=2 VTO=-0.76 GAMMA=0.6 CGSO=3.35E-10)}

\begin{center}
\begin{tabular}{|l|l|c|l|l|}
\hline
\multicolumn{1}{|c}{MODEL}&
\multicolumn{1}{|c}{SPICE}&
\multicolumn{1}{|c}{ELEMENT}&
\multicolumn{1}{|c}{DESCRIPTION}&
\multicolumn{1}{|c|}{PAGE}\\
\multicolumn{1}{|c}{TYPE}&
\multicolumn{1}{|c}{VERSION|}& NAME&&\\
\hline \hline
\multicolumn{5}{|c|}{PASSIVE DEVICE MODELS}
\\
\hline
RES \modelversion{}     & R & resistor model    &
   {\pageref{RESmodelsp3},}
   \pageref{RESmodelpspice} \\
CAP \modelversion{}     & C & capacitor model   &
   {\pageref{CAPmodelsp3},}
   \pageref{CAPmodelpspice}
   \\
IND \modelversion{\pspice}  & L & inductor model    &
   \pageref{INDmodelpspice} \\
\hline
\hline
\multicolumn{5}{|c|}{SEMICONDUCTOR DEVICE MODELS}
 \\
\hline
D   \modelversion{} & D & diode model       & \pageref{DIODEmodel} \\
NPN \modelversion{} & Q & NPN bjt model & \pageref{NPNmodel} \\
PNP \modelversion{} & Q & PNP bjt model & \pageref{PNPmodel} \\
LPNP    \modelversion{\pspice}&Q& Lateral PNP bjt model& \pageref{PNPmodel} \\
NJF \modelversion{} & J& N-channel junction FET (JFET)&\pageref{NJFmodel} \\
PJF \modelversion{} & J& P-channel junction FET (JFET)& \pageref{PJFmodel}\\
GASFET  \modelversion{\pspice}&B&N-channel GaAs MESFET&
    {\pageref{GASFETmodelsp3},}
    \pageref{GASFETmodelpspice}
    \\
NMF \modelversion{\spicethree}&B&N-channel GaAs MESFET&\\ % \pageref{NMFmodel} \\
PMF \modelversion{\spicethree}&B&P-channel GaAs MESFET&\\ 
% Device Types  \pageref{PMFmodel} \\
NMOS    \modelversion{} & M & N-channel MOSFET  & \pageref{NMOSmodel} \\
PMOS    \modelversion{} & M & P-channel MOSFET  & \pageref{PMOSmodel} \\
SWMODEL \modelversion{\spicethree}&&& \pageref{SWMODELmodel} \\
URC \modelversion{\spicethree}& T   &Lossy RC transmission line&
        \pageref{URCmodel} \\
\hline
\hline
\multicolumn{5}{|c|}{MISCELLANEOUS DEVICE MODELS}
\\
\hline
CORE    \modelversion{\pspice}  &K  & nonlinear, transformer (magnetic core)
    & \pageref{COREmodelpspice} \\
VSWITCH \modelversion{\pspice}  &S  & voltage switch
    & \pageref{VSWITCHmodel} \\
ISWITCH \modelversion{\pspice}  &W  & current switch
    & \pageref{ISWITCHmodel} \\
SW  \modelversion{\spicethree}  &   &
    & \pageref{SWMODELmodel} \\
CSW \modelversion{\spicethree}  &   &
    & \pageref{CSWmodel} \\
\hline
\hline
\multicolumn{5}{|c|}{DISTRIBUTED DEVICE MODELS}
\\
\hline
URC \modelversion{} &   &Uniform Distributed RC model& \pageref{URCmodel} \\
\hline
\end{tabular}
\end{center}

\begin{center}
\begin{tabular}{|l|l|l|l|l|}
\hline
\multicolumn{1}{|c}{MODEL}&
\multicolumn{1}{|c}{SPICE}&
\multicolumn{1}{|c}{ELEMENT}&
\multicolumn{1}{|c}{DESCRIPTION}&
\multicolumn{1}{|c|}{PAGE}\\
\multicolumn{1}{|c}{TYPE}&
\multicolumn{1}{|c}{VERSION}& \multicolumn{1}{|c}{NAME}&&\\
\hline
\hline
\multicolumn{5}{|c|}{DIGITAL INTERFACE AND DIGITAL DEVICE MODELS}
\\
\hline
DINPUT  \modelversion{\pspice} &N& digital input model
    & \pageref{DINPUTmodel} \\
DOUTPUT \modelversion{\pspice} &O& digital output model
    & \pageref{DOUTPUTmodel} \\
UIO    \modelversion{\pspice} &U& digital I/O model
    & \\ %  \pageref{UIOmodel} \\
UGATE   \modelversion{\pspice} &U& standard gate
    & \\ % \pageref{UGATEmodel} \\
UTGATE  \modelversion{\pspice} &U& tri-state gate
    & \\ % \pageref{UTGATEmodel} \\
UEFF    \modelversion{\pspice} &U& edge-triggered flip-flop
    & \\ % \pageref{UEFFmodel} \\
UGFF    \modelversion{\pspice} &U& gated flip-flop
    & \\ % \pageref{UGFFmodel} \\
UWDTH   \modelversion{\pspice} &U& pulse width checker
    & \\ % \pageref{UWDTHmodel} \\
USUHD   \modelversion{\pspice} &U& setup and hold checker
    & \\ % \pageref{USUHDmodel} \\
UDLY    \modelversion{\pspice} &U& digital delay line
    & \\ % \pageref{UDLYmodel} \\
UADC    \modelversion{\pspice} &U& multi-bit analog-to-digital converter
    & \\ % \pageref{UDCmodel} \\
UDAC    \modelversion{\pspice} &U& multi-bit digital-to-analog converter
        &\\ % \pageref{UDACmodel} \\
\hline
\end{tabular}
\end{center}
Virtually all of the device model parameters have default values which
generally result in typical operation.\\[0.1in]

\specialnote{Tolerance and Distribution Assignment}{}
\label{.MODELstatementTOL}
Immediately following the specification of a model parameter
a statistical distribution and tolerance can be assigned.
These are used in conjunction with Monte Carlo analysis which is controlled
by the {\tt .MC} statement described on page
\pageref{.MCstatement}
The Monte Carlo analysis is a statistical analysis of the circuit causing
the circuit to be analyzed many times with a random change of certain model
parameters). The form of the tolerance and distribution assignment is\\[0.1in]
\hspace*{\fill}\offsetparbox{
{\tt DEV}\B{\tt /DeviceTrackingIndex}\E\B{\tt /}{\it
         DistributionType}\E Tolerance\B{\tt \%}\E\\
{\tt +}
{\tt LOT}\B{\tt /LotTrackingIndex}\E\B{\tt /}{\it
         DistributionType}\E Tolerance\B{\tt \%}\E
}\\[0.1in]
\begin{widelist}

\item[{\tt DEV}] is the keyword for the device tolerance specification.

\item[{\it DeviceTrackingIndex}] specifies which random number to use.
           It must be one of 1, 2, ... 10.  Two parameters with the same
           {\it DeviceTrackingIndex} are correlated. Conversely
           if two parameters have different
           {\it DeviceTrackingIndex} then they are uncorrelated.\\
           {\it LotTrackingIndex} and {\it DeviceTrackingIndex} refer to
       different random numbers and so choosing the same
           {\it LotTrackingIndex} and {\it DeviceTrackingIndex} does not
       correlate the {\tt LOT} and {\tt DEV} distributions.\\
           (Units: none; Optional; Default: 1)


\item[{\it DistributionType}] is the type of the statistical distribution
           used in generating random numbers.  There are two predefined
       distribution types and a user specified distribution type can be
       used.  The allowable types are:
       \begin{widelist}
       \item[{\tt UNIFORM}] specifies a uniform distribution of random
       numbers between -1 and +1.
       \item[{\tt GAUSS}] specifies a Gaussian distribution of random
       numbers between $\pm 4 \sigma$.  In this case {\it Tolerance}
       specifies $\sigma$.
       \item[{\it DistributionName}] is the name of a user specified
       distribution specified in a {\tt .DISTRIBUTION} statement (see page
       \pageref{.DISTRIBUTIONstatement}).
       \end{widelist}

\item[{\it LotTolerance}] is the lot tolerance specification.
It may be specified as an absolute quantity or as a percentage if
followed by \%. It is converted to a fraction of the
nominal value.  The magnitude of the fractional value must be $\le$ 1.

\item[{\tt LOT}] is the keyword for the lot tolerance specification.

\item[{\it LotTrackingIndex}] specifies which random number to use.
           It must be one of 1, 2, ... 10.  Two parameters with the same
           {\it LotTrackingIndex} are correlated. Conversely
           two parameters have different
           {\it LotTrackingIndex}'s then they are uncorrelated.\\
           {\it LotTrackingIndex} and {\it DeviceTrackingIndex} refer to
       different random numbers and so choosing the same
           {\it LotTrackingIndex} and {\it DeviceTrackingIndex} does not
       correlate the {\tt LOT} and {\tt DEV} distributions.\\
           (Units: none; Optional; Default: 1)

\item[{\it DeviceTolerance}] is the device tolerance specification.
It may be specified as an absolute quantity or as a percentage if
followed by \%. It is converted to a fraction of the
nominal value.  The magnitude of the fractional value must be $\le$ 1.

\end{widelist}

{
\example{
.MODEL
}

}

\note{
\item Both lot ({\tt LOT}) and device ({\tt DEV}) tolerances can be
specified separately. Lot and device tolerances combine for the purposes
of determining the total tolerance of a model parameter.  The sum of device
and lot tolerances must be less than the nominal model parameter value
(i.e. is less than 100\%).

\item A total of 20 random number are generated internally. Of these 10 are
for lot tolerancing and 10 are for device tolerancing. Parameters with the
same lot tolerance index ({\it LotToleranceIndex}) use the same random number
to generate statistical variations and so are fully correlated. Similarly
parameters with the
same device tolerance index ({\it DeviceToleranceIndex}) use the same random
number. However the random number for a lot and that for a device are always
uncorrelated.   If $X(T)$ is the value of a model parameter at temperature
$T$, $R_{\ms{LOT}}({\it LotToleranceIndex})$ is random number for lot
variations and
$R_{\ms{DEV}}({\it DeviceToleranceIndex})$ is random number for device
variations, $X_{\ms{LOT}}$ is the fractional lot tolerance
$X_{\ms{DEV}}$ is the fractional device tolerance, the value used in a Monte
Carlo simulation is
\begin{equation}
X'(T) = X(T) \left(1 +
R_{\ms{LOT}}({\it LotTrackingIndex}) X_{\ms{LOT}} +
R_{\ms{DEV}}({\it DeviceTrackingIndex}) X_{\ms{DEV}}
\right)
\end{equation}
}


%
% .NODESET
%
\statement{.NODESET}{Node Voltage Initialization}

\form{{\tt .NODESET V(}NodeName{\tt )}=Voltage \B
{\tt V(}NodeName{\tt )}=Voltage $\ldots$ \E}

\begin{widelist}
\item[{\it V}] is the keyword specifying a node voltage
\item[{\it NodeName}]
     is the name of a node.
     {Note that in \spicetwo\ and \spicethree\ {\it NodeName}
     must be an integer.}
\end{widelist}

\example{ .NODESET V(11)=4.9 V(2)=2.5}

\note{
\item
   This statement can be used if convergence problems are encountered
   in a \dc\ analysis.
   For most circuits \spice\ will be able to determine the
   \dc\ voltage or initial transient solution.
\item
   The statement can be used with astable circuits such as multivibrators and
   flip flops to ensure that these circuits are initialized in a particular
   state.
}

%
% .NOISE
%
\statement{.NOISE}{Small-Signal Noise Analysis}

In noise  analysis the noise generated by active devices and resistors
is evaluated.

\form{{\tt .NOISE } OutputVoltageSpecification InputSourceName OutputInterval}

\spicethreeform{{\tt .NOISE}
OutputVoltageSpecification InputSourceName {\tt DEC}\\{\tt +}
FrequenciesPerDecade FStart FStop \B OutputInterval \E \\[0.1in]
{\tt .NOISE} OutputVoltageSpecification InputSourceName {\tt OCT}\\{\tt +}
FrequenciesPerOctave FStart FStop \B OutputInterval \E \\[0.1in]
{\tt .NOISE} OutputVoltageSpecification InputSourceName {\tt LIN}\\{\tt +}
NumberPoints FStart FStop OutputInterval}

\begin{widelist}
\item[{\it OutputVoltageSpecification}]
     is a specification of an output voltage which is to be the output
     of the noise analysis. It acts as a summing point for the
     noise contributions of the individual noise current generators.
     The noise voltage appearing at the output for each noise generator is summed
     in the RMS sense.
     Any voltage specification may be used including
     the voltage at a node, e.g. using V(5), or the voltage between
     nodes, e.g. using V(5,3).
     The noise is reported in units of V/$\sqrt{\mbox{Hz}}$.

\item[{\it InputSourceName}]
     is the name of the independent voltage ({\tt V}) or current ({\tt I})
     source that is to be the input reference source to which equivalent
     input noise is referred. The input source does not produce noise itself.
     If the input source is an independent voltage source then the equivalent
     input noise is reported in units of V/$\sqrt{\mbox{Hz}}$.
     If the input source is an independent current source then the equivalent
     input noise is reported in units of A/$\sqrt{\mbox{Hz}}$.

\item[{\it OutputInterval}]
     is the optional output reporting interval at which the values of the noise
     current generators internal to the elements of the circuit are reported.
     The report is produced every {\it OutputInterval}~th frequency.
     If zero or omitted no detailed output is produced.

\item[{\tt DEC}] is the decade sweep keyword specifying that the
     noise analysis is to be evaluated at a number of frequency points.
     The frequency is swept logarithmically by decades.

\item[{\it FrequenciesPerDecade}]
     specifies the number of frequencies per decade.

\end{widelist}

\begin{widelist}
\item[{\tt OCT}] is the octave sweep keyword specifying that the
     noise analysis is to be evaluated at a number of frequency points.
     The frequency is swept logarithmically by octaves.

\item[{\it FrequenciesPerOctave}]
     specifies the number of frequencies per octave.

\item[{\tt LIN}] is the linear frequency sweep keyword. At each frequency a
     noise analysis is performed.

\item[{\it NumberPoints}]
     specifies the total number of frequencies in a linear sweep.

\item[{\it FStart}]
     is the starting frequency of the sweep.

\item[{\it FStop}]
     is the stopping frequency of the sweep.

\end{widelist}

\example{.NOISE V(5) VIN DEC 10 1kHZ 100Mhz\\
         .NOISE V(5,3) V1 OCT 8 1.0 1.0e6 1}

\example{.NOISE V(5) VIN \\
         .NOISE V(5,3) I1}



\note{
\item
In noise  analysis the noise generated by active devices and resistors
is evaluated as a noise spectral density.  The densities are integrated over the 
the range of frequencies to obtain a gross noise measure for the circuit
(for the specified frequency range).  The finer the frequency spacing the
more accurate will be the noise analysis.  The noise contributions of
individual noise generators are summed at the node or branch specified by
{\it OutputVoltageSpecification}.
The noise at this output port is reported as well as the equivalent input
noise (the output noise referred to the input) at the input source identified
by {\it InputSourceName}.

%The  calculated noise  value
%corresponds to the variance of the circuit variable (voltage or current)
%viewed as a stationary Gaussian process.

\item
Two types of output are produced by noise analysis:
\begin{enumerate}
\item noise spectral  density  versus frequency, and
{
\item total integrated noise  over  the  specified  frequency  range.
}
\end{enumerate}

{\item
In \spicethree\ the \ac\ frequency sweep for noise analysis must be specified in
the {\tt .NOISE} statement. In \pspice\ the frequency sweep specified in the
{\tt .AC statement} is used.}

\item
{
The noise table is produced while analysis is being performed.}
Noise reporting
is produced using the {\tt .PRINT} and {\tt .PLOT} statements.

\item
Noise is generated by resistors and by semiconductor devices. Resistors generate
thermal noise while the noise model of semiconductor devices includes thermal
noise, shot noise and flicker noise.  Noise models of individual elements are
discussed in the element catalog beginning on page \pageref{chapter:element}.
}

\noindent
\underline{\it Two-Port Noise and Gain Calculations}

This is supported in a few versions of Spice.

For all circuits the output {\tt ONOISE} and effective input noise
{\tt INOISE} are calculated. Also the voltage gain {\tt GAIN} is calculated
as the output voltage devided by the voltage across the source.
Extended gain and noise parameters are available if the circuit is defined
as a two port.  The two-port parameters that are calculated are defined in
terms of the signal and noise powers shown in Fig. 
\ref{fig:signal:noise:two:port}
\begin{figure}
\centerline{\epsfxsize=5.5in\pfig{signal_noise_two_port.id}}
\caption{Signal and noise definitions for a two-port.
\label{fig:signal:noise:two:port}}
\end{figure}
The actual noise and signal powers that are delivered to the circuit are
$P_i$ and $N_i$.
Also the actual noise and signal powers that are delivered to the load
resistance are $P_o$ an $N_o$.
The available noise and signal powers are the powers that would be
delivered to the circuit with ideal lossless matching networks.
That is, when the input and output impedances equal the source ($R_S$) and
load ($R_L$) resistances --- so that
$P_i = P_{Ai}$ and $P_o = P_{Ao}$; and
$N_I = N_A$ and $N_O = N_A$.

Two-port noise analysis yields the following quantities which can be
specified in the {\tt .PRINT} and {\tt .PLOT} statements:\\
      \offset\begin{tabular}{lclp{3.5in}}
      {\tt ONOISE} &-&$V_{NO}$&RMS output noise voltage in
                          V/$\sqrt{\mbox{Hz}}$\\
      {\tt INOISE} &-&$V_{NO}$&RMS equivalent input noise voltage in
                          V/$\sqrt{\mbox{Hz}}$\\
      {\tt GAIN} &-&$G$&voltage gain\\
      {\tt GT} &-&$G_T$&transducer gain\\
      {\tt NF} &-&$NF'$&spot noise factor\\
      {\tt SNR} &-&$\mbox{SNR}_i$&output voltage signal-to-noise ratio\\
      {\tt TNOISE} &-&$T_{\mbox{NOISE}}$&output noise temperature in celsius.
      \end{tabular}

All of the quantities can be output in dB with the exception of
$T_{\mbox{NOISE}}$.
When {\tt DB(NF)} is used the spot noise figure is obtained.

The transducer gain, $G_T$, for the two port is
defined as
\begin{equation}
G_T = P_o / P_{Ai}
\end{equation}

The most common measure of the noise performance of a two port is the noise
factor or noise figure.  The spot noise factor {NF'}
of a linear two-port network
is defined as the ratio of the noise power delivered by the network to the 
load impedance to the fraction of the noise power due to the input
termination alone.   This noise is calculated with the input termination,
 $R_S$, at the standard temperature $T_0$ = 290~K.  Note that this differs
 from the default analysis temperature of spice which is 300~K or 16.85~c.
The noise power delivered to the output is the total noise power indicated
by {\tt ONOISE} less the noise power contributed to the output by $R_L$ since it
is not part of the two-port.  These subtractions must be done using squared
voltage quantities because the noises are uncorrelated.  The noise power at the output
due to $R_S$ is the voltage gain squared multiplied by the square of the noise
voltage in series with $R_S$.  The noise factor calculated by \sspice\ is
the spot noise factor as the noise powers are not averaged over
frequency.  The spot noise factor is
\begin{equation}
NF' = { { _0V_{NO}^2 - V_{NO,RL}^2 } \over {_0V_{NO,RS}^2} }
\end{equation}
Here the leading zero subscript indicates that the noise is calculated with
$R_S$ at $T_0$.
$_0V_{NO}$ is the output noise voltage with $R_S$ at $T_0$,
$V_{NO,RL}$ is the component due to the noise generated by $R_L$ and
$_0V_{NO,RS}$ is the component due to the noise generated by $R_S$ at
$T_0$.

The noise temperature in Kelvin is
\begin{equation}
T_{NOISE} = T_K (NF' -1)
\end{equation}
where $T_K$ is the analysis temperature in Kelvin.

The output voltage signal-to-noise ratio is the ratio of the signal voltage to
the noise voltage:
\begin{equation}
\mbox{SNR}_o = {{V''_O }\over{ V''_{NO} \sqrt{2}}}
\end{equation}
where the $\sqrt{2}$ is required since noise voltages are specified in RMS
terms but the signal voltages are specified as a peak voltage.
$V_O$ is the signal voltage at the output taking into account the
signal-to-noise ratio, $\mbox{SNR}_i$, at the input of the circuit.
$\mbox{SNR}_i$ can either be specified on the voltage source line or, if
not, calculated usnig the thermal voltage of $R_S$.


Three parameters affect the results of the noise analysis.  These are the
source resistance $R_S$, the load resistance, $R_L$ and the input signal to
noise ratio, $\mbox{SNR}_i$.  The values used for the parameters are as follows:
\\\offset
\begin{tabular}{rcp{4in}}
$R_S$ &=& The resistance of port 1 ({\tt ZL}, or if Port 1 is not defined,\\
      &=& The {\tt RS} resistance specified on the source line, or if not
          specified.\\
      &=& 50~$\Omega$.\\
\\
$R_L$ &=& The resistance of port 2 ({\tt ZL}, or if Port 2  is not defined.\\
      &=& The {\tt RL} resistance specified on the source line, or if not
          specified,\\
      &=& 50~$\Omega$.\\
      \\
$\mbox{SNR}_i$ &=& The {\tt SNR} specified on the source line, or if not
             specified.\\
      &=& it is calculated as the signal voltage specified on the input line
          devided by the thermal noise voltage of $R_S$ with appropriate
          correction for the difference between RMS and peak quantities.\\
\end{tabular}
\\\offset

\noindent
{\it Example of Two-Port Noise and Gain Analysis}

The netlist for performing a two-port noise and gain analysis of the circuit 
in Fig. \ref{fig:noiser} is as follows.
\begin{figure}
\centerline{\epsfxsize=3in\pfig{noiser_cir.id}}
\caption{Circuit used as an example for specifying noise analysis.
\label{fig:noiser}}
\end{figure}

{\tt
\noindent
Gain and noise analysis of resistive attenuator\\
vin 1 0 AC 1u RS=50 SNR=100\\
RS 1 2 50\\
R1 2 0 55\\
R2 2 3 500\\
R3 3 0 55\\
RL 3 0 50\\
*The following sets the analysis temperature to the standard temperature\\
.TEMP 16.85\\
.AC DEC 1 1MEG 2G\\
.NOISE V(3,0) VIN 1\\
.PRINT NOISE nf db(nf) gt db(gt) gain snr inoise onoise\\
.END}

\medskip

The example below performs the same analysis using ports and also prints
the scattering parameters of the circuit.

{\tt 
\noindent
Gain and noise analysis of resistive attenuator using ports.\\
vin 1 0 AC 1u SNR=100\\
PIN PNR=1 ZL=50\\
R1 2 0 55\\
R2 2 3 500\\
R3 3 0 55\\
POUT PNR=2 ZL=50\\
*The following sets the analysis temperature to the standard temperature\\
.TEMP 16.85\\
.AC DEC 1 1MEG 2G\\
.NOISE V(3) VIN 1\\
.PRINT NOISE nf db(nf) gt db(gt) gain snr inoise onoise\\
.PRINT AS S(1,1) S(1,2), S(2,1), S(2,2)\\
.END}



%
% .OP
%
\statement{.OP}{Operating Point Analysis}

In the operating point analysis a \dc\ analysis is performed
to determine the \dc\ voltages and currents without performing
any sweeps.

\form{{\tt.OP}}

\note{
\item
The operating point analysis is performed by default prior to \ac\ small-signal
({\tt .AC}) and transient ({\tt .TRAN}) analyses
to determine the operating point about which the circuit is linearized
for \ac\ analysis and as the initial starting point for transient analysis.

\item
The ({\tt .OP}) analysis is performed if no other analyses are specified.

\item
The operating point analysis is performed with inductors shorted and capacitors
opened.

\item Following \dc\ analysis the nonlinear devices are linearized to determine
their \ac\ small-signal models.

\item
A {\tt .TRAN} analysis performs its own \dc\
analysis to determine the initial conditions (or bias point) for transient
analysis ignoring the bias point determined in a {\tt .OP} analysis.
}
%
% .OPTIONS
%
\statement{.OPTIONS}{Option Specification}
\index{option, see {\tt .OPTIONS}}

The options specification provides the user control over the program
and a way of setting defaults for certain elements and analyses.

\form{{\tt .OPTIONS} \B Keyword\E ... \B Keyword=Value\E ...}

Multiple keywords can be included in a single {\tt .OPTIONS} statement
and  in any order.\\[0.1in]

\noindent Keywords:
\begin{widelist}

\optionitem{ACCT}{}
Sets reporting of accounting and statistics.
This option is a flag and does not have a value.
\\(Default: not set)

\optionitem{ABSTOL}{=\it Value}
Sets the absolute current error tolerance.
\\(Default: 1~pA ($10^{-12}$))

{
\optionitem{BYPASS}{}
The bypass option.\\
\spicethree\ only.}

\optionitem{CHGTOL}{=\it Value}
Resets  the  charge  tolerance  of  the  program.
\\(Units: C; Default: 10~fC $10^{-14}$)

{
\optionitem{CPTIME}{=\it Value}
Sets the maximum CPU-time.
\\(Units: s; Default: $\infty$)\\\spicetwo\ only.}

\optionitem{DEFAD}{=\it Value}
Sets the default value of the  MOS  drain  diffusion  area ({\tt AD})
used in the M element (see page \pageref{Melement}).
\\(Units: m$^2$ Default: 0)

\optionitem{DEFAS}{=\it Value}
Sets the default value of the MOS  source  diffusion  area ({\tt AS})
used in the M element (see page \pageref{Melement}).
\\(Units: m$^2$ Default: 0)

\end{widelist}

\begin{widelist}
\optionitem{DEFL}{=\it Value}
Sets the value of the MOS channel length ({\tt L})
used in the M element (see page \pageref{Melement}).
\\(Units: m; Default: 100$\mu$m (1E-4))

\optionitem{DEFW}{=\it Value}
Resets the value for MOS channel width ({\tt W})
used in the M element (see page \pageref{Melement}).
\\(Units: m; Default: 100 $\mu$m (1E-4))

\optionitem{EXPAND}{}
Reports in output logfile the devices and nodes created in subcircuit expansions.
This option is a flag and does not have a value.
\\(Default: no expansion)
{\\\pspice\ only.}

\sspicetwoonly{
\optionitem[{\tt FMAX}]  Maximum frequency for simulation
of distributed circuit.
(Units Hz:; Default: 10~GHz ($10^10$))}

\optionitem{GMIN}{=\it Value}
Resets the  value of the minimum conductance $G_{\ms{MIN}}$. The usage of
$G_{\ms{MIN}}$ is controlled by the code implementing individual
elements.   Generally is the minimum conductance between nodes.
It is used to aid convergence.
\\(Units: S; Default: $10^{-12}$ (1E-12))

\optionitem{ITL1}{=\it IntegerValue}
Sets the limit on the number of \dc\ iterations.
\\(Default: 100)

\optionitem{ITL2}{=\it IntegerValue}
Sets the \dc\  transfer curve iteration limit.
\\(Default: 50)

{
\optionitem{ITL3}{=\it IntegerValue}
Sets the minimum number of iterations used in transient analysis.
\\(Default: 4)\\\spicetwo\ only.

\optionitem{ITL4}{=\it IntegerValue}
Sets the maximum of transient iterations at each time point.
\\(Default: 10)\\\spicetwo\ only.
}

\optionitem{ITL5}{=\it IntegerValue}
Sets the transient analysis  total  iteration  limit. If
ITL5=0 this test is omitted. Used in transient analysis ({\tt .TRAN}).
{\\\pspice\ only.}
\\(Default: 5000)

\optionitem{LIBRARY}{}
Reports in output logfile the statements and devices extracted from
a library file
This option is a flag and does not have a value.
\\(Default: no report)
{\\\pspice\ only.}

{
\optionitem{LIMPTS}{=\it Value}
Sets the maximum number of points that can be printed or plotted in
a \dc\ , \ac\ or transient analysis.
\\(Default: 201)\\\spicetwo\ only.

\optionitem{LIMTIM}{=\it Value}
Sets the maximum CPU time for generating plots.  Used only if the program
was terminated because the time specified by the {\tt CPTIME} option
was exceeded.
\\(Units: s; Default: 10)\\\spicetwo\ only.
}

\end{widelist}

\begin{widelist}

\optionitem{LIST}{}
Sets summary reporting of circuit elements in input NETLIST.
\\(Default: not set)

{
\optionitem{LVLCOD}{=\it IntegerValue}
Sets an internal option of the program when running on CDC computers.
If {\tt LVLCOD}=1 machine code for the matrix solution is generated.
\\(Default: 2)\\\spicetwo\ only.

\optionitem{LVLTIM}{=\it IntegerValue}
If {\tt LVLTIM}=1 then iteration time step control is used.
If {\tt LVLTIM}=2 then the time step indicated by the truncation error is
used.
If {\tt METHOD}={\tt GEAR} and MAXORD>2 then LVLTIM is set to 2 by SPICE.
\\(Default: 2)\\\spicetwo\ only.

\optionitem{MAXORD}{=\it IntegerValue}
Sets the maximum order of the integration method if {\tt METHOD}={\tt GEAR}
\\(Default: 2; 2$\le${\tt MAXORD}$\le$6)\\\spicetwo\ only.

\optionitem{METHOD}{=\it String}
Sets the numerical integration method to be used.
If {\tt METHOD}={\tt GEAR} then Gear's method is used.
If {\tt METHOD}={\tt TRAPEZOIDAL} then the trapezoidal method is used.
\\(Default: {\tt TRAPEZOIDAL})\\\spicetwo\ only.
}

\optionitem{NODE}{}
Sets reporting of the node table.
\\(Default: not set)

\optionitem{NOECHO}{}
Suppresses listing of input file in output log file.
\\(Default: input lines listed.)
{\\\pspice\ only.}

\spicetwoonly{\optionitem[{\tt NOFREQ}] Number of frequency points to be used
       in simulation of distributed circuit.  Must be a power of 2.\\
       (Default: 1024)}

\optionitem{NOMOD}{}
Un-sets reporting of model parameters.
\\(Default: set)

\optionitem{NOPAGE}{}
Un-sets page breaks in the output log file.
Useful if the log file is to printed by a
program which automatically paginates the output.
This option is a flag and does not have a value.
\\(Default: set)

\optionitem{NUMDGT}{=\it IntegerValue}
Sets the number of significant digits used in printing values in the output
log file.
\\(Default: 4; $0<{\it IntegerValue}\le 8$)
{\\\spicetwo\ and \pspice\ only.}

\end{widelist}

\begin{widelist}
\optionitem{OPTS}{}
Sets reporting of the option values.
\\(Default: not set)

\optionitem{PIVREL}{=\it Value}
Sets the minimum acceptable pivot value used in partial pivoting
in the solution of the network equations
{(such as solving for the nodal
voltages $\m{v}_n$ in (\ref{nodal:yvj}) on page \pageref{nodal:yvj})}.
{\tt PIVREL} is the minimum acceptable ratio of
an acceptable pivot value to the  largest  column entry.
%In the  numerical  pivoting  algorithm  the
%allowed   minimum   pivot   value   is   determined  by
%{\tt EPSREL} = {\tt AMAX1}({\tt PIVREL} * {\tt MAXVAL}, {\tt PIVTOL}) where
%{\tt MAXVAL} is  the maximum  element  in the column where a pivot is sought.
\\(Default: 0.001)

\optionitem{PIVTOL}{=\it Value}
Sets the minimum value of a matrix element for it to be used as a pivot.
\\(Default: $10^{-13}$)

\optionitem{RELTOL}{=\it Value}
Sets the relative error tolerance of voltages and currents.
\\(Default: 0.001)


\optionitem{TNOM}{=\it Value}
Sets the nominal temperature.
This is assumed to be the temperature at which the model parameters were
measured.  In some cases it is overwritten by a temperature parameter in the
{\tt .MODEL} statement.
\\(Units: $^{\circ}$C; Default: 27$^{\circ}$C (300K))

\optionitem{TRTOL}{=\it Value}
Sets the factor by which the approximated truncation error evaluated
in transient analysis is scaled.
\\(Default: 7)

\optionitem{VNTOL}{=\it Value}
Sets the absolute voltage error tolerance.
\\(Units: V; Default: 1$\mu$V (1E-6))

\end{widelist}

%
% .PARAM
%
\statement{.PARAM}{\pspiceninetytwo\ Only}{Parameter Definition}

This statement defines parameters that can be used in subsequent statements
and element lines.

\pspiceform{ {\tt .PARAM}
\B ParameterName = NumericValue $\ldots$ \E
\B ParameterName = {\tt \{} Expression {\tt \}} $\ldots$ \E}

\begin{widelist}
\item[{\it ParameterName}] Name of a parameter with first character being
     alphabetic (a-z).
     If this is the name of a
     previously defined parameter at the same level of subcircuit expansion
     then the parameter value is changed.
     If the {\tt .PARAM} statement is
     is in the top level circuit then the parameter value is global.
     If the {\tt .PARAM} statement is
     is in a subcircuit then the parameter value is local and can be used at
     the current subcircuit expansion level or lower in the subcircuit
     expansion hierarchy.

\item[{\it NumericValue}] is a numeric value which can be an integer or
     floating point number followed by optional scale factor and/or unit.
     (e.g. 1.E-9, 1N, 1NV and 1.E-9V are equivalent.)

\item[{\it Expression}] is a standard expression as described in the Algebraic
Expressions section on page \pageref{section:algebraic:expression}.
     Note that the expression must be enclosed in braces
     ({\tt \{} $\ldots$ {\tt \}}).

\end{widelist}

{\example{
.PARAM VDD = 10V, VSS = 0 VREF = 2.5\\
.PARAM VREF = { VDD/2}
.PARAM LENGTH = {1.10*L}
}}

\note{

\item
Predefined parameters are supported and the user must avoid using these.
Predefined parameters:\\
{\offset\begin{tabular}{|l|l|l|}
\hline
{\bf Name} & {\bf Value}   &  {\bf Description}  \\
\hline
\hline
TEMP       & not supported & Analysis temperature.\\
           & Reserved for future expansion\\\\
VT         & not supported & Thermal voltage.\\
           & Reserved for future expansion\\
\hline
\end{tabular}
}

\item In \pspice\ , in most places where a numeric value is required
an expression (within braces {\tt \{ $\ldots$ \})} can be used instead.
An expression can contain any support mathematical operation, constant numeric
values or expressions. Exceptions are
     \begin{itemize}
     \item Polynomial coefficients.
     \item The values of the transmission line device parameters {\tt NL} and
           {\tt F}.
     \item The values of the piece-wise linear characteristic in the {\tt PWL}
           form of the independent voltage ({\tt V}) and current ({\tt I})
           sources.
     \item[and]
     \item The values of the resistor device parameter {\tt TC}.
     \item As node numbers.
     \item Values of most statements (such as .TEMP, .AC, .TRAN etc.)
     \end{itemize}
Specifically included are
     \begin{itemize}
     \item The values of all other device parameters.
     \item The values in {\tt .IC} and {\tt .NODESET} statements.
     \item The values in {\tt .SUBCKT} statements.
     \item[and]
     \item The values of all model parameters.  {\tt F}.
     \end{itemize}

\item Hierarchical usage of {\tt .PARAM} statements in subcircuits is
supported.  The parameters defined in a {\tt .PARAM} statement are
available in the subcircuit in which they are defined or in lower nested
subcircuits. Thus parameters defined in a subcircuit are not available
higher in the hierarchy.  The same concept applies to values of a parameter
changed in a subcircuit.  Value changes are local and are available in the
current subcircuit and lower nested subcircuits.

{\item Libraries are searched for parameters not defined in the
      circuit NETLIST or in included files. A {\tt .PARAM} statement does
      not have to be within a subcircuit in a library.}

}

%
% .PLOT
%
\statement{.PLOT}{Plot Specification}

The plot specification controls the information that is plotted as the result
of various analyses.

{
\form{{\tt .PLOT TRAN } OutputSpecification \B PlotLimits\E
     \\{\tt +} \B OutputSpecification \B PlotLimits\E $\ldots$ \E\\[0.1in]
      {\tt .PLOT AC   } OutputSpecification \B PlotLimits\E
	   \\{\tt +} \B OutputSpecification \B PlotLimits\E $\ldots$ \E\\[0.1in]
      {\tt .PLOT DC   } OutputSpecification  \B PlotLimits\E
	   \\{\tt +} \B OutputSpecification \B PlotLimits\E $\ldots$ \E\\[0.1in]
      {\tt .PLOT NOISE} NoiseOutputSpecification 
		    \B {\tt (}DistortionReportType{\tt )}\E \B PlotLimits\E
       \\{\tt +} \B NoiseOutputSpecification
	    \B {\tt (}DistortionReportType{\tt )}\E \B PlotLimits\E\\[0.1in]
      {\tt .PLOT DISTO} DistortionOutputSpecification
		   \B {\tt (}DistortionReportType{\tt )}\E \B PlotLimits\E
		   \\{\tt +} \B DistortionOutputSpecification
		   \B {\tt (}DistortionReportType{\tt )}\E \B PlotLimits\E
		      $\ldots$ \E
	    \B {\tt (}DistortionReportType{\tt )}\E \B PlotLimits\E\\[0.1in]
     }}

\pspiceform{{\tt .PLOT TRAN } OutputSpecification \B PlotLimits\E
     \\{\tt +} \B OutputSpecification \B PlotLimits\E $\ldots$ \E\\[0.1in]
      {\tt .PLOT AC   } OutputSpecification \B PlotLimits\E
	   \\{\tt +} \B OutputSpecification \B PlotLimits\E $\ldots$ \E\\[0.1in]
      {\tt .PLOT DC   } OutputSpecification  \B PlotLimits\E
	   \\{\tt +} \B OutputSpecification \B PlotLimits\E $\ldots$ \E\\[0.1in]
      {\tt .PLOT NOISE} NoiseOutputSpecification 
		    \B {\tt (}DistortionReportType{\tt )}\E \B PlotLimits\E
       \\{\tt +} \B NoiseOutputSpecification
	    \B {\tt (}DistortionReportType{\tt )}\E \B PlotLimits\E\\[0.1in]
     }


\begin{widelist}

\item[{\tt TRAN}] is the keyword specifying that this {\tt .PLOT} statement
     controls the reporting of results of a transient analysis initiated by the
     {\tt .TRAN} statement.

\item[{\tt AC}] is the keyword specifying that this {\tt .PLOT} statement
     controls the reporting of results of a small-signal \ac\ analysis initiated by
     the {\tt .AC} statement.

\item[{\tt DC}] is the keyword specifying that this {\tt .PLOT} statement
     controls the reporting of results of a \dc\ analysis initiated by the
     {\tt .DC} statement.

\item[{\tt NOISE}] is the keyword specifying that this {\tt .PLOT} statement
     controls the reporting of results of a noise analysis initiated by the
     {\tt .NOISE} statement.

{
\item[{\tt DISTO}] is the keyword specifying that this {\tt .PLOT} statement
     controls the reporting of results of a small-signal \ac\ distortion analysis
     initiated by the {\tt .DISTO} statement.
     }

\end{widelist}

\begin{widelist}
\item[{\it OutputSpecification}] specifies the voltage or current to be plotted
     against the sweep variable. The sweep variable is dependent on the type
     of analysis.

     Many forms of {\tt OutputSpecification} are supported by
     \pspice\ .
     {Below is a description of the basic forms that are
     supported both by \spicetwo\ and \pspice\ .
     }
     A comprehensive
     description of {\tt OutputSpecification} supported by
     \pspice\ is given in the section on output specification
     on page \pageref{.PRINToutputspecification}.
     %Up to 8 {\it OutputSpecification}s 
     \\[0.1in]

     {
     \underline{Voltages} may be specified as an absolute voltage at a node:
     {\tt V({\it NodeName})} or
     the voltage at one node with respect to that at another node,
     e.g. {\tt V(Node1Name,Node2Name)}.

     For the reporting of the results of an \ac\ analysis the following outputs
     can be specified by replacing the {\tt V} as follows:\\
      \offset\begin{tabular}{lcp{3.5in}}
      {\tt VR} &-&real part\\
      {\tt VI} &-&imaginary part\\
      {\tt VM} &-&magnitude\\
      {\tt VP} &-&phase\\
      {\tt VDB} &-&$10\,\mbox{log}(10\,magnitude)$
      \end{tabular}\\
      In \ac\  analysis the default is {\tt VM} for magnitude.\\[0.1in]

     \underline{Currents} are specified by referencing the name of the voltage
     source through which the current is measured, e.g.
     {\tt I(V{\it VoltageSourceName})}.

     For the reporting of the results of an \ac\ analysis the following outputs
     can be specified by replacing the {\tt I} as follows:\\
      \offset\begin{tabular}{lcp{3.5in}}
      {\tt IR} &-&real part\\
      {\tt II} &-&imaginary part\\
      {\tt IM} &-&magnitude\\
      {\tt IP} &-&phase\\
      {\tt IDB} &-&$10\,\mbox{log}(10\,magnitude)$
      \end{tabular}
      In \ac\  analysis the default is {\tt IM} for magnitude.
      }

\item[{\it PlotLimits}] are optional and can be placed after any output
     specification. {\it PlotLimits} has the form
     {\tt (}{\it LowerLimit}{\tt ,}{\it UpperLimit}{\tt )} .
     All quantities will be plotted using the same
     {\it PlotLimits}.  The default is to automatically scale the plot and
     perhaps use different scales for each of the quantities to be plotted.
      
\end{widelist}

\begin{widelist}
\item[{\it NoiseOutputSpecification}]
     specifies the noise measure to be reported. The two options are
     {\tt ONOISE} which reports the output noise and
     {\tt INOISE} which reports the equivalent input noise.
     See the {\tt .NOISE} statement on page \pageref{.NOISEstatement}
     for a detailed explanation.

     It  must be one of the following:\\
      \offset\begin{tabular}{lcp{3.5in}}
      {\tt ONOISE} &-&magnitude of the output noise\\
      {\tt DB(ONOISE)} &-&output noise in dB\\
      {\tt INOISE} &-&magnitude of the equivalent input noise\\
      {\tt DB(INOISE)} &-&equivalent input noise in dB\\
      {\tt GAIN} &-&voltage gain\\
      {\tt DB(GAIN)} &-&voltage gain in dB (= 20 log({\tt GAIN})\\
      {\tt GT} &-&transducer gain\\
      {\tt DB(GT)} &-&transducer gain in dB (= 10 log({\tt GT})\\
      {\tt NF} &-&spot noise factor\\
      {\tt DB(NF)} &-& spot noise figure (= 10 log({\tt NF})\\
      {\tt SNR} &-&output signal-to-noise ratio\\
      {\tt DB(SNR)} &-&output signal-to-noise ratio in dB (= 20 log({\tt SNR})\\
      {\tt TNOISE} &-&output noise temperature.
      \end{tabular}

\item[{\it SParameterOutputSpecification}]
      specifies the S-parameter output variables that are to be printed.
      Each variable must have one of the following forms:\\
      \offset\begin{tabular}{lcp{3.5in}}
      {\tt S(i,j)} &-&Magnitude of $S_{ij}$\\
      {\tt SR(i,j)} &-&Real part of $S_{ij}$\\
      {\tt SI(i,j)} &-&Imaginary part of $S_{ij}$\\
      {\tt SP(i,j)} &-&Phase of $S_{ij}$ in degrees\\
      {\tt SDB(i,j)} &-&Magnitude of $S_{ij}$ in dB
                         (= 20 log({\tt {\tt S(i,j)}}))\\ 
      {\tt SG(i,j)} &-&Group delay of $S_{ij}$
      \end{tabular}
      The port numbers are $i,j$  which are specified using the {\tt PNR} keywor
      when the port (`P') element is specified.


{
\item[{\it DistortionOutputSpecification}]
     specifies the distortion component to be reported
     in a tabular format of up to 8 columns plus an initial column with the sweep
     variable.  The {\it DistortionOutputSpecification} is one of the
     following:\\
      \offset\begin{tabular}{lcp{3.5in}}
      {\tt HD2} &-&the second harmonic distortion\\
      {\tt HD3} &-&the second harmonic distortion\\
      {\tt SIM2} &-&the sum frequency intermodulation component\\
      {\tt DIM2} &-&the difference frequency intermodulation component\\
      {\tt DIM3} &-&the third order intermodulation component
      \end{tabular}
      See the {\tt .DISTO} statement on page \pageref{.DISTOstatement} for
      a description of these distortion components.

\item[{\it DistortionReportType }]
     specifies the format for reporting the distortion components.
     It  must be one of the
     following:\\
      \offset\begin{tabular}{lcp{3.5in}}
      R &-&real part\\
      I &-&imaginary part\\
      M &-&magnitude\\
      P &-&phase\\
      DB &-&$10\,\mbox{log}(10\,magnitude)$
      \end{tabular}
      The default is {\tt M} for magnitude.
}
\end{widelist}

\example{
.PLOT TRAN V(10) V(5,3) I(VIN)\\
.PLOT AC VM(10) VR(5,3) IP(VLOAD)\\
.PLOT DC V(10) V(5,3) I(VIN)\\
.PLOT NOISE ONOISE INOISE DB(ONOISE) DB(INOISE)\\
.PLOT NOISE GAIN DB(GT) DB(NF) SNR TNOISE\\
.PLOT AS SDB(1,1) SP(1,1) SDB(1,2) SP(1,2)\\
{.PLOT DISTO HD2 HD3 SIM2(DB)}
}


\note{
\item There can be any number of {\tt .PLOT} statements.

\item All of the output quantities specified on a single {\tt .PLOT} statement
will be plotted on the same graph using ASCII characters.
An overlap will be indicated by the letter
{\tt X}.  The plot produced by the {\tt .PLOT} statement is a line printer plot.
While plotting is primitive it can be plotted on any printer and is
incorporated in the output log file.

\item The plot output of the results of an \ac\ analysis always have a
logarithmic vertical scale.

\item The current through any element can be found by inserting independent
voltage sources in series with the elements.
This is generally what is required in \spicetwo and \spicethree.
However \pspice\ supports direct specification of the voltage and currents
of most elements. See the section on page
\pageref{.PRINToutputspecification}.

\item
More elaborate plotting is available
{with \spicethree\ using the \nutmeg\ plotting program described on
the \nutmeg\ chapter beginning on page \pageref{ch:nutmeg}; and}
with \pspice\ using the {\tt .PROBE} statement described on page \pageref{.PROBEstatement}.

}


%
% .PRINT
%
\statement{.PRINT}{Print Specification}

The print specification controls the information that is reported as the result
of various analyses.

\form{{\tt .PRINT TRAN } OutputSpecification \B OutputSpecification $\ldots$
      \E\\[0.1in]
      {\tt .PRINT AC   } OutputSpecification \B OutputSpecification $\ldots$
      \E\\[0.1in]
      {\tt .PRINT DC   } OutputSpecification \B OutputSpecification $\ldots$
      \E\\[0.1in]
    {
      {\tt .PRINT DISTO} DistortionOutputSpecification
              {\tt ( }DistortionReportType{ \tt )}\\{\tt +}
              \B DistortionOutputSpecification
              {\tt ( }DistortionReportType{ \tt )} $\ldots$ \E}
     }

\begin{widelist}

\item[{\tt TRAN}] is the keyword specifying that this {\tt .PRINT} statement
     controls the reporting of results of a transient analysis initiated by the
     {\tt .TRAN} statement.

\item[{\tt AC}] is the keyword specifying that this {\tt .PRINT} statement
     controls the reporting of results of a small-signal \ac\ analysis initiated by
     the {\tt .AC} statement.

\item[{\tt DC}] is the keyword specifying that this {\tt .PRINT} statement
     controls the reporting of results of a \dc\ analysis initiated by the
     {\tt .DC} statement.

\item[{\tt NOISE}] is the keyword specifying that this {\tt .PRINT} statement
     controls the reporting of results of a noise analysis initiated by the
     {\tt .NOISE} statement.

{
\item[{\tt DISTO}] is the keyword specifying that this {\tt .PRINT} statement
     controls the reporting of results of a small-signal \ac\ distortion analysis
     initiated by the {\tt .DISTO} statement.
     }

\end{widelist}

\begin{widelist}
\item[{\it OutputSpecification}] specifies the voltage or current to be reported
     in a tabular format of up to 8 columns plus an initial column with the
     sweep variable.

     Many forms of {\tt OutputSpecification} are supported by
     \pspice\ .
     {Below is a description of the basic forms that are
     supported both by \spicetwo\ and \pspice\ .  }
     A comprehensive
     description of {\tt OutputSpecification} supported by
     \pspice\ is given in the section on output specification
     on page \pageref{.PRINToutputspecification}.
     %Up to 8 {\it OutputSpecification}s
     \\[0.1in]

{
     \underline{Voltages} may be specified as an absolute voltage at a node:
     {\tt V({\it NodeName})} or
     the voltage at one node with respect to that at another node,
     e.g. {\tt V(Node1Name,Node2Name)}.

     For the reporting of the results of an \ac\ analysis the following outputs
     can be specified by replacing the {\tt V} as follows:\\
      \offset\begin{tabular}{lcp{3.5in}}
      {\tt VR} &-&real part\\
      {\tt VI} &-&imaginary part\\
      {\tt VM} &-&magnitude\\
      {\tt VP} &-&phase\\
      {\tt VDB} &-&$10\,\mbox{log}(10\,magnitude)$
      \end{tabular}\\
      In \ac\  analysis the default is {\tt VM} for magnitude.\\[0.1in]

     \underline{Currents} are specified by referencing the name of the voltage
     source through which the current is measured, e.g.
     {\tt I(V{\it VoltageSourceName})}.

     For the reporting of the results of an \ac\ analysis the following outputs
     can be specified by replacing the {\tt I} as follows:\\
      \offset\begin{tabular}{lcp{3.5in}}
      {\tt IR} &-&real part\\
      {\tt II} &-&imaginary part\\
      {\tt IM} &-&magnitude\\
      {\tt IP} &-&phase\\
      {\tt IDB} &-&$10\,\mbox{log}(10\,magnitude)$
      \end{tabular}
      In \ac\  analysis the default is {\tt IM} for magnitude.
      }

\item[{\it NoiseOutputSpecification}]
     specifies the noise measure to be reported. The two options are
     {\tt ONOISE} which reports the output noise and
     {\tt INOISE} which reports the equivalent input noise.
     See the {\tt .NOISE} statement on page \pageref{.NOISEstatement}
     for a detailed explanation.

     It  must be one of the following:\\
      \offset\begin{tabular}{lcp{3.5in}}
      {\tt ONOISE} &-&RMS output noise voltage\\
      {\tt DB(ONOISE)} &-&output noise voltage in dB (= 20 log({\tt ONOISE})\\
      {\tt INOISE} &-&RMS equivalent input noise voltage\\
      {\tt DB(INOISE)} &-&equivalent input noise voltage in dB (= 20 log({\tt INOISE})\\
      {\tt GAIN} &-&voltage gain\\
      {\tt DB(GAIN)} &-&voltage gain in dB (= 20 log({\tt GAIN})\\
      {\tt GT} &-&transducer gain\\
      {\tt DB(GT)} &-&transducer gain in dB (= 10 log({\tt GT})\\
      {\tt NF} &-&spot noise factor\\
      {\tt DB(NF)} &-& spot noise figure (= 10 log({\tt NF})\\
      {\tt SNR} &-&output signal-to-noise ratio\\
      {\tt DB(SNR)} &-&output signal-to-noise ratio in dB (= 20 log({\tt SNR})\\
      {\tt TNOISE} &-&output noise temperature.
      \end{tabular}

\item[{\it SParameterOutputSpecification}]
      specifies the S-parameter output variables that are to be printed.
      Each variable must have one of the following forms:\\
      \offset\begin{tabular}{lcp{3.5in}}
      {\tt S(i,j)} &-&Magnitude of $S_{ij}$\\
      {\tt SR(i,j)} &-&Real part of $S_{ij}$\\
      {\tt SI(i,j)} &-&Imaginary part of $S_{ij}$\\
      {\tt SP(i,j)} &-&Phase of $S_{ij}$ in degrees\\
      {\tt SDB(i,j)} &-&Magnitude of $S_{ij}$ in dB
                         (= 20 log({\tt {\tt S(i,j)}}))\\
      {\tt SG(i,j)} &-&Group delay of $S_{ij}$
      \end{tabular}\\[0.1in]
      The port numbers are $i,j$  which are specified using the {\tt PNR} keyword
      when the port element is specified.

\end{widelist}

{
\begin{widelist}
\item[{\it DistortionOutputSpecification}]
     specifies the distortion component to be reported
     in a tabular format of up to 8 columns plus an initial column with the sweep
     variable.  The {\it DistortionOutputSpecification} is one of the
     following:\\
      \offset\begin{tabular}{lcp{3.5in}}
      {\tt HD2} &-&the second harmonic distortion\\
      {\tt HD3} &-&the second harmonic distortion\\
      {\tt SIM2} &-&the sum frequency intermodulation component\\
      {\tt DIM2} &-&the difference frequency intermodulation component\\
      {\tt DIM3} &-&the third order intermodulation component
      \end{tabular}\\[0.1in]
      See the {\tt .DISTO} statement on page \pageref{.DISTOstatement} for
      a description of these distortion components.

\item[{\it DistortionReportType }]
     specifies the format for reporting the distortion components.
     It  must be one of the
     following:\\
      \offset\begin{tabular}{lcp{3.5in}}
      R &-&real part\\
      I &-&imaginary part\\
      M &-&magnitude\\
      P &-&phase\\
      DB &-&$10\,\mbox{log}(10\,magnitude)$
      \end{tabular}\\[0.1in]
      The default is {\tt M} for magnitude.
\end{widelist}
      }

\example{
.PRINT TRAN V(10) V(5,3) I(VIN)\\
.PRINT AC VM(10) VR(5,3) IP(VLOAD)\\
.PRINT DC V(10) V(5,3) I(VIN)\\
.PRINT NOISE ONOISE INOISE DB(ONOISE) DB(INOISE)\\
.PRINT NOISE GAIN DB(GT) DB(NF) SNR TNOISE\\
.PRINT AS SDB(1,1) SP(1,1) SDB(1,2) SP(1,2)\\
{.PRINT DISTO HD2 HD3 SIM2(DB)}
}

\note{
\item There can be any number of {\tt .PRINT} statements.

\item The number of significant digits of the results reported is
{\tt NUMDGT} which is set in a {\tt .OPTIONS} statement (see page
\pageref{.OPTIONNUMDGT}).

\item The current through any element can be found by inserting independent
voltage sources in series with the elements.
This is generally what is required in \spicetwo and \spicethree.
However \pspice\ supports direct specification of the voltage and currents
of most elements. See the section on page
\pageref{.PRINToutputspecification}.
}


\specialnote{Output Specification{ for \pspice}}{}
\vspace{0.1in}
\label{.PRINToutputspecification}
\pspice\ supports a relatively large variety of output specifications compared
to that available with \spicetwo\ and \spicethree . The output
specifications described in the following can be used {\tt .PRINT}
and {\tt .PLOT} statements.  The various forms of output specifications
enable the current and voltages of virtually all devices to be
examined>\\[0.1in]
\underline{{\tt DC} and {\tt TRAN} Reporting}\\
The output specifications available for the \dc\ sweep and transient
analyses are

\begin{widelist}

\item[{\tt I(}{\it DeviceName}{\tt )}]
Current through a two terminal device (such as a resistor {\tt R} element) or
the output of a controlled voltage or current source. e.g.
{\tt I(R22)} is the current flowing through resistor {\tt R22}
from node $N_1$ to $N_2$ of {\tt R22}.

\item[{\tt I}{$\,$TerminalName}{\tt (}{\it DeviceName}{\tt )}]
Current flowing into terminal named {\it TerminalName} (such as {\tt B} for
gate) from the device named {\it DeviceName} (such as {\tt Q12}).
e.g. {\tt IB(Q12)}

\item[{\tt I}{$\,$PortName}{\tt (}{\it TransmissionLineName}{\tt )}]
Current at port named {\it PortName} (either {\tt A} or {\tt B}) of the
transmission line device named   {\it TransmissionLineName}

\item[{\tt V(}{\it NodeName}{\tt )}]
Voltage at a node of name {\it NodeName}.

\item[{\tt V(}{$n_1,n_2$}{\tt )}]
Voltage at node $n_1$ with respect to the voltage
 at node $n_2$.

\item[{\tt V(}{\it DeviceName}{\tt )}]
Voltage across a two terminal device (such
as a resistor {\tt R} element) or
at the output of a controlled voltage or current source.

\item[{\tt V}{$\,$TerminalName}{\tt (}{\it DeviceName}{\tt )}]
Voltage at terminal named {\it TerminalName} (such as {\tt G} for gate) of the
device named   {\it DeviceName} (such as {\tt M12}). e.g. {\tt VG(M12)}

\item[{\tt V}{$\,$TerminalName1$\,$TerminalName2}{\tt (}{\it DeviceName}{\tt )}]
Voltage at terminal named {\it TerminalName1} (such as {\tt G} for gate) th respect to the terminal name {\it TerminalName2} (such as {\tt S} for source) of the
device named   {\it DeviceName} (such as {\tt M12}). e.g. {\tt VGS(M12)}

\item[{\tt V}{$\,$PortName}{\tt (}{\it TransmissionLineName}{\tt )}]
Voltage at port named {\it PortName} (either {\tt A} or {\tt B}) of the
transmission line device named   {\it TransmissionLineName} (such as
{\tt T5}). e.g. {\tt VA(M5)}

\end{widelist}

\hspace*{\fill}\\[0.1in]

\noindent\underline{Two Terminal Device Types Supported for \dc and
Transient Analysis Reporting}\\[0.1in]
The single character identifier  for the following elements
as well as the rest of the device name can be used as the {\it DeviceName}
in the {\tt I(}{\it DeviceName}{\tt )}  and {\tt I(}{\it DeviceName}{\tt )}
output specifications.\\
\hspace*{\fill}
\begin{tabular}{|p{1in}|p{3in}|}
\hline
Element Type & Description\\
\hline
{\tt C} & capacitor\\
{\tt D} & diode\\
{\tt E} & voltage-controlled voltage source\\
{\tt F} & current-controlled current source\\
{\tt G} & voltage-controlled current source\\
{\tt H} & current-controlled voltage source\\
{\tt I} & independent current source\\
{\tt L} & inductor\\
{\tt R} & resistor\\
{\tt V} & independent voltage source\\
\hline
\end{tabular}\\[0.1in]

\noindent
\underline{Multi-Terminal Device Types Supported for \dc and Transient Analysis
Reporting}\\[0.1in]
The single character identifier for the following elements as well as the rest
of the device name can be used as the {\it DeviceName}
in the
{\tt I}{$\,$TerminalName}{\tt (}{\it DeviceName}{\tt )},
{\tt V}{$\,$TerminalName}{\tt (}{\it DeviceName}{\tt )} and\newline
{\tt V}{$\,$TerminalName1$\,$TerminalName2}{\tt (}{\it DeviceName}{\tt )}
output specifications.\\[0.1in]
\hspace*{\fill}
\begin{tabular}{|p{1in}|p{3in}|}
\hline
Element Type & Description\\
\hline
{\tt B} & GaAs MESFET\ \ \ Terminals:\newline
\hspace*{1in}D --- drain\newline
\hspace*{1in}G --- gate\newline
\hspace*{1in}S --- source\\
{\tt J} & JFET\ \ \ Terminals:\newline
\hspace*{1in}D --- drain\newline
\hspace*{1in}G --- gate\newline
\hspace*{1in}S --- source\\
{\tt M} & MOSFET\ \ \ Terminals:\newline
\hspace*{1in}B --- bulk or substrate\newline
\hspace*{1in}D --- drain\newline
\hspace*{1in}G --- gate\newline
\hspace*{1in}S --- source\\
{\tt Q} & BJT\ \ \ Terminals\newline
\hspace*{1in}C --- collector\newline
\hspace*{1in}B --- base\newline
\hspace*{1in}E --- emitter\newline
\hspace*{1in}S --- source\\
\hline
\end{tabular}\\[0.1in]

\noindent
\underline{{\tt AC} Reporting}\\
The output specifications available for reporting the results of an
\ac\ frequency sweep analysis includes all of the specification formats discussed
above for \dc\ and transient analysis together with a number of possible
suffixes:\\
      \offset\begin{tabular}{lcp{3.5in}}
      {\tt DB} &-&$10\,\mbox{log}(10\,magnitude)$\\
      {\tt M} &-&magnitude\\
      {\tt P} &-&phase\\
      {\tt R} &-&real part\\
      {\tt I} &-&imaginary part\\
      {\tt G} &-&group delay = ${\partial \phi/ \partial f}$\newline
                 where $\phi$ is the phase of the quantity being reported
         and $f$ is the analysis frequency.
      \end{tabular}\\
In \ac\  analysis the default suffix is {\tt M} for magnitude.\\[0.1in]

\noindent\underline{Two-Terminal Device Types Supported for
\ac\ Reporting}\\[0.1in]

The single character identifier  for the following elements
as well as the rest of the device name can be used as the {\it DeviceName}
in the {\tt I(}{\it DeviceName}{\tt )}  and {\tt I(}{\it DeviceName}{\tt )}
output specifications.\\
\hspace*{\fill}
\begin{tabular}{|p{1in}|p{3in}|}
\hline
Element Type & Description\\
\hline
{\tt C} & capacitor\\
{\tt D} & diode\\
{\tt I} & independent current source\\
{\tt L} & inductor\\
{\tt R} & resistor\\
{\tt V} & independent voltage source\\
\hline
\end{tabular}\\[0.1in]

\noindent
\underline{Multi-Terminal Device Types Supported for \dc and Transient Analysis
Reporting}\\[0.1in]
The single character identifier for the following elements as well as the rest
of the device name can be used as the {\it DeviceName}
in the
{\tt I}{$\,$TerminalName}{\tt (}{\it DeviceName}{\tt )},
{\tt V}{$\,$TerminalName}{\tt (}{\it DeviceName}{\tt )} and\newline
{\tt V}{$\,$TerminalName1$\,$TerminalName2}{\tt (}{\it DeviceName}{\tt )}
output specifications.\\[0.1in]
\hspace*{\fill}
\begin{tabular}{|p{1in}|p{3in}|}
\hline
Element Type & Description\\
\hline
{\tt B} & GaAs MESFET\ \ \ Terminals:\newline
\hspace*{1in}D --- drain\newline
\hspace*{1in}G --- gate\newline
\hspace*{1in}S --- source\\
\hline
{\tt J} & JFET\ \ \ Terminals:\newline
\hspace*{1in}D --- drain\newline
\hspace*{1in}G --- gate\newline
\hspace*{1in}S --- source\\
\hline
{\tt M} & MOSFET\ \ \ Terminals:\newline
\hspace*{1in}B --- bulk or substrate\newline
\hspace*{1in}D --- drain\newline
\hspace*{1in}G --- gate\newline
\hspace*{1in}S --- source\\
\hline
{\tt Q} & BJT\ \ \ Terminals\newline
\hspace*{1in}C --- collector\newline
\hspace*{1in}B --- base\newline
\hspace*{1in}E --- emitter\newline
\hspace*{1in}S --- source\\
\hline
\end{tabular}\\[0.1in]


%
% .PROBE
%
\statement{.PROBE}{Data Output Specification}

{\large\bf There is a big problem here --- Probe is a trademark of Microsim
corporation}

The {\tt .PROBE} statement saves the node voltages and device currents in a
file for subsequent probing.

\pspiceform{
{\tt .PROBE}\B {\tt /CSDF}\E \B OutputSpecification $\ldots$ \E
}

\begin{widelist}

\item[{\tt /CSDF}] is a keyword resulting in the output probing file being
     written in ASCII format.  By default the probing file is output in
     the more efficient binary format. However, only the ASCII formated is
     fully portable between computers and operating systems.

\item[{\it OutputSpecification}] specifies a node voltage or device current
     to be included in the probe data file.
     The output specifications supported are those supported for the
     {\tt .PRINT} and {\tt .PLOT} statements.  A comprehensive
     description of the {\tt OutputSpecification} supported is
     given in the section on output specification
     on page \pageref{.PRINToutputspecification}.\\
     If an {\it OutputSpecification} is not given then all node voltages
     and device currents are stored in the probing file.
\end{widelist}

\example{
.PROBE V(10) V(5,3) I(VIN)\\
.PROBE VM(10) VR(5,3) IP(VLOAD)\\
.PROBE/CSDF V(10) IG(VIN)\\
.PROBE\\[0.1in]
}
     The first example will output data from {\tt .DC}, {\tt .AC} and
     {\tt .TRAN} analyses.\\
     The second example will output data only from a {\tt .AC} analysis.\\
     The third example will output the node voltage at node 10 for
     {\tt .DC}, {\tt .AC} and {\tt .TRAN} analyses but output the group
     delay of the current in the independent voltage source {\tt VIN}
     only from an {\tt .AC} analysis. As well the data will be output in ASCII
     format.\\
     The fourth example results in all node voltages and device currents
     being stored.

\note{

\item
The probing data is stored in the file {\tt PROBE.DAT} .

\item
The results of \dc , \ac\ and transient analyses are saved.
An {\it OutputSpecification} which is unique to a particular type of
analysis is ignored when the the results of analyses are being selected to
storage.
An example is {\tt VI(10)} which selects the imaginary part of the voltage at
node 10 and so only applies for storage of the results for an \ac\
small-signal analysis.

\item
The results stored in the probe file can subsequently be viewed using
{the \probe program described in chapter \ref{chapter:probe}.}
}


%
% .PZ
%
{
\statement{.PZ}{Pole-Zero Analysis}

In pole zero analysis the poles and zeros of the small signal \ac
transfer function of a two-port is evaluated.

%\begin{figure}[h]
%\centering
%\ pfig{pz.ps}
%\caption{Two port definition for pole-zero analysis. \label{fig:pz}}
%\end{figure}

{
\form{{\tt .PZ} Node1 Node2 Node3 Node4 {\tt CUR POL}\\
      {\tt .PZ} Node1 Node2 Node3 Node4 {\tt CUR ZER}\\
      {\tt .PZ} Node1 Node2 Node3 Node4 {\tt CUR PZ}\\
      {\tt .PZ} Node1 Node2 Node3 Node4 {\tt VOL POL}\\
      {\tt .PZ} Node1 Node2 Node3 Node4 {\tt VOL ZER}\\
      {\tt .PZ} Node1 Node2 Node3 Node4 {\tt VOL PZ}
}
}


\begin{widelist}
\item[{\it Node1}] is the positive input node.

\item[{\it Node2}] is the negative input node.

\item[{\it Node3}] is the positive output node.

\item[{\it Node4}] is the negative output node.

\item[{\tt CUR}] is the keyword to evaluate the transfer function (output
voltage)/(input  current)

\item[{\tt VOL}] is the keyword to evaluate the transfer function (output
voltage)/(input  voltage)

\item[{\tt POL}] is the keyword to evaluate the poles of the transfer
function only.

\item[{\tt ZERO}] is the keyword to evaluate the zeroes of the transfer
function only.

\item[{\tt PZ}] is the keyword to evaluate the poles and zeroes of the transfer
function.
\end{widelist}

\example{.PZ 1 0 3 0 CUR POL\\
         .PZ 2 3 5 0 VOL ZER\\
         .PZ 4 1 4 1 CUR PZ}

\note{
\item
The pole-zero analysis works  with  resistors {\tt R}, capacitors {\tt C},
inductors {\tt L},
linear controlled  sources {\tt E},{\tt F},{\tt G} and {\tt H};  independent
voltage and current sources {\tt V} and {\tt I}, bipolar junction
transistors, {\tt Q};  MOSFETs, {\tt M};  JFETs {\tt J};
and  diodes {\tt D}.  In particular distributed devices such as Transmission
lines are not supported as these do not have a pole-zero description.

\item
In interactive mode, the command  syntax  is  the  same
except  that the first field is PZ instead of .PZ.  To print
the results, one should use the command ``{\tt PRINT ALL}''.

\item
The program first computes the \dc\ operating point  and  then
determines  the  linearized, small-signal models for all the
nonlinear devices in the circuit. This circuit is then  used
to find the poles and zeros.
}

}
%
% .SAVEBIAS
%
{\statement{.SAVEBIAS}{Save Bias
Conditions}

}

%
% .SENS
%
\statement{.SENS}{Sensitivity Analysis}

The sensitivity analysis calculates the \dc\ small-signal sensitivities of each
output quantity with respect to every circuit parameter.

\form{{\tt .SENS} OutputSpecification \B OutputSpecification $\ldots$ \E }

\begin{widelist}

\item[{\it OutputSpecification}]  is the specification of the small-signal output
     quantity.  It has the same format as the
     {\it OutputSpecification} in a {\tt .PRINT} statement (see page
     \pageref{.PRINTstatement}).

\end{widelist}

\example{.SENS V(10) V(10,2) I(VLOAD)}

\note{
\item
The {\tt .SENS} statement initiates a small-signal \ac\ analysis.

\item
The transfer function computed is the sensitivity (or partial derivative) of the
\dc\ value of the output quantity
with respect to the each and every circuit parameter.  For example, if a resistor
is specified with value $x$ and a capacitor is specified with value $c$ then, for
the above examples the following sensitivities
are computed:\\[0.1in]
${{\textstyle\partial\tt V(10)}\over{\textstyle\partial x}}$\hfill
${{\textstyle\partial\tt V(10,2)}\over{\textstyle\partial x}}$\hfill
${{\textstyle\partial\tt VLOAD}\over{\textstyle\partial x}}$\hfill
${{\textstyle\partial\tt V(10)}\over{\textstyle\partial c}}$\hfill
${{\textstyle\partial\tt V(10,2)}\over{\textstyle\partial c}}$\hfill
${{\textstyle\partial\tt VLOAD}\over{\textstyle\partial c}}$
}

%
% .STEP
%
{\statement{.STEP}{Parameteric Analysis}}

%
% .SUBCKT
%
\statement{.SUBCKT}{Subcircuit Statement}

\begin{figure}[h]
\centering
\ \epsfxsize=2in\pfig{x_spice.ps}
\caption{Subcircuit.}
\end{figure}

{
\form{ {\tt .SUBCKT} SubcircuitName $N_1$ \B $N_2$ $N_3$ ... $N_N$\E}
}

\pspiceform{ {\tt .SUBCKT} SubcircuitName $N_1$ \B $N_2$ $N_3$ ... $N_N$\E
\\{\tt +} \B {\tt PARAMS:} \B Keyword = {\tt \{} Expression {\tt \}} $\ldots$ \E
\B Keyword = Value $\ldots$ \E \E}

\pspiceninetytwoform{ {\tt .SUBCKT} SubcircuitName $N_1$ \B $N_2$ $N_3$ ... $N_N$\E
\\{\tt +} \B PARAMS: \B keyword = {\tt \{} Expression {\tt \}} $\ldots$ \E
\B Keyword = Value $\ldots$ \E \E}

\begin{widelist}
\item[{\it SubcircuitName}] is the name of the subcircuit.
\item[$N_1$] is the first node of the subcircuit.
\item[$N_N$] is the $N$th node of the subcircuit.
\item[{\tt PARAMS:}] indicates that parameters are to be passed to
the subcircuit.
\item[{\it Keyword:}] is a keyword which may be replaced by a value
specified on a subcircuit call ({\tt X}) element.
(See page \pageref{Xelement}).
\item[{\it Value:}] is a numeric value.
\item[{\it Expression:}] is an algebraic expression which evaluates to a numeric
value.
(See section \ref{section:algebraic:expression}
on page \pageref{section:algebraic:expression} for allowable expressions).
\end{widelist}
\example{.SUBCKT MULTI 2 4 17 3 1}

\note{

\item The global ground node, node 0
  {(or in \pspice\ node {\tt GND})}
must not be one of the subcircuit nodes.
\item
Subcircuits are incorporated by using the ``{\tt X}'' element.
The number of nodes of the ``{\tt X}'' element must correspond to
the number of nodes in the definition of the subcircuit (i.e. is
in the {\tt .SUBCKT} statement). See page \pageref{Xelement} for
a description of the {\tt X} element.


\item
The last line in a subcircuit definition is the
.ENDS line (see page \pageref{.ENDSstatement}).

\item
The only restriction on the statements within a subcircuit
is that control lines such as {\tt .AC}, {\tt .DC} or {\tt .OPTIONS}
are not allowed. However, element lines, model statements and
other subcircuit definitions and subcircuit calls are allowed.

\item
Device  models  or  subcircuit  definitions
included as part of a  subcircuit  definition  are
local only.  Subcircuits and models that are to be ``known''
by all elements and subcircuits must be defined at the top level
of the circuit hierarchy.

\item
The nodes of elements in the subcircuit definition are local except for
those that also appear on the {\tt .SUBCKT} statement and for the ground
({\tt 0}) node.
The local nodes are given the unique name\\
\hspace*{\fill}
{\it SubcircuitName1}:[{\it SubcircuitName2}: $\ldots$ ]{\it LocalNodeName} .
\hspace*{\fill}\\
and local devices are given the unique name\\
\hspace*{\fill}
{\it DeviceType}:{\it SubcircuitName1}:[{\it SubcircuitName2}
$\ldots$ ]{\it DeviceName} .
\hspace*{\fill}
}


%
% .TEMP
%
\statement{.TEMP}{Temperature Specification}

The {\tt .TEMP} statement specifies the temperatures at which the circuit
is to be simulated.

\form{{\tt .TEMP} $T_1$ \B $T_2$ $\ldots$ $T_N$ \E}

\begin{widelist}
\item[{$T_1$}] is the first temperature at which the circuit is to be
simulated.
\item[{$T_N$}] is the $N$th temperature at which the circuit is to be
simulated.
\end{widelist}

\note{

\item
The circuit is first simulated at temperature $T_1$ and then resimulated at
temperature $T_2$ and so on.
\item If the {\tt .TEMP} statement is missing then $T_1$ is assumed to be
$T_{\ms{NOM}}$ specified in a {\tt .OPTIONS} statement.

\item
Model parameters are specified at $T_{\ms{NOM}}$ and prior to the simulation
at a new temperature temperature dependent device parameters are
reevaluated.
}

%
% .TEXT
%
{
\statement{.TEXT}{Text Parameter Definition}

} % end

%
% .TF
%
\statement{.TF}{Transfer Function Specification}

The transfer function specifies a small-signal \dc\ analysis from which
a small-signal transfer function and input and output resistances are computed..

\form{{\tt .TF} OutputSpecification InputSourceName}

\begin{widelist}

\item[{\it OutputSpecification}]  is the specification of the small-signal output
     quantity.  It has the same format as the
     {\it OutputSpecification} in a {\tt .PRINT} statement (see page
     \pageref{.PRINTstatement}).

\item[{\it InputSourceName}]  specifies the the name of the small-signal input
     independent voltage ({\tt V}) or current ({\tt I}) source.
\end{widelist}

\example{.TF V(10) VINPUT\\
     .TF V(10,2) ISOURCE\\
     .TF I(VLOAD) ISOURCE}

\note{
\item
The {\tt .TF} statement initiates a small-signal \dc\ analysis from which
a small-signal transfer function and input and output resistances are computed.

\item
The transfer function computed is the ratio of the \dc\ value of the output quantity
to the input quantity.  In the above examples the following transfer functions are
computed:\\[0.1in]
\hspace*{\fill}
\begin{tabular}{|l|l|}
\hline
EXAMPLE & Transfer Function\\
\hline
&\\
.TF V(10) VINPUT   &${{\textstyle\tt V(10)}\over{\textstyle\tt VINPUT}}$\\&\\
.TF V(10,2) ISOURCE&${{\textstyle\tt V(10,2)}\over{\textstyle\tt ISOURCE}}$\\&\\
.TF I(VLOAD) ISOURCE&${{\textstyle\tt I(VLOAD)}\over{\textstyle\tt ISOURCE}}$\\&\\
\hline
\end{tabular} \hspace*{\fill}
}
%
% TITLE
%
\statement{TITLE}{Title Line}

The {\tt TITLE} line must be the first line of the input file. The string
on this line included as the banner in the output log file appearing at
the top of each page.


%
% .TRAN
%

\statement{.TRAN}{Transient Analysis}

In transient analysis the current and voltages in a circuit are computed
as a function of time.

General form:

\form{{\tt .TRAN} TSTEP TSTOP \B TSTART \B TMAX\E\E \B {\tt UIC}\E }

\begin{widelist}
\item[{\it TSTEP}] is the time increment for reporting transient
simulation results.
\\(Units: s)

\item[{\it TSTOP}] is the final analysis time.
\\(Units: s)

\item[{\it TSTART}] is the start time for reporting the
        transient results. Transient analysis always begins at time 0.
        Before the time {\tt TSTART} no results are recorded.
        \\(Units: s; Optional; Default: 0)

\item[{\it TMAX}]
is the maximum  step size used in incrementing the time during transient
analysis.
        \\(Units: s; Optional; Default: the smaller of {\tt TSTEP}
    and ({\tt TSTOP}-{\tt TSTART})/50)

\item[{\tt UIC}] is the optional keyword to use initial conditions specified
on the element line, by a {\tt .IC} statement or a {\tt .NODESET} statement.
Normally the operating point is determined (using a \dc\ analysis) before a
transient analysis is initiated. If the {\tt UIC} keyword is present
the initial \dc\ analysis is omitted and instead the initial conditions
specified by the {\tt IC} parameter supported by certain elements,
by the {\tt .IC} statement,
or by the
{\tt .NODESET} statement are used.
{
In addition \pspice\ supports using an operating point solution
that was previously saved using the {\tt .SAVEBIAS} statement.}
\end{widelist}


\example{.TRAN 1NS 100NS\\
         .TRAN 1NS 1000NS 500NS\\
         .TRAN 10NS 1US UIC}

\note{

\item  If the {\tt UIC} keyword is \underline{not present}
      a \dc\ analysis is automatically
      performed prior  to  a  transient ({\tt .TRAN})  analysis to find the
      operating point of the circuit.
      All sources which are not time dependent (for example,  power  supplies)
      are  set  to their \dc\ value.  The transient time interval is
      specified on a .TRAN control line.  The operating point solution
      is used as the initial conditions for a transient analysis.
      Individual operating point solutions are overridden by initial transient
      conditions specified on the {\tt .TRAN} statement or by the initial
      conditions specified for specific elements by the {\tt IC} keyword.

\item
Normally the operating point is determined (using a \dc\ analysis) before a
transient analysis is initiated. If the {\tt UIC} keyword is present
the initial \dc\ analysis is omitted and instead the initial conditions
specified by the {\tt IC} parameter supported by certain elements,
by the {\tt .IC} statement (see \pageref{.ICstatement}),
or by the
{\tt .NODESET} statement (see \pageref{.NODESETstatement}) are used.
{
In addition \pspice\ supports using an operating point solution
that was previously saved using the {\tt .SAVEBIAS} statement described
on page \pageref{.SAVEBIASstatement}.}
\item  If the {\tt UIC} keyword is \underline{present} the initial conditions
            specified in
        the {\tt .IC} statement (described on page \pageref{.ICstatement})
        are used
        to establish the initial conditions.  Initial conditions specified
        for individual elements using the {\tt IC} parameter on the element
        line will always have precedence over those specified in a {\tt .IC}
        statement.\\
        No \dc\ analysis is performed prior to a transient analysis. Thus it
        is important to establish the initial conditions at all nodes using
        the {\tt .IC} statement or using the {\tt IC} element parameter.

}

%
% .WATCH
%
{
\statement{.WATCH}{Watch Analysis Statement}
}

%
% .WCASE
%
{
\statement{.WCASE}{Sensitivity and Worst Case Analysis}

The sensitivity and worst case analysis is a statistical analysis of
a circuit causing
the circuit to analyzed many times with a random change of model parameters
(parameters in a {\tt .MODEL} statement).

\form{{\tt .WCASE} NumberOfRuns AnalysisType OutputSpecification OutputFunction
                \B {\tt LIST}\E\\
                {\tt +} \B {\tt OUTPUT(}OutputSampleType {\tt )}\E
                \B {\tt RANGE(}LowValue{\tt ,} HighValue{\tt )}\E
     }

\begin{widelist}

\item[{\it NumberOfRuns}] is the total number of runs to do. This number
includes the initial nominal run.

\item[{\it AnalysisType}] is the type of analysis to be performed in Monte
     Carlo runs after the initial nominal run.  All analyses
     specified in the {\tt NETLIST} are performed in the nominal run.
     The {\it AnalysisType} must be one of the following:
     \begin{widelist}
     \item to be filled in
     \end{widelist}

\item[{\tt DC}] is a keyword indicating that the \dc\ analysis as
     specified by the {\tt .DC} statement is repeated.
     The sweep variable used in analyzing the output
     {\it OutputSpecification} is the value of the independent voltage
     or current source specified in the {\tt .DC} statement which is discussed
     on page \pageref{.DCstatement}.

\item[{\tt AC}] is a keyword indicating that the \ac\ small-signal analysis as
     specified by the {\tt .AC} statement is repeated.
     The sweep variable used in analyzing the output
     {\it OutputSpecification} is frequency.

\item[{\tt TRAN}] is a keyword indicating that the transient analysis as
     specified by the {\tt .TRAN} statement is repeated.
     The sweep variable used in analyzing the output
     {\it OutputSpecification} is time.


\item[{\it OutputSpecification}]  specifies the quantity to be reported
     as the result of the Monte Carlo Analysis.  It has the same format as the
     {\it OutputSpecification} in a {\tt .PRINT} statement (see page
     \pageref{.PRINTstatement}). The result is the value of the
     {\it OutputSpecification} with respect to a sweep for {\tt DC} and {\tt AC}
     analysis, and as a waveform for {\tt TRAN analysis}.

\item[{\it OutputFunction}] indicates the function to be performed on the
     output indicated by {\it OutputSpecification} to reduce the sweep
     or waveform at each run to a single numeric value. The
     {\it OutputFunction} must be one of the following keywords:
     \item[{\tt YMAX}] which produces the greatest deviation of the sweep or
                       waveform from the nominal run.


     \item[{\tt MAX}] which results in the maximum value in each sweep or
                      waveform.

     \item[{\tt MIN}] which results in the minimum value in each sweep or
                      waveform.

     \item[{\tt RISE\_EDGE({\it Value})}] which reports as the result the
                      first run when the waveform crosses above the threshold
              {\it Value}.
              The algorithm used requires that
              one point in the waveform be below {\it Value}
              and the succeeding point be above {\it Value}.
\end{widelist}
\clearpage

\begin{widelist}
     \item[{\tt FALL\_EDGE({\it Value})}] which reports as the result the
                      first run when the waveform crosses below the threshold
              {\it Value}.
              The algorithm used requires that
              one point in the waveform be above {\it Value}
              and the succeeding point be below {\it Value}.


\item[{\tt RANGE}] is an optional range indicating the range of the sweep
                  variable over which {\it OutputFunction} is to be performed.
          If this keyword is missing output is produced
          the range is not restricted.  The range of the sweep
          variable to be considered is from {\it LowValue}
          to {\it HighValue} inclusive.

\item[{\it LowValue}] is the low end of the sweep variable to be considered
                  in evaluating {\it OutputFunction}.


\item[{\it HighValue}] is the low end of the sweep variable to be considered
                  in evaluating {\it OutputFunction}.

\end{widelist}

\begin{widelist}
\item[{\tt LIST}] is an optional keyword that results in the model parameter
                  values that are statistically varied
                  being printed out prior to each run. If it is omitted
          then the the statistically generated model parameter
          values are not produced prior to each run.


\item[{\tt OUTPUT}] is an optional keyword indicating the type of output to
                  be produced by runs after the initial nominal run. The
          output produced for each run sampled is determined by the
          {\tt .PLOT}, {\tt .PRINT} and {\tt .PROBE} statements in
          the NETLIST. If this keyword is missing output is produced
          only for the nominal run.



\item[{\it SEED}] is the keyword for the seed of the random number generator
                  used in Monte Carlo Analysis

\item[{\it SeedValue}] is the value of the seed used in the random number
generator used to select sample runs at random.
    (Optional; Default: 17,533; $1 \le$ {\it SeedValue}$ \le 32,767$)

\end{widelist}

\note{
\item If the {\it AnalysisType} is {\tt DC} only one independent voltage or
      current source can be specified in the {\tt .DC} statement (discussed on
      page \pageref{.DCstatement}).

\item The random number generator is the subtractive method generator described
      by Knuth \cite[p. 171]{knuth:subtractive}.

\item The initial run uses the nominal parameter values given in the NETLIST.
      Subsequent runs statistically vary model parameters indicated as having
      either lot {\tt LOT} or device {\tt DEV} tolerances. These tolerances
      are specified in a {\tt .MODEL} statement (see page
      \pageref{.MODELstatement}).
}

} % end

%
% .WIDTH
%
\statement{.WIDTH}{Width Specification}

\form{{\tt .WIDTH OUT=}{\it ColumnWidth}}

\begin{widelist}
\item[{\tt OUT}] is the keyword for column width of the output file.
\item[{\it ColumnWidth}] is the column width of the output file.
                 It must be either 80 or 132.  If there is no {\tt .WIDTH}
                 statement the ColumnWidth defaults to 80.
\end{widelist}

\example{
.WIDTH OUT=80\\
.WIDTH OUT=132
}
