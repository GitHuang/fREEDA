\element{W}{Current Controlled Switch}
\begin{figure}[h]
\centering
\ \pfig{w_spice.ps}
\caption{W --- current controlled switch.}
\end{figure}

\form{{\tt W}name $N_1$ $N_2$ VoltageSourceName ModelName \B ON\E \B OFF\E  }

\pspiceform{{\tt W}name $N_1$ $N_2$ VoltageSourceName ModelName}

\vspace{-0.2in}
\begin{widelist}
\item[$N_{+}$] is the positive node of the switch.
\item[$N_{-}$] is the negative node of the switch.
\item[{\it VoltageSourceName}] is the name of the voltage source the current
through which is the controlling current.
The voltage source must be a {\tt V} element.
\item[{\tt ON}] is the optional initial condition.
It is intended for use with the {\tt UIC} option
on  the  {\tt .TRAN}  line,  when  a transient analysis is desired
starting from other than the quiescent operating point.
It is also the initial condition on the device for \dc\ analysis.
\item[{\tt OFF}] is the optional initial condition.
If specified the \dc\ operating point is calculated with the
terminal voltages set to zero.  Once convergence is obtained, the
program continues to iterate to obtain the exact  value of the
terminal  voltages.  The OFF option is used to enforce the
solution to  correspond  to  a  desired  state if the circuit has
more than one stable state.
\end{widelist}
\modeltype{ISWITCH}
\model{ISWITCH}{Current-Controlled Switch Model}
\begin{figure}[h]
\centering
\ \pfig{iswitch.ps}
\caption{ISWITCH --- current controlled switch model. \label{iswitch}}
\end{figure}

The current-controlled switch model is supported by
both \spicethree\ and \pspice. However the model keywords differ slightly.\\[0.1in]

\kw{\spicethree\ keywords:}{
{\tt IT}   & threshold current\sym{I_{\ms{ON}}}& A     & 0.0    \X
{\tt IH}   & hysteresis current\sym{I_{\ms{OFF}}} & A     & 0.0    \X
{\tt RON}  & on resistance      \sym{R_{\ms{ON}}}&$\Omega$&1.0    \X
{\tt ROFF} & off resistance     \sym{R_{\ms{OFF}}}&$\Omega$&1/GMIN\X
}

\kw{\pspice\ keywords:}{
{\tt ION}   & threshold current\sym{I_{\ms{ON}}}& A     & 0.0    \X
{\tt IOFF}   & hysteresis current\sym{I_{\ms{OFF}}} & A     & 0.0    \X
{\tt RON}  & on resistance      \sym{R_{\ms{ON}}}&$\Omega$&1.0    \X
{\tt ROFF} & off resistance     \sym{R_{\ms{OFF}}}&$\Omega$&1/GMIN\X
}

Care must be exercised in using the switch.  An instantaneous switch
is highly nonlinear and will very likely lead to convergence problems.
This problem is alleviated in the {\tt ISWITCH} model by ramping the resistance
of the switch from its off value to its on value.  For this ramping action to be
effective the difference between $I_{\ms{ON}}$ and $I_{\ms{OFF}}$
must not be too small. Also the values of $R_{\ms{ON}}$ and $R_{\ms{OFF}}$
should not be extreme.
The ration $R_{\ms{ON}}/R_{\ms{OFF}}$ should be be as small as possible.

If $R_{\ms{ON}}/R_{\ms{OFF}}$  is large, e.g.
$R_{\ms{ON}}/R_{\ms{OFF}}$ $>$ $10^{12}$, then the default error tolerances
{\tt TRTOL} and {\tt CHGTOL}, specified in a {\tt .OPTIONS} statement
(see page \pageref{.OPTIONSstatement}) may need to be changed.
\begin{widelist}
\item[{\tt TRTOL}] Change to 1.0 from 7.0 idf there are convergence problems
during transient analysis.
\item[{\tt CHGTOL}] If a switch is across a capacitor then {\tt CHGTOL} should be
reduced to $10^{-16}$ if there are convergence problems
during transient analysis.
\end{widelist}
\noindent\underline{\bf \large Switch Model}\\[0.1in]

The switch is modeled by a current variable resistor $R$, see figure \ref{iswitch}.

\noindent\underline{\sl \large Standard Calculations}\\[0.1in]
\begin{eqnarray}
R_{\ms{MEAN}} & = & \sqrt{R_{\ms{ON}} + R_{\ms{OFF}}} \\
R_{\ms{RATIO}}& = &       R_{\ms{ON}}/ R_{\ms{OFF}} \\
I_{\ms{MEAN}} & = & \sqrt{I_{\ms{ON}} + I_{\ms{OFF}}} \\
I_{\Delta}& = & \left({{\textstyle i - I_{\ms{MEAN}}} \over
              {\textstyle I_{\ms{ON}}- I_{\ms{OFF}}}}\right)
\end{eqnarray}
If $I_{\ms{ON}}> I_{\ms{OFF}}$
the switch resistance
\begin{equation}
R = \left\{
\begin{array}{ll}
R_{\ms{ON}}                            & i \ge I_{\ms{ON}} \\
R_{\ms{OFF}}                            & i \le I_{\ms{OFF}}\\
R_{\ms{MEAN}}\,
  R_{\ms{RATIO}}^{\textstyle 1.5 I_{\Delta}}\,
  R_{\ms{RATIO}}^{\textstyle 1.5 I_{\Delta}^3}
  & I_{\ms{OFF}} < i < I_{\ms{ON}} \\
\end{array} \right. %}
\end{equation}
If $I_{\ms{ON}}< I_{\ms{OFF}}$
the switch resistance
\begin{equation}
R = \left\{
\begin{array}{ll}
R_{\ms{ON}}                            & i \le I_{\ms{ON}} \\
R_{\ms{OFF}}                            & i \ge I_{\ms{OFF}}\\
R_{\ms{MEAN}}\,
  R_{\ms{RATIO}}^{\textstyle 1.5 I_{\Delta}}\,
  R_{\ms{RATIO}}^{\textstyle 1.5 I_{\Delta}^3}
  & I_{\ms{OFF}} < i < I_{\ms{ON}} \\
\end{array} \right. %}
\end{equation}
\noindent\underline{\sl \large Noise Analysis}\\[0.1in]
\index{ISWITCH, Noise Model}
\index{ISWITCH, Noise Analysis}
The current controlled switch noise model accounts for thermal noise generated
in the switch resistance.
The rms (root-mean-square) values of
thermal noise current generators shunting the switch resistance is
\begin{equation}
I_{n} = \sqrt{4kT/R}~\mbox{A/}\sqrt{\mbox{Hz}}
\end{equation}
where $T$ is the analysis temperature in kelvin (K), and $k$ (=
$1.3806226\,10^{-23}$~J/K) is Boltzmanns constant.
