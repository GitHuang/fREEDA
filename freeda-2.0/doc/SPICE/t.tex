\element{T}{Transmission Line}
\begin{figure}[h]
\centering
\ \pfig{t_spice.eps}
\caption{T --- transmission line element.}
\end{figure}

\form{{\tt T}name $n_1$ $n_2$ $n_3$ $n_4$ {\tt Z0}=CharacteristicImpedance
    {\tt TD}=TimeDelay
    \B {\tt IC}=$V_1,I_1,V_2,I_2$\E  \\
    \\
    {\tt T}name $n_1$ $n_2$ $n_3$ $n_4$ {\tt Z0}=CharacteristicImpedance
       {\tt F}=ReferenceFrequency \newline
    {\tt +} \B {\tt NL}=NormalizedElectricalLength\E
    \B {\tt IC}=$V_1,I_1,V_2,I_2$\E}

\spicethreeform{{\tt T}name $n_1$ $n_2$ $n_3$ $n_4$
    \B ModelName \E
    {\tt Z0}=CharacteristicImpedance
    {\tt TD}=TimeDelay
    \B {\tt IC}=$V_1,I_1,V_2,I_2$\E  \\
    \\
    {\tt T}name $n_1$ $n_2$ $n_3$ $n_4$ \B ModelName \E
       {\tt Z0}=CharacteristicImpedance\newline
    {\tt +}   {\tt F}=ReferenceFrequency
    {\tt +} \B {\tt NL}=NormalizedElectricalLength\E
    \B {\tt IC}=$V_1,I_1,V_2,I_2$\E}

\pspiceform{{\tt T}name $n_1$ $n_2$ $n_3$ $n_4$ {\tt Z0}=CharacteristicImpedance {\tt TD}=TimeDelay \\
    \\
    {\tt T}name $n_1$ $n_2$ $n_3$ $n_4$ {\tt Z0}=CharacteristicImpedance \newline
    {\tt +}{\tt F}=Frequency \B {\tt NL}=NormalizedElectricalLength\E}


\begin{widelist}
\item[{\it $n_1$}] positive node at port 1.
\item[{\it $n_2$}] negative node at port 1.
\item[{\it $n_3$}] positive node at port 2.
\item[{\it $n_4$}] negative node at port 2.
\item[{\tt ModelName}]  is  the  model name.
\item[{\tt Z0}] is the characteristic impedance. (Z-zero)\\
               (Units: $\Omega$; Required; Symbol: $Z_0$; Default: none)
\item[{\tt TD}] transmission line delay.
               (Units: s; Either {\tt TD} or {\tt F} Required;
           Symbol: $T_D$; Default: none)
\item[{\tt F}]  reference frequency.
               (Units: Hz; Either {\tt TD} or {\tt F} Required;
           Symbol: $F$; Default: none)
\end{widelist}
\clearpage

\begin{widelist}
\item[{\tt NL}] normalized electrical length. Normalization is with respect to
                the wavelength in free space at the
                reference frequency {\tt F}.\\
               (Units: none; Optional; Symbol: $L_{\ms{NORMALIZED}}$; Default:
           0.25)

\item[{\tt IC}] is the optional  initial condition specification
using {\tt IC=} $V_1,I_1,V_2,I_2$ is intended for use with the {\tt UIC} option
on  the  {\tt .TRAN}  line,  when  a transient analysis is desired
starting from other than the quiescent operating point.
Specification of the transient initial conditions using the {\tt .IC}
statement (see page \pageref{.ICstatement}) is preferred and is more
convenient.
\end{widelist}
\example{T1 1 0 2 0 Z0=50 TD=10NS\\ \\
         TLONG 1 0 2 0 Z0=50 F=1G NL=10\\ \\
         TLONG 1 0 2 0 Z0=50 F=1G
    }

\note{
\item
The length of the line may be expressed in either of two  forms.
The  transmission  delay,  {\it TD}, may be specified directly (as
TD=10ns, for example).  Alternatively, a frequency F may  be
given, together with NL, the normalized electrical length of
the transmission line with respect to the wavelength in  the
line at the frequency F.  If a frequency is specified but NL
is omitted, 0.25 is  assumed  (that  is,  the  frequency  is
assumed  to  be  the  quarter-wave  frequency).   Note  that
although both forms for expressing the line length are indicated
 as optional, one of the two must be specified.

\item
Note that only 3 distinct nodes should be specified as this element
describes a single  propagating
mode.  With four distinct nodes specified, two propagating modes may exist
on the actual line.  If there are four distinct nodes then two lines are
required.
\item
The transmission line {\tt T} element is modeled as a bidirectional ideal
delay element.
% as shown in figure \ref{t:model:fig}.
The maximum time step in
\spice\ is limited to half of the time delay along the line.  Thus
short transmission lines can result in many time steps in a transient
analysis.  Unnecessary short transmission lines should be avoided.
}

%\begin{figure}[hbp]
%\centering
%tmodel.ps
%%\ \pfig{tmodel.ps}
%\caption[Ideal bidirectional delay element model of transmission lines]{Ideal
%bidirectional delay element model of the transmission line {\tt T} element.
%\label{t:model:fig}}
%\end{figure}

\modeltype{URC}

\model{URC}{Lossy RC Transmission Line Model}
\begin{figure}[h]
\centering
\ \pfig{uurc.eps}
\caption[URC --- lossy RC transmission line model]{URC --- lossy RC transmission
line model: (a) linear RC transmission
line model; (b) nonlinear transmission line model. \label{urc:fig}}
\end{figure}
\clearpage
\form{ {\tt .MODEL} ModelName {\tt URC(} \B  \B keyword = value\E  ... \E
{\tt )}}

\example{ .MODEL LONGLINE URC( ) }


\begin{table}[h]
\caption{URC model parameters. \label{urctable}}
\keywordtable{
{\tt K} & Propagation Constant               & -     & 2.0\X
{\tt FMAX} & Maximum Frequency of interest      & Hz    & 1.0G\X
{\tt RPERL}&Resistance per unit length\sym{I_{S,\ms{PERL}}}& $\Omega$/m & 1000\X
{\tt CPERL}&Capacitance per unit length\sym{I_{S,\ms{PERL}}}& F/m   & 1.0E-15\X
{\tt ISPERL}&Saturation current per unit length\sym{I_{S,\ms{PERL}}}&A/m&\omit\X
{\tt RSPERL}&Diode Resistance per unit length\sym{I_{S,\ms{PERL}}}
       &$\Omega$/m&0\X
   }
\end{table}

The URC model was originally proposed by Gertzberrg \cite{gertzberg:74}
%L.  Gertzberrg in 1974.
In this model a transmission line is represented by the cascade of a number of
transmission line segments each of which is modeled by an
RC or R-Diode subcircuit.  The lengths of the line segments
increases in a geometric progression towards the middle of the line.
The number of line segments is
\begin{equation}
N =
\end{equation}

and the length of the $i$th line segment is
\begin{equation}
l_i =
\end{equation}
If {\tt ISPERL} is not specified
then a linear transmission line is modeled, see figure \ref{urc:fig}, with
\begin{eqnarray}
R_i &=& R_{\ms{PERL}} l_i\\
C_i &=& C_{\ms{PERL}} l_i
\end{eqnarray}
If {\tt ISPERL} is not
then a diode loaded nonlinear transmission line is modeled,
see figure \ref{urc:fig}, with
\begin{eqnarray}
R_i &=& R_{\ms{PERL}} l_i\\
R_{S,i} &=& R_{S,\ms{PERL}} l_i\\
C_i &=& C_{J,i} \left(1 -
      {{\textstyle \phi}\over{\textstyle V_{J,i}}}\right)^{\textstyle
      -\frac{ 1}{2}}\\
I_S &=& I_{S,i}
        \left( e^{\textstyle{{\textstyle V_{J,i}}\over{\textstyle V_{\ms{TH}}}}
         -1} \right)
\end{eqnarray}
where the zero-bias capacitance of the $i$th diode is
\begin{eqnarray}
C_{J,i} &=& C_{\ms{PERL}} l_i
\end{eqnarray}
its reverse saturation current is
\begin{eqnarray}
I_{S,i} &=& I_{S,\ms{PERL}}
\end{eqnarray}
