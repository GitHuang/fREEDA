\chapter{Hints}
\section{Introduction}
\section{Convergence}

\subsection{Transient Analysis}

Both dc and transient  solutions  are  obtained  by  an
iterative  process which is terminated when both of the following conditions
 hold:\\
\begin{enumerate}
\item The nonlinear branch  currents  converge  to  within  a
tolerance  of  0.1  percent or 1 picoamp (1.0E-12 Amp),
whichever is larger.

\item The node voltages converge to within a tolerance of 0.1
percent  or  1  microvolt  (1.0E-6  Volt), whichever is
larger.
\end{enumerate}

Although the algorithm used in SPICE has been found  to
be  very reliable, in some cases it will fail to converge to
a solution.  When this failure occurs, the program will terminate the job.

Failure to converge in dc analysis is usually due to an
error  in specifying circuit connections, element values, or
model parameter values.  Regenerative switching circuits  or
circuits  with  positive feedback probably will not converge
in the dc analysis unless the OFF option is used for some of
the  devices  in  the feedback path, or the .NODESET line is
used to force the circuit to converge to the desired state.

\subsection{Switches}
The use of an ideal element that is  highly  non-linear
such as a switch can cause large discontinuities to occur in
the circuit node voltages.  A  rapid  change  such  as  that
associated  with a switch changing state can cause numerical
roundoff or tolerance problems leading to erroneous  results
or  timestep difficulties.  The user of switches can improve
the situation by taking the following steps:

First of all it is wise to set ideal switch  impedences
only  high  and  low enough to be negligible with respect to
other circuit elements.  Using switch  impedences  that  are
close  to "ideal" in all cases will aggravate the problem of
discontinuities mentioned above.  Of course,  when  modeling
real  devices  such  as MOSFETS, the on resistance should be
adjusted to a realistic level depending on the size  of  the
device being modelled.

If a wide range of ON to OFF resistance must be used in
the switches (ROFF/RON >1e+12), then the tolerance on errors
allowed during transient analysis  should  be  decreased  by
using the .OPTIONS line and specifying TRTOL to be less than
the default value of 7.0.  When switches are  placed  around
capacitors,  then  the option CHGTOL should also be reduced.
Suggested values for these two options  are  1.0  and  1e-16
respectively.   These changes inform SPICE3 to be more careful
around the switch points so that no errors are made  due
to the rapid change in the circuit.

