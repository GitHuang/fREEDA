\element{Z}{Distributed Discontinuity}
Only a few versions of \spice\ support this.
\begin{figure}[h]
\centering
\epsfxsize=2in \pfig{z_spice.ps}
\caption{Z --- distributed discontinuity.}
\end{figure}
\form{ {\tt Z}name N1 ... Nn Mname}

%\offset \offsetparbox{
The Z element defines a subcircuit of L`s and C's
describing a distributed element. The L and C subcircuit is calculated
using field theoretic based models and are optimized for maximum accuracy
at 3GHz. In a contrast a quasistatic model is evaluated at \dc\ and its
accuracy will degrade as frequencies increase.
% That is calculated at 3 GHz
The upper frequency of validity is selectable by the model parameter {\tt N}.
{\tt N}  is the harmonic of 3 GHz at which the model is valid.  The default
for {\tt N} is 5 so that by default the models are valid to a frequency
of 15 GHz.

{\it Mname} is the name of a model which can be one of several types:\\
\begin{tabular}{ll}
 {\tt ZSTEP}    &   Microstrip impedance step \\
 {\tt LBEND}    &   Microstrip right-angle bend \\
 {\tt MBEND}    &   Microstrip right-angle bend \\
 {\tt TJUNC}    &   Microstrip T-junction (three way junction) \\
 {\tt XJUNC}    &   Microstrip X-junction (four way bend)
 \end{tabular}
%  }
\\[0.1in]
\example{Z1 1 5 7 Tjunc1}

\clearpage
%
% LBEND
%
\model{LBEND}{Microstrip Right-Angle Bend Model}

\begin{figure}[h]
\centering
\epsfxsize=5in \pfig{lbend.ps}
\caption{LBEND --- microstrip right-angle bend model.}
\end{figure}

\form{{\tt .MODEL} Mname {\tt LBEND} \B {\tt ER}=value\E
     {\tt H}=xvalue {\tt W1}=xvalue {\tt W2}=xvalue \B {\tt N}=ivalue\E }

\keyword{
{\tt   ER}   & Permittivity of dielectric layer & - & 1.0      \X
{\tt   H}    & Height of dielectric layer   & mm& \reqd \X
{\tt   W1}   & First line width         & mm& \reqd \X
{\tt   W2}   & Second line width        & mm& required \X
{\tt   N}    & Number of calculated harmonics (0-15)& - & 5        \X
   }

\example{.MODEL MB1 LBEND ER=9.8 H=0.635 W1=1.2 W2=3.2}

%
% MBEND
%
\clearpage
\model{MBEND}{Microstrip Mitered Right-Angle Bend Model}

\begin{figure}[h]
\centering
\ \epsfxsize=5in\pfig{mbend.ps}
\caption{MBEND --- microstrip mitered right-angle bend model.}
\end{figure}

\form{{\tt .MODEL} Mname {\tt MBEND} \B {\tt ER}=value\E
     {\tt H}=xvalue {\tt W1}=xvalue {\tt W2}=xvalue \B {\tt N}=ivalue\E }

\keyword{
{\tt   ER}   & Permittivity of dielectric layer & - & 1.0      \X
{\tt   H}    & Height of dielectric layer   & mm& \reqd \X
{\tt   W1}   & First line width         & mm& \reqd \X
{\tt   W2}   & Second line width        & mm& \reqd \X
{\tt   N}    & Number of calculated harmonics (0-15)& - & 5     \X
   }

\example{.MODEL MB1 MBEND ER=9.8 H=0.635 W1=1.2 W2=3.2}

%
% TJUNC
%
\clearpage
\model{TJUNC}{Microstrip T-Junction Model}

\begin{figure}[h]
\centering
\ \epsfxsize=5in\pfig{tjunc.ps}
\caption{TJUNC --- microstrip T-junction model.}
\end{figure}

\form{{\tt .MODEL} Mname {\tt TJUNC} \B {\tt ER}=value\E
     {\tt H}=xvalue {\tt W1}=xvalue {\tt W2}=xvalue {\tt W3}=xvalue
     \B {\tt S}=xvalue\E  \B {\tt N}=ivalue\E }

\keyword{
{\tt   ER}   & Permittivity of dielectric layer & - & 1.0    \X
{\tt   H}    & Height of dielectric layer   & mm& \reqd \X
{\tt   W1}   & First line width         & mm& \reqd \X
{\tt   W2}   & Second line width        & mm& \reqd \X
{\tt   W3}   & Third line width     & mm& \reqd \X
{\tt   S}    & Edge offset of line 3        & mm& 0        \X
{\tt   N}    & Number of calculated harmonics (0-15)& - & 5\X
   }


\example{.MODEL TJ1 TJUNC ER=9.8 H=0.635 W1=1.2 W2=3.2 W3=1.1 W4=1}

%
% XJUNC
%
\clearpage
\model{XJUNC}{Microstrip X-junction}

\begin{figure}[h]
\centering
\ \epsfxsize=5in\pfig{xjunc.ps}
\caption{ XJUNC --- microstrip X-junction.}
\end{figure}

\form{{\tt .MODEL} Mname {\tt XJUNC} \B {\tt ER}=value\E
     {\tt H}=xvalue {\tt W1}=xvalue {\tt W2}=xvalue
     {\tt W3}=xvalue {\tt W4}=xvalue \B {\tt S}=xvalue\E
     \B {\tt N}=ivalue\E }

\keyword{
{\tt   ER}   & Permittivity of dielectric layer & - & 1.0    \X
{\tt   H}    & Height of dielectric layer   & mm& \reqd \X
{\tt   W1}   & First line width         & mm& \reqd \X
{\tt   W2}   & Second line width        & mm& \reqd \X
{\tt   W3}   & Third line width     & mm& \reqd \X
{\tt   S}    & Edge offset of line 3        & mm& 0        \X
{\tt   N}    & Number of calculated harmonics (0-15)& - & 5\X
   }


\example{.MODEL TJ1 TJUNC ER=9.8 H=0.635 W1=1.2 W2=3.2 W3=1.1 W4=1}

%
% ZSTEP
%
\clearpage
\model{ZSTEP}{Microstrip Impedance Step Model}

\begin{figure}[h]
\centering
\ \epsfxsize=5in\pfig{zstep.ps}
\caption{ZSTEP --- microstrip impedance step model.}
\end{figure}

\form{{\tt .MODEL} Mname {\tt ZSTEP} \B {\tt ER}=value\E
     {\tt H}=xvalue {\tt W1}=xvalue {\tt W2}=xvalue \B {\tt S}=xvalue\E
     \B {\tt N}=ivalue\E }

\keyword{
{\tt   ER}   & Permittivity of dielectric layer & - & 1.0   \X
{\tt   H}    & Height of dielectric layer   & mm& required \X
{\tt   W1}   & First line width         & mm& required \X
{\tt   W2}   & Second line width        & mm& required \X
{\tt   S}    & Displacement (see figure)    & mm& 0        \X
{\tt   N}    & Number of calculated harmonics (0-15)& - & 5\X
   }

\example{.MODEL STEP1 ZSTEP ER=9.8 H=0.635 W1=1.2 W2=3.2}

