%\documentclass{article}
%\usepackage{epsf}
%\newcommand{\fig}[1]{J:/eos.ncsu.edu/users/m/mbs/mbs_group/figures/#1}
%\newcommand{\fig}[1]{../figures/#1}
%\newcommand{\pfig}[1]{\epsfbox{\fig{#1}}}
%\newcommand{\ms}[1]{\mbox{\scriptsize #1}}
%\newcommand{\B}{{ \rm [}}     % begin optional parameter in \form{}
%\newcommand{\E}{{\ \rm\hspace{-0.04in}] }}   % end optional parameter in \form{}

\oddsidemargin 10mm \topmargin 0.0in \textwidth 5.5in \textheight 7.375in
\evensidemargin 1.0in \headheight 0.18in \footskip 0.16in
%%%%%%%%%%%%%%%%%%%%%%%%%%%%%%%%%%%%%%%% The title
%\begin{document}
\section[H \- Current-Controlled Voltage Source]{\noindent{\LARGE \textbf{Current-Controlled Voltage Source} \hspace{35mm}\huge\textbf{H}}}
%\newline
\linethickness{1mm}
\line(1,0){425}
\normalsize
%%%%%%%%%%%%%%%%%%%%%%%%%%%%%%%%%%%%%%%% the resistor figure
\begin{figure}[h]
\centerline{\epsfxsize=2.5in\pfig{h_spice.ps}} \caption{H ---
current-controlled voltage source element.}
\end{figure}
%%%%%%%%%%%%%%%%%%%%%%%%%%%%%%%%%%%%%%%% form for \FDA
\linethickness{0.5mm} \line(1,0){425}
\newline
\textit{Form:}
\newline
 {\tt H}name $N_{+}$ $N_{-}$ {\it VoltageSourceName Transresistance}\\
     {\tt H}name $N_{+}$ $N_{-}$ {\tt POLY(} D {\tt )}
     {\it VoltageSourceName}$_1$ ... {\it VoltageSourceName}$_D$ {\it
     PolynomialCoefficients}
\newline
%%%%%%%%%%%%%%%%%%%%%%%%%%%%%%%%%%%%%%%%
\begin{tabular}{r l}
$N_{+}$ & is the positive voltage source node.\\
$N_{-}$ & is the negative voltage source node.\\
{\it VoltageSourceName} & is the name of the voltage source the
current through which is \\
& the controlling current. The voltage source must be a {\tt V} element.\\
{\it Transresistance} & is the Transresistance of the element.\\
{\tt POLY} & is the identifier for the polynomial form of the
element.\\
{\it D} & is the degree of the polynomial. The number of\\
& pairs of controlling nodes must be equal to {\it Degree}.\\
{\it VoltageSourceName$_i$} & is the name of the voltage source
the current through \\
& which is the $i$th controlling current. The voltage source must be a {\tt V} element.\\
{\it PolynomialCoefficients} & is the set of polynomial
coefficients which must be specified\\
& in the standard polynomial coefficient format discussed in the
description.
\end{tabular}
%%%%%%%%%%%%%%%%%%%%%%%%%%%%%%%%%%%%%%% Parameter list
%%%%%%%%%%%%%%%%%%%%%%%%%%%%%%%%%%%%%%% example in \FDA
\newline
\linethickness{0.5mm} \line(1,0){425}
\newline
\textit{Example:}
\newline
\texttt{H1 2 3 14 1 2.0}
\newline
\linethickness{0.5mm} \line(1,0){425}
\newline
\textit{Description:}\\
Polynomial expressions can be used with the controlled source
elements ({\tt E}, {\tt F}, {\tt G} and {\tt H}) to realize
nonlinear controlled sources. The specification of the polynomial
must be at the end of the input line and has two forms.

The format for a current-controlled voltage source (the {\tt H}
element) is\\[0.1in]
\hspace*{\fill}\offsetparbox{ {\tt POLY({\it N}) {\it
VoltageSourceName$_1$} $\ldots$ {\it VoltageSourceName$_N$} $C_0$
$C_1$ $C_2$ $C_3$ $\ldots$ }}\\
where
\newline
\begin{tabular}{r l}
{\tt POLY} & is the keyword indicating that that a polynomial
description follows.\\
{\it N} & is the degree of the polynomial.\\
{\it VoltageSourceName$_1$} & is the name of the voltage source
the current through\\
& which is control current $I_1$.\\
{\it VoltageSourceName$_N$} & is the name of the voltage source
the current through\\
& which is control current $I_N$.\\
$C_0$ $C_1$ $\ldots$ & are the polynomial coefficients.
\end{tabular}

For these elements the output is calculated as
\begin{eqnarray}
{\rm OUTPUT} & = & C_0 \nonumber\\
         &   & + C_1V_1 + \ldots + C_NV_N\nonumber\\
         &   & + C_{N+1}V_1V_1 + C_{N+2}V_1V_2 + \ldots + C_{N+N}V_1V_N
             \nonumber\\
         &   & + C_{2N+1}V_2V_2 + C_{2N+2}V_2V_3 +\ldots + C_{2N+N-1}V_2V_N
             \nonumber\\
         &   & \vdots
             \nonumber\\
         &   & + C_{N!/(2(N-2)!)+2N}V_NV_N
             \nonumber\\
         &   & + C_{N!/(2(N-2)!)+2N+1}V_1V_1V_1 +
               C_{N!/(2(N-2)!)+2N+2}V_1V_1V_2\\
         &   & \ \ \ \ \ \ \ + \ldots +
               C_{N!/(2(N-2)!)+2N+N-1}V_1V_1V_N
                   \nonumber\\
         &   & + C_{N!/(2(N-2)!)+3N}V_1V_2V_2 + \ldots +
               C_{N!/(2(N-2)!)+3N+N-2}V_1V_2V_N
                   \nonumber\\
         &   & \vdots
\end{eqnarray}

An example of a current-controlled voltage source is:\\[0.1in]
\hspace*{\fill}\offsetparbox{H1 2 3 POLY(2) VIN V2 0.5 1 1 0.2 0.3
0.2}\\[0.1in]


\noindent\underline{Linear Transresistance Instance}
\\[0.1in]\hspace*{\fill}\offsetparbox{\it
     {\tt H}name $N_{+}$ $N_{-}$ $N_{C+}$ $N_{C-}$ Transresistance }\\[0.1in]
The value of the voltage generator is linearly proportional to the
controlling current:
\begin{equation}
v_o = Transresistance\,v_c
\end{equation}
\\[0.2in]\noindent\underline{{\tt POLY}nomial Instance}
\\[0.1in]\hspace*{\fill}\offsetparbox{\it
     {\tt H}name $N_{+}$ $N_{-}$ {\tt POLY(} D {\tt )}
     {\tt (}$N_{C1+}$ $N_{C1-}{\tt)}$
     ... {\tt (}$N_{CD+}$ $N_{CD-}{\tt )}$ PolynomialCoefficients }\\[0.1in]
The value of the voltage generator is a polynomial function of the
controlling voltages:
\begin{equation}
v_o = f(i_{c1}, ...,  i_{ci}, ...  i_{cD}
\end{equation}
where the number of controlling currents is $D$ --- the degree of
the polynomial specified on the element line. $i_{ci}$ is the
$i$th controlling current and is the current flowing from the $+$
terminal to the $-$ terminal in the $i$th voltage source of name
{\it VoltageSourceName}.
\newline
\linethickness{0.5mm} \line(1,0){425}
\newline
\textit{Notes:}\\
The actual element in \FDA is the \texttt{ccvs} element.
See \texttt{ccvs} for full documentation.\\
%%%%%%%%%%%%%%%%%%%%%%%%%%%%%%%%%%%%%%%%%%%%% Credits
\linethickness{0.5mm} \line(1,0){425}
\newline
\textit{Credits:}\\
\begin{tabular}{l l l l}
Name & Affiliation & Date & Links \\
Satish Uppathil & NC State University & Sept 2000 & \epsfxsize=1in\pfig{logo.eps} \\
svuppath@eos.ncsu.edu & & & www.ncsu.edu    \\
\end{tabular}
%\end{document}
