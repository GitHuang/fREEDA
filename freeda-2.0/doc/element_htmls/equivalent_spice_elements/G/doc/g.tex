%\documentclass{article}
%\usepackage{epsf}
%\newcommand{\fig}[1]{J:/eos.ncsu.edu/users/m/mbs/mbs_group/figures/#1}
%\newcommand{\fig}[1]{../figures/#1}
%\newcommand{\pfig}[1]{\epsfbox{\fig{#1}}}
%\newcommand{\ms}[1]{\mbox{\scriptsize #1}}
%\newcommand{\B}{{ \rm [}}     % begin optional parameter in \form{}
%\newcommand{\E}{{\ \rm\hspace{-0.04in}] }}   % end optional parameter in \form{}

\oddsidemargin 10mm \topmargin 0.0in \textwidth 5.5in \textheight 7.375in
\evensidemargin 1.0in \headheight 0.18in \footskip 0.16in
%%%%%%%%%%%%%%%%%%%%%%%%%%%%%%%%%%%%%%%% The title
%\begin{document}
\section[G \- Voltage-Controlled Current Source]{\noindent{\LARGE \textbf{Voltage-Controlled Current Source} \hspace{35mm}\huge\textbf{G}}}
%\newline
\linethickness{1mm}
\line(1,0){425}
\normalsize
%%%%%%%%%%%%%%%%%%%%%%%%%%%%%%%%%%%%%%%% the resistor figure
\begin{figure}[h]
\centerline{\epsfxsize=3in\pfig{g_spice.ps}} \caption{G ---
voltage-controlled current source element.}
\end{figure}
%%%%%%%%%%%%%%%%%%%%%%%%%%%%%%%%%%%%%%%% form for \FDA
\linethickness{0.5mm} \line(1,0){425}
\newline
\textit{Form:}
\newline
{\tt G}name $N_{+}$ $N_{-}$ $N_{C+}$ $N_{C-}$ {\it Transconductance}\\
     {\tt G}name $N_{+}$ $N_{-}$ {\tt POLY(} D {\tt )} $N_{C+}$ $N_{C-}$
        {\it PolynomialCoefficients}
\newline
%%%%%%%%%%%%%%%%%%%%%%%%%%%%%%%%%%%%%%%%
\begin{tabular}{r l}
$N_{+}$ & is the positive voltage source node.\\
$N_{-}$ & is the negative voltage source node.\\
$N_{C+}$ & is the positive controlling node.\\
$N_{C-}$ & is the negative controlling node.\\
{\it Transconductance} & is the transconductance.\\
{\tt POLY} & is the identifier for the polynomial form of the
element.\\
{\it D} & is the degree of the polynomial. The number of
pairs of controlling nodes \\
& must be equal to {\it Degree}.\\
$N_{Ci+}$ & the positive node of the $i$ th controlling node
pair.\\
$N_{Ci-}$ & the negative node of the $i$ th controlling node
pair.\\
{\it PolynomialCoefficients} & is the set of polynomial
coefficients which must be specified\\
& in the standard polynomial coefficient format discussed in the description.\\
\notforsspice{ {\tt VALUE} & is the identifier for the
\underline{value form} of the element.\\
{\it Expression} & This is an expression of the form discussed in the description.\\
{\tt TABLE} & is the identifier for the \underline{table form} of
the element.\\
{\it TableInput} & This is the independent input of the table. See
the {\tt TABLE} parameter above.\\
{\it TableInput} & This is the dependent output of the table. See
the {\tt TABLE} parameter above.\\
{\tt LAPLACE} & is the identifier for the \underline{laplace form}
of the element.\\
{\it TransformExpression}\\
{\tt FREQ} & is the identifier for the \underline{frequency form} of the element.\\
{\it Frequency}\\
{\it Magnitude}\\
{\it Phase}\\
{\tt CHEBYSHEV} & is the identifier for the \underline{chebyshev
form} of the element.\\
{\it Type}\\
{\it CutoffFrequency}\\
{\it Phase}}
\end{tabular}
%%%%%%%%%%%%%%%%%%%%%%%%%%%%%%%%%%%%%%% Parameter list
%%%%%%%%%%%%%%%%%%%%%%%%%%%%%%%%%%%%%%% example in \FDA
\newline
\linethickness{0.5mm} \line(1,0){425}
\newline
\textit{Example:}
\newline
\texttt{G1 2 3 14 1 2.0}
\newline
\linethickness{0.5mm} \line(1,0){425}
\newline
\textit{Description:}\\
Polynomial expressions can be used with the controlled source
elements ({\tt E}, {\tt F}, {\tt G} and {\tt H}) to realize
nonlinear controlled sources. The specification of the polynomial
must be at the end of the input line and has two forms. The
polynomial format for a voltage-controlled current source (the
{\tt G} element) is\\[0.1in]
\hspace*{\fill}\offsetparbox{ {\tt POLY({\it N}) ($N_{C1+}$,{\it
$N_{C1-}$}) $\ldots$ ($N_{CN+}$, $N_{CN-}$) $C_0$ $C_1$ $C_2$
$C_3$ $\ldots$ }}
\newline
where
\newline
%%%%%%%%%%%%%%%%%%%%%%%%%%%%%%%%%%%%%%%%
\begin{tabular}{r l}
{\tt POLY} & is the keyword indicating that a polynomial
description follows.\\
{\it N} & is the degree of the polynomial.\\
$N_{C1+}$, $N_{C1-}$ & The voltage at the node $N_{C1+}$ with
respect to\\
& the voltage at the node $N_{C1-}$ is the controlling voltage
$V_1$.\\
$N_{CN+}$, $N_{CN-}$ & The voltage at the node $N_{CN+}$ with
respect to\\
& the voltage at the node $N_{CN-}$ is the controlling voltage
$V_N$.\\
$C_0$ $C_1$ $\ldots$ & are the polynomial coefficients. Not all
of\\
& the coefficients need be specified as the trailing
coefficients\\
& that are not specified are treated as if they are zero.
\end{tabular}

Note that in spice parentheses, ``('' and ``)'', and commas,
``,'', are treated as if they are spaces. The use of parentheses
and commas serves only to make the netlist more easily read. {\bf
The exception to this is their use in expressions.}

For voltage-controlled elements the output is calculated as
\begin{eqnarray}
{\rm OUTPUT} & = & C_0 \nonumber\\
         &   & + C_1V_1 + \ldots + C_NV_N\nonumber\\
         &   & + C_{N+1}V_1V_1 + C_{N+2}V_1V_2 + \ldots + C_{N+N}V_1V_N
               \nonumber\\
         &   & + C_{2N+1}V_2V_2 + C_{2N+2}V_2V_3 +\ldots + C_{2N+N-1}V_2V_N
               \nonumber\\
         &   & \vdots \nonumber\\
         &   & + C_{N!/(2(N-2)!)+2N}V_NV_N
               \nonumber\\
         &   & + C_{N!/(2(N-2)!)+2N+1}V_1V_1V_1 +
               C_{N!/(2(N-2)!)+2N+2}V_1V_1V_2\nonumber\\
         &   &\ \ \ \ \ \ \ \ \ + \ldots +
               C_{N!/(2(N-2)!)+2N+N-1}V_1V_1V_N
                     \nonumber\\
         &   & + C_{N!/(2(N-2)!)+3N}V_1V_2V_2 + \ldots +
               C_{N!/(2(N-2)!)+3N+N-2}V_1V_2V_N
                     \nonumber\\
         &   & \vdots\nonumber
\end{eqnarray}
A one dimensional polynomial (with only one pair of controlling
nodes) is evaluated as
\[
{\rm OUTPUT} = C_0 + C_1V_1 + C_2V_1^2 + C_3V_1^3 + \ldots
C_NV_1^N
\]
An example of a voltage-controlled current source is\\[0.1in]
\hspace*{\fill}\offsetparbox{{\tt G1 2 3 POLY(4) (10,0) (12,2)
(11,0) (13,0)
   0.5 1 1 1 1 0.2 0.3 0.2}}\\[0.1in]
\newline

Several form of the voltage-controlled voltage source element are
supported
      in addition to the \underline{Linear Transconductance}
      form which is the default.
      The other forms are selected based on the the identifier
      {\tt POLY},
      {\tt VALUE},
      {\tt TABLE},
      {\tt LAPLACE},
      {\tt FREQ} or
      {\tt CHEBYSHEV}.

\noindent\underline{Linear Transconductance Instance}
\\[0.1in]\hspace*{\fill}\offsetparbox{\it
     {\tt G}name $N_{+}$ $N_{-}$ $N_{C+}$ $N_{C-}$ Transconductance }\\[0.1in]
The value of the voltage generator is linearly proportional to the
controlling voltage:
\begin{equation}
v_o = Transconductance\,v_c
\end{equation}
\\[0.2in]\noindent\underline{{\tt POLY}nomial Instance}
\\[0.1in]\hspace*{\fill}\offsetparbox{\it
     {\tt G}name $N_{+}$ $N_{-}$ {\tt POLY(} D {\tt )}
     {\tt (}$N_{C1+}$ $N_{C1-}{\tt)}$
     ... {\tt (}$N_{CD+}$ $N_{CD-}{\tt )}$ PolynomialCoefficients }\\[0.1in]
The value of the voltage generator is a polynomial function of the
controlling voltages:
\begin{equation}
v_o = f(v_{c1}, ...,  v_{ci}, ...  v_{cD})
\end{equation}
where the number of controlling voltages is $D$ --- the degree of
the polynomial specified on the element line. $v_{ci}$ is the
$i$th controlling voltage and is the voltage of the $n_{ci+}$ node
with respect to the $n_{ci+}$ node. \notforsspice{
\\[0.2in]\noindent\underline{{\tt VALUE} Instance} --- \pspice92 only
\\[0.1in]\hspace*{\fill}\offsetparbox{\it
     {\tt G}name $N_{+}$ $N_{-}$ {\tt VALUE= \{} Expression {\tt \}} }\\[0.1in]
The value of the voltage generator is the resultant of an
expression evaluation.
\begin{equation}
v_o = f(v_c)
\end{equation}
\\[0.2in]\noindent\underline{{\tt TABLE} Instance} --- \pspice92 only
\\[0.1in]\hspace*{\fill}\offsetparbox{\it
     {\tt G}name $N_{+}$ $N_{-}$ {\tt TABLE \{}
          Expression {\tt \}=(} TableInput {\tt ,} TableOutput {\tt )} ...}\\[0.1in]
\begin{equation}
v_o = f(v_c)
\end{equation}
\\[0.2in]\noindent\underline{{\tt LAPLACE} Instance} --- \pspice92 only
\\[0.1in]\hspace*{\fill}\offsetparbox{\it
     {\tt G}name $N_{+}$ $N_{-}$ {\tt LAPLACE \{}
            Expression {\tt \}=\{} TransformExpression {\tt \}} }\\[0.1in]
\begin{equation}
v_o = f(v_c)
\end{equation}
\\[0.2in]\noindent\underline{{\tt FREQ}} --- \pspice92 only
\\[0.1in]\hspace*{\fill}\offsetparbox{\it
     {\tt G}name $N_{+}$ $N_{-}$ {\tt FREQ \{}
       Expression {\tt \}=(} Frequency{\tt ,} Magnitude{\tt ,} Phase {\tt )} ...}
       \\[0.1in]
\begin{equation}
v_o = f(v_c)
\end{equation}
\\[0.2in]\noindent\underline{{\tt CHEBYSHEV}} --- \pspice92 only
\\[0.1in]\hspace*{\fill}\offsetparbox{\it
     {\tt G}name $N_{+}$ $N_{-}$ {\tt CHEBYSHEV \{}
          Expression {\tt \}=} Type{\tt ,} CutoffFrequency ... {\tt ,}
      Phase ...  }\\[0.1in]
      }
\newline
\linethickness{0.5mm} \line(1,0){425}
\newline
\textit{Notes:}\\
The actual element in \FDA is the \texttt{G} element.
See \texttt{G} for full documentation.\\
%%%%%%%%%%%%%%%%%%%%%%%%%%%%%%%%%%%%%%%%%%%%% Credits
\linethickness{0.5mm} \line(1,0){425}
\newline
\textit{Credits:}\\
\begin{tabular}{l l l l}
Name & Affiliation & Date & Links \\
Satish Uppathil & NC State University & Sept 2000 & \epsfxsize=1in\pfig{logo.eps} \\
svuppath@eos.ncsu.edu & & & www.ncsu.edu    \\
\end{tabular}
%\end{document}
