%\documentclass{article}
%\usepackage{epsf}
%\newcommand{\fig}[1]{J:/eos.ncsu.edu/users/m/mbs/mbs_group/figures/#1}
%\newcommand{\fig}[1]{../figures/#1}
%\newcommand{\pfig}[1]{\epsfbox{\fig{#1}}}

\oddsidemargin 10mm \topmargin 0.0in \textwidth 5.5in \textheight
7.375in
\evensidemargin 1.0in \headheight 0.18in \footskip 0.16in
%%%%%%%%%%%%%%%%%%%%%%%%%%%%%%%%%%%%%%%% The title
%\begin{document}
\section[J \- JFET]{\LARGE \textbf{Junction Field Effect Transistor} \hspace{40mm}\huge \textbf{J}}
%\newline
\linethickness{1mm} \line(1,0){425} \normalsize
%\newline
%%%%%%%%%%%%%%%%%%%%%%%%%%%%%%%%%%%%%%%% the diode figure
\begin{figure}[h]
\centerline{\epsfxsize=2.4in\pfig{j.ps}} \caption{J --- JFET
element (a) n-channel JFET ; (b) p-channel JFET}
\end{figure}
\newline
\linethickness{0.5mm} \line(1,0){425}
\newline
%%%%%%%%%%%%%%%%%%%%%%%%SPICE FORM%%%%%%%%%%%%%%%%%%%%%%%%%%%%%%%%%%%%%%%%%
\noindent\texttt{SPICE} \textit{Form}:
\newline
{\tt J}name  \textit{NDrain} \textit{NGate} \textit{NSource} \texttt{[}\textit{ModelName}\texttt{]} [Area] [{\tt OFF}] [{\tt IC=}Vbe,Vce]\\

\begin{tabular}{r l}
{\it NDrain} & is the drain node, \\
{\it NGate} & is the gate node, \\
{\it NSource} & is the source node, \\
{\it ModelName} & is the model name, (\textit{ModelType} is either \texttt{NJF} or \texttt{PJF}.) \\
{\it Area} & is  the  area  factor.(Units: none; Optional; Default: 1; Symbol: $Area$) \\
{\tt OFF} & indicates an (optional) initial condition on the device for the dc\ analysis. If \\
& specified the dc\ operating point is calculated with the terminal voltages set to \\
& zero. Once convergence is obtained, the program continues to iterate to obtain \\
& the exact value of the  terminal  voltages.  The OFF option is used to enforce \\
& the solution to  correspond to  a  desired state if the circuit has \\
& more than one stable state. \\
{\tt IC} & is the optional initial condition specification using  {\tt IC=}$V_{BE},V_{CE}$ is  intended \\
& for use with the {\tt UIC} option on the {\tt .TRAN} line, when a transient analysis is \\
& desired  starting  from  other than  the  quiescent  operating  point. See  the {\it .IC} \\
& line description for a better way to set transient initial
conditions.
\end{tabular}
\newline
%\vspace{4mm}
%%%%%%%%%%%%%%%%%%%%%%%%%%%%%%%%%%%%%%%%%%%%%%%%%%%%%%%%%%%%%%%%%%%%% example in SPICE
%\newline
\linethickness{0.5mm} \line(1,0){425}
\newline
\noindent\textit{Example}:
\newline
J2 12 10 3 OFF \\
%%%%%%%%%%%%%%%%%%%%%%%%%%%%%%%%%%%%%%%%%%%%%%% Parameter table
%\newpage
\linethickness{0.5mm} \line(1,0){425}
\newline
\noindent\textit{Model Parameters}:\\[0.1in]
The parameters remain the same for an n-channel and p-channel JFET
\newline
%%%%%%%%%%%%%%%%%%%%%%%%%%%%%%%%%%%%%%%%%%%
\begin{tabular}{|r|l|c|c|}
\hline
\textbf{Name} & \textbf{Description} & \textbf{Units} & \textbf{Default} \\
\hline
{\tt AF} & flicker noise exponent ($A_F$) & - & 1 \\
\hline
{\tt BETA} & transconductance parameter ($\beta$) & $\mbox{A/V}^2$ & 1.0E-4 \\
\hline
{\tt CGS} & zero-bias G-S junction capacitance per unit area ($C'_{GS}$) & F & 0 \\
\hline
{\tt CGD} & zero-bias G-D junction capacitance per unit area ($C'_{GD}$) & F & 0 \\
\hline
{\tt FC} & coefficient for forward-bias depletion capacitance formula ($F_C$) & - & 0.5 \\
\hline
{\tt IS}& gate junction saturation current ($I_S$) & A & 1.0E-14 \\
\hline
{\tt KF} & flicker noise coefficient ($K_F$) & - & 0 \\
\hline
{\tt LAMBDA} & channel length modulation parameter ($\lambda$) & 1/V & 0 \\
\hline
{\tt PB} & gate junction potential ($\phi_J$) & V & 1 \\
\hline
{\tt RD} & drain ohmic sheet resistance ($R_D$) & $\Omega$ & 0 \\
\hline
{\tt RS} & source ohmic sheet resistance ($R_S$) & $\Omega$ & 0 \\
\hline
{\tt VTO} & threshold voltage ($V_{T0}$) & V & -2.0 \\
          & {\tt VTO} $<$ 0 indicates a depletion mode JFET & & \\
          & {\tt VTO} $\ge$ 0 indicates an enhancement mode JFET & & \\
\hline
{\tt BETATC}& temperature coefficient of {\tt BETA} ($T_{C,\beta}$) & /$^\circ$C & 0 \\
\hline
{\tt M} & gate {\it p-n} junction grading  coefficient ($M$) & - & 0.5 \\
\hline
{\tt VTOTC}& temperature coefficient of threshold voltage {\tt VTO} ($T_{C,VT0}$) & V/$^\circ$C & 0 \\
\hline
\end{tabular}
\newline
%%%%%%%%%%%%%%%%%%%%%%%%%%DESCRIPTION%%%%%%%%%%%%%%%%%%%%%%%%%%%%%%%%%%%%%%%%%%%%%%%%%
\textit{Description}:\\
\textit{\underline{JFET Model}}:
\newline
\begin{figure}[h]
\centerline{\epsfxsize=3.5in\pfig{jfet_eq.ps}} \caption{Schematic
of the JFET Model}
\end{figure}
\newline
%\vspace{5mm}
%\newline
The physical constants used in the model evaluation are
\begin{center}
\begin{tabular}{|l|l|l|}
\hline
$k$ & Boltzmann's constant           &  $1.3806226\,10^{-23}$~J/K\\
$q$ & electronic charge             & $1.6021918\,10^{-19}$~C\\
\hline
\end{tabular}
\end{center}
\noindent\textit{\underline{\large Standard Calculations}}:\\[0.1in]
Absolute temperatures (in kelvins, K) are used. The thermal
voltage
\begin{equation}
V_{TH} = {{kT_{NOM}} \over q} .
\end{equation}
\noindent The silicon bandgap energy
\begin{equation}
E_G = 1.16 - 0.000702{{4T_{NOM}^2} \over {T_{NOM}+1108}} .
\end{equation}
%%%%%%%%%%%%%%%%%%%%%%%%%%%%%%%%%%%%% Current Characteristics
\noindent\textit{\underline{\large Current Characteristics}}:\\[0.1in]
The current/voltage characteristics are evaluated after first
determining the mode (normal: $V_{DS} \ge 0$ or inverted: $V_{DS}
< 0$) and the region (cutoff, linear or saturation) of the current
$(V_{DS}, V_{GS})$ operating point.\\[0.1in]

\noindent{\sl Normal Mode: ($V_{DS} \ge 0$)}\\[0.2in]
Regions of operation:
\newline
\hspace*{12mm}{
\begin{tabular}{l l}
$V_{GS} - V_{T0} \leq 0$ & Cutoff Region \\
$0 \leq V_{DS} < V_{GS} - V_{T0}$ & Linear Region \\
$0 < V_{GS} - V_{T0} \leq V_{DS}$ & Saturation Region
\end{tabular}}\\[0.1in]
Then
\begin{equation}
I_{D} = \left\{ \begin{array}{ll}
      0  & \mbox{cutoff region} \\ \\
      Area\,\beta \left(1 + \lambda V_{DS}\right)V_{DS}
      \left[2\left(V_{GS}-V_{T0}\right)-V_{DS}\right]
         &\mbox{linear region}\\
      Area\,\beta \left(1 + \lambda V_{DS}\right)
      \left(V_{GS}-V_{T0}\right)^2
         &\mbox{saturation region} \end{array} \right. %}
\end{equation}

\noindent{\sl Inverted Mode: ($V_{DS} < 0)$}\\[0.2in]
Regions of operation:
\newline
\hspace*{12mm}{
\begin{tabular}{l l}
$V_{GD} - V_{T0} \leq 0$ & Cutoff Region \\
$0 \leq -V_{DS} < V_{GD} - V_{T0}$ & Linear Region \\
$0 < V_{GD} - V_{T0} \leq -V_{DS}$ & Saturation Region
\end{tabular}}\\[0.1in]
\begin{equation}
I_{D} = \left\{ \begin{array}{ll}
      0  & \mbox{cutoff region} \\ \\
      Area\,\beta \left(1 - \lambda V_{DS}\right)V_{DS}
      \left[2\left(V_{GD}-V_{T0}\right)+V_{DS}\right]
         &\mbox{linear region}\\
      Area\,(-\beta) \left(1 - \lambda V_{DS}\right)
      \left(V_{GD}-V_{T0}\right)^2
         &\mbox{saturation region} \end{array} \right. %}
\end{equation}\\[0.2in]
%%%%%%%%%%%%%%%%%%%%%%%%%%%%%%%%%%%%%%%% Leakage Currents
\noindent\textit{\underline{\large Leakage Currents}}:\\[0.1in]
Current flows across the normally reverse biased source-bulk and
drain-bulk junctions. The gate-source leakage current
\begin{equation}
I_{GS} = Area\,I_{S}e^{(\textstyle V_{GS}/V_{TH} -1)}
\end{equation}
and the gate-source leakage current
\begin{equation}
I_{GD} = Area\,I_{S}e^{(\textstyle V_{GD}/V_{TH} -1)}
\end{equation}
\newline
%\vspace{4mm}
%%%%%%%%%%%%%%%%%%%%%%%%%%%%%%%%%%%%%% Capacitances
\textit{\underline{\large Capacitances}}:\\[0.1in]
The drain-source capacitance
\begin{equation}
C_{DS} = Area\,C'_{DS}
\end{equation}
The gate-source capacitance
\begin{equation}
C_{GS} = \left\{ \begin{array}{ll}
         Area\,C'_{GS}\left(1 - {{\textstyle V_{GS}}\over{\textstyle \phi_J}}
         \right)^{-M}
         & V_{GS} \le F_C \phi_J\\
         Area\,C'_{GS}\left(1 -F_C\right)^{-(1+M)}
         \left[1-F_C(1+M)+M {{\textstyle V_{GS}}\over{\textstyle \phi_J}}
         \right]^{-M}
         & V_{GS} > F_C \phi_J
         \end{array} \right. %}
\end{equation}
models charge storage at the gate-source depletion layer. The
gate-drain capacitance
\begin{equation}
C_{GD} = \left\{ \begin{array}{ll}
         Area\,C'_{GD}\left(1 - {{\textstyle V_{GD}}\over{\textstyle \phi_J}}
         \right)^{-M}
         & V_{GD} \le F_C \phi_J\\
         Area\,C'_{GD}\left(1 -F_C\right)^{-(1+M)}
         \left[1-F_C(1+M)+M {{\textstyle V_{GD}}\over{\textstyle \phi_J}}
         \right]^{-M}
         & V_{GD} > F_C \phi_J
         \end{array} \right. %}
\end{equation}
models charge storage at the gate-drain depletion layer.
\newline
\textit{Notes:}\\
The actual elements in \FDA are the \texttt{njfet} and \texttt{pjfet} elements.
See \texttt{njfet} and \texttt{pjfet} for full documentation. \\
%%%%%%%%%%%%%%%%%%%%%%%%%%%%%%%%%%%%%%%%%%%%% Credits
\linethickness{0.5mm}
\line(1,0){425}\\[0.1in]
\noindent\large{\textit{Credits}}: \normalsize
\newline
\begin{tabular}{l l l l}
Name & Affiliation & Date & Links \\
Nikhil Kriplani & North Carolina State University & April 2001 & \epsfxsize=1in\pfig{logo.eps} \\
Shubha Vijaychand & North Carolina State University & April 2001 &
\epsfxsize=1in\pfig{logo.eps}\\
nmkripla@eos.ncsu.edu & & & www.ncsu.edu \\
svijayc@eos.ncsu.edu & & & \\
\end{tabular}
%\end{document}
