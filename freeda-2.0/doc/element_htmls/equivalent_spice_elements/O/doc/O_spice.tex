%\documentclass{article}
%\usepackage{epsf,html}
%\newcommand{\fig}[1]{J:/eos.ncsu.edu/users/m/mbs/mbs_group/figures/#1}
%\newcommand{\fig}[1]{../figures/#1}
%\newcommand{\pfig}[1]{\epsfbox{\fig{#1}}}

\oddsidemargin 10mm \topmargin 0.0in \textwidth 5.5in \textheight
7.375in \evensidemargin 1.0in \headheight 0.18in \footskip 0.16in
%%%%%%%%%%%%%%%%%%%%%%%%%%%%%%%%%%%%%%%% The title
%\begin{document}
\section[O \- Digital Output Interface]{\noindent{\LARGE \textbf{Digital Output Interface}} \hspace{70mm}\huge\textbf{O}}
\linethickness{1mm}
\line(1,0){425}
\normalsize
%%%%%%%%%%%%%%%%%%%%%%%%%%%%%%%%%%%%%%%% the resistor figure
\begin{figure}[h]
\centerline{\epsfxsize=2in\pfig{o_spice.ps}} \caption{O ---
Digital output interface element.}
\end{figure}
%\newline
%%%%%%%%%%%%%%%%%%%%%%%%%%%%%%%%%%%%%%%%%%% SPICE form
%\vspace{2mm}
\newline
\linethickness{0.5mm} \line(1,0){425}
\newline
\texttt{SPICE} \textit{Form:}
\newline
{\tt O}name {\it InterfaceNode ReferenceNode ModelName}
  \B{\tt SIGNAME = } {\it DigitalSignalName} \E
\newline
%%%%%%%%%%%%%%%%%%%%%%%%%%%%%%%%%%%%%%%%%%%%%%% explanation of terms in the SPICE form
\begin{tabular}{r l}
{\it InterfaceNode} & \\
     & Identifier of node interfacing between digital signal and
     continuous time circuit.\\
{\it ReferenceNode} & \\
     & Identifier of reference node. Normally this is ground.\\
{\it ModelName} & \\
     & Name of the model specifying transitions times and resistances and
     capacitances\\
     & of each logic state.\\
{\tt SIGNAME} & \\
     & Keyword for digital signal name. (optional)\\
{\it DigitalSignalName} & \\
     & Digital signal name.\\
\end{tabular}
%\newline
%%%%%%%%%%%%%%%%%%%%%%%%%%%%%%%%%%%%%%%%%%%%%%% Parameter table
%\vspace{4mm}
\linethickness{0.5mm} \line(1,0){425}
\newline
\textit{Example:}
\newline
\texttt{O100 1 0 INTERFACE\_TO\_MEMORY SIGNAME=MEM1 \\
    OADD1 1 0 2 ADD1}
\newline
\linethickness{0.5mm} \line(1,0){425}
\newline
\textit{Description:}\\
\modeltype{DOUTPUT}

The digital output interface is modeled by time variable
resistances between the {\it Interface Node} and the {\it Low
Level Node} and between the the {\it Interface Node} and the {\it
High Level Node}. The variable resistances are shunted by fixed
capacitances. The parameters are controlled by parameters
specified in the model. The resistance varies exponentially from
the old state to the new state over the time period indicated for
the new state. This approximates the output of a digital gate.

\model{DOUTPUT}{Digital Output Interface Model}
\keyword{{\tt
FILE}  & digital output filename.
          If more than one model refers to the same file then the
                  filenames specified must be identical and not logicly
                  equivalent.  This ensures that the file is opened
                  only once.
        & - & \reqd \X
{\tt FORMAT}    & digital output file format    & - & 1 \X {\tt
TIMESTEP}& digital output file time step   & s & 1NS   \X {\tt
TIMESCALE}& digital output file time scale & s & 1     \X {\tt
CHGONLY}   & Output type flag: \newline
                  = 0 $\rightarrow$ output at each {\tt TIMESTEP} \newline
                  = 1 $\rightarrow$ output only on state change
        & - & 0 \X
{\tt CLOAD} & capacitance           & F & 0 \X {\tt RLOAD} &
resistance                    & $\Omega$& 1000\X {\tt S{\it
n}NAME}  & state ``n'' character abreviation\newline
      {\it n} = 0, 2, ..., or 19
    & - & \reqd \X
{\tt S{\it n}VLO}   & state ``n'' low level voltage\newline
      {\it n} = 0, 2, ..., or 19
    & s & \reqd \X
{\tt S{\it n}VHI}   & state ``n'' high level voltage\newline
      {\it n} = 0, 2, ..., or 19
    & s & \reqd \X
} The digital output interface is modeled by a resistance {\it
RLoad} and capacitance {\it CLoad} between the {\it InterfaceNode}
and the {\it Reference Node }. The values of {\it Rload} and {\it
CLoad} are specified in the model {\it ModelName}.

A state transition from state $n$ ($n$ = one of 0, 1, 2, ... 19)
is indicated if the interface voltage $V_{\mbox{\it
InterfaceNode}}$ $-$ $V_{\mbox{\it ReferenceNode}}$ between the
{\it InterfaceNode} and the {\it ReferenceNode} node is outside
the range {\tt S{\it n}VHI} $-$ {\tt S{\it n}VLO}. If there is a
state transistion then the valid voltage range of each state $k$
is considered in order from state k =  0 to state 19 to determine
which voltage range {\tt S{\it k}VHI} $-$ {\tt S{\it k}VLO}
brackets the current interface voltage $V_{\mbox{\it
InterfaceNode}}$ $-$ $V_{\mbox{\it ReferenceNode}}$. The first
valid state becomes the new state.  If there is no valid state
then the new state is indeterminate and designated by ``{\tt ?}''.
At each {\tt TIME} being a multiple integer of {\tt TIMESTEP} a
line is written to the digital output file {\it OutputFileName}.
If the new state at the time $t_i$ = $i \cdot {\tt TIMESTEP}$ is
$n$ then the $i$~th line is: \boxed{int($i \cdot {\tt TIMESCALE}$)
$n$} \\
where int() is the integer operation. An example of the
first few lines of {\it OutputFileName} with a {\tt TIMESTEP} of
1~ns and {\tt TIMESCALE} of 2 is:
\boxed{0.0 1\\
       2 0\\
       4 2\\
       6 3\\
       8 ?\\
      10 1\\
      12 0\\
      14 1
      }
\rm
\newline
\linethickness{0.5mm} \line(1,0){425}
\newline
\textit{Notes:}\\
There is no equivalent element in \FDA.
%%%%%%%%%%%%%%%%%%%%%%%%%%%%%%%%%%%%%%%%%%%%% Credits
\newline
\linethickness{0.5mm} \line(1,0){425}
\newline
\textit{Credits:}
\newline
\begin{tabular}{l l l l}
Name & Affiliation & Date & Links \\
Carlos E. Christofferson & NC State University & Sept 2000 & \epsfxsize=1in\pfig{logo.eps} \\
cechrist@ieee.org & & & www.ncsu.edu    \\
\end{tabular}
%\end{document}
