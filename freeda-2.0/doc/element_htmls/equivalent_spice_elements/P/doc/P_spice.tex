%\documentclass{article}
%\usepackage{epsf,html}
%\newcommand{\fig}[1]{J:/eos.ncsu.edu/users/m/mbs/mbs_group/figures/#1}
%\newcommand{\fig}[1]{../figures/#1}
%\newcommand{\pfig}[1]{\epsfbox{\fig{#1}}}
\oddsidemargin 10mm \topmargin 0.0in \textwidth 5.5in \textheight
7.375in \evensidemargin 1.0in \headheight 0.18in \footskip 0.16in
%%%%%%%%%%%%%%%%%%%%%%%%%%%%%%%%%%%%%%%% The title
\section[P \- Port Element]{\noindent{\LARGE \textbf{Port Element}}
\hspace{70mm}\huge\textbf{P}}
\linethickness{1mm}\line(1,0){425} \normalsize
%%%%%%%%%%%%%%%%%%%%%%%%%%%%%%%%%%%%%%%% the resistor figure
\begin{figure}[h]
\centerline{\epsfxsize=1.5in\pfig{p_spice.ps}} \caption{P --- port
element.\label{fig:port}}
\end{figure}
%\newline
%%%%%%%%%%%%%%%%%%%%%%%%%%%%%%%%%%%%%%%%%%% SPICE form
%\vspace{2mm}
\newline
\linethickness{0.5mm} \line(1,0){425}
\newline
\texttt{SPICE} \textit{Form:}
\newline
{\tt P}{\it name} $N_{+}$ $N_{-}$ {\tt PNR=} {\it PortNumber}
      \B{\tt ZL=} {\it ReferenceImpedance}\E
\newline
%%%%%%%%%%%%%%%%%%%%%%%%%%%%%%%%%%%%%%%%%%%%%%% explanation of terms in the SPICE form
\newline
\begin{tabular}{r l}
$N_{+}$ & is the positive element node,\\
$N_{-}$ & is the negative element node and \\
{\it PNR} & is the integer index of the port. The port index
must\\
& be numbered sequentially beginning at 1. That is, the first\\
& occurrence of a {\tt P} element in the input netlist must
have\\
& {\tt PNR=1} the second occurrence {\tt PNR=2}, etc.\\
             & (Units: none; Required; Symbol: $PortNumber$;)\\
{\it ZL} & is the reference impedance of port.\\
              & (Units: $\Omega$; Optional; Default: 50~$\Omega$; Symbol: $Z_L$;)\\
\end{tabular}
\newline
%%%%%%%%%%%%%%%%%%%%%%%%%%%%%%%%%%%%%%%%%%%%%%% Parameter table
%\vspace{4mm}
\linethickness{0.5mm} \line(1,0){425}
\newline
\textit{Example:}
\newline
\texttt{PORT1 1 0 PNR=1 ZL=75}
\newline
\linethickness{0.5mm} \line(1,0){425}
\newline
\textit{Description:}\\
As an example of using the port specification with a source,
consider the partial circuit in Fig. \ref{fig:port_source}. The
spice code defining this is\\
{\it {\tt P}name  $N_{+}$ $N_{-}$\ {\tt PNR=} PortNumber \B{\tt
ZL=}\ ReferenceImpedance\E\\ \B{\tt VIN $N_{-}$ 0 {\tt PULSE}
(Pulse Specification)\E}}
\begin{figure}[h]
\centering \ \pfig{pex.ps} \caption{Example of the usage of a P
element with a pulse voltage source. \label{fig:port_source}}
\end{figure}
\newline
\linethickness{0.5mm} \line(1,0){425}
\newline
\textit{Notes:}\\
There is no equivalent element in \FDA. $V_{AS}$ in Fig.
\ref{fig:port} is not visible to the user and is used by the
program to test for the S parameters.
%%%%%%%%%%%%%%%%%%%%%%%%%%%%%%%%%%%%%%%%%%%%% Credits
\newline
\linethickness{0.5mm} \line(1,0){425}
\newline
\textit{Credits:}
\newline
\begin{tabular}{l l l l}
Name & Affiliation & Date & Links \\
Carlos E. Christofferson & NC State University & Sept 2000 & \epsfxsize=1in\pfig{logo.eps} \\
cechrist@ieee.org & & & www.ncsu.edu    \\
\end{tabular}
%\end{document}
