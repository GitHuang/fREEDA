
\chapter{Output Control} \label{ch_output}

fREEDA$^{\mathrm{TM}}$ has an interpretive output language which uses a reverse
polish notation syntax.  The operators operate on a stack and as an
operation is performed zero or more arguements are consumed by an
opertor.  This is an extremely powerful way of controlling output.

\section{Output Commands}

{\tt .out write
( (<{\it qualifier}> <{\it value}>* )* <{\it operator}> )*
      in <{\it filename}>} \medskip
\\
or \medskip
\\
{\tt .out plot
( (<{\it qualifier}> <{\it value}>* )* <{\it operator}> )*
[[<{\it gnuplotPostambleScript}>] <{\it gnuplotPreambleScript}>]
in <{\it filename}>} \medskip
\\
or \medskip
\\
{\tt .out system <{\it string}>}

\subsection{Writing}

{\tt .out write
( (<{\it qualifier}> <{\it value}>* )* <{\it operator}> )*
      in <{\it filename}>} \medskip
\\
The {\tt write} command writes what is left on the stack into the file
{\it filename}.

\subsubsection{Example}

{\tt .out write term 4 vt in "4v.out"} \medskip
\\
This writes the time domain voltage at terminal 4 using the file
4v.out as an output file.

\subsection{Plotting}

{\tt .out plot
( (<{\it qualifier}> <{\it value}>* )* <{\it operator}> )*
[[<{\it gnuplotPostambleScript}>] <{\it gnuplotPreambleScript}>]
in <{\it filename}>} \medskip
\\
The {\tt plot} command writes what is left on the stack into the file
{\it filename} and initiates a plot. The file can be plotted
interactively using the fREEDA$^{\mathrm{TM}}$ Output Viewer. Also, a file named
{\tt<{\it filename}>.cmd} is created. This file is a gnuplot
\cite{gnuplot} script file that plots the desired data. The Scripts
are optional strings and are used to send additional commands to the
gnuplot program.

{\tt<{\it gnuplotPreambleScript}>} is a string of semicolon delineated
gnuplot commands prior to the plot command which is automatically
issued.

{\tt<{\it gnuplotPostambleScript}>} is a string of semicolon delineated
gnuplot commands after the plot command.

If the option {\tt gnuplot} is present in the {\tt .options} card, the
gnuplot program will be called automatically by fREEDA$^{\mathrm{TM}}$. Note that
this is generally not needed when using the Output Viewer.


\subsubsection{Example}

{\tt .out plot term 4 vt in "4v.out"} \medskip
\\
There are no script commands here. This plots the time domain voltage
at terminal 4 using the file 4v.out as an output file. This functions
as both a write and a plot.

\subsection{Running a System Command}

{\tt .out system <{\it string}>} \medskip
\\
Use this to run the string as a command to the operating system.

\subsubsection{Example}

{\tt .out system "echo Hello"} \medskip
\\
Prints ``Hello'' on the screen.

\section{Nomenclature}

The following nomenclature is used in describing the output operators.

\begin{tabular}{ll}
{\bf type} & {\bf description} \\
\multicolumn{2}{l}{\sl scalar numeric types} \\
\\
{\it i} & integer \\
{\it f} & floating-point \\
{\it r} & real (integer or floating-point) \\
{\it c} & complex \\
{\it s} & scalar (integer, floating-point or complex) \\
\\
\multicolumn{2}{l}{\sl scalar and mixed numeric types} \\
\\
{\it fv} & floating-point vector \\
{\it cv} & complex vector \\
{\it v} & floating-point or complex vector \\
{\it fsv} & floating-point scalar or vector \\
{\it csv} & complex scalar or vector \\
{\it sv} & scalar or vector (any) \\
{\it prom} & an appropriately-promoted numeric type \\
{\it -x} & (suffix to vector types) x data required \\
\\
\multicolumn{2}{l}{\sl other types} \\
\\
{\it any} & any type \\
{\it string} & character string \\
{\it var} & variable name \\
{\it file} & data file \\
{\it func} & function pointer
\end{tabular}

\section{Qualifiers}

\begin{tabular}{ll}
{\bf type} & {\bf description} \\
\\
\multicolumn{2}{l}{\sl qualifiers  (network types)} \\
\\
{\it term} & terminal (or node)\\
{\it element} & circuit element
\end{tabular}

\clearpage

\section{Operators}


\subsection{General Operators}

\begin{tabular}{p{.8in}p{2.5in}p{1.0in}p{.75in}}
{\bf operator} & {\bf function} & {\bf argument(s)} & {\bf result} \\
\\
\\
{\tt dup} & duplicate object & {\it any} & {\it same} \\
{\tt get} & get element of vector & {\it arg:v \newline index:i} & {\it s} \\
{\tt put} & modify element of vector & {\it arg:v \newline index:i \newline
  val:s} & {\it v} \\
{\tt stripx} & remove x data & {\it vx} & {\it v} \\
{\tt pack} & concatenates last {\it vx}'s on stack & variable number of {\it vx} & {\it m} \\
{\tt system} & execute shell command & {\it string} & {\it none}
\end{tabular}

\subsection{Network Operators}

\begin{tabular}{p{.8in}p{2.5in}p{1.0in}p{.75in}}
{\tt vf} & complex freq. domain voltage vector at a terminal &
    {\it term} & {\it cv} \\
{\tt if} & complex freq. domain current vector at a terminal &
    {\it term} & {\it cv} \\
{\tt xf} & complex freq. domain state variable vector at a terminal &
    {\it term} & {\it cv} \\
{\tt vt} & time domain voltage vector at a terminal &
    {\it term} & {\it fv} \\
{\tt it} & time domain current vector at a terminal &
    {\it term} & {\it fv} \\
{\tt ut} & time domain voltage vector at an element port &
    {\it elem} & {\it fv} \\
\end{tabular}

\subsection{RPN Arithmetic Operators}
Arithmetic Operators for reverse polish notation
e.g. 3 4 add = 7

\begin{tabular}{p{.8in}p{2.5in}p{1.0in}p{.75in}}
{\tt add} & addition & {\it sv \newline sv} & {\it prom} \\
{\tt sub} & subtraction & {\it sv \newline sv} & {\it prom} \\
{\tt mult} & multiplication & {\it sv \newline sv} & {\it prom} \\
{\tt div} & division & {\it sv \newline sv} & {\it prom} \\
{\tt real} & real part & {\it csv} & {\it fsv} \\
{\tt imag} & imaginary part & {\it csv} & {\it fsv} \\
{\tt mag} & magnitude & {\it csv} & {\it fsv} \\
{\tt abs} & absolute value or magnitude & {\it sv} & {\it fsv} \\
{\tt contphase} & continuous phase & {\it csv} & {\it fsv} \\
{\tt prinphase} & principal value phase & {\it csv} & {\it fsv} \\
{\tt conj} & complex conjugate & {\it csv} & {\it csv} \\
{\tt neg} & additive inverse (negative) & {\it sv} & {\it sv} \\
{\tt recip} & reciprocal & {\it sv} & {\it sv}
\end{tabular}

% \subsection{Conventional Arithmetic Operators}
% Conventional Arithmetic Operators
% e.g. 3 + 4 = 7
% Not fully implemented. Do not use and reserved for future expansion.
%
% \begin{tabular}{p{.8in}p{2.5in}p{1.0in}p{.75in}}
% {\verb:+:} & addition & {\it sv \newline sv} & {\it prom} \\
% {\verb:-:} & subtraction & {\it sv \newline sv} & {\it prom} \\
% {\verb:*:} & multiplication & {\it sv \newline sv} & {\it prom} \\
% {\verb:/:} & division & {\it sv \newline sv} & {\it prom} \\
% {\verb:^:} & poer & {\it sv \newline sv} & {\it prom} \\
% {\verb:**:} & poer & {\it sv \newline sv} & {\it prom} \\
% \end{tabular}

\subsection{Mathematical Operators}

\begin{tabular}{p{.8in}p{2.5in}p{1.0in}p{.75in}}
{\tt db} & dB ($20 \log_{10}$) & {\it sv} & {\it fsv} \\
{\tt db10} & dB applied to power ($10 \log_{10}$) & {\it sv}
    & {\it fsv} \\
{\tt rad2deg} & convert radians to degrees & {\it fsv} & {\it fsv} \\
{\tt deg2rad} & convert degrees to radians & {\it fsv} & {\it fsv} \\
{\tt minlmt} & limit the minimum value & {\it arg:fsv \newline min:f}
    & {\it fsv} \\
{\tt maxlmt} & limit the maximum value & {\it arg:fsv \newline max:f}
    & {\it fsv} \\
{\tt diff} & differences & {\it fsv} & {\it fsv} \\
{\tt deriv} & derivative & {\it fsv} & {\it fsv} \\
{\tt sum} & sums & {\it fsv} & {\it fsv} \\
{\tt integ} & integral & {\it fsv} & {\it fsv}
\end{tabular}

\clearpage

\subsection{Signal Processing Operators}

\begin{tabular}{p{.8in}p{2.5in}p{1.0in}p{.75in}}
{\tt smpltime} & current analysis timebase as x {\em and} y of result &
  {\it none} & {\it fv} \\
{\tt sweepfrq} & current analysis sweep frequencies as x {\em and}
  y of result & {\it none} & {\it fv} \\
{\tt smplcvt} & interpolate {\em signal1} over timebase of {\em signal2} &
  {\it signal1:v \newline signal2:vx} & {\it vx} \\
{\tt sweepcvt} & interpolate {\em frq1} over sweep frequencies of {\em frq2} &
  {\it frq1:v \newline frq2:vx} & {\it vx} \\
{\tt maketime} & create timebase starting at $t=0$ in x {\em and} y of result &
  {\it tmax:r \newline pts:i} & {\it vx} \\
{\tt makesweep} & create sweep frequencies starting at $f=0$ in x
  {\em and} y of result & {\it fmax:r \newline pts:i} & {\it vx} \\
{\tt fft} & FFT (argument should have $2^{k}$ points) &
  {\it timedata:fv} & {\it cv} \\
{\tt invfft} & inverse FFT (argument should have $2^{k}-1$ points) &
  {\it frqdata:cv} & {\it fv} \\
{\tt cconv} & real circular (FFT) convolution with zero padding &
  {\it signal1:fv \newline signal2:fv} & {\it fv} \\
{\tt upcconv} & unpadded real circular (FFT) convolution &
  {\it signal1:fv \newline signal2:fv} & {\it fv} \\
{\tt sconv} & slow (time-domain) real convolution &
  {\it signal1:fv \newline signal2:fv} & {\it fv} \\
{\tt fconv} & fast (approximate) real convolution &
  {\it signal1:fv \newline signal2:fv} & {\it fv} \\
{\tt lpbwfrq} & lowpass Butterworth filter frequency response &
  {\it frqvec:vx \newline corner:f \newline order:i} & {\it cvx}
\end{tabular}

