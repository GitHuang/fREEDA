\chapter[fREEDA Commands]{fREEDA Commands\label{chap:dotted_commands}}

\section[.inc \- Include Statement]{.inc \- Include Statement}

\noindent The {\tt .inc} statement is an efficient way to include subcircuits and common netlist code.

\form{ {\tt .lib} \B FileName \E}\\
{\it Filename} is the name of the file to be read.

\note{
\item   It must contain only {\tt .model} statements, subcircuit
        definitions (between {\tt .subckt} and {\tt .ends} statements),
        and {\tt .lib} statements.

\item The include file {\it Filename} is searched in the current directory.
}

\offsetenumerate{
\item[] {\tt .INC} and
      the {\tt .LIB} function similarly with the exception that .LIB searches for the file in a specific directory while .INC searches for the file in current directory.}

\clearpage
\section[.lib \- Library Statement]{.lib \- Library Statement}

\noindent The {\tt .lib} statement is an efficient way to include
{\tt .model} statements and subcircuits.

\form{ {\tt .lib} \B FileName \E}\\
{\it Filename} is the name of the library file.

\note{
\item   It must contain only {\tt .model} statements, subcircuit
        definitions (between {\tt .subckt} and {\tt .ends} statements),
        and {\tt .lib} statements.

\item The library file {\it Filename} is searched in the directory pointed to by the environment variable  specified by the environment variable FREEDA\_LIBRARY .
}

\offsetenumerate{
\item[] Libraries could be included using either the {\tt .INC} statement or by
      the {\tt .LIB} statement.    {\tt .INC} and
      the {\tt .LIB} function similarly with the exception that .LIB searches for the file in a specific directory while .INC searches for the file in current directory.}

\clearpage
\section{.locate \- Identify Location of Terminals} \noindent\textit{Form:}
\newline
{\tt .locate $\langle$ \tt{terminal\ name}$\rangle$ $\langle$
\tt{X}$\rangle$ $\langle$ \tt{Y}$\rangle$} \\[0.2in]
\newline
\textit{Description:}\\
.LOCATE is used to identify the position of a terminal.\\[0.2in]

\textit{Credits:}\\[0.1in]
\begin{tabular}{l l l l}
Name & Affiliation & Date & Links \\
Michael Steer & NC State University & Sept 2000 & \epsfxsize=1in\pfig{logo.eps} \\
m.b.steer@ieee.org & & & www.ncsu.edu    \\
\end{tabular}

\clearpage
\section[.plot \- Plot Specification]{.plot \- Plot Specification}

NOT FULLY FUNCTIONAL as of \FDA1.3

The plot specification controls the information that is plotted as
the result of various analyses.\\

\noindent\textit{Form:} \notforsspice{ \form{{\tt .PLOT TRAN}
OutputSpecification \B PlotLimits\E
     \\{\tt +} \B OutputSpecification \B PlotLimits\E $\ldots$ \E\\[0.1in]
      {\tt .PLOT AC   } OutputSpecification \B PlotLimits\E
       \\{\tt +} \B OutputSpecification \B PlotLimits\E $\ldots$ \E\\[0.1in]
      {\tt .PLOT DC   } OutputSpecification  \B PlotLimits\E
       \\{\tt +} \B OutputSpecification \B PlotLimits\E $\ldots$ \E\\[0.1in]
      {\tt .PLOT NOISE} NoiseOutputSpecification
            \B {\tt (}DistortionReportType{\tt )}\E \B PlotLimits\E
       \\{\tt +} \B NoiseOutputSpecification
        \B {\tt (}DistortionReportType{\tt )}\E \B PlotLimits\E\\[0.1in]
      {\tt .PLOT DISTO} DistortionOutputSpecification
           \B {\tt (}DistortionReportType{\tt )}\E \B PlotLimits\E
           \\{\tt +} \B DistortionOutputSpecification
           \B {\tt (}DistortionReportType{\tt )}\E \B PlotLimits\E
              $\ldots$ \E
        \B {\tt (}DistortionReportType{\tt )}\E \B PlotLimits\E\\[0.1in]
     }}

\begin{widelist}
\item[{\tt tran}] is the keyword specifying that this {\tt .plot} statement
     controls the reporting of results of a transient analysis initiated by the
     {\tt .TRAN} statement.

\item[{\tt ac}] is the keyword specifying that this {\tt .plot} statement
     controls the reporting of results of a small-signal \ac\ analysis initiated by
     the {\tt .ac} statement.

\item[{\tt dc}] is the keyword specifying that this {\tt .plot} statement
     controls the reporting of results of a \dc\ analysis initiated by the
     {\tt .dc} statement.

\item[{\tt noise}] is the keyword specifying that this {\tt .plot} statement
     controls the reporting of results of a noise analysis initiated by the
     {\tt .noise} statement.

\sspiceonly{
\item[{\tt as}] is the keyword specifying that this {\tt .plot} statement
     controls the reporting of results of a scattering parameter analysis.
     }
\end{widelist}

\begin{widelist}
\item[{\it OutputSpecification}] specifies the voltage or current to be plotted
     against the sweep variable. The sweep variable is dependent on the type
     of analysis.\\[0.1in]

     \notforsspice{
     \underline{Voltages} may be specified as an absolute voltage at a terminal:
     {\tt V({\it TerminalName})} or
     the voltage at one terminal with respect to that at another terminal,\newline
     e.g. {\tt V(Terminal1Name,Terminal2Name)}.

     For the reporting of the results of an \ac\ analysis the following outputs
     can be specified by replacing the {\tt V} as follows:\\
      \offset\begin{tabular}{lcp{3.5in}}
      {\tt VR} &-&real part\\
      {\tt VI} &-&imaginary part\\
      {\tt VM} &-&magnitude\\
      {\tt VP} &-&phase\\
      {\tt VDB} &-&$10\,\mbox{log}(10\,magnitude)$
      \end{tabular}\\
      In \ac\  analysis the default is {\tt VM} for magnitude.\\[0.1in]

     \underline{Currents} are specified by referencing the name of the voltage
     source through which the current is measured, e.g.
     {\tt I(V{\it VoltageSourceName})}.

     For the reporting of the results of an \ac\ analysis the following outputs
     can be specified by replacing the {\tt I} as follows:\\
      \offset\begin{tabular}{lcp{3.5in}}
      {\tt IR} &-&real part\\
      {\tt II} &-&imaginary part\\
      {\tt IM} &-&magnitude\\
      {\tt IP} &-&phase\\
      {\tt IDB} &-&$10\,\mbox{log}(10\,magnitude)$
      \end{tabular}\\
      In \ac\  analysis the default is {\tt IM} for magnitude.
      }

\item[{\it PlotLimits}] are optional and can be placed after any output
     specification. {\it PlotLimits} has the form
     {\tt (}{\it LowerLimit}{\tt ,}{\it UpperLimit}{\tt )} .
     All quantities will be plotted using the same
     {\it PlotLimits}.  The default is to automatically scale the plot and
     perhaps use different scales for each of the quantities to be plotted.
\end{widelist}

\begin{widelist}
\item[{\it NoiseOutputSpecification}]
     specifies the noise measure to be reported. The two options are
     {\tt ONOISE} which reports the output noise and
     {\tt INOISE} which reports the equivalent input noise.
     See the {\tt .NOISE} statement
     for a detailed explanation.

     It  must be one of the following:\\
      \offset\begin{tabular}{lcp{3.5in}}
      {\tt ONOISE} &-&magnitude of the output noise\\
      {\tt DB(ONOISE)} &-&output noise in dB\\
      {\tt INOISE} &-&magnitude of the equivalent input noise\\
      {\tt DB(INOISE)} &-&equivalent input noise in dB\\
      {\tt GAIN} &-&voltage gain\\
      {\tt DB(GAIN)} &-&voltage gain in dB (= 20 log({\tt GAIN})\\
      {\tt GT} &-&transducer gain\\
      {\tt DB(GT)} &-&transducer gain in dB (= 10 log({\tt GT})\\
      {\tt NF} &-&spot noise factor\\
      {\tt DB(NF)} &-& spot noise figure (= 10 log({\tt NF})\\
      {\tt SNR} &-&output signal-to-noise ratio\\
      {\tt DB(SNR)} &-&output signal-to-noise ratio in dB (= 20 log({\tt SNR})\\
      {\tt TNOISE} &-&output noise temperature.
      \end{tabular}
\end{widelist}

\begin{widelist}

\item[{\it SParameterOutputSpecification}]
      specifies the S-parameter output variables that are to be printed.
      Each variable must have one of the following forms:\\
      \offset\begin{tabular}{lcp{3.5in}}
      {\tt S(i,j)} &-&Magnitude of $S_{ij}$\\
      {\tt SR(i,j)} &-&Real part of $S_{ij}$\\
      {\tt SI(i,j)} &-&Imaginary part of $S_{ij}$\\
      {\tt SP(i,j)} &-&Phase of $S_{ij}$ in degrees\\
      {\tt SDB(i,j)} &-&Magnitude of $S_{ij}$ in dB
                         (= 20 log({\tt {\tt S(i,j)}}))\\
      {\tt SG(i,j)} &-&Group delay of $S_{ij}$
      \end{tabular}\\
      The port numbers are $i,j$  which are specified using the {\tt PNR} keywor
      when the port (`P') element is specified.

\notforsspice{
\item[{\it DistortionOutputSpecification}]
     specifies the distortion component to be reported
     in a tabular format of up to 8 columns plus an initial column with the sweep
     variable.  The {\it DistortionOutputSpecification} is one of the
     following:\\
      \offset\begin{tabular}{lcp{3.5in}}
      {\tt HD2} &-&the second harmonic distortion\\
      {\tt HD3} &-&the second harmonic distortion\\
      {\tt SIM2} &-&the sum frequency intermodulation component\\
      {\tt DIM2} &-&the difference frequency intermodulation component\\
      {\tt DIM3} &-&the third order intermodulation component
      \end{tabular}\\
      See the {\tt .DISTO} statement for
      a description of these distortion components.

\item[{\it DistortionReportType }]
     specifies the format for reporting the distortion components.
     It  must be one of the
     following:\\
      \offset\begin{tabular}{lcp{3.5in}}
      R &-&real part\\
      I &-&imaginary part\\
      M &-&magnitude\\
      P &-&phase\\
      DB &-&$10\,\mbox{log}(10\,magnitude)$
      \end{tabular}\\
      The default is {\tt M} for magnitude.
}
\end{widelist}

\note{
\item There can be any number of {\tt .PLOT} statements.

\item All of the output quantities specified on a single {\tt .PLOT} statement
will be plotted on the same graph using ASCII characters. An
overlap will be indicated by the letter {\tt X}.  The plot
produced by the {\tt .PLOT} statement is a line printer plot.
While plotting is primitive it can be plotted on any printer and
is incorporated in the output log file.

\item The plot output of the results of an \ac\ analysis always have a
logarithmic vertical scale.}

\clearpage
\section[.print \- Print Specification]{.print \- Print Specification}

NOT FULLY FUNCTIONAL as of \FDA1.3

\noindent\textit{Form:}
\newline
The print specification controls the information that is reported
as the result of various analyses.

\form{{\tt .print TRAN } OutputSpecification \B
OutputSpecification $\ldots$
      \E\\[0.1in]
      {\tt .print AC   } OutputSpecification \B OutputSpecification $\ldots$
      \E\\[0.1in]
      {\tt .print DC   } OutputSpecification \B OutputSpecification $\ldots$
      \E\\[0.1in]
\sspiceonly{
      {\tt .print NOISE} NoiseOutputSpecification
              \B NoiseOutputSpecification $\ldots$ \E\\[0.1in]
      {\tt .print AS} SParameterOutputSpecification
              \B\B SParameterOutputSpecification\E $\ldots$ \E\\[0.1in]
              }
    \notforsspice{
      {\tt .print DISTO} DistortionOutputSpecification
              {\tt ( }DistortionReportType{ \tt )}\\{\tt +}
              \B DistortionOutputSpecification
              {\tt ( }DistortionReportType{ \tt )} $\ldots$ \E}
     }

\begin{widelist}
\item[{\tt TRAN}] is the keyword specifying that this {\tt .print} statement
     controls the reporting of results of a transient analysis initiated by the
     {\tt .TRAN} statement.

\item[{\tt AC}] is the keyword specifying that this {\tt .print} statement
     controls the reporting of results of a small-signal \ac\ analysis initiated by
     the {\tt .AC} statement.

\item[{\tt DC}] is the keyword specifying that this {\tt .print} statement
     controls the reporting of results of a \dc\ analysis initiated by the
     {\tt .DC} statement.

\item[{\tt NOISE}] is the keyword specifying that this {\tt .print} statement
     controls the reporting of results of a noise analysis initiated by the
     {\tt .NOISE} statement.

\sspiceonly{
\item[{\tt AS}] is the keyword specifying that this {\tt .print} statement
     controls the reporting of results of a scattering parameter analysis.
     }

\notforsspice{
\item[{\tt DISTO}] is the keyword specifying that this {\tt .print} statement
     controls the reporting of results of a small-signal \ac\ distortion analysis
     initiated by the {\tt .DISTO} statement.
     }

\item[{\it OutputSpecification}] specifies the voltage or current to be reported
     in a tabular format of up to 8 columns plus an initial column with the
     sweep variable.\\[0.1in]

     \underline{Voltages} may be specified as an absolute voltage at a node:
     {\tt V({\it TerminalName})} or
     the voltage at one node with respect to that at another node,\newline
     e.g. {\tt V(Terminal1Name,Terminal2Name)}.

     For the reporting of the results of an \ac\ analysis the following outputs
     can be specified by replacing the {\tt V} as follows:\\
      \offset\begin{tabular}{lcp{3.5in}}
      {\tt VR} &-&real part\\
      {\tt VI} &-&imaginary part\\
      {\tt VM} &-&magnitude\\
      {\tt VP} &-&phase\\
      {\tt VDB} &-&$10\,\mbox{log}(10\,magnitude)$
      \end{tabular}\\
      In \ac\  analysis the default is {\tt VM} for magnitude.\\[0.1in]

     \underline{Currents} are specified by referencing the name of the voltage
     source through which the current is measured, e.g.
     {\tt I(V{\it VoltageSourceName})}.

     For the reporting of the results of an \ac\ analysis the following outputs
     can be specified by replacing the {\tt I} as follows:\\
      \offset\begin{tabular}{lcp{3.5in}}
      {\tt IR} &-&real part\\
      {\tt II} &-&imaginary part\\
      {\tt IM} &-&magnitude\\
      {\tt IP} &-&phase\\
      {\tt IDB} &-&$10\,\mbox{log}(10\,magnitude)$
      \end{tabular}\\
      In \ac\  analysis the default is {\tt IM} for magnitude.

\item[{\it NoiseOutputSpecification}]
     specifies the noise measure to be reported. The two options are
     {\tt ONOISE} which reports the output noise and
     {\tt INOISE} which reports the equivalent input noise.
     See the {\tt .NOISE} statement
     for a detailed explanation.

It  must be one of the following:\\
{\tt ONOISE} - RMS output noise voltage\\
{\tt DB(ONOISE)} - output noise voltage in dB (= 20 log({\tt ONOISE})\\
{\tt INOISE} - RMS equivalent input noise voltage\\
{\tt DB(INOISE)} - equivalent input noise voltage in dB (= 20 log({\tt INOISE})\\
{\tt GAIN} - voltage gain\\
{\tt DB(GAIN)} - voltage gain in dB (= 20 log({\tt GAIN})\\
{\tt GT} - transducer gain\\
{\tt DB(GT)} - transducer gain in dB (= 10 log({\tt GT})\\
{\tt NF} - spot noise factor\\
{\tt DB(NF)} - spot noise figure (= 10 log({\tt NF})\\
{\tt SNR} - output signal-to-noise ratio\\
{\tt DB(SNR)} - output signal-to-noise ratio in dB (= 20 log({\tt SNR})\\
{\tt TNOISE} - output noise temperature.

\item[{\it SParameterOutputSpecification}] specifies the S-parameter output variables that are to be printed.
Each variable must have one of the following forms:\\
{\tt S(i,j)} - Magnitude of $S_{ij}$\\
{\tt SR(i,j)} - Real part of $S_{ij}$\\
{\tt SI(i,j)} - Imaginary part of $S_{ij}$\\
{\tt SP(i,j)} -Phase of $S_{ij}$ in degrees\\
{\tt SDB(i,j)} - Magnitude of $S_{ij}$ in dB
(= 20 log({\tt {\tt S(i,j)}}))\\
{\tt SG(i,j)} - Group delay of $S_{ij}$\\

The port numbers are $i,j$  which are specified using the {\tt
PNR} keyword when the port element is specified.
\end{widelist}

\notforsspice{
\begin{widelist}
\item[{\it DistortionOutputSpecification}]
     specifies the distortion component to be reported
     in a tabular format of up to 8 columns plus an initial column with the sweep
     variable.  The {\it DistortionOutputSpecification} is one of the
     following:\\
      \offset\begin{tabular}{lcp{3.5in}}
      {\tt HD2} &-&the second harmonic distortion\\
      {\tt HD3} &-&the second harmonic distortion\\
      {\tt SIM2} &-&the sum frequency intermodulation component\\
      {\tt DIM2} &-&the difference frequency intermodulation component\\
      {\tt DIM3} &-&the third order intermodulation component
      \end{tabular}\\
      See the {\tt .DISTO} statement for
      a description of these distortion components.

\item[{\it DistortionReportType }]
     specifies the format for reporting the distortion components.
     It  must be one of the
     following:\\
      \offset\begin{tabular}{lcp{3.5in}}
      R &-&real part\\
      I &-&imaginary part\\
      M &-&magnitude\\
      P &-&phase\\
      DB &-&$10\,\mbox{log}(10\,magnitude)$
      \end{tabular}\\
      The default is {\tt M} for magnitude.
\end{widelist}
}

\note{
\item There can be any number of {\tt .print} statements.

\item The number of significant digits of the results reported is
{\tt NUMDGT} which is set in a {\tt .options} statement.}

\underline{{\tt dc} and {\tt tran} Reporting}\\
The output specifications available for the \dc\ sweep and
transient analyses are

\begin{widelist}

\item[{\tt I(}{\it DeviceName}{\tt )}]
Current through a two terminal device (such as a resistor {\tt R}
element) or the output of a controlled voltage or current source.
e.g. {\tt I(R22)} is the current flowing through resistor {\tt
R22} from node $N_1$ to $N_2$ of {\tt R22}.

\item[{\tt I}{$\,$TerminalName}{\tt (}{\it DeviceName}{\tt )}]
Current flowing into terminal named {\it TerminalName} (such as
{\tt B} for gate) from the device named {\it DeviceName} (such as
{\tt Q12}). e.g. {\tt IB(Q12)}

\item[{\tt I}{$\,$PortName}{\tt (}{\it TransmissionLineName}{\tt )}]
Current at port named {\it PortName} (either {\tt A} or {\tt B})
of the transmission line device named   {\it TransmissionLineName}

\item[{\tt V(}{\it TerminalName}{\tt )}]
Voltage at a node of name {\it TerminalName}.

\item[{\tt V(}{$n_1,n_2$}{\tt )}]
Voltage at node $n_1$ with respect to the voltage
 at node $n_2$.

\item[{\tt V(}{\it DeviceName}{\tt )}]
Voltage across a two terminal device (such as a resistor {\tt R}
element) or at the output of a controlled voltage or current
source.

\item[{\tt V}{$\,$TerminalName}{\tt (}{\it DeviceName}{\tt )}]
Voltage at terminal named {\it TerminalName} (such as {\tt G} for
gate) of the device named   {\it DeviceName} (such as {\tt M12}).
e.g. {\tt VG(M12)}

\item[{\tt V}{$\,$TerminalName1$\,$TerminalName2}{\tt (}{\it DeviceName}{\tt )}]
Voltage at terminal named {\it TerminalName1} (such as {\tt G} for
gate) th respect to the terminal name {\it TerminalName2} (such as
{\tt S} for source) of the device named   {\it DeviceName} (such
as {\tt M12}). e.g. {\tt VGS(M12)}

\item[{\tt V}{$\,$PortName}{\tt (}{\it TransmissionLineName}{\tt )}]
Voltage at port named {\it PortName} (either {\tt A} or {\tt B})
of the transmission line device named   {\it TransmissionLineName}
(such as {\tt T5}). e.g. {\tt VA(M5)}

\end{widelist}

\hspace*{\fill}\\[0.1in]

\noindent\underline{Two Terminal Device Types Supported for \dc
and
Transient Analysis Reporting}\\[0.1in]
The single character identifier  for the following elements as
well as the rest of the device name can be used as the {\it
DeviceName} in the {\tt I(}{\it DeviceName}{\tt )}  and {\tt
I(}{\it DeviceName}{\tt )}
output specifications.\\
\hspace*{\fill}
\begin{tabular}{|p{1in}|p{3in}|}
\hline
Element Type & Description\\
\hline
{\tt C} & capacitor\\
{\tt D} & diode\\
{\tt E} & voltage-controlled voltage source\\
{\tt F} & current-controlled current source\\
{\tt G} & voltage-controlled current source\\
{\tt H} & current-controlled voltage source\\
{\tt I} & independent current source\\
{\tt L} & inductor\\
{\tt R} & resistor\\
{\tt V} & independent voltage source\\
\hline
\end{tabular}\\[0.1in]

\noindent \underline{Multi-Terminal Device Types Supported for \dc
and Transient Analysis Reporting}\\[0.1in]
The single character identifier for the following elements as well
as the rest of the device name can be used as the {\it DeviceName}
in the {\tt I}{$\,$TerminalName}{\tt (}{\it DeviceName}{\tt )},
{\tt V}{$\,$TerminalName}{\tt (}{\it DeviceName}{\tt )} and {\tt
V}{$\,$TerminalName1$\,$TerminalName2}{\tt (}{\it DeviceName}{\tt
)}
output specifications.\\[0.1in]
\hspace*{\fill}
\begin{tabular}{|p{1in}|p{3in}|}
\hline
Element Type & Description\\
\hline {\tt B} & GaAs MESFET\ \ \ Terminals:\newline
\hspace*{1in}D --- drain\newline \hspace*{1in}G --- gate\newline
\hspace*{1in}S --- source\\
{\tt J} & JFET\ \ \ Terminals:\newline \hspace*{1in}D ---
drain\newline \hspace*{1in}G --- gate\newline
\hspace*{1in}S --- source\\
{\tt M} & MOSFET\ \ \ Terminals:\newline \hspace*{1in}B --- bulk
or substrate\newline \hspace*{1in}D --- drain\newline
\hspace*{1in}G --- gate\newline
\hspace*{1in}S --- source\\
{\tt Q} & BJT\ \ \ Terminals\newline \hspace*{1in}C ---
collector\newline \hspace*{1in}B --- base\newline \hspace*{1in}E
--- emitter\newline
\hspace*{1in}S --- source\\
\hline
\end{tabular}\\[0.1in]

\noindent
\underline{{\tt AC} Reporting}\\
The output specifications available for reporting the results of
an \ac\ frequency sweep analysis includes all of the specification
formats discussed above for \dc\ and transient analysis together
with a number of possible
suffixes:\\
      \offset\begin{tabular}{lcp{3.5in}}
      {\tt DB} &-&$10\,\mbox{log}(10\,magnitude)$\\
      {\tt M} &-&magnitude\\
      {\tt P} &-&phase\\
      {\tt R} &-&real part\\
      {\tt I} &-&imaginary part\\
      {\tt G} &-&group delay = ${\partial \phi/ \partial f}$\newline
                 where $\phi$ is the phase of the quantity being reported
         and $f$ is the analysis frequency.
      \end{tabular}\\
In \ac\  analysis the default suffix is {\tt M} for magnitude.\\[0.1in]

\noindent\underline{Two-Terminal Device Types Supported for
\ac\ Reporting}\\[0.1in]

The single character identifier  for the following elements as
well as the rest of the device name can be used as the {\it
DeviceName} in the {\tt I(}{\it DeviceName}{\tt )}  and {\tt
I(}{\it DeviceName}{\tt )}
output specifications.\\
\hspace*{\fill}
\begin{tabular}{|p{1in}|p{3in}|}
\hline
Element Type & Description\\
\hline
{\tt C} & capacitor\\
{\tt D} & diode\\
{\tt I} & independent current source\\
{\tt L} & inductor\\
{\tt R} & resistor\\
{\tt V} & independent voltage source\\
\hline
\end{tabular}\\[0.1in]

\noindent \underline{Multi-Terminal Device Types Supported for \dc
and Transient Analysis
Reporting}\\[0.1in]
The single character identifier for the following elements as well
as the rest of the device name can be used as the {\it DeviceName}
in the {\tt I}{$\,$TerminalName}{\tt (}{\it DeviceName}{\tt )},
{\tt V}{$\,$TerminalName}{\tt (}{\it DeviceName}{\tt )} and {\tt
V}{$\,$TerminalName1$\,$TerminalName2}{\tt (}{\it DeviceName}{\tt
)}
output specifications.\\[0.1in]
\hspace*{\fill}
\begin{tabular}{|p{1in}|p{3in}|}
\hline
Element Type & Description\\
\hline {\tt B} & GaAs MESFET\ \ \ Terminals:\newline
\hspace*{1in}D --- drain\newline \hspace*{1in}G --- gate\newline
\hspace*{1in}S --- source\\
\hline {\tt J} & JFET\ \ \ Terminals:\newline \hspace*{1in}D ---
drain\newline \hspace*{1in}G --- gate\newline
\hspace*{1in}S --- source\\
\hline {\tt M} & MOSFET\ \ \ Terminals:\newline \hspace*{1in}B ---
bulk or substrate\newline \hspace*{1in}D --- drain\newline
\hspace*{1in}G --- gate\newline
\hspace*{1in}S --- source\\
\hline {\tt Q} & BJT\ \ \ Terminals\newline \hspace*{1in}C ---
collector\newline \hspace*{1in}B --- base\newline \hspace*{1in}E
--- emitter\newline
\hspace*{1in}S --- source\\
\hline
\end{tabular}

\linethickness{0.5mm} \line(1,0){425}
\newline
\textit{Credits:}\\[0.1in]
\begin{tabular}{l l l l}
Name & Affiliation & Date & Links \\
Michael Steer & NC State University & Sept 2000 & \epsfxsize=1in\pfig{logo.eps} \\
m.b.steer@ieee.org & & & www.ncsu.edu    \\
\end{tabular}

\clearpage
\section{Structure of a fREEDA Netlist}

There are four types of elements used in TRANSIM: \footnote{
element: a model of a physical component of a network.} nodes,
edges, edge coupling groups (ECGs) and node coupling groups
(NCGs).  Within those broad classifications there are a wide
variety of individual element types, for example, ``mlin''
(microstrip line), ``coax'' (coaxial cable), and ``idealj'' (ideal
junction).  ``element'' and ``model'' are used synonomously.

\subsection{Lexical Rules}

A lexical rule defines an identifiable object in the input file.
That is it defines the equivalent of words.  Words put togetehr in
a particular order define a grammar. fREEDA recognizes many "words"
but the important one are as follows.

\begin{tabular}{p{1.5in}p{3in}}
{\bf whitespace}& a blank\\
                &  a newline followed immediately by a + sign.\\
                &  a tab\\
                &  a vertical tab\\
                &  a newpage
\end{tabular}

\begin{tabular}{p{1.5in}p{3in}}
{\bf identifier} & A character sequence beginning with an
alphabetic character \[A-Za-z\]
\end{tabular}

\begin{tabular}{p{1.5in}p{3in}}
{\bf variables} & A variable must begin with an alphabetic
character or a \$
                followed by alphanumeric characters or `\_' or
                `.'\\
                & Example: \\
                & \indent {\tt HEIGHT}\\
                & \indent {\tt \$height}\\
                & \indent {\tt height.1\_1}\\
                & Note that {\tt HEIGHT} and {\tt height} are identical as
                case is not preserved.
\end{tabular}

\begin{tabular}{p{1.5in}p{3in}}
{\bf strings} & Either as an identifier (a continuous sequence of
alphanumeric characters) or enclosed within double quotes. \\
               &The following special escaped characters are allowed in
               strings defined within double quotes.\\
                & {\tt " } To include a double quote in a string.\\
                & {\tt $\backslash$n } To indicate a newline \\
               & Examples:\\
               & \indent {\tt gate}\\
               & \indent {\tt "VOLTAGE WAVEFORM"}\\
               & Note: Strings may continue across lines using the
               continuation syntax:\\
               & {\tt "VOLTAGE}\\
               & {\tt + WAVEFORM"}\\
               & or simply by continuing across a line as in\\
               & {\tt "VOLTAGE}\\
               & {\tt WAVEFORM"}
\end{tabular}

\begin{tabular}{p{1.5in}p{3in}}
{\bf numbers}] & ``E'' or ``e'' to indicate exponent.
\end{tabular}

\begin{tabular}{p{1.5in}p{3in}}
{\bf dotted command} & A ``.'' followed by alphabetic characters
at the beginning of a line.
\end{tabular}

\begin{tabular}{p{1.5in}p{3in}}
{\bf lf}& A line feed or carriage return.
\end{tabular}

\subsubsection{Capitalization}

The case of identifiers and keywords is ignored in TRANSIM
netlists. The significance of case is retained only within quoted
strings, and in that case it is always retained. Internally
characters are mapped to lower case.

\clearpage
\section{SPICE Elements}

All regular SPICE elements have the same syntax as in standard
SPICE but with the following additions.
\begin{enumerate}
\item A {\tt .model} specification is allowed for all elements.
\item Anything that can appear in a {\tt .model} specification can be
      included in the specification of the element.
\item If a parameter is not specified either through an  element
specification or a {\tt.model} specification then the default
parameters for that model will apply to this element.
\end{enumerate}

{\tt <term\_id>} is either an integer or a string in double
quotes, and is the name of a terminal in the network.  {\tt
<term\_id\_list>} is a list of one or more terminal id's separated
by white space.

\section{General File Comments}
The first line of an input file is used as the identifier string
and is associated with various output files to identify their
origin. It is seen strictly as a text string and no processing is
done on it. If a particular statement won't fit on a single line,
it may be continued by placing a ``+'' at the beginning of each
additional line.  All comments are proceeded by an ``*'' (an
asterisk) and there is no limit to the number of comment lines
used in a file. A comment may begin anywhere on a line (such as
after a statement) and any text after the asterisk is ignored by
the parser.

\section{Element Instance Syntax}

Each instance of an element in TRANSIM netlist is declared in the
same manner with each declaration existing on a separate line.
The syntax is:
\begin{tt}
\begin{verbatim}
        element:instance_id term1 term2....  model = "identifier"
\end{verbatim}
\end{tt}
The terms {\em element} and {\em identifier} are the same as those
used in the description of the {\em .model} statement and {\em
instance\_id} is a unique string that identifies this instance of
{\em identifier}. {\em term1}, {\em term2}, etc. are the terminal
specifiers which maybe a string or numeric values.

\section{Netlist Variables}

Local variables for use inside a netlist may be set with the {\em
.options} command using the same syntax as used to set system
variables.  For example

\begin{tt}
\begin{verbatim}
                      .options logic1 = 5.0
                      .options logic0 = 0.6
                      .options  vdiff = logic1 - logic2
\end{verbatim}
\end{tt}

These local variables do not need to be declared before being set
but they must be set before being used.  Local variables are
designed so that common parameters (such as microstrip width) may
be declared in each {\em .model} statement as a variable with the
variables value set once at the top of the netlist.  Changing
width requires changing one variable rather than multiple
declarations in different {\em .model} statements. The third {\em
.options} statement above illustrates the use of mathematical
operations on local variables, in this case the difference between
{\bf logic1} and {\bf logic0} is assigned to {\bf vdiff}. Various
analyses rely on variables set in a {\em .options} statement.  The
variables are defined in the description of the particular
analyses.

\clearpage
\section{.couple  --- Couple Elements}

{\em NOT FUNCTIONAL in current version}
\noindent\textit{Form:}
\newline
{\tt .couple $\langle$ \tt{element\ instance\ name}$\rangle$ ... $\langle$
\tt{element\ instance\ name}$\rangle$ $\langle$ \tt{terminal\
name}$\rangle$ $\langle$ \tt{terminal\ name}$\rangle$}\\[0.2in]
\newline
\textit{Description:}\\
.couple is used to identify the elements that combine to create a coupled element.\\[0.2in]

This command is used to indicate which edges (or nodes) to be
simulated as coupled lines.  The syntax is:
\begin{tt}
\begin{verbatim}
                 .couple line_1 line_2 line_3....
\end{verbatim}
\end{tt}
where {\em line\_1} etc. are the specific names given to each
instance of a line (or node).  Note that the type of model used
for coupled edges or nodes must be able to handle coupling.  In
general, a single line or node that may also be coupled is just a
subset of the coupled line case.  In other words, if a coupled
line model (such as {\bf cmlin}) is specified as the line model
and the {\em .couple} statement is not used, then the simulator
will default to using the uncoupled model (in this case {\bf
mlin}).  This is not a runtime option but is fixed inside the code
modules for each model.

\clearpage
\subsection{.locate --- Identify Location of Terminals}

This command is used to define the physical location or a terminal.  These cartesian
coordinates refer to the locations of the ``logical'' terminals of
the device.  The units are meters.  The syntax is:
\begin{tt}
\begin{verbatim}
                       .locate term x y
                       .locate term x y z
\end{verbatim}
\end{tt}
where {\tt term} is one of the terminals of a device in the
netlist and {\tt x}, {\tt y} and {\tt z} (if provided) are the
coordinates of that terminal. Be default {\tt z}=0.

\noindent\textit{Credits:}\\[0.1in]
\begin{tabular}{l l l l}
Name & Affiliation & Date & Links \\
Michael Steer & NC State University & Sept 2000 & \epsfxsize=1in\pfig{logo.eps} \\
m.b.steer@ieee.org & & & www.ncsu.edu    \\
\end{tabular}

\clearpage
\section{.model}

\noindent
\textit{Description:}\\
.model is used to identify the elements that combine to create a coupled element.\\[0.2in]

The syntax of the {\em .model} statement is:

\begin{tt}
\begin{verbatim}
              .model identifier element (par1 par2 ...)
\end{verbatim}
\end{tt}
\begin{itemize}
\item({\em identifier}) is any character string name assigned by the user
by which this particular model will be referred.
\item({\em model\_type}) is the model name as defined in the .c file associated
with this model and as declared in pd\_physdef.c.
\item(par1 par2...) is the parameter list.
\end{itemize}

The {\em .model} statement must be used before it is referred to in the netlist.  All fREEDA elements can utilize a {\em .model} statement.

\clearpage
\section{.options}

\noindent
\textit{Description:}\\
.options is used to identify the elements that combine to create a coupled element.\\[0.2in]

This command allows various runtime options and user defined
netlist variables to be set prior to execution.  The various
system options will be discussed later in this appendix but the
general syntax is:
\begin{tt}
\begin{verbatim}
                .options variable = value
                .options variable = "string"
\end{verbatim}
\end{tt}

The first case is used for assigning a numeric value to a variable
and the second is used to assign a string.  Note that double quote
marks (``...'') must be used to surround the string.  Not
typecasting of numeric variables is performed in the .options
command and thus no distinction is made between floating point and
integer values. Therefore 2 is the same as 2.00 until the value is
actually used in the simulator.  Exponential notation is denoted
by the ``e'' operator (i.e. 0.001 = 1.0e-3). Note that string
variables may contain any symbols but must be continuous with no
white space between characters (i.e. ``V\_high'' not ``V high'').

\clearpage
\section{.out}
\noindent\textit{Form:}\\
{\tt .out write ( [[$<${\it qualifier}$>$] $<${\it value}$>$*]
$<${\it operator}$>$ )*
      in $<${\it filename}$>$} \\
This write what is left on the stack into the file {\it filename}
or \\
{\tt .out system ( [[<{\it qualifier}$>$] <{\it value}$>$*] [<{\it
operator}$>$] )* } This performs a system call of the string
equivalent of whatever is left on
the stack.  \\
\newline
{\tt .out $\langle$ \tt{terminal\ name}$\rangle$ $\langle$
\tt{terminal\ name}$\rangle$ $\langle$ \tt{terminal\
name}$\rangle$ $\langle$ \tt{terminal\ name}$\rangle$}\\[0.2in]
\newline
\textit{Description:}\\
.out is used to identify the elements that combine to create a coupled element.\\[0.2in]

The {\em .out} command is used to process and output data
resulting from a fREEDA run.  The {\em .out} statement uses stacks
and has a syntax much like a reverse polish notation calculator.
It is a powerful output engine and can be utilized in its own
right or in conjunction with the more usual {\em .print} and {\em
.plot} statements although these provide much less functionality.
A variety of signal processing functions including arithmetic
operators may be used to manipulate the data prior to writing it
to a file, plotting to the screen or piping it to a system call.

fREEDA has an interpretive output language which uses a reverse
polish syntax. The operators operate on a stack and as an
operation is performed zero or more arguments are consumed by an
operator.

Details of the various options will be shown at the end of this
section but for most situations and netlists, the voltages and
currents at the various external ports are to be written to output
files in standard ASCII format. An example is shown below:

\begin{tt}
\begin{verbatim}
                 .out write term 1 vt in "1v.out"
                 .out write term 2 it in "2i.out"
\end{verbatim}
\end{tt}

In the first example, the voltage at terminal 1 is written out to
file ``1v.out''.  The second example writes the current going {\em
into} terminal 2 to the file ``2i.out''.

\clearpage
\subsection{Qualifiers}

\begin{tabular}{ll}
{\bf type} & {\bf description} \\
\\
\multicolumn{2}{l}{\sl qualifiers  (network types)} \\
\\
{\it term} & terminal reference\\
{\it junct} & junction (node) reference (NOT CURRENTLY AVAILABLE) \\
{\it line} & line (edge) reference (NOT CURRENTLY AVAILABLE)
\end{tabular}

\subsection{Nomenclature}

The following nomenclature is used in describing the output
operators.

%\begin{tabular}{ll}
\begin{longtable}{ll}
{\bf type} & {\bf description} \\
\multicolumn{2}{l}{\sl scalar numeric types} \\
\\
{\it i} & integer \\
{\it f} & floating-point \\
{\it r} & real (integer or floating-point) \\
{\it c} & complex \\
{\it s} & scalar (integer, floating-point or complex) \\
\\
\multicolumn{2}{l}{\sl scalar and mixed numeric types} \\
\\
{\it fv} & floating-point vector \\
{\it cv} & complex vector \\
{\it v} & floating-point or complex vector \\
{\it fsv} & floating-point scalar or vector \\
{\it csv} & complex scalar or vector \\
{\it sv} & scalar or vector (any) \\
{\it prom} & an appropriately-promoted numeric type \\
{\it -x} & (suffix to vector types) x data required \\
\\
\multicolumn{2}{l}{\sl other types} \\
\\
{\it any} & any type \\
{\it string} & character string \\
{\it var} & variable name \\
{\it file} & data file \\
{\it func} & function pointer
\end{longtable}
%\end{tabular}
\clearpage
\subsection{Operators}

\subsection*{General Operators}

\begin{tabular}{p{.8in}p{2.5in}p{1.0in}p{.75in}}
{\bf operator} & {\bf function} & {\bf argument(s)} & {\bf result} \\
\\
\\
{\tt dup} & duplicate object & {\it any} & {\it same} \\
{\tt get} & get element of vector & {\it arg:v \newline index:i} & {\it s} \\
{\tt put} & modify element of vector & {\it arg:v \newline index:i
\newline
  val:s} & {\it v} \\
{\tt stripx} & remove x data & {\it vx} & {\it v} \\
{\tt shell} & execute shell command (UNIX ENVIRONMENT ONLY)& {\it
string} & {\it none}
\end{tabular}

\subsection*{Network Operators}

\begin{tabular}{p{.8in}p{2.5in}p{1.0in}p{.75in}}
{\tt v} & complex voltage vector at a terminal & {\it term} & {\it cv} \\
{\tt i} & complex current vector at a terminal & {\it term} & {\it cv} \\
{\tt vt} & transient voltage vector at a terminal & {\it term} & {\it fv} \\
{\tt it} & transient current vector at a terminal & {\it term} & {\it fv} \\
{\tt zl} & load impedance at a terminal (NOT CURRENTLY SUPPORTED)& {\it term} & {\it cv} \\
{\tt ymelem} & element of the y-parameter matrix of a junction
(NOT CURRENTLY SUPPORTED)&
  {\it junct \newline row:i \newline col:i} & {\it cv} \\
{\tt z0} & characteristic impedance of a line & {\it line}  (NOT CURRENTLY SUPPORTED)& {\it cv} \\
{\tt gamma} & complex attenuation of a line & {\it line}  (NOT CURRENTLY SUPPORTED)& {\it cv} \\
{\tt yp} & admittance parameter of two terminals & {\it term} (NOT
CURRENTLY SUPPORTED)& {\it fv}
\end{tabular}
\newpage
\subsection*{Arithmetic Operators}

\begin{tabular}{p{.8in}p{2.5in}p{1.0in}p{.75in}}
{\tt add} & addition & {\it sv \newline sv} & {\it prom} \\
{\verb:+:} & addition & {\it sv \newline sv} & {\it prom} \\
{\tt sub} & subtraction & {\it sv \newline sv} & {\it prom} \\
{\verb:-:} & subtraction & {\it sv \newline sv} & {\it prom} \\
{\tt mult} & multiplication & {\it sv \newline sv} & {\it prom} \\
{\verb:*:} & multiplication & {\it sv \newline sv} & {\it prom} \\
{\tt div} & division & {\it sv \newline sv} & {\it prom} \\
{\verb:/:} & division & {\it sv \newline sv} & {\it prom} \\
{\tt real} & real part & {\it csv} & {\it fsv} \\
{\tt imag} & imaginary part & {\it csv} & {\it fsv} \\
{\tt mag} & magnitude & {\it csv} & {\it fsv} \\
{\tt abs} & absolute value or magnitude & {\it sv} & {\it fsv} \\
{\tt contphase} & continuous phase & {\it csv} & {\it fsv} \\
{\tt prinphase} & principal value phase & {\it csv} & {\it fsv} \\
{\tt conj} & complex conjugate & {\it csv} & {\it csv} \\
{\tt neg} & additive inverse (negative) & {\it sv} & {\it sv} \\
{\tt recip} & reciprocal & {\it sv} & {\it sv}
\end{tabular}
\newpage
\subsection*{Mathematical Operators}

\begin{tabular}{p{.8in}p{2.5in}p{1.0in}p{.75in}}
{\tt db} & dB ($20 \log_{10}$) & {\it sv} & {\it fsv} \\
{\tt db10} & dB applied to power ($10 \log_{10}$) & {\it sv}
  & {\it fsv} \\
{\tt rad2deg} & convert radians to degrees & {\it fsv} & {\it fsv} \\
{\tt deg2rad} & convert degrees to radians & {\it fsv} & {\it fsv} \\
{\tt minlmt} & limit the minimum value & {\it arg:fsv \newline min:f} &           {\it fsv} \\
{\tt maxlmt} & limit the maximum value & {\it arg:fsv \newline
max:f} &
  {\it fsv} \\
{\tt diff} & differences & {\it fsv} & {\it fsv} \\
{\tt deriv} & derivative & {\it fsv} & {\it fsv} \\
{\tt sum} & sums & {\it fsv} & {\it fsv} \\
{\tt integ} & integral & {\it fsv} & {\it fsv}
\end{tabular}

\newpage
\subsection*{Signal Processing Operators}

\begin{tabular}{p{.8in}p{2.5in}p{1.0in}p{.75in}}
{\tt smpltime} & current analysis timebase as x {\em and} y of
result &
  {\it none} & {\it fv} \\
{\tt sweepfrq} & current analysis sweep frequencies as x {\em and}
  y of result & {\it none} & {\it fv} \\
{\tt smplcvt} & interpolate {\em signal1} over timebase of {\em
signal2} &
  {\it signal1:v \newline signal2:vx} & {\it vx} \\
{\tt sweepcvt} & interpolate {\em frq1} over sweep frequencies of
{\em frq2} &
  {\it frq1:v \newline frq2:vx} & {\it vx} \\
{\tt maketime} & create timebase starting at $t=0$ in x {\em and}
y of result &
  {\it tmax:r \newline pts:i} & {\it vx} \\
{\tt makesweep} & create sweep frequencies starting at $f=0$ in x
  {\em and} y of result & {\it fmax:r \newline pts:i} & {\it vx} \\
{\tt fft} & FFT (argument should have $2^{k}$ points) &
  {\it timedata:fv} & {\it cv} \\
{\tt invfft} & inverse FFT (argument should have $2^{k}-1$ points)
&
  {\it frqdata:cv} & {\it fv} \\
{\tt cconv} & real circular (FFT) convolution with zero padding &
  {\it signal1:fv \newline signal2:fv} & {\it fv} \\
{\tt upcconv} & unpadded real circular (FFT) convolution &
  {\it signal1:fv \newline signal2:fv} & {\it fv} \\
{\tt sconv} & slow (time-domain) real convolution &
  {\it signal1:fv \newline signal2:fv} & {\it fv} \\
{\tt fconv} & fast (approximate) real convolution &
  {\it signal1:fv \newline signal2:fv} & {\it fv} \\
{\tt lpbwfrq} & lowpass Butterworth filter frequency response &
  {\it frqvec:vx \newline corner:f \newline order:i} & {\it cvx}
\end{tabular}

\newpage
\subsection*{Other Operators}

\bigskip
\begin{tabular}{p{.8in}p{2.5in}p{1.0in}p{.75in}}
{\tt catalog} & produce catalog of elements &
  {\it none} & {\it func}
\end{tabular}

\noindent Example

\begin{verbatim}
.out write catalog in "list.txt"
\end{verbatim}
Writes the catalog of the elements in the current fREEDA build and
puts the catalog in the file `list.txt'.

\clearpage
\section{.tran}

Similar to SPICE's {\em .tran} card with syntax:
\begin{tt}
\begin{verbatim}
                 .tran start stop delta
\end{verbatim}
\end{tt}
where {\tt start} is the starting transient analysis time, {\tt
stop} is the ending time and {\tt delta} is the time increment. If
{\tt delta} is zero, the finest time increment is used (determined
by the highest frequency, {\tt sfrq} and the number of frequency
points {\tt spts}).

\section{.tran2}

Under construction

\section{.tran4}

Under construction

\clearpage
\section{.tran basel}

NO LONGER SUPPORTED

But it will be.  The analysis was used in an earlier version and performs convolution-based analysis.

This section defines the options used in Mark Basel's particular
form of transient analysis.  This analysis is not publicly
available.  The variables set in in a {\em .options} statement for
this analysis are shown in Table \ref{tran_basel_runtimeops.tab}

\begin{table}
\caption{fREEDA runtime options}
\label{tran_basel_runtimeops.tab}
\centerline{\begin{tabular}{|c|c|c|} \hline \hline
Variable Name & Definition & Use \\
\hline
iterationdump & Debugging dump for  & ON or OFF \\
              & each iteration of   & \\
              & transient analysis  & \\
\hline
dump  & Debugging dump of & ON or OFF \\
      & various variables & \\
\hline
dumpnet  & Debugging dump of network & ON or OFF \\
         & as interpreted by TRANSIM & \\
\hline
dcNormal & Switch for using & ON or OFF \\
         & threshold error & \\
         & correction      & \\
\hline
spts & number of frequency & int: power of 2 \\
     & points used in y(f) & \\
\hline
$Z_m$ & Matching network impedance & float, ohms \\
type  & form of analysis & ``transient'' \\
      &                  & ``hb'' \\
\hline
sfrq & Maximum frequency & float: hz \\
\hline
LPFOrder & low pass filter order & int: 1,2 or 3 \\
\hline
impulselength & fraction of impulse & float: 0-1 \\
              & response to use in  & \\
              & transient analysis  & \\
\hline
impulsescale & scale factor for & float: any \\
             & impulse responses & \\
\hline
ytthresthru & relative threshold level  & float: 0-1\\
            & for thru and self impulse & \\
            & response terms            & \\
\hline
ytthrescross & relative threshold level  & float: 0-1\\
             & for cross impulse         & \\
             & response terms            & \\
\hline
tolerance & stopping difference & float: any\\
          & for successive values & \\
          & in Newton iteration   & \\
\hline
maxNoOfIterates & Maximum number of & int: any \\
                & Newton iteration  & \\
                & steps per analysis point & \\
\hline
LPFCornerFrequency & corner frequency when & float: hz\\
                   & using LP filter & \\
\hline
\end{tabular}
}
\end{table}
