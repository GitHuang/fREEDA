
\chapter{Algebraic Expressions}
This page is still under development so there may be problems.
The big one is that expressions do not work at all now.

Most places where a numeric value is normally used
an expression (within braces {\tt \{ $\ldots$ \})} can be used instead.
An expression can contain any supported mathematical operation,
constant numeric values, parenthesees ``( $\ldots$ )'' to indicate precedence,
commas ``,'' to separate arguments of a function,
or parameters. Valid parameters must begin with an alphabetic character.
Places where expressions cannot be used are 
     \begin{itemize}
     \item As polynomial coefficients.
     \item The values of the transmission line device parameters {\tt NL} and
           {\tt F}.
     \item The values of the piece-wise linear characteristic in the {\tt PWL}
           form of the independent voltage ({\tt V}) and current ({\tt I})
           sources.
     \item The values of the resistor device parameter {\tt TC}.
     \item As node numbers.
     \item[and]
     \item Values of most statements (such as .TEMP, .AC, .TRAN etc.)
     \end{itemize}
Specifically included are
     \begin{itemize}
     \item The values of all other device parameters.
     \item The values in {\tt .IC} and {\tt .NODESET} statements.
     \item The values in {\tt .SUBCKT} statements.
     \item[and]
     \item The values of all model parameters.
     \end{itemize}

Operators that can be used in expressions are listed in Table
\ref{table:expression:operators}. Here $x$ and $y$ maybe numbers,
parameters or sub-expressions.
The result of the logical operations is either 0 or 1.
Operands are treated as 1 if they are not exactly zero.
The precedence of the operators is also given in Table
\ref{table:expression:operators} and follows normal practice.
A higher precedence number indicates higher prcedence and
operators of the same precedence are evaluated from left to right.
For example\\

    \begin{tabular}{lp{3.5in}}
    {\tt idss* ( vgs + vgs\^{}2 )} & is evaluated as 
			     ${\tt idss} * ( {\tt vgs} + ({\tt vgs} ^ 2) )$
			     \newline
                             where {\tt idss} and {\tt vgs} are parameters
			     defined elsewhere.\\
                     
    {\tt 3 + 5 -- 4||(3\^{}-4)$>$=23}
				& is evaluated as \newline
				 $(((3 + 5) - (-4)) || ((3 ^ (-4)) >= 23))$\\
    \end{tabular}

\begin{table}
\caption{Expression operators.\label{table:expression:operators}}
\begin{center}
\begin{tabular}{|l|c|l|l|}
\hline
\multicolumn{1}{|c}{\bf Operator } &
\multicolumn{1}{|c}{\bf Precedence} &
\multicolumn{1}{|c}{\bf Syntax } &
\multicolumn{1}{|c|}{\bf Description}  \\
\multicolumn{1}{|c}{\bf  Name } &
\multicolumn{1}{|c}{\bf  Index } &
\multicolumn{1}{|c}{} &
\multicolumn{1}{|c|}{} \\
\hline
\hline
\multicolumn{4}{|c|}{Arithmetic Operators} \\
UNARY\_PLUS	& 10& {\tt +}$x$            	& unary plus \\
UNARY\_MINUS	& 10& {\tt -}$x$            	& unary minus \\
POWER		& 9 & $x${\tt \^{}}$y$  or
                  $x${\tt **}$y$ 		& raise to a power, $x^y$ \\
MULTIPLY	& 8 & $x${\tt *}$y$		& multiply \\
DIVIDE		& 8 & $y${\tt /}$x$	        & divide \\
PLUS		& 7 & $x${\tt +}$y$		& plus \\
MINUS		& 7 & $x${\tt -}$y$		& minus \\
\hline
\hline
\multicolumn{4}{|c|}{Logical Operators} \\
\hline
NOT		& 10& {\tt !}$x$		& NOT \\
GREATER\_OR\_EQUAL& 6 & $x>=y$			& greater than or equal \\
LESS\_OR\_EQUAL	& 6 & $x<=y$			& less than or equal \\
GREATER\_THAN	& 6 & $x>y$			& greater than \\
LESS\_THAN	& 6 & $x<y$			& less than \\
EQUAL		& 5 & $x==y$			& equality \\
NOT\_EQUAL	& 5 & $x!=y$			& no equal \\
AND		& 4 & $x${\tt \&}$y$           	& logical and \\
OR		& 3 & $x${\tt |}$y$           	& logical or \\
XOR		& 2 & $x${\tt ~}$y$		& exclusive or \\
\hline
\end{tabular}
\end{center}
\end{table}

Functions that can be used in expressions are listed in Table
\ref{table:expression:functions}. 

\begin{table}
\caption{Expression functions.\label{table:expression:functions}}
\begin{center}
\begin{tabular}{|l|l|l|}
\hline
{\bf Operator }	& {\bf   Syntax } & {\bf Description}  \\
\hline
SIN		&{\tt sin($x$)}	& sine, argument in radians \\
COS		& {\tt cos($x$)}	& cosine, argument in radians \\
TAN		& {\tt tan($x$)}	& tangent, argument in radians \\
ASIN		& {\tt asin($x$)} or {\tt arcsin($x$)}
					& arcsine, result in radians \\
ACOS		& {\tt acos($x$)} or {\tt arcsin($x$)}
					& arccosine, result in radians \\
ATAN		& {\tt atan($x$)} or {\tt arcsin($x$)}
					& arctangent, result in radians \\
SINH		& {\tt sinh($x$)}	& hyperbolic sine \\
COSH		& {\tt cosh($x$)}	& hyperbolic cosine \\
TANH		& {\tt tanh($x$)}	& hyperbolic tangent \\
EXP		& {\tt exp($x$)}	& exponentiation, $e^x$ \\
ASINH		& {\tt asinh($x$)} or {\tt arcsinh($x$)}
                                        & arc-hyberbolic sine \\
ACOSH		& {\tt acosh($x$)} or {\tt arccosh($x$)}
					& arc-hyberbolic cosine \\
ATANH		& {\tt atanh($x$)} or {\tt arctanh($x$)}
					& arc-hyberbolic tangent \\
ABS		& {\tt abs($x$)}	& absolute, $|x|$ \\
SQRT		& {\tt sqrt($x$)}	& square root, $\sqrt{x}$ \\
LN		& {\tt ln($x$)}		& log to base e of $x$ \\
LOG		& {\tt log($x$)} or log10($x$)	& log to base 10 of $x$ \\
SIGN		& {\tt sign($x$)}		& sign of $x$
						$= \left\{ \begin{array}{rl}
                                                   1 & \mbox{if}~ x >= 0 \\
                                                  -1 & \mbox{if}~ x<0 \\
				          	\end{array}  \right. $ \\ %} 
DB		& {\tt db($x$)}		& decibell  = 10 log($x$) \\
LIMIT		& {\tt limit($x$,{\it min},{\it max})} & limit
					$= \left\{ \begin{array}{rl}
                                        {\it min} & \mbox{if}~ x < {\it min} \\
                                        {\it max} & \mbox{if}~ x < {\it max} \\
                                            x & \mbox{otherwise} \\
				       	\end{array}  \right. $ \\ %}
TABLE		& {\tt table($x,x_1,y_1,...,x_n,y_n$)} & table lookup, 
					  see note 1 \\
DUPLICATE	& {\tt dup($x$)}	& duplicates $x$, 
					  see note 2 \\
IF		& {\tt if($x,y,z$)}     & conditional,
					$= \left\{ \begin{array}{rl}
                                        y & \mbox{if}~ x~ \mbox{not zero} \\
                                        z & \mbox{if}~ x~ \mbox{is zero} \\
				       	\end{array} \right. $ \\ %}
\hline
\end{tabular}
\end{center}
\end{table}

Note:
\begin{enumerate}
\item[1.] The table look up function returns a $y$ value
given an $x$ value and a set of ($x$,$y$) points defining piece-wise linear.
The number of $x-y$ pairs in the table function is limited
approximately to 100.  A further limit is imposed by the amounjt of
information thatmustbe retained during expression evalurion.  To
obtain the full 100 print capability in a complicated expression it may be
necessary to use an intermediate variable.
expre

\item The dup() function duplicates an operand.
      It provides a means to use a sub-expression twice while only evaluating
      it once. For example 

      \medskip

      \begin{tabular}{lll}
       \multicolumn{1}{c}{Operation} &
       \multicolumn{1}{c}{Expansion} \\
       \hline
      (dup($x$)+)           & $\longrightarrow$ & $x + x$ \\
      (dup(dup($x$))+*)     & $\longrightarrow$ & $(x + x)*x$ \\
      limit(dup($x$),max) & $\longrightarrow$ & $ \left\{ \begin{array}{rl}
				      x & \mbox{if}~ x~ < ~ \mbox{max} \\
				      \mbox{max} & \mbox{if}~ x~ > ~ \mbox{max}
                                           \end{array} \right.$ \\ % }
      if(dup(dup($x$))>0, *2,*3)&
			 $\longrightarrow$ & $ \left\{ \begin{array}{rl}
				      2x & \mbox{if}~ x~ > ~ 0 \\
				      3x & \mbox{if}~ x~ < ~ 0 \\
                                           \end{array} \right.$ % }
      \end{tabular}

      \medskip

      \noindent
      It is good practice to enforce precedence by using parentheses. That
      is, use ({\tt dup($x$)}+) rather than {\tt dup($x$)}+ .
\end{enumerate}
