\chapter{Graphical User Interface} \label{ch_gui}

\section{Introduction}

fREEDA supports three interactive front ends:
\begin{itemize}
\item iFREEDA --- the preferred interface and part of the fREEDA distribution.
\item Electric Editor --- Very good for VLSI layout, not documented.
\item fREEDA GUI --- not currently distributed but described in this chapter.
\end{itemize}

The simulation engine in fREEDA${^\mathrm{TM}}$ can be used in the traditional way as
a stand-alone program, for example in batch jobs. In this mode, the
program reads an input netlist, process its contents and writes the
requested output files.

fREEDA also provides a Graphical User Interface (GUI), which is more
convenient for interactive use of the program. This GUI is written
using the Java language, so it can be used in every system where Java
is supported. In this chapter we describe the different components of
the GUI. This has now been replaced by iFREEDA but this documentation is provided for completeness and the code is available.

\section{The Netlist Editor}

The netlist editor is a simple text editor combined with a simulation
manager. The editor window is shown in Figure \ref{fig:editorw}.
%
\begin{figure*}
\centerline{\epsfxsize=10cm \pfig{editw.eps}}
\caption{Netlist Editor window.} \label{fig:editorw}
\end{figure*}
%
Besides the normal editing commands, the editor provides buttons and
keyboard shortcuts to analyze the netlist being edited and see the
output of the simulation.

The editor can edit several files and handle multiple simulations at
once by spawning multiple windows.

\section{The Analysis Window}

The analysis window is used to show the progress of a simulation
(Figure \ref{fig:analyzew}).
%
\begin{figure*}
\centerline{\epsfxsize=10cm \pfig{analyzew.eps}}
\caption{Analysis window.} \label{fig:analyzew}
\end{figure*}
%
The upper subwindow displays important messages such as when the
program starts or stops, and also warnings and errors that may occur
during the simulation. The lower subwindow shows the progress of the
simulation.

The buttons are self-explaining. The ``Analyze'' button changes to
``Stop'' when the engine is running.

\section{The Output Viewer Window}

This window is perhaps the most useful of all. It is shown in Figure
\ref{fig:outputw}.
%
\begin{figure*}
\centerline{\epsfxsize=10cm \pfig{outputv.eps}}
\caption{Output Viewer window.} \label{fig:outputw}
\end{figure*}
%
The output file is displayed at the left. This file contains detailed
information about the simulation.

At the right there is a list of files available for
plotting. After selecting one or more of these files and depressing
the ``Plot'' button, a plot window appears showing the desired data
(see Figure \ref{fig:plotw}).
%
\begin{figure}
\centerline{\epsfxsize=10cm \pfig{plotw.eps}}
\caption{Plot window.} \label{fig:plotw}
\end{figure}
%
Any number of plots can be requested. Also, the plot data is kept in
memory by the plot window, so it is possible to re-run a simulation
with different parameters and compare the new and old graphs on the
screen.  An encapsulated postscript file can be generated pressing the
corresponding button.

There are several features provided in the plot window. One of the
most remarkable is that it is possible to zoom in or out the graph by
dragging the left or right mouse button, respectively.

The plotting facility is provided by the ptplot \cite{ptplot} library.
