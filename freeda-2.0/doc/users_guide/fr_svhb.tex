
\chapter{Harmonic Balance Analysis} \label{ch_svhb}

%--------------------------------------------------------------------------
\section{Introduction}

Steady state (harmonic balance) analysis of microwave circuits
continues to be the dominant approach to the simulation of nonlinear
RF and microwave circuits. Development efforts are currently being
directed at extending the approach to accommodate a very large number
of tones, improve robustness, and development of matrix-free methods
to handle very large problems without explicit storage of the
Jacobian. One system that is of particular concern is the simulation
of quasi-optical power combiners with hundreds of active devices with
significant thermal effects and distributed over an electrically large
region. This exacerbates the problem of robustness and of model
development.

In this chapter we present neccesary developments which makes the
steady-state simulation of large systems possible. The issue of
robustness is addressed by
\begin{enumerate}
\item Using state variables which enables the use of parametrized
device models including nonlinear behavioral models.
\item Using the minimum number of state variables and error
functions and so eliminating the redundancy in the Jacobian.
\item A formulation permitting the use of various nonlinear equation
solvers.
\end{enumerate}
Model development is greatly simplified using a novel implementation
of automatic differentiation technology.

With respect to the memory requirements and running time, fREEDA$^{\mathrm{TM}}$
performs on average as well as commercial tools for small and medium
size problems.

%--------------------------------------------------------------------------
\section{Review of Harmonic Balance Formulations}

\subsection{Conventional formulation}

 The idea of harmonic balance~\cite{steer2:1996} is most easily
demonstrated using the approach of Nakhla and
Vlach~\cite{nakhla:vlach:76}. This approach is based on partitioning
of the linear and nonlinear portions of a circuit as shown in
Figure~\ref{hb1}.

\begin{figure}[htpb]
\centerline{\epsfxsize=3in {\pfig{HB_concept.eps}}}
\caption{Partition representation of the harmonic
balance method} \label{hb1}
\end{figure}

A solution at the $k^{th}$ node up to the $n^{th}$ harmonic is assumed
of the form
\begin{equation}
  v_k(t)= \Re \left\{ \sum_{m=0}^n V_m 
  \exp(j \omega m t + \phi_m) \right\} \label{eq_hb1}
\end{equation}
with the unknown variables being the amplitude and phase of each
frequency component at each interfacial node. Note that phasor
amplitude and phase at each frequency is fixed with respect to time.
The solution is, then, a finite sum of commensurate discrete
spectra. Conservation of current and charge in the frequency domain
at each of the interfacial nodes of Figure~\ref{hb1} gives
\begin{equation}
  F(V)= I(V) + {\bf \Omega} Q(V) + {\bf Y} V + I_S = 0 \label{eq9}
\end{equation}
where ${\bf Y}$ is the nodal admittance matrix of the linear part and
$I_S$ is the source vector. The first two terms on the right side are
calculated as:
\begin{equation}
  I_{NL}(V)= {\mathcal{F}}\{{\bf i}[{\mathcal{F}}^{-1}(V)]\} + {\bf \Omega}
             {\mathcal{F}}\{{\bf q}[{\mathcal{F}}^{-1}(V)]\}
\end{equation}
where $\mathcal{F}$ is the Fourier transform operator and ${\bf \Omega}$
is a diagonal matrix of $j\omega m$ terms representing
frequency domain differentiation. Solution is carried out in the
frequency domain, with nonlinearities transformed to the frequency
domain using FFT with precisely spaced time samples. 

\subsection{State-Variable Formulation} \label{subsection:rizzoli}

Let the nonlinear subnetwork be described by the following generalized
parametric equations~\cite{rizzoli:92:1}
\begin{eqnarray}
  {\bf v}(t) &=& u[{\bf x}(t), \frac{d{\bf x}}{dt}, \dots , 
                 \frac{d^n{\bf x}}{dt^n}, {\bf x}_D(t)] \\
  {\bf i}(t) &=& w[{\bf x}(t), \frac{d{\bf x}}{dt}, \dots ,
                 \frac{d^n{\bf x}}{dt^n}, {\bf x}_D(t)] 
  \label{eqriz3}
\end{eqnarray}
where $v(t)$, $i(t)$ are vectors of voltages and currents at the
common ports, ${\bf x}(t)$ is a vector of state variables and ${\bf
x}_D(t)$ a vector of time-delayed state variables, i.e.,
$x_{Di}(t)=x_i(t-\tau_i)$. The time delays $\tau_i$ may be functions
of the state variables. All vectors in (\ref{eqriz3}) have a same size
$n_d$ equal to the number of common (device) ports. This kind of
representation is very convenient from the physical viewpoint, because
it is in fact equivalent to a set of implicit integro-differential
equations in the port currents and voltages. This allows an effective
minimization of the number of subnetwork ports, and what is more
important, results in extreme generality in device modeling
capabilities.

The quasi-periodic electrical regime of the nonlinear circuit
resulting from a multi-tone excitation is completely defined by a set of
time-dependent state variables of the form (\ref{eq_hb1}), or
equivalently by the vector $X$ of the real and imaginary parts of
their harmonics.

The linear subnetwork may be represented by the frequency domain
equation
\begin{equation}
  {\bf Y}(\omega) V(\omega) + N(\omega) + I(\omega) = 0 \label{eqriz4}
\end{equation}
where $V(\omega)$, $I(\omega)$ are vectors of voltage and current
phasors, $Y(\omega)$ is the linear subnetwork admittance matrix, and
$N(\omega)$ is a vector of Norton equivalent current sources. Thus a
set of complex harmonic-balance errors at a generic IM product
$\Omega_k$ has the expression
\begin{equation}
  F_k(X) = {\bf Y}(\Omega_k) U_k(X) + N(\Omega_k) + W_k(X) \label{eqriz5}
\end{equation}
The nonlinear analysis problem is reduced to the solution of a
nonlinear algebraic system by imposing that all the HB errors
vanish. In order to avoid the use of negative frequencies, the
nonlinear solving system is formulated in terms of a vector $F$ of
real and imaginary parts of the HB errors given by (\ref{eqriz5}), and
thus written as a systems of $N_t$ real equations in $N_t$ unknowns,
namely
\begin{equation}
  F(X) = 0 \label{eqriz6}
\end{equation}

%--------------------------------------------------------------------------
\section{Review of Frequency-Time Conversion \\
Techniques}

Until 1984, harmonic balance was only used to analyze circuits with a
periodic response~\cite{kundert:vincentelli:90}. The reason for this
is that with the linear devices evaluated in the frequency domain and
the nonlinear devices evaluated in the time domain, a transform is
needed to convert signals between the two domains. The FFT was used to
perform this operation, however the FFT is only applicable to periodic
signals, which excludes a very important class of circuits whose
steady state response is not periodic: mixers. The signals present in
mixers consist of sinusoids at the sum and difference frequencies of
the two or more input frequencies and their harmonics. In general,
these input frequencies are not harmonically related, and so the
signals found in mixers are not periodic, but rather quasi-periodic.

There are currently five different methods available for transforming
signals between time and frequency domains that are suitable for use
with quasi-periodic harmonic balance. The first three of these methods
are general in nature. Two methods, those of Ushida and
Chua~\cite{ushida:84}, and Gilmore and Rosenbaum~\cite{gilmore:84},
are considered less efficient than the remaining methods and thus are
no longer used. The third method is the almost periodic Fourier
transform (APFT)~\cite{kundert:vincentelli:90}. This transform has a
simple operator notation and so is useful for theoretical
manipulation.

The last two methods exploit the fact that in harmonic balance one
desires the frequency domain response of a nonlinear device to a
frequency domain stimulus and the time domain waveforms generated
along the way are of no interest. These methods transform spectra into
waveforms with distorted time axis back into the spectra, and so as
long as the nonlinearities being evaluated in the time domain are
algebraic, which was the basic assumption with harmonic balance, then
the resulting spectra are correct. Of these two remaining methods, the
first is based in the multidimensional DFT~\cite{bava:82,ushida:87},
and it is restricted to box truncation. The second is based on the one
dimensional DFT. It is faster and has fewer restrictions than the
multidimensional DFT approach, and is the one presented
here~\cite{kundert:vincentelli:90}.

\subsection{Artificial frequency mapping} \label{subsection:freq}

This method allows the use of the DFT --- and hence the FFT --- with
harmonic balance even when the desired solution is quasi-periodic.
Here are the conditions to use this method:
\begin{enumerate}
\item The nonlinearities must be algebraic and time invariant.
\item The signals must be quasi-periodic.
\item The time domain signals must be of no interest.
\end{enumerate}

Consider a nonlinear resistor with the constitutive equation
\[
  i(v) = v^2
\]
Assume that the resistor is being driven with the voltage waveform 
\[
  v(t) = \cos(\alpha t) + \cos(\beta t)
\]
The resistor responds with a current waveform of
\[
  \begin{array}{ccl}
  i(v(t)) & = & 1 + \frac{1}{2} \cos(2 \alpha t) +
                    \cos(\alpha t - \beta t) + \\
          &   & +  \cos(\alpha t + \beta t) +
                   \frac{1}{2} \cos(2 \beta t) 
  \end{array}
\]
Notice that the coefficients of the cosines (i.e. the spectrum of the
response signal) are independent of the frequencies $\alpha$ and
$\beta$. This is true, whenever the nonlinearities are
algebraic. Thus, for the purpose of evaluating the nonlinear devices,
the actual fundamental frequencies are of no importance and can be
chosen freely. In particular, the fundamentals can be chosen to be
multiples of some arbitrary frequency so that the resulting signals
will be periodic. Once the fundamentals are chosen in this manner, the
DFT can be used. It is important to realize that these artificially
chosen fundamental frequencies are not actually used in the harmonic
balance equations, such as $Y$ and $\Omega$. Rather, they are used
when determining in which order to place the terms in the spectra so
that the DFT can be used when evaluating the nonlinear devices.

For simplicity, the way in which the artificial frequencies are chosen
is illustrated by examples. The actual artificial frequencies, and the
scale factors that convert the original fundamental to the artificial
frequencies, are of no interest except in determining the
correspondence between the quasi-periodic and periodic harmonic indices.

\begin{figure}[htpb]
\centerline{\epsfxsize=2.5in {\pfig{artificial_mapping.eps}}}
\caption{The mapping of quasi-periodic frequencies into periodic
frequencies for box truncations.} \label{art_map}
\end{figure}
%
Consider the following set of frequencies:
\[
  \begin{array}{ccl}
  \Lambda_K  & = & \{\omega:\omega = k_1 \lambda_1 + k_2 \lambda_2; 
              0 \leq k_1 \leq H_1, \\
             &   & |k_2| \leq H_2, k_1 \neq 0 \mbox{~if~} k_2 < 0\}
  \end{array}
\]
Let $\alpha_1 = 1$ and $\alpha_2 = \lambda_1 /[\lambda_2(2H_2+1)]$ be
the scaling factors of the two fundamentals. Then the scaled set of
frequencies is equally spaced and no two frequencies overlap. This
scaling, which is ideal for box truncations, is illustrated in
Figure~\ref{art_map}. The correspondence between original and
artificial frequencies is given by
\[
  k \lambda_0 = k_1 \alpha_1 \lambda_1 +
                k_2 \alpha_2 \lambda_2 
\]
where 
\[
  \lambda_0 = \frac{\lambda_1}{2 H_2 + 1}
\]
and
\[
  k = (2 H_2 + 1 ) k_1 + k_2
\]
This approach can be extended to the case where more than two
fundamentals are applied or to other truncation schemes.

%--------------------------------------------------------------------------
\section{Review of Newton-based methods}

With a Newton-Raphson type solution method, eq.~\ref{eq9} can be
solved iteratively as:
\begin{equation}
  V^{j+1}= V^j - {\bf J}^{-1} F(V^j)  \label{eq735}
\end{equation}
where ${\bf J}$ is the network Jacobian function. 

The major drawback of Newton's method is that it is not globally
convergent so, in practice, it fails too often to be a useful general
method. 

In the case of large circuits, or strongly nonlinear ones, the
Jacobian becomes large, and a great contribution to the computation
burden is due to the LU decomposition required by each
iteration~\cite{damore:94}. In the literature, several algorithms have
been presented.

In the Newton-Samanskii method~\cite{steer1:1996}, the Jacobian is
re-used in a `chord' iteration. There is also the block Newton method,
in which terms relating different frequency components are deleted
from the Jacobian. Thus the Jacobian is block diagonal and the
inversion of it requires the inversion of the much smaller
submatrices~\cite{chang:1990}. The block Newton iteration scheme can
be combined with the chord method for a further reduction in
computational complexity. However these methods can only be employed
for almost linear circuits, since they present poor convergence
properties.

The globally convergent methods~\cite{NNES} are the {\em line
search methods} and the {\em trust region methods}. These methods have
better convergence properties when the initial guess is not close to
the solution.

The line search method expands on the strict Newton method. The point
generated by equation~\ref{eq735} is not necessarily accepted as a new
point; it must meet some acceptance criterion. To do this, some
measure of merit of the current point is required, i.e., is this point
preferred versus the previous ones? A common method is to define an
objective function $f$ such that
\begin{equation}
  f = \frac{1}{2}F(X)F^T(X) \label{eqnnes12}
\end{equation}
where the factor $1/2$ is for convenience only. When $f$ is at its
global minimum, 0, a solution has been found. However, local minima of
$f$ can exist which are not solutions; this is the main downfall of
global Newton-based methods.

The trust region~\cite{powell:1988} method attempts to combine the
better global convergence properties of searches in the steepest
descent direction with the local quadratic convergence of the Newton
method. It is similar to the well-known Levenberg-Marquardt method for
nonlinear least squares.

There exist methods to update the inverse of the Jacobian matrix at
each iteration without having to perform neither the derivative
calculation nor the matrix decomposition. These are known as {\em
quasi-Newton} update methods (for example, Broyden's method). These
methods can be combined with Newton-based methods to reduce the
overall calculation time.

%---------------

There are reports of using variations of the Newton-Powell hybrid
method to solve the harmonic balance equations~\cite{damore:94}. This
method is one of the so called trust region methods combined with
quasi-Newton updates.  The convergence of this method is very good.

Also, variations of globally convergent methods have been used to
solve the harmonic balance equations~\cite{rizzoli:92:1}. These
methods combined with a good election of the state variables are the
best solution found until now for the harmonic balance problem.


%--------------------------------------------------------------------------
\section{Nonlinear Equation Formulation}

In this section the harmonic equations are formulated with the minimum
number of unknowns and error functions starting from a modified nodal
admittance matrix of the linear part of the circuit.  This approach
has the advantage that the flexibility of the modified nodal
admittance matrix is kept together with the robustness advantage given
by the state variable approach.

%The use of the minimum number of error functions is critical to the
%robustness and we believe that the work presented here is the first
%implementation of this. Anything other than the minimum number of
%error functions indicates redundancy in the Jacobian.

The formulation of the system equations begins with the partitioned
network of Fig.~\ref{fig:network}, with the nonlinear elements
replaced by variable voltage or current
sources~\cite{nakhla:vlach:76}.
%
\begin{figure}[htpb]
\centerline{\epsfxsize=3in \pfig{svHB_circuit_formulation.eps}}
\caption{Network with nonlinear elements.} \label{fig:network}
\end{figure}
%
For each nonlinear element one terminal is taken as the reference and
the element is replaced by a set of sources connected to the reference
terminal. Clearly, the state of the element can be determined
considering only the current of the sources, or the voltages across
the sources, or some combination of the voltages and currents.
Identifying the local element reference eliminates one potential state
variable and one error function term that would otherwise be
considered if all of the nonlinear element terminals were treated
equally. For example, in a conventional voltage-current formulation
the voltage and current associated with the reference node need not be
considered in the nonlinear formulation. With the three-terminal
element in Fig.~\ref{fig:network}, terminal 2 is chosen as the
reference node and only two voltages, $V_{12} = (V_1 - V_2)$ and
$V_{32} = (V_3 - V_2)$, and two currents, $I_1$ and $I_3$, are needed.

Using the state variable concept, let ${\bf X}$ be the {\em state
variable vector}. Each element of ${\bf X}$ is a vector with all the
frequency components of the same state variable, thus
%
\begin{equation}
  {\bf X} = \left[ \begin{array}{ccccc}
  x_{1,f0} & x_{1,f1} & x_{1,f2} & \cdots & x_{1,fm} \\
  x_{2,f0} & x_{2,f1} & x_{2,f2} & \cdots & x_{2,fm} \\
  \vdots \\
  x_{n,f0} & x_{n,f1} & x_{n,f2} & \cdots & x_{n,fm} 
  \end{array} \right]
\end{equation}
%
We denote $X_i$ as the $i$~th row of the matrix, i.e., all the
frequency components of the $i$~th state variable. 
% Also, each
% column of the matrix is denoted by $X_{fj}$, where $f_j$ is the
% corresponding frequency.

Now the equations can be formulated.  Unlike\cite{rizzoli:92:1}, here
the equations are formulated in the frequency domain. The current and
voltage at the $i$~th terminal of each nonlinear device are
%
\begin{eqnarray} \label{eq_nldev}
  V_i & = & V_i(X_k, \dots , X_l) \\ 
  I_i & = & I_i(X_k, \dots , X_l)
\end{eqnarray}
%
where the element under consideration depends only on the $k$~th to
$l$~th components of the state variable vector. Since the state
variable approach is used, both the current and the voltage of the
nonlinear elements are needed for developing the error function of the
network.

Since the nonlinear elements are replaced by sources, the modified
nodal admittance matrix (MNAM) of the entire circuit can be written
for each frequency:
%
\begin{equation}
  {\bf M}_{fi}  V_{fi} = S_{fi}
\end{equation}
%
The source vector $S_{fi}$ is composed of the fixed sources present in
the circuit and the state variable-dependent sources due to the
nonlinear elements. For simplicity, assume that all the
nonlinear devices are replaced by current sources. Let $I_{NL,fi}(X)$
be the vector with the currents of all the nonlinear elements at
frequency $f_i$. These currents are `applied' to the linear
circuit. Then, the source vector can be written as:
%
\begin{equation}
  S_{fi} = S_{\mbox{fixed}, fi} + {\bf W}  I_{NL,fi}({\bf X})
\end{equation}
%
Where ${\bf W}$ is a matrix that maps the nonlinear currents to the
proper place in the source vector. This matrix is composed only of
zeros, ones and minus ones, and it is the same for all frequencies
since it only contains topology information.

Provided that the MNAM is not singular, we can calculate the
voltages at the linear circuit given the values of the state
variables ${\bf X}$:
%
\begin{equation}
  V = {\bf M}_{fi}^{-1}  S_{\mbox{fixed},fi} + {\bf M}_{fi}^{-1} 
                                      {\bf W}  I_{NL,fi}({\bf X})
  \label{eq_voltage}
\end{equation}
%
% The condition for the MNAM not to be singular is that there are not
% current sources in serial connection. Therefore, no nonlinear
% elements can be connected in series if we consider them as current
% sources. 
%
We denote $U_{NL,fi}({\bf X})$ the vector with the voltages of all
the current sources due to the nonlinear elements at frequency $f_i$
(port voltages of the nonlinear elements). The error function is
formulated by comparing the port voltages of the nonlinear elements
$U_{NL,fi}({\bf X})$ with the port voltages at the linear circuit
$U_{L,fi}({\bf X})$. In order to obtain $U_{L,fi}({\bf X})$, we need
to perform differences between the nodal voltages of the current
source's terminals. This can be accomplished by multiplying the
nodal voltage vector $V$ by a matrix ${\bf T}$ such that the result
are the desired port voltages. It can be shown that ${\bf W}$ is the
transpose of ${\bf T}$ so that:
%
\begin{equation}
  U_{L,fi}({\bf X}) = {\bf T} {\bf M}_{fi}^{-1}  S_{\mbox{fixed},fi} + 
             {\bf T} {\bf M}_{fi}^{-1}  {\bf T}^T  I_{NL,fi}({\bf X})
\end{equation}
%
The first term in this sum is a constant vector, and the second is a
constant matrix multiplied by a function of ${\bf X}$. Then, 
%
\begin{eqnarray}
  S_{sv,fi} & = & {\bf T}{\bf M}_{fi}^{-1} S_{\mbox{fixed},fi} \label{eq:ssv}\\
  {\bf M}_{sv,fi} & = & {\bf T} {\bf M}_{fi}^{-1}  {\bf T}^T  \label{eq:msv}
\end{eqnarray}
%
$S_{sv,fi}$ is the state variable source vector, and the matrix
${\bf M}_{sv,fi}$ is the state variable impedance matrix.

The solution of the problem is achieved by finding the zero of each
error function 
%
\begin{equation} 
  F_{fi}({\bf X}) = U_{L,fi}({\bf X}) - U_{NL,fi}({\bf X}) = 0 \label{eq_errfi}
\end{equation}
%
The complete system of equations is then
%
\begin{equation} 
  \begin{array}{ccl}
  F({\bf X}) & = & 
  \left[ \begin{array}{l}
  S_{sv,f0} + {\bf M}_{sv,f0} I_{NL,f0}({\bf X}) - U_{NL,f0}({\bf X}) \\
  S_{sv,f1} + {\bf M}_{sv,f1} I_{NL,f1}({\bf X}) - U_{NL,f1}({\bf X}) \\
  \vdots \\
  S_{sv,fm} + \dots \\
% {\bf M}_{sv,fm} I_{NL,fm}({\bf X}) - U_{NL,fm}({\bf X}) \\
  \end{array} \right] \\
  & = & {\bf 0} 
  \end{array}  \label{eq_final}
\end{equation}
%
The resulting system of equations is composed of $m+1$ systems each
with $n$ equations. If $f_0$ is denoted as being DC, then the
dimension of the system (and the number of unknowns) is $(2m+1)n$,
since for each state variable a real component is needed for DC, and a
real and imaginary component for each alternating frequency. This is
the minimum number of unknowns that can be achieved, and it only
depends on the nonlinear elements and the frequencies considered.

After solving~(\ref{eq_final}), the value of the state variable vector
is known, so finding the voltages (and the current of the ideal
voltage sources) for the entire network is straightforward using
(\ref{eq_voltage}).

Note that the entire analysis was performed in the frequency
domain. All the frequency mixing is performed in the functions of
(\ref{eq_nldev}), i.e., inside the nonlinear devices.  The particular
method used to calculate the currents and voltages from the state
variables in the frequency domain is not relevant here.


%--------------------------------------------------------------------------
\section{Implementation}

Figure~\ref{fig:genflow} (which is somewhat outdated now) shows the
flow diagram of the program. Note that the nonlinear equation solver
knows nothing about harmonic balance. This fact allows us to integrate
any general solver into the program. See chapter \ref{ch_arch} for
details.
%
\begin{figure}[htpb]
\centerline{\epsfxsize=2in \pfig{svHB_routine_flow_diagram.eps}}
\caption{General flow diagram of the SVHB analysis.} \label{fig:genflow}
\end{figure}
%

Multi-tone analysis is supported by using artificial frequency mapping
described in~\ref{subsection:freq}. This approach is more efficient
than the multidimensional Fourier transform, and it is simpler to
implement~\cite{kundert:vincentelli:90}. One of the advantages is that
the analysis remains essentially the same as with single-tone
excitation.


