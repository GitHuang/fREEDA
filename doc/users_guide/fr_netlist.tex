
\chapter{Netlist Format} \label{ch_netlist}

The netlist input of fREEDA is almost compatible to Spice.  There are
a number of additional features and these are documented below.  The
focus of the additions is to facilitate the addition of new models,
allow variables, and support hierarchical descriptions of coupling in
a network.

%--------------------------------------------------------------------------
\section{Structure of fREEDA$^{\mathrm{TM}}$'s Netlist}

The fREEDA netlist mainly consists in a title, an analysis
specification, a list of connected elements\footnote{element: a model
of a physical component of a network.}, and a list of output commands.

\subsection{Lexical}

fREEDAs grammatical rules are very similar to those of spice:
\begin{description}
\item[whitespace] blank\\
          a newline followed immediately by a $+$ sign.
          a tab\\
          a vertical tab\\
          a newpage
\item[identifier] A character sequence beginning with an Alphabetic
               character \[A-Za-z\]
\item[variables] A variable must begin with an alphabetic character or a \$
        followed by alphanumeric characters or `\_' or `.' \\[0.2in]
        Example: \\
        \begin{verbatim}
        HEIGHT
        \$height
        height.1_1
        \end{verbatim}
        Note that {\tt HEIGHT} and {\tt height} are identical as
        case is not preserved.
\item[strings] Either as an identifier (a continuous sequence of
alphanumeric characters or enclosed within double quotes. \\
           The following special escaped characters are allowed in
           strings defined within double quotes.
           \begin{itemize}
           \item[{\tt \\"}] To include a double quote in a string.
           \item[{\tt \\n}] To indicate a newline
           \end{itemize}
           Examples:
           \begin{verbatim}
           gate
           "VOLTAGE WAVEFORM"
           \end{verbatim}
           Note: Strings may continue across lines using the Spice
           continuation syntax:
           \begin{verbatim}
           "VOLTAGE
           + WAVEFORM"
           \end{verbatim}
           or simply by continuing across a line as in
           \begin{verbatim}
           "VOLTAGE
            WAVEFORM"
           \end{verbatim}

\item[numbers] ``E'' or ``e'' to indicate exponent.
\item[dotted command] A ``.'' folowed by alphabetic characters
\item[lf] A line feed or cariage return.
\end{description}

\subsubsection{Capitalization}

The case of identifiers and keywords is ignored in fREEDA$^{\mathrm{TM}}$  netlists.
The significance of case is retained only within quoted strings, and
in that case it is always retained.  Internally characters are mapped
to lower case.

\subsection{Continuation of Line}

A line beginning with a plus sign is considered to be the continuation
of the previous one.

\subsection{Title Line}

{\tt *** Unit Cell Folded Slot Antenna ***} \medskip \\
As in Spice, the first line of the netlist file is the title and does
not contain commands.

\subsection{Comments}

{\tt * Local reference nodes} \medskip \\
As in Spice, comment lines begin with an asterisk.

\subsection{.options}

Used to set up quantities similar to spice syntax.  The general syntax
is \medskip\\
{\tt .options <{\it identifier}> = <{\it value}>} \medskip \\
All identifiers set in a .options card are treated as a
variable.  {\it value} may be an number or a previously defined
variable.

\subsubsection{Preset Options}
Some variables are preset:

\begin{tabular}{lll}
{\bf variable} & {\bf default} & {\bf value} \\
defl    & OPTIONS\_DEFAULT\_DEFL & $100\times 10^{-6}$\\
defw    & OPTIONS\_DEFAULT\_DEFW & $100\times 10^{-6}$\\
defad   & OPTIONS\_DEFAULT\_DEFAD & 0 \\
defas   & OPTIONS\_DEFAULT\_DEFAS & 0\\
tnom    & OPTIONS\_DEFAULT\_TNOM & 27\\
numdgt  & OPTIONS\_DEFAULT\_NUMDGT & 4\\
cptime  & OPTIONS\_DEFAULT\_CPTIME & $1\times 10^6$\\
limpts  & OPTIONS\_DEFAULT\_LIMPTS & 201\\
itl1    & OPTIONS\_DEFAULT\_ITL1 & 40\\
itl2    & OPTIONS\_DEFAULT\_ITL2 & 20\\
itl4    & OPTIONS\_DEFAULT\_ITL4 & 10\\
itl5    & OPTIONS\_DEFAULT\_ITL5 & 5000\\
reltol  & OPTIONS\_DEFAULT\_RELTOL & 0.001\\
trtol   & OPTIONS\_DEFAULT\_TRTOL & 7.0\\
abstol  & OPTIONS\_DEFAULT\_ABSTOL & $1\times 10^{-12}$\\
chgtol  & OPTIONS\_DEFAULT\_CHGTOL &$0.01\times 10^{-12}$\\
vntol   & OPTIONS\_DEFAULT\_VNTOL &$1\times 10^{-6}$\\
pivrel  & OPTIONS\_DEFAULT\_PIVREL &$1\times 10^{-13}$\\
gmin    & OPTIONS\_DEFAULT\_GMIN &$1\times 10^{-12}$
\end{tabular}\\[0.2in]
(For the developer: the defaults are defined in spice.h )

\subsubsection{Control Options}

Under Construction.

\subsection{.model}

{\tt .model c\_line tlinp4 ( z0mag=75.00 k=7 fscale=1.e10 \\
+ alpha = 59.9 )} \medskip \\
Each {\tt .model} is a statement that associates a name ({\tt <{\it
model name}>}) with a list of parameter values ({\tt <{\it
param-list}>}).  The parameter names given must be the names of
parameters of the element specified after the model keyword.  Thus,
{\tt alpha} and {\tt z0mag} must be parameters of {\tt tlinp4} in the
above example.

Further, the values assigned to parameters must be of an appropriate
type.  The parser goes to some lengths to coerce types where the
result is sensible (i.e., if you give an integer value ``1'' to a
parameter of float type, the parameter will be assigned the
floating-point value ``1.0'').  However, you can't assign string
values to float parameters, or vice-versa.

The {\tt .model} statements define the default characteristics of the
different physical elements (``models'') in a network.

The syntax is as for spice \medskip \\
{\tt .model  <{\it model name}> <{\it type name}> ([<{\it parameter name}> =
<{\it value}>]*) }

Here, {\tt <{\it model name }>} is an identifier by which the model is
referred to.  {\tt <{\it type name}>} is the physical element name
that the model refers to.  the {\it parameter name} must be a valid
parameter for the element (indicated by {\tt <{\it type name}>})
referred to.

Any number of models may be specified for a single element.

\subsection{Analysis Specification}

{\tt .ac start = 3.6GHz stop = 4.8GHz n\_freqs = 7} \medskip \\
This line consists in a dot followed by the analysis name and a list
of parameters. See the analysis catalog for a list of analysis and the
description of the paramenters.

Note that 4.8GHz is equivalent to 4.8e9 or 4.8g. This is the same as
in Spice.

\subsection{Element Specification}

{\tt nport:cpw\_2 10 20 100 200 filename = "unitcell.yp"} \medskip \\
Elements are specified with the element type name ({\tt nport} in this
example) followed by a colon and the element instance name. Then a
list of nodes (or terminals) to which the element is connected
followed by a paramenter list. See the element catalog for a list of
available elements and the description of the paramenters.

The terminals can be named using integer numbers or strings. When
using strings, they must be enclosed in quotes.

Regular passive elements (R, L and C) also support the standard Spice
syntax with the following additions common to all elements in fREEDA$^{\mathrm{TM}}$:
\begin{enumerate}
\item A {\tt .model} specification is allowed for all elements.
\item Anything that can appear in a {\tt .model} specification can be
      included in the specification of the element.
\item If a parameter is not specified either through an  element
specification or a {\tt.model} specification then the default
parameters for that model will apply to this element.
\end{enumerate}

\subsection{End of Netlist}

Every netlist must finish with a {\tt .end} control card.


\subsection{Subcircuits}

The subcircuit definition and instantiation is pretty much as in
Spice. The definition may appear after or before the instatiation in
the netlist. Node names inside the subcircuit are local to the
subcircuit.  The following is an example for a three-terminal
subcircuit.
\begin{verbatim}

...

.subckt era6 1 5 "gnd1"

... (subcircuit definition)

.ends

...

xamp1 10 50 0 era6

...
\end{verbatim}
The name of the subcircuit instance must begin with {\tt x}.


%-------------------------------------------------------------------
\section{Output Control} \label{sec_output}

fREEDA$^{\mathrm{TM}}$ has an interpretive output language which uses a reverse
polish notation syntax.  The operators operate on a stack and as an
operation is performed zero or more arguements are consumed by an
opertor.  This is an extremely powerful way of controlling output.

\subsubsection{Output Commands}

{\tt .out write
( (<{\it qualifier}> <{\it value}>* )* <{\it operator}> )*
      in <{\it filename}>} \medskip
\\
or \medskip
\\
{\tt .out plot
( (<{\it qualifier}> <{\it value}>* )* <{\it operator}> )*
[[<{\it gnuplotPostambleScript}>] <{\it gnuplotPreambleScript}>]
in <{\it filename}>} \medskip
\\
or \medskip
\\
{\tt .out system <{\it string}>}

\subsection{Writing}

{\tt .out write
( (<{\it qualifier}> <{\it value}>* )* <{\it operator}> )*
      in <{\it filename}>} \medskip
\\
The {\tt write} command writes what is left on the stack into the file
{\it filename}.

\subsubsection{Example}

{\tt .out write term 4 vt in "4v.out"} \medskip
\\
This writes the time domain voltage at terminal 4 using the file
4v.out as an output file.

\subsection{Plotting}

{\tt .out plot
( (<{\it qualifier}> <{\it value}>* )* <{\it operator}> )*
[[<{\it gnuplotPostambleScript}>] <{\it gnuplotPreambleScript}>]
in <{\it filename}>} \medskip
\\
The {\tt plot} command writes what is left on the stack into the file
{\it filename} and initiates a plot. The file can be plotted
interactively using the fREEDA$^{\mathrm{TM}}$ Output Viewer. Also, a file named
{\tt<{\it filename}>.cmd} is created. This file is a gnuplot
\cite{gnuplot} script file that plots the desired data. The Scripts
are optional strings and are used to send additional commands to the
gnuplot program.

{\tt<{\it gnuplotPreambleScript}>} is a string of semicolon delineated
gnuplot commands prior to the plot command which is automatically
issued.

{\tt<{\it gnuplotPostambleScript}>} is a string of semicolon delineated
gnuplot commands after the plot command.

If the option {\tt gnuplot} is present in the {\tt .options} card, the
gnuplot program will be called automatically by fREEDA$^{\mathrm{TM}}$. Note that
this is generally not needed when using the Output Viewer.


\subsubsection{Example}

{\tt .out plot term 4 vt in "4v.out"} \medskip
\\
There are no script commands here. This plots the time domain voltage
at terminal 4 using the file 4v.out as an output file. This functions
as both a write and a plot.

\subsection{Running a System Command}

{\tt .out system <{\it string}>} \medskip
\\
Use this to run the string as a command to the operating system.

\subsubsection{Example}

{\tt .out system "echo Hello"} \medskip
\\
Prints ``Hello'' on the screen.

\subsection{Nomenclature}

The following nomenclature is used in describing the output operators.

\begin{tabular}{ll}
{\bf type} & {\bf description} \\
\multicolumn{2}{l}{\sl scalar numeric types} \\
\\
{\it i} & integer \\
{\it f} & floating-point \\
{\it r} & real (integer or floating-point) \\
{\it c} & complex \\
{\it s} & scalar (integer, floating-point or complex) \\
\\
\multicolumn{2}{l}{\sl scalar and mixed numeric types} \\
\\
{\it fv} & floating-point vector \\
{\it cv} & complex vector \\
{\it v} & floating-point or complex vector \\
{\it fsv} & floating-point scalar or vector \\
{\it csv} & complex scalar or vector \\
{\it sv} & scalar or vector (any) \\
{\it prom} & an appropriately-promoted numeric type \\
{\it -x} & (suffix to vector types) x data required \\
\\
\multicolumn{2}{l}{\sl other types} \\
\\
{\it any} & any type \\
{\it string} & character string \\
{\it var} & variable name \\
{\it file} & data file \\
{\it func} & function pointer
\end{tabular}

\subsection{Qualifiers}

\begin{tabular}{ll}
{\bf type} & {\bf description} \\
\\
\multicolumn{2}{l}{\sl qualifiers  (network types)} \\
\\
{\it term} & terminal (or node)\\
{\it element} & circuit element
\end{tabular}


\subsection{Operators}


\subsubsection{General Operators}

\begin{tabular}{p{.8in}p{2.5in}p{1.0in}p{.75in}}
{\bf operator} & {\bf function} & {\bf argument(s)} & {\bf result} \\
\\
\\
{\tt dup} & duplicate object & {\it any} & {\it same} \\
{\tt get} & get element of vector & {\it arg:v \newline index:i} & {\it s} \\
{\tt put} & modify element of vector & {\it arg:v \newline index:i \newline
  val:s} & {\it v} \\
{\tt stripx} & remove x data & {\it vx} & {\it v} \\
{\tt pack} & concatenates last {\it vx}'s on stack & variable number of {\it vx} & {\it m} \\
{\tt system} & execute shell command & {\it string} & {\it none}
\end{tabular}

\subsection{Network Operators}

\begin{tabular}{p{.8in}p{2.5in}p{1.0in}p{.75in}}
{\tt vf} & complex freq. domain voltage vector at a terminal &
    {\it term} & {\it cv} \\
{\tt if} & complex freq. domain current vector at a terminal &
    {\it term} & {\it cv} \\
{\tt xf} & complex freq. domain state variable vector at a terminal &
    {\it term} & {\it cv} \\
{\tt vt} & time domain voltage vector at a terminal &
    {\it term} & {\it fv} \\
{\tt it} & time domain current vector at a terminal &
    {\it term} & {\it fv} \\
{\tt ut} & time domain voltage vector at an element port &
    {\it elem} & {\it fv} \\
\end{tabular}

\subsubsection{RPN Arithmetic Operators}
Arithmetic Operators for reverse polish notation
e.g. 3 4 add = 7

\begin{tabular}{p{.8in}p{2.5in}p{1.0in}p{.75in}}
{\tt add} & addition & {\it sv \newline sv} & {\it prom} \\
{\tt sub} & subtraction & {\it sv \newline sv} & {\it prom} \\
{\tt mult} & multiplication & {\it sv \newline sv} & {\it prom} \\
{\tt div} & division & {\it sv \newline sv} & {\it prom} \\
{\tt real} & real part & {\it csv} & {\it fsv} \\
{\tt imag} & imaginary part & {\it csv} & {\it fsv} \\
{\tt mag} & magnitude & {\it csv} & {\it fsv} \\
{\tt abs} & absolute value or magnitude & {\it sv} & {\it fsv} \\
{\tt contphase} & continuous phase & {\it csv} & {\it fsv} \\
{\tt prinphase} & principal value phase & {\it csv} & {\it fsv} \\
{\tt conj} & complex conjugate & {\it csv} & {\it csv} \\
{\tt neg} & additive inverse (negative) & {\it sv} & {\it sv} \\
{\tt recip} & reciprocal & {\it sv} & {\it sv}
\end{tabular}

% \subsection{Conventional Arithmetic Operators}
% Conventional Arithmetic Operators
% e.g. 3 + 4 = 7
% Not fully implemented. Do not use and reserved for future expansion.
%
% \begin{tabular}{p{.8in}p{2.5in}p{1.0in}p{.75in}}
% {\verb:+:} & addition & {\it sv \newline sv} & {\it prom} \\
% {\verb:-:} & subtraction & {\it sv \newline sv} & {\it prom} \\
% {\verb:*:} & multiplication & {\it sv \newline sv} & {\it prom} \\
% {\verb:/:} & division & {\it sv \newline sv} & {\it prom} \\
% {\verb:^:} & poer & {\it sv \newline sv} & {\it prom} \\
% {\verb:**:} & poer & {\it sv \newline sv} & {\it prom} \\
% \end{tabular}

\subsubsection{Mathematical Operators}

\begin{tabular}{p{.8in}p{2.5in}p{1.0in}p{.75in}}
{\tt db} & dB ($20 \log_{10}$) & {\it sv} & {\it fsv} \\
{\tt db10} & dB applied to power ($10 \log_{10}$) & {\it sv}
    & {\it fsv} \\
{\tt rad2deg} & convert radians to degrees & {\it fsv} & {\it fsv} \\
{\tt deg2rad} & convert degrees to radians & {\it fsv} & {\it fsv} \\
{\tt minlmt} & limit the minimum value & {\it arg:fsv \newline min:f}
    & {\it fsv} \\
{\tt maxlmt} & limit the maximum value & {\it arg:fsv \newline max:f}
    & {\it fsv} \\
{\tt diff} & differences & {\it fsv} & {\it fsv} \\
{\tt deriv} & derivative & {\it fsv} & {\it fsv} \\
{\tt sum} & sums & {\it fsv} & {\it fsv} \\
{\tt integ} & integral & {\it fsv} & {\it fsv}
\end{tabular}

\subsubsection{Signal Processing Operators}

\begin{tabular}{p{.8in}p{2.5in}p{1.0in}p{.75in}}
{\tt smpltime} & current analysis timebase as x {\em and} y of result &
  {\it none} & {\it fv} \\
{\tt sweepfrq} & current analysis sweep frequencies as x {\em and}
  y of result & {\it none} & {\it fv} \\
{\tt smplcvt} & interpolate {\em signal1} over timebase of {\em signal2} &
  {\it signal1:v \newline signal2:vx} & {\it vx} \\
{\tt sweepcvt} & interpolate {\em frq1} over sweep frequencies of {\em frq2} &
  {\it frq1:v \newline frq2:vx} & {\it vx} \\
{\tt maketime} & create timebase starting at $t=0$ in x {\em and} y of result &
  {\it tmax:r \newline pts:i} & {\it vx} \\
{\tt makesweep} & create sweep frequencies starting at $f=0$ in x
  {\em and} y of result & {\it fmax:r \newline pts:i} & {\it vx} \\
{\tt fft} & FFT (argument should have $2^{k}$ points) &
  {\it timedata:fv} & {\it cv} \\
{\tt invfft} & inverse FFT (argument should have $2^{k}-1$ points) &
  {\it frqdata:cv} & {\it fv} \\
{\tt cconv} & real circular (FFT) convolution with zero padding &
  {\it signal1:fv \newline signal2:fv} & {\it fv} \\
{\tt upcconv} & unpadded real circular (FFT) convolution &
  {\it signal1:fv \newline signal2:fv} & {\it fv} \\
{\tt sconv} & slow (time-domain) real convolution &
  {\it signal1:fv \newline signal2:fv} & {\it fv} \\
{\tt fconv} & fast (approximate) real convolution &
  {\it signal1:fv \newline signal2:fv} & {\it fv} \\
{\tt lpbwfrq} & lowpass Butterworth filter frequency response &
  {\it frqvec:vx \newline corner:f \newline order:i} & {\it cvx}
\end{tabular}



%--------------------------------------------------------------------------
\section{Example: Simulation of a Folded Slot Antenna}

The netlist format is illustrated using an example. This example uses
local reference nodes. For a discussion of the local reference node
concept see chapter \ref{ch_local}. fREEDA$^{\mathrm{TM}}$ provides the local
references as a convenience tool, but it is still possible to treat
circuits in a conventional way using the node ``0'' or ``{\tt gnd}''
as a global reference.

As a component of a spatial power combining circuit the CPW
folded-slot antenna~\cite{rodwell}, see Fig.~\ref{fig:cpwunit}, with
polarizers transmits an amplified version of an incident propagating
field. In Fig.~\ref{fig:cpwunit} the two orthogonal folded-slots are
connected to each other by a CPW with an inserted MMIC amplifier.The
system is modeled using the circuit of Fig.~\ref{fig:cpwcircuit} where
electromagnetic modeling of this structure is discussed in
\cite{steer:abdullah:1998,mostafa,usman}. Note that three different
local reference nodes are required.
%
\begin{figure*}
\centerline{\epsfxsize=10cm \pfig{carlos_cpwUnit.eps}}
\caption{Unit cell of the CPW antenna array.} \label{fig:cpwunit}
\end{figure*}
%
\begin{figure}
\centerline{\epsfxsize=9cm \pfig{freeda_cpwcircuit.eps}}
\caption{Circuit model of the unit cell. The diamond symbol indicates
a local reference node.} \label{fig:cpwcircuit}
\end{figure}
%
EM modeling yields a port-based $y$ parameters of the antennas at each
frequency of interest. The transfer of data between the EM and circuit
simulators (typically a file) includes a header with port grouping
information (a port grouping includes terminals associated with a
specific local reference node). This is required by the circuit
simulator in order to expand the port-based matrix into nodal form and
also to check the connectivity of the spatially-distributed circuit.

Below is shown the data file for this example. Each port belong to a
different group, so the element has four terminals. After reading the
header the circuit simulator knows the number of elements of the
matrix and the port number and local reference corresponding to each
row and column of the matrix.

\begin{verbatim}
# port:group
1:1
1:2
# GHZ Y RI R 50
4
0.00355603      -0.0233196
-0.00121905     -0.00496212
-0.00121905     -0.00496212
0.00355603      -0.0233196
...
\end{verbatim}

The rest of the file consist in a list of frequencies and the
associated matrix elements in complex form.

The parser provides several facilities such as the {\tt .model}
statement support for any element type and a complete reverse polish
notation calculator for the output data. The corresponding netlist is
shown below.

{
\begin{verbatim}
*** Unit Cell Folded Slot Antenna ***

.ac start = 3.6GHz stop = 4.8GHz n_freqs = 7

* Local reference nodes
.ref 100
.ref 200

* CPW structure
nport:cpw_2 10 20 100 200 filename = "unitcell.yp"

* Transistor small signal model
nport:amplifier 1 2 0 filename = "feedback_ne3210s1.yp"

* Field excitation
gridex:iin 10 100 20 200 ifilename = "unitcell.i"
+ efilename = "dummy.e"

* current meters
vsource:amp1 10 11
vsource:amp2 20 22

* CPW Transmission lines
.model fsa1 cpw (s=.369m w=1m t=10u er=10.8 tand=.001)
cpw:t1 11 100 1 0 model="fsa1" length=8.5m
cpw:t2 22 200 2 0 model="fsa1" length=17.5m

.out write element "vsource:amp1" 0 if in "i1.out"
.out write element "vsource:amp2" 0 if in "i2.out"

* Plot magnitude of current gain
.out plot element "vsource:amp2" 0 if element
+ "vsource:amp1" 0 if div mag db in "igain.out"

* Plot magnitude of voltage gain
.out plot term 20 vf term 10 vf div mag db in "gain.out"

.end
\end{verbatim}}

