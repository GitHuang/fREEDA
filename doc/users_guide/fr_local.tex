
\chapter{Local Reference Node} \label{ch_local}


\section{Introduction}

The distributed nature of many microwave and millimeter-wave circuits
necessitates electromagnetic modeling. Thus the full application of
computer-aided engineering (CAE) to these circuits requires the
integration of electromagnetic models of distributed structures with
conventional circuit analysis. In the formative stages of the CAE of
microwave circuits, ports were used in specifying the
interconnectivity of networks.  The utilization of ports avoided the
issue of specifying reference nodes.  In analysis, using matrix
manipulations or signal flow graphs, one of the terminals of a port
was used implicitly as a reference node and generally ignored in
formulating the mathematical model.  In one commercial nonlinear
microwave simulator ports are still used in circuit specification.
The connection of discrete elements is specified nodally but at the
highest level of the hierarchy port-based descriptions are used. While
it is possible to analyze any circuit with this arrangement it becomes
increasingly difficult to specify the connections of large spatially
distributed circuits, and also to specify and extract desired output
quantities.  The alternative to using a port-only descriptions is to
exclusively use the nodal connectivity description the only method
used in general purpose circuit simulators.  The conventional nodal
specification enables circuit elements to be connected in any possible
combination and only one reference node (commonly called the global
reference node or simply ground) is used.  So nodal specification
provides tremendous flexibility in circuit specification but the
multiple reference nodes leads to M-fold indefiniteness of the
(modified) nodal admittance matrix.  Also, with spatially distributed
circuits it is possible to make illegal connections such as connecting
a non-spatially distributed element, say a resistor, across a
spatially distributed element, \emph{e.g.} a transmission line.  The
purpose of this paper is two-fold: firstly to describe an
implementation of the local reference node theory described by Khalil
and Steer in Reference .  Secondly to develop and describe the
implementation of a scheme for checking the legal connections of a
network with mixed spatially distributed and conventional elements.



\section{Background}

A circuit is a graphical construct (which can be specified in textual
form as a netlist) for coupling together algebraic and first order
differential equations. These equations arise from the constitutive
relations of the individual elements, which specify the actual form of
the individual equations, and from Kirchoff's current law which
specifies how the equations are coupled. With but a few modifications,
the analysis thus described is called nodal admittance analysis. It is
important to note that circuits have no sense of space a circuit is
defined as though it existed at a point. The problem is then how to
incorporate an electromagnetic model of a structure that is inherently
distributed in space. Of two approaches, one is to insert the device
equations into an appropriate time-stepping electromagnetic simulator
such as a finite difference time domain (FDTD) simulator
\cite{imtiaz,kuo}. This reduces the level of abstraction of the
`circuit' and embeds the constitutive relations of the conventional
circuit elements into the analysis grid of the FDTD method. An
alternative is to retain the high-level circuit abstraction and
incorporate the results of a field analysis (the spatially distributed
circuit) into a circuit structure
\cite{larique,todd1,antenna,kunisch}. Thus the problem is one of
taking an electromagnetic solution and using a physically consistent
approach to inserting it is a circuit element. This is a less direct
approach, that is more abstract, but potentially much more general and
compatible with the large body of existing computer-aided circuit
analysis theory and knowledge that has been developed.

At this point it is useful to review modified nodal admittance (MNA)
analysis to indicate the basis for the work presented here. MNA
analysis was developed to handle elements that do not have nodal
admittance descriptions.  For each such element one or more additional
equations are added to the nodal admittance equations and these new
equations become additional rows and columns in the evolving matrix
system of equations.  A similar approach can be followed for the
electromagnetic elements.  The process is a little more sophisticated
as it is no longer sufficient to add additional rows.  Instead the
concept of local reference nodes was developed
\cite{local:reference:node:khalil} as a generalization of the
compression matrix approach \cite{kunisch}. This concept provides
another way to incorporate alternate equations in the evolving MNA
matrix. However, rather than adding additional constitutive relations,
the local reference node concept changes the way the port-based
parameters are used.  shows a circuit with a spatially distributed
circuit and with local reference nodes indicated by the diagonal
symbol.  In a conventional circuit only one reference node (ground) is
possible so that application of KCL to the global reference node
introduces just one additional redundant row and column in the
indefinite form of the MNA matrix. For a spatially distributed
circuit, KCL is applied to each locally referenced group one at a time
and, as the local reference nodes are electrically connected only
through a spatially distributed element, each application results in a
redundant row and column in the MNA matrix. This is a particular
property of the port-based characterization of the spatially
distributed element \cite{local:reference:node:khalil}. Thus only
appropriate electromagnetic analyses are suitable. The mathematical
description of the required extension to the nodal admittance analysis
is given in \cite{local:reference:node:khalil}.

\section{Method}

The basis of the method is that in a multi-port element, the terminals
of ports with different local reference nodes can be considered
isolated.  For example, if we write the equivalent circuit of a
two-element derived from the port-based admittance ($y$) parameters,
\[
  \left[ \begin{array}{c}
    i1 \\
    i2 \end{array} \right] =
 \left[ \begin{array}{cc}
    y_{11} & y_{12} \\
    y_{21} & y_{22}
  \end{array} \right]
  \left[ \begin{array}{c}
    v1 \\
    v2 \end{array} \right]
\]
we obtain the equivalent circuit in Fig. \ref{fig:2_port_y}. Note that
no current can flow
%
\begin{figure}[htpb]
\centerline{\epsfxsize=2.5in \pfig{2_port_y.eps}}
\caption{Equivalent circuit of a two-port distributed element.}
\label{fig:2_port_y}
\end{figure}
%
between the two ports. This type of circuit model can be generalized
for an n-port element where external and internal local reference
nodes are defined. The local references shown in are internal because
they are used internally in the element to measure the voltages at its
ports. An external local reference terminal, on the other hand, is an
arbitrarily chosen terminal from a locally referenced group. In
general, for any spatially distributed circuit
(Fig. \ref{fig:port_node_2}) there is no current between port groups
inside the spatially-distributed element. Thus the circuit can be
divided into subcircuits, each with a local reference terminal, as
shown in Fig. \ref{fig:potatoes}. Each subcircuit is isolated with
respect to the others, so there is no change in the solution if all
the reference nodes are connected together. The difference is that now
there is only one global reference node for all the circuit, and
standard nodal methods can be used to formulate the circuit equations.
%
\begin{figure}[htpb]
\centerline{\epsfxsize=2.5in \pfig{port_node_2.eps}}
\caption{Generic spatially distributed circuit.} \label{fig:port_node_2}
\end{figure}
%
%
\begin{figure}[htpb]
\centerline{\epsfxsize=2.in \pfig{potatoes.eps}}
\caption{The locally referenced groups are isolated,
therefore the solution is unchanged if the local references
are connected.} \label{fig:potatoes}
\end{figure}
%
%
\begin{figure}[htpb]
\centerline{\epsfxsize=2.5in \pfig{current_loop.eps}}
\caption{Illegal connection between groups.} \label{fig:current_loop}
\end{figure}
%
The problem becomes that of detecting violations to the assumption
that each locally referenced group is isolated from the others and
that there is exactly one reference terminal for each one. Such a
situation is shown in Fig. \ref{fig:current_loop}. The effect of this
connection is the creation of a current loop that is non-physical
since the two circuits cannot be connected instantaneously. On the
other hand, it is valid for two subcircuits to be connected by more
than one spatially distributed element. It is also possible for two
ports of a spatially distributed element to be connected to the same
locally referenced group (for example a delay line). In that case, the
external local reference node would be the same for both ports of the
line, but internally, the line still has two local reference nodes.
After all the checking is done, all the local reference nodes are
merged into a single global reference node, and the solution of the
circuit is found by standard procedures. The nodal voltages found are
coincident with the voltages referred to each local reference node.


\section{Implementation}

Implementation of the technique can be described using a conventional
procedural approach or an object oriented approach.  While not obvious
to those not conversant with object oriented practice, the object
oriented view maps onto the circuit analysis problem cleanly.  A
circuit is a collection of objects (resistors, inductors, transmission
lines, etc.) that are related to each other (they share nodes).  With
the introduction of spatially distributed elements along with the
conventional elements there are certain rules that describe allowable
interconnections. The rule is \medskip \\ \emph{Spatially distributed
objects can be connected together in any manner but the
interconnections of non-spatially distributed elements cannot span
spatially distributed elements.} \medskip \\ In fREEDA, each element
and each terminal is a node in a graph structure. The algorithm to
detect violations is a variation of the depth-search algorithm and is
best illustrated using what is called a collaboration diagram. The
algorithm could also be implemented using conventional procedural
programming \cite{cormen:90}. The collaboration diagram \cite{rational}
is part of the Unified Modeling Language (UML), a language for
specifying, visualizing, and constructing the artifacts of software
systems as well as for business modeling. The UML represents a
collection of `best engineering practices' that have proven successful
in the modeling of large and complex systems. Behavior is implemented
by sets of objects that exchange messages within an overall
interaction to accomplish a purpose. To understand the mechanisms used
in a design, it is important to see only the objects and the messages
involved in accomplishing a purpose or a related set of purposes,
projected from the larger system of which they are part for other
purposes. Such a static construct is called a collaboration. A
collaboration is a set of participants and relationships that are
meaningful for a given set of purposes. The identification of
participants and their relationships does not have global meaning. In
the collaboration diagram, each box represents an object (in this
case, either a terminal or an element). The lines between boxes
represent associations and the arrows are messages. The order in which
messages are sent is shown by the numbers.

%
\begin{figure}[htpb]
\centerline{\epsfxsize=2.5in \pfig{circuit_view.eps}}
\caption{Circuit containing an illegal element. The numbers at the
nodes indicate the order in which terminals are discovered.}
\label{fig:circuit_view}
\end{figure}
%
%
\begin{figure}[htpb]
\centerline{\epsfxsize=3in \pfig{collaboration_error.eps}}
\caption{Collaboration diagram showing the violation detection
algorithm.} \label{fig:collaboration_error}
\end{figure}
%
Figure \ref{fig:circuit_view} shows the circuit view corresponding to
the collaboration diagram of Figure \ref{fig:collaboration_error}. The
terminal numbers indicate the order in which terminals are discovered
by the algorithm. Note also that the internal local reference
terminals of the spatially distributed element do not need to be
coincident with the external references. The search begins at each
local reference terminal. The local reference terminal sends a message
to propagate its identifier (id) to all the adjacent elements. If the
element that receive the message has only one local reference terminal
(lumped), it continues the propagation to the other adjacent
terminals. Otherwise it only propagates the id to the element
terminals with the same internal reference. If an element attempts to
propagate a reference id to a terminal already marked with another
reference id, then a violation is detected: there is an illegal
connection between two locally referenced groups. A check to detect
floating parts of the circuit can be made by finding any terminal that
does not belong to any locally referenced group.  All these checks are
implemented in fREEDA. Note that the bias circuit often violates the
condition that the locally referenced groups are isolated between
them. To overcome this either the distributed elements should include
the DC characteristics or an independent DC source should be
considered in each group.  In the input netlist, the local reference
nodes are identified with the command {\tt.ref <{\it terminal}>}. By
default the nodes named 0 and {\tt gnd} are assumed to be reference
nodes.


