
\chapter{Convolution Transient Analysis} \label{ch_svtr}

%--------------------------------------------------------------------------
\section{Introduction}

Transient analysis of distributed microwave circuits is complicated by
the inability of frequency independent primitives (such as resistors,
inductors and capacitors) to model distributed circuits. More
generally, the linear part of a microwave circuit is described in the
frequency domain by network parameters especially where numerical
field analysis is used to model a spatially distributed
structure. Inverse Fourier transformation of these network parameters
yields the impulse response of the linear circuit. This has been used
with convolution to achieve transient analysis of distributed
circuits~\cite{basel:paper,Brazil:New,gamma,delta}.

The major drawback of convolution analysis has been the large time and
memory requirements that result from recording and convolving the
nodal voltages of the nonlinear elements. The situation worsens
because of convergence considerations which require small time
steps. This chapter develops a convolution-based transient analysis
which uses state variables instead of node voltages to capture the
nonlinear response. This has two desirable effects. First, the number
of state variables required is less than the number of nodes of the
nonlinear elements and so fewer quantities need to be recorded and
convolved. Second, parameterized device models can be used as the
nonlinearity is not restricted to the form of current as a nonlinear
function of voltage. Parameterized device models result in stable
transient analysis for larger time steps and so less discretized time
history is required. Also with improved stability constant time steps
can be used further improving the robustness of the convolution.

Observation about efficiency of state-variable approach: if we have
only one element that uses convolution and we use SPICE-like
transient, the convolution would only be performed at the ports of the
convolution-based element. That approach is probably much more
efficient for a large group of circuits. In the case of quasi-optical
systems, since a big part of the circuit needs to be treated using
convolution, our approach can be equally or even more efficient if the
number of nonlinear element ports is less than the number of
convolution-based element ports.



%--------------------------------------------------------------------------
\section{Background}

Distributed linear microwave circuits are described by
frequency-dependent network parameters and only a few methods are
available to accommodate these circuits in transient simulation. These
methods include Impulse Response and Convolution (IRC), Asymptotic Waveform
Evaluation (AWE), and Laplace Inversion.

\subsection{Impulse Response and Convolution}

One of the first implementations of IRC to distributed microwave
circuits was by Djordjevic and Sarkar~\cite{sarkar} in 1987.  Since
that time there have been several efforts to reduce the significant
aliasing errors that result from the Inverse Fast Fourier Transform
(I-FFT) operation.

Minimization of aliasing in the I-FFT requires that the imaginary part
of the frequency response be zero at the maximum frequency. Low pass
filtering has been used to achieve this~\cite{alpha}. The introduction
of a small time delay also achieves this result with presumably less
distortion~\cite{delta}. Insertion of an augmentation network in the
linear network at the interface with the nonlinear network achieves
the same result for special types of circuits~\cite{basel:paper}. The
effect of the augmentation network is compensated in the nonlinear
iteration scheme. This, however, is not a general technique. It is
also important to limit the length of the impulse response to reduce
memory requirements and the resistive augmentation achieves this
result~\cite{basel:paper}. Augmentation is also effectively achieved
using $s$ parameters~\cite{theta}, where the reference impedance
effectively dampens multiple reflections.

Distributed networks are characterized partly by reflections so that
an impulse response tends to have regions of low value between regions
of rapid change. In this case thresholding greatly reduces the number
of impulse response discretizations that need to be
retained~\cite{basel:paper}. In high speed digital interconnect
circuitry this can reduce the number of significant impulse responses
by a factor of 10 to 100, depending on the desired
accuracy~\cite{basel:paper}.

Convolution as generally practiced uses a rectangular integration
scheme (essentially an impulse response is treated as being constant
in a time step interval). P. Stenius \emph{et al.}~\cite{aplac}
developed a trapezoidal form of the convolution integration which
should have superior convergence properties than the previous block
integration.

\subsection{Asymptotic Waveform Evaluation}

The frequency dependent network parameters can be modeled using
frequency independent primitives (resistors, inductors and capacitors)
if a rational polynomial transfer function is fitted to the network
parameters. In practice, this procedure results in an impossibly large
circuit. The AWE method addresses this problem by reducing the
dimension of the rational polynomial while minimizing
distortion~\cite{chan:AWE,multi:pade}. This method works well for
interconnects in digital systems and lower frequency microwave
circuits~\cite{Nakhla:Eli,Trithy}. However, higher frequency circuits
can only be modeled approximately using AWE due to the infinite number
of poles and zeros of such a circuit.

Most of the AWE methods use the Pad\'e approximation and this and
other approximations used can have stability problems~\cite{chan:AWE}.


\subsection{Numerical Inversion of Laplace Transform Technique}

This technique does not have aliasing problems since it does not
assume that the function is periodic. The inverse transform exists for
both periodic and non-periodic functions. There is no causality
problem for double sided Laplace Transforms, either. Unlike FFT
methods, the desired part of the response can be achieved without
doing tedious and unnecessary calculations for the other parts of the
response. Laplace techniques suffer from the series approximations and
the nonlinear iterations involved. The advantages and the limitations
of the Inverse Laplace Methods are discussed in detail in
\cite{Nakhla:Grif}.


%--------------------------------------------------------------------------
\section{Formulation of the Transient Analysis}

The system being modeled is partitioned into linear and nonlinear
subnetworks\footnote{The linear/nonlinear classification is not strict
here since some 'linear' elements such as sources or sometimes
inductors and capacitors can be treated in the time domain, for
example to avoid problems with the impulse response.} with the linear
subnetwork comprising the spatially distributed network characterized
by EM analysis. This yields a port-based admittance matrix which is
augmented by other linear circuit elements. Each nonlinear subnetwork
interfaces with the linear network via an LRG of nodes which contains
a single LRN \cite{local:reference:node:khalil}.  Sources are included
in the nonlinear subnetworks as they must be treated in the time
domain.  From the frequency- and time-domain state variables (${\bf
X}(f), {\bf x}(t)$), describing the state of the nonlinear
subnetworks, the voltages with respect to the LRN's, (${\bf
V}_{NL}(f), {\bf v}_{NL}(t)$), and the currents flowing into the
nonlinear networks at the interface nodes other than the LRN's (${\bf
I}_{NL}(f), {\bf i}_{NL}(t)$) are determined by the constitutive
relations of the nonlinear elements so that ${\bf I}_{NL}(f)$ $=$
${\bf I}_{NL}({\bf X},f)$ and ${\bf V}_{NL}(f)$ $=$ ${\bf V}_{NL}({\bf
X},f)$.  As voltages and currents at the element's reference nodes are
not considered, the number of state variables, $n_S$, is equal to the
number of interface ports.  With the current flowing into the linear
network ${\bf I}_L(f)$ $=$ $-{\bf I}_{NL}(f)$, the frequency domain
voltages at the interface required by the linear network is
%
\begin{equation}
  {\bf V}_{L}({\bf X},f) = {\bf M}_{SV}{\bf I}_{NL}({\bf X},f) \label{eq.fd}
\end{equation}
%
where ${\bf M}_{SV}$ is the state variable impedance
matrix~\cite{ims99} and it is obtained from a modified nodal
admittance matrix that supports the concept of local reference
nodes~\cite{local:reference:node:christoffersen}.  Following discrete
Fourier transformation the $i$ th component of ${\bf V}_{L}(f)$ at the
discretized time step $n_t$ is
%
\begin{equation}
  v_{Li}(n_t) = \sum_{j=1}^{n_S} v_{Lij}(n_t)
\end{equation}
%
the contribution of the current at the $j$ th interface node to the
voltage at the $i$ th node is \cite{svtr}\footnote{Here we assume that
$m_{i,j}(0)$ is already divided by two after the IFFT operation
\cite{aplac}.}
%
\begin{eqnarray}
\lefteqn{{v_L}_{i,j}({\bf \hat{X}},n_t) =} \nonumber \\
 & &	\left\{ \begin{array}{l}
	   {m_{i,j}(0) \: i_{NL}({\bf \hat{X}},n_t)} \\
           + \: \sum_{n_{\tau}=1}^{n_t - 1} m_{i,j}(n_{\tau}) \:
                    i_j(n_t - n_{\tau}), \\
	\multicolumn{1}{r}{\mbox{if $n_t < N_T$}} \\
	   {m_{i,j}(0) \: {i_{NL}}_j({\bf \hat{X}},n_t)} \\
           + \: \sum_{n_{\tau}=1}^{N_T-1} m_{i,j}(n_{\tau}) \:
                    i_j(n_t - n_{\tau}), \\
	\multicolumn{1}{r}{\mbox{if $n_t \ge N_T$}}
	\end{array} \right. \label{eq.conv} 
\end{eqnarray}
%
where $m_{ij}(n_t)$ is the I-FFT of ${\bf M}_{SV}(f)$ with duration
$N_T$ and is the discrete impulse response contribution of the current
at the $j$ th interface node to the voltage at the $i$ th node.  Note
that $m_{ij}(n_t)$ $=$ $0$ for $n_t$ $<$ $0$ and for $n_t$ $\ge$
$N_T$. Following application of this the error function vector ${\bf
f}$ $=$ $[f_1 \dots f_i \dots f_{n_S}]^{\mbox{\scriptsize T}}$ is
formed where
\begin{equation}
  f_i = \sum_{j=1}^{n_S} v_{Li,j}(n_t) - v_{NLi}(n_t)
\label{eq:error}
\end{equation}
%
and in vector form, the complete system is
%
\begin{eqnarray}
{\bf F}({\bf \hat{X}}) = 
	  \left[ \begin{array}{l} 
	  \sum_{j=1}^{n_s} {v_L}_{1,j}({\bf \hat{X}},n_t) - 
				  {v_{NL}}_1({\bf \hat{X}},n_t)  \\
	  \sum_{j=1}^{n_s} {v_L}_{2,j}({\bf \hat{X}},n_t) - 
				  {v_{NL}}_2({\bf \hat{X}},n_t) \\ 
	  \vdots\\
	  \sum_{j=1}^{n_s} {v_L}_{n_s,j}({\bf \hat{X}},n_t) - 
				  {v_{NL}}_{n_s}({\bf \hat{X}},n_t) 
	  \end{array} \right] \label{error_function}
\end{eqnarray}
%
The error function in Eq.~(\ref{error_function}) is solved for
each time step.


%--------------------------------------------------------------------------
\section{Implementation}

The formulation developed above resulted in a standard nonlinear
problem that can be solved using the Newton method or quasi-Newton
method. As the error function changes only slightly from time step to
time step, efficient iterative matrix solve schemes can be used as a
very good preconditioner is available from the previous time step.
The general flow of the analysis is shown in
Figure~\ref{transient_flow} and was implemented in the freeda$^{\mathrm{TM}}$
program.
%
\begin{figure}[ht]
\centerline{\epsfxsize=1.6in \pfig{transient_flow.eps}}
\caption{Flow diagram of the analysis code} \label{transient_flow}
\end{figure}
Discretization errors and aliasing errors arise in developing the
discrete impulse response. Also the time required to perform
convolution-based transient analysis is very long if a sinusoidal
steady-state response is to be determined at the same time as
evaluation of the dc operating level.  This section presents two
workarounds to these problems.


\subsection{Impulse Response Determination}

The impulse response is obtained using the inverse real Fourier
transformation of each element of ${\bf M}_{SV}$. The transform
requires special characteristics of the frequency domain variables at
high frequencies so that aliasing is minimized. Problems occur with
ideal inductors and capacitors as the matrix elements can go to
infinity. This can be corrected by cascading a resistive augmentation
circuit with the linear circuit and so ensure finite
parameters~\cite{mete,basel:paper}. The use of scattering parameters
achieves similar resistive augmentation~\cite{theta}. Being resistive,
the effect of the augmentation network can be removed in transient
simulation as the resistors are unaffected by the Fourier
transformation. The current work uses the resistive augmentation
network shown in Fig.~\ref{augmentation}. The advantage of this
topology is that no extra nodes are added to the circuit and the size
of the MNAM is not changed. The augmentation network provides a better
matrix conditioning for the I-FFT operation at the expense of
increased error due to finite numerical accuracy. Ideally, the effect
of the augmentation would be exactly compensated.
%
\begin{figure}[ht]
\centerline{\epsfxsize=2.7in \pfig{augmentation.eps}}
\caption{Augmentation and compensation network}
\label{augmentation}
\end{figure}
%
The resistive augmentation network also serves to limit the extent of
the impulse response as energy is absorbed and multiple reflections
are damped. Again, note this effect is fully compensated.

Before converting the frequency domain ${\bf M}_{SV}$ (impedance-like)
matrix to the time domain, the imaginary part of the frequency
response needs to be band-limited so that the function has a periodic
(or circular) frequency response. This significantly reduces the
aliasing errors in the I-FFT operation.

In freeda$^{\mathrm{TM}}$, frequency response limiting is implemented by filtering
the last part of the frequency spectrum of each matrix element so the
imaginary part goes to zero and the real part goes to the DC value. In
this way, the frequency response is periodic. This is in contrast to
low pass filtering~\cite{basel:paper} which zeroes the high frequency
response and introducing additional time-delay to take the imaginary
response to zero~\cite{Brazil:New}. The filtering factor $k_i$ is,
\begin{equation} 
  k_i = \left(\frac{n-i}{n-i_0}\right)^2
\end{equation}
where $n$ is the number of discrete frequencies, $i$ is the frequency
index and $i_0$ is the frequency index at which filtering starts.
For all $i$ greater than $i_0$, the filtering operation on each state
variable impedance matrix entry is
\begin{eqnarray}
  \Re\{m_i\} &=& k_i \times \Re\{m_i\} + (1 - k_i) \times m_0 \\
  \Im\{m_i\} &=& k_i \times \Im\{m_i\}
\end{eqnarray}
The amount of filtering is an analysis option since in some cases it
is not necessary. By default, only the last 3\% of the spectrum is
filtered to achieve an acceptable alias reduction.  If this is not
done the correct impulse response is not assured as the continuity
condition is inherent in the I-FFT (or FFT) operation.

Causality imposes special considerations for the response obtained
from the I-FFT operation. A discontinuity of the impulse response
generally occurs at $t=0$, as the negative time response must be
zero~\cite{aplac}. In this situation the I-FFT operation yields a
value which is the average of the one-sided limits of the function at
that point. That is the value calculated at $t=0$ is one half of the
value at $t=0^+$. However, the upper side limit (at $t=0^+$) is
required in numerical integration and so the value of the I-FFT
calculated time response must be multiplied by 2 to obtain the
required impulse response. Note that in Equation~(\ref{eq.conv}) the
first point of the impulse is divided by two, so actually the
multiplication can be saved.

Fourier transformation is implemented using a real I-FFT algorithm
from the FFTW library~\cite{FFTW}. This algorithm requires only the
positive frequency samples since the negative frequency samples are
the complex conjugate of the corresponding positive frequencies. The
FFTW library is a comprehensive collection of C routines for computing
the discrete Fourier transform in one or more dimensions, of both real
and complex data, and of arbitrary input size.


\subsection{Initial Operating Point}

The quickest way to determine the steady state response is to avoid
the initial transient which largely determines the dc operating point
with a superimposed time-varying response. The dc voltage at the $i$th 
node is
\begin{equation}
  V_{DCi} = \sum_{j=1}^{n_S} V_{DCi,j} \label{eq.dc1}
\end{equation}
where
\begin{equation}
  V_{DCi,j} = M_{DCi,j} I_{DCj}
\end{equation}
$I_{DCj}$ is the dc current flowing into the nonlinear circuit at the
$j$ th node, and the $ij$ th DC response (corresponding to resistance)
is
\begin{equation}
  M_{DCi,j} = \sum_{n_{\tau}=0}^{N_T-1} m_{i,j}(n_{\tau})
\label{eq:mdcij}
\end{equation}
Setting $n_t = 0$ and replacing $v_{Li,j}(0)$ by $V_{DCi,j}$ in
(\ref{eq:error}) the DC operating point (defined by the set of
$V_{DCi,j}$'s and $I_{DCj}$'s) is determined when the error function
is zeroed.  These can be used as initial conditions in solving for the
steady-state response with the same effect as establishing the voltage
and current conditions before time 0.  The initial condition must
slowly be removed as the simulation progresses but for $n_t$ $<$ $N_T$
the initial condition contribution ${^Iv}_{i,j}(n_t)$ is, from
(\ref{eq.conv}),
%
\begin{equation}
  {^Iv}_{i,j}(n_t) = I_{DCj} \sum_{n_{\tau}=n_t}^{N_T-1} m_{i,j}(n_{\tau})
\end{equation}
This can be evaluated more efficiently as
\begin{equation}
  {^Iv}_{i,j}(n_t) = \left\{
  \begin{array}{lr}
    M_{DCi,j}I_{DCj}, & \mbox{if $n_t = 0$} \\
    \multicolumn{2}{l}{{^Iv}_{i,j}(n_t-1) - I_{DCj}m_{i,j}(n_t-1)},  \\
    \multicolumn{2}{r}{\mbox{if $1 \le n_t < N_T$}} \\
    0, & \mbox{if $n_t \ge N_T$} \\
\end{array}
\right.
\end{equation}
%
Now the previous expression for $v_{Li,j}$, (\ref{eq.conv}), becomes
%
\begin{eqnarray}
\lefteqn{{v_L}_{i,j}({\bf \hat{X}},n_t) =} \nonumber \\
 & &	\left\{ \begin{array}{l}
        {m_{i,j}(0) \: i_{NL}({\bf \hat{X}},n_t)} + {{^Iv}_{DC}}_{i,j}(n_t) \\
           + \: \sum_{n_{\tau}=1}^{n_t - 1} m_{i,j}(n_{\tau}) \:
                    i_j(n_t - n_{\tau}), \\
	\multicolumn{1}{r}{\mbox{if $n_t < N_T$}} \\
	   {m_{i,j}(0) \: {i_{NL}}_j({\bf \hat{X}},n_t)} \\
           + \: \sum_{n_{\tau}=1}^{N_T-1} m_{i,j}(n_{\tau}) \:
                    i_j(n_t - n_{\tau}), \\
	\multicolumn{1}{r}{\mbox{if $n_t \ge N_T$}}
	\end{array} \right. \label{eq.dcconv} 
\end{eqnarray}
%
Thus, the DC initial condition only adds one term to the convolution
sum.




\subsection{Impulse Response Correction}

The finite length discrete impulse response has inherent errors and
this is particularly problematic in determining the dc response. The
linear network is characterized by its transient step response
\begin{equation}
{^{STEP}R_i} = \sum_{j=1}^{n_S} M_{DCi,j}
\end{equation}
and its dc response (which is known to be more accurate)
\begin{equation}
{^{DC}R_i} = \sum_{j=1}^{n_S} M_{SVi,j}(f = 0)
\end{equation}
The dc error is characterized by an absolute error
%
\begin{equation}
  E_i = {^{DC}R_i} - {^{STEP}R_i}
\end{equation}
%
and a relative error
%
\begin{equation}
  {\epsilon_i} = E_i  / {^{DC}R_i}
\end{equation}
%
The dc error of the time domain solution can be corrected by
multiplying each row $i$ of impulse responses of $M_{SV}$ by the
following factor
%
\begin{equation}
  f_{Ei} = {^{DC}R_i} / {^{STEP}R_i}
\end{equation}
%
which is ideally 1.  When the dc response $R_i$ is very small a
correction would be erroneous.  The importance is taken as
${^{DC}R_i}$ relative to $E_i$. Now the error correction algorithm is
for each $i$\\
\medskip
\centerline{
\begin{tabular}{ll}
if $|{^{DC}R_i}| > |E_i|$ & $m_{i,j}(n_t) \leftarrow 
	f_{Ei} m_{i,j}(n_t)$\\
else if ${^{DC}R_i} \ne 0$
      & $m_{i,j}(n_t) \leftarrow m_{i,j}(n_t) + E_i /(n_S N_T)$ \\
else &  do nothing
\end{tabular}}


\subsection{Parameterized Nonlinear Models}

Let the nonlinear subnetwork be described by the following generalized
parametric equations~\cite{rizzoli:92:1}
\begin{eqnarray}
  {\bf v_{NL}}(t) &=& u[{\bf x}(t), \frac{d{\bf x}}{dt}, \dots , 
                 \frac{d^n{\bf x}}{dt^n}, {\bf x}_D(t)] \\
  {\bf i_{NL}}(t) &=& w[{\bf x}(t), \frac{d{\bf x}}{dt}, \dots ,
                 \frac{d^n{\bf x}}{dt^n}, {\bf x}_D(t)] 
  \label{eq1riz3}
\end{eqnarray}
where $v_{NL}(t)$, $i_{NL}(t)$ are vectors of voltages and currents at
the common ports, ${\bf x}(t)$ is a vector of state variables and
${\bf x}_D(t)$ a vector of time-delayed state variables, i.e.,
$x_{Di}(t)=x_i(t-\tau_i)$. The time delays $\tau_i$ may be functions
of the state variables. All vectors in (\ref{eq1riz3}) have the same
size $n_d$ equal to the number of common (device) ports. This kind of
representation is convenient from the physical viewpoint, as it is
equivalent to a set of implicit integro-differential equations in the
port currents and voltages. This allows an effective minimization of
the number of subnetwork ports, and what is more important, results in
extreme generality in device modeling capabilities.

Here a parametric diode model for the diode element is
presented~\cite{rizzoli:92:1}. The full model includes capacitances
and nonlinear resistance, but here a simple case is shown for didactic
purposes.  The conventional current equation for the diode is
\begin{equation}
  i(t) = I_s (\exp(\alpha v(t)) - 1) \label{eq_dio1}
\end{equation}
and, based on previous work~\cite{rizzoli:92:1}, the parametric model
is as follows
%
\begin{eqnarray}
  v(t) & = & \left\{ \begin{array}{l}
	x(t)            \\
          \multicolumn{1}{r}{\mbox{if $x(t) \leq V_1$}}  \\
	V_1+\frac{1}{\alpha} \ln(1 + \alpha (x(t) - V_1)) \\
	  \multicolumn{1}{r}{\mbox{if $x(t) > V_1$}}
  \end{array} \right. \\
  i(t) & = & \left\{ \begin{array}{l}
	I_s (\exp(\alpha x(t)) - 1) \\
          \multicolumn{1}{r}{\mbox{if $x(t) \leq V_1$}}  \\
	I_s \exp(\alpha V_1) (1 + \alpha (x(t)-V_1)) - I_s \\
	  \multicolumn{1}{r}{\mbox{if $x(t) > V_1$}}
  \end{array} \right. 
\end{eqnarray}
%
where $V_1$ is some threshold value. The model requires the current,
voltage and derivatives to be continuous at $x = V_1$. This implies
that
%
\begin{equation}
  V_1 = \frac{\ln ( G_1 / \alpha I_s )}{\alpha}
\end{equation}
%
where $G$ is the slope $\partial i / \partial v$ and is chosen to be
$\mbox{5e8} \cdot I_s$. Note that using this value, $V_1$ becomes
independent of the saturation current. In this way, the maximum value
of the exponential function is limited to 5e8.

The plots of Figures~\ref{fig:vi}, \ref{fig:xi} and~\ref{fig:xv} show
the improvement in the behavior of the model when using the state
variable approach. In a diode the current has an exponential
dependence on voltage. This causes convergence problems when the
voltage is updated during nonlinear iteration. At voltages greater
than the threshold, small voltage increments can result in large
current changes and hence changes in the error function. The
possibility of large changes is eliminated through the use of
parameterization which ensures smooth, well behaved current, voltage
and error function variations when the state variable is updated. See
Figures~\ref{fig:xi} and~\ref{fig:xv}.
%
\begin{figure}[htpb]
\centerline{\epsfxsize=3.5in \pfig{diode_iv.eps}}
\caption{Relation between $v$ and $i$ in a diode.} \label{fig:vi}
\end{figure}
%
\begin{figure}[htpb]
\centerline{\epsfxsize=3.5in \pfig{diode_ix.eps}}
\caption{Relation between $x$ and $i$ in a diode.} \label{fig:xi}
\end{figure}
%
\begin{figure}[htpb]
\centerline{\epsfxsize=3.5in \pfig{diode_vx.eps}}
\caption{Relation between $x$ and $v$ in a diode.} \label{fig:xv}
\end{figure}
%



\subsection{Convolution}

Convolution of the impulse response of the linear circuit with the
outputs of the nonlinear elements has been proposed for transient
analysis of distributed circuits with
\cite{theta,Brazil:New,gamma,delta,alpha,basel:paper,Blazeck:Mittra}. The
major problem identified with this type of analysis is the rapid
growth in computation and memory requirements. The convolution
integral, which becomes a convolution sum in computer implementations,
is $O(n_{MAX}^2 {n_S}^2)$ where $n_{MAX}$ is the length of the impulse
response and $n_S$ is the number of state variables. Memory usage is
$O(n_{MAX} \times {n_S}^2)$, principally due to storage of the matrix
impulse response. Thus reducing the number of state variables, $n_S$,
(i.e., using the minimum number of state variables rather than the
voltages at all nodes of the nonlinear network) dramatically reduces
memory and compute time. As state variables permit the use of
parameterized device models, the stability of the convolution analysis
is improved allowing larger time steps thus reducing $n_{MAX}$.

