\addcontentsline{toc}{chapter}{\numberline {}References}
%--------------------------------------------------------------------------
\begin{thebibliography}{99}
%
\bibitem{steer:global:1999} M. B. Steer, J. F. Harvey, J. W. Mink,
M. N. Abdulla, C. E. Christoffersen, H. M. Gutierrez, P. L. Heron,
C. W. Hicks, A. I. Khalil, U. A. Mughal, S. Nakazawa, T. W. Nuteson,
J. Patwardhan, S. G. Skaggs, M. A. Summers, S. Wang, and
A. B. Yakovlev, ``Global modeling of spatially distributed microwave
and millimeter-wave systems,'' \emph{IEEE Trans. Microwave Theory
Tech.}, June 1999, pp. 830--839.
%
\bibitem{svhb} C. E. Christoffersen, M. B. Steer and M. A. Summers,
``Harmonic balance analysis for systems with circuit-field
interactions,'' \emph{1998 IEEE Int. Microwave Symp. Dig.}, June 1998,
pp. 1131--1134.
%
\bibitem{kundert:vincentelli:90} K. S. Kundert, J. K. White and
A. Sangiovanni-Vincentelli, \emph{Steady-state methods for simulating
analog and microwave circuits}, Boston, Dordrecht, Kluwer Academic
Publishers, 1990.
%
\bibitem{rizzoli:92:1} V. Rizzoli, A. Lipparini, A. Costanzo,
F. Mastri, C. Ceccetti, A. Neri and D. Masotti, ``State-of-the-Art
Harmonic-Balance Simulation of Forced Nonlinear Microwave Circuits by
the Piecewise Technique,'' IEEE Trans. on Microwave Theory and
Tech., Vol. 40, No. 1, Jan 1992.
%
\bibitem{aplac1} M. Valtonen and T. Veijola, ``A microcomputer tool
especially suited for microwave circuit design in frequency and time
domain,'' \emph{Proc. URSI/IEEE National Convention on Radio
Science}, Espoo, Finland, 1986, p. 20,
%
\bibitem{aplac2} M. Valtonen, P. Heikkil\"a, A. Kankkunen, K.
Mannersalo, R. Niutanen, P. Stenius, T. Veijola and J.  Virtanen,
``APLAC - A new approach to circuit simulation by object
orientation,'' \emph{10th European Conference on Circuit Theory and
Design Dig.}, 1991.
%
\bibitem{codecs} K. Mayaram and D. O. Pederson, ``CODECS: an
object-oriented mixed-level circuit and device simulator,'' \emph{1987
IEEE Int. Symp. on Circuits and Systems Digest}, 1987, pp 604--607.
%
\bibitem{davis1} A. Davis, ``An object-oriented approach to circuit
simulation,'' \emph{1996 IEEE Midwest Symp. on Circuits and
Systems Dig.}, 1996, pp 313--316.
%
\bibitem{feldmann} B. Melville, P. Feldmann and S. Moinian, ``A C++
environment for analog circuit simulation,'' \emph{1992 IEEE
Int. Conf. on Computer Design: VLSI in Computers and Processors.}
%
\bibitem{ngoya} P. Carvalho, E. Ngoya, J. Rousset and J. Obregon,
``Object-oriented design of microwave circuit simulators,'' \emph{1993
IEEE MTT-S Int. Microwave Symp. Digest}, June 1993, pp 1491--1494.
%
\bibitem{local:reference:node:khalil} A. I. Khalil and M. B. Steer
``Circuit theory for spatially distributed microwave circuits,''
\emph{IEEE Trans. on Microwave Theory and Techn.}, Vol. 46, Oct. 1998,
pp 1500--1503.

\bibitem{mete} M. Ozkar, \emph{Transient Analysis of Spatially Distributed
Microwave Circuits Using Convolution and State Variables},
M. S. Thesis, Department of Electrical and Computer Engineering,
North Carolina State University.

\bibitem{sarkar} A. R. Djordjevic and T. K. Sarkar, ``Analysis of time
response of lossy multiconductor transmission line networks,''
\emph{IEEE Trans. on Microwave Theory and Techn.}, Vol. MTT-35,
Oct. 1987, pp. 898--908.

\bibitem{alpha} D. Winkelstein, R. Pomerleau and M. B. Steer, ``Transient
simulation of complex, lossy, multi-port transmission line networks
with nonlinear digital device termination using a circuit simulator,''
\emph{Conf. Proc. IEEE SOUTHEASTCON}, Vol. 3, pp. 1239--1244.

\bibitem{delta} T. J. Brazil, ``Causal convolution---a new method for
the transient analysis of linear systems at microwave frequencies,''
\emph{IEEE Trans. on Microwave Theory and Techn.}, Vol. 43, Feb. 1995,
pp. 315--23.

\bibitem{basel:paper} M. S. Basel, M. B. Steer and P. D. Franzon,
``Simulation of high speed interconnects using a convolution-based
hierarchical packaging simulator,'' \emph{IEEE Trans. on Components,
Packaging, and Manufacturing Techn.}, Vol. 18, February 1995,
pp. 74--82.

\bibitem{theta} J. E. Schutt-Aine and R. Mittra, ``Nonlinear transient
analysis of coupled transmission lines,'' \emph{IEEE Trans. on Circuits
and Systems}, Vol. 36, Jul. 1989, pp. 959--967.

\bibitem{aplac} P. Stenius, P. Heikkil\"a and M. Valtonen,
``Transient analysis of circuits including frequency-dependent
components using transgyrator and convolution,''
\emph{Proc. of the 11th European Conference on Circuit Theory and
Design}, Part II, 1993, pp. 1299--1304.

\bibitem{Nakhla:Grif} R. Griffith and M. S. Nakhla, ``Mixed frequency/time
domain analysis of nonlinear circuits,'' \emph{IEEE Trans. on Computer
Aided Design}, Vol.11, Aug. 1992, pp. 1032--43.

\bibitem{chan:AWE} P. K. Chan, Comments on ``Asymptotic waveform
evaluation for timing analysis,'' \emph{IEEE Trans. on Computer Aided
Design}, Vol. 10, Aug. 1991, pp. 1078--79.

\bibitem{multi:pade} M. Celik, O. Ocali, M. A. Tan, and A. Atalar,
``Pole-zero computation in microwave circuits using multipoint Pad\'e
approximation,'' \emph{IEEE Trans. on Circuits and Systems},
Jan. 1995, pp. 6--13.

\bibitem{Nakhla:Eli} E. Chiprout and M. Nakhla, ``Fast nonlinear
waveform estimation for large distributed networks,'' \emph{1992 IEEE
MTT-S Int. Microwave Symp. Digest}, Vol.3, Jun. 1992,
pp. 1341--1344.

\bibitem{Trithy} R. J. Trihy and Ronald A. Rohrer, ``AWE macromodels
for nonlinear circuits,'' \emph{Proceedings of the 36th Midwest
Symposium on Circuits and Systems}, Vol. 1, Aug. 1993, pp. 633--636.

\bibitem{beyene:98} W. T. Beyene and J. E. Schutt-Aine\'e, ``Efficient
Transient Simulation of High-Speed Interconnects Characterized by
Sampled Data'', \emph{IEEE Trans. on Components, Packaging and
Manufacturing Techn. --- Part B}, Vol. 21, No. 1, Feb. 1998,
pp. 105--114.

\bibitem{borisovich:96} D. Borisovich and J. E. Schutt-Aine\'e,
``Optimal Transient Simulation of Transmission Lines,'' \emph{IEEE
Trans. on Circuits and Systems---I: Fundamental Theory and
Applications}, Vol. 43, No. 2, Feb. 1996, pp. 110--121.

\bibitem{leung:95} W. Leung and F. Chang, ``Transient analysis via
fast wavelet-based convolution,'' \emph{1995 ISCAS Symp. Digest},
Vol. 3, pp. 1884--1887, 1995.

\bibitem{leung:96} W. Leung and F. Chang, ``Wavelet-based waveform
relaxation simulation of lossy transmission lines,'' \emph{1996
ISCAS Symp. Digest}, Vol. 4, pp. 739--742, 1996.

\bibitem{al-rawi:98} A. Al-Rawi and M. Devaney, ``Wavelets and power
system transient analysis,'' \emph{1998 IEEE Instr. and
Meas. Techn. Conf. Digest}, Vol. 2 , pp. 1331--1334, 1998.

\bibitem{hisakado:99} T. Hisakado and K. Okumura, ``Steady states
prediction in nonlinear circuit by wavelet transform,'' \emph{1999
ISCAS Symp. Digest}, 1999.

\bibitem{arturi:99} C. M. Arturi, A. Gandelli, S. Leva, S. Marchi and
A. P. Morando, ``Multiresolution analysis of time-variant electrical
networks,'' \emph{1999 ISCAS Symp. Digest}, 1999.

\bibitem{cai:96} W. Cai and J. Wang, ``Adaptive multiresolution
collocation methods for initial boundary value problems of nonlinear
PDEs,'' \emph{SIAM J. Numer. Anal.}, Vol. 33, No. 3, pp. 937--970, June
1996.

\bibitem{zhou:95} D. Zhou, N. Chen and W. Cai, ``A fast wavelet
collocation method for high-speed VLSI circuit simulation,''
\emph{1995 IEEE/ACM ICCAD Symp. Digest}, pp 115--122, 1995.

\bibitem{zhou:97} D. Zhou, X. Li, W. Zhang and W. Cai, ``Nonlinear
circuit simulation based on adaptive wavelet method,'' \emph{1997
ISCAS Symp. Digest}, Vol. 3, pp. 1720--1723, 1997.

\bibitem{cai:98} W. Cai and J. Wang, ``An adaptative spline wavelet
ADI (SW-ADI) method for two-dimensional reaction-diffusion
equations,'' \emph{J. of Computational Physics}, No. 139, pp. 92--126,
1998.

\bibitem{zhou1:99} D. Zhou and W. Cai, ``A fast wavelet
collocation method for high-speed VLSI circuit simulation,''
\emph{IEEE Trans. on Circuits and Systems---I: Fundamental Theory and
Appl.}, Vol. 46, pp 920--930, Aug. 1999.

\bibitem{zhou2:99} D. Zhou, W. Cai and W. Zhang, ``An adaptive
wavelet method for nonlinear circuit simulation,'' \emph{IEEE
Trans. on Circuits and Systems---I: Fundamental Theory and Appl.},
Vol. 46, pp 931--938, Aug. 1999.

\bibitem{wenzler} A. Wenzler and E. Lueder, ``Analysis of the periodic
steady-state in nonlinear circuits using an adaptive function base,''
\emph{1999 IEEE Int. Symposium on Circuits and Systems Digest}.

\bibitem{lippert:98} R. A. Lippert, T. A. Arias and A. Edelman,
``Multiscale computation with interpolating wavelets,'' \emph{J. of
Computational Physics}, No. 140, pp. 278--310, 1998.

\bibitem{graps:95} A. Graps, ``An introduction to wavelets,''
\emph{IEEE Computational Science and Engineering}, Vol. 2, No. 2,
pp. 50--61, Summer 1995.

\bibitem{daubechies:92} I. Daubechies, ``Ten Lectures on Wavelets,''
\emph{SIAM Publication}, Philadelphia, 1992.

\bibitem{bertoluzza:97} S. Bertoluzza, ``An adaptive collocation
method based on interpolating wavelets,'' \emph{Multiscale Wavelet
Methods for Partial Differential Equations}, Academic Press, 1997.

\bibitem{ruehli:94} P. Debefve F. Odeh and A. E. Ruehli, ``Waveform
Techniques'' \emph{Circuit Analysis, Simulation and Design},
North-Holland, 1994.
%
\bibitem{nakhla:vlach:76} M. S. Nakhla and J. Vlach, ``A Piecewise
Harmonic Balance Technique for Determination of Periodic Response of
Nonlinear Systems,'' \emph{IEEE Trans. on Circuits and Systems}, Vol
CAS-23, No. 2, Feb 1976.
%
\bibitem{steer:1991:1} M. B. Steer, C. Chang and G. W. Rhyne,
``Computer-Aided Analysis of Nonlinear Microwave Circuits Using
Frequency-Domain Nonlinear Analysis Tech.: The State of the
Art,'' \emph{Int. Journal of Microwave and Millimeter-Wave
Computer-Aided Engineering}, Vol. 1, No. 2, 181--200, 1991.

\bibitem{carvalho:pedro:1} N. Borges de Carvalho and J. C. Pedro,
``Multitone frequency-domain simulation of nonlinear circuits in
large- and small-signal regimes,'' \emph{IEEE MTT}, Vol. 46, no.  12,
pp. 2016--2024, 1998.

\bibitem{borich:99} V. Borich, J. East and G. Haddad, ``An efficient
Fourier transform algorithm for multitone harmonic balance,''
\emph{IEEE Transactions on Microwave Theory and Tech.}, Vol. 47,
no. 2, Feb. 1999, pp 182--188.

\bibitem{vlach:singhal} J. Vlach and K. Singhal, \emph{Computer
Methods for Circuit Analysis and Design}, Van Nostrand Reinhold, 1994.

\bibitem{Brazil:New} T. J. Brazil, ``A new method for the transient
simulation of causal linear systems described in the frequency
domain,'' \emph{1992 IEEE MTT-S Int. Microwave Symp. Digest}, June
1992, pp. 1485--1488.

\bibitem{gamma} P. Perry and T. J. Brazil, ``Hilbert-transform-derived
relative group delay,'' \emph{IEEE Trans. on Microwave Theory
and Techn.}, Vol 45, Aug. 1997, pt. 1, pp. 1214--1225.

\bibitem{ims99} C. E. Christoffersen, S. Nakazawa, M. A. Summers, and
M. B. Steer, ``Transient analysis of a spatial power combining
amplifier'', \emph{1999 IEEE MTT-S Int. Microwave Symp. Dig.}, June
1999, pp. 791--794.
%
\bibitem{FFTW} M. Frigo and S. G. Johnson, \emph{FFTW User's Manual},
Massachusetts Institute of Technology, September 1998.

\bibitem{Blazeck:Mittra} C. Gordon, T. Blazeck and R. Mittra, ``Time
domain simulation of multiconductor transmission lines with
frequency-dependent losses,'' \emph{IEEE Trans. on Computer
Aided Design of Integrated Circuits and Systems}, Vol. 11, Nov. 1992
pp. 1372--87.

\bibitem{mike:thesis} M. G. Case, \emph{Nonlinear transmission lines
for picosecond pulse, impulse and millimeter-wave harmonic
generation}, Ph.D Dissertation, Department of Electrical and Computer
Engineering, University of California, Santa Barbara, California,
U.S.A., 1993.

\bibitem{rodwell:paper} M. J. W. Rodwell, M. Kamegawa, R. Yu, M. Case,
E. Carman and K. S. Giboney, ``GaAs nonlinear transmission lines for
picosecond pulse generation and millimeter-wave sampling,'' \emph{IEEE
Trans. on Microwave Theory and Techn.}, Vol. 39, July 1991,
pp. 1194--1204.
%
\bibitem{shi} H. Shi, C. W. Domier and N. C. Luhmann, ``A monolithic
nonlinear transmission line system for the experimental study of
lattice solutions,'' \emph{J. of Applied Physics}, Vol 4., August
1995, pp. 2558--64.
%
\bibitem{mharm:94} Compact Software, \emph{Microwave Harmonica
Elements Library}, (1994).
%
\bibitem{brambilla:93} A. Brambilla, D. D'Amor and M. Pillan,
``Convergence improvements of the harmonic balance method,''
\emph{Proceedings IEEE Int. Symposium on Circuits and
Systems}, Vol. 4 1993, Publ. by IEEE, IEEE Service Center, Piscataway,
NJ, USA. p 2482--2485.
%
\bibitem{NNES} R. S. Bain, \emph{NNES user's manual}, 1993.

\bibitem{rodrigues} P. J. C. Rodrigues, \emph{Computer Aided Analysis
of Nonlinear Microwave Circuits}, Artech House, 1998.
%
\bibitem{eliens} A. Eli\"ens, \emph{Principles of object-oriented
software development}, Adison-Wesley, 1995.
%
\bibitem{dep_inv} R. C. Martin. ``The dependency inversion principle,''
\emph{C++ Report}, May 1996.
%
\bibitem{open} R. C. Martin, ``The Open Closed Principle,'' \emph{C++
Report}, Jan. 1996.
%
\bibitem{liskov} R. C. Martin, ``The Liskov Substitution Principle,''
\emph{C++ Report}, March 1996.
%
\bibitem{int_seg} R. C. Martin, ``The Interface Segregation Principle,''
\emph{C++ Report}, Aug 1996.
%
\bibitem{uml_tut} R. C. Martin, ``UML Tutorial: Part 1 --- Class
Diagrams,'' Engineering Notebook Column, \emph{C++ Report}, Aug. 1997.
%
\bibitem{kai} A. D. Robison, ``C++ Gets Faster for Scientific
Computing,'' \emph{Computers in Physics}, Vol. 10, pp. 458--462, 1996.
%
\bibitem{c++fortran} J. R. Cary and S. G. Shasharina, ``Comparison of
C++ and Fortran 90 for Object-Oriented Scientific Programming,''
Available from Los Alamos National Laboratory as Report
No. LA-UR-96-4064.
%
\bibitem{oonpage} The Object Oriented Numerics Page,
http://oonumerics.org/.
%
\bibitem{STL} Silicon Graphics, \emph{Standard Template Library
Programmer's Guide}, http://www.sgi.com/Technology/STL/.
%
\bibitem{todd} T. Veldhuizen, {Tech. for Scientific C++ - Version
0.3}, Indiana University, Computer Science Department,
1999. (http://extreme.indiana.edu/~tveldhui/papers/Tech./)
%
\bibitem{adol-c:96} A. Griewank, D. Juedes and J. Utke, ``Adol-C: A
Package for the Automatic Differentiation of Algorithms Written in
C/C++,'' \emph{ACM TOMS}, Vol. 22(2), pp. 131--167, June 1996.
%
\bibitem{mv++} R. Pozo, {MV++ v. 1.5a}, \emph{Reference Guide}, National
Institute of Standards and Technology, 1997.
%
\bibitem{Sparse} K. S. Kundert and A. Songiovanni-Vincentelli,
\emph{Sparse user's guide - a sparse linear equation solver}, Dept. of
Electrical Engineering and Computer Sciences, University of
California, Berkeley, Calif. 94720, Version 1.3a, Apr 1988.
%
\bibitem{speelpenning} B. Speelpenning. \emph{Compiling Fast Partial
Derivatives of Functions Given by Algorithms}, Ph.D. thesis (Under the
supervision of W. Gear), Department of Computer Science, University of
Illinois at Urbana-Champaign, Urbana-Champaign, Ill., January 1980.
%
\bibitem{coleman} T. F. Coleman and G. F. Jonsson, ``The Efficient
Computation of Structured Gradients using Automatic Differentiation,''
\emph{Cornell Theory Center Technical Report CTC97TR272}, April 28,
1997

\bibitem{imtiaz} S. M. S. Imtiaz and S. M. El-Ghazaly, ``Global
modeling of millimeter-wave circuits: electromagnetic simulation of
amplifiers,'' \emph{IEEE Trans. on Microwave Theory and Tech.}, vol
45, pp. 2208--2217.  Dec. 1997.

\bibitem{kuo} C.-N. Kuo, R.-B. Wu, B. Houshmand, and T. Itoh, Modeling
of microwave active devices using the FDTD analysis based on the
voltage-source approach, \emph{IEEE Microwave Guided Wave Lett.},
Vol. 6, pp. 199--201, May 1996.

\bibitem{larique} E. Larique, S. Mons, D. Baillargeat, S. Verdeyme,
M. Aubourg, P.  Guillon, and R. Quere, ``Electromagnetic analysis for
microwave FET modeling,'' \emph{IEEE microwave and guided wave
letters}, Vol 8, pp. 41--43, Jan. 1998.

\bibitem{todd1} T. W. Nuteson, H. Hwang, M. B. Steer, K. Naishadham,
J.W.Mink, and J. Harvey, ``Analysis of finite grid structures with
lenses in quasi-optical systems,'' \emph{IEEE Trans. Microwave Theory
Tech.}, pp. 666--672, May 1997.

\bibitem{antenna} M. B. Steer, M. N. Abdullah, C. Christoffersen,
M. Summers, S. Nakazawa, A. Khalil, and J. Harvey, ``Integrated
electro-magnetic and circuit modeling of large microwave and
millimeter-wave structures,'' \emph{Proc. 1998 IEEE Antennas and
Propagation Symp.}, pp. 478--481, June 1998.

\bibitem{kunisch} J. Kunisch and I. Wolff, ``Steady-state analysis of
nonlinear forced and autonomous microwave circuits using the
compression approach,'' \emph{Int. J. of Microwave and Millimeter-Wave
Computer-Aided Engineering}, Vol. 5, No. 4, pp. 241--225, 1995
%
\bibitem{cormen:90} T. H. Cormen, C. E. Leiserson and R. L. Rivest
\emph{Introduction to Algorithms}, The MIT Press, McGraw-Hill Book
Company, 1990.
%
\bibitem{rational} Rational Software, UML Resources,
\emph{http://www.rational.com/}.
%
\bibitem{rodwell} H. S. Tsai, M. J. W. Rodwell and R. A. York,
``Planar amplifier array with improved bandwidth using folded-slots,''
\emph{IEEE Microwave and Guided Wave Letters}, Vol. 4, April 1994,
pp. 112--114.
%
\bibitem{steer:abdullah:1998} M. B. Steer, M. N. Abdullah,
C. Christoffersen, M. Summers, S. Nakazawa, A. Khalil, and J.  Harvey,
``Integrated electro-magnetic and circuit modeling of large microwave
and millimeter-wave structures,''
\emph{Proc. 1998 IEEE Antennas and Propagation Symp.},
pp. 478--481, June 1998.
%
\bibitem{mostafa} M. N. Abdulla, U.A. Mughal, and M B. Steer,
``Network Charactarization for a Finite Array of Folded-Slot Antennas
for Spatial Power Combining Application,''
\emph{Proc. 1999 IEEE Antennas and Propagation Symp.},
July 1999.
%
\bibitem{usman} U. A. Mughal, ``Hierarchical approach to global
modeling of active antenna arrays,'' \emph{M.S. Thesis}, North
Carolina State University, 1999.
%
\bibitem{mark} M. A. Summers, C. E. Christoffersen, A. I. Khalil,
S. Nakazawa, T. W. Nuteson, M. B. Steer and J. W. Mink, ``An
integrated electromagnetic and nonlinear circuit simulation
environment for spatial power combining systems,'' \emph{1998 IEEE
MTT-S Int. Microwave Symp. Dig.}, June 1998, pp. 1473--1476.

\bibitem{hector1} H. Gutierrez, C. E. Christoffersen and M. B. Steer,
``An integrated environment for the simulation of electrical, thermal
and electromagnetic interactions in high-performance integrated
circuits,'' \emph{Proc. IEEE 6 th Topical Meeting on Electrical
Performance of Electronic Packaging}, Sept. 1999.

\bibitem{bill4} W. Batty, C. E. Christoffersen, S. David, A. J. Panks,
R. G. Johnson, C. M. Snowden and M. B. Steer, ``Electro-thermal cad of
power devices and circuits with fully physical time-dependent thermal
mmodelling of complex 3-d systems,'' submitted to the \emph{IEEE
Trans. on Component and Packaging Technologies}.

\bibitem{bill1} W. Batty, C. E. Christoffersen, S. David, A. J. Panks,
R. G. Johnson, C. M. Snowden and M. B. Steer, ``Fully physical,
time-dependent thermal modelling of complex 3-dimensional systems for
device and circuit level electro-thermal CAD,'' submitted to
\emph{Semi-Therm XVII}, San Jose, March 2001.

\bibitem{bill2} W. Batty, C. E. Christoffersen, S. David, A. J. Panks,
R. G. Johnson, C. M. Snowden and M. B. Steer, ``Predictive microwave
device design by coupled electro-thermal simulation based on a fully
physical thermal model,'' \emph{EDMO 2000}, Glasgow UK,
November 2000.

\bibitem{bill3} W. Batty, C. E. Christoffersen, S. David, A. J. Panks,
R. G. Johnson, C. M. Snowden and M. B. Steer, ``Steady-state and
transient electro-thermal simulation of power devices and circuits
based on a fully physical thermal model,'' \emph{THERMINIC 2000
Digest}, Budapest, September 2000.
%
\bibitem{ptplot}
Ptplot. \emph{http://ptolemy.eecs.berkeley.edu/java/ptplot}.

\bibitem{foty:1997}
Foty, \textit{MOSFET modeling with SPICE: Principles and
Practice}, Prentice Hall, 1997.

\bibitem{liu:2001}
Liu, \textit{MOSFET Models for SPICE simulation including BSIM3v3
and BSIM4}, John Wiley and Sons, 2001.

\bibitem{shi:1968}
H. Shichman and D. Hodges, ``Modeling and Simulation of
Insulated-Gate Field-Effect Transistor Switching Circuits,''
\textit{IEEE J. Sol. St. Circ.} vol. 3, pp. 285-289 (1968)

\bibitem{fettext:1966}
\textit{Field Effect Transistors}, (ed. by J. Wallmark and H.
Johnson), Prentice-Hall, 1966.

\bibitem{lee:1993}
Lee, Shur, Fjeldy and Ytterdal, \textit{Semiconductor Device
Modeling for VLSI: with the AIM-spice circuit simulator}, Prentice
Hall, 1993.

\bibitem{tor:1998}
Fjeldy, Ytterdal and Shur, \textit{Introduction to device modeling
and circuit simulation}, A Wiley-Interscience Publication, 1998.

\bibitem{kiel:1995}
Ron M. Kielkowski, \textit{SPICE Practical Device Modeling},
McGraw-Hill, Inc., 1995.

\bibitem{svtr} C. E. Christoffersen, M. Ozkar, M. B. Steer, M. G. Case
and M. Rodwell, ``State variable-based transient analysis using
convolution,'' \emph{IEEE Transactions on Microwave Theory and
Tech.}, Vol. 47, June 1999, pp. 882--889.

\bibitem{wavelet} C. E. Christoffersen and M. B. Steer,
``State-variable microwave circuit simulation using wavelets,'' to
be published in the \emph{IEEE Microwave and Guided Waves
Letters}, 2001.

\bibitem{oopaper} C. E. Christoffersen, U. A. Mughal and M. B. Steer,
``Object Oriented Microwave Circuit Simulation,'' \emph{Int. J.
of RF and Microwave Computer-Aided Engineering}, Vol. 10, Issue 3,
2000, pp. 164--182.

\bibitem{carlos_phd} C. E. Christoffersen \emph{Global modeling of nonlinear
microwave circuits}, Ph. D. Dissertation, North Carolina State
University, December 2000.

\bibitem{l_element} C. E. Christoffersen, ``Adding Linear Element to
Transim'', April 2001.

\bibitem{adol-c} A. Griewank, D. Juedes, J. Utke, ``Adol-C: A Package for
the Automatic Differentiation of Algorithms Written in C/C++'',
Version 1.8.2, March 1999.

\bibitem{etrcat} H. S. Kanj, ``Electro-Thermal Resistor Catalog'', June
2001.

\bibitem{lref1} A.I. Khalil and M.B. Steer, ``Circuit theory for spacially
distributed microwave circuits'', IEEE Trans on Microwave Theory
and Technique 46 (1998), 1500-1503.

\bibitem{lref2} C. E. Christoffersen and M.B. Steer, ``Implementation of the
Local Reference Node Concept for Spatially Distributed Circuits'',
John Willey \& Sons, Inc. Int J RF and Microwave CAE 9: 376-384,
1999.

% %
% \bibitem{gnuplot} Gnuplot. Copyright(C) 1986 - 1993, 1998 Thomas
% Williams, Colin Kelley and many others.
% \bibitem{local:reference:node:christoffersen} C. E. Christoffersen and
% M. B. Steer ``Implementation of the local reference concept for
% spatially distributed circuits,'' \emph{Int. J. of RF and Microwave
% Computer-Aided Eng.}, Vol. 9, No. 5, 1999.
% \bibitem{rational} Rational Software, UML Resources,
% \emph{http://www.rational.com/}.
% %
% %
% % HB papers
% %
% \bibitem{rizzoli:96:1} V. Rizzoli, F. Mastri, F. Sgallari and
% G. Spaletta, ``Harmonic-Balance Simulation of Strongly Nonlinear very
% Large-Size Microwave Circuits by Inexact Newton Methods,'' \emph{1996
% IEEE MTT-S Int. Microwave Symp. Dig.}
% %
% \bibitem{rizzoli:95:1} V. Rizzoli, A. Costanzo, and A. Lipparini, ``An
% Electrothermal Functional Model of the Microwave FET Suitable for
% Nonlinear Simulation,'' \emph{Int. Journal of Microwave and
% Millimeter-Wave Computer-Aided Engineering}, Vol. 5, No. 2, 104-121
% (1995).
% %
% \bibitem{rizzoli:92:1} V. Rizzoli, A. Lipparini, A. Costanzo,
% F. Mastri, C. Ceccetti, A. Neri and D. Masotti, ``State-of-the-Art
% Harmonic-Balance Simulation of Forced Nonlinear Microwave Circuits by
% the Piecewise Technique,'' IEEE Trans. on Microwave Theory and
% Tech., Vol. 40, No. 1, Jan 1992.
% %
% \bibitem{gounary:1997} M. M. Gourary, S. G. Rusakov, S. L. Ulyanov,
% M. M. Zharov, K. K. Gullapalli, and B. J. Mulvaney, ``Iterative
% Solution of Linear Systems in Harmonic Balance Analysis,'' \emph{IEEE
% MTT-S Int. Microwave Symp. Dig.}, 1997.
% %
% \bibitem{moret:1987} I. Moret, ``On the Convergence of Inexact
% Quasi-Newton Methods,'' \emph{Int. J. of Computer Math.},
% Vol. 28, pp. 117-137, 1987.
% %
% \bibitem{materka:kacprzac:85} A. Materka and T. Kacprzak, ``Computer
% Calculation of Large-Signal GaAs FET Amplifier Characteristics,''
% \emph{IEEE Trans. on Microwave Theory and Tech.}, Vol MTT-33,
% No. 2, Feb 1985.
% %
% \bibitem{steer2:1996} J. F. Sevic, M. B. Steer, and A. M. Pavio,
% ``Nonlinear Analysis Methods for the Simulation of Digital Wireless
% Communication Systems,'' \emph{Int. Journal of Microwave and
% Millimiter-Wave Computer-Aided Engineering}, Vol. 6, No. 3, 197-216,
% 1996.
% %
% \bibitem{kunisch:wolff:95} J. Kunisch and I. Wolff, ``Steady-State
% Analysis of Nonlinear Forced and Autonomous Microwave Circuits Using
% the Compression Approach,'' Int. \emph{Journal of Microwave
% and Millimeter-Wave Computer-Aided Engineering}, Vol. 5, No. 4,
% 241-255 (1995).
% %
% \bibitem{ngoya:suarez:quere:95} E. Ngoya, A. S. R. Sommet and
% R. Qu\'er\'e, ``Steady State Analysis of Free or Forced Oscillators by
% Harmonic Balance and Stability Investigation of Periodic and
% Quasi-Periodic Regimes,'' \emph{Int. Journal of Microwave and
% Millimeter-Wave Computer-Aided Engineering}, Vol. 5, No. 3, 210-223
% (1995).
% %
% \bibitem{powell:1988} M. J. D. Powell, ``A hybrid method for nonlinear
% equations,'' \emph{Numerical Methods for Nonlinear Algebraic
% Equations}, P. Rabinowitz, Editor, Gordon and Breach, 1988.
% %
% \bibitem{gilmore:1991:1} R. J. Gilmore and M. B. Steer, ``Nonlinear
% Circuit Analysis Using the Method of Harmonic Balance---A Review of
% the Art. II. Advanced Concepts,'' \emph{Int. Journal of
% Microwave and Millimeter-Wave Computer-Aided Engineering}, Vol. 1,
% No. 2, 159-180, 1991.
% %
% \bibitem{chang:1990} C. R. Chang, \emph{Computer-Aided Analysis of
% Nonlinear Microwave Analog Circuits Using Frequency-Domain Spectral
% Balance}, Ph.D. Thesis, Department of Electrical and Computer
% Engineering, North Carolina State University, Raleigh, NC, 1990.
% %
% \bibitem{damore:94} D. D'Amore, P. Maffezzoni and M. Pillan, ``A
% Newton-Powell Modification Algorithm for Harmonic Balance-Based
% Circuit Analysis,'' \emph{IEEE Transactions on Circuits and
% Systems---I: Fundamental Theory and Applications}, Vol. 41, No. 2,
% February 1994.
% %
% \bibitem{Thodesen:Kundert:96} Y. Thodesen and K. Kundert, ``Parametric
% harmonic balance,'' \emph{IEEE MTT S. Int. Microwave Symp.  Digest},
% Vol 3, 1996, IEEE, Piscataway, NJ, USA, pp. 1361-1364.
% %
% \bibitem{ushida:84} A. Ushida and L. O. Chua.  ``Frequency-domain
% analysis of nonlinear circuits driven by multi-tone signals,''
% \emph{IEEE Transactions on Circuits and Systems}, Vol. CAS-31, No. 9,
% September 1984, pp. 766-778.
% %
% \bibitem{ushida:87} A. Ushida, L. O. Chua and T. Sugawara. ``A
% substitution algorithm for solving nonlinear circuits with
% multi-frequency components,'' \emph{Int. Journal on Circuit
% Theory and Application}, Vol. 15, 1987, pp. 327-355.
% %
% \bibitem{gilmore:84} R. J. Gilmore and F. J. Rosenbaum, ``Modelling of
% nonlinear distortion in GaAs MESFETs,'' \emph{1984 IEEE MTT-S Int.
% Microwave Symp. Digest}, May 1984, pp. 430-431.
% %
% \bibitem{bava:82} G. P. Bava, S. Benedetto, E. Biglieri, F. Filicori,
% V. A. Monaco, C. Naldi, U. Pisani and V. Pozzolo, ``Modelling and
% perfomance simulation Tech. of GaAs MESFETs for microwave power
% amplifiers,'' \emph{ESA-ESTEC Report}, Noordwijk, Holland, March 1982.
% %
% \bibitem{hiroaki:93} H. Makino and H. Asai, ``Relaxation-based circuit
% simulation Tech. in the frequency domain,'' \emph{IEICE
% Transactions on Fundamentals of Electronics, Communications and
% Computer Sciences}, Vol E76-A, No. 4 Apr 1993, p 626-630.
% %
% \bibitem{rizzoli:92:2} V. Rizzoli, A. Costanzo, P. R. Ghigi,
% F. Mastri, D. Masotti and C. Cecchetti, ``Recent advances in
% harmonic-balance Tech. for nonlinear microwave circuit
% simulation,'' \emph{AEU Arch Elektron Uebertrag Electron Commun},
% Vol. 46, No. 4 Jul 1992, p 286-297.
% %
% \bibitem{celik:96} M. Celik, A. Atalar and M. A. Tan, ``New method for
% the steady-state analysis of periodically excited nonlinear
% circuits,'' \emph{IEEE Transactions on Circuits and Systems I:
% Fundamental Theory and Applications}, Vol. 43, No. 12 Dec 1996, p
% 964-972.
% %
% \bibitem{Brachtendorf:95:1} H. G. Brachtendorf, G. Welsch and R. Laur,
% ``Fast simulation of the steady-state of circuits by the harmonic
% balance technique,'' \emph{Proc. IEEE Int. Symp. on Circuits and
% Systems}, Vol. 2 1995, IEEE, Piscataway, NJ, USA, p 1388-1391.
% %
% \bibitem{Brachtendorf:95:2} H. G. Brachtendorf, G. Welsch and R. Laur,
% ``Simulation tool for the analysis and verification of the steady
% state of circuit designs,'' \emph{Int. Journal of Circuit Theory and
% Applications}, Vol. 23, No 4 Jul-Aug 1995, p 311-323.
% %
% \bibitem{barbancho:96} I. Barbancho Perez and I. Molina Fernandez,
% ``Predictor strategies for continuation methods applied to nonlinear
% circuit analysis,'' \emph{Industrial Applications in Power Systems,
% Computer Science and Telecommunications Proceedings of the
% Mediterranean Electrotechnical Conference MELECON}, Vol. 3 1996, IEEE,
% Piscataway, NJ, USA, p 1419-1422.
% %
%
% \bibitem{frolich:97} J. Fr\"ohlich and K. Schneider, ``An
% adaptive wavelet-vaguelette algorithm for the solution of PDEs,''
% \emph{J. of Computational Physics}, No. 130, pp. 174-190, 1997.
%
% %\bibitem{jawerth:94} B. Jawerth and W. Sweldens, SIAM Rev., No. 36,
% %p. 377, 1994.
%
% \bibitem{lippert:98} R. A. Lippert, T. A. Arias and A. Edelman,
% ``Multiscale computation with interpolating wavelets,'' \emph{J. of
% Computational Physics}, No. 140, pp. 278-310, 1998.
%
% \bibitem{jameson:98} J. S. Hesthaven and L. M Jameson, ``A wavelet
% optimized adaptive multi-domain method,'' \emph{J. of Computational
% Physics}, No. 145, pp. 280-296, 1998.
%
% \bibitem{vasilyev:97} O. V. Vasilyev and S. Paolucci, ``A fast
% adaptive wavelet collocation algorithm for multidimensional PDEs,''
% \emph{J. of Computational Physics}, No. 138, pp. 16-56, 1997.
%
% \bibitem{beylkin:97} G. Beylkin and J. M. Keiser, ``An adaptive
% pseudo-wavelet approach for solving nonlinear partial differential
% equations,'' \emph{Multiscale Wavelet Methods for Partial Differential
% Equations}, Academic Press, 1997.
%

\end{thebibliography}
