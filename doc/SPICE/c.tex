\element{C}{Capacitor}
\begin{figure}[h]
\centering
\ \pfig{c_spice.ps}
\caption{C --- capacitor element. \label{c:fig}}
\end{figure}
\form{ {\tt C}name  $N_1$ $N_2$ CapacitorValue \B {\tt IC=}$V_C$\E
   \B{\tt L=} Length\E \B{\tt W=} Width\E }
\spicethreeform{ {\tt C}name  $N_1$ $N_2$ \B ModelName \E CapacitorValue
   \B {\tt IC=}$V_C$\E
   \B{\tt L=} Length\E \B{\tt W=} Width\E }
\pspiceform{ {\tt C}name  $N_1$ $N_2$ \B ModelName \E
 ~CapacitorValue \B {\tt IC=}$V_C$\E }
\begin{widelist}
\item[$N_1$] is the positive element node,
\item[$N_2$] is the negative element  node, and
\item[{\it ModelName}]  is  the optional  model name.
\item[{\it CapacitorValue}]  is  the  capacitance.  This is
               modified if {\it ModelName} is specified.\\
               (Units: F; Required; Symbol: $CapacitorValue$)

\item[{\it L}] is the length {\it Length} of the integrated capacitor.\\
               (Units: m; Required in \spicethree\ if {\it ModelName}
               specified; Symbol: $\Length$;)
\item[{\it W}] is the width {\it Length} of the integrated capacitor.\\
               (Units: m; Optional; Default: Default width {\tt DEFW}
               specified in model {\it ModelName}; Symbol: $\Length$;)\\

\item[{\tt IC}] is the optional  initial condition specification
      Using {\tt IC=}$V_C$ is intended for use with the {\tt UIC} option
      on  the  {\tt .TRAN}  line,  when  a transient analysis is desired
      with an initial voltage $V_C$ across the capacitor
      rather than the quiescent operating point.
      Specification of the transient initial conditions using the {\tt .IC}
      statement (see page \pageref{.ICstatement}) is preferred and is more
      convenient.
\end{widelist}
\example{ CAP1 1 GND 12.3PF \\
          C1 node1 0 ((12.3 + 2.1)/2) }

\modeltype{CAP}

\modelx{CAP}{\spicethree\ Only}{Capacitor Model}
\label{CAPmodelsp3}

\form{ {\tt .MODEL} ModelName {\tt CAP(} \B  \B keyword = value\E  ... \E
{\tt )}}

\example{ .MODEL SMALLCAP CAP( CJ CJSW DEFW$10^{-6}$ NARROW) }

\modelkeyword{
{\tt CJ} &junction bottom capacitance\sym{C_J}&$\mbox{F/m}^2$    &\reqd \X
{\tt CJSW}    &junction sidewall capacitance\sym{C_{J,\ms{SW}}}&F/m        &\reqd \X
{\tt DEFW}    &default device width\sym{W_{\ms{DEF}}}&meters     &$10^{-6}$\X
{\tt NARROW}  &narrowing due to side etching\sym{X_{\ms{NARROW}}}&meters   &0\X
    }

The \spicethree\ capacitance model is a process model for a monolithicly
fabricated capacitor enabling the capacitance to be determined from geometric
information.  If the parameter {\tt W} is not specified on the element
line then $Width$
defaults to {\tt DEFW} = $W_{\ms{DEF}}$. The effective dimensions are reduced
by etching so that the effective length of the capacitor is
\begin{equation}
L_{\ms{EFF}} = (Length - X_{\ms{NARROW}})
\end{equation}
and the effective width is
\begin{equation}
W_{\ms{EFF}} = (Width - X_{\ms{NARROW}})
\end{equation}
The new value of the capacitance
\begin{equation}
{\it NewCapacitorValue} = C_J L_{\ms{EFF}} W_{\ms{EFF}}
          + 2 C_{J,\ms{SW}}( L_{\ms{EFF}} +  W_{\ms{EFF}} )
\end{equation}

\modelx{CAP}{\pspice\ Only}{Capacitor Model}

\label{CAPmodelpspice}

\form{ {\tt .MODEL} ModelName {\tt CAP(} \B  \B keyword = value\E  ... \E
{\tt )}}

\example{ .MODEL SMALLCAP GASFET(C=2.5 VC1=0.01 VC2=0.001 TC1=0.02 TC2=0.005)}

\modelkeyword{
{\tt C} &capacitance multiplier\sym{C_{\ms{MULTIPLIER}}}&-    &1 \X
{\tt VC1}    &linear voltage coefficient\sym{V_{C1}}&1/V        &0\X
{\tt VC2}    &quadratic voltage coefficient\sym{V_{C2}}&1/V        &0\X
{\tt TC1}    &linear temperature coefficient\sym{T_{C1}}&1/$^{\circ}C$     &0\X
{\tt TC2}    &quadratic temperature coefficient\sym{T_{C1}}&1/$^{\circ}C$  &0\X
    }

The \pspice\ capacitance model is a nonlinear temperature dependent capacitor
model.  It is assumed that the model parameters were determined or
measured at the nominal temperature $T_{\ms{NOM}}$ (default
$27^{\circ}$C) specified in the most recent {\tt .OPTIONS} statement
preceeding the {\tt .MODEL} statement.

If the CAP model is specified then a new capacitance is evaluated
as
\begin{eqnarray}
C & = &CapacitorValue\,C_{\ms{MULTIPLIER}}
  \left[ 1 + V_{C1} V_C + V_{C2} V_C^2 \right] \\
  && \times\left[ 1 + T_{C1} (T-T_{\ms{NOM}}) + T_{C2} (T-T_{NOM})^2 \right]
\end{eqnarray}
where $V_C$ is the voltage across the capacitor as in figure \ref{c:fig} and
$T$ is the current temperature.
If \pspice\ CAP model is not specified then the capacitance specified on the element line is used.
