\addcontentsline{toc}{part}{PART I  SPICE BASICS}

\chapter{Introduction}

\section{Introduction}

\spice\ is a general-purpose circuit  simulation  program
for  nonlinear  \dc ,  nonlinear transient, and small-signal \ac\ analyses.
Circuits may contain resistors,  capacitors,  inductors,
mutual  inductors,  independent  voltage  and current
sources,  dependent  sources,  transmission
lines, switches,
and the five most common semiconductor devices:
diodes, BJTs, JFETs, MESFETs, and MOSFETs.
\justspice\ was developed at the
University of California at Berkeley and after many years of effort
culminated in the landmark \spicetwo\ version.  This was the last FORTRAN
language
version of \justspice\ distributed by UC Berkeley and its syntax and analysis
options have become a standard
for \justspice -like simulators.  With few exceptions, all commercial versions
and University versions of \justspice\ are upwards compatible with
\spicetwo\ in that they support the complete syntax and analyses of
\spicetwo .
Since \spicetwo\ was released \spicethree\ was developed at UC Berkeley
initially
as a C language equivalent of \spicetwo .  The new capabilities of
\spicethree\ include pole-zero analysis, and new transistor models
for MESFETs and for short and narrow channel MOSFETs as well improved
numerical methods.
Many commercial
versions of \justspice\ are based directly on \spicethree .  However
there is a group of commercial \justspice -like simulators
that have significant advances over \spicetwo\ and \spicethree\ in the
areas of enhanced input syntax, improved convergence, better device models and
more analysis types.
{Many of these enhanced \justspice\ programs were completely
rewritten and not ports of the Berkeley software.
As can be expected, the effort put into these commercial programs is
reflected in their price.}
The first \justspice\ version for personal computers was the
commerical program \justpspice\ by MicroSim corporation.
{
\justpspice now has the largest
customer base of all commerical \justspice\ programs.}
Consequently the syntax of \justpspice\ has become a second
``standard''.
{
The \pspice\ syntax
is upwards compatible to the \spicetwo\ syntax.
However, there are some incompatibilities
between the \spicethree\ and \pspice\ syntaxes as \pspice\ was released
before \spicethree\ became available.
The effect of this development is that all \justspice\ simulators
(including commercial programs)
accept a \spicetwo\ netlist but perhaps not a \spicethree\ netlist.
Conflicts with \spicethree\ generally exist in the naming of additional
elements and in the use of new models.}


\section{How to use this book}

This manual was used as a guide in developing the fREEDA simulator (see http://www.freeda.org).
If you are generally unfamiliar with how to use \justspice,
or are not familiar with all of its features then
Chapter \ref{chapter:chap2} and Chapter \ref{chapter:chap3} are provided
to get you started.  The aim in Chapter 2 is just to
help you write, run, and understand your first \justspice\
file.  In Chapter \ref{chapter:chap3}, each of the major types of analyses
\justspice\ can do for you are introduced, by example.
{In contrast, Chapter \ref{chapter:technic} is intended for those
wishing to understand how \spice\ works internally.}
Chapter \ref{chapter:input} describes in the format of the \spice\ input file
or netlist.
{Part II (Chapters \ref{chapter:statement} and
\ref{chapter:element}) describe the syntax of the \spice\
language and the predefined expressions provided within it.}
{Part III summarizes the \spice\ syntax, statements and
elements in a quick look-up form suitable for the experienced user.}
%Also provided are full descriptions of how \spice\ internally models each
%circuit element.
{
Chapter 9 presents more elaborate
\spice\ examples.
Also provided is a quick reference guide to \spice 's error
messages and their meaning (Appendix E).}


\section{What \spice\ Does}

Many different types of analyses are supported by different versionss
of \justspice.
Most versions allow all of the analysis types of \spicetwo\ plus a few
additional analyses.  One of the exceptions is the distortion analysis which
proved to be unreliable in \spicetwo.
The table below identifies the analyses that are common to virtually all
\justspice\ programs and the extended analyses
{by the \spicethree\ and \pspice\ versions included in this book.}
\begin{center}
\begin{tabular}{|l|l|}
\hline
\multicolumn{2}{|c|}{COMMON \justspice\ ANALYSES}\\
\hline
\hline
.AC & \ac\ Analysis\\
.DC & \dc\ Analysis\\
.FOUR & Fourier Analysis\\
.NOISE & Small-Signal Noise Analysis\\
.OP & Operating Point Analysis\\
.SENS & Sensitivity Analysis\\
.TRAN & Transient Analysis\\
\hline
\multicolumn{2}{|c|}{ANALYSES SPECIFIC TO \spicetwo}\\
\hline
\hline
.DISTO & Small-Signal Distortion Analysis\\
\hline
\multicolumn{2}{|c|}{ANALYSES SPECIFIC TO \spicethree}\\
\hline\\
\hline
.DISTO & Small-Signal Distortion Analysis\\
.PZ    & Pole-Zero Analysis\\
\multicolumn{2}{|c|}{ANALYSES SPECIFIC TO \pspice}\\
.TF & Transfer Function Specification \pspice\ Only\\
.MC & Monte Carlo Analysis (\pspice\ only)\\
.SAVEBIAS & Save Bias Conditions\\
.STEP & Parameteric Analysis\\
.WCASE & Sensistivity and Worst Case Analysis\\
\hline
\end{tabular}
\end{center}

{
\section{justspice\ versions}

This book is a manual for five versions of \justspice: \spicetwo,
\spicethree , \pspice and \hspice . For these the input syntax,
and models are described.
Particularly emphasis is given to \spicethree\ and \pspice\ as these are the
most widley used \justspice\ versions.
For these the graphical user interface is also described.

The syntax, analysis types, and elements of
\spicetwo form a common denominator with the capabilities of
\spicethree and \pspice being extensions.
Adhering to the \spicetwo\ syntax ensures maximum portability of
\spice\ netlists.  Major restrictions of this syntax compared to commercial
versions include using integers to designate nodes.
The \spicethree syntax is just a small extension
of the \spicetwo\ syntax and is fully upwards compatible from \spicetwo.
The \pspice syntax is a considerable enhancement over the \spicetwo syntax.
Highlights of the enhanced syntax are that node names are allowed which
greatly increases the readibility of the netlist, the use of
symbolic expressions in place of numeric values, passing parameters
to subcircuits, and many more analysis
types.  The \pspice\ syntax has become a second ``standard'' syntax.

Part II of this book serves as a combined user and reference manual while
Part III is a condensed reference manual aimed at the experiendenced user
needing to check syntax.
Descriptions of statements and elements are based on the \spicetwo\ syntax
with the \spicethree\ and \pspice\ extensions clearly identified.

\section{Documentation Conventions}

In this manual the general forms of statements and elements use the following
conventions to identify the type of input required:
\begin{enumerate}
\item Actual characters that must be typed by the user are in a typewriter
      font.
\item Input that must be replaced by a word or a numeric value is italicized.
\item Optional input is enclosed between square brackets ``\B\ \ \ \E''.
\item Input that can be optionally repeated is followed by a string of
      dots ``$\ldots$''.
\item As in \justspice\ input syntax a line is continued when a plus sign
      ``{\tt +}'' appears in the first character position of the continued line.
\end{enumerate}

As example the general form of resistor is\\[0.05in]
\hspace*{\fill}\offsetparbox{
{\it{\tt R}name $N_1$ $N_2$\\{\tt +}  ResistorValue {\tt IC=}$V_R$\E } }\\[0.05in]
Here the first character on the first line is {\tt R} which indicates that
this line describes a resistor element.  The full name of the resistor is
{\it {\tt R}name} where {\it name} can be replaced by any alphanumeric character
string that uniquely identifies the element. Thus ``{\tt R1}'', ``{\tt Rgate}''
and ``{\tt ROP\_AMP\_16\_2}'' are names of resistors. It should be noted that
\spice\ does not distinguish between upper and lower case characters.
{\it ResistorValue} must be replaced by the numeric value of the resistor
possibly including a scale factor. Thus {\tt 1MEG}, {\tt 1E6}, and {\tt 1000000}.
The complete \justspice\ input syntax is described in Chapter \ref{chapter:input}.
With the exception of the line continuation indicated by the leading {\tt +}
sign a \spice\ element or statement must be
fully contained on a single line.
