\element{B}{GaAs MESFET}
\kwversions{\pspice}

\begin{figure}[h]
\centering
\ \pfig{b.ps}
\caption{B --- GASFET element.
%: (a) enhancement-mode GASFET; (b) depletion-mode GASFET.
\label{b.ps}}
\end{figure}


\pspiceform{ {\tt B}name  NDrain NGate NSource ModelName \B Area\E }

\begin{widelist}
\item[{\it NDrain}] is the drain node
\item[{\it NGate}] is the gate node
\item[{\it NSource}] is the source node
\item[{\it ModelName}]  is  the  model name
\item[{\it Area}] is the area factor in dimensionless units\\
               (Units: none; Optional; Default: 1; Symbol: $Area$)
\end{widelist}

\example{ B1 1 2 3 GAAS12 \\
          B1 1 2 3 GAAS12 0.5 }

\modeltype{GASFET}
%
% GASFET
%
\modelx{GASFET}{}{GaAs MESFET Model}\label{BGASFETmodel}
\label{GASFETmodelpspice}
\begin{figure}[h]
\centering
\ \epsfxsize=2.75in\pfig{gaas.eps}
\caption[Schematic of the GASFET model]{Schematic of the GASFET model.
\label{bgasfet} $V_{GS}$, $V_{DS}$, and $V_{GD}$ are intrinsic gate-source,
drain-source and gate-drain voltages between the internal gate, drain, and
source terminals designated $G$, $D$, and $S$ respectively.
$R_I$ is not used in \pspice .}

\end{figure}
\vfill
\form{{\tt .MODEL} ModelName {\tt GASFET(} \B\B keyword {\tt =} value\E  ... \E
{\tt )}}

\example{.MODEL GAAS12 GASFET( LEVEL=1 ) }

\pspice\ provides three MESFET device models some microwave versions provide six.
The parameter {\tt LEVEL} specifies the model to be used:
\index{Curtice Quadratic Model, see GASFET, LEVEL 1}
\index{GASFET, LEVEL 1 (Curtice Quadratic) Model}
\index{Statz Model, see GASFET, LEVEL 2}
\index{GASFET, LEVEL 2 (Raytheon or Statz) Model}
\index{Raytheon Model, see GASFET, LEVEL 2}
\index{TOM Model, see GASFET, LEVEL 3}
\index{GASFET, LEVEL 3 (TOM) Model}
\index{Triquint Model, see GASFET, LEVEL 3}
\index{Curtice-Ettenberg Cubic Model, see GASFET, LEVEL 4}
\index{GASFET, LEVEL 4 (Curtice Cubic) Model}
\index{Materka-Kacprzak Model, see GASFET, LEVEL 5}
\index{GASFET, LEVEL 5 (Materka-Kacprzak) Model}
\index{Angelov Model, see GASFET, LEVEL 6}
\index{GASFET, LEVEL 6 (Angelov) Model}
\index{TOM2 Model, see GASFET, LEVEL -1}
\index{GASFET, LEVEL -1 (TOM2) Model}
\\[0.2in]
\noindent\begin{longtable}{p{0.9in}p{0.2in}p{4in}}
         LEVEL = 1& $\rightarrow$ &  Curtice Quadratic model\newline
         This was the first widely accepted model for a GaAs MESFET
         and described in \cite{curtice:80}.  It
         uses rather simple empirical fits to measured data.
         See page \pageref{b:level1234:start}.
         \hspace*{\fill}\versions{\pspice}\\[0.1in]
         LEVEL = 2& $\rightarrow$ &  Raytheon model\newline
         This model is also known as the Statz model and
         model was developed at Raytheon for the modeling of GaAs
         MESFETs used in digital circuits.  It is also based on empirical
         fits to measured data \cite{statz:87}.
         See page \pageref{b:level1234:start}.
         \hspace*{\fill}\versions{\pspice}
\\[0.1in]
         LEVEL = 3& $\rightarrow$ &  TOM or TriQuint model\newline
         The name of this model derives from {\it TriQuint's Own Model}
     \cite{mccamant:mccormack:90}.
         See page \pageref{b:level1234:start}.
         \hspace*{\fill}\versions{\pspice}
\\[0.1in]
         LEVEL = 4& $\rightarrow$ &  Curtice-Ettenberg Cubic model\newline
         This is a refinement on the LEVEL 1 model
         \cite{curtice:ettenberg:85}.  It also
         uses simple empirical fits to measured data.
         See page \pageref{b:level1234:start}.
         \hspace*{\fill}
\\[0.1in]
         LEVEL = 5& $\rightarrow$ &  Materka-Kacprzak model
         \cite{kacprzak:materka:83}\newline
         A distinguishing characteristic is that the drain-source current
         is analytic and so it has better convergence characteristics than
         the other models.
         See page \pageref{b:level5:start}.
         \hspace*{\fill}
\\[0.1in]
         LEVEL = 6& $\rightarrow$ &  Angelov model\newline
         Another empirical GASFET model with analytic characteristics.
         See page \pageref{b:level6:start}.
         \hspace*{\fill}
\\[0.1in]
         LEVEL = -1& $\rightarrow$ &  TOM-2 model\newline
         An improved model from TriQuint.
         Another empirical GASFET model with analytic characteristics.
         See page \pageref{b:level-1:start}.
         \hspace*{\fill}
%\end{tabular}
\end{longtable}

\newpage
\ \myThickLine
\label{b:level1234:start}
LEVEL 1, 2, 3 and 4 GASFET models
\myline

Many parameters of the LEVEL 1, 2, 3 and 4 {\tt GASFET} models are the same
and so these models will be considered together.  The parameter keywords
are given in table \ref{btable1234}.
It is assumed that the model parameters were determined or
measured at the nominal temperature $T_{\ms{NOM}}$ (default
$27^{\circ}$C) specified in the most recent {\tt .OPTIONS} statement
preceding the {\tt .MODEL} statement.

\begin{longtable}[h]{|p{0.6in}|p{3.5in}|p{0.6in}|p{0.6in}|}
\caption[MESFET model parameters.]{MESFET model parameters.}\\

\hline
\multicolumn{1}{|c}{\bf Name} &
\multicolumn{1}{|c}{\parbox{2.77in}{\bf Description}}  &
\multicolumn{1}{|c}{\bf Units} &
\multicolumn{1}{|c|}{\bf Default}\\ \hline
\endhead

\hline \multicolumn{4}{|r|}{{Continued on next page}} \\ \hline
\endfoot

\hline \hline
\endlastfoot

{\tt A0}& drain saturation current for $V_{GS}$ = 0
          ({\tt LEVEL=4}) \kwversion{\sspice} \sym{A_0}&A& 0.1\X
{\tt A1}& coefficient of $V_1$ (primary transconductance
          parameter)
          ({\tt LEVEL=4}) \version{\sspice} \sym{A_1}&A/V& 0.05\X
{\tt A2}& coefficient of $V_1^2$
          ({\tt LEVEL=4}) \hfill \version{\sspice} \sym{A_1}&$\mbox{A/V}^2$& 0\X
{\tt A3}& coefficient of $V_1^3$
          ({\tt LEVEL=4}) \hfill \version{\sspice} \sym{A_1}&$\mbox{A/V}^3$& 0\X
{\tt AA0}& temperature cofficient of {\tt A0}
          ({\tt LEVEL=4}) \kwversion{\sspice} \sym{A_{A0}}&A& 0\X
{\tt AA1}& temperature cofficient of {\tt A1}
          ({\tt LEVEL=4}) \kwversion{\sspice} \sym{A_{A1}}&A& 0\X
{\tt AA2}& temperature cofficient of {\tt A2}
          ({\tt LEVEL=4}) \kwversion{\sspice} \sym{A_{A2}}&A& 0\X
{\tt AA3}& temperature cofficient of {\tt A3}
          ({\tt LEVEL=4}) \kwversion{\sspice} \sym{A_{A3}}&A& 0\X
{\tt AALF} & linear temperature coefficient of {\tt ALPHA}
           ({\tt LEVEL=1,2,3,-1})
       \version{\sspice} \sym{A_{\alpha}}&$^{\circ}\mbox{C}^{-1}$& 0\X
{\tt AB} & linear temperature coefficient of {\tt B} ({\tt LEVEL=2,3})
        \kwversion{\sspice} \sym{A_B} & $^{\circ}\mbox{C}^{-1}$& 0  \X
{\tt ABET} & linear temperature coefficient of {\tt BETA}
        \kwversion{\sspice} \sym{A_{\beta}} & \%/$^{\circ}$C    & 0 \X
{\tt ACGS} & linear temperature coefficient of {\tt CGS} ({\tt LEVEL=1,3,4,-1})
       \kwversion{\sspice} \sym{A_{CGS}}&$^{\circ}\mbox{C}^{-1}$& 0\X
{\tt ACGD} & linear temperature coefficient of {\tt CGD} ({\tt LEVEL=1,3,4},-1)
       \kwversion{\sspice} \sym{A_{CGD}}&$^{\circ}\mbox{C}^{-1}$& 0\X
{\tt ADEL} & linear temperature coefficient of {\tt DELTA} ({\tt LEVEL=3,-1})
       \kwversion{\sspice} \sym{A_{\delta}}&$^{\circ}\mbox{C}^{-1}$& 0\X
{\tt AF}& flicker noise exponent\sym{A_F}& -    & 1 \X
{\tt AGAM}& linear temperature coefficient of {\tt GAMA} ({\tt LEVEL=1,3,4})
           \kwversion{\sspice} \sym{A_{\lambda}}& \%/$^{\circ}$C& 0 \X
{\tt ALAM}& linear temperature coefficient of {\tt LAMBDA} ({\tt LEVEL=1,2,3})
           \kwversion{\sspice} \sym{A_{\lambda}}& \%/$^{\circ}$C& 0 \X
{\tt ALPHA} & saturation voltage parameter
      ({\tt LEVEL=1,2,3,-1})\sym{\alpha}&$\mbox{V}^{-1}$& 2 \X
{\tt ALFA}& alternative keyword of {\tt ALPHA} ({\tt LEVEL=1,2,3,-1})
       \kwversion{\sspice} \sym{\alpha}&$\mbox{V}^{-1}$& 2\X
{\tt AQ} & linear temperature coefficient of {\tt Q} ({\tt LEVEL=3})
        \kwversion{\sspice} \sym{A_Q} & $^{\circ}\mbox{C}^{-1}$& 0  \X
{\tt AR1}    & linear temperature coefficient of {\tt R1} ({\tt LEVEL=1,4})
       \kwversion{\sspice} \sym{A_{R1}}&$^{\circ}\mbox{C}^{-1}$& 0\X
{\tt AR2}    & linear temperature coefficient of {\tt R2} ({\tt LEVEL=1,4})
       \kwversion{\sspice} \sym{A_{R2}}&$^{\circ}\mbox{C}^{-1}$& 0\X
{\tt ALPHATCE}   & exponential temperature coefficient of {\tt ALPHA}
           ({\tt LEVEL=-1})
        \sym{T_{C,\alpha}}& \%/$^{\circ}$C   & 0\\
{\tt ARD}    & alternative keyword for {\tt TRD1}
       \sym{A_{RD}}&$^{\circ}\mbox{C}^{-1}$& 0\X
{\tt ARF}    & linear temperature coefficient of {\tt RF} ({\tt LEVEL=1,4})
        \sym{A_{RF}}&$^{\circ}\mbox{C}^{-1}$& 0\X
{\tt ARG}    & alternative keyword for {\tt TRG1}
        \sym{A_{RG}}&$^{\circ}\mbox{C}^{-1}$& 0\X
{\tt ARI}    & linear temperature coefficient of {\tt RI}
        \sym{A_{RI}}&$^{\circ}\mbox{C}^{-1}$& 0\X
{\tt ARS}    & alternative keyword for {\tt TRS1}
        \sym{A_{RS}}&$^{\circ}\mbox{C}^{-1}$& 0\X
{\tt AT}    & linear temperature coefficient of {\tt TAU}
       \sym{A_{\tau}}&$^{\circ}\mbox{C}^{-1}$& 0\X
{\tt AU}    & linear temperature coefficient of {\tt U} ({\tt LEVEL=1,3})
        \sym{A_U}&$^{\circ}\mbox{C}^{-1}$& 0\X
{\tt AVDS} & linear temperature coefficient of {\tt VDS0}
          ({\tt LEVEL}=4)  \sym{A_{VDS0}}
                        &$^{\circ}\mbox{C}^{-1}$ & 0\X
{\tt AVBD} & linear temperature coefficient of {\tt VBD} \sym{A_{VBD}}
                        &$^{\circ}\mbox{C}^{-1}$ & 0\X
{\tt AVT0} & (AVT-zero) linear temperature coefficient of {\tt VTO}
           \sym{A_{VT0}} &$^{\circ}\mbox{C}^{-1}$ & 0\X
{\tt B} & doping tail extending parameter \sym{B} \newline
          \version{\pspice} ({\tt LEVEL=2}) \newline
          \version{\sspice} ({\tt LEVEL=2,3})
    &$\mbox{V}^{-1}$& 0.3 \X
{\tt BETA}& transconductance coefficient
       ({\tt LEVEL=1,2,3}) \sym{\beta}&$\mbox{A/V}^2$& 0.1\X
{\tt BETA}& transconductance coefficient
           ({\tt LEVEL=4}) \kwversion{\sspice} \sym{\beta}&$\mbox{1/V}^2$& 0\X
{\tt BETATCE} & exponential temperature coefficient of\newline {\tt BETA}
         \sym{T_{C,{\beta}}} & \%/$^{\circ}$C   & 0\X
{\tt BRD}   & quadratic  temperature coefficient of {\tt RD}
                  \kwversion{\sspice}  \sym{B_{RD}}
    & $^{\circ}\mbox{C}^{-2}$ & 0 \X
{\tt BRG}   & quadratic  temperature coefficient of {\tt RG}
                  \kwversion{\sspice}  \sym{B_{RG}}
    & $^{\circ}\mbox{C}^{-2}$ & 0 \X
{\tt BRI}   & quadratic  temperature coefficient of {\tt RI}
                  \kwversion{\sspice}  \sym{B_{RI}}
    & $^{\circ}\mbox{C}^{-2}$ & 0 \X
{\tt BRS}   & quadratic  temperature coefficient of {\tt RS}
                  \kwversion{\sspice}  \sym{B_{RD}}
    & $^{\circ}\mbox{C}^{-2}$ & 0 \X
{\tt BVT0}  & quadratic  temperature coefficient of {\tt VT0}
                  \kwversion{\sspice}  \sym{B_{VT0}}
    & $^{\circ}\mbox{C}^{-2}$ & 0 \\
{\tt CDS}& drain-source capacitance
        \sym{C_{DS}}& F & 0 \X
{\tt CGD}& zero-bias gate-drain p-n capacitance \sym{C_{GD}}& F & 0\X
{\tt CGD0}& alternative keyword for {\tt CGD}
        \kwversion{\sspice} \sym{C_{GD}}& F & 0\X
{\tt CGS}& zero-bias gate-source p-n capacitance
        \sym{C_{GS}}& F & 0 \X
{\tt CGS0}& alternative keyword for {\tt CGS}
        \kwversion{\sspice} \sym{C_{GS}}& F & 0\X
{\tt DELTA}& output feedback parameter ({\tt LEVEL=3})
         \sym{\delta}&$\mbox{(AV)}^{-1}$& 0 \X
{\tt DELTA}& output feedback parameter ({\tt LEVEL=-1})
         \sym{\delta}&$\mbox{(AV)}^{-1}$& 0.2   \X
{\tt DELT}& alternative keyword for {\tt DELTA} ({\tt LEVEL=3})
         \kwversion{\sspice} \sym{\delta}&$\mbox{(AV)}^{-1}$& 0 \X
{\tt DELT}& alternative keyword for {\tt DELTA} ({\tt LEVEL=-1})
         \kwversion{\sspice} \sym{\delta}&$\mbox{(AV)}^{-1}$& 0 \X
{\tt DLVL}& breakdown model flag\hfill {\tt not used}\newline
        \begin{tabular}{ll}
        {\tt DLVL} &= 1 use original model\\
                   &= 2 use enhanced model
        \end{tabular}
      &-&1\X
{\tt E}& drain current power law coefficient
        \kwversion{\sspice} ({\tt LEVEL=1}) \sym{E}& -  & 2\X
{\tt EG} & bandgap voltage (barrier height) at 0~K\sym{E_G(0)}
                      \kwnote{ Schottky Barrier Diode: 0.69}
                      \kwnote{ Silicon: 1.16}
                      \kwnote{ Gallium Arsenide: 1.52}
                      \kwnote{ Germanium: 0.67}
         ({\tt LEVEL=1,2,3,4})  & eV    & 1.52  \X
{\tt EG} & bandgap voltage (barrier height) at 0~K\sym{E_G(0)}
                      \kwnote{ Schottky Barrier Diode: 0.69}
                      \kwnote{ Silicon: 1.16}
                      \kwnote{ Gallium Arsenide: 1.52}
                      \kwnote{ Germanium: 0.67}
         ({\tt LEVEL=-1})  & eV & 1.11  \X
{\tt FC} & forward-bias depletion capacitance factor\sym{F_C}& -    & 0.5   \X
{\tt GAMMA} & Static feedback parameter
              also known as \newline voltage slope parameter of
                  pinch-off voltage
         \kwversion{\pspice} ({\tt LEVEL=3})
         \kwversion{\sspice} ({\tt LEVEL=1,3})
         \sym{\gamma} & -   & 0 \X
{\tt GAMA}  &  alternative keyword for {\tt GAMMA}
         \kwversion{\sspice} \hfill ({\tt LEVEL 1,3,-1}) \sym{\gamma} & -& 0\X
{\tt GAMA}  & Slope of drain characteristic in the linear region
         \newline ({\tt LEVEL=4}) \hfill \version{\sspice}
         \sym{\gamma'} & -  & 1.5   \X
{\tt GAMMATCE} & exponential temperature coefficient of\newline {\tt GAMMA}
         \sym{T_{C,{\gamma}}} & \%/$^{\circ}$C  & 0\X
{\tt GAP1}  & First bandgap correction factor
           \sym{F_{\ms{GAP1}}}
                     \kwnote{  Silicon: 0.000473}
                     \kwnote{  Old Value for Silicon: 0.000702}
                     \kwnote{  Gallium Arsenide: 0.000541}
                     \kwnote{  Germanium: 0.000456}
          \kwversion{\hspice ; \sspice}
     &  eV/$^{\circ}$C& 0.000541    \\
{\tt GAP2}  & Second bandgap correction factor
           \sym{F_{\ms{GAP2}}}
                     \kwnote{  Silicon: 0.000636}
                     \kwnote{  Old Value for Silicon: 0.001108}
                     \kwnote{  Gallium Arsenide: 0.000204}
                     \kwnote{  Germanium: 0.000210}
          \kwversion{\hspice ; \sspice}
     &  $^{\circ}$C& 0.000204   \X
{\tt GMAX}& enhanced breakdown model parameter {\tt not used}&S&0\X
{\tt IS}& gate p-n saturation current\sym{I_S}&A    & 1E-14 \X
{\tt K1}& enhanced breakdown model parameter {\tt not used}&V$^{-1}$&0\X
{\tt K2}& enhanced breakdown model parameter {\tt not used}&V&0\X
{\tt K3}& enhanced breakdown model parameter {\tt not used}&V$^2$&0\X
{\tt KF}& flicker noise coefficient\sym{K_F}& - & 0 \X
{\tt LAMBDA}& channel-length modulation ({\tt LEVEL=1,2,3,-1})
              \sym{\lambda}&$\mbox{V}^{-1}$& 0 \X
{\tt LAMB}& alternative keyword for {\tt LAMBDA} ({\tt LEVEL=1,2,3})
           \kwversion{\sspice} \sym{\lambda}&$\mbox{V}^{-1}$& 0 \X
{\tt LEVEL}& model index
              \hfill  1 $\rightarrow$ Curtice quadratic model
              \kwnote{  2 $\rightarrow$ Raytheon model}
              \kwnote{  3 $\rightarrow$ TOM (Triquint) model}
              \kwnote{  4 $\rightarrow$ Curtice cubic model}
              \kwnote{  5 $\rightarrow$ Materka-Kacprzak model}
              \kwnote{  6 $\rightarrow$ Angelov model}\newline
{\tt LEVEL}s  4,5,6 \sspice\ only
        & -  &  1   \X
{\tt M} & gate p-n grading coefficient \sym{M}& -& 0.5  \X
{\tt MGS}& gate-source p-n grading coefficient
           \kwversion{\sspice} \sym{M_{GS}}& - & M\X
{\tt MGD}& gate-drain p-n grading coefficient
           \kwversion{\sspice}  \sym{M_{GS}}& - & M\X
{\tt N} & gate p-n emission coefficient\sym{n}& -   & 1 \X
{\tt NG}& constant part of threshold ideality factor
{\tt LEVEL} -1 \sspice\ only \sym{N_G}& -   & 1 \X
{\tt ND}& part of threshold ideality factor that depends on $V_{DS}$
{\tt LEVEL} -1 \sspice\ only \sym{N_D}& V$^{-1}$& 0 \X
{\tt NPLT} &  ({\sc not used})
          \version{\sspice} \sym{V_{GMN}}& s    &  0    \X
{\tt Q} & power-law parameter ({\tt LEVEL=3,-1})
         \sym{Q}& - &  2    \X
{\tt R1}    & breakdown gate-drain resistance ({\tt LEVEL}=1,4)
          \kwversion{\sspice} \hfill{\sc not used} \sym{R_1}&$\Omega$& $\infty$ \X
{\tt R2}    & breakdown dependency on channel current
        ({\tt LEVEL}=1,4)\kwversion{\sspice}\hfill({\tt LEVEL}=1,4)
    \hfill{\sc not used}\sym{R_2}&$\Omega$&0\X
{\tt RD}    & drain resistance
          \sym{R_D}&$\Omega$& 0 \X
{\tt RF}    & forward-biased gate-source resistance ({\tt LEVEL}=1,4)
          \kwversion{\sspice} \sym{R_F}&$\Omega$& $\infty$  \\
{\tt RG}& gate resistance
          \sym{R_G}&$\Omega$& 0 \X
{\tt RI}& channel resistance\hfill ({\sc not used}) \hfill
          \kwversion{\sspice}
          \sym{R_I}&$\Omega$& 0 \X
{\tt RS}    & source resistance
          \sym{R_S}&$\Omega$& 0\X
{\tt T} & alternative keyword for {\tt TAU}
          \kwversion{\sspice} \sym{\tau}& s &  0    \X
{\tt TBET}  & alternative keyword for {\tt BETATCE}
          \kwversion{\sspice} \sym{T_{C,{\beta}}}& s    &  0    \X
{\tt TJ} & junction temperature \hfill  ({\sc not used}) \newline
          \version{\sspice} \sym{T_J}& s    &  0    \X
{\tt TM} &  ({\sc not used})
          \version{\sspice} \sym{T_M}& s    &  0    \X
{\tt TME} &  ({\sc not used})
          \version{\sspice} \sym{T_{ME}}& s &  0    \X
{\tt TNOM} & nominal temperature ({\sc not used})
          \kwversion{\sspice} \sym{T_{\ms{NOM}}}& s &  0    \X
{\tt TAU}  & conduction current delay time\sym{\tau}& s &  0    \X
{\tt TRG1} & linear temperature coefficient of {\tt RG}\sym{A_{RG}}
           & $^{\circ}\mbox{C}^{-1}$ & 0    \X
{\tt TRD1} & linear temperature coefficient of {\tt RD}\sym{A_{RD}}
           & $^{\circ}\mbox{C}^{-1}$ & 0    \X
{\tt TRS1} & linear temperature coefficient of {\tt RS}\sym{A_{RS}}
           & $^{\circ}\mbox{C}^{-1}$ & 0    \X
{\tt U}& critical field parameter for mobility degradation
           ({\tt LEVEL=1,2,3}) \kwversion{\sspice} \sym{U}&V/m& 0   \X
{\tt VBI}& gate p-n potential\sym{V_{\ms{BI}}}& V&1\X
{\tt VBR}& enhanced breakdown model parameter {\tt not used}&V$^2$&$\infty$\X
{\tt VBD}  & breakdown voltage
          \kwversion{\sspice} \sym{A_{VBD)}} & V & $\infty$\X
{\tt VDELTA}
    & capacitance transistion voltage \newline
          ({\tt LEVEL}=2,3) \sym{V_{\Delta}}
                        & V & 0.2   \X
{\tt VDS0}
    & $V_{DS}$ at which {\tt BETA} was measured\newline
          ({\tt LEVEL}=4) \hfill \sym{V_{\Delta}}
                        & V & 4 \X
{\tt VGMN} &  ({\sc not used})
           \sym{V_{GMN}}& s    &  0    \X
{\tt VGMX} &  ({\sc not used})
           \sym{V_{GMN}}& s    &  0    \X
{\tt VDMX} &  ({\sc not used})
           \sym{V_{GMN}}& s    &  0    \X
{\tt VMAX} & capacitance limiting voltage \newline ({\tt LEVEL=2,3})
\sym{V_{\ms{MAX}}}& V   & 0.5   \X
{\tt VTO}    & (VT-oh)  pinch-off voltage \sym{V_{T0}}& V  & -2.5   \X
{\tt VT0} & (VT-0) alternative keyword for {\tt VTO}
          \kwversion{\sspice} \sym{A_{VDS0)}} & V & -2.5\X
{\tt VTOTC} & linear temperature coefficient of {\tt VTO}\sym{T_{C,VT0}}
            & V/$^\circ$C   & 0 \X
{\tt XTI} & temperature exponent of {\tt IS}\sym{X_{TI}}& - & 0\X
{\tt VBITC} & linear temperature coefficient of {\tt VBI}\sym{T_{C,BI}}& -  & 0\\
\hline
\end{longtable}

Some versions of \justspice\ use incorrect default
       values for some of the parameters.  One example is the default
       value for EG.  The accepted value has changed with time.  It is
       always a good idea not to rely on default values other than 0.
       %\cite{sze2e,szehighspeed}

The physical constants used in the model evaluation are
\begin{center}
\begin{tabular}{|l|l|l|}
\hline
$k$ & Boltzmann's constant           &  $1.3806226\,10^{-23}$~J/K\\
$q$ & electronic charge             & $1.6021918\,10^{-19}$~C\\
\hline
\end{tabular}
\end{center}

\noindent\underline{\sl \large Standard Calculations}\\[0.1in]
Absolute temperatures (in kelvins, K) are used.
The thermal voltage
$V_{\ms{TH}} = kT / q$
and the band gap energy at the nominal temperature is
\begin{equation}
E_G(T_{\ms{NOM}})=
    E_G(0)-F_{\ms{GAP1}}{{4T_{\ms{NOM}}^2}/
            \left({T_{\ms{NOM}}+F_{\ms{GAP2}}} \right) }.
\end{equation}
Here $E_G(0)$ is the parameter {\tt EG} --- the band gap energy at 0~K.
$F_{\ms{GAP1}}$ and $F_{\ms{GAP2}}$ are not parameters in
\justpspice.  \justpspice\ documentation indicates that \pspice\ uses
$F_{\ms{GAP1}}$ = 0.000702 and $F_{\ms{GAP2}}$ = 1108.
\\[0.2in]

\noindent\underline{\sl \large Temperature Dependence}
\index{GASFET, Temperature Dependence}
\index{Temperature Dependence, see GASFET}
\\[0.1in]
Temperature effects are incorporated as follows where $T$ and $T_{\ms{NOM}}$
are absolute temperatures in Kelvins (K).

\begin{align}
\alpha(T) & =  \alpha(T_{\ms{NOM}}) \left(
                1.01^{\textstyle(T_{C,\alpha}(T-T_{\ms{NOM}}))}
                + A_{\alpha}(T-T_{\ms{NOM}}) \right)
                \\
\beta(T) & =  \beta(T_{\ms{NOM}}) \left(
                1.01^{\textstyle(T_{C,\beta}(T-T_{\ms{NOM}}))}
                + A_{\beta}(T-T_{\ms{NOM}}) \right)
                \\
I_S (T) & =  I_S(T_{\ms{NOM}})
        e^{\left( \textstyle E_G(T) {T / {T_{\ms{NOM}}}}
            - E_G(T) \right) {\textstyle /(nV_{\ms{TH}})}}
    \left({{\textstyle T} / {\textstyle T_{\ms{NOM}}}}
    \right)^{ \left( \textstyle X_{TI}/n \right) }
\end{align}

~

\begin{align}
C'_{GS} (T)&= \left\{ \begin{array}{l}
       C_{GS}(T_{\ms{NOM}})
       \left\{1 + M_{GS} \left[0.0004(T-T_{\ms{NOM}})+\left(1-
   {{\textstyle V_{BI} (T)} \over {\textstyle V_{BI}(T_{\ms{NOM}})}}\right)
    \right]\right\} \nonumber\\
    \hspace*{\fill} A_{CGS} \mbox{ not specified} \\
       C_{GS}(T_{\ms{NOM}})(1 + A_{CGS}(T-T_{\ms{NOM}}))
    \hspace*{\fill} A_{CGS} \mbox{ specified} \\
      \end{array} \right. \\  %}
C'_{GD} (T)&= \left\{ \begin{array}{l}
       C_{GD}(T_{\ms{NOM}})\left\{1 + M_{GD}
        \left[0.0004(T-T_{\ms{NOM}})+\left(1-
   {{\textstyle V_{BI} (T)} \over
    {\textstyle V_{BI}(T_{\ms{NOM}})}}\right) \right]\right\}\nonumber\\
    \hspace*{\fill} A_{CGD} \mbox{ not specified} \\
       C_{GD}(T_{\ms{NOM}})(1 + A_{CGD}(T-T_{\ms{NOM}}))
   \hspace*{\fill}  A_{CGD} \mbox{ specified} \\
      \end{array} \right. %}
\end{align}

~

\begin{align}
E_G(T) & =  E_G(0) -
      F_{\ms{GAP1}}{{4T^2} / \left( {T+F_{\ms{GAP2}}} \right) }\\
\lambda(T) &= \lambda(T_{\ms{NOM}})(1 + A_{\lambda}(T-T_{\ms{NOM}} )) \\
\alpha(T) &= \left\{ \begin{array}{l}
              \alpha(T_{\ms{NOM}})(1 + A_{\alpha}(T-T_{\ms{NOM}} ))
          \hspace*{\fill} \mbox{LEVEL = 1,2,3}\\
              \alpha(T_{\ms{NOM}})(1.01^
          {T_{C,\alpha}(T-T{\ms{NOM}})} + A_{\alpha}(T-T_{\ms{NOM}} )) \\
          A_{\alpha}(T-T_{\ms{NOM}} ))
          \hspace*{\fill} \mbox{LEVEL = -1}\\
           \\ \end{array}
          \right. %}
\end{align}

~

\begin{align}
U(T) &= U(T_{\ms{NOM}})(1 + A_{U}(T-T_{\ms{NOM}} )) \\
A_0(T) &= A_0(T_{\ms{NOM}})(1 + A_{A0}(T-T_{\ms{NOM}} )) \\
A_1(T) &= A_1(T_{\ms{NOM}})(1 + A_{A1}(T-T_{\ms{NOM}} )) \\
A_2(T) &= A_2(T_{\ms{NOM}})(1 + A_{A2}(T-T_{\ms{NOM}} ))\\
A_3(T) &= A_3(T_{\ms{NOM}})(1 + A_{A3}(T-T_{\ms{NOM}} ))
\end{align}

~

\begin{align}
\delta(T) &= \delta(T_{\ms{NOM}})(1 + A_{\delta}(T-T_{\ms{NOM}} )) \\
\gamma(T) &= \gamma(T_{\ms{NOM}})(1 + A_{\gamma}(T-T_{\ms{NOM}} )) \\
Q(T) &= Q(T_{\ms{NOM}})(1 + A_{Q}(T-T_{\ms{NOM}} ))\\
R_1(T) &= R_1(T_{\ms{NOM}})(1 + A_{R1}(T-T_{\ms{NOM}} )) \\
R_2(T) &= R_2(T_{\ms{NOM}})(1 + A_{R2}(T-T_{\ms{NOM}} )) \\
R_D(T) & =  R_D(T_{\ms{NOM}}) \left( 1 + A_{RD} (T-T_{\ms{NOM}})
                                    + B_{RD}(T-T_{\ms{NOM}})^2 \right)\\
R_F(T) &= R_F(T_{\ms{NOM}})(1 + A_{RF}(T-T_{\ms{NOM}} )) \\
R_G(T) & =  R_G(T_{\ms{NOM}}) \left( 1 + A_{RG}(T-T_{\ms{NOM}})
                                    + B_{RG}(T-T_{\ms{NOM}})^2 \right)\\
R_S(T) & =  R_S(T_{\ms{NOM}}) \left( 1 + A_{RS}(T-T_{\ms{NOM}})
                                    + B_{RS}(T-T_{\ms{NOM}})^2 \right)\\
R_I(T) &= R_I(T_{\ms{NOM}}) \left( 1 + A_{RI}(T-T_{\ms{NOM}} )
                                    + B_I(T-T_{\ms{NOM}})^2 \right)
\end{align}

~

\begin{align}
\tau(T) &= \tau(T_{\ms{NOM}})(1 + A_{\tau}(T-T_{\ms{NOM}} )) \\
V_{BD}(T) &= V_{BD}(T_{\ms{NOM}})(1 + A_{VBD}(T-T_{\ms{NOM}} )) \\
V_{BI} (T) & =  \left\{ \begin{array}{l}
 V_{BI}(T_{\ms{NOM}}) {T / {T_{\ms{NOM}}}}
 - 3V_{\ms{TH}} \ln{\left( {{T} / {T_{\ms{NOM}}}}\right) }
        \hfill {\tt LEVEL \ne -1} \\
 V_{BI}(T_{\ms{NOM}}) {T / {T_{\ms{NOM}}}}
+ T_{C,VBI}({T- T_{\ms{NOM}}})
        \hfill {\tt LEVEL = -1} \\
                         \end{array} \right.   %}
 \nonumber\\
           & ~~+ E_G (T_{\ms{NOM}}) {T / {T_{\ms{NOM}}}} -E_G(T)\\
V_{DS0}(T) &= V_{DS0}(T_{\ms{NOM}})(1 + A_{VDS0}(T-T_{\ms{NOM}} ))\\
V_{T0}(T)&=V_{T0} \left( 1 + A_{VT0}(T- T_{NOM})
                            + B_{VT0}(T- T_{NOM})^2 \right)\nonumber\\
            &\hspace*{1in} + T_{C,VT0}(T- T_{NOM})\\
 V_{VMAX}(T)&=V_{VMAX}(T_{\ms{NOM}}) + T_{C,VMAX}({T- T_{\ms{NOM}}})
        \hfill {\tt LEVEL = -1}
\end{align}

\noindent\underline{\sl \large Parasitic Resistances}\\[0.1in]
\index{Parasitic Resistances, see GASFET, \pspice}
\index{GASFET, \pspice\ Parasitic Resistance}
\index{GASFET, \pspice\ $R_S$}
\index{GASFET, \pspice\ $R_G$}
\index{GASFET, \pspice\ $R_D$}
\index{MESFET, see GASFET (\pspice)}

The resistive parasitics
$R'_S$, and $R'_D$ are calculated from the sheet resistivities
{\tt RS} (= $R_S$) and {\tt RD} (= $R_D$), and the
$Area$ specified on the element line.
{\tt RG} (= $R_G$) is used as supplied.

\begin{eqnarray}
R'_D & = & R_D/Area\\
R'_G & = & \left\{ \begin{array}{ll}
                   R_G      & \mbox{\pspice}\\
                   R_G/Area & \mbox{\sspice}
              \end{array}\right. %}
\\
R'_S & = & R_S/Area
\end{eqnarray}

The parasitic resistance parameter dependencies are summarized in
figure \ref{b1234para}.
%\bigskip
\begin{figure}[h]
\parbox[t]{1.3in}{
\begin{tabular}[t]{|p{1in}|}
\hline
\multicolumn{1}{|c|}{KEYWORD} \\
\multicolumn{1}{|c|}{PARAMETERS} \\
\hline
\hline
{\tt RD} \hfill $R_D$\\
{\tt RG} \hfill $R_G$\\
{\tt RS} \hfill $R_S$\\
\hline
\end{tabular}
}
\hfill
\parbox{0.2in}{\ \vspace*{0.2in}\newline +}
\hfill
\begin{tabular}[t]{|p{1in}|}
\hline
\multicolumn{1}{|c|}{GEOMETRY} \\
\multicolumn{1}{|c|}{PARAMETER} \\
\hline
$Area$\\
\hline
\end{tabular}
\hfill
\parbox{0.2in}{\ \vspace*{0.2in}\newline $\rightarrow$}
\hfill
\begin{tabular}[t]{|p{1.8in}|}
\hline
\multicolumn{1}{|c|}{DEVICE}\\
\multicolumn{1}{|c|}{PARAMETERS}\\
\hline
\hspace*{\fill} $R'_D = f(Area, R_D)$\\
\hspace*{\fill} $R'_G = f(Area, R_G)$\\
\hspace*{\fill} $R'_S = f(Area, R_S)$\\
\hline
\end{tabular}
\caption{MESFET parasitic resistance parameter
relationships. \label{b1234para}}
\end{figure}

\noindent\underline{\bf Leakage Currents}\\[0.1in]

\index{Leakage Currents, see GASFET}
\index{GASFET, I/V Characteristics, Leakage Currents}
\index{GASFET, Leakage Currents}
Current flows across the normally reverse biased gate-source and gate-drain
junctions.
The gate-source leakage current
$I_{GS} = Area\,I_{S}e^{(\textstyle V_{GS}/V_{\ms{TH}} -1)}$\inlineeq
and the gate-drain leakage current
$I_{GD} = Area\,I_{S}e^{(\textstyle V_{GD}/V_{\ms{TH}} -1)}$\inlineeq

The dependencies of the parameters describing the leakage current
are summarized in figure \ref{bleakage}.\\[0.2in]
\begin{figure}[h]
\begin{tabular}[t]{|p{1in}|}
\hline
\multicolumn{1}{|c|}{KEYWORD} \\
\multicolumn{1}{|c|}{PARAMETERS} \\
\hline
\hline
{\tt IS} \hfill $I_S$\\
\hline
\end{tabular}
\hfill
\parbox{0.2in}{\ \vspace*{0.2in}\newline +}
\hfill
\begin{tabular}[t]{|p{1in}|}
\hline
\multicolumn{1}{|c|}{GEOMETRY} \\
\multicolumn{1}{|c|}{PARAMETER} \\
\hline
$Area$\\
\hline
\end{tabular}
\hfill
\parbox{0.2in}{\ \vspace*{0.2in}\newline $\rightarrow$}
\hfill
\begin{tabular}[t]{|p{1.8in}|}
\hline
\multicolumn{1}{|c|}{DEVICE} \\
\multicolumn{1}{|c|}{PARAMETERS} \\
\hline
\hspace*{\fill}$I_{GS} = f(I_S, Area)$\\
\hspace*{\fill}$I_{GD} = f(I_S, Area)$\\
\hline
\end{tabular}
\caption{GASFET leakage current parameter dependencies. \label{bleakage}}
\end{figure}

\noindent{\bf LEVEL 1 (Curtice Model)}\myline
\noindent\underline{\sl {\tt LEVEL} 1 (Curtice Model)
I/V Characteristics}\\[0.1in]

\index{GASFET, LEVEL 1 (Curtice) Model, I/V}
\index{GASFET, LEVEL 1 (Curtice) Model, I/V}
\index{I/V Characteristics, see GASFET}
\index{GASFET, I/V Characteristics}
\index{I-V characteristics, see GASFET}
The {\tt LEVEL 1} current/voltage characteristics are evaluated after first
determining the mode (normal: $V_{DS} \ge 0$ or inverted:
$V_{DS} < 0$) and the region (cutoff,
linear or saturation) of the current
$(V_{DS}, V_{GS})$ operating point. Curtice \cite{curtice:80}
proposed two \dc\ current models: a quadratic channel current model
and a cubic channel current model.  The quadratic channel model is
implemented as the {\tt LEVEL 1} model.\\[0.1in]
\noindent{\sl Normal Mode: ($V_{DS} \ge 0$)}\\[0.2in]
The regions of operation are defined as follows
with $V_{GST} = V_{GS}-(V_{T0}-\gamma V_{DS})$ \inlineeq
\hspace*{\fill}\offsetparbox{
\begin{tabular}{ll}
cutoff region:&$V_{GST}(t-\tau) \le 0$\\
linear and saturation regions:&$V_{GST}(t-\tau) > 0$\\
\end{tabular}}\\[0.1in]
Then
\begin{equation}
I_{DS} = \left\{ \begin{array}{ll}
      0  & \mbox{cutoff region} \\ \\
      Area {{\textstyle \beta \left(1 + \lambda V_{DS}\right)}
      \over {\textstyle 1+UV_{GST}(t-\tau)}}
       V_{GST}^E(t-\tau)
      \mbox{tanh}\left(\alpha V_{DS}\right)
         &\mbox{linear, saturation regions} \end{array} \right. %}
      \label{b1id}
\end{equation}
\noindent{\sl Inverted Mode: ($V_{DS} < 0)$}\\[0.2in]
In the inverted mode the MESFET I/V characteristics are evaluated as in the
normal mode (\ref{b1id}) but with the drain and source subscripts
exchanged, and $V_{GD}$ is the controlling voltage instead of $V_{GS}$.

The relationships of the parameters describing the I/V
characteristics for the {\tt LEVEL} 1 model are summarized in figure
\ref{blevel1i/v}.\\[0.1in]
\begin{figure}[h]
\begin{tabular}[t]{|p{1in}|}
\hline
\multicolumn{1}{|c|}{KEYWORD} \\
\multicolumn{1}{|c|}{PARAMETERS} \\
\hline
\hline
{\tt ALPHA} \hfill $\alpha$\\
{\tt BETA} \hfill $\beta$\\
{\tt LAMBDA} \hfill $\lambda$\\
{\tt U} \hfill $U$\\
{\tt VTO} \hfill $V_{T0}$\\
\hline
\end{tabular}
\hfill
\parbox{0.2in}{\ \vspace*{0.2in}\newline +}
\hfill
\begin{tabular}[t]{|p{1in}|}
\hline
\multicolumn{1}{|c|}{GEOMETRY} \\
\multicolumn{1}{|c|}{PARAMETER} \\
\hline
\hspace*{\fill}$Area$\\
\hline
\end{tabular}
\hfill
\parbox{0.2in}{\ \vspace*{0.2in}\newline $\rightarrow$}
\hfill
\begin{tabular}[t]{|p{1.8in}|}
\hline
\multicolumn{1}{|c|}{DEVICE} \\
\multicolumn{1}{|c|}{PARAMETERS} \\
\hline
$I_{DS} = f(Area, \alpha, \beta, \lambda, U, V_{T0})$\\
\hline
\end{tabular}
\caption{LEVEL 1 (Curtice model) I/V dependencies. \label{blevel1i/v}}
\end{figure}

\noindent\underline{\sl LEVEL 1 (Curtice Model) Capacitances}\\[0.1in]
The drain-source capacitance
\begin{equation}
C'_{DS} = Area\,C_{DS}
\end{equation}
The gate-source capacitance
\begin{equation}
C'_{GS} = \left\{ \begin{array}{ll}
         Area\,C_{GS}\left(1 - {{\textstyle V_{GS}}\over{\textstyle V_{BI}}}
         \right)^{-M_{GS}}
         & V_{GS} \le F_C V_{BI}\\
         Area\,C_{GS}\left(1 -F_C\right)^{-(1+M_{GS})}
         \left[1-F_C(1+M_{GS})+M_{GS} {{\textstyle V_{GS}}\over{\textstyle V_{BI}}}
         \right]
         & V_{GS} > F_C V_{BI}
         \end{array} \right. %}
\end{equation}
The gate-drain capacitance
\begin{equation}
C'_{GD} = \left\{ \begin{array}{ll}
         Area\,C_{GD}\left(1 - {{\textstyle V_{GD}}\over{\textstyle V_{BI}}}
         \right)^{\textstyle -M_{GD}}
         & V_{GD} \le F_C V_{BI}\\
         Area\,C_{GD}\left(1 -F_C\right)^{\textstyle -(1+M_{GD})}
         \left[1-F_C(1+M_{GD})+M_{GD} {{\textstyle V_{GD}}\over
         {\textstyle V_{BI}}} \right]
         & V_{GS} > F_C V_{BI}
         \end{array} \right. %}
\end{equation}

The {\tt LEVEL} 1 capacitance parameter dependencies are summarized in figure
\ref{blevel1cap}.\\[0.2in]
\begin{figure}[h]
\begin{tabular}[t]{|p{1in}|}
\hline
\multicolumn{1}{|c|}{KEYWORD} \\
\multicolumn{1}{|c|}{PARAMETERS} \\
\hline
\hline
{\tt CGD} \hfill $C_{GD}$\\
{\tt CGS} \hfill $C_{GS}$\\
{\tt CDS} \hfill $C_{DS}$\\
{\tt FC} \hfill $F_C$\\
{\tt VBI} \hfill $V_{BI}$\\
{\tt MGS} \hfill $M_{GS}$\\
{\tt MGD} \hfill $M_{GD}$\\
\hline
\end{tabular}
\hfill
\parbox{0.2in}{\ \vspace*{0.2in}\newline +}
\hfill
\begin{tabular}[t]{|p{1in}|}
\hline
\multicolumn{1}{|c|}{GEOMETRY} \\
\multicolumn{1}{|c|}{PARAMETER} \\
\hline
\hspace*{\fill}$Area$\\
\hline
\end{tabular}
\hfill
\parbox{0.2in}{\ \vspace*{0.2in}\newline $\rightarrow$}
\hfill
\begin{tabular}[t]{|p{1.8in}|}
\hline
\multicolumn{1}{|c|}{DEVICE} \\
\multicolumn{1}{|c|}{PARAMETERS} \\
\hline
$C'_{DS} = f(Area, C_{DS})$\\
$C'_{GD} =$\newline\hspace*{\fill}$ f(Area, C_{GD},F_C,V_{BI},M_{GD})$\\
$C'_{GS} =$\newline\hspace*{\fill}$ f(Area, C_{GS},F_C,V_{BI},M_{GS})$\\
\hline
\end{tabular}
\caption{LEVEL 1 (Curtice model) capacitance dependencies. \label{blevel1cap}}
\end{figure}

{\bf LEVEL 2 (Raytheon Model)}\myline
\noindent\underline{\sl {\tt LEVEL} 2 (Raytheon Model)
I/V Characteristics}\\[0.1in]
\index{GASFET, LEVEL 2 (Raytheon) Model, I/V}
\index{GASFET, LEVEL 2 (Raytheon) Model, I/V}
\index{I/V Characteristics, see GASFET}
\index{GASFET, I/V Characteristics}
\index{I-V characteristics, see GASFET}
The {\tt LEVEL 2} current/voltage characteristics are evaluated after first
determining the mode (normal: $V_{DS} \ge 0$ or inverted:
$V_{DS} < 0$) and the region (cutoff,
linear or saturation) of the current
$(V_{DS}, V_{GS})$ operating point.\\[0.1in]

\noindent{\sl Normal Mode: ($V_{DS} \ge 0$)}\\[0.2in]
The regions of operation are defined as follows
with $V_{GST} = V_{GS}-V_{T0}$ \inlineeq
\hspace*{\fill}\offsetparbox{
\begin{tabular}{ll}
cutoff region:&$V_{GST}(t-\tau) \le 0$\\
linear region:&$V_{GST}(t-\tau) > 0 \mbox{ and } V_{DS} \le 3/\alpha$\\
saturation region:&$V_{GS}(t-\tau) > V_{T0} \mbox{ and } V_{DS} > 3/\alpha$\\
\end{tabular}}\\[0.1in]
Then
\begin{equation}
I_{DS} = \left\{ \begin{array}{ll}
      0  & \mbox{cutoff region} \\ \\
      Area\,{{\textstyle\beta}\over{\textstyle 1 + U V_{GST}}}
      \left(1 + \lambda V_{DS}\right)
      {{\textstyle V_{GST}(t-\tau)^2}\over
      {\textstyle 1 + BV_{GST}(t-\tau)}}
      \mbox{Ktanh}
         &\mbox{linear and saturation}\\
         &\mbox{regions} \end{array} \right. %}
      \label{b2id}
\end{equation}

where
\begin{equation}
\mbox{Ktanh} = \left\{ \begin{array}{ll}
       1 - \left(1 - V_{DS} \frac{\textstyle\alpha}{\textstyle 3}\right)^3
        & \mbox{linear region} \\ \\
      1
         &\mbox{saturation regions} \end{array} \right. %}
\end{equation}
is a Taylor series approximation to the tanh function
of the {\tt LEVEL} 1 model.\\[0.1in]
\noindent{\sl Inverted Mode: ($V_{DS} < 0)$}\\[0.2in]
In the inverted mode the MESFET I/V characteristics are evaluated as in the
normal mode (\ref{b2id}) but with the drain and source subscripts
exchanged, and $V_{GD}$ is the controlling voltage instead of $V_{GS}$.

The relationships of the parameters describing the I/V
characteristics of the {\tt LEVEL} 2 model are summarized in figure
\ref{blevel2i/v}.\\[0.1in]
\begin{figure}[h]
\begin{tabular}[t]{|p{1in}|}
\hline
\multicolumn{1}{|c|}{KEYWORD} \\
\multicolumn{1}{|c|}{PARAMETERS} \\
\hline
\hline
{\tt ALPHA} \hfill $\alpha$\\
{\tt B} \hfill $B$\\
{\tt BETA} \hfill $\beta$\\
{\tt LAMBDA} \hfill $\lambda$\\
{\tt U} \hfill $\mu$\\
{\tt VTO} \hfill $V_{T0}$\\
\hline
\end{tabular}
\hfill
\parbox{0.2in}{\ \vspace*{0.2in}\newline +}
\hfill
\begin{tabular}[t]{|p{1in}|}
\hline
\multicolumn{1}{|c|}{GEOMETRY} \\
\multicolumn{1}{|c|}{PARAMETER} \\
\hline
\hspace*{\fill}$Area$\\
\hline
\end{tabular}
\hfill
\parbox{0.2in}{\ \vspace*{0.2in}\newline $\rightarrow$}
\hfill
\begin{tabular}[t]{|p{1.8in}|}
\hline
\multicolumn{1}{|c|}{DEVICE} \\
\multicolumn{1}{|c|}{PARAMETERS} \\
\hline
$I_{DS} = f(Area, \alpha, $\newline\hspace*{\fill}$
B, \beta, \lambda, U V_{T0})$\\
\hline
\end{tabular}
\caption{LEVEL 2 (Raytheon model) I/V dependencies. \label{blevel2i/v}}
\end{figure}

\noindent\underline{\sl LEVEL 2 (Raytheon Model) Capacitances}
\label{b:raytheon:capacitance}
\\[0.1in]
This is a symmetrical capacitance model.  The drain-source capacitance
$C'_{DS} = Area\,C_{DS}$\inlineeq
The gate-source capacitance
\begin{equation}
C'_{GS} = Area\left[C_{GS}F_1F_2\left(1 - {{\textstyle V_{\ms{NEW}}}
   \over{\textstyle V_{BI}}}\right)^{\textstyle - \frac{1}{2}}
   + C_{GD}F_3\right]
\end{equation}
The gate-drain capacitance
\begin{equation}
C'_{GD} = Area\left[C_{GS}F_1F_3\left(1 - {{\textstyle V_{\ms{NEW}}}
   \over{\textstyle V_{BI}}}\right)^{\textstyle - \frac{1}{2}}
   + C_{GD}F_2\right]
\end{equation}
where
\begin{eqnarray}
F_1 & = & {{\textstyle 1}\over{\textstyle 2}} \left\{ 1 +
    {{\textstyle V_{\ms{EFF}}-V_{T0}}
    \over{\textstyle\sqrt{\left( V_{\ms{EFF}}-V_{T0}\right)^2+V_{\Delta}^2}}}\right\}\\
F_2 & = & {{\textstyle 1}\over{\textstyle 2}} \left\{ 1 +
    {{\textstyle V_{GS}-V_{GD}}
    \over{\textstyle\sqrt{\left( V_{GS}-V_{GD}\right)^2+\alpha^{-2}}}}\right\}\\
F_3 & = & {{\textstyle 1}\over{\textstyle 2}} \left\{ 1 -
    {{\textstyle V_{GS}-V_{GD}}
    \over{\textstyle\sqrt{\left( V_{GS}-V_{GD}\right)^2+\alpha^{-2}}}}\right\}\\
V_{\ms{EFF}} & = & {{\textstyle 1}\over{\textstyle 2}} \left\{V_{GS}+V_{GD}+
    \sqrt{\left( V_{GS}-V_{GD}\right)^2+\alpha^{-2}}\right\}\\
V_{\ms{NEW}} & = & \left\{ \begin{array}{ll}
    A_1 & A_1 < V_{\ms{MAX}} \\
    V_{\ms{MAX}} & A_1 \ge V_{\ms{MAX}} \\
      \end{array} \right. \\ %}
    A_1 & = & \frac{1}{2}\left[V_{\ms{EFF}} + V_{T0}
          + \sqrt{(V_{\ms{EFF}}-V_{T0})^2+V_{\Delta}^2}\right]
\end{eqnarray}

The capacitance parameter dependencies are summarized in figure
\ref{blevel2cap}.
\begin{figure}[h]
\begin{tabular}[t]{|p{1in}|}
\hline
\multicolumn{1}{|c|}{KEYWORD} \\
\multicolumn{1}{|c|}{PARAMETERS} \\
\hline
\hline
{\tt ALPHA} \hfill $\alpha$\\
{\tt CGD} \hfill $C_{GD}$\\
{\tt CGS} \hfill $C_{GS}$\\
{\tt CDS} \hfill $C_{DS}$\\
{\tt VBI} \hfill $V_{BI}$\\
{\tt VT0} \hfill $V_{T0}$\\
{\tt VDELTA} \hfill $V_{\Delta}$\\
{\tt VMAX} \hfill $V_{\ms{MAX}}$\\
\hline
\end{tabular}
\hfill
\parbox{0.2in}{\ \vspace*{0.2in}\newline +}
\hfill
\begin{tabular}[t]{|p{1in}|}
\hline
\multicolumn{1}{|c|}{GEOMETRY} \\
\multicolumn{1}{|c|}{PARAMETER} \\
\hline
\hspace*{\fill}$Area$\\
\hline
\end{tabular}
\hfill
\parbox{0.2in}{\ \vspace*{0.2in}\newline $\rightarrow$}
\hfill
\begin{tabular}[t]{|p{1.8in}|}
\hline
\multicolumn{1}{|c|}{DEVICE} \\
\multicolumn{1}{|c|}{PARAMETERS} \\
\hline
$C'_{DS} = f(Area, C_{DS})$\\
$C'_{GD} = f(Area, C_{GD}, \alpha,$\newline\hspace*{\fill}$ B, F_C, V_{BI}, V_{T0})$\\
$C'_{GS} = f(Area, C_{GS}, \alpha,$\newline\hspace*{\fill}$ B, F_C, V_{BI},
V_{T0})$\\
\hline
\end{tabular}
\caption{LEVEL 2 (Raytheon model) capacitance dependencies. \label{blevel2cap}}
\end{figure}
The above capacitance model does not satisfy drain-source charge
conservation.  Over a cycle, charge can be pumped from the drain to the
source.  In practice this is often not a problem when this capacitance model
is used buyt the user must be ware.  If it is a problem the user will see a
periodic reponse can not be obtained even if the excotation is periodic.
For a further discussion see References \cite{divekar:87,smith:87}.

The above capacitance model does not satisfy drani-source charge
conservation.  Over a cycle charge can be pumped from the drain to the
source.   In practice this is often not a problem when this capacitance
model is used but the user must be wary.
If it is a problem the user will see that a periodic response can not be
obtained even if the excitation is periodeic.  For a further discussion see
References \cite{divekar:87} and \cite{smith:87}.

{\noindent\bf LEVEL 3 (TOM Model)}\myline
The {\tt LEVEL} 3 model is an implementation of the TOM model (``Triquint's
Own Model) \cite{mccamant:mccormack:90}.\\

\noindent\underline{\sl {\tt LEVEL} 3 (TOM Model)
I/V Characteristics}\\[0.1in]
\index{GASFET, LEVEL 3 (Triquint) Model, I/V}
\index{GASFET, LEVEL 3 (TOM) Model, I/V}
\index{I/V Characteristics, see GASFET}
\index{GASFET, I/V Characteristics}
\index{I-V characteristics, see GASFET}
The {\tt LEVEL 3} current/voltage characteristics are evaluated after first
determining the mode (normal: $V_{DS} \ge 0$ or inverted:
$V_{DS} < 0$) and the region (cutoff,
linear or saturation) of the current
$(V_{DS}, V_{GS})$ operating point.

\noindent

\noindent{\sl Normal Mode: ($V_{DS} \ge 0$)}

\medskip

\noindent
The regions of operation are defined as follows
with \hfill $V_{GST} = V_{GS}-V_P$ \inlineeq
and\hfill$V_P = V_{T0} - \gamma V_{DS}$\inlineeq

\medskip

\hspace*{\fill}\offsetparbox{
\begin{tabular}{ll}
cutoff region:&$V_{GST}(t-\tau) \le 0$\\
linear region:&$V_{GST}(t-\tau) > 0 \mbox{ and } V_{DS} \le 3/\alpha$\\
saturation region:&$V_{GST}(t-\tau) > 0 \mbox{ and } V_{DS} > 3/\alpha$\\
\end{tabular}}\\[0.1in]
Then
\begin{eqnarray}
I_{DS} &=&  Area\, {{\textstyle I_{DS0}}/ {\left( \textstyle 1 + \delta I_{DS0} V_{DS}
           \right) }} \label{b3id} \\
I_{DS0} &=& \left\{ \begin{array}{ll}
      0  & \mbox{cutoff region} \\
      \beta (1 + \lambda V_{DS})
      V^Q_{GST}(t-\tau)
      \mbox{Ktanh}
         &\mbox{linear and saturation regions} \end{array} \right. \\ %}
\mbox{Ktanh} &=& \left\{ \begin{array}{ll}
       1 - \left(1 - V_{DS} \frac{\textstyle\alpha}{\textstyle 3}\right)^3
        & \mbox{linear region} \\
      1
         &\mbox{saturation region} \end{array} \right. %}
\end{eqnarray}
Ktanh is a taylor series approximation to the tanh function
of the {\tt LEVEL} 1 model.
\\[0.1in]
\noindent{\sl Inverted Mode: ($V_{DS} < 0)$}\\[0.2in]
In the inverted mode the MESFET I/V characteristics are evaluated as in the
normal mode (\ref{b3id}) but with the drain and source subscripts
exchanged, and $V_{GD}$ used as the controlling voltage instead of $V_{GS}$.
%
%  The following equations are correct but commented out.
%
%The regions are as follows:\\[0.1in]
%\hspace*{\fill}\offsetparbox{
%\begin{tabular}{ll}
%cutoff region:&$V_{GD}(t-\tau) < V_P$\\
%linear region:&$V_{GD}(t-\tau) > V_P \mbox{ and } -V_{DS} \le 3/\alpha$\\
%saturation region:&$V_{GD}(t-\tau) > V_P \mbox{ and } -V_{DS} > 3/\alpha$\\
%\end{tabular}}\\[0.1in]
%where
%\begin{equation}
%V_P = V_{T0} + \gamma V_{DS}
%\end{equation}
%Then
%\begin{equation}
%I_{DS} =  {{\textstyle I_{DS0}}\over{\textstyle 1 + \delta I_{DS0} V_{DS}}}
%      \label{b3idinv}
%\end{equation}
%\begin{equation}
%I_{DS0} = \left\{ \begin{array}{ll}
%      0  & \mbox{cutoff region} \\ \\
%      -Area\,\beta
%      \left[V_{GD}(t-\tau)-V_P\right]^Q
%      \mbox{Ktanh}
%         &\mbox{linear and saturation regions} \end{array} \right. %}
%\end{equation}
%where
%\begin{equation}
%\mbox{Ktanh} = \left\{ \begin{array}{ll}
%       1 - \left(1 + V_{DS} \frac{\alpha}{3}\right)^3
%        & \mbox{linear region} \\ \\
%      1
%         &\mbox{saturation region} \end{array} \right. %}
%\end{equation}
%is a taylor series approximation to the tanh function
%of the {\tt LEVEL} 1 model.\\[0.1in]

\noindent
The following description does apply to SuperSpice.

\noindent
The relationships of the parameters describing the I/V
characteristics of the {\tt LEVEL} 3 model are summarized in figure
\ref{fig:blevel3i/v}.\\[0.1in]
\begin{figure}[h]
\begin{tabular}[t]{|p{1in}|}
\hline
\multicolumn{1}{|c|}{KEYWORD} \\
\multicolumn{1}{|c|}{PARAMETERS} \\
\hline
\hline
{\tt ALPHA} \hfill $\alpha$\\
{\tt BETA} \hfill $\beta$\\
{\tt DELTA} \hfill $\delta$\\
{\tt GAMMA} \hfill $\gamma$\\
{\tt LAMBDA} \hfill $\lambda$\\
{\tt U} \hfill $\mu$\\
{\tt VTO} \hfill $V_{T0}$\\
\hline
\end{tabular}
\hfill
\parbox{0.2in}{\ \vspace*{0.2in}\newline +}
\hfill
\begin{tabular}[t]{|p{1in}|}
\hline
\multicolumn{1}{|c|}{GEOMETRY} \\
\multicolumn{1}{|c|}{PARAMETER} \\
\hline
\hspace*{\fill}$Area$\\
\hline
\end{tabular}
\hfill
\parbox{0.2in}{\ \vspace*{0.2in}\newline $\rightarrow$}
\hfill
\begin{tabular}[t]{|p{1.8in}|}
\hline
\multicolumn{1}{|c|}{DEVICE} \\
\multicolumn{1}{|c|}{PARAMETERS} \\
\hline
$I_{DS} = f(Area, \alpha, \beta, \delta, \gamma,$\newline\hspace*{\fill}
$\lambda, U V_{T0})$\\
\hline
\end{tabular}
\caption{LEVEL 3 (TOM model) I/V dependencies. \label{fig:blevel3i/v}}
\end{figure}

\noindent\underline{\sc LEVEL 3 (TOM Model) Capacitances
 \label{blevel3cap.txt}}\\[0.1in]
The drain-source capacitance
$C'_{DS} = Area\,C_{DS}$\inlineeq
The gate-source capacitance
\begin{equation}
C'_{GS} = Area\left[C_{GS}F_1F_2\left(1 - {{\textstyle V_{\ms{NEW}}}
   \over{\textstyle V_{BI}}}\right)^{\textstyle - \frac{1}{2}}
   + C_{GD}F_3\right]
\end{equation}
The gate-drain capacitance is given by
\begin{eqnarray}
C'_{GD} &=& Area\left[C_{GS}F_1F_3\left(1 - {{\textstyle V_{\ms{NEW}}}
   \over{\textstyle V_{BI}}}\right)^{\textstyle - \frac{1}{2}}
   + C_{GD}F_2\right]\\
F_1 & = & {{\textstyle 1}\over{\textstyle 2}} \left\{ 1 +
    {{\textstyle V_{\ms{EFF}}-V_P}
    \over{\textstyle\sqrt{\left( V_{\ms{EFF}}-V_{T0}\right)^2+V_{\Delta}^2}}}\right\}\\
F_2 & = & {{\textstyle 1}\over{\textstyle 2}} \left\{ 1 +
    {{\textstyle V_{GS}-V_{GD}}
    \over{\textstyle\sqrt{\left( V_{GS}-V_{GD}\right)^2+\alpha^{-2}}}}\right\}\\
F_3 & = & {{\textstyle 1}\over{\textstyle 2}} \left\{ 1 -
    {{\textstyle V_{GS}-V_{GD}}
    \over{\textstyle\sqrt{\left( V_{GS}-V_{GD}\right)^2+\alpha^{-2}}}}\right\}\\
V_{\ms{EFF}} & = & {{\textstyle 1}\over{\textstyle 2}} \left\{V_{GS}+V_{GD}+
    \sqrt{\left( V_{GS}-V_{GD}\right)^2+\alpha^{-2}}\right\}\\
V_{\ms{NEW}}&=&\left\{ \begin{array}{ll}
    A_1 & A_1 < V_{\ms{MAX}} \\
    V_{\ms{MAX}} & A_1 \ge V_{\ms{MAX}} \\
      \end{array} \right. %}
\end{eqnarray}
and\hfill$A_1 = \frac{1}{2}\left[V_{\ms{EFF}} + V_P + \sqrt{(V_{\ms{EFF}}-V_P)^2+V_{\Delta}^2}\right]$
\inlineeq

The capacitance parameter dependencies are summarized in figure
\ref{blevel3cap}.\\[0.2in]
\begin{figure}[h]
\begin{tabular}[t]{|p{1in}|}
\hline
\multicolumn{1}{|c|}{KEYWORD} \\
\multicolumn{1}{|c|}{PARAMETERS} \\
\hline
\hline
{\tt ALPHA} \hfill $\alpha$\\
{\tt CGD} \hfill $C_{GD}$\\
{\tt CGS} \hfill $C_{GS}$\\
{\tt CDS} \hfill $C_{DS}$\\
{\tt VBI} \hfill $V_{BI}$\\
{\tt VT0} \hfill $V_{T0}$\\
{\tt LAMBDA} \hfill $V_{T0}$\\
{\tt VMAX} \hfill $V_{\ms{MAX}}$\\
{\tt VDELTA} \hfill $V_{\Delta}$\\
\hline
\end{tabular}
\hfill
\parbox{0.2in}{\ \vspace*{0.2in}\newline +}
\hfill
\begin{tabular}[t]{|p{1in}|}
\hline
\multicolumn{1}{|c|}{GEOMETRY} \\
\multicolumn{1}{|c|}{PARAMETER} \\
\hline
\hspace*{\fill}$Area$\\
\hline
\end{tabular}
\hfill
\parbox{0.2in}{\ \vspace*{0.2in}\newline $\rightarrow$}
\hfill
\begin{tabular}[t]{|p{1.8in}|}
\hline
\multicolumn{1}{|c|}{DEVICE} \\
\multicolumn{1}{|c|}{PARAMETERS} \\
\hline
$C'_{DS} = f(Area, C_{DS})$\\
$C'_{GD} = f(Area, C_{GD},$\newline\hspace*{\fill}$
\alpha, B, F_C, V_{BI}, V_{T0})$\\
$C'_{GS} = f(Area, C_{GS},$\newline\hspace*{\fill}$
\alpha, B, F_C, V_{BI}, V_{T0})$\\
\hline
\end{tabular}
\caption{LEVEL 3 (TOM model) capacitance dependencies. \label{blevel3cap}}
\end{figure}
\label{b:level-1:start}
{\noindent\bf LEVEL -1 (TOM-2 Model)}\myline
The {\tt LEVEL} -1 model is an enhancement of the {\tt LEVEL 3} TOM model.

\medskip

\noindent\underline{\sl {\tt LEVEL} -1 (TOM Model)
I/V Characteristics}\\[0.1in]
\index{GASFET, LEVEL -1 (TOM-2) Model, I/V}
\index{I/V Characteristics, see GASFET}
\index{GASFET, I/V Characteristics}
\index{I-V characteristics, see GASFET}
The {\tt LEVEL -1} current/voltage characteristics are evaluated after first
determining the mode (normal: $V_{DS} \ge 0$ or inverted:
$V_{DS} < 0$) and the region (cutoff,
linear or saturation) of the current
$(V_{DS}, V_{GS})$ operating point.

\noindent

\noindent{\sl Normal Mode: ($V_{DS} \ge 0$)}

\medskip

\noindent
The regions of operation are defined as follows
with \hfill $V_{GST} = V_{GS}-V_P$ \inlineeq
and\hfill$V_P = V_{T0} - \gamma V_{DS}$\inlineeq

\medskip

\hspace*{\fill}\offsetparbox{
\begin{tabular}{ll}
cutoff region:&$V_{GST}(t-\tau) \le 0$\\
linear region:&$V_{GST}(t-\tau) > 0 \mbox{ and } V_{DS} \le 3/\alpha$\\
saturation region:&$V_{GST}(t-\tau) > 0 \mbox{ and } V_{DS} > 3/\alpha$\\
\end{tabular}}\\[0.1in]
Then
\begin{eqnarray}
I_{DS} &=&  Area\, {{\textstyle I_{DS0}}/ {\left( \textstyle 1 + \delta I_{DS0} V_{DS}
           \right) }} \label{b-1id} \\
I_{DS0} &=& \left\{ \begin{array}{ll}
      0  & \mbox{cutoff region} \\
      {{\textstyle\beta}\over{\textstyle 1 + U V_{GST}}}
      V^Q_{GST}(t-\tau) F_d(\alpha V_{DS})
         &\mbox{linear and saturation regions} \end{array} \right. \\ %}
F_d(\alpha V_{DS})&=&{{ \textstyle x } \over { \sqrt{1 + x^2}}} \\
   V_G&=&Q V_{ST} \mbox{ln}
     \left[ exp \left( V_{GST}/(Q V_{ST} \right) + 1 \right]\\
   V_{ST}&=&N_{ST} (kT/q) \\
   N_{ST}&=&N_G + N_D V_{DS}
\end{eqnarray}
\\[0.1in]
\noindent{\sl Inverted Mode: ($V_{DS} < 0)$}\\[0.2in]
In the inverted mode the MESFET I/V characteristics are evaluated as in the
normal mode (\ref{b-1id}) but with the drain and source subscripts
exchanged, and $V_{GD}$ used as the controlling voltage instead of $V_{GS}$.

The relationships of the parameters describing the I/V
characteristics of the {\tt LEVEL} -1 model are summarized in figure
\ref{blevel-1i/v}.\\[0.1in]
\begin{figure}[h]
\begin{tabular}[t]{|p{1in}|}
\hline
\multicolumn{1}{|c|}{KEYWORD} \\
\multicolumn{1}{|c|}{PARAMETERS} \\
\hline
\hline
{\tt ALPHA} \hfill $\alpha$\\
{\tt BETA} \hfill $\beta$\\
{\tt DELTA} \hfill $\delta$\\
{\tt GAMMA} \hfill $\gamma$\\
{\tt LAMBDA} \hfill $\lambda$\\
{\tt U} \hfill $\mu$\\
{\tt VTO} \hfill $V_{T0}$\\
\hline
\end{tabular}
\hfill
\parbox{0.2in}{\ \vspace*{0.2in}\newline +}
\hfill
\begin{tabular}[t]{|p{1in}|}
\hline
\multicolumn{1}{|c|}{GEOMETRY} \\
\multicolumn{1}{|c|}{PARAMETER} \\
\hline
\hspace*{\fill}$Area$\\
\hline
\end{tabular}
\hfill
\parbox{0.2in}{\ \vspace*{0.2in}\newline $\rightarrow$}
\hfill
\begin{tabular}[t]{|p{1.8in}|}
\hline
\multicolumn{1}{|c|}{DEVICE} \\
\multicolumn{1}{|c|}{PARAMETERS} \\
\hline
$I_{DS} = f(Area, \alpha, \beta, \delta, \gamma,$\newline\hspace*{\fill}
$\lambda, U V_{T0})$\\
\hline
\end{tabular}
\caption{LEVEL -1 (TOM-2 model) I/V dependencies. \label{blevel-1i/v}}
\end{figure}
\noindent\underline{\sc LEVEL -1 (TOM-2 Model) Capacitances}\\[0.1in]
These are evaluated the same way the {\tt LEVEL} 3 capacitanceaas
described on \pageref{blevel3cap.txt} are evaluated.

{\noindent\bf LEVEL -1 (TOM2 Model)}\myline
The {\tt LEVEL} -1 model is an implementation of the TOM model (``Triquint's
Own Model) \cite{mccamant:mccormack:90}.

\noindent\underline{\sl {\tt LEVEL} -1 (TOM2 Model)
I/V Characteristics}\\[0.1in]
\index{GASFET, LEVEL -1 (Triquint) Model, I/V}
\index{GASFET, LEVEL -1 (TOM) Model, I/V}
\index{I/V Characteristics, see GASFET}
\index{GASFET, I/V Characteristics}
\index{I-V characteristics, see GASFET}
The {\tt LEVEL -1} current/voltage characteristics are evaluated after first
determining the mode (normal: $V_{DS} \ge 0$ or inverted:
$V_{DS} < 0$) and the region (cutoff,
linear or saturation) of the current
$(V_{DS}, V_{GS})$ operating point.

\noindent

\noindent{\sl Normal Mode: ($V_{DS} \ge 0$)}

\medskip

\noindent
The regions of operation are defined as follows
with \hfill $V_{GST} = V_{GS}-V_P$ \inlineeq
and\hfill$V_P = V_{T0} - \gamma V_{DS}$\inlineeq

\medskip

\hspace*{\fill}\offsetparbox{
\begin{tabular}{ll}
cutoff region:&$V_{GST}(t-\tau) \le 0$\\
linear region:&$V_{GST}(t-\tau) > 0 \mbox{ and } V_{DS} \le 3/\alpha$\\
saturation region:&$V_{GST}(t-\tau) > 0 \mbox{ and } V_{DS} > 3/\alpha$\\
\end{tabular}}\\[0.1in]
Then
\begin{eqnarray}
I_{DS} &=&  Area\, {{\textstyle I_{DS0}}/ {\left( \textstyle 1 + \delta I_{DS0} V_{DS}
           \right) }} \label{b3id} \\
I_{DS0} &=& \left\{ \begin{array}{ll}
      0  & \mbox{cutoff region} \\
      V^Q_G(t-\tau)F_d(\alpha V_{DS})
         &\mbox{linear and saturation regions} \end{array} \right. \\ %}
F_d(x) = {{\textstyle x} \over {\textstyle \sqrt{1+x^2}}} \\
V_G = Q V_{ST} \mbox{ln} \left[ \mbox{exp} \left(
          {{\textstyle V_{GST}}\over{\textstyle Q V_{ST}}} \right) +1) \right]
      \\
V_{ST} = N_{ST} \left( {{\textstyle k T} \over { \textstyle q}} \right) \\
N_{ST} = N_G + N_D V_{DS}
\end{eqnarray}
\\[0.1in]
\noindent{\sl Inverted Mode: ($V_{DS} < 0)$}\\[0.2in]
In the inverted mode the MESFET I/V characteristics are evaluated as in the
normal mode (\ref{b3id}) but with the drain and source subscripts
exchanged, and $V_{GD}$ used as the controlling voltage instead of $V_{GS}$.

\noindent
The relationships of the parameters describing the I/V
characteristics of the {\tt LEVEL} -1 model are summarized in figure
\ref{fig:blevel3i/v}.\\[0.1in]
\begin{figure}[h]
\begin{tabular}[t]{|p{1in}|}
\hline
\multicolumn{1}{|c|}{KEYWORD} \\
\multicolumn{1}{|c|}{PARAMETERS} \\
\hline
\hline
{\tt ALPHA} \hfill $\alpha$\\
{\tt BETA} \hfill $\beta$\\
{\tt DELTA} \hfill $\delta$\\
{\tt GAMMA} \hfill $\gamma$\\
{\tt LAMBDA} \hfill $\lambda$\\
{\tt U} \hfill $\mu$\\
{\tt VTO} \hfill $V_{T0}$\\
\hline
\end{tabular}
\hfill
\parbox{0.2in}{\ \vspace*{0.2in}\newline +}
\hfill
\begin{tabular}[t]{|p{1in}|}
\hline
\multicolumn{1}{|c|}{GEOMETRY} \\
\multicolumn{1}{|c|}{PARAMETER} \\
\hline
\hspace*{\fill}$Area$\\
\hline
\end{tabular}
\hfill
\parbox{0.2in}{\ \vspace*{0.2in}\newline $\rightarrow$}
\hfill
\begin{tabular}[t]{|p{1.8in}|}
\hline
\multicolumn{1}{|c|}{DEVICE} \\
\multicolumn{1}{|c|}{PARAMETERS} \\
\hline
$I_{DS} = f(Area, \alpha, \beta, \delta, \gamma,$\newline\hspace*{\fill}
$\lambda, U V_{T0})$\\
\hline
\end{tabular}
\caption{LEVEL 3 (TOM model) I/V dependencies. \label{fig:blevel3i/v}}
\end{figure}

\noindent\underline{\sc LEVEL -1 (TOM2 Model) Capacitances
 \label{blevel3cap.txt}}\\[0.1in]
 The {\sc LEVEL -1 } (TOM2 Model) capacitances are identical to that of the
 LEVEL 3 model. This capacitance model is based on the model proposed by
 Statz as used in the level 2 model.  This is a charge conserving
 symmetrical capacitance model.

The drain-source capacitance
$C'_{DS} = Area\,C_{DS}$\inlineeq
The gate-source capacitance
\begin{equation}
C'_{GS} = Area\left[C_{GS}F_1F_2\left(1 - {{\textstyle V_{\ms{NEW}}}
   \over{\textstyle V_{BI}}}\right)^{\textstyle - \frac{1}{2}}
   + C_{GD}F_3\right]
\end{equation}
The gate-drain capacitance is given by
\begin{eqnarray}
C'_{GD} &=& Area\left[C_{GS}F_1F_3\left(1 - {{\textstyle V_{\ms{NEW}}}
   \over{\textstyle V_{BI}}}\right)^{\textstyle - \frac{1}{2}}
   + C_{GD}F_2\right]\\
F_1 & = & {{\textstyle 1}\over{\textstyle 2}} \left\{ 1 +
    {{\textstyle V_{\ms{EFF}}-V_P}
    \over{\textstyle\sqrt{\left( V_{\ms{EFF}}-V_{T0}\right)^2+V_{\Delta}^2}}}\right\}\\
F_2 & = & {{\textstyle 1}\over{\textstyle 2}} \left\{ 1 +
    {{\textstyle V_{GS}-V_{GD}}
    \over{\textstyle\sqrt{\left( V_{GS}-V_{GD}\right)^2+\alpha^{-2}}}}\right\}\\
F_3 & = & {{\textstyle 1}\over{\textstyle 2}} \left\{ 1 -
    {{\textstyle V_{GS}-V_{GD}}
    \over{\textstyle\sqrt{\left( V_{GS}-V_{GD}\right)^2+\alpha^{-2}}}}\right\}\\
V_{\ms{EFF}} & = & {{\textstyle 1}\over{\textstyle 2}} \left\{V_{GS}+V_{GD}+
    \sqrt{\left( V_{GS}-V_{GD}\right)^2+\alpha^{-2}}\right\}\\
V_{\ms{NEW}}&=&\left\{ \begin{array}{ll}
    A_1 & A_1 < V_{\ms{MAX}} \\
    V_{\ms{MAX}} & A_1 \ge V_{\ms{MAX}} \\
      \end{array} \right. %}
\end{eqnarray}
and\hfill$A_1 = \frac{1}{2}\left[V_{\ms{EFF}} + V_P + \sqrt{(V_{\ms{EFF}}-V_P)^2+V_{\Delta}^2}\right]$
\inlineeq

The capacitance parameter dependencies are summarized in figure
\ref{blevel3cap}.\\[0.2in]
\begin{figure}[h]
\begin{tabular}[t]{|p{1in}|}
\hline
\multicolumn{1}{|c|}{KEYWORD} \\
\multicolumn{1}{|c|}{PARAMETERS} \\
\hline
\hline
{\tt ALPHA} \hfill $\alpha$\\
{\tt CGD} \hfill $C_{GD}$\\
{\tt CGS} \hfill $C_{GS}$\\
{\tt CDS} \hfill $C_{DS}$\\
{\tt VBI} \hfill $V_{BI}$\\
{\tt VT0} \hfill $V_{T0}$\\
{\tt LAMBDA} \hfill $V_{T0}$\\
{\tt VMAX} \hfill $V_{\ms{MAX}}$\\
{\tt VDELTA} \hfill $V_{\Delta}$\\
\hline
\end{tabular}
\hfill
\parbox{0.2in}{\ \vspace*{0.2in}\newline +}
\hfill
\begin{tabular}[t]{|p{1in}|}
\hline
\multicolumn{1}{|c|}{GEOMETRY} \\
\multicolumn{1}{|c|}{PARAMETER} \\
\hline
\hspace*{\fill}$Area$\\
\hline
\end{tabular}
\hfill
\parbox{0.2in}{\ \vspace*{0.2in}\newline $\rightarrow$}
\hfill
\begin{tabular}[t]{|p{1.8in}|}
\hline
\multicolumn{1}{|c|}{DEVICE} \\
\multicolumn{1}{|c|}{PARAMETERS} \\
\hline
$C'_{DS} = f(Area, C_{DS})$\\
$C'_{GD} = f(Area, C_{GD},$\newline\hspace*{\fill}$
\alpha, B, F_C, V_{BI}, V_{T0})$\\
$C'_{GS} = f(Area, C_{GS},$\newline\hspace*{\fill}$
\alpha, B, F_C, V_{BI}, V_{T0})$\\
\hline
\end{tabular}
\caption{LEVEL 3 (TOM model) capacitance dependencies. \label{blevel3cap}}
\end{figure}
\label{b:level-1:start}
{\noindent\bf LEVEL -1 (TOM-2 Model)}\myline
The {\tt LEVEL} -1 model is an enhancement of the {\tt LEVEL 3} TOM model.

\medskip

\noindent\underline{\sl {\tt LEVEL} -1 (TOM Model)
I/V Characteristics}\\[0.1in]
\index{GASFET, LEVEL -1 (TOM-2) Model, I/V}
\index{I/V Characteristics, see GASFET}
\index{GASFET, I/V Characteristics}
\index{I-V characteristics, see GASFET}
The {\tt LEVEL -1} current/voltage characteristics are evaluated after first
determining the mode (normal: $V_{DS} \ge 0$ or inverted:
$V_{DS} < 0$) and the region (cutoff,
linear or saturation) of the current
$(V_{DS}, V_{GS})$ operating point.

\noindent

\noindent{\sl Normal Mode: ($V_{DS} \ge 0$)}

\medskip

\noindent
The regions of operation are defined as follows
with \hfill $V_{GST} = V_{GS}-V_P$ \inlineeq
and\hfill$V_P = V_{T0} - \gamma V_{DS}$\inlineeq

\medskip

\hspace*{\fill}\offsetparbox{
\begin{tabular}{ll}
cutoff region:&$V_{GST}(t-\tau) \le 0$\\
linear region:&$V_{GST}(t-\tau) > 0 \mbox{ and } V_{DS} \le 3/\alpha$\\
saturation region:&$V_{GST}(t-\tau) > 0 \mbox{ and } V_{DS} > 3/\alpha$\\
\end{tabular}}\\[0.1in]
Then
\begin{eqnarray}
I_{DS} &=&  Area\, {{\textstyle I_{DS0}}/ {\left( \textstyle 1 + \delta I_{DS0} V_{DS}
           \right) }} \label{b-1id} \\
I_{DS0} &=& \left\{ \begin{array}{ll}
      0  & \mbox{cutoff region} \\
      {{\textstyle\beta}\over{\textstyle 1 + U V_{GST}}}
      V^Q_{GST}(t-\tau) F_d(\alpha V_{DS})
         &\mbox{linear and saturation regions} \end{array} \right. \\ %}
F_d(\alpha V_{DS})&=&{{ \textstyle x } \over { \sqrt{1 + x^2}}} \\
   V_G&=&Q V_{ST} \mbox{ln}
     \left[ exp \left( V_{GST}/(Q V_{ST} \right) + 1 \right]\\
   V_{ST}&=&N_{ST} (kT/q) \\
   N_{ST}&=&N_G + N_D V_{DS}
\end{eqnarray}
\\[0.1in]
\noindent{\sl Inverted Mode: ($V_{DS} < 0)$}\\[0.2in]
In the inverted mode the MESFET I/V characteristics are evaluated as in the
normal mode (\ref{b-1id}) but with the drain and source subscripts
exchanged, and $V_{GD}$ used as the controlling voltage instead of $V_{GS}$.

The relationships of the parameters describing the I/V
characteristics of the {\tt LEVEL} -1 model are summarized in figure
\ref{blevel-1i/v}.\\[0.1in]
\begin{figure}[h]
\begin{tabular}[t]{|p{1in}|}
\hline
\multicolumn{1}{|c|}{KEYWORD} \\
\multicolumn{1}{|c|}{PARAMETERS} \\
\hline
\hline
{\tt ALPHA} \hfill $\alpha$\\
{\tt BETA} \hfill $\beta$\\
{\tt DELTA} \hfill $\delta$\\
{\tt GAMMA} \hfill $\gamma$\\
{\tt LAMBDA} \hfill $\lambda$\\
{\tt U} \hfill $\mu$\\
{\tt VTO} \hfill $V_{T0}$\\
\hline
\end{tabular}
\hfill
\parbox{0.2in}{\ \vspace*{0.2in}\newline +}
\hfill
\begin{tabular}[t]{|p{1in}|}
\hline
\multicolumn{1}{|c|}{GEOMETRY} \\
\multicolumn{1}{|c|}{PARAMETER} \\
\hline
\hspace*{\fill}$Area$\\
\hline
\end{tabular}
\hfill
\parbox{0.2in}{\ \vspace*{0.2in}\newline $\rightarrow$}
\hfill
\begin{tabular}[t]{|p{1.8in}|}
\hline
\multicolumn{1}{|c|}{DEVICE} \\
\multicolumn{1}{|c|}{PARAMETERS} \\
\hline
$I_{DS} = f(Area, \alpha, \beta, \delta, \gamma,$\newline\hspace*{\fill}
$\lambda, U V_{T0})$\\
\hline
\end{tabular}
\caption{LEVEL -1 (TOM-2 model) I/V dependencies. \label{blevel-1i/v}}
\end{figure}
\noindent\underline{\sc LEVEL -1 (TOM-2 Model) Capacitances}\\[0.1in]
These are evaluated the same way the {\tt LEVEL} 3 capacitanceaas
described on \pageref{blevel3cap.txt} are evaluated.

\noindent{\bf LEVEL 4 (Curtice Cubic Model)}\myline
\noindent\underline{\sl {\tt LEVEL} 4 (Curtice Cubic Model)
I/V Characteristics}\\[0.1in]
 \index{GASFET,
LEVEL 4 (Curtice) Model, I/V} \index{GASFET, LEVEL 4 (Curtice)
Model, I/V} \index{I/V Characteristics, see GASFET} \index{GASFET,
I/V Characteristics} \index{I-V characteristics, see GASFET} The
{\tt LEVEL 4} current/voltage characteristics are evaluated after
first determining the mode (normal: $V_{DS} \ge 0$ or inverted:
$V_{DS} < 0$) and the region (cutoff,linear or saturation) of the
current $(V_{DS}, V_{GS})$ operating point. Curtice
\cite{curtice:80} proposed two \dc\ current models: a quadratic
channel current model and a cubic channel current model.  The
quadratic channel model is implemented as the {\tt LEVEL 4} model.

\medskip

\noindent{\sl Normal Mode: ($V_{DS} \ge 0$)}

\medskip

\noindent
The regions of operation are defined as follows with
with \hfill $V_{GST} = V_{GS}-V_{T0}$ \inlineeq
\hspace*{\fill}\offsetparbox{
\begin{tabular}{ll}
cutoff region:&$V_{GS}(t-\tau) < V_{T0}$\\
linear and saturation regions:&$V_{GS}(t-\tau) > V_{T0}$\\
\end{tabular}}\\[0.1in]
Then thedrain source current is given by
\begin{eqnarray}
I_{DS} &=& \left\{ \begin{array}{ll}
      0  & \mbox{cutoff region} \\
      Area \left( A_0 + A_1 V_x + A_2 V_x^2 + A_3 V_x^3 \right)
      \mbox{tanh}\left(\gamma V_{DS}\right)
         &\mbox{linear and saturation}\\
         &\mbox{regions} \end{array} \right. %}
      \label{b4id} \\
      V_x &=& V_{GS}(t-\tau)\left[ 1 + \beta (V_{DS0} - V_{DS}) \right]
\end{eqnarray}
\noindent{\sl Inverted Mode: ($V_{DS} < 0)$}\\[0.2in]
In the inverted mode the MESFET I/V characteristics are evaluated as in the
normal mode (\ref{b4id}) but with the drain and source subscripts
exchanged, and $V_{GD}$ is the controlling voltage instead of $V_{GS}$.

The relationships of the parameters describing the I/V
characteristics for the {\tt LEVEL} 4 model are summarized in figure
\ref{blevel4i/v}.\\[0.1in]
\begin{figure}[h]
\begin{tabular}[t]{|p{1in}|}
\hline
\multicolumn{1}{|c|}{KEYWORD} \\
\multicolumn{1}{|c|}{PARAMETERS} \\
\hline
\hline
{\tt A0} \hfill $A_0$\\
{\tt A1} \hfill $A_1$\\
{\tt A2} \hfill $A_2$\\
{\tt A3} \hfill $A_3$\\
{\tt BETA} \hfill $\beta$\\
{\tt GAMMA} \hfill $\gamma$\\
{\tt T, TAU} \hfill $\tau$\\
{\tt VTO} \hfill $V_{T0}$\\
\hline
\end{tabular}
\hfill
\parbox{0.2in}{\ \vspace*{0.2in}\newline +}
\hfill
\begin{tabular}[t]{|p{1in}|}
\hline
\multicolumn{1}{|c|}{GEOMETRY} \\
\multicolumn{1}{|c|}{PARAMETER} \\
\hline
\hspace*{\fill}$Area$\\
\hline
\end{tabular}
\hfill
\parbox{0.2in}{\ \vspace*{0.2in}\newline $\rightarrow$}
\hfill
\begin{tabular}[t]{|p{1.8in}|}
\hline
\multicolumn{1}{|c|}{DEVICE} \\
\multicolumn{1}{|c|}{PARAMETERS} \\
$I_{DS}=f(Area,\newline\hspace*{\fill}A_0,A_1,A_2,A_3,\beta,\gamma,V_{T0})$\\
\hline
\end{tabular}
\caption{LEVEL 4 (Curtice cubic model) I/V dependencies. \label{blevel4i/v}}
\end{figure}

\noindent\underline{\sl LEVEL 4 (Curtice Cubic Model) Capacitances}\\[0.1in]
The drain-source capacitance
$C'_{DS} = Area\,C_{DS}$\inlineeq
the gate-source and gate-drain capacitances are
\begin{eqnarray}
C'_{GS} &=& \left\{ \begin{array}{ll}
         Area\,C_{GS}\left(1 - {{\textstyle V_{GS}}\over{\textstyle V_{BI}}}
         \right)^{\textstyle -M_{GS}}
         & V_{GS} \le F_C V_{BI}\\
         Area\,C_{GS}\left(1 -F_C\right)^{\textstyle -(1+M_{GS})}
         \left[1-F_C(1+M_{GS})+M_{GS}
          {{\textstyle V_{GS}}\over{\textstyle V_{BI}}} \right]
         & V_{GS} > F_C V_{BI}
         \end{array} \right. \\ %}
C_{GD} &=& \left\{ \begin{array}{ll}
         Area\,C_{GD}\left(1 - {{\textstyle V_{GD}}\over{\textstyle V_{BI}}}
         \right)^{\textstyle -M_{GD}}
         & V_{GD} \le F_C V_{BI}\\
         Area\,C_{GD}\left(1 -F_C\right)^{\textstyle -(1+M_{GD})}
         \left[1-F_C(1+M_{GD})+M_{GD}
            {{\textstyle V_{GD}}\over{\textstyle V_{BI}}}
         \right]
         & V_{GS} > F_C V_{BI}
         \end{array} \right.  %}
\end{eqnarray}

The {\tt LEVEL} 4 capacitance parameter dependencies are summarized in figure
\ref{blevel4cap}.\\[0.2in]
\begin{figure}[h]
\begin{tabular}[t]{|p{1in}|}
\hline
\multicolumn{1}{|c|}{KEYWORD} \\
\multicolumn{1}{|c|}{PARAMETERS} \\
\hline
\hline
{\tt CGD} \hfill $C_{GD}$\\
{\tt CGS} \hfill $C_{GS}$\\
{\tt CDS} \hfill $C_{DS}$\\
{\tt FC} \hfill $F_C$\\
{\tt VBI} \hfill $V_{BI}$\\
{\tt MGS} \hfill $M_{GS}$\\
{\tt MGD} \hfill $M_{GD}$\\
\hline
\end{tabular}
\hfill
\parbox{0.2in}{\ \vspace*{0.2in}\newline +}
\hfill
\begin{tabular}[t]{|p{1in}|}
\hline
\multicolumn{1}{|c|}{GEOMETRY} \\
\multicolumn{1}{|c|}{PARAMETER} \\
\hline
\hspace*{\fill}$Area$\\
\hline
\end{tabular}
\hfill
\parbox{0.2in}{\ \vspace*{0.2in}\newline $\rightarrow$}
\hfill
\begin{tabular}[t]{|p{1.8in}|}
\hline
\multicolumn{1}{|c|}{DEVICE} \\
\multicolumn{1}{|c|}{PARAMETERS} \\
\hline
$C'_{DS} = f(Area, C_{DS})$\\
$C'_{GD} =$\newline\hspace*{\fill}$ f(Area, C_{GD},F_C,V_{BI},M_{GD})$\\
$C'_{GS} =$\newline\hspace*{\fill}$ f(Area, C_{GS},F_C,V_{BI},M_{GS})$\\
\hline
\end{tabular}
\caption{LEVEL 4 (Curtice Cubic model) capacitance dependencies.
 \label{blevel4cap}}
\end{figure}

\ \myThickLine \label{b:level5:start}
{\bf LEVEL 5
(Materka-Kacprzak Model)}\myline

The parameter keywords of the Materka-Kacprzak model are given in table
\ref{btable5}.



\begin{longtable}[h]{|p{0.6in}|p{3.5in}|p{0.6in}|p{0.6in}|}
\caption[GASFET level 5 (Materka-Kacprzak) model keywords]{GASFET
 level 5 (Materka-Kacprzak) model keywords. \sspice\ only.
Parameters that are {\sc NOT USED} are reserved for future
expansion.
         \label{btable5}}\\

\hline
\multicolumn{1}{|c}{\bf Name} &
\multicolumn{1}{|c}{\parbox{2.77in}{\bf Description}}  &
\multicolumn{1}{|c}{\bf Units} &
\multicolumn{1}{|c|}{\bf Default}\\ \hline
\endhead

\hline \multicolumn{4}{|r|}{{Continued on next page}} \\ \hline
\endfoot

\hline \hline
\endlastfoot

{\tt AC10} & temperature coefficient of {\tt C10}
       \sym{A_{C10}}&$^{\circ}\mbox{C}^{-1}$& 0\X
{\tt ACF0} & temperature coefficient of {\tt CF0}
       \sym{A_{CF0}}&$^{\circ}\mbox{C}^{-1}$& 0\X
{\tt AF}& flicker noise exponent\sym{A_F}& -    & 1 \X
{\tt AE} & temperature coefficient of {\tt E}
       \sym{A_{E}}&V$^{-1}$& 0\X
{\tt AFAB} & slope factor of breakdown current ($\ge$ 0)
       \sym{A_{FAB}}&$^{\circ}\mbox{C}^{-1}$& 0\X
{\tt AFAG} & slope factor of gate conduction current
       \sym{A_{FAG}}&V$^{-1}$& 38.696\X
{\tt AGAM} & temperature coefficient of {\tt GAMA}
       \sym{A_{\gamma}}&$^{\circ}\mbox{C}^{-1}$& 0\X
{\tt AIDS}& linear temperature coefficient of {\tt IDSS}
           \sym{A_{\lambda}}& \%/$^{\circ}$C& 0 \X
{\tt AKE}    & temperature coefficient of {\tt KE}
       \sym{A_{KE}}&$^{\circ}\mbox{C}^{-1}$& 0\X
{\tt AKG}    & temperature coefficient of {\tt KG}
       \sym{A_{KG}}&$^{\circ}\mbox{C}^{-1}$& 0\X
{\tt ARD}    & alternative keyword for {\tt TRD1}
       \sym{A_{RD}}&$^{\circ}\mbox{C}^{-1}$& 0\X
{\tt ARG}    & alternative keyword for {\tt TRG1}
       \sym{A_{RG}}&$^{\circ}\mbox{C}^{-1}$& 0\X
{\tt ARS}    & alternative keyword for {\tt TRS1}
       \sym{A_{RS}}&$^{\circ}\mbox{C}^{-1}$& 0\X
{\tt AR10}  & temperature coefficient of {\tt R10}
       \sym{A_{R10}}&$^{\circ}\mbox{C}^{-1}$& 0\X
{\tt ASL}   & temperature coefficient of {\tt SL}
       \sym{A_{SL}}&$^{\circ}\mbox{C}^{-1}$& 0\X
{\tt ASS}   & temperature coefficient of {\tt SS}
       \sym{A_{SS}}&$^{\circ}\mbox{C}^{-1}$& 0\X
{\tt AT}    & temperature coefficient of {\tt T}
       \sym{A_{T}}&$^{\circ}\mbox{C}^{-1}$& 0\X
{\tt AVBC} & linear temperature coefficient of {\tt VDC}
             \sym{A_{VBC)}} &$^{\circ}\mbox{C}^{-1}$ & 0\X
{\tt AVP0} & linear temperature coefficient of {\tt VP0}
          \sym{A_{VP0)}} &$^{\circ}\mbox{C}^{-1}$ & 0\X
{\tt C10}& gate-source Schottky barrier capacitance for\newline $V_{GS}$ = 0.
        \sym{C_{10}}& F & 0 \X
{\tt CF0}& gate-drain feedback capacitacne for $V_{GD}$ = 0.
        \sym{C_{F0}}& F & 0 \X
{\tt C1S}& constant parasitic component of gate-source capacitance
        \sym{C_{1S}}& F & 0 \X
{\tt CDE}& drain-source electrode capacitance
        {\sc not used} \sym{C_{DE}}& F  & 0 \X
{\tt CDGE}& drain-gate electrode capacitance
        {\sc not used} \sym{C_{DGE}}& F & 0 \X
{\tt CDS}& drain-source capacitance
        \sym{C_{DS}}& F & 0 \X
{\tt CDSD}& low frequency trapping capacitance ({\sc not used})
        \newline\sym{C_{DSD}}& F    & 0 \X
{\tt CGE}& gate-source electrode capacitance
        {\sc not used} \sym{C_{DE}}& F  & 0 \X
{\tt CGS0}& zero-bias gate-source p-n capacitance in Raytheon capacitance
           model. Used if {\tt CLVL} = 2 \sym{C_{GS}}& F    & 0 \X
{\tt CGD0}& zero-bias gate-drain p-n capacitance in Raytheon capacitance
           model. Used if {\tt CLVL} = 2 \sym{C_{GS}}& F    & 0 \X
{\tt CLVL}& capacitance model flag
           \newline {\tt CLVLV = 1} use Materka-Kacprzak capacitance model
           \newline {\tt CLVLV = 2} use Raytheon capacitance model
           \sym{C_{LVL}}& - & 1 \X
{\tt E}& constant part of drain current power \sym{E}& -    & 2\X
{\tt FCC} & \sym{F_{CC}}& - & 0.8   \X
{\tt GAMA}  & Voltage slope parameter of pinch-off voltage \sym{\gamma}
            &$\mbox{V}^{-1}$& 0\X
{\tt IB0}& current parameter of gate-drain breakdown
            \newline  source ($\ge$ 0)\sym{I_{B0}}&A&0\X
{\tt IDSS}& frain saturation current for $V_{GS}$ = 0\sym{I_{DSS}}&A    & 0.1 \X
{\tt IG0}& saturation current of gate-source schottky\newline
            barrier\sym{I_{G0}}&A&0\X
{\tt K1}& slope parameter of gate-source capacitance\sym{K_1}
            &$\mbox{V}^{-1}$& 1.25\X
{\tt KE}& dependence of drain current  power on $V_{GS}$\sym{K_E}
            &$\mbox{V}^{-1}$& 0\X
{\tt KF}& slope parameter of gate-drain feedback capacitance\sym{K_F}& -    & 0 \X
{\tt KFL}& flicker noise coefficient\sym{K_{FL}}& - & 0 \X
{\tt KG}& linearregion drain current $V_{GS}$ dependence \sym{K_G}
            &$\mbox{V}^{-1}$& 0\X
{\tt KR}& slope factor of intrinsic channel resistance\sym{K_R}
            &$\mbox{V}^{-1}$& 0\X
{\tt LEVEL}& model index \hfill must be 5 & -  &  1 \X
{\tt M} & gate p-n grading coefficient\sym{M}& -    & 0.5   \X
{\tt MGS}& gate-source p-n grading coefficient \sym{M_{GS}}& - & M  \X
{\tt MGD}& gate-drain p-n grading coefficient \sym{M_{GS}}& - & M   \X
{\tt R10}   & intrinsic channel resistance for $V_{GS}$ = 0.
          \sym{R_{10}}&$\Omega$& $\infty$   \X
{\tt RD}    & drain resistance \sym{R_D}&$\Omega$& 0    \X
{\tt RDSD}  & channel trapping resistance {\sc not used} \sym{R_{DSD}}
          &$\Omega$& $\infty$   \X
{\tt RG}& gate resistance \sym{R_G}&$\Omega$& 0 \X
{\tt RS}    & source resistance \sym{R_S}&$\Omega$& 0\X
{\tt SL}    & slope of $V_{GS}$ = 0 drain
                  current, linear region \sym{S_L}&S& 0.15\X
{\tt SS}    & slope of $V_{GS}$ = 0 drain
                  current, saturated region \sym{S_S}&S& 0\X
{\tt T} & channel transit time delay \sym{\tau}& s  &  0    \X
{\tt TJ}    & junction temperature \version{\sspice} \sym{T_J} &K& 298 \X
{\tt TNOM}  & model reference temperature ($> 0$) \version{\sspice}
          \sym{T_{NOM}} &K& 298 \X
{\tt TRG1} & temperature coefficient of {\tt RG}\sym{A_{RG}}
           & $^{\circ}\mbox{C}^{-1}$ & 0    \X
{\tt TRD1} & temperature coefficient of {\tt RG}\sym{A_{RD}}
           & $^{\circ}\mbox{C}^{-1}$ & 0    \X
{\tt TRS1} & temperature coefficient of {\tt RG}\sym{A_{RS}}
           & $^{\circ}\mbox{C}^{-1}$ & 0    \X
{\tt VBC}  & breakdown voltage ($\ge 0$ \sym{V_{BC}} & V & $\infty$\X
{\tt VBI}   & gate p-n potential in Raytheon capacitance
           model. Used if {\tt CLVL} = 2.  \sym{V_{\ms{BI}}}& V & 1 \X
{\tt VP0} &(VP-zero) pinch-off voltage for $V_{DS}$ = 0 \sym{V_{P0}}&V &-2.5\X
\end{longtable}

\noindent\underline{\sl \large Temperature Dependence}
\index{GASFET, Temperature Dependence}
\index{Temperature Dependence, see GASFET}

\medskip

\noindent
The Materka-Kacprzak temperature effects are as follows where $T$ and $T_{\ms{NOM}}$
are absolute temperatures in Kelvins (K).

\begin{align}
I_{DSS}(T) & =  I_{DSS}(T_{\ms{NOM}}
                 \left( 1 + A_{IDSS}(T-T_{\ms{NOM}} ) \right)\\
C_{10}(T)  & =  C_{10}(T_{\ms{NOM}}
                 \left( 1 + A_{C10}(T-T_{\ms{NOM}} ) \right)\\
C_{F0}(T)  & =  C_{F0}(T_{\ms{NOM}}
                 \left( 1 + A_{CF0}(T-T_{\ms{NOM}} ) \right)\\
E(T)       & =  E(T_{\ms{NOM}}
                 \left( 1 + A_E(T-T_{\ms{NOM}} ) \right)\\
K_E(T)     & =  K_E(T_{\ms{NOM}}
                 \left( 1 + A_{KE}(T-T_{\ms{NOM}} ) \right)\\
K_G(T)     & =  K_G(T_{\ms{NOM}}
                 \left( 1 + A_{KG}(T-T_{\ms{NOM}} ) \right)\\
\gamma (T) & =  \gamma(T_{\ms{NOM}}
                 \left( 1 + A_{\gamma}(T-T_{\ms{NOM}}) \right)\\
R_{10}(T)  & =  R_{10}(T_{\ms{NOM}}
                 \left( 1 + A_{R10}(T-T_{\ms{NOM}} ) \right)\\
R_G(T)     & =  R_G(T_{\ms{NOM}}
                 \left( 1 + A_{RG}(T-T_{\ms{NOM}} ) \right)\\
R_D(T)     & =  R_D(T_{\ms{NOM}}
                 \left( 1 + A_{RD}(T-T_{\ms{NOM}} ) \right)\\
R_S(T)     & =  R_S(T_{\ms{NOM}}
                 \left( 1 + A_{RS}(T-T_{\ms{NOM}} ) \right)\\
S_L(T)     & =  S_L(T_{\ms{NOM}}
                 \left( 1 + A_{SL}(T-T_{\ms{NOM}} ) \right)\\
S_S(T)     & =  S_S(T_{\ms{NOM}}
                 \left( 1 + A_{SS}(T-T_{\ms{NOM}} ) \right)\\
\tau(T)     & =  \tau(T_{\ms{NOM}}
                 \left( 1 + A_{\tau}(T-T_{\ms{NOM}} ) \right)\\
V_{BC}(T)  & =  V_{BC}(T_{\ms{NOM}}
                 \left( 1 + A_{VBC}(T-T_{\ms{NOM}} ) \right)\\
V_{P0}(T)  & =  V_{P0}(T_{\ms{NOM}}
                 \left( 1 + A_{VP0}(T-T_{\ms{NOM}} ) \right)
\end{align}


\noindent\underline{\sl \large Parasitic Resistances}\\[0.1in]
\index{Parasitic Resistances, see GASFET, \pspice}
\index{GASFET, \pspice\ Parasitic Resistance}
\index{GASFET, \pspice\ $R_S$}
\index{GASFET, \pspice\ $R_G$}
\index{GASFET, \pspice\ $R_D$}
\index{MESFET, see GASFET (\pspice)}

The resistive parasitics
$R'_S$, and $R'_D$ are calculated from the sheet resistivities
{\tt RS} (= $R_S$) and {\tt RD} (= $R_D$), and the
$Area$ specified on the element line.
{\tt RG} (= $R_G$) is used as supplied.
\begin{eqnarray}
R'_D & = & R_D/Area\\
R'_G & = & \left\{ \begin{array}{ll}
                          R_G      & \mbox{\pspice}\\
                          R_G/Area & \mbox{\sspice}\\\end{array}\right. \\ %}
R'_G & = & R_G/Area
\\
R'_S & = & R_S/Area
\end{eqnarray}
The parasitic resistance parameter dependencies are summarized in
figure \ref{b5para}.
\begin{figure}[h]
\parbox[t]{1.3in}{
\begin{tabular}[t]{|p{1in}|}
\hline
\multicolumn{1}{|c|}{KEYWORD} \\
\multicolumn{1}{|c|}{PARAMETERS} \\
\hline
\hline
{\tt RD} \hfill $R_D$\\
{\tt RG} \hfill $R_G$\\
{\tt RS} \hfill $R_S$\\
\hline
\end{tabular}
}
\hfill
\parbox{0.2in}{\ \vspace*{0.2in}\newline +}
\hfill
\begin{tabular}[t]{|p{1in}|}
\hline
\multicolumn{1}{|c|}{GEOMETRY} \\
\multicolumn{1}{|c|}{PARAMETER} \\
\hline
$Area$\\
\hline
\end{tabular}
\hfill
\parbox{0.2in}{\ \vspace*{0.2in}\newline $\rightarrow$}
\hfill
\begin{tabular}[t]{|p{1.8in}|}
\hline
\multicolumn{1}{|c|}{DEVICE}\\
\multicolumn{1}{|c|}{PARAMETERS}\\
\hline
\hspace*{\fill} $R'_D = f(Area, R_D)$\\
\hspace*{\fill} $R'_G = f(Area, R_G$\\
\hspace*{\fill} $R'_S = f(Area, R_S)$\\
\hline
\end{tabular}
\caption{MESFET parasitic resistance parameter
relationships. \label{b5para}}
\end{figure}

\noindent\underline{\sl {\tt LEVEL} 5 (Materka-Kacprzak Model)
I/V Characteristics}\\[0.1in]
\index{GASFET, LEVEL 5 (Materka-Kacprzak) Model, I/V}
\index{GASFET, LEVEL 5 (Materka-Kacprza) Model, I/V}
\index{I/V Characteristics, see GASFET}
\index{GASFET, I/V Characteristics}
\index{I-V characteristics, see GASFET}
The {\tt LEVEL 5} current/voltage characteristics are analytic
except for the channel resistance $R_I$ determination.
\begin{eqnarray}
I_{DS}&=&Area I_{DSS} \left[ 1 + S_S {{V_{DS}} \over {I_{DSS}}} \right]
         \left[ 1 - {{V_{GS}(t-\tau)} \over {V_{P0} + \gamma V_{DS}}}
         \right]^{\textstyle (E + K_E V_{GS}(t-\tau))} \nonumber\\
      & &\hspace*{1in} \times \mbox{tanh}\left[  {{S_L V_{DS}} \over
                {I_{DSS} ( 1 - K_G V_{GS}(t-\tau))}} \right]
      \label{b5id} \\
I_{GS}&=&Area I_{G0} \left[ e^{\textstyle A_{FAG}V_{GS}} -1 \right]
         - I_{B0} \left[ e^{\textstyle -A_{FAB}(V_{GS}+ V_{BC}}  \right]
      \label{b5igs} \\
I_{GD}&=&Area I_{G0} \left[ e^{\textstyle A_{FAG}V_{GD}} -1 \right]
         - I_{B0} \left[ e^{\textstyle -A_{FAB}(V_{GD}+ V_{BC}}  \right]
      \label{b5igd} \\
R_I&=& \left\{ \begin{array}{ll}
       R_{10} ( 1 - K_R V_{GS} )/Area  & K_R V_{GS} < 1.0\\
       0                         & K_R V_{GS} \ge 1.0\\
      \end{array} \right. %}
      \label{b5ri}
\end{eqnarray}

The relationships of the parameters describing the I/V
characteristics for the {\tt LEVEL} 5 model are summarized in figure
\ref{blevel5i/v}.\\[0.1in]
\begin{figure}[h]
\begin{tabular}[t]{|p{1in}|}
\hline
\multicolumn{1}{|c|}{KEYWORD} \\
\multicolumn{1}{|c|}{PARAMETERS} \\
\hline
\hline
{\tt AFAG} \hfill $A_{FAG}$\\
{\tt AFAB} \hfill $A_{FAB}$\\
{\tt GAMA} \hfill $\gamma$\\
{\tt IDSS} \hfill $I_{DSS}$\\
{\tt IB0} \hfill $I_{B0}$\\
{\tt IG0} \hfill $I_{G0}$\\
{\tt KG} \hfill $K_G$\\
{\tt KE} \hfill $K_E$\\
{\tt KR} \hfill $K_R$\\
{\tt R10} \hfill $R_{10}$\\
{\tt SS} \hfill $S_S$\\
{\tt SL} \hfill $S_L$\\
{\tt T} \hfill $\tau$\\
{\tt VBC} \hfill $V_{BC}$\\
{\tt VP0} \hfill $V_{P0}$\\
\hline
\end{tabular}
\hfill
\parbox{0.2in}{\ \vspace*{0.2in}\newline +}
\hfill
\begin{tabular}[t]{|p{1in}|}
\hline
\multicolumn{1}{|c|}{GEOMETRY} \\
\multicolumn{1}{|c|}{PARAMETER} \\
\hline
\hspace*{\fill}$Area$\\
\hline
\end{tabular}
\hfill
\parbox{0.2in}{\ \vspace*{0.2in}\newline $\rightarrow$}
\hfill
\begin{tabular}[t]{|p{1.8in}|}
\hline
\multicolumn{1}{|c|}{DEVICE} \\
\multicolumn{1}{|c|}{PARAMETERS} \\
\hline
$I_{DS} = f(Area, E, \gamma, I_{DSS},
       \newline\hspace*{\fill} K_E, K_G, S_L, S_S, \tau, V_{P0})$\\
$I_{GS} = f(Area, A_{FAB}, A_{FAG},
       \newline\hspace*{\fill} I_{B0}, I_{G0}, V_{BC})$ \\
$I_{GD} = f(Area, A_{FAB}, A_{FAG},
       \newline\hspace*{\fill} I_{B0}, I_{G0}, V_{BC})$ \\
$R_I =  f(Area, K_R, R_{10}, V_{GS})$ \\
\hline
\end{tabular}
\caption{LEVEL 5 (Materka-Kacprzak model) I/V dependencies. \label{bleve51i/v}}
\end{figure}

\noindent\underline{\sl LEVEL 5 (Materka-Kacprzak Model) Capacitances}\\[0.1in]
Two capacitance models are available depending on the value of the $C_{LVL}$
({\tt CLVL}) parameter.  With $C_{LVL}$ = 1 (default) the standard Materka
capacitance model described below is used.
With $C_{LVL}$ = 0 the Raytheon capacitance model
described on page \pageref{b:raytheon:capacitance} is used.
The Materka-Kacprzak capacitances are
\begin{eqnarray}
C'_{DS}&=&Area\, C_{DS} \\
C'_{GS}&=&\left\{ \begin{array}{ll}
         Area \left[ C_{10}( 1 - K_1 V_{GS} )^{M_{GS}} + C_{1S} \right]
           & K_1V_{GS} < F_{CC}\\
         Area \left[ C_{10}( 1 - F_{CC} )^{M_{GS}} + C_{1S} \right]
           & K_1V_{GS} \ge F_{CC}\\
         \end{array} \right. \\ %}
C'_{GD}&=&\left\{ \begin{array}{ll}
         Area \left[ C_{F0}( 1 - K_1 V_1 )^{M_{GD}} \right]
           & K_1V_1 < F_{CC}\\
         Area \left[ C_{F0}( 1 - F_{CC} )^{M_{GD}} \right]
           & K_1V_1 \ge F_{CC}\\
         \end{array} \right. %}
\end{eqnarray}

The {\tt LEVEL} 5 capacitance parameter dependencies are summarized in figure
\ref{blevel5cap}.\\[0.2in]
\begin{figure}[h]
\begin{tabular}[t]{|p{1in}|}
\hline
\multicolumn{1}{|c|}{KEYWORD} \\
\multicolumn{1}{|c|}{PARAMETERS} \\
\hline
\hline
{\tt CGD} \hfill $C_{GD}$\\
{\tt CGS} \hfill $C_{GS}$\\
{\tt CDS} \hfill $C_{DS}$\\
{\tt FC} \hfill $F_C$\\
{\tt VBI} \hfill $V_{BI}$\\
{\tt M} \hfill $M$\\
\hline
\end{tabular}
\hfill
\parbox{0.2in}{\ \vspace*{0.2in}\newline +}
\hfill
\begin{tabular}[t]{|p{1in}|}
\hline
\multicolumn{1}{|c|}{GEOMETRY} \\
\multicolumn{1}{|c|}{PARAMETER} \\
\hline
\hspace*{\fill}$Area$\\
\hline
\end{tabular}
\hfill
\parbox{0.2in}{\ \vspace*{0.2in}\newline $\rightarrow$}
\hfill
\begin{tabular}[t]{|p{1.8in}|}
\hline
\multicolumn{1}{|c|}{DEVICE} \\
\multicolumn{1}{|c|}{PARAMETERS} \\
\hline
$C'_{DS} = f(Area, C_{DS})$\\
$C'_{GD} = f(Area, C_{10}, C_{1S},
            \newline\hspace*{\fill} F_{CC}, K_1, V_{GS}, M_{GS})$\\
$C'_{GS} = f(Area, C_{F0}, C_{1S},
            \newline\hspace*{\fill} F_{CC}, K_1, V_1, V_{GS}, M_{GS})$\\
\hline
\end{tabular}
\caption{LEVEL 5 (Materka-Kacprzak model) capacitance dependencies.
\label{blevel5cap}}
\end{figure}

\ \myThickLine
\label{b:level6:start}
{\bf LEVEL 6 (Angelov Model)}\myline

The parameter keywords of the Materka-Kacprzak model are given in table
\ref{btable6}.


\begin{longtable}[h]{|p{0.6in}|p{3.5in}|p{0.6in}|p{0.6in}|}
\caption[GASFET model 6 (Angelov) keywords.]{GASFET model 6 (Angelov)
keywords.
Parameters that are {\sc NOT USED} are reserved for future
expansion.
         \label{btable6}}\\

\hline
\multicolumn{1}{|c}{\bf Name} &
\multicolumn{1}{|c}{\parbox{2.77in}{\bf Description}}  &
\multicolumn{1}{|c}{\bf Units} &
\multicolumn{1}{|c|}{\bf Default}\\ \hline
\endhead

\hline \multicolumn{4}{|r|}{{Continued on next page}} \\ \hline
\endfoot

\hline \hline
\endlastfoot

{\tt AF}& flicker noise exponent\sym{A_F}& -    & 1 \X
{\tt ACGS}  & linear temperature coefficient of {\tt CGS} \sym{A_{CGS}}
    & $^{\circ}\mbox{C}^{-1}$ & 0 \X
{\tt ACGD}  & linear temperature coefficient of {\tt CGD} \sym{A_{CGD}}
    & $^{\circ}\mbox{C}^{-1}$ & 0 \X
{\tt ALFA}  & saturation voltage parameter ($\ge 0$) \sym{\alpha}& V$^{-1}$ & 1.5 \X
{\tt ARI}   & linear temperature coefficient of {\tt RI} \sym{A_{RI}}
    & $^{\circ}\mbox{C}^{-1}$ & 0 \X
{\tt ARG}   & linear temperature coefficient of {\tt RG} \sym{A_{RG}}
    & $^{\circ}\mbox{C}^{-1}$ & 0 \X
{\tt ARD}   & linear temperature coefficient of {\tt RD} \sym{A_{RD}}
    & $^{\circ}\mbox{C}^{-1}$ & 0 \X
{\tt ARS}   & linear temperature coefficient of {\tt RS} \sym{A_{RS}}
    & $^{\circ}\mbox{C}^{-1}$ & 0 \X
{\tt B1}    & unsaturated coefficient of {\tt P1} \sym{B_1} &-& 0 \X
{\tt B2}    & drain voltage slope parameter for {\tt B1} \sym{B_2}
    &-& 3.0 \X
{\tt BCGD}  & quadratic  temperature coefficient of {\tt CGD} \sym{B_{CGD}}
    & $^{\circ}\mbox{C}^{-2}$ & 0 \X
{\tt BCGS}  & quadratic  temperature coefficient of {\tt CGS} \sym{B_{CGS}}
    & $^{\circ}\mbox{C}^{-2}$ & 0 \X
{\tt BRI}   & quadratic  temperature coefficient of {\tt RI} \sym{B_{RI}}
    & $^{\circ}\mbox{C}^{-2}$ & 0 \X
{\tt BRG}   & quadratic  temperature coefficient of {\tt RG} \sym{B_{RG}}
    & $^{\circ}\mbox{C}^{-2}$ & 0 \X
{\tt BRS}   & quadratic  temperature coefficient of {\tt RS} \sym{B_{RS}}
    & $^{\circ}\mbox{C}^{-2}$ & 0 \X
{\tt BRD}   & quadratic  temperature coefficient of {\tt RD} \sym{B_{RD}}
    & $^{\circ}\mbox{C}^{-2}$ & 0 \X
{\tt CGD0}  & gate-source capacitance at $\Psi_3=\Psi_4=0$ (>= 0)
         \sym{C_{GD0}}&F&0\X
{\tt CGS0}  & gate-source capacitance at $\Psi_1=\Psi_2=0$ (>= 0)
         \sym{C_{GS0}}&F&0\X
{\tt EG}    & barrier height at 0~K \sym{E_G} &V& 0.8 \X
{\tt GAMA}  & voltage slope parameter for pinch-off
                 \newline ({\sc NO LONGER USED})
        &V$^{-1}$& 0 \X
{\tt IB0}   & breakdown saturation current ($\ge 0$) \sym{I_{B0}} &A& 0 \X
{\tt IPK}   & drain current at peak $g_m$ ($\ge 0$) \sym{I_{PK}} &A& 0.1 \X
{\tt IS}    & diode saturation current ($\ge 0$) \sym{I_S} &A& 0 \X
{\tt LAMB}& slope of the drain characteristic \sym{\lambda} &V$^{-1}$& 0 \X
{\tt LEVEL}& model index \hfill must be 6 & -  &  1 \X
{\tt N} & diode ideality factor ($> 0$\sym{N} &-& 1 \X
{\tt NR}    & breakdown ideality factor ($> 0$ \sym{N_R} &-& 10 \X
{\tt P2}    & polynomial coefficient of channel current \sym{P_2}
         & V$^{-2}$ & 0 \X
{\tt P3}    & polynomial coefficient of channel current \sym{P_3}
         & V$^{-3}$ & 0 \X
{\tt P4}    & polynomial coefficient of channel current \sym{P_4}
         & V$^{-4}$ & 0 \X
{\tt P5}    & polynomial coefficient of channel current \sym{P_5}
         & V$^{-5}$ & 0 \X
{\tt P6}    & polynomial coefficient of channel current \sym{P_6}
         & V$^{-6}$ & 0 \X
{\tt P7}    & polynomial coefficient of channel current \sym{P_7}
         & V$^{-7}$ & 0 \X
{\tt P8}    & polynomial coefficient of channel current \sym{P_7}
         & V$^{-8}$ & 0 \X
{\tt P10}   & polynomial coefficient of g-s capacitance \sym{P_{10}}
    & - & 0 \X
{\tt P11}   & polynomial coefficient of g-s capacitance \sym{P_{11}}
    & V$^{-1}$ & 0 \X
{\tt P12}   & polynomial coefficient of g-s capacitance \sym{P_{12}}
    & V$^{-2}$ & 0 \X
{\tt P13}   & polynomial coefficient of g-s capacitance \sym{P_{13}}
    & V$^{-3}$ & 0 \X
{\tt P14}   & polynomial coefficient of g-s capacitance \sym{P_{14}}
    & V$^{-4}$ & 0 \X
{\tt P20}   & polynomial coefficient of g-s capacitance \sym{P_{20}}
    & - & 0 \X
{\tt P21}   & polynomial coefficient of g-s capacitance \sym{P_{21}}
    & V$^{-1}$ & 0 \X
{\tt P22}   & polynomial coefficient of g-s capacitance \sym{P_{22}}
    & V$^{-2}$ & 0 \X
{\tt P23}   & polynomial coefficient of g-s capacitance \sym{P_{23}}
    & V$^{-3}$ & 0 \X
{\tt P24}   & polynomial coefficient of g-s capacitance \sym{P_{24}}
    & V$^{-4}$ & 0 \X
{\tt P30}   & polynomial coefficient of g-d capacitance \sym{P_{30}}
    & - & 0 \X
{\tt P31}   & polynomial coefficient of g-d capacitance \sym{P_{31}}
    & V$^{-1}$ & 0 \X
{\tt P32}   & polynomial coefficient of g-d capacitance \sym{P_{32}}
    & V$^{-2}$ & 0 \X
{\tt P33}   & polynomial coefficient of g-d capacitance \sym{P_{33}}
    & V$^{-3}$ & 0 \X
{\tt P34}   & polynomial coefficient of g-d capacitance \sym{P_{34}}
    & V$^{-4}$ & 0 \X
{\tt P40}   & polynomial coefficient of g-d capacitance \sym{P_{40}}
    & - & 0 \X
{\tt P41}   & polynomial coefficient of g-d capacitance \sym{P_{41}}
    & V$^{-1}$ & 0 \X
{\tt P42}   & polynomial coefficient of g-d capacitance \sym{P_{42}}
    & V$^{-2}$ & 0 \X
{\tt P43}   & polynomial coefficient of g-d capacitance \sym{P_{43}}
    & V$^{-3}$ & 0 \X
{\tt P44}   & polynomial coefficient of g-d capacitance \sym{P_{44}}
    & V$^{-4}$ & 0 \X
{\tt P1CC}  & polynomial coefficient of g-d capacitance \sym{P_{1CC}}
    & & 0 \X
{\tt PSAT}  & polynomial coefficient of channel current {\tt PSAT}  \sym{P_1}
    & V$^{-1}$ & 1.3 \X
{\tt RI}    & channel resistacne ($\ge 0$) {\sc NOT USED} \sym{R_I} & & 0 \X
{\tt T} & channel time delay ($\ge 0$) \sym{\tau} &s& 0 \X
%{\tt TCGD1}& alternative keyword for {\tt ACGD}\sym{A_{CGD}} & C$^{-1}$ & 0 \X
%{\tt TCGD2}& alternative keyword for {\tt BCGD}\sym{B_{CGD}} & C$^{-1}$ & 0 \X
%{\tt TCGS1}& alternative keyword for {\tt ACGS}\sym{A_{CGS}} & C$^{-1}$ & 0 \X
%{\tt TCGS2}& alternative keyword for {\tt BCGS}\sym{B_{CGS}} & C$^{-1}$ & 0 \X
%{\tt TCGD1}& alternative keyword for {\tt ACGD}\sym{A_{CGD}} & C$^{-1}$ & 0 \X
%{\tt TCGD2}& alternative keyword for {\tt BCGD}\sym{B_{CGD}} & C$^{-1}$ & 0 \X
{\tt TJ}    & junction temperature \sym{T_J} &K& 298 \X
{\tt TM}    & $I_{DS}$ linear temperature coefficient \sym{T_M} & & 0 \X
{\tt TME}   & $I_{DS}$ power law temperature coefficient \sym{T_{ME}}&&0 \X
{\tt TNOM}  & model reference temperature ($> 0$) \sym{T_{NOM}} &K& 298 \X
%{\tt TRD1}& alternative keyword for {\tt ARD}\sym{A_{RD}} & C$^{-1}$ & 0 \X
%{\tt TRD2}& alternative keyword for {\tt BRD}\sym{B_{RD}} & C$^{-1}$ & 0 \X
%{\tt TRG1}& alternative keyword for {\tt ARG}\sym{A_{RG}} & C$^{-1}$ & 0 \X
%{\tt TRG2}& alternative keyword for {\tt BRG}\sym{B_{RG}} & C$^{-1}$ & 0 \X
%{\tt TRI1}& alternative keyword for {\tt ARI}\sym{A_{RI}} & C$^{-1}$ & 0 \X
%{\tt TRI2}& alternative keyword for {\tt BRI}\sym{B_{RI}} & & 0 \X
%{\tt TRS1}& alternative keyword for {\tt ARS}\sym{A_{RS}} & & 0 \X
%{\tt TRS2}& alternative keyword for {\tt BRS}\sym{B_{RS}} & & 0 \X
{\tt VBD}   & breakdown voltage \sym{V_{BD}} & & 0 \X
{\tt VPK0}  & gate-source voltage for unsaturated peak $g_m$\sym{V_{PK0}}
    &V& -0.5 \X
{\tt VPKS}  & gate-source voltage for peak $g_m$ \sym{V_{PKS}}
    &V& 0 \X
{\tt XTI}   & saturation current temperature exponent \sym{X_{TI}}
    & & 2 \X
\end{longtable}

\noindent\underline{\sl \large Temperature Dependence}
\index{GASFET, Temperature Dependence}
\index{Temperature Dependence, see GASFET}
\\[0.1in]
Temperature effects are incorporated as follows where $T$ and $T_{\ms{NOM}}$
are absolute temperatures in Kelvins (K) and
the thermal voltage $V_{\ms{TH}} = kT / q$.
\begin{eqnarray}
\alpha(T)   &=& \alpha(T_{\ms{NOM}})  \left(
                    1 + A_{\alpha} (T-T_{\ms{NOM}}) \right) \\
C_{GS0}(T)  &=& C_{GS0}(T_{\ms{NOM}}) \left( 1 +
    A_{CGS} (T-T_{\ms{NOM}}) + B_{CGS} (T-T_{\ms{NOM}})^2  \right) \\
C_{GD0}(T)  &=& C_{GD0}(T_{\ms{NOM}}) \left( 1 +
    A_{CGD} (T-T_{\ms{NOM}}) + B_{CGD} (T-T_{\ms{NOM}})^2  \right) \\
E_G(T) & = & E_G(0) -
      F_{\ms{GAP1}}{{4T^2} / \left( {T+F_{\ms{GAP2}}} \right) }\\
\gamma(T)   &=& \gamma(T_{\ms{NOM}})  \left(
                    1 + A_{\gamma} (T-T_{\ms{NOM}}) \right) \\
I_{PK}(T)   &=& I_{PK}(T_{\ms{NOM}})  \left(
                    1 + A_{IPK} (T-T_{\ms{NOM}}) \right) \\
R_I(T)      &=& R_I(T_{\ms{NOM}})  \left( 1 +
    A_{RI} (T-T_{\ms{NOM}}) + B_{RI} (T-T_{\ms{NOM}})^2  \right) \\
R_G(T)      &=& R_G(T_{\ms{NOM}})  \left( 1 +
    A_{RG} (T-T_{\ms{NOM}}) + B_{RG} (T-T_{\ms{NOM}})^2  \right) \\
R_D(T)      &=& R_D(T_{\ms{NOM}})  \left( 1 +
    A_{RD} (T-T_{\ms{NOM}}) + B_{RD} (T-T_{\ms{NOM}})^2  \right) \\
R_S(T)      &=& R_S(T_{\ms{NOM}})  \left( 1 +
    A_{RS} (T-T_{\ms{NOM}}) + B_{RS} (T-T_{\ms{NOM}})^2  \right) \\
V_{PK}(T)   &=& V_{PK}(T_{\ms{NOM}})  \left(
                    1 + A_{VPK} (T-T_{\ms{NOM}}) \right) \\
V_{BD}(T)   &=& V_{BD}(T_{\ms{NOM}})  \left(
                    1 + A_{VBD} (T-T_{\ms{NOM}}) \right) \\
&&  {\tt TM} I_{DS} {\tt TME}  ????
\end{eqnarray}



\noindent\underline{\sl \large Parasitic Resistances}\\[0.1in]
\index{Parasitic Resistances, see GASFET, \pspice}
\index{GASFET, \pspice\ Parasitic Resistance}
\index{GASFET, \pspice\ $R_S$}
\index{GASFET, \pspice\ $R_G$}
\index{GASFET, \pspice\ $R_D$}
\index{MESFET, see GASFET (\pspice)}
The resistive parasitics
$R'_S$, and $R'_D$ are calculated from the sheet resistivities
{\tt RS} (= $R_S$) and {\tt RD} (= $R_D$), and the
$Area$ specified on the element line.
{\tt RG} (= $R_G$) is used as supplied.
\begin{eqnarray}
R'_D & = & R_D/Area\\
R'_G & = & R_G/Area\\
R'_S & = & R_S/Area
\end{eqnarray}
The parasitic resistance parameter dependencies are summarized in
Figure \ref{b6para}.

\begin{figure}[h]
\parbox[t]{1.3in}{
\begin{tabular}[t]{|p{1in}|}
\hline
\multicolumn{1}{|c|}{KEYWORD} \\
\multicolumn{1}{|c|}{PARAMETERS} \\
\hline
\hline
{\tt RD} \hfill $R_D$\\
{\tt RG} \hfill $R_G$\\
{\tt RS} \hfill $R_S$\\
\hline
\end{tabular}
}
\hfill
\parbox{0.2in}{\ \vspace*{0.2in}\newline +}
\hfill
\begin{tabular}[t]{|p{1in}|}
\hline
\multicolumn{1}{|c|}{GEOMETRY} \\
\multicolumn{1}{|c|}{PARAMETER} \\
\hline
$Area$\\
\hline
\end{tabular}
\hfill
\parbox{0.2in}{\ \vspace*{0.2in}\newline $\rightarrow$}
\hfill
\begin{tabular}[t]{|p{1.8in}|}
\hline
\multicolumn{1}{|c|}{DEVICE}\\
\multicolumn{1}{|c|}{PARAMETERS}\\
\hline
\hspace*{\fill} $R'_D = f(Area, R_D)$\\
\hspace*{\fill} $R'_G = f(Area, R_G)$\\
\hspace*{\fill} $R'_S = f(Area, R_S)$\\
\hline
\end{tabular}
\caption{MESFET parasitic resistance parameter
relationships. \label{b6para}}
\end{figure}

\noindent\underline{\sl {\tt LEVEL} 6 (Angelov Model)
I/V Characteristics}\\[0.1in]
\index{GASFET, LEVEL 6 (Angelov) Model, I/V}
\index{GASFET, LEVEL 6 (Angelov) Model, I/V}
\index{I/V Characteristics, see GASFET}
\index{GASFET, I/V Characteristics}
\index{I-V characteristics, see GASFET}
The {\tt LEVEL 6} current/voltage characteristics are analytic:
\begin{eqnarray}
I_{DS}&=& I_{PK} [ 1 + \mbox{tanh}(\Psi)]
                 [ 1 + \lambda V_{DS}]
                 \mbox{tanh}(\alpha V_{DS}) \\
\Psi  &=& P_1 \left(V_{GS} - V_{PK} \right)
        + P_2 \left(V_{GS} - V_{PK} \right) ^2
        + P_3 \left(V_{GS} - V_{PK} \right) ^3 \nonumber\\
     && + P_4 \left(V_{GS} - V_{PK} \right) ^4
        + P_5 \left(V_{GS} - V_{PK} \right) ^5
        + P_6 \left(V_{GS} - V_{PK} \right) ^6\nonumber\\
     && + P_7 \left(V_{GS} - V_{PK} \right) ^7
        + P_8 \left(V_{GS} - V_{PK} \right) ^8
    \\
P_1 &=& P_{\ms{SAT}}
       \left[ 1 + B_1/\left(\mbox{cosh}^2(B_2 V_{DS}) \right)\right]\\
V_{PK} &=& V_{PK0} + (V_{PKS}-V_{PK0}) \mbox{tanh}(\alpha V_{DS})\\
I_{GS}&=&I_S \left[ e^{V_{GS}/(N V_{\ms{TH}})} -1 \right]
         - I_{B0} \left[ e^{-(V_{GS}+ V_{BD})/(N_R V_{\ms{TH}})}  \right]
      \label{b6igs} \\
I_{GD}&=&I_S \left[ e^{V_{GD}/(N V_{\ms{TH}})} -1 \right]
         - I_{B0} \left[ e^{-(V_{GD}+ V_{BD})/(N_R V_{\ms{TH}})}  \right]
      \label{b6igd} \\
      R_I &=& R'_I/Area
\end{eqnarray}

The relationships of the parameters describing the I/V
characteristics for the {\tt LEVEL} 6 model are summarized in figure
\ref{blevel6i/v}.

\medskip

\begin{figure}[h]
\begin{tabular}[t]{|p{1in}|}
\hline
\multicolumn{1}{|c|}{KEYWORD} \\
\multicolumn{1}{|c|}{PARAMETERS} \\
\hline
\hline
{\tt IPK} \hfill $ I_{PK}$\\
{\tt LAMB} \hfill $ \lambda$\\
{\tt ALFA} \hfill $ \alpha$\\
{\tt P1} \hfill $ P_1$\\
{\tt P2} \hfill $ P_2$\\
{\tt P3} \hfill $ P_3$\\
{\tt P4} \hfill $ P_4$\\
{\tt P5} \hfill $ P_5$\\
{\tt P6} \hfill $ P_6$\\
{\tt P7} \hfill $ P_7$\\
{\tt P8} \hfill $ P_8$\\
{\tt IS} \hfill $ I_S$\\
{\tt N} \hfill $ N$\\
{\tt IB0} \hfill $I_{B0}$\\
{\tt NR} \hfill $ N_R$\\
{\tt VBD} \hfill $ V_{BD}$\\
{\tt VPK0} \hfill $ V_{PK0}$\\
{\tt VPKS} \hfill $ V_{PKS}$\\
\hline
\end{tabular}
\hfill
\parbox{0.2in}{\ \vspace*{0.2in}\newline +}
\hfill
\begin{tabular}[t]{|p{1in}|}
\hline
\multicolumn{1}{|c|}{GEOMETRY} \\
\multicolumn{1}{|c|}{PARAMETER} \\
\hline
$Area$\\
\hline
\end{tabular}
\hfill
\parbox{0.2in}{\ \vspace*{0.2in}\newline $\rightarrow$}
\hfill
\begin{tabular}[t]{|p{1.8in}|}
\hline
\multicolumn{1}{|c|}{DEVICE} \\
\multicolumn{1}{|c|}{PARAMETERS} \\
\hline
$I_{DS} = f(Area, I_{PK}, \lambda, \alpha,$\newline\hspace*{\fill}$
 P_1, P_2, P_3, P_4, P_5, P_6, P_7, P_8,$\newline\hspace*{\fill}$
 V_{PK0}, V_{PKS},)$\\
$I_{GS} =$\newline\hspace*{\fill}$ f(Area, I_S, N, I_{B0}, V_{BD}, N_R)$\\
$I_{GS} =$\newline\hspace*{\fill}$ f(Area, I_S, N, I_{B0}, V_{BD}, N_R)$\\
\hline
\end{tabular}
\caption{LEVEL 6 (Angelov model) I/V dependencies. \label{bleve61i/v}}
\end{figure}

\noindent\underline{\sl LEVEL 6 (Angelov Model) Capacitances}\\[0.1in]
The Angelov capacitances are
\begin{eqnarray}
C'_{DS}&=&Area\, C_{DS} \\
C'_{GS}&=&Area C_{GS0} \left[ 1 + \mbox{tanh} (\Psi_1) \right]
                       \left[ 1 + \mbox{tanh} (\Psi_2) \right]\\
&&\Psi_1 = P_{10}+P_{11}V_{GS}+P_{12}V_{GS}^2+P_{13}V_{GS}^3+P_{14}V_{GS}^4\\
&&\Psi_2 = P_{20}+P_{11}V_{DS}+P_{22}V_{DS}^2+P_{23}V_{DS}^3+P_{24}V_{DS}^4\\
C'_{GD}&=&Area C_{GD0} \left[ 1 + \mbox{tanh} (\Psi_3) \right]
                       \left[ 1 - \mbox{tanh} (\Psi_4) \right]\\
&&\Psi_3 = P_{30}+P_{31}V_{GS}+P_{32}V_{GS}^2+P_{33}V_{GS}^3+P_{34}V_{GS}^4\\
&&\Psi_4 = P_{40}+    \left( P_{41}+P_{1CC} V_GS \right) V_{DS}
         + P_{42}V_{DS}^2+P_{43}V_{DS}^3+P_{44}V_{DS}^4
\end{eqnarray}

The {\tt LEVEL} 6 capacitance parameter dependencies are summarized in figure
\ref{blevel6cap}.\\[0.2in]
\begin{figure}[h]
\begin{tabular}[t]{|p{1in}|}
\hline
\multicolumn{1}{|c|}{KEYWORD} \\
\multicolumn{1}{|c|}{PARAMETERS} \\
\hline
\hline
{\tt CGD} \hfill $C_{GD}$\\
{\tt CGS} \hfill $C_{GS}$\\
{\tt CDS} \hfill $C_{DS}$\\
{\tt FC} \hfill $F_C$\\
{\tt VBI} \hfill $V_{BI}$\\
{\tt M} \hfill $M$\\
\hline
\end{tabular}
\hfill
\parbox{0.2in}{\ \vspace*{0.2in}\newline +}
\hfill
\begin{tabular}[t]{|p{1in}|}
\hline
\multicolumn{1}{|c|}{GEOMETRY} \\
\multicolumn{1}{|c|}{PARAMETER} \\
\hline
\hspace*{\fill}$Area$\\
\hline
\end{tabular}
\hfill
\parbox{0.2in}{\ \vspace*{0.2in}\newline $\rightarrow$}
\hfill
\begin{tabular}[t]{|p{1.8in}|}
\hline
\multicolumn{1}{|c|}{DEVICE} \\
\multicolumn{1}{|c|}{PARAMETERS} \\
\hline
$C'_{DS} = f(Area, C_{DS})$\\
$C'_{GD} = f(Area, C_{10}, C_{1S},
      \newline\hspace*{\fill} F_{CC}, K_1, V_{GS}, M_{GS})$\\
$C'_{GS} = f(Area, C_{F0}, C_{1S},
      \newline\hspace*{\fill} F_{CC}, K_1, V_1, V_{GS}, M_{GS})$\\
\hline
\end{tabular}
\caption{LEVEL 6 (Angelov model) capacitance dependencies.
\label{blevel6cap}}
\end{figure}

\noindent\underline{\sl \large AC Analysis}
\index{GASFET, AC Analysis}
\begin{figure}[h]
\centering
\ \epsfxsize=2.75in\pfig{gaasAC.ps}
\caption[Small signal GASFET model]{Small signal GASFET model
showing noise sources $I_{n,G}$, $I_{n,D}$, $I_{n,S}$, and
$I_{n,DS}$ used in the noise analysis.
$R_I$ is not used in \pspice .
\label{fig:gaasAC} }
\end{figure}
The AC analysis uses the model of figure \ref{fig:gaasAC} with the capacitor
values evaluated at the \dc\ operating point with\\
\hspace*{\fill}
$g_m={{\textstyle\partial I_{DS}} \over {\textstyle\partial V_{GS}}}
\hfill
R_{GD}={{\textstyle\partial I_{GD}} \over {\textstyle\partial V_{GD}}}
\hfill
R_{GS}={{\textstyle\partial I_{GS}} \over {\textstyle\partial V_{GS}}}
\hfill
R_{DS}={{\textstyle\partial I_{DS}} \over {\textstyle\partial V_{DS}}}$
\inlineeq

\noindent\underline{\sl \large Noise Analysis}\\[0.1in]
\index{GASFET, Noise Model}
\index{GASFET, Noise Analysis}
The MESFET noise model, see figure \ref{fig:gaasAC}, accounts for thermal
noise generated in the
parasitic resistamces and shot and flicker noise generated in the
drain source current generator.  The rms (root-mean-square) values of
thermal noise current generators shunting the three parasitic resistance
$R_D$, $R_G$ and $R_S$ are

\begin{eqnarray}
I_{n,D} &=& \sqrt{4kT/R'_D}~\mbox{A/}\sqrt{\mbox{Hz}}\\
I_{n,G} &=& \sqrt{4kT/R'_G}~\mbox{A/}\sqrt{\mbox{Hz}}\\
I_{n,S} &=& \sqrt{4kT/R'_S}~\mbox{A/}\sqrt{\mbox{Hz}}
\end{eqnarray}
Shot and flicker noise are modeled by
a noise current generator in series with the drain-source current generator
$I_{DS}$.
The rms value of this noise generator is given by
\begin{eqnarray}
I_{n,DS} &=& \sqrt{I_{\ms{SHOT},DS}^2 + I_{\ms{FLICKER},DS}^2}
~~~~\mbox{A/}\sqrt{\mbox{Hz}}\\
I_{\ms{SHOT},DS} &=& \sqrt{4kTg_m\frac{2}{3}}
~~~~\mbox{A/}\sqrt{\mbox{Hz}}\\
I_{\ms{FLICKER},DS} &=& \sqrt{{{\textstyle\KF I_{DS}^{\AF}}
                         \over {\textstyle f }}}
~~~~\mbox{A/}\sqrt{\mbox{Hz}}
\end{eqnarray}
where $f$ is the analysis frequency.
