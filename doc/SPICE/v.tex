\element{V}{Independent Voltage Source}
%\newcommand{\tenrm}{\small}
\begin{figure}[h]
\centering
\ \pfig{v_spice.ps}
\caption{V --- Independent voltage source.}
\end{figure}

\form{{\tt V}name $N_{+}$ $N_{-}$ \B \B DC\E \B DCvalue\E \newline
      {\tt +}\B {\tt AC} \B ACmagnitude \B ACphase\E \E \E  \newline
      {\tt +} \B {\tt DISTOF1} \B F1Magnitude \B F1Phase\E \E \E
      {\tt +} \B {\tt DISTOF2} \B F2Magnitude \B F2Phase\E \E \E }

\spicethreeform{{\tt V}name $N_{+}$ $N_{-}$ \B \B DC\E \B DCvalue\E \newline
      {\tt +}\B {\tt AC} \B ACmagnitude \B ACphase\E \E \E  \newline
      {\tt +} \B TransientSpecification \E\newline
      {\tt +} \B {\tt DISTOF1} \B F1Magnitude \B F1Phase\E \E \E\newline
      {\tt +} \B {\tt DISTOF2} \B F2Magnitude \B F2Phase\E \E \E }

\pspiceform{{\tt V}name $N_{+}$ $N_{-}$ \B \B DC\E  \B DCvalue\E
      \B {\tt AC} \B ACmagnitude \B ACphase\E \E \E\\
      {\tt +} \B TransientSpecification \E
       \B {\tt SNR} InputVoltageSNR \E
             \B {\tt RS} SourceResistanceValue \E
             \B {\tt RL} LoadResistanceValue \E }

\example{VBIAS 1 0 5.0\\
         VCLOCK 20 5 PULSE(0 5 1N 2N 1.5N 21.9N 5N 20N)\\
         VSSIGNAL AC 1U 90}

\begin{widelist}
\item[$N_{+}$] is the positive voltage source node.
\item[$N_{-}$] is the negative voltage source node.

\item[{\tt DC}] is the optional keyword for the \dc\ value of the source.

\item[{\it DCvalue}] is the \dc\ voltage value of the source.\\
               (Units: V; Optional; Default: 0; Symbol: $V_{DC}$)

\item[{\tt AC}] is the keyword for the \ac\ value of the source.

\item[{\it ACmagnitude}] is the \ac\ magnitude of the source used during
\ac\ analysis.
That is, it is the peak \ac\ voltage so that the \ac\ signal is
$\mbox{{\it ACmagnitude}}\,\mbox{sin}(\omega t + \mbox{ACphase})$.
{\it ACmagnitude} is ignored for other types of analyses.\\
               (Units: V; Optional; Default: 1; Symbol: $V_{AC}$)
\item[{\it ACphase}] is the ac  phase of the source.  It is used only in
               \ac\ analysis.\\
               (Units: Degrees; Optional; Default: 0; Symbol: $\phi_{\ms{AC}}$)
\end{widelist}

\begin{widelist}
\item[{\tt DISTOF1}] is the distortion keyword for distortion component 1
              which hass frequency {\tt F1}.
              (see  the  description  of  the {\tt .DISTO}  statement on
              page \pageref{.DISTOstatement}).

\item[{\it F1magnitude}] is the magnitude of the distortion component at
              {\tt F1}. See {\tt .DISTOF1} keyword above.\\
               (Units: V; Optional; Default: 1; Symbol: $V_{F1}$)

\item[{\it F1phase}] is the phase of the distortion component at
              {\tt F1}. See {\tt .DISTOF1} keyword above.\\
               (Units: Degrees; Optional; Default: 0; Symbol: $\phi_{F1}$)


\item[{\tt DISTOF2}] is the distortion keyword for distortion component 2
              which hass frequency {\tt F2}.
              (see  the  description  of  the {\tt .DISTO}  statement on
              page \pageref{.DISTOstatement}).

\item[{\it F2magnitude}] is the magnitude of the distortion component at
              {\tt F2}. See {\tt .DISTOF2} keyword above.\\
               (Units: V; Optional; Default: 1; Symbol: $V_{F2}$)

\item[{\it F2phase}] is the phase of the distortion component at
              {\tt F2}. See {\tt .DISTOF2} keyword above.\\
               (Units: Degrees; Optional; Default: 0; Symbol: $\phi_{F2}$)

\item[{\tt SNR}] is the input signal-to-noise ratio keyword. \sspice

\item[{\it InputVoltageSNR}] is the value of the signal-to-noise ratio at the
               input.\\
               (Units: None; Optional; Default: use thermal noise of $R_S$;
                Symbol: $\mbox{SNR}_I$)  \sspice

\item[{\tt RS}] is the source resistance keyword.  \sspice

\item[{\it SourceResistanceValue}] is the value of the source resistance.\\
               (Units: Ohms; Optional; Default: 50$\Omega$; Symbol: $R_S$)\\
               Note: if port 1 is specified then the resistance specified
               for the port takes precedence.  \sspice

\item[{\tt RL}] is the source resistance keyword.  \sspice

\item[{\it LoadResistanceValue}] is the value of the load resistance.\\
               (Units: Ohms; Optional; Default: 50$\Omega$; Symbol: $R_L$)\\
               Note: if port 2 is specified then the resistance specified
               for the port takes precedence.  \sspice

\item[{\it TransientSpecification}] is the optional transient specification
          described more fully below.
\end{widelist}
\note{
\item
The independent voltage source has three different sets of
parameters to describe the source for
DC analysis (see {\tt .DC} on page \pageref{.DCstatement}),
AC analysis (see {\tt .AC} on page \pageref{.ACstatement}), and
transient analysis (see {\tt .TRAN} on page \pageref{.TRANstatement}).
The \dc\ value of the source is used during bias point evaluation and \dc\ analysis
is {\it DCValue}. It is also the constant value of the voltage source if no
{\it TransientSpecification} is supplied.
It may also be used in conjunction with the {\tt PWL} transient specification
if a time zero value is not provided as part of the transient specification.
The \ac\ specification, indicated by the keyword {\tt AC} is independent
of the \dc\ parameters and the {\it Transient Specification}.

\item See the {\tt .NOISE} statement description for a discussion of how
$\mbox{SNR}_I$ $R_S$, $R_L$ are used in noise calculations.

\item
The original documentation distributed with \spicetwo\ and \spicethree\
incorrectly stated that if a {\it TransientSpecification} was supplied then
the time-zero transient voltage was used in \dc\ analysis and in determiniong
the operating point.}

\noindent{\large \bf  Transient Specification}

Five transient specification forms are supported:
pulse ({\tt PULSE}),  exponential ({\tt EXP}),  sinusoidal ({\tt SIN}),
piece-wise  linear ({\tt PWL}),   and
single-frequency FM ({\tt SFFM}).  The default values of some of the parameters
of these transient specifications include
{\tt TSTEP} which is the printing increment and {\tt TSTOP} which is the
final time (see the {\tt .TRAN} statement on page \pageref{.TRANstatement}
for further explanation of these quantities).
In the following $t$ is the transient analysis time.

\noindent{\underline{\bf Exponential:}}
\\[0.2in]
\form{{\tt EXP(} $V_1$ $V_2$ \B $T_{D1}$ \E \B $\tau_1$ \E
       \B $T_{D2}$ \E \B$\tau_2$ \E {\tt )}}
\vspace{-0.5in}
\keywordtable{
$V_1   $&initial voltage    & A &  \reqd   \X
$V_2   $&pulsed voltage     & A &  \reqd   \X
$T_{D1}$&rise delay time    & s &  0.0     \X
$\tau_1$&rise time constant & s &  {\tt TSTEP}    \X
$T_{D2}$&fall delay time    & s &  $T_{D1}$
                 +\newline\hspace*{\fill}{\tt TSTEP}\X
$\tau_2$&fall time constant & s &  {\tt TSTEP}\X
}
The exponential transient is a single-shot event specifying two exponentials.
The voltage is
$V_1$ for the first $T_{D1}$ seconds at which it begins increasing exponentially
towards $V_2$ with a time constant of $\tau_1$ seconds.  At time
$T_{D2}$ the voltage exponentially decays towards $V_1$ with a time constant
of $\tau_2$. That is,
\begin{equation}
v = \left\{ \begin{array}{ll}
     V_1                                           & t \le T_{D1}\\
     V_1+(V_2-V_1)(1-e^{\textstyle (-(t-T_{D1})/\tau_1)})  & T_{D1} < t \le T_{D2}\\
     V_1+(V_2-V_1)(1-e^{\textstyle (-(t-T_{D1})/\tau_1)})
        +(V_1-V_2)(1-e^{\textstyle (-(t-T_{D2})/\tau_2)})  &  t > T_{D2}
     \end{array} \right. %}
\end{equation}
\vspace*{-0.2in}
\begin{figure}[hbp]
\centering
%set samples 200
%set yrange [0:2]
%i(x) = (x < 1)? 0.5 : ( (x < 2)? (0.5 + (1-exp(-(x-1)/0.35))) :\
%       (0.5 + (1-exp(-(x-1)/0.35))- (1-exp(-(x-2)/1))) )
%i1(x) = (x < 1)? 0.5 : ( 0.5 + (1-exp(-(x-1)/0.35)))
%i2(x) = 0.5
%set term latex
%set output 'iexp.tex'
%set size 1.0,1.0
%set border
%set nokey
%set xtics
%set ytics
%set xzeroaxis
%set yzeroaxis
%plot [0:4] i(x) with line 1, i1(x) with line 4, i2(x) with line 4,\
%   0 with line 4
%set term x11
%plot [0:4] i(x) with line 1, i1(x) with line 4, i2(x) with line 4,\
%   0 with line 4
% GNUPLOT: LaTeX picture
\setlength{\unitlength}{0.240900pt}
\ifx\plotpoint\undefined\newsavebox{\plotpoint}\fi
\sbox{\plotpoint}{\rule[-0.175pt]{0.350pt}{0.350pt}}%
\begin{picture}(1500,860)(0,40)
%\tenrm
\sbox{\plotpoint}{\rule[-0.175pt]{0.350pt}{0.350pt}}%
\put(264,158){\rule[-0.175pt]{282.335pt}{0.350pt}}
\put(264,158){\rule[-0.175pt]{0.350pt}{151.526pt}}
\put(264,158){\rule[-0.175pt]{4.818pt}{0.350pt}}
\put(242,158){\makebox(0,0)[r]{0}}
\put(1416,158){\rule[-0.175pt]{4.818pt}{0.350pt}}
\put(264,221){\rule[-0.175pt]{4.818pt}{0.350pt}}
\put(242,221){\makebox(0,0)[r]{0.2}}
\put(1416,221){\rule[-0.175pt]{4.818pt}{0.350pt}}
\put(264,284){\rule[-0.175pt]{4.818pt}{0.350pt}}
\put(242,284){\makebox(0,0)[r]{0.4}}
\put(1416,284){\rule[-0.175pt]{4.818pt}{0.350pt}}
\put(264,347){\rule[-0.175pt]{4.818pt}{0.350pt}}
\put(242,347){\makebox(0,0)[r]{0.6}}
\put(1416,347){\rule[-0.175pt]{4.818pt}{0.350pt}}
\put(264,410){\rule[-0.175pt]{4.818pt}{0.350pt}}
\put(242,410){\makebox(0,0)[r]{0.8}}
\put(1416,410){\rule[-0.175pt]{4.818pt}{0.350pt}}
\put(264,473){\rule[-0.175pt]{4.818pt}{0.350pt}}
\put(242,473){\makebox(0,0)[r]{1}}
\put(1416,473){\rule[-0.175pt]{4.818pt}{0.350pt}}
\put(264,535){\rule[-0.175pt]{4.818pt}{0.350pt}}
\put(242,535){\makebox(0,0)[r]{1.2}}
\put(1416,535){\rule[-0.175pt]{4.818pt}{0.350pt}}
\put(264,598){\rule[-0.175pt]{4.818pt}{0.350pt}}
\put(242,598){\makebox(0,0)[r]{1.4}}
\put(1416,598){\rule[-0.175pt]{4.818pt}{0.350pt}}
\put(264,661){\rule[-0.175pt]{4.818pt}{0.350pt}}
\put(242,661){\makebox(0,0)[r]{1.6}}
\put(1416,661){\rule[-0.175pt]{4.818pt}{0.350pt}}
\put(264,724){\rule[-0.175pt]{4.818pt}{0.350pt}}
\put(242,724){\makebox(0,0)[r]{1.8}}
\put(1416,724){\rule[-0.175pt]{4.818pt}{0.350pt}}
\put(264,787){\rule[-0.175pt]{4.818pt}{0.350pt}}
\put(242,787){\makebox(0,0)[r]{2}}
\put(1416,787){\rule[-0.175pt]{4.818pt}{0.350pt}}
\put(264,158){\rule[-0.175pt]{0.350pt}{4.818pt}}
\put(264,113){\makebox(0,0){0}}
\put(264,767){\rule[-0.175pt]{0.350pt}{4.818pt}}
\put(411,158){\rule[-0.175pt]{0.350pt}{4.818pt}}
\put(411,113){\makebox(0,0){0.5}}
\put(411,767){\rule[-0.175pt]{0.350pt}{4.818pt}}
\put(557,158){\rule[-0.175pt]{0.350pt}{4.818pt}}
\put(557,113){\makebox(0,0){1}}
\put(557,767){\rule[-0.175pt]{0.350pt}{4.818pt}}
\put(704,158){\rule[-0.175pt]{0.350pt}{4.818pt}}
\put(704,113){\makebox(0,0){1.5}}
\put(704,767){\rule[-0.175pt]{0.350pt}{4.818pt}}
\put(850,158){\rule[-0.175pt]{0.350pt}{4.818pt}}
\put(850,113){\makebox(0,0){2}}
\put(850,767){\rule[-0.175pt]{0.350pt}{4.818pt}}
\put(997,158){\rule[-0.175pt]{0.350pt}{4.818pt}}
\put(997,113){\makebox(0,0){2.5}}
\put(997,767){\rule[-0.175pt]{0.350pt}{4.818pt}}
\put(1143,158){\rule[-0.175pt]{0.350pt}{4.818pt}}
\put(1143,113){\makebox(0,0){3}}
\put(1143,767){\rule[-0.175pt]{0.350pt}{4.818pt}}
\put(1290,158){\rule[-0.175pt]{0.350pt}{4.818pt}}
\put(1290,113){\makebox(0,0){3.5}}
\put(1290,767){\rule[-0.175pt]{0.350pt}{4.818pt}}
\put(1436,158){\rule[-0.175pt]{0.350pt}{4.818pt}}
\put(1436,113){\makebox(0,0){4}}
\put(1436,767){\rule[-0.175pt]{0.350pt}{4.818pt}}
\put(264,158){\rule[-0.175pt]{282.335pt}{0.350pt}}
\put(1436,158){\rule[-0.175pt]{0.350pt}{151.526pt}}
\put(264,787){\rule[-0.175pt]{282.335pt}{0.350pt}}
\put(264,158){\rule[-0.175pt]{0.350pt}{151.526pt}}
\put(264,315){\usebox{\plotpoint}}
\put(264,315){\rule[-0.175pt]{69.861pt}{0.350pt}}
\put(554,316){\usebox{\plotpoint}}
\put(555,317){\usebox{\plotpoint}}
\put(556,318){\usebox{\plotpoint}}
\put(557,319){\usebox{\plotpoint}}
\put(558,320){\rule[-0.175pt]{0.350pt}{0.683pt}}
\put(559,322){\rule[-0.175pt]{0.350pt}{0.683pt}}
\put(560,325){\rule[-0.175pt]{0.350pt}{0.683pt}}
\put(561,328){\rule[-0.175pt]{0.350pt}{0.683pt}}
\put(562,331){\rule[-0.175pt]{0.350pt}{0.683pt}}
\put(563,334){\rule[-0.175pt]{0.350pt}{0.683pt}}
\put(564,337){\rule[-0.175pt]{0.350pt}{0.642pt}}
\put(565,339){\rule[-0.175pt]{0.350pt}{0.642pt}}
\put(566,342){\rule[-0.175pt]{0.350pt}{0.642pt}}
\put(567,344){\rule[-0.175pt]{0.350pt}{0.642pt}}
\put(568,347){\rule[-0.175pt]{0.350pt}{0.642pt}}
\put(569,350){\rule[-0.175pt]{0.350pt}{0.642pt}}
\put(570,352){\rule[-0.175pt]{0.350pt}{0.642pt}}
\put(571,355){\rule[-0.175pt]{0.350pt}{0.642pt}}
\put(572,358){\rule[-0.175pt]{0.350pt}{0.642pt}}
\put(573,360){\rule[-0.175pt]{0.350pt}{0.642pt}}
\put(574,363){\rule[-0.175pt]{0.350pt}{0.642pt}}
\put(575,366){\rule[-0.175pt]{0.350pt}{0.642pt}}
\put(576,368){\rule[-0.175pt]{0.350pt}{0.562pt}}
\put(577,371){\rule[-0.175pt]{0.350pt}{0.562pt}}
\put(578,373){\rule[-0.175pt]{0.350pt}{0.562pt}}
\put(579,376){\rule[-0.175pt]{0.350pt}{0.562pt}}
\put(580,378){\rule[-0.175pt]{0.350pt}{0.562pt}}
\put(581,380){\rule[-0.175pt]{0.350pt}{0.562pt}}
\put(582,383){\rule[-0.175pt]{0.350pt}{0.562pt}}
\put(583,385){\rule[-0.175pt]{0.350pt}{0.562pt}}
\put(584,387){\rule[-0.175pt]{0.350pt}{0.562pt}}
\put(585,390){\rule[-0.175pt]{0.350pt}{0.562pt}}
\put(586,392){\rule[-0.175pt]{0.350pt}{0.562pt}}
\put(587,394){\rule[-0.175pt]{0.350pt}{0.562pt}}
\put(588,397){\rule[-0.175pt]{0.350pt}{0.522pt}}
\put(589,399){\rule[-0.175pt]{0.350pt}{0.522pt}}
\put(590,401){\rule[-0.175pt]{0.350pt}{0.522pt}}
\put(591,403){\rule[-0.175pt]{0.350pt}{0.522pt}}
\put(592,405){\rule[-0.175pt]{0.350pt}{0.522pt}}
\put(593,407){\rule[-0.175pt]{0.350pt}{0.522pt}}
\put(594,409){\rule[-0.175pt]{0.350pt}{0.482pt}}
\put(595,412){\rule[-0.175pt]{0.350pt}{0.482pt}}
\put(596,414){\rule[-0.175pt]{0.350pt}{0.482pt}}
\put(597,416){\rule[-0.175pt]{0.350pt}{0.482pt}}
\put(598,418){\rule[-0.175pt]{0.350pt}{0.482pt}}
\put(599,420){\rule[-0.175pt]{0.350pt}{0.482pt}}
\put(600,422){\rule[-0.175pt]{0.350pt}{0.482pt}}
\put(601,424){\rule[-0.175pt]{0.350pt}{0.482pt}}
\put(602,426){\rule[-0.175pt]{0.350pt}{0.482pt}}
\put(603,428){\rule[-0.175pt]{0.350pt}{0.482pt}}
\put(604,430){\rule[-0.175pt]{0.350pt}{0.482pt}}
\put(605,432){\rule[-0.175pt]{0.350pt}{0.482pt}}
\put(606,434){\rule[-0.175pt]{0.350pt}{0.530pt}}
\put(607,436){\rule[-0.175pt]{0.350pt}{0.530pt}}
\put(608,438){\rule[-0.175pt]{0.350pt}{0.530pt}}
\put(609,440){\rule[-0.175pt]{0.350pt}{0.530pt}}
\put(610,442){\rule[-0.175pt]{0.350pt}{0.530pt}}
\put(611,445){\rule[-0.175pt]{0.350pt}{0.401pt}}
\put(612,446){\rule[-0.175pt]{0.350pt}{0.401pt}}
\put(613,448){\rule[-0.175pt]{0.350pt}{0.401pt}}
\put(614,449){\rule[-0.175pt]{0.350pt}{0.401pt}}
\put(615,451){\rule[-0.175pt]{0.350pt}{0.401pt}}
\put(616,453){\rule[-0.175pt]{0.350pt}{0.401pt}}
\put(617,454){\rule[-0.175pt]{0.350pt}{0.402pt}}
\put(618,456){\rule[-0.175pt]{0.350pt}{0.401pt}}
\put(619,458){\rule[-0.175pt]{0.350pt}{0.401pt}}
\put(620,459){\rule[-0.175pt]{0.350pt}{0.401pt}}
\put(621,461){\rule[-0.175pt]{0.350pt}{0.401pt}}
\put(622,463){\rule[-0.175pt]{0.350pt}{0.401pt}}
\put(623,464){\rule[-0.175pt]{0.350pt}{0.361pt}}
\put(624,466){\rule[-0.175pt]{0.350pt}{0.361pt}}
\put(625,468){\rule[-0.175pt]{0.350pt}{0.361pt}}
\put(626,469){\rule[-0.175pt]{0.350pt}{0.361pt}}
\put(627,471){\rule[-0.175pt]{0.350pt}{0.361pt}}
\put(628,472){\rule[-0.175pt]{0.350pt}{0.361pt}}
\put(629,474){\rule[-0.175pt]{0.350pt}{0.361pt}}
\put(630,475){\rule[-0.175pt]{0.350pt}{0.361pt}}
\put(631,477){\rule[-0.175pt]{0.350pt}{0.361pt}}
\put(632,478){\rule[-0.175pt]{0.350pt}{0.361pt}}
\put(633,480){\rule[-0.175pt]{0.350pt}{0.361pt}}
\put(634,481){\rule[-0.175pt]{0.350pt}{0.361pt}}
\put(635,483){\usebox{\plotpoint}}
\put(636,484){\usebox{\plotpoint}}
\put(637,485){\usebox{\plotpoint}}
\put(638,487){\usebox{\plotpoint}}
\put(639,488){\usebox{\plotpoint}}
\put(640,489){\usebox{\plotpoint}}
\put(641,491){\usebox{\plotpoint}}
\put(642,492){\usebox{\plotpoint}}
\put(643,493){\usebox{\plotpoint}}
\put(644,495){\usebox{\plotpoint}}
\put(645,496){\usebox{\plotpoint}}
\put(646,497){\usebox{\plotpoint}}
\put(647,499){\usebox{\plotpoint}}
\put(648,500){\usebox{\plotpoint}}
\put(649,501){\usebox{\plotpoint}}
\put(650,502){\usebox{\plotpoint}}
\put(651,503){\usebox{\plotpoint}}
\put(652,504){\usebox{\plotpoint}}
\put(653,505){\usebox{\plotpoint}}
\put(654,507){\usebox{\plotpoint}}
\put(655,508){\usebox{\plotpoint}}
\put(656,509){\usebox{\plotpoint}}
\put(657,510){\usebox{\plotpoint}}
\put(658,511){\usebox{\plotpoint}}
\put(659,512){\usebox{\plotpoint}}
\put(660,514){\usebox{\plotpoint}}
\put(661,515){\usebox{\plotpoint}}
\put(662,516){\usebox{\plotpoint}}
\put(663,517){\usebox{\plotpoint}}
\put(664,519){\usebox{\plotpoint}}
\put(665,520){\usebox{\plotpoint}}
\put(666,521){\usebox{\plotpoint}}
\put(667,522){\usebox{\plotpoint}}
\put(668,523){\usebox{\plotpoint}}
\put(669,524){\usebox{\plotpoint}}
\put(670,526){\usebox{\plotpoint}}
\put(671,527){\usebox{\plotpoint}}
\put(672,528){\usebox{\plotpoint}}
\put(673,529){\usebox{\plotpoint}}
\put(674,530){\usebox{\plotpoint}}
\put(676,531){\usebox{\plotpoint}}
\put(677,532){\usebox{\plotpoint}}
\put(678,533){\usebox{\plotpoint}}
\put(679,534){\usebox{\plotpoint}}
\put(680,535){\usebox{\plotpoint}}
\put(681,536){\usebox{\plotpoint}}
\put(682,537){\usebox{\plotpoint}}
\put(683,538){\usebox{\plotpoint}}
\put(684,539){\usebox{\plotpoint}}
\put(685,540){\usebox{\plotpoint}}
\put(686,541){\usebox{\plotpoint}}
\put(688,542){\usebox{\plotpoint}}
\put(689,543){\usebox{\plotpoint}}
\put(690,544){\usebox{\plotpoint}}
\put(691,545){\usebox{\plotpoint}}
\put(692,546){\usebox{\plotpoint}}
\put(694,547){\usebox{\plotpoint}}
\put(695,548){\usebox{\plotpoint}}
\put(696,549){\usebox{\plotpoint}}
\put(697,550){\usebox{\plotpoint}}
\put(698,551){\usebox{\plotpoint}}
\put(700,552){\rule[-0.175pt]{0.361pt}{0.350pt}}
\put(701,553){\rule[-0.175pt]{0.361pt}{0.350pt}}
\put(703,554){\rule[-0.175pt]{0.361pt}{0.350pt}}
\put(704,555){\rule[-0.175pt]{0.361pt}{0.350pt}}
\put(706,556){\rule[-0.175pt]{0.361pt}{0.350pt}}
\put(707,557){\rule[-0.175pt]{0.361pt}{0.350pt}}
\put(709,558){\rule[-0.175pt]{0.361pt}{0.350pt}}
\put(710,559){\rule[-0.175pt]{0.361pt}{0.350pt}}
\put(712,560){\usebox{\plotpoint}}
\put(713,561){\usebox{\plotpoint}}
\put(714,562){\usebox{\plotpoint}}
\put(715,563){\usebox{\plotpoint}}
\put(717,564){\rule[-0.175pt]{0.361pt}{0.350pt}}
\put(718,565){\rule[-0.175pt]{0.361pt}{0.350pt}}
\put(720,566){\rule[-0.175pt]{0.361pt}{0.350pt}}
\put(721,567){\rule[-0.175pt]{0.361pt}{0.350pt}}
\put(723,568){\rule[-0.175pt]{0.482pt}{0.350pt}}
\put(725,569){\rule[-0.175pt]{0.482pt}{0.350pt}}
\put(727,570){\rule[-0.175pt]{0.482pt}{0.350pt}}
\put(729,571){\rule[-0.175pt]{0.482pt}{0.350pt}}
\put(731,572){\rule[-0.175pt]{0.482pt}{0.350pt}}
\put(733,573){\rule[-0.175pt]{0.482pt}{0.350pt}}
\put(735,574){\rule[-0.175pt]{0.482pt}{0.350pt}}
\put(737,575){\rule[-0.175pt]{0.482pt}{0.350pt}}
\put(739,576){\rule[-0.175pt]{0.482pt}{0.350pt}}
\put(741,577){\rule[-0.175pt]{0.482pt}{0.350pt}}
\put(743,578){\rule[-0.175pt]{0.482pt}{0.350pt}}
\put(745,579){\rule[-0.175pt]{0.482pt}{0.350pt}}
\put(747,580){\rule[-0.175pt]{0.482pt}{0.350pt}}
\put(749,581){\rule[-0.175pt]{0.482pt}{0.350pt}}
\put(751,582){\rule[-0.175pt]{0.482pt}{0.350pt}}
\put(753,583){\rule[-0.175pt]{0.482pt}{0.350pt}}
\put(755,584){\rule[-0.175pt]{0.482pt}{0.350pt}}
\put(757,585){\rule[-0.175pt]{0.482pt}{0.350pt}}
\put(759,586){\rule[-0.175pt]{0.723pt}{0.350pt}}
\put(762,587){\rule[-0.175pt]{0.723pt}{0.350pt}}
\put(765,588){\rule[-0.175pt]{0.402pt}{0.350pt}}
\put(766,589){\rule[-0.175pt]{0.402pt}{0.350pt}}
\put(768,590){\rule[-0.175pt]{0.401pt}{0.350pt}}
\put(770,591){\rule[-0.175pt]{0.723pt}{0.350pt}}
\put(773,592){\rule[-0.175pt]{0.723pt}{0.350pt}}
\put(776,593){\rule[-0.175pt]{0.723pt}{0.350pt}}
\put(779,594){\rule[-0.175pt]{0.723pt}{0.350pt}}
\put(782,595){\rule[-0.175pt]{0.723pt}{0.350pt}}
\put(785,596){\rule[-0.175pt]{0.723pt}{0.350pt}}
\put(788,597){\rule[-0.175pt]{0.723pt}{0.350pt}}
\put(791,598){\rule[-0.175pt]{0.723pt}{0.350pt}}
\put(794,599){\rule[-0.175pt]{1.445pt}{0.350pt}}
\put(800,600){\rule[-0.175pt]{0.723pt}{0.350pt}}
\put(803,601){\rule[-0.175pt]{0.723pt}{0.350pt}}
\put(806,602){\rule[-0.175pt]{0.723pt}{0.350pt}}
\put(809,603){\rule[-0.175pt]{0.723pt}{0.350pt}}
\put(812,604){\rule[-0.175pt]{1.445pt}{0.350pt}}
\put(818,605){\rule[-0.175pt]{1.204pt}{0.350pt}}
\put(823,606){\rule[-0.175pt]{0.723pt}{0.350pt}}
\put(826,607){\rule[-0.175pt]{0.723pt}{0.350pt}}
\put(829,608){\rule[-0.175pt]{1.445pt}{0.350pt}}
\put(835,609){\rule[-0.175pt]{1.445pt}{0.350pt}}
\put(841,610){\rule[-0.175pt]{1.445pt}{0.350pt}}
\put(847,611){\rule[-0.175pt]{0.723pt}{0.350pt}}
\put(850,610){\rule[-0.175pt]{0.723pt}{0.350pt}}
\put(853,609){\usebox{\plotpoint}}
\put(854,608){\usebox{\plotpoint}}
\put(855,607){\usebox{\plotpoint}}
\put(856,606){\usebox{\plotpoint}}
\put(857,605){\usebox{\plotpoint}}
\put(859,604){\usebox{\plotpoint}}
\put(860,603){\usebox{\plotpoint}}
\put(861,602){\usebox{\plotpoint}}
\put(862,601){\usebox{\plotpoint}}
\put(863,600){\usebox{\plotpoint}}
\put(865,599){\usebox{\plotpoint}}
\put(866,598){\usebox{\plotpoint}}
\put(867,597){\usebox{\plotpoint}}
\put(868,596){\usebox{\plotpoint}}
\put(869,595){\usebox{\plotpoint}}
\put(871,594){\usebox{\plotpoint}}
\put(872,593){\usebox{\plotpoint}}
\put(873,592){\usebox{\plotpoint}}
\put(874,591){\usebox{\plotpoint}}
\put(875,590){\usebox{\plotpoint}}
\put(877,589){\usebox{\plotpoint}}
\put(878,588){\usebox{\plotpoint}}
\put(879,587){\usebox{\plotpoint}}
\put(880,586){\usebox{\plotpoint}}
\put(881,585){\usebox{\plotpoint}}
\put(882,584){\usebox{\plotpoint}}
\put(883,583){\usebox{\plotpoint}}
\put(884,582){\usebox{\plotpoint}}
\put(885,581){\usebox{\plotpoint}}
\put(886,580){\usebox{\plotpoint}}
\put(888,579){\usebox{\plotpoint}}
\put(889,578){\usebox{\plotpoint}}
\put(890,577){\usebox{\plotpoint}}
\put(891,576){\usebox{\plotpoint}}
\put(892,575){\usebox{\plotpoint}}
\put(894,574){\usebox{\plotpoint}}
\put(895,573){\usebox{\plotpoint}}
\put(896,572){\usebox{\plotpoint}}
\put(897,571){\usebox{\plotpoint}}
\put(898,570){\usebox{\plotpoint}}
\put(900,569){\rule[-0.175pt]{0.361pt}{0.350pt}}
\put(901,568){\rule[-0.175pt]{0.361pt}{0.350pt}}
\put(903,567){\rule[-0.175pt]{0.361pt}{0.350pt}}
\put(904,566){\rule[-0.175pt]{0.361pt}{0.350pt}}
\put(906,565){\usebox{\plotpoint}}
\put(907,564){\usebox{\plotpoint}}
\put(908,563){\usebox{\plotpoint}}
\put(909,562){\usebox{\plotpoint}}
\put(910,561){\usebox{\plotpoint}}
\put(912,560){\rule[-0.175pt]{0.361pt}{0.350pt}}
\put(913,559){\rule[-0.175pt]{0.361pt}{0.350pt}}
\put(915,558){\rule[-0.175pt]{0.361pt}{0.350pt}}
\put(916,557){\rule[-0.175pt]{0.361pt}{0.350pt}}
\put(918,556){\usebox{\plotpoint}}
\put(919,555){\usebox{\plotpoint}}
\put(920,554){\usebox{\plotpoint}}
\put(921,553){\usebox{\plotpoint}}
\put(922,552){\usebox{\plotpoint}}
\put(924,551){\rule[-0.175pt]{0.361pt}{0.350pt}}
\put(925,550){\rule[-0.175pt]{0.361pt}{0.350pt}}
\put(927,549){\rule[-0.175pt]{0.361pt}{0.350pt}}
\put(928,548){\rule[-0.175pt]{0.361pt}{0.350pt}}
\put(930,547){\usebox{\plotpoint}}
\put(931,546){\usebox{\plotpoint}}
\put(932,545){\usebox{\plotpoint}}
\put(933,544){\usebox{\plotpoint}}
\put(934,543){\usebox{\plotpoint}}
\put(935,542){\rule[-0.175pt]{0.361pt}{0.350pt}}
\put(936,541){\rule[-0.175pt]{0.361pt}{0.350pt}}
\put(938,540){\rule[-0.175pt]{0.361pt}{0.350pt}}
\put(939,539){\rule[-0.175pt]{0.361pt}{0.350pt}}
\put(941,538){\rule[-0.175pt]{0.361pt}{0.350pt}}
\put(942,537){\rule[-0.175pt]{0.361pt}{0.350pt}}
\put(944,536){\rule[-0.175pt]{0.361pt}{0.350pt}}
\put(945,535){\rule[-0.175pt]{0.361pt}{0.350pt}}
\put(947,534){\rule[-0.175pt]{0.361pt}{0.350pt}}
\put(948,533){\rule[-0.175pt]{0.361pt}{0.350pt}}
\put(950,532){\rule[-0.175pt]{0.361pt}{0.350pt}}
\put(951,531){\rule[-0.175pt]{0.361pt}{0.350pt}}
\put(953,530){\rule[-0.175pt]{0.361pt}{0.350pt}}
\put(954,529){\rule[-0.175pt]{0.361pt}{0.350pt}}
\put(956,528){\rule[-0.175pt]{0.361pt}{0.350pt}}
\put(957,527){\rule[-0.175pt]{0.361pt}{0.350pt}}
\put(959,526){\rule[-0.175pt]{0.361pt}{0.350pt}}
\put(960,525){\rule[-0.175pt]{0.361pt}{0.350pt}}
\put(962,524){\rule[-0.175pt]{0.361pt}{0.350pt}}
\put(963,523){\rule[-0.175pt]{0.361pt}{0.350pt}}
\put(965,522){\rule[-0.175pt]{0.361pt}{0.350pt}}
\put(966,521){\rule[-0.175pt]{0.361pt}{0.350pt}}
\put(968,520){\rule[-0.175pt]{0.361pt}{0.350pt}}
\put(969,519){\rule[-0.175pt]{0.361pt}{0.350pt}}
\put(971,518){\rule[-0.175pt]{0.361pt}{0.350pt}}
\put(972,517){\rule[-0.175pt]{0.361pt}{0.350pt}}
\put(974,516){\rule[-0.175pt]{0.361pt}{0.350pt}}
\put(975,515){\rule[-0.175pt]{0.361pt}{0.350pt}}
\put(977,514){\rule[-0.175pt]{0.361pt}{0.350pt}}
\put(978,513){\rule[-0.175pt]{0.361pt}{0.350pt}}
\put(980,512){\rule[-0.175pt]{0.361pt}{0.350pt}}
\put(981,511){\rule[-0.175pt]{0.361pt}{0.350pt}}
\put(983,510){\rule[-0.175pt]{0.402pt}{0.350pt}}
\put(984,509){\rule[-0.175pt]{0.402pt}{0.350pt}}
\put(986,508){\rule[-0.175pt]{0.401pt}{0.350pt}}
\put(988,507){\rule[-0.175pt]{0.361pt}{0.350pt}}
\put(989,506){\rule[-0.175pt]{0.361pt}{0.350pt}}
\put(991,505){\rule[-0.175pt]{0.361pt}{0.350pt}}
\put(992,504){\rule[-0.175pt]{0.361pt}{0.350pt}}
\put(994,503){\rule[-0.175pt]{0.361pt}{0.350pt}}
\put(995,502){\rule[-0.175pt]{0.361pt}{0.350pt}}
\put(997,501){\rule[-0.175pt]{0.361pt}{0.350pt}}
\put(998,500){\rule[-0.175pt]{0.361pt}{0.350pt}}
\put(1000,499){\rule[-0.175pt]{0.482pt}{0.350pt}}
\put(1002,498){\rule[-0.175pt]{0.482pt}{0.350pt}}
\put(1004,497){\rule[-0.175pt]{0.482pt}{0.350pt}}
\put(1006,496){\rule[-0.175pt]{0.361pt}{0.350pt}}
\put(1007,495){\rule[-0.175pt]{0.361pt}{0.350pt}}
\put(1009,494){\rule[-0.175pt]{0.361pt}{0.350pt}}
\put(1010,493){\rule[-0.175pt]{0.361pt}{0.350pt}}
\put(1012,492){\rule[-0.175pt]{0.482pt}{0.350pt}}
\put(1014,491){\rule[-0.175pt]{0.482pt}{0.350pt}}
\put(1016,490){\rule[-0.175pt]{0.482pt}{0.350pt}}
\put(1018,489){\rule[-0.175pt]{0.482pt}{0.350pt}}
\put(1020,488){\rule[-0.175pt]{0.482pt}{0.350pt}}
\put(1022,487){\rule[-0.175pt]{0.482pt}{0.350pt}}
\put(1024,486){\rule[-0.175pt]{0.361pt}{0.350pt}}
\put(1025,485){\rule[-0.175pt]{0.361pt}{0.350pt}}
\put(1027,484){\rule[-0.175pt]{0.361pt}{0.350pt}}
\put(1028,483){\rule[-0.175pt]{0.361pt}{0.350pt}}
\put(1030,482){\rule[-0.175pt]{0.482pt}{0.350pt}}
\put(1032,481){\rule[-0.175pt]{0.482pt}{0.350pt}}
\put(1034,480){\rule[-0.175pt]{0.482pt}{0.350pt}}
\put(1036,479){\rule[-0.175pt]{0.401pt}{0.350pt}}
\put(1037,478){\rule[-0.175pt]{0.401pt}{0.350pt}}
\put(1039,477){\rule[-0.175pt]{0.401pt}{0.350pt}}
\put(1040,476){\rule[-0.175pt]{0.482pt}{0.350pt}}
\put(1043,475){\rule[-0.175pt]{0.482pt}{0.350pt}}
\put(1045,474){\rule[-0.175pt]{0.482pt}{0.350pt}}
\put(1047,473){\rule[-0.175pt]{0.482pt}{0.350pt}}
\put(1049,472){\rule[-0.175pt]{0.482pt}{0.350pt}}
\put(1051,471){\rule[-0.175pt]{0.482pt}{0.350pt}}
\put(1053,470){\rule[-0.175pt]{0.482pt}{0.350pt}}
\put(1055,469){\rule[-0.175pt]{0.482pt}{0.350pt}}
\put(1057,468){\rule[-0.175pt]{0.482pt}{0.350pt}}
\put(1059,467){\rule[-0.175pt]{0.482pt}{0.350pt}}
\put(1061,466){\rule[-0.175pt]{0.482pt}{0.350pt}}
\put(1063,465){\rule[-0.175pt]{0.482pt}{0.350pt}}
\put(1065,464){\rule[-0.175pt]{0.482pt}{0.350pt}}
\put(1067,463){\rule[-0.175pt]{0.482pt}{0.350pt}}
\put(1069,462){\rule[-0.175pt]{0.482pt}{0.350pt}}
\put(1071,461){\rule[-0.175pt]{0.482pt}{0.350pt}}
\put(1073,460){\rule[-0.175pt]{0.482pt}{0.350pt}}
\put(1075,459){\rule[-0.175pt]{0.482pt}{0.350pt}}
\put(1077,458){\rule[-0.175pt]{0.723pt}{0.350pt}}
\put(1080,457){\rule[-0.175pt]{0.723pt}{0.350pt}}
\put(1083,456){\rule[-0.175pt]{0.482pt}{0.350pt}}
\put(1085,455){\rule[-0.175pt]{0.482pt}{0.350pt}}
\put(1087,454){\rule[-0.175pt]{0.482pt}{0.350pt}}
\put(1089,453){\rule[-0.175pt]{0.401pt}{0.350pt}}
\put(1090,452){\rule[-0.175pt]{0.401pt}{0.350pt}}
\put(1092,451){\rule[-0.175pt]{0.401pt}{0.350pt}}
\put(1093,450){\rule[-0.175pt]{0.723pt}{0.350pt}}
\put(1097,449){\rule[-0.175pt]{0.723pt}{0.350pt}}
\put(1100,448){\rule[-0.175pt]{0.482pt}{0.350pt}}
\put(1102,447){\rule[-0.175pt]{0.482pt}{0.350pt}}
\put(1104,446){\rule[-0.175pt]{0.482pt}{0.350pt}}
\put(1106,445){\rule[-0.175pt]{0.482pt}{0.350pt}}
\put(1108,444){\rule[-0.175pt]{0.482pt}{0.350pt}}
\put(1110,443){\rule[-0.175pt]{0.482pt}{0.350pt}}
\put(1112,442){\rule[-0.175pt]{0.723pt}{0.350pt}}
\put(1115,441){\rule[-0.175pt]{0.723pt}{0.350pt}}
\put(1118,440){\rule[-0.175pt]{0.723pt}{0.350pt}}
\put(1121,439){\rule[-0.175pt]{0.723pt}{0.350pt}}
\put(1124,438){\rule[-0.175pt]{0.482pt}{0.350pt}}
\put(1126,437){\rule[-0.175pt]{0.482pt}{0.350pt}}
\put(1128,436){\rule[-0.175pt]{0.482pt}{0.350pt}}
\put(1130,435){\rule[-0.175pt]{0.723pt}{0.350pt}}
\put(1133,434){\rule[-0.175pt]{0.723pt}{0.350pt}}
\put(1136,433){\rule[-0.175pt]{0.482pt}{0.350pt}}
\put(1138,432){\rule[-0.175pt]{0.482pt}{0.350pt}}
\put(1140,431){\rule[-0.175pt]{0.482pt}{0.350pt}}
\put(1142,430){\rule[-0.175pt]{0.602pt}{0.350pt}}
\put(1144,429){\rule[-0.175pt]{0.602pt}{0.350pt}}
\put(1147,428){\rule[-0.175pt]{0.723pt}{0.350pt}}
\put(1150,427){\rule[-0.175pt]{0.723pt}{0.350pt}}
\put(1153,426){\rule[-0.175pt]{0.723pt}{0.350pt}}
\put(1156,425){\rule[-0.175pt]{0.723pt}{0.350pt}}
\put(1159,424){\rule[-0.175pt]{0.723pt}{0.350pt}}
\put(1162,423){\rule[-0.175pt]{0.723pt}{0.350pt}}
\put(1165,422){\rule[-0.175pt]{0.723pt}{0.350pt}}
\put(1168,421){\rule[-0.175pt]{0.723pt}{0.350pt}}
\put(1171,420){\rule[-0.175pt]{0.723pt}{0.350pt}}
\put(1174,419){\rule[-0.175pt]{0.723pt}{0.350pt}}
\put(1177,418){\rule[-0.175pt]{0.723pt}{0.350pt}}
\put(1180,417){\rule[-0.175pt]{0.723pt}{0.350pt}}
\put(1183,416){\rule[-0.175pt]{0.723pt}{0.350pt}}
\put(1186,415){\rule[-0.175pt]{0.723pt}{0.350pt}}
\put(1189,414){\rule[-0.175pt]{0.723pt}{0.350pt}}
\put(1192,413){\rule[-0.175pt]{0.723pt}{0.350pt}}
\put(1195,412){\rule[-0.175pt]{0.602pt}{0.350pt}}
\put(1197,411){\rule[-0.175pt]{0.602pt}{0.350pt}}
\put(1200,410){\rule[-0.175pt]{0.723pt}{0.350pt}}
\put(1203,409){\rule[-0.175pt]{0.723pt}{0.350pt}}
\put(1206,408){\rule[-0.175pt]{0.723pt}{0.350pt}}
\put(1209,407){\rule[-0.175pt]{0.723pt}{0.350pt}}
\put(1212,406){\rule[-0.175pt]{0.723pt}{0.350pt}}
\put(1215,405){\rule[-0.175pt]{0.723pt}{0.350pt}}
\put(1218,404){\rule[-0.175pt]{1.445pt}{0.350pt}}
\put(1224,403){\rule[-0.175pt]{0.723pt}{0.350pt}}
\put(1227,402){\rule[-0.175pt]{0.723pt}{0.350pt}}
\put(1230,401){\rule[-0.175pt]{0.723pt}{0.350pt}}
\put(1233,400){\rule[-0.175pt]{0.723pt}{0.350pt}}
\put(1236,399){\rule[-0.175pt]{0.723pt}{0.350pt}}
\put(1239,398){\rule[-0.175pt]{0.723pt}{0.350pt}}
\put(1242,397){\rule[-0.175pt]{1.445pt}{0.350pt}}
\put(1248,396){\rule[-0.175pt]{0.602pt}{0.350pt}}
\put(1250,395){\rule[-0.175pt]{0.602pt}{0.350pt}}
\put(1253,394){\rule[-0.175pt]{1.445pt}{0.350pt}}
\put(1259,393){\rule[-0.175pt]{0.723pt}{0.350pt}}
\put(1262,392){\rule[-0.175pt]{0.723pt}{0.350pt}}
\put(1265,391){\rule[-0.175pt]{1.445pt}{0.350pt}}
\put(1271,390){\rule[-0.175pt]{0.723pt}{0.350pt}}
\put(1274,389){\rule[-0.175pt]{0.723pt}{0.350pt}}
\put(1277,388){\rule[-0.175pt]{1.445pt}{0.350pt}}
\put(1283,387){\rule[-0.175pt]{0.723pt}{0.350pt}}
\put(1286,386){\rule[-0.175pt]{0.723pt}{0.350pt}}
\put(1289,385){\rule[-0.175pt]{1.445pt}{0.350pt}}
\put(1295,384){\rule[-0.175pt]{1.445pt}{0.350pt}}
\put(1301,383){\rule[-0.175pt]{0.602pt}{0.350pt}}
\put(1303,382){\rule[-0.175pt]{0.602pt}{0.350pt}}
\put(1306,381){\rule[-0.175pt]{1.445pt}{0.350pt}}
\put(1312,380){\rule[-0.175pt]{1.445pt}{0.350pt}}
\put(1318,379){\rule[-0.175pt]{0.723pt}{0.350pt}}
\put(1321,378){\rule[-0.175pt]{0.723pt}{0.350pt}}
\put(1324,377){\rule[-0.175pt]{1.445pt}{0.350pt}}
\put(1330,376){\rule[-0.175pt]{1.445pt}{0.350pt}}
\put(1336,375){\rule[-0.175pt]{1.445pt}{0.350pt}}
\put(1342,374){\rule[-0.175pt]{1.445pt}{0.350pt}}
\put(1348,373){\rule[-0.175pt]{1.445pt}{0.350pt}}
\put(1354,372){\rule[-0.175pt]{0.602pt}{0.350pt}}
\put(1356,371){\rule[-0.175pt]{0.602pt}{0.350pt}}
\put(1359,370){\rule[-0.175pt]{1.445pt}{0.350pt}}
\put(1365,369){\rule[-0.175pt]{1.445pt}{0.350pt}}
\put(1371,368){\rule[-0.175pt]{1.445pt}{0.350pt}}
\put(1377,367){\rule[-0.175pt]{1.445pt}{0.350pt}}
\put(1383,366){\rule[-0.175pt]{1.445pt}{0.350pt}}
\put(1389,365){\rule[-0.175pt]{1.445pt}{0.350pt}}
\put(1395,364){\rule[-0.175pt]{1.445pt}{0.350pt}}
\put(1401,363){\rule[-0.175pt]{1.445pt}{0.350pt}}
\put(1407,362){\rule[-0.175pt]{1.204pt}{0.350pt}}
\put(1412,361){\rule[-0.175pt]{1.445pt}{0.350pt}}
\put(1418,360){\rule[-0.175pt]{1.445pt}{0.350pt}}
\put(1424,359){\rule[-0.175pt]{2.891pt}{0.350pt}}
\sbox{\plotpoint}{\rule[-0.250pt]{0.500pt}{0.500pt}}%
\put(264,315){\usebox{\plotpoint}}
\put(264,315){\usebox{\plotpoint}}
\put(284,315){\usebox{\plotpoint}}
\put(305,315){\usebox{\plotpoint}}
\put(326,315){\usebox{\plotpoint}}
\put(347,315){\usebox{\plotpoint}}
\put(367,315){\usebox{\plotpoint}}
\put(388,315){\usebox{\plotpoint}}
\put(409,315){\usebox{\plotpoint}}
\put(430,315){\usebox{\plotpoint}}
\put(450,315){\usebox{\plotpoint}}
\put(471,315){\usebox{\plotpoint}}
\put(492,315){\usebox{\plotpoint}}
\put(513,315){\usebox{\plotpoint}}
\put(533,315){\usebox{\plotpoint}}
\put(554,316){\usebox{\plotpoint}}
\put(563,334){\usebox{\plotpoint}}
\put(570,353){\usebox{\plotpoint}}
\put(577,373){\usebox{\plotpoint}}
\put(585,392){\usebox{\plotpoint}}
\put(594,411){\usebox{\plotpoint}}
\put(603,429){\usebox{\plotpoint}}
\put(613,448){\usebox{\plotpoint}}
\put(623,466){\usebox{\plotpoint}}
\put(635,483){\usebox{\plotpoint}}
\put(647,499){\usebox{\plotpoint}}
\put(661,515){\usebox{\plotpoint}}
\put(675,530){\usebox{\plotpoint}}
\put(690,544){\usebox{\plotpoint}}
\put(707,557){\usebox{\plotpoint}}
\put(724,568){\usebox{\plotpoint}}
\put(743,578){\usebox{\plotpoint}}
\put(761,586){\usebox{\plotpoint}}
\put(781,594){\usebox{\plotpoint}}
\put(800,600){\usebox{\plotpoint}}
\put(820,605){\usebox{\plotpoint}}
\put(841,610){\usebox{\plotpoint}}
\put(861,613){\usebox{\plotpoint}}
\put(882,617){\usebox{\plotpoint}}
\put(902,619){\usebox{\plotpoint}}
\put(923,620){\usebox{\plotpoint}}
\put(943,622){\usebox{\plotpoint}}
\put(964,624){\usebox{\plotpoint}}
\put(985,625){\usebox{\plotpoint}}
\put(1005,626){\usebox{\plotpoint}}
\put(1026,626){\usebox{\plotpoint}}
\put(1047,627){\usebox{\plotpoint}}
\put(1067,628){\usebox{\plotpoint}}
\put(1088,628){\usebox{\plotpoint}}
\put(1109,628){\usebox{\plotpoint}}
\put(1130,629){\usebox{\plotpoint}}
\put(1150,629){\usebox{\plotpoint}}
\put(1171,629){\usebox{\plotpoint}}
\put(1192,629){\usebox{\plotpoint}}
\put(1213,629){\usebox{\plotpoint}}
\put(1233,629){\usebox{\plotpoint}}
\put(1254,629){\usebox{\plotpoint}}
\put(1275,629){\usebox{\plotpoint}}
\put(1296,630){\usebox{\plotpoint}}
\put(1316,630){\usebox{\plotpoint}}
\put(1337,630){\usebox{\plotpoint}}
\put(1358,630){\usebox{\plotpoint}}
\put(1379,630){\usebox{\plotpoint}}
\put(1399,630){\usebox{\plotpoint}}
\put(1420,630){\usebox{\plotpoint}}
\put(1436,630){\usebox{\plotpoint}}
\put(264,315){\usebox{\plotpoint}}
\put(264,315){\usebox{\plotpoint}}
\put(284,315){\usebox{\plotpoint}}
\put(305,315){\usebox{\plotpoint}}
\put(326,315){\usebox{\plotpoint}}
\put(347,315){\usebox{\plotpoint}}
\put(367,315){\usebox{\plotpoint}}
\put(388,315){\usebox{\plotpoint}}
\put(409,315){\usebox{\plotpoint}}
\put(430,315){\usebox{\plotpoint}}
\put(450,315){\usebox{\plotpoint}}
\put(471,315){\usebox{\plotpoint}}
\put(492,315){\usebox{\plotpoint}}
\put(513,315){\usebox{\plotpoint}}
\put(533,315){\usebox{\plotpoint}}
\put(554,315){\usebox{\plotpoint}}
\put(575,315){\usebox{\plotpoint}}
\put(596,315){\usebox{\plotpoint}}
\put(616,315){\usebox{\plotpoint}}
\put(637,315){\usebox{\plotpoint}}
\put(658,315){\usebox{\plotpoint}}
\put(679,315){\usebox{\plotpoint}}
\put(699,315){\usebox{\plotpoint}}
\put(720,315){\usebox{\plotpoint}}
\put(741,315){\usebox{\plotpoint}}
\put(762,315){\usebox{\plotpoint}}
\put(782,315){\usebox{\plotpoint}}
\put(803,315){\usebox{\plotpoint}}
\put(824,315){\usebox{\plotpoint}}
\put(845,315){\usebox{\plotpoint}}
\put(865,315){\usebox{\plotpoint}}
\put(886,315){\usebox{\plotpoint}}
\put(907,315){\usebox{\plotpoint}}
\put(928,315){\usebox{\plotpoint}}
\put(948,315){\usebox{\plotpoint}}
\put(969,315){\usebox{\plotpoint}}
\put(990,315){\usebox{\plotpoint}}
\put(1011,315){\usebox{\plotpoint}}
\put(1031,315){\usebox{\plotpoint}}
\put(1052,315){\usebox{\plotpoint}}
\put(1073,315){\usebox{\plotpoint}}
\put(1094,315){\usebox{\plotpoint}}
\put(1114,315){\usebox{\plotpoint}}
\put(1135,315){\usebox{\plotpoint}}
\put(1156,315){\usebox{\plotpoint}}
\put(1177,315){\usebox{\plotpoint}}
\put(1197,315){\usebox{\plotpoint}}
\put(1218,315){\usebox{\plotpoint}}
\put(1239,315){\usebox{\plotpoint}}
\put(1260,315){\usebox{\plotpoint}}
\put(1281,315){\usebox{\plotpoint}}
\put(1301,315){\usebox{\plotpoint}}
\put(1322,315){\usebox{\plotpoint}}
\put(1343,315){\usebox{\plotpoint}}
\put(1364,315){\usebox{\plotpoint}}
\put(1384,315){\usebox{\plotpoint}}
\put(1405,315){\usebox{\plotpoint}}
\put(1426,315){\usebox{\plotpoint}}
\put(1436,315){\usebox{\plotpoint}}
\put(264,158){\usebox{\plotpoint}}
\put(264,158){\usebox{\plotpoint}}
\put(284,158){\usebox{\plotpoint}}
\put(305,158){\usebox{\plotpoint}}
\put(326,158){\usebox{\plotpoint}}
\put(347,158){\usebox{\plotpoint}}
\put(367,158){\usebox{\plotpoint}}
\put(388,158){\usebox{\plotpoint}}
\put(409,158){\usebox{\plotpoint}}
\put(430,158){\usebox{\plotpoint}}
\put(450,158){\usebox{\plotpoint}}
\put(471,158){\usebox{\plotpoint}}
\put(492,158){\usebox{\plotpoint}}
\put(513,158){\usebox{\plotpoint}}
\put(533,158){\usebox{\plotpoint}}
\put(554,158){\usebox{\plotpoint}}
\put(575,158){\usebox{\plotpoint}}
\put(596,158){\usebox{\plotpoint}}
\put(616,158){\usebox{\plotpoint}}
\put(637,158){\usebox{\plotpoint}}
\put(658,158){\usebox{\plotpoint}}
\put(679,158){\usebox{\plotpoint}}
\put(699,158){\usebox{\plotpoint}}
\put(720,158){\usebox{\plotpoint}}
\put(741,158){\usebox{\plotpoint}}
\put(762,158){\usebox{\plotpoint}}
\put(782,158){\usebox{\plotpoint}}
\put(803,158){\usebox{\plotpoint}}
\put(824,158){\usebox{\plotpoint}}
\put(845,158){\usebox{\plotpoint}}
\put(865,158){\usebox{\plotpoint}}
\put(886,158){\usebox{\plotpoint}}
\put(907,158){\usebox{\plotpoint}}
\put(928,158){\usebox{\plotpoint}}
\put(948,158){\usebox{\plotpoint}}
\put(969,158){\usebox{\plotpoint}}
\put(990,158){\usebox{\plotpoint}}
\put(1011,158){\usebox{\plotpoint}}
\put(1031,158){\usebox{\plotpoint}}
\put(1052,158){\usebox{\plotpoint}}
\put(1073,158){\usebox{\plotpoint}}
\put(1094,158){\usebox{\plotpoint}}
\put(1114,158){\usebox{\plotpoint}}
\put(1135,158){\usebox{\plotpoint}}
\put(1156,158){\usebox{\plotpoint}}
\put(1177,158){\usebox{\plotpoint}}
\put(1197,158){\usebox{\plotpoint}}
\put(1218,158){\usebox{\plotpoint}}
\put(1239,158){\usebox{\plotpoint}}
\put(1260,158){\usebox{\plotpoint}}
\put(1281,158){\usebox{\plotpoint}}
\put(1301,158){\usebox{\plotpoint}}
\put(1322,158){\usebox{\plotpoint}}
\put(1343,158){\usebox{\plotpoint}}
\put(1364,158){\usebox{\plotpoint}}
\put(1384,158){\usebox{\plotpoint}}
\put(1405,158){\usebox{\plotpoint}}
\put(1426,158){\usebox{\plotpoint}}
\put(1436,158){\usebox{\plotpoint}}
\put(300,630){$I_2$}
\put(300,335){$I_1$}
\put(585,500){$\tau_1$}
\put(1080,500){$\tau_2$}
\put(545,195){$T_{D1}$}
\put(558,248){\rule{1pt}{10pt}}
\put(835,195){$T_{D2}$}
\put(847,248){\rule{1pt}{10pt}}
%\put(258,165){$0$}
\put(1200,145){$t$}
%\put(264,208){\rule{1pt}{20pt}}
\put(380,630){\rule[-0.175pt]{25pt}{0.683pt}}
\put(100,480){$v$}
\put(90,700){(V)}
\put(820,40){$t$}
\put(1300,40){(s)}
\end{picture}

\caption[Voltage source exponential ({\tt EXP}) waveform]{Voltage source exponential
({\tt EXP}) waveform for {\tt EXP(0.1 0.8 1 0.35 2 1)} \label{fig:vexp} }
\end{figure}

\noindent\underline{\bf Single-Frequency FM:}
\\[0.2in]
\form{{\tt SFFM(} $V_O$ $V_A$ $F_C$ $\mu$ $F_S$ {\tt )}}

\keywordtable{
$V_O$&offset voltage     & A  &        \X
$V_A$& peak amplitude of \ac\ voltage& A  &        \X
$F_C$&carrier frequency  & Hz &1/{\tt TSTOP} \X
$\mu$&modulation index   & -  & 0      \X
$F_S$&signal frequency   & Hz &1/{\tt TSTOP} \X
}

The single frequency frequency modulated transient response is described by
\begin{equation}
v = V_O + V_A\sin{(2 \pi \, F_C t +  \mu\sin{(2 \pi F_S t)})}
\end{equation}
\begin{figure}[hbp]
\centering
\documentclass{article}
\usepackage{epsf}\usepackage{here}
% SUMMARY OF USEFUL MACROS
%
% 1. \marginpar[LeftText]{RightText} Standard Latex \marginpar
%
% 2. \mymarginpar{MarginText}{BodyText} Places MarginText in margin and
%     BodyText in main text opposite margin. Places lineas above and below
%     text.  Used in element and model catalogs and elsewhere.
%
% 3. \marginlabel{text} Places large text in margin and underlines. Used
%    to put element name in margin at top of page for continued
%    descriptions.  Does not automaticly put it at the top of a page
%    and so a \clearpage is required.
%
% 4. \marginid{text} Used to place a short identifier in the margin. Just one
%    word and justified to the outside edge.
%
% 5. \offset a standard indent.
%
% 6. \offsetparbox{} Places text in an offset parbox.  Use
%    \hspace*{\fill} \offsetparbox{text}        to insert an offset text.
%
% 7. \boxed{} similar to above but starts a new line and right justifies
%    offsetparbox.
%
% 8. \form
%
% 9. \example
%
%10. \begin{widelist}  .... \end{widelist} 
%    A widelist that has 1 inch wide labels that has additional left hand
%    margin of 0.6 inch.
%
%11. \spicethreeonly{} Text is inserted only if output is for spice3.
%
%12. \pspiceonly{} 
%    Text is inserted only if output is for pspice (superspice for the most
%    part is upwards compatible to pspice).
%
%13. \sspiceonly{} 
%    Text is inserted only if output is for superspice but not for pspice.

%14. \sym{}
%    Right justifies a symbol in a keyword table.
%
%15. Use the following wherever as then we can have a standard way to
%    report them.  Note that "\dc\ " is required to get a space after dc.
%    \dc
%    \ac
%    \SPICE
%
%16. \kwnote{}  This is a convenient way to include notes in a keyword table.
%    It right justifies a note in the description column and starts it on a
%    newline.
\newcommand{\kwnote}[1]{\newline\hspace*{\fill} #1}

%
%17. \kwversion{} This is a convenient way to indicate versions in a keyword
%    table. It does a \kwnote .  It should not printed when outputing for
%     specific versions.
% typical usage is \kwversion{\sspice ; \hspice} to produce the output
%            VERSIONS: SUPERSPICE; HSPICE
% typical usage is \kwversionNote{\sspice ; \hspice} to produce the output
%                 |                    VERSIONS: SUPERSPICE; HSPICE |
%for unifomity the recommended version order is:
%    \hspice \pspice \sspice \spicetwo \spicethree
\newcommand{\version}[1]{({\sc version: #1})}
\newcommand{\kwversion}[1]{\newline({\sc version: #1})}
\newcommand{\kwversionNote}[1]{\kwnote{({\sc version: #1})}}
\newcommand{\versions}[1]{({\sc versions: #1})}
\newcommand{\kwversions}[1]{\newline({\sc versions: #1})}
\newcommand{\kwversionsNote}[1]{\kwnote{({\sc versions: #1})}}

%


\newcommand{\mymargin}{1in}   % Use to set width of margin notes
\newcommand{\mymarginparsep}{0.1in}
\newcommand{\mymarginparsepplustext}{5.6in}
\newcommand{\mymarginplus}{1.1in}
\newcommand{\mymarginplustext}{6.6in}
\newcommand{\underlinehead}{\\[-0.1in] \rule{\mymarginplustext}{0.01in}}
\newcommand{\overlinefoot}{\rule{\mymarginplustext}{0.01in}\\}
% WIDEPARBOX uses margin space as well.
\newcommand{\wideparbox}[1]{\parbox[t]{\mymarginplustext}{#1}}
%
% HEADER AND FOOT
% Standard header and foot, but includes copyright notice.a
% To omit copyrite notice do not include copyrite.sty
%
\newcommand{\mycopyrite}{}
\def\ps@headings{\let\@mkboth\markboth
\def\@oddfoot{\wideparbox{\overlinefoot\rm\today \hfill \mycopyrite}}
\def\@evenfoot{\hspace*{-\mymarginplus}\wideparbox{\overlinefoot
\mycopyrite\hfill \today}}
\def\@evenhead{\hspace*{-\mymarginplus}\wideparbox{\rm
\thepage\hfill \sl \leftmark \underlinehead }}
\def\@oddhead{\wideparbox{\hbox{}\sl \rightmark \hfill
\rm\thepage \underlinehead}}\def\chaptermark##1{\markboth {\uppercase{\ifnum
\c@secnumdepth
>\m@ne
 \@chapapp\ \thechapter. \ \fi ##1}}{}}\def\sectionmark##1{\markright
 {\uppercase{\ifnum \c@secnumdepth >\z@
  \thesection. \ \fi ##1}}}}

%\oddsidemargin 0.25in \topmargin 0.0in \textwidth 6.5in \textheight 9in
%\evensidemargin 0.25in \headheight 0.18in \footskip 0.16in

%%   EPSF.TEX macro file:
%   Written by Tomas Rokicki of Radical Eye Software, 29 Mar 1989.
%   Revised by Don Knuth, 3 Jan 1990.
%   Revised by Tomas Rokicki to accept bounding boxes with no
%      space after the colon, 18 Jul 1990.
%
%   TeX macros to include an Encapsulated PostScript graphic.
%   Works by finding the bounding box comment,
%   calculating the correct scale values, and inserting a vbox
%   of the appropriate size at the current position in the TeX document.
%
%   To use with the center environment of LaTeX, preface the \epsffile
%   call with a \leavevmode.  (LaTeX should probably supply this itself
%   for the center environment.)
%
%   To use, simply say
%   \input epsf           % somewhere early on in your TeX file
%   \epsfbox{filename.ps} % where you want to insert a vbox for a figure
%
%   Alternatively, you can type
%
%   \epsfbox[0 0 30 50]{filename.ps} % to supply your own BB
%
%   which will not read in the file, and will instead use the bounding
%   box you specify.
%
%   The effect will be to typeset the figure as a TeX box, at the
%   point of your \epsfbox command. By default, the graphic will have its
%   `natural' width (namely the width of its bounding box, as described
%   in filename.ps). The TeX box will have depth zero.
%
%   You can enlarge or reduce the figure by saying
%     \epsfxsize=<dimen> \epsfbox{filename.ps}
%   (or
%     \epsfysize=<dimen> \epsfbox{filename.ps})
%   instead. Then the width of the TeX box will be \epsfxsize and its
%   height will be scaled proportionately (or the height will be
%   \epsfysize and its width will be scaled proportiontally).  The
%   width (and height) is restored to zero after each use.
%
%   A more general facility for sizing is available by defining the
%   \epsfsize macro.    Normally you can redefine this macro
%   to do almost anything.  The first parameter is the natural x size of
%   the PostScript graphic, the second parameter is the natural y size
%   of the PostScript graphic.  It must return the xsize to use, or 0 if
%   natural scaling is to be used.  Common uses include:
%
%      \epsfxsize  % just leave the old value alone
%      0pt         % use the natural sizes
%      #1          % use the natural sizes
%      \hsize      % scale to full width
%      0.5#1       % scale to 50% of natural size
%      \ifnum#1>\hsize\hsize\else#1\fi  % smaller of natural, hsize
%
%   If you want TeX to report the size of the figure (as a message
%   on your terminal when it processes each figure), say `\epsfverbosetrue'.
%
\newread\epsffilein    % file to \read
\newif\ifepsffileok    % continue looking for the bounding box?
\newif\ifepsfbbfound   % success?
\newif\ifepsfverbose   % report what you're making?
\newdimen\epsfxsize    % horizontal size after scaling
\newdimen\epsfysize    % vertical size after scaling
\newdimen\epsftsize    % horizontal size before scaling
\newdimen\epsfrsize    % vertical size before scaling
\newdimen\epsftmp      % register for arithmetic manipulation
\newdimen\pspoints     % conversion factor
%
\pspoints=1bp          % Adobe points are `big'
\epsfxsize=0pt         % Default value, means `use natural size'
\epsfysize=0pt         % ditto
%
\def\epsfbox#1{\global\def\epsfllx{72}\global\def\epsflly{72}%
   \global\def\epsfurx{540}\global\def\epsfury{720}%
   \def\lbracket{[}\def\testit{#1}\ifx\testit\lbracket
   \let\next=\epsfgetlitbb\else\let\next=\epsfnormal\fi\next{#1}}%
%
\def\epsfgetlitbb#1#2 #3 #4 #5]#6{\epsfgrab #2 #3 #4 #5 .\\%
   \epsfsetgraph{#6}}%
%
\def\epsfnormal#1{\epsfgetbb{#1}\epsfsetgraph{#1}}%
%
\def\epsfgetbb#1{%
%
%   The first thing we need to do is to open the
%   PostScript file, if possible.
%
\openin\epsffilein=#1
\ifeof\epsffilein\errmessage{I couldn't open #1, will ignore it}\else
%
%   Okay, we got it. Now we'll scan lines until we find one that doesn't
%   start with %. We're looking for the bounding box comment.
%
   {\epsffileoktrue \chardef\other=12
    \def\do##1{\catcode`##1=\other}\dospecials \catcode`\ =10
    \loop
       \read\epsffilein to \epsffileline
       \ifeof\epsffilein\epsffileokfalse\else
%
%   We check to see if the first character is a % sign;
%   if not, we stop reading (unless the line was entirely blank);
%   if so, we look further and stop only if the line begins with
%   `%%BoundingBox:'.
%
          \expandafter\epsfaux\epsffileline:. \\%
       \fi
   \ifepsffileok\repeat
   \ifepsfbbfound\else
    \ifepsfverbose\message{No bounding box comment in #1; using defaults}\fi\fi
   }\closein\epsffilein\fi}%
%
%   Now we have to calculate the scale and offset values to use.
%   First we compute the natural sizes.
%
\def\epsfclipstring{}% do we clip or not?  If so,
\def\epsfclipon{\def\epsfclipstring{ clip}}%
\def\epsfclipoff{\def\epsfclipstring{}}%
%
\def\epsfsetgraph#1{%
   \epsfrsize=\epsfury\pspoints
   \advance\epsfrsize by-\epsflly\pspoints
   \epsftsize=\epsfurx\pspoints
   \advance\epsftsize by-\epsfllx\pspoints
%
%   If `epsfxsize' is 0, we default to the natural size of the picture.
%   Otherwise we scale the graph to be \epsfxsize wide.
%
   \epsfxsize\epsfsize\epsftsize\epsfrsize
   \ifnum\epsfxsize=0 \ifnum\epsfysize=0
      \epsfxsize=\epsftsize \epsfysize=\epsfrsize
      \epsfrsize=0pt
%
%   We have a sticky problem here:  TeX doesn't do floating point arithmetic!
%   Our goal is to compute y = rx/t. The following loop does this reasonably
%   fast, with an error of at most about 16 sp (about 1/4000 pt).
% 
     \else\epsftmp=\epsftsize \divide\epsftmp\epsfrsize
       \epsfxsize=\epsfysize \multiply\epsfxsize\epsftmp
       \multiply\epsftmp\epsfrsize \advance\epsftsize-\epsftmp
       \epsftmp=\epsfysize
       \loop \advance\epsftsize\epsftsize \divide\epsftmp 2
       \ifnum\epsftmp>0
          \ifnum\epsftsize<\epsfrsize\else
             \advance\epsftsize-\epsfrsize \advance\epsfxsize\epsftmp \fi
       \repeat
       \epsfrsize=0pt
     \fi
   \else \ifnum\epsfysize=0
     \epsftmp=\epsfrsize \divide\epsftmp\epsftsize
     \epsfysize=\epsfxsize \multiply\epsfysize\epsftmp   
     \multiply\epsftmp\epsftsize \advance\epsfrsize-\epsftmp
     \epsftmp=\epsfxsize
     \loop \advance\epsfrsize\epsfrsize \divide\epsftmp 2
     \ifnum\epsftmp>0
        \ifnum\epsfrsize<\epsftsize\else
           \advance\epsfrsize-\epsftsize \advance\epsfysize\epsftmp \fi
     \repeat
     \epsfrsize=0pt
    \else
     \epsfrsize=\epsfysize
    \fi
   \fi
%
%  Finally, we make the vbox and stick in a \special that dvips can parse.
%
   \ifepsfverbose\message{#1: width=\the\epsfxsize, height=\the\epsfysize}\fi
   \epsftmp=10\epsfxsize \divide\epsftmp\pspoints
   \vbox to\epsfysize{\vfil\hbox to\epsfxsize{%
      \ifnum\epsfrsize=0\relax
        \special{PSfile=#1 llx=\epsfllx\space lly=\epsflly\space
            urx=\epsfurx\space ury=\epsfury\space rwi=\number\epsftmp
            \epsfclipstring}%
      \else
        \epsfrsize=10\epsfysize \divide\epsfrsize\pspoints
        \special{PSfile=#1 llx=\epsfllx\space lly=\epsflly\space
            urx=\epsfurx\space ury=\epsfury\space rwi=\number\epsftmp\space
            rhi=\number\epsfrsize \epsfclipstring}%
      \fi
      \hfil}}%
\global\epsfxsize=0pt\global\epsfysize=0pt}%
%
%   We still need to define the tricky \epsfaux macro. This requires
%   a couple of magic constants for comparison purposes.
%
{\catcode`\%=12 \global\let\epsfpercent=%\global\def\epsfbblit{%BoundingBox}}%
%
%   So we're ready to check for `%BoundingBox:' and to grab the
%   values if they are found.
%
\long\def\epsfaux#1#2:#3\\{\ifx#1\epsfpercent
   \def\testit{#2}\ifx\testit\epsfbblit
      \epsfgrab #3 . . . \\%
      \epsffileokfalse
      \global\epsfbbfoundtrue
   \fi\else\ifx#1\par\else\epsffileokfalse\fi\fi}%
%
%   Here we grab the values and stuff them in the appropriate definitions.
%
\def\epsfempty{}%
\def\epsfgrab #1 #2 #3 #4 #5\\{%
\global\def\epsfllx{#1}\ifx\epsfllx\epsfempty
      \epsfgrab #2 #3 #4 #5 .\\\else
   \global\def\epsflly{#2}%
   \global\def\epsfurx{#3}\global\def\epsfury{#4}\fi}%
%
%   We default the epsfsize macro.
%
\def\epsfsize#1#2{\epsfxsize}
%
%   Finally, another definition for compatibility with older macros.
%
\let\epsffile=\epsfbox

\usepackage{epsf}
\oddsidemargin 0.25in \evensidemargin 0.25in
\topmargin 0.0in
\textwidth 6.5in \textheight 9in
\headheight 0.18in \footskip 0.16in
\leftmargin -0.5in \rightmargin -0.5in

\newcommand{\fig}[1]{figures/#1}
\newcommand{\pfig}[1]{\epsfbox{\fig{#1}}}
\newcommand{\newfig}[1]{\epsffile{\fig{#1}}}

\newcommand{\fdaelement}[1]{elements/#1}
\newcommand{\spiceelement}[1]{equivalent_spice_elements/#1}

\newcommand{\FREEDA}{{\Huge{\textsl{\textsf{f}}}${\mathsf{REEDA}}^{{\tiny{\textsf{TM}}}}$}}
\newcommand{\FDA}{{\textsl{\textsf{f}}}${\mathsf{REEDA}}^{{\tiny{\textsf{TM}}}}$}
\newcommand{\notforsspice}[1]{#1}
\newcommand{\spicetwoonly}[1]{#1}
\newcommand{\spicethreeonly}[1]{#1}
\newcommand{\pspiceonly}[1]{#1}
\newcommand{\pspiceninetytwoonly}[1]{#1}
\newcommand{\sspiceonly}[1]{}
\newcommand{\fornutmeg}[1]{}
\newcommand{\sspicetwoonly}[1]{}

%%%%%%%%%%%%%%%%%%%%%%%%%%%%%%%%%%%%%%%%%%%%%%%%%%%%%%%%%%%%%%%%%%%%%%%%%%%%%%%%
\marginparwidth=\mymargin
\marginparsep=\mymarginparsep
\newcommand{\textplusmarginparwidth}{\textwidth+\marginparsep+\mymargin}
\newcommand{\X}{\\ \hline}    % line terminataion in keyword environment
\newcommand{\B}{{ \rm [}}     % begin optional parameter in \form{}
\newcommand{\E}{{\ \rm\hspace{-0.04in}] }}   % end optional parameter in \form{}
\newcommand{\expr}{{\sc Expressions supported}}
\newcommand{\reqd}{{\scriptsize REQUIRED}}
\newcommand{\omitted}{{\scriptsize OMITTED}}
\newcommand{\inferred}{{\scriptsize INFERRED}}
\newcommand{\para}{\newline{\scriptsize (PARASITIC)}}
\newcommand{\opt}{{\tiny  OPTIONAL}}

\newcommand{\Spice}{{\sc Spice}}
\newcommand{\spice}{{\sc Spice}}
\newcommand{\justspice}{{\sc Spice}}
\newcommand{\nutmeg}{{\sc NUTMEG}}
\newcommand{\probe}{{\sc Probe}}

\newcommand{\accusim}{{\sc AccuSim}}
\newcommand{\contec}{{\sc ContecSpice}}
\newcommand{\cdsspice}{{\sc CDS Spice}}
\newcommand{\hpimpulse}{{\sc HP Impulse}}
\newcommand{\hspice}{{\sc HSpice}}
\newcommand{\igspice}{{\sc IG\_SPICE}}
\newcommand{\isspice}{{\sc IsSpice}}
\newcommand{\mspice}{{\sc Microwave Spice}}
\newcommand{\pspice}{{\sc PSpice}}
\newcommand{\justpspice}{{\sc PSpice}}
\newcommand{\spectre}{{\sc Spectre}}
\newcommand{\spicetwo}{{\sc Spice2g6}}
\newcommand{\spicethree}{{\sc Spice3}}
\newcommand{\spiceplus}{{\sc SpicePlus}}
\newcommand{\sspice}{{\sc SuperSpice}}

\newcommand{\modelversion}[1]{& #1}
%%%%%%%%%%%%%%%%%%%%%%%%%%%%%%%%%%%%%%%%%%%%%%%%%%%%%%%%%%%%%%%%%%%%%%%%%%%%%%%%
% set up a counter for all occasions
%
\newcounter{count}

% set up new commands
%
\newcommand{\vshift}{\vspace{0.2in}}
% OFFSET
\newcommand{\offset}{\hspace*{0.45in} }
% OFFSETPARBOXWIDTH argument should be \textwith - offset
\newcommand{\offsetparbox}[1]{\parbox[t]{5in}{#1}}

% note: no labeling
\newcommand{\elementx}[3]{\clearpage\rm\markright{#3:#2}
\addcontentsline{toc}{section}{#1, #2: #3}
\mymarginparx{#1}{#2}{#3}\index{#1:#2}\index{#3:#2}}

% macros for sub elements
\newcommand{\subelement}[2]{\clearpage 
   \noindent\rule{\textwidth /2}{.5mm} \\[0.1in]
   {\large \bf #1} \hspace{0.2in} {\bf #2} \\
   \noindent\rule{\textwidth /2}{.5mm} \index{#1} \index{#2}}
% macros for models
\newcommand{\model}[2]{{
   \noindent\vspace{0.2in}\parbox{\textwidth}{
   \noindent\rule{\textwidth}{.5mm} \\[0.1in]
   {\large \bf #1 Model} \label{#1model} \hfill {\large #2} \\
   \noindent\rule{\textwidth}{.5mm} \index{#1} \index{#2}}}}
\newcommand{\modelx}[3]{{
   \noindent\vspace{0.2in}\parbox{\textwidth}{
   \noindent\rule{\textwidth}{.5mm} \\[0.1in]
   {\large \bf #1 Model} \hfill {\large #2} \hfill {\large #3} \\
   \noindent\rule{\textwidth}{.5mm} \index{#1} \index{#2}}}}



% BOXED
\newcommand{\boxed}[1]{\noindent
\newline \vshift \hspace*{\fill} {\tt \offsetparbox{\tt #1}}
 \vshift}


%
% KEYWORD
%
\newcommand{\keywordtable}[1]{
        \sloppy
        \hyphenation{ca-pac-i-t-an-ce} 
        \begin{center}
    \sf
        \begin{tabular}[t]
        {|p{0.58in}|p{3.07in}|p{0.55in}|p{0.60in}|}
        \hline
        \multicolumn{1}{|c}{\bf Name} &
        \multicolumn{1}{|c}{\parbox{2.77in}{\bf Description}}  &
        \multicolumn{1}{|c}{\bf Units} &
        \multicolumn{1}{|c|}{\bf Default} \X
        #1
        \end{tabular}
        \end{center}
    }

\newcommand{\keywordtwotable}[2]{
        \sloppy
        \hyphenation{ca-pac-i-t-an-ce} 
        \begin{center}
    \sf
        \begin{tabular}[t]
        {|p{0.58in}|p{2.38in}|p{0.55in}|p{0.60in}|p{0.53in}|}
        \hline
        \multicolumn{1}{|c}{\bf Name} &
        \multicolumn{1}{|c}{\parbox{2.20in}{\bf Description}}  &
        \multicolumn{1}{|c}{\bf Units} &
        \multicolumn{1}{|c}{\bf Default} &
        \multicolumn{1}{|c|}{\bf #1} \X
        #2
        \end{tabular}
        \end{center}
    }

\newcommand{\kw}[2]{
     \samepage{
     \noindent {\sl #1} \vspace{-0.5in} \\
     \keywordtable{#2} }}

\newcommand{\kwtwo}[3]{
     \samepage{
     \noindent {\sl #1} \vspace{-0.4in} \\
     \keywordtwotable{#2}{#3} }}

\newcommand{\keyword}[1]{\kw{Keywords:}{#1}}
\newcommand{\keywordtwo}[2]{\kwtwo{Keywords:}{#1}{#2}}
\newcommand{\modelkeyword}[1]{\kw{Model Keywords}{#1}}
\newcommand{\modelkeywordtwo}[2]{\kwtwo{Model Keywords}{#1}{#2}}

\newcommand{\myline}{\\[-0.1in]
\noindent \rule{\textwidth}{0.01in} \newline}

\newcommand{\myThickLine}{\\[-0.1in]
\noindent \rule{\textwidth}{0.02in} \newline}


% SPICE 2G6 FORM
%\newcommand{\spicetwoform}[1]{
%\spicetwoonly{\samepage{\noindent{\sl\spicetwo\form{#1}}}}}

% PSPICE88 FORM
%\newcommand{\pspiceform}[1]{
%\pspiceonly{\samepage{\noindent{\sl\pspice}\form{#1}}}}

% PSPICE92 FORM
%\newcommand{\pspiceninetytwoform}[1]{
%\pspiceninetytwoonly{\samepage{\noindent{\sl\pspice92\form{#1}}}}}


% SPICE3E2 FORM
%\newcommand{\spicethreeform}[1]{
%\spicethreeonly{\samepage{\noindent{\sl\spicethree\form{#1}}}}}

% FORM
\newcommand{\form}[1]{\samepage{\noindent
 {\sl Form} \myline
% \hspace*{\fill} % For some reason \fill = 0 when \pspiceform{} is used?
\offset
\it  \offsetparbox{#1}}
\\[0.1in]}

% ELEMENT FORM
\newcommand{\elementform}[1]{\samepage{\noindent
 {\sl Element Form} \myline
% \hspace*{\fill} % For some reason \fill = 0 when \pspiceform{} is used?
\offset
\it  \offsetparbox{#1}}
\\[0.1in]}

% MODEL FORM
\newcommand{\modelform}[1]{\samepage{\noindent
 {\sl Model Form} \myline
% \hspace*{\fill} % For some reason \fill = 0 when \pspiceform{} is used?
\offset
\it  \offsetparbox{#1}}
\\[0.1in]}

% LIMITS
\newcommand{\mylimits}[1]{\samepage{\noindent
 {\sl Limits} \myline
 \hspace*{\fill} \it  \offsetparbox{#1}}
 \vshift}

% EXAMPLE
\newcommand{\example}[1]{\samepage{\noindent
{\sl Example} \myline
\offset \tt  \offsetparbox{#1}}
 \vshift}

% PSPICE88 EXAMPLE
\newcommand{\pspiceexample}[1]{\samepage{\noindent
{\sl \pspice\ Example} \myline
\offset \tt  \offsetparbox{#1}}
 \vshift}

% MODEL TYPES
\newcommand{\modeltype}[1]{\samepage{\noindent
{\sl Model Type} \myline
 \hspace*{\fill} \tt \offsetparbox{#1}}
 \\[0.1in]}

% MODEL TYPES
\newcommand{\modeltypes}[1]{\samepage{\noindent
{\sl Model Types:} \myline
 \hspace*{\fill} \tt \offsetparbox{\tt #1}}
 \vshift}

% OFFSET ENUMERATE
\newcommand{\offsetenumerate}[1]{
     \offset \hspace*{-0.1in} {\begin{enumerate} #1 \end{enumerate}}}

% NOTE
\newcommand{\note}[1]{
\vshift\samepage{\noindent {\sl Note}\myline\vspace{-0.24in}}
 \offsetenumerate{#1} }

% SPECIAL NOTE
\newcommand{\specialnote}[2]{
\vshift\samepage{\noindent {\sl #1}\myline\vspace{-0.24in}}\\#2}

\newcommand{\dc}{\mbox{\tt DC}}
\newcommand{\ac}{\mbox{\tt AC}}
\newcommand{\SPICE}{\mbox{\tt SPICE}}
\newcommand{\m}[1]{{\bf #1}}                           % matrix command  \m{}

% ////// Changing nodes to terminals///////
% print terminals in \tt and enclose in a circle use outside
\newcommand{\terminal}[1]{\: \mbox{\tt #1} \!\!\!\! \bigcirc }
%
% set up environment for example
%
\newcounter{excount}
\newcounter{dummy}
\newenvironment{eg}{\vspace{0.1in}\noindent\rule{\textwidth}{.5mm}
   \begin{list}
   {{\addtocounter{excount}{1}
   \em Example\/ \arabic{chapter}.\arabic{excount}\/}:}
   {\usecounter{dummy}
   \setlength{\rightmargin}{\leftmargin}}
   }{\end{list} \rule{\textwidth}{.5mm}\vspace{0.1in}}
%
% set up environment for block
% currently this draws a horizontal line at the start of block and another
% at the end of block.
%
\newenvironment{block}{\vspace{0.1in}\noindent\rule{\textwidth}{.5mm}
   }{\rule{\textwidth}{.5mm}\vspace{0.1in}}
%

%
% Macros for element summaries
%
% macros for element summary
%\newcommand{\summaryelement}[2]{
%   \vspace{0.4in}
%   \mymarginpar{#1}{#2} 
%   \addcontentsline{toc}{section}{#1, #2}
%   \vspace{-0.6in} \\
%   \noindent Full description on page \pageref{#1element}. \vshift\\
%   }

% Summary MODEL TYPE
%\newcommand{\summarymodeltype}[1]{\samepage{\noindent
%{\sl Model Type} \myline
% \hspace*{\fill} \tt \offsetparbox{\tt #1
% \hfill Summary on \pageref{#1summary} \index{#1}}}
% }

% Summary MODEL TYPES
%\newcommand{\summarymodeltypes}[1]{\samepage{\noindent
%{\sl Model Types} \myline
% \hspace*{\fill} \tt \offsetparbox{\tt #1
% \hfill Summary on \pageref{#1summary} \index{#1}}}
% }

%
% macros for model summary
%\newcommand{\summarymodel}[2]{\clearpage
%   \addcontentsline{toc}{section}{#1, #2}
%   \mymarginpar{#1}{#2} \label{#1summary}
%   \index{#1} \index{#2}
%   \noindent Full description on page \pageref{#1model} \\[0.1in]
%   }



%
% set up wide descriptive list
%
\newenvironment{widelist}
    {\begin{list}{}{\setlength{\rightmargin}{0in} \setlength{\itemsep}{0.1in}
    \setlength{\labelwidth}{0.95in} \setlength{\labelsep}{0.1in}
\setlength{\listparindent}{0in} \setlength{\parsep}{0in}
    \setlength{\leftmargin}{1.0in}}
    }{\end{list}}

\newcommand{\STAR}{\hspace*{\fill} * \hspace*{\fill}}

\newcommand{\sym}[1]{\hspace*{\fill} ($#1$)}

%
% The thebibliography environment is redefined so the the word References is
% not output
%\def\thebibliography#1{\list
% {[\arabic{enumi}]}{\settowidth\labelwidth{[#1]}\leftmargin\labelwidth
% \advance\leftmargin\labelsep
% \usecounter{enumi}}
% \def\newblock{\hskip .11em plus .33em minus -.07em}
% \sloppy\clubpenalty4000\widowpenalty4000
% \sfcode`\.=1000\relax}
%\let\endthebibliography=\endlist
% END thebibliography environment redefinition



%\newcommand{\eskipv}[1]{\clearpage\marginlabel{#1}}
%\newcommand{\eskip}[1]{\vspace*{\fill}\clearpage\marginlabel{#1}}
%\newcommand{\eskipnv}[1]{\newpage\marginlabel{#1}}
%\newcommand{\eskipn}[1]{\vspace*{\fill}\newpage\marginlabel{#1}}

%marginlabel is very wide
%\newcommand{\eskipfullv}[1]{\clearpage\marginlabelfull{#1}}
%\newcommand{\eskipfull}[1]{\vspace*{\fill}\clearpage\marginlabelfull{#1}}
%\newcommand{\eskipfullnv}[1]{\newpage\marginlabelfull{#1}}
%\newcommand{\eskipfulln}[1]{\vspace*{\fill}\newpage\marginlabelfull{#1}}

%\newcommand{\mycontentsline}[5]{\parbox{#1}{#2}#3
%\hspace{0.1in}\dotfill\parbox{0.3in}{\hfill\pageref{#4#5}}\\[0.1in]}
%\newcommand{\mysline}[2]{\mycontentsline{1.2in}{#1}{#2}{#1}{statement}}
%\newcommand{\mymline}[2]{\mycontentsline{0.7in}{#1}{#2}{#1}{model}}
%\newcommand{\myeline}[2]{\mycontentsline{0.7in}{#1}{#2}{#1}{element}}

%\newcommand{\myincontentsline}[5]{\vspace{0.05in}\noindent\parbox{#1}{#2}#3
%\hspace{0.1in}
%\dotfill\parbox{0.7in}{\hfill Page \pageref{#4#5}}\\[0.05in]\noindent}
%\newcommand{\myinsline}[2]{\myincontentsline{1.2in}{#1}{#2}{#1}{statement}}
%\newcommand{\myinmline}[2]{\myincontentsline{0.5in}{#1}{#2}{#1}{model}}
%\newcommand{\myineline}[2]{\myincontentsline{0.5in}{#1}{#2}{#1}{element}}

%
%
% The following a symbols that could used alot.
\newcommand{\ms}[1]{\mbox{\scriptsize #1}}
\newcommand{\AF}{A_F}
\newcommand{\CBD}{C'_{BD}}
\newcommand{\CBS}{C'_{BS}}
\newcommand{\CGBO}{C_{GBO}}
\newcommand{\CGDO}{C_{GDO}}
\newcommand{\CGSO}{C_{GSO}}
\newcommand{\CJ}{C_J}
\newcommand{\CJSW}{C_{J,\ms{SW}}}
\newcommand{\DELTA}{\delta}
\newcommand{\ETA}{\eta}
\newcommand{\FC}{F_C}
\newcommand{\GAMMA}{\gamma}
\newcommand{\IS}{I_S}
\newcommand{\JS}{J_S}
\newcommand{\KAPPA}{\kappa}
\newcommand{\KF}{K_F}
\newcommand{\KP}{K_P}
\newcommand{\LAMBDA}{\lambda}
\newcommand{\LD}{X_{JL}}
\newcommand{\LEVEL}{M_J}
\newcommand{\MJ}{M_J}
\newcommand{\MJSW}{M_{J,\ms{SW}}}
\newcommand{\NSUB}{N_B}
\newcommand{\NSS}{N_{\ms{SS}}}
\newcommand{\NFS}{N_{\ms{FS}}}
\newcommand{\NEFF}{N_{\ms{EFF}}}
\newcommand{\PB}{\phi_J}
\newcommand{\PHI}{2\phi_B}
\newcommand{\RD}{R_D}
\newcommand{\RS}{R_S}
\newcommand{\RSH}{R_{\ms{SH}}}
\newcommand{\THETA}{\theta}
\newcommand{\TOX}{T_{OX}}
\newcommand{\TPG}{T_{\ms{PG}}}
\newcommand{\UCRIT}{U_C}
\newcommand{\UEXP}{U_{\ms{EXP}}}
\newcommand{\UO}{\mu_0}
\newcommand{\UTRA}{U_{\ms{TRA}}}
\newcommand{\VMAX}{V_{\ms{MAX}}}
\newcommand{\VTZERO}{V_{T0}}
\newcommand{\VTO}{V_{T0}}
\newcommand{\XJ}{X_J}
\newcommand{\Length}{L} %  \L already used
\newcommand{\N}{N}
\newcommand{\PBSW}{\phi_{J,{\ms{SW}}}}
\newcommand{\RB}{R_B}
\newcommand{\RG}{R_B}
\newcommand{\RDS}{R_{DS}}
\newcommand{\TT}{\tau_T}
\newcommand{\W}{W}
\newcommand{\WD}{W_D}
\newcommand{\XQC}{X_{QC}}
\newcommand{\JSSW}{J_{S,{\ms{SW}}}}
\newcommand{\DL}{\Delta_L}
\newcommand{\DW}{\Delta_W}
\newcommand{\DELL}{\Delta_{L,\ms{SW}}}
\newcommand{\KONE}{K_1}
\newcommand{\KTWO}{K_2}
\newcommand{\MUS}{\mu_S}
\newcommand{\MUZ}{\mu_Z}
\newcommand{\NZERO}{N_0}
\newcommand{\NB}{N_B}
\newcommand{\ND}{N_D}
\newcommand{\TEMP}{T}
\newcommand{\VDD}{V_{DD}}
\newcommand{\WDF}{W_{\ms{DF}}}
\newcommand{\VFB}{V_{\ms{FB}}}
\newcommand{\UZERO}{U_0}
\newcommand{\UONE}{U_1}
\newcommand{\XTWOE}{X_{2E}}
\newcommand{\XTWOMS}{X_{2\ms{MS}}}
\newcommand{\XTWOMZ}{X_{2\ms{MZ}}}
\newcommand{\XTWOUZERO}{X_{2\ms{U0}}}
\newcommand{\XTWOUONE}{X_{2\ms{U1}}}
\newcommand{\XTHREEE}{X_{3E}}
\newcommand{\XTHREEMS}{X_{3\ms{MS}}}
\newcommand{\XTHREEMZ}{X_{3\ms{MZ}}}
\newcommand{\XTHREEUZERO}{X_{3\ms{U0}}}
\newcommand{\XTHREEUONE}{X_{3\ms{U1}}}
\newcommand{\XPART}{X_{\ms{PART}}}
\newcommand{\PS}{P_S}
\newcommand{\PD}{P_D}
\newcommand{\NRS}{N_{RS}}
\newcommand{\NRG}{N_{RG}}
\newcommand{\NRB}{N_{RB}}
\newcommand{\NRD}{N_{RD}}


\newcommand{\Net}{{${\cal N}$}}                          % network \N
\newcommand{\Nprime}{{${\cal N}^{\prime}$}}            % another network \Nprime
\newcommand{\Nold}{{${\cal N}^{\mbox{old}}$}}          % old network  \Nold
\newcommand{\Nnew}{{${\cal N}^{\mbox{new}}$}}          % new network  \Nnew

\newcommand{\GMIN}{{G_{\ms{MIN}}}}

\newcommand{\optionitem}[2]{
\item[{\tt #1}{#2}]\label{.OPTION#1}\index{.OPTIONS, #1}\index{#1}}

\newcommand{\error}[1]{\vspace{0.1in}\noindent{\tt #1}\\}


%For numbering an equation which is incoorporated
%with text.
\newcommand{\inlineeq}{\hspace*{\fill}\refstepcounter{equation}{\rm
(\theequation)}\\}

\begin{document}
\noindent{\LARGE \textbf{Single Frequency FM voltage source}
\hspace{\fill}\textbf{vsffm}}
\hrulefill\linethickness{0.5mm}\line(1,0){425}
\normalsize
\newline
% the resistor figure
\begin{figure}[h]
\centerline{\epsfxsize=0.5in\epsfbox{v_source.eps}}
\caption{Independent Voltage Source Element.}
\end{figure}
\newline
% form for \FDA
\linethickness{0.5mm} \line(1,0){425}
\newline
\textit{Form:}
$\tt sffm$:$\langle \tt{instance\ name}\rangle$
$n_1\ n_2\ $ $\langle \tt{parameter\ list}\rangle$
\newline
\begin{tabular}{r l}
$n_1$ & is the positive element node, \\
&  \\
$n_2$ & is the negative element node. \\
%&  \\
%parameter list & see table 1 for parameter list
\end{tabular}
% Parameter list
\newline
\textit{Parameters:}
\begin{table}[H]
\begin{tabular}{|c|c|c|c|}
\hline
Parameter&Type&Default value&Required?\\
\hline
vo: Offset voltage(V) & DOUBLE & 0 & no\\
\hline
va: RMS voltage amplitude (V) & DOUBLE & 0 & no\\
\hline
fcarrier: AC frequency (Hz) & DOUBLE & 0 & no\\
\hline
mdi: modulation index (Dimensionless) & DOUBLE & 0 & no\\
\hline
fsignal: Signal frequency (Hz) & DOUBLE & 0 & no\\
\par
\hline
\end{tabular}
\end{table}
% example in \FDA
%\newline
\noindent\linethickness{0.5mm}\line(1,0){425}
\newline
\textit{Example:}
\newline
\texttt{vsffm:vsignal\ 8\ 0\ io=0.2 va=0.7 fcarrier=4 mdi=0.9
fsignal=1}
\newline
\linethickness{0.5mm} \line(1,0){425}
\newline
\textit{Description:}\\
The waveform shape for this source is
\begin{equation}
i = i_o + i_a[\sin(2.\pi.f_{carrier}.t) +
mdi\sin(2.\pi.f_{signal}.t)]
\end{equation}
\begin{figure}[hbp]
\centerline{\epsfxsize=3in\epsfbox{vsffm.eps}} \caption{Voltage
source single frequency frequency modulation waveform for
\texttt{vsffm:vsignal\ 8\ 0\ vo=0.2 va=0.7 fcarrier=4 mdi=0.9
fsignal=1}}
\end{figure}
\newline
\linethickness{0.5mm} \line(1,0){425}
\newline
\textit{Notes:}\\
This is the \texttt{V} element in the SPICE compatible netlist.\\
\linethickness{0.5mm} \line(1,0){425}
\newline
\textit{Version:}\\
2002.05.15 \\
% Credits
\newpage
\noindent\linethickness{0.5mm}\line(1,0){425}
\newline
\textit{Credits:}\\
\begin{tabular}{l l l l}
Name & Affiliation & Date & Links \\
Satish Uppathil & NC State University & May 2002 & \epsfxsize=1in\epsfbox{logo.eps} \\
svuppath@eos.ncsu.edu & & & www.ncsu.edu    \\
\end{tabular}
\end{document}

\caption[Voltage source single frequency frequency modulation ({\tt SFFM})
waveform]{Voltage source single frequency frequency modulation ({\tt SFFM})
waveform for\newline \hspace*{\fill}{\tt SFFM(0.2 0.7 4 0.9 1)}\hspace*{\fill}.
\label{fig:vsffm}}
\end{figure}

\noindent{\underline{\bf Pulse:}}
\\[0.2in]
\form{{\tt PULSE(} $V_1$ $V_2$ \B $T_D$ \E \B $T_R$ \E \B $T_F$\E
       \B $W$ \E \B $T$ \E {\tt )}}

\keywordtable{
$V_1  $&initial voltage & A &  \reqd   \X
$V_2  $&pulsed voltage  & A &  \reqd   \X
$T_D  $&delay time    & s &   0.0    \X
$T_R  $&rise time     & s &{\tt TSTEP}  \X
$T_F  $&fall time     & s &{\tt TSTEP}  \X
$W    $&pulse width   & s &{\tt TSTOP}  \X
$T    $&period        & s &{\tt TSTOP}  \X
}
The pulse transient waveform is defined by
\begin{equation}
v = \left\{ \begin{array}{ll}
V_1                         & t \le T_D\\
V_1 + {{\textstyle t'} \over {\textstyle T_R}} ({V_2}-{V_1})&0<t' \le T_R\\
V_2                         &{T_R} < t'< (T_R+W)\\
V_2 - {{\textstyle t'-W} \over {\textstyle T_F}} (V_1-V_2)
                   &(T_R+W) < t' < (T_R+W+T_F)\\
V_1           &(T_R+W+T_F) < t' < T
     \end{array} \right. %}
\end{equation}
where
\begin{equation}
t' = t - T_D -(n-1)T
\end{equation}
and $t$ is the voltage analysis time and $n$ is the cycle index.  The effect
of this is that after an initial time delay $T_D$ the transient waveform
repeats itself every cycle.

\begin{figure}[hbp]
\centering
% GNUPLOT: LaTeX picture
\setlength{\unitlength}{0.240900pt}
\ifx\plotpoint\undefined\newsavebox{\plotpoint}\fi
\sbox{\plotpoint}{\rule[-0.175pt]{0.350pt}{0.350pt}}%
\begin{picture}(1500,900)(0,0)
%\tenrm
\sbox{\plotpoint}{\rule[-0.175pt]{0.350pt}{0.350pt}}%
\put(264,158){\rule[-0.175pt]{282.335pt}{0.350pt}}
\put(264,158){\rule[-0.175pt]{0.350pt}{151.526pt}}
\put(264,158){\rule[-0.175pt]{4.818pt}{0.350pt}}
\put(242,158){\makebox(0,0)[r]{0}}
\put(1416,158){\rule[-0.175pt]{4.818pt}{0.350pt}}
\put(264,221){\rule[-0.175pt]{4.818pt}{0.350pt}}
\put(242,221){\makebox(0,0)[r]{0.2}}
\put(1416,221){\rule[-0.175pt]{4.818pt}{0.350pt}}
\put(264,284){\rule[-0.175pt]{4.818pt}{0.350pt}}
\put(242,284){\makebox(0,0)[r]{0.4}}
\put(1416,284){\rule[-0.175pt]{4.818pt}{0.350pt}}
\put(264,347){\rule[-0.175pt]{4.818pt}{0.350pt}}
\put(242,347){\makebox(0,0)[r]{0.6}}
\put(1416,347){\rule[-0.175pt]{4.818pt}{0.350pt}}
\put(264,410){\rule[-0.175pt]{4.818pt}{0.350pt}}
\put(242,410){\makebox(0,0)[r]{0.8}}
\put(1416,410){\rule[-0.175pt]{4.818pt}{0.350pt}}
\put(264,473){\rule[-0.175pt]{4.818pt}{0.350pt}}
\put(242,473){\makebox(0,0)[r]{1}}
\put(1416,473){\rule[-0.175pt]{4.818pt}{0.350pt}}
\put(264,535){\rule[-0.175pt]{4.818pt}{0.350pt}}
\put(242,535){\makebox(0,0)[r]{1.2}}
\put(1416,535){\rule[-0.175pt]{4.818pt}{0.350pt}}
\put(264,598){\rule[-0.175pt]{4.818pt}{0.350pt}}
\put(242,598){\makebox(0,0)[r]{1.4}}
\put(1416,598){\rule[-0.175pt]{4.818pt}{0.350pt}}
\put(264,661){\rule[-0.175pt]{4.818pt}{0.350pt}}
\put(242,661){\makebox(0,0)[r]{1.6}}
\put(1416,661){\rule[-0.175pt]{4.818pt}{0.350pt}}
\put(264,724){\rule[-0.175pt]{4.818pt}{0.350pt}}
\put(242,724){\makebox(0,0)[r]{1.8}}
\put(1416,724){\rule[-0.175pt]{4.818pt}{0.350pt}}
\put(264,787){\rule[-0.175pt]{4.818pt}{0.350pt}}
\put(242,787){\makebox(0,0)[r]{2}}
\put(1416,787){\rule[-0.175pt]{4.818pt}{0.350pt}}
\put(264,158){\rule[-0.175pt]{0.350pt}{4.818pt}}
\put(264,113){\makebox(0,0){0}}
\put(264,767){\rule[-0.175pt]{0.350pt}{4.818pt}}
\put(381,158){\rule[-0.175pt]{0.350pt}{4.818pt}}
\put(381,113){\makebox(0,0){0.5}}
\put(381,767){\rule[-0.175pt]{0.350pt}{4.818pt}}
\put(498,158){\rule[-0.175pt]{0.350pt}{4.818pt}}
\put(498,113){\makebox(0,0){1}}
\put(498,767){\rule[-0.175pt]{0.350pt}{4.818pt}}
\put(616,158){\rule[-0.175pt]{0.350pt}{4.818pt}}
\put(616,113){\makebox(0,0){1.5}}
\put(616,767){\rule[-0.175pt]{0.350pt}{4.818pt}}
\put(733,158){\rule[-0.175pt]{0.350pt}{4.818pt}}
\put(733,113){\makebox(0,0){2}}
\put(733,767){\rule[-0.175pt]{0.350pt}{4.818pt}}
\put(850,158){\rule[-0.175pt]{0.350pt}{4.818pt}}
\put(850,113){\makebox(0,0){2.5}}
\put(850,767){\rule[-0.175pt]{0.350pt}{4.818pt}}
\put(967,158){\rule[-0.175pt]{0.350pt}{4.818pt}}
\put(967,113){\makebox(0,0){3}}
\put(967,767){\rule[-0.175pt]{0.350pt}{4.818pt}}
\put(1084,158){\rule[-0.175pt]{0.350pt}{4.818pt}}
\put(1084,113){\makebox(0,0){3.5}}
\put(1084,767){\rule[-0.175pt]{0.350pt}{4.818pt}}
\put(1202,158){\rule[-0.175pt]{0.350pt}{4.818pt}}
\put(1202,113){\makebox(0,0){4}}
\put(1202,767){\rule[-0.175pt]{0.350pt}{4.818pt}}
\put(1319,158){\rule[-0.175pt]{0.350pt}{4.818pt}}
\put(1319,113){\makebox(0,0){4.5}}
\put(1319,767){\rule[-0.175pt]{0.350pt}{4.818pt}}
\put(1436,158){\rule[-0.175pt]{0.350pt}{4.818pt}}
\put(1436,113){\makebox(0,0){5}}
\put(1436,767){\rule[-0.175pt]{0.350pt}{4.818pt}}
\put(264,158){\rule[-0.175pt]{282.335pt}{0.350pt}}
\put(1436,158){\rule[-0.175pt]{0.350pt}{151.526pt}}
\put(264,787){\rule[-0.175pt]{282.335pt}{0.350pt}}
\put(264,158){\rule[-0.175pt]{0.350pt}{151.526pt}}
\put(264,252){\usebox{\plotpoint}}
\put(264,252){\rule[-0.175pt]{55.407pt}{0.350pt}}
\put(494,252){\usebox{\plotpoint}}
\put(495,253){\usebox{\plotpoint}}
\put(496,254){\usebox{\plotpoint}}
\put(497,255){\usebox{\plotpoint}}
\put(498,257){\usebox{\plotpoint}}
\put(499,258){\usebox{\plotpoint}}
\put(500,260){\rule[-0.175pt]{0.350pt}{1.927pt}}
\put(501,268){\rule[-0.175pt]{0.350pt}{1.927pt}}
\put(502,276){\rule[-0.175pt]{0.350pt}{1.927pt}}
\put(503,284){\rule[-0.175pt]{0.350pt}{1.927pt}}
\put(504,292){\rule[-0.175pt]{0.350pt}{1.927pt}}
\put(505,300){\rule[-0.175pt]{0.350pt}{1.566pt}}
\put(506,306){\rule[-0.175pt]{0.350pt}{1.566pt}}
\put(507,313){\rule[-0.175pt]{0.350pt}{1.566pt}}
\put(508,319){\rule[-0.175pt]{0.350pt}{1.566pt}}
\put(509,326){\rule[-0.175pt]{0.350pt}{1.566pt}}
\put(510,332){\rule[-0.175pt]{0.350pt}{1.566pt}}
\put(511,339){\rule[-0.175pt]{0.350pt}{1.606pt}}
\put(512,345){\rule[-0.175pt]{0.350pt}{1.606pt}}
\put(513,352){\rule[-0.175pt]{0.350pt}{1.606pt}}
\put(514,358){\rule[-0.175pt]{0.350pt}{1.606pt}}
\put(515,365){\rule[-0.175pt]{0.350pt}{1.606pt}}
\put(516,372){\rule[-0.175pt]{0.350pt}{1.606pt}}
\put(517,378){\rule[-0.175pt]{0.350pt}{1.566pt}}
\put(518,385){\rule[-0.175pt]{0.350pt}{1.566pt}}
\put(519,392){\rule[-0.175pt]{0.350pt}{1.566pt}}
\put(520,398){\rule[-0.175pt]{0.350pt}{1.566pt}}
\put(521,405){\rule[-0.175pt]{0.350pt}{1.566pt}}
\put(522,411){\rule[-0.175pt]{0.350pt}{1.566pt}}
\put(523,418){\rule[-0.175pt]{0.350pt}{1.606pt}}
\put(524,424){\rule[-0.175pt]{0.350pt}{1.606pt}}
\put(525,431){\rule[-0.175pt]{0.350pt}{1.606pt}}
\put(526,437){\rule[-0.175pt]{0.350pt}{1.606pt}}
\put(527,444){\rule[-0.175pt]{0.350pt}{1.606pt}}
\put(528,451){\rule[-0.175pt]{0.350pt}{1.606pt}}
\put(529,457){\rule[-0.175pt]{0.350pt}{1.566pt}}
\put(530,464){\rule[-0.175pt]{0.350pt}{1.566pt}}
\put(531,471){\rule[-0.175pt]{0.350pt}{1.566pt}}
\put(532,477){\rule[-0.175pt]{0.350pt}{1.566pt}}
\put(533,484){\rule[-0.175pt]{0.350pt}{1.566pt}}
\put(534,490){\rule[-0.175pt]{0.350pt}{1.566pt}}
\put(535,497){\rule[-0.175pt]{0.350pt}{1.606pt}}
\put(536,503){\rule[-0.175pt]{0.350pt}{1.606pt}}
\put(537,510){\rule[-0.175pt]{0.350pt}{1.606pt}}
\put(538,517){\rule[-0.175pt]{0.350pt}{1.606pt}}
\put(539,523){\rule[-0.175pt]{0.350pt}{1.606pt}}
\put(540,530){\rule[-0.175pt]{0.350pt}{1.606pt}}
\put(541,537){\rule[-0.175pt]{0.350pt}{1.566pt}}
\put(542,543){\rule[-0.175pt]{0.350pt}{1.566pt}}
\put(543,550){\rule[-0.175pt]{0.350pt}{1.566pt}}
\put(544,556){\rule[-0.175pt]{0.350pt}{1.566pt}}
\put(545,563){\rule[-0.175pt]{0.350pt}{1.566pt}}
\put(546,569){\rule[-0.175pt]{0.350pt}{1.566pt}}
\put(547,576){\rule[-0.175pt]{0.350pt}{1.606pt}}
\put(548,582){\rule[-0.175pt]{0.350pt}{1.606pt}}
\put(549,589){\rule[-0.175pt]{0.350pt}{1.606pt}}
\put(550,596){\rule[-0.175pt]{0.350pt}{1.606pt}}
\put(551,602){\rule[-0.175pt]{0.350pt}{1.606pt}}
\put(552,609){\rule[-0.175pt]{0.350pt}{1.606pt}}
\put(553,616){\rule[-0.175pt]{0.350pt}{1.879pt}}
\put(554,623){\rule[-0.175pt]{0.350pt}{1.879pt}}
\put(555,631){\rule[-0.175pt]{0.350pt}{1.879pt}}
\put(556,639){\rule[-0.175pt]{0.350pt}{1.879pt}}
\put(557,647){\rule[-0.175pt]{0.350pt}{1.879pt}}
\put(558,654){\rule[-0.175pt]{0.350pt}{1.606pt}}
\put(559,661){\rule[-0.175pt]{0.350pt}{1.606pt}}
\put(560,668){\rule[-0.175pt]{0.350pt}{1.606pt}}
\put(561,675){\rule[-0.175pt]{0.350pt}{1.606pt}}
\put(562,681){\rule[-0.175pt]{0.350pt}{1.606pt}}
\put(563,688){\rule[-0.175pt]{0.350pt}{1.606pt}}
\put(564,695){\rule[-0.175pt]{0.350pt}{1.164pt}}
\put(565,699){\rule[-0.175pt]{0.350pt}{1.164pt}}
\put(566,704){\rule[-0.175pt]{0.350pt}{1.164pt}}
\put(567,709){\rule[-0.175pt]{0.350pt}{1.164pt}}
\put(568,714){\rule[-0.175pt]{0.350pt}{1.164pt}}
\put(569,719){\rule[-0.175pt]{0.350pt}{1.164pt}}
\put(570,723){\usebox{\plotpoint}}
\put(570,724){\rule[-0.175pt]{55.407pt}{0.350pt}}
\put(800,722){\rule[-0.175pt]{0.350pt}{0.442pt}}
\put(801,720){\rule[-0.175pt]{0.350pt}{0.442pt}}
\put(802,718){\rule[-0.175pt]{0.350pt}{0.442pt}}
\put(803,716){\rule[-0.175pt]{0.350pt}{0.442pt}}
\put(804,714){\rule[-0.175pt]{0.350pt}{0.442pt}}
\put(805,713){\rule[-0.175pt]{0.350pt}{0.442pt}}
\put(806,709){\rule[-0.175pt]{0.350pt}{0.964pt}}
\put(807,705){\rule[-0.175pt]{0.350pt}{0.964pt}}
\put(808,701){\rule[-0.175pt]{0.350pt}{0.964pt}}
\put(809,697){\rule[-0.175pt]{0.350pt}{0.964pt}}
\put(810,693){\rule[-0.175pt]{0.350pt}{0.964pt}}
\put(811,689){\rule[-0.175pt]{0.350pt}{0.964pt}}
\put(812,685){\rule[-0.175pt]{0.350pt}{0.923pt}}
\put(813,681){\rule[-0.175pt]{0.350pt}{0.923pt}}
\put(814,677){\rule[-0.175pt]{0.350pt}{0.923pt}}
\put(815,673){\rule[-0.175pt]{0.350pt}{0.923pt}}
\put(816,669){\rule[-0.175pt]{0.350pt}{0.923pt}}
\put(817,666){\rule[-0.175pt]{0.350pt}{0.923pt}}
\put(818,661){\rule[-0.175pt]{0.350pt}{1.156pt}}
\put(819,656){\rule[-0.175pt]{0.350pt}{1.156pt}}
\put(820,651){\rule[-0.175pt]{0.350pt}{1.156pt}}
\put(821,646){\rule[-0.175pt]{0.350pt}{1.156pt}}
\put(822,642){\rule[-0.175pt]{0.350pt}{1.156pt}}
\put(823,638){\rule[-0.175pt]{0.350pt}{0.964pt}}
\put(824,634){\rule[-0.175pt]{0.350pt}{0.964pt}}
\put(825,630){\rule[-0.175pt]{0.350pt}{0.964pt}}
\put(826,626){\rule[-0.175pt]{0.350pt}{0.964pt}}
\put(827,622){\rule[-0.175pt]{0.350pt}{0.964pt}}
\put(828,618){\rule[-0.175pt]{0.350pt}{0.964pt}}
\put(829,614){\rule[-0.175pt]{0.350pt}{0.923pt}}
\put(830,610){\rule[-0.175pt]{0.350pt}{0.923pt}}
\put(831,606){\rule[-0.175pt]{0.350pt}{0.923pt}}
\put(832,602){\rule[-0.175pt]{0.350pt}{0.923pt}}
\put(833,598){\rule[-0.175pt]{0.350pt}{0.923pt}}
\put(834,595){\rule[-0.175pt]{0.350pt}{0.923pt}}
\put(835,591){\rule[-0.175pt]{0.350pt}{0.964pt}}
\put(836,587){\rule[-0.175pt]{0.350pt}{0.964pt}}
\put(837,583){\rule[-0.175pt]{0.350pt}{0.964pt}}
\put(838,579){\rule[-0.175pt]{0.350pt}{0.964pt}}
\put(839,575){\rule[-0.175pt]{0.350pt}{0.964pt}}
\put(840,571){\rule[-0.175pt]{0.350pt}{0.964pt}}
\put(841,567){\rule[-0.175pt]{0.350pt}{0.964pt}}
\put(842,563){\rule[-0.175pt]{0.350pt}{0.964pt}}
\put(843,559){\rule[-0.175pt]{0.350pt}{0.964pt}}
\put(844,555){\rule[-0.175pt]{0.350pt}{0.964pt}}
\put(845,551){\rule[-0.175pt]{0.350pt}{0.964pt}}
\put(846,547){\rule[-0.175pt]{0.350pt}{0.964pt}}
\put(847,543){\rule[-0.175pt]{0.350pt}{0.923pt}}
\put(848,539){\rule[-0.175pt]{0.350pt}{0.923pt}}
\put(849,535){\rule[-0.175pt]{0.350pt}{0.923pt}}
\put(850,531){\rule[-0.175pt]{0.350pt}{0.923pt}}
\put(851,527){\rule[-0.175pt]{0.350pt}{0.923pt}}
\put(852,524){\rule[-0.175pt]{0.350pt}{0.923pt}}
\put(853,520){\rule[-0.175pt]{0.350pt}{0.964pt}}
\put(854,516){\rule[-0.175pt]{0.350pt}{0.964pt}}
\put(855,512){\rule[-0.175pt]{0.350pt}{0.964pt}}
\put(856,508){\rule[-0.175pt]{0.350pt}{0.964pt}}
\put(857,504){\rule[-0.175pt]{0.350pt}{0.964pt}}
\put(858,500){\rule[-0.175pt]{0.350pt}{0.964pt}}
\put(859,496){\rule[-0.175pt]{0.350pt}{0.964pt}}
\put(860,492){\rule[-0.175pt]{0.350pt}{0.964pt}}
\put(861,488){\rule[-0.175pt]{0.350pt}{0.964pt}}
\put(862,484){\rule[-0.175pt]{0.350pt}{0.964pt}}
\put(863,480){\rule[-0.175pt]{0.350pt}{0.964pt}}
\put(864,476){\rule[-0.175pt]{0.350pt}{0.964pt}}
\put(865,472){\rule[-0.175pt]{0.350pt}{0.964pt}}
\put(866,468){\rule[-0.175pt]{0.350pt}{0.964pt}}
\put(867,464){\rule[-0.175pt]{0.350pt}{0.964pt}}
\put(868,460){\rule[-0.175pt]{0.350pt}{0.964pt}}
\put(869,456){\rule[-0.175pt]{0.350pt}{0.964pt}}
\put(870,452){\rule[-0.175pt]{0.350pt}{0.964pt}}
\put(871,448){\rule[-0.175pt]{0.350pt}{0.923pt}}
\put(872,444){\rule[-0.175pt]{0.350pt}{0.923pt}}
\put(873,440){\rule[-0.175pt]{0.350pt}{0.923pt}}
\put(874,436){\rule[-0.175pt]{0.350pt}{0.923pt}}
\put(875,432){\rule[-0.175pt]{0.350pt}{0.923pt}}
\put(876,429){\rule[-0.175pt]{0.350pt}{0.923pt}}
\put(877,424){\rule[-0.175pt]{0.350pt}{1.156pt}}
\put(878,419){\rule[-0.175pt]{0.350pt}{1.156pt}}
\put(879,414){\rule[-0.175pt]{0.350pt}{1.156pt}}
\put(880,409){\rule[-0.175pt]{0.350pt}{1.156pt}}
\put(881,405){\rule[-0.175pt]{0.350pt}{1.156pt}}
\put(882,401){\rule[-0.175pt]{0.350pt}{0.964pt}}
\put(883,397){\rule[-0.175pt]{0.350pt}{0.964pt}}
\put(884,393){\rule[-0.175pt]{0.350pt}{0.964pt}}
\put(885,389){\rule[-0.175pt]{0.350pt}{0.964pt}}
\put(886,385){\rule[-0.175pt]{0.350pt}{0.964pt}}
\put(887,381){\rule[-0.175pt]{0.350pt}{0.964pt}}
\put(888,377){\rule[-0.175pt]{0.350pt}{0.923pt}}
\put(889,373){\rule[-0.175pt]{0.350pt}{0.923pt}}
\put(890,369){\rule[-0.175pt]{0.350pt}{0.923pt}}
\put(891,365){\rule[-0.175pt]{0.350pt}{0.923pt}}
\put(892,361){\rule[-0.175pt]{0.350pt}{0.923pt}}
\put(893,358){\rule[-0.175pt]{0.350pt}{0.923pt}}
\put(894,354){\rule[-0.175pt]{0.350pt}{0.964pt}}
\put(895,350){\rule[-0.175pt]{0.350pt}{0.964pt}}
\put(896,346){\rule[-0.175pt]{0.350pt}{0.964pt}}
\put(897,342){\rule[-0.175pt]{0.350pt}{0.964pt}}
\put(898,338){\rule[-0.175pt]{0.350pt}{0.964pt}}
\put(899,334){\rule[-0.175pt]{0.350pt}{0.964pt}}
\put(900,330){\rule[-0.175pt]{0.350pt}{0.964pt}}
\put(901,326){\rule[-0.175pt]{0.350pt}{0.964pt}}
\put(902,322){\rule[-0.175pt]{0.350pt}{0.964pt}}
\put(903,318){\rule[-0.175pt]{0.350pt}{0.964pt}}
\put(904,314){\rule[-0.175pt]{0.350pt}{0.964pt}}
\put(905,310){\rule[-0.175pt]{0.350pt}{0.964pt}}
\put(906,306){\rule[-0.175pt]{0.350pt}{0.964pt}}
\put(907,302){\rule[-0.175pt]{0.350pt}{0.964pt}}
\put(908,298){\rule[-0.175pt]{0.350pt}{0.964pt}}
\put(909,294){\rule[-0.175pt]{0.350pt}{0.964pt}}
\put(910,290){\rule[-0.175pt]{0.350pt}{0.964pt}}
\put(911,286){\rule[-0.175pt]{0.350pt}{0.964pt}}
\put(912,282){\rule[-0.175pt]{0.350pt}{0.923pt}}
\put(913,278){\rule[-0.175pt]{0.350pt}{0.923pt}}
\put(914,274){\rule[-0.175pt]{0.350pt}{0.923pt}}
\put(915,270){\rule[-0.175pt]{0.350pt}{0.923pt}}
\put(916,266){\rule[-0.175pt]{0.350pt}{0.923pt}}
\put(917,263){\rule[-0.175pt]{0.350pt}{0.923pt}}
\put(918,261){\rule[-0.175pt]{0.350pt}{0.442pt}}
\put(919,259){\rule[-0.175pt]{0.350pt}{0.442pt}}
\put(920,257){\rule[-0.175pt]{0.350pt}{0.442pt}}
\put(921,255){\rule[-0.175pt]{0.350pt}{0.442pt}}
\put(922,253){\rule[-0.175pt]{0.350pt}{0.442pt}}
\put(923,252){\rule[-0.175pt]{0.350pt}{0.442pt}}
\put(924,252){\rule[-0.175pt]{38.303pt}{0.350pt}}
\put(1083,252){\rule[-0.175pt]{0.350pt}{1.124pt}}
\put(1084,256){\rule[-0.175pt]{0.350pt}{1.124pt}}
\put(1085,261){\rule[-0.175pt]{0.350pt}{1.124pt}}
\put(1086,265){\rule[-0.175pt]{0.350pt}{1.124pt}}
\put(1087,270){\rule[-0.175pt]{0.350pt}{1.124pt}}
\put(1088,275){\rule[-0.175pt]{0.350pt}{1.124pt}}
\put(1089,279){\rule[-0.175pt]{0.350pt}{1.927pt}}
\put(1090,288){\rule[-0.175pt]{0.350pt}{1.927pt}}
\put(1091,296){\rule[-0.175pt]{0.350pt}{1.927pt}}
\put(1092,304){\rule[-0.175pt]{0.350pt}{1.927pt}}
\put(1093,312){\rule[-0.175pt]{0.350pt}{1.927pt}}
\put(1094,320){\rule[-0.175pt]{0.350pt}{1.566pt}}
\put(1095,326){\rule[-0.175pt]{0.350pt}{1.566pt}}
\put(1096,333){\rule[-0.175pt]{0.350pt}{1.566pt}}
\put(1097,339){\rule[-0.175pt]{0.350pt}{1.566pt}}
\put(1098,346){\rule[-0.175pt]{0.350pt}{1.566pt}}
\put(1099,352){\rule[-0.175pt]{0.350pt}{1.566pt}}
\put(1100,359){\rule[-0.175pt]{0.350pt}{1.606pt}}
\put(1101,365){\rule[-0.175pt]{0.350pt}{1.606pt}}
\put(1102,372){\rule[-0.175pt]{0.350pt}{1.606pt}}
\put(1103,378){\rule[-0.175pt]{0.350pt}{1.606pt}}
\put(1104,385){\rule[-0.175pt]{0.350pt}{1.606pt}}
\put(1105,392){\rule[-0.175pt]{0.350pt}{1.606pt}}
\put(1106,398){\rule[-0.175pt]{0.350pt}{1.566pt}}
\put(1107,405){\rule[-0.175pt]{0.350pt}{1.566pt}}
\put(1108,412){\rule[-0.175pt]{0.350pt}{1.566pt}}
\put(1109,418){\rule[-0.175pt]{0.350pt}{1.566pt}}
\put(1110,425){\rule[-0.175pt]{0.350pt}{1.566pt}}
\put(1111,431){\rule[-0.175pt]{0.350pt}{1.566pt}}
\put(1112,438){\rule[-0.175pt]{0.350pt}{1.606pt}}
\put(1113,444){\rule[-0.175pt]{0.350pt}{1.606pt}}
\put(1114,451){\rule[-0.175pt]{0.350pt}{1.606pt}}
\put(1115,457){\rule[-0.175pt]{0.350pt}{1.606pt}}
\put(1116,464){\rule[-0.175pt]{0.350pt}{1.606pt}}
\put(1117,471){\rule[-0.175pt]{0.350pt}{1.606pt}}
\put(1118,477){\rule[-0.175pt]{0.350pt}{1.566pt}}
\put(1119,484){\rule[-0.175pt]{0.350pt}{1.566pt}}
\put(1120,491){\rule[-0.175pt]{0.350pt}{1.566pt}}
\put(1121,497){\rule[-0.175pt]{0.350pt}{1.566pt}}
\put(1122,504){\rule[-0.175pt]{0.350pt}{1.566pt}}
\put(1123,510){\rule[-0.175pt]{0.350pt}{1.566pt}}
\put(1124,517){\rule[-0.175pt]{0.350pt}{1.606pt}}
\put(1125,523){\rule[-0.175pt]{0.350pt}{1.606pt}}
\put(1126,530){\rule[-0.175pt]{0.350pt}{1.606pt}}
\put(1127,537){\rule[-0.175pt]{0.350pt}{1.606pt}}
\put(1128,543){\rule[-0.175pt]{0.350pt}{1.606pt}}
\put(1129,550){\rule[-0.175pt]{0.350pt}{1.606pt}}
\put(1130,557){\rule[-0.175pt]{0.350pt}{1.566pt}}
\put(1131,563){\rule[-0.175pt]{0.350pt}{1.566pt}}
\put(1132,570){\rule[-0.175pt]{0.350pt}{1.566pt}}
\put(1133,576){\rule[-0.175pt]{0.350pt}{1.566pt}}
\put(1134,583){\rule[-0.175pt]{0.350pt}{1.566pt}}
\put(1135,589){\rule[-0.175pt]{0.350pt}{1.566pt}}
\put(1136,596){\rule[-0.175pt]{0.350pt}{1.606pt}}
\put(1137,602){\rule[-0.175pt]{0.350pt}{1.606pt}}
\put(1138,609){\rule[-0.175pt]{0.350pt}{1.606pt}}
\put(1139,616){\rule[-0.175pt]{0.350pt}{1.606pt}}
\put(1140,622){\rule[-0.175pt]{0.350pt}{1.606pt}}
\put(1141,629){\rule[-0.175pt]{0.350pt}{1.606pt}}
\put(1142,636){\rule[-0.175pt]{0.350pt}{1.879pt}}
\put(1143,643){\rule[-0.175pt]{0.350pt}{1.879pt}}
\put(1144,651){\rule[-0.175pt]{0.350pt}{1.879pt}}
\put(1145,659){\rule[-0.175pt]{0.350pt}{1.879pt}}
\put(1146,667){\rule[-0.175pt]{0.350pt}{1.879pt}}
\put(1147,674){\rule[-0.175pt]{0.350pt}{1.606pt}}
\put(1148,681){\rule[-0.175pt]{0.350pt}{1.606pt}}
\put(1149,688){\rule[-0.175pt]{0.350pt}{1.606pt}}
\put(1150,695){\rule[-0.175pt]{0.350pt}{1.606pt}}
\put(1151,701){\rule[-0.175pt]{0.350pt}{1.606pt}}
\put(1152,708){\rule[-0.175pt]{0.350pt}{1.606pt}}
\put(1153,715){\rule[-0.175pt]{0.350pt}{0.361pt}}
\put(1154,716){\rule[-0.175pt]{0.350pt}{0.361pt}}
\put(1155,718){\rule[-0.175pt]{0.350pt}{0.361pt}}
\put(1156,719){\rule[-0.175pt]{0.350pt}{0.361pt}}
\put(1157,721){\rule[-0.175pt]{0.350pt}{0.361pt}}
\put(1158,722){\rule[-0.175pt]{0.350pt}{0.361pt}}
%\put(1159,724){\rule[-0.175pt]{66.729pt}{0.350pt}}
\put(1159,724){\rule[-0.175pt]{10pt}{0.350pt}}
\sbox{\plotpoint}{\rule[-0.250pt]{0.500pt}{0.500pt}}%
%\put(264,252){\usebox{\plotpoint}}
%\put(264,252){\usebox{\plotpoint}}
%\put(284,252){\usebox{\plotpoint}}
%\put(305,252){\usebox{\plotpoint}}
%\put(326,252){\usebox{\plotpoint}}
%\put(347,252){\usebox{\plotpoint}}
%\put(367,252){\usebox{\plotpoint}}
%\put(388,252){\usebox{\plotpoint}}
%\put(409,252){\usebox{\plotpoint}}
%\put(430,252){\usebox{\plotpoint}}
%\put(450,252){\usebox{\plotpoint}}
%\put(471,252){\usebox{\plotpoint}}
%\put(492,252){\usebox{\plotpoint}}
%\put(501,268){\usebox{\plotpoint}}
%\put(503,289){\usebox{\plotpoint}}
%\put(506,310){\usebox{\plotpoint}}
%\put(509,330){\usebox{\plotpoint}}
%\put(512,351){\usebox{\plotpoint}}
%\put(515,371){\usebox{\plotpoint}}
%\put(519,392){\usebox{\plotpoint}}
%\put(522,412){\usebox{\plotpoint}}
%\put(525,433){\usebox{\plotpoint}}
%\put(528,453){\usebox{\plotpoint}}
%\put(531,474){\usebox{\plotpoint}}
%\put(534,494){\usebox{\plotpoint}}
%\put(537,515){\usebox{\plotpoint}}
%\put(540,535){\usebox{\plotpoint}}
%\put(543,556){\usebox{\plotpoint}}
%\put(547,576){\usebox{\plotpoint}}
%\put(550,597){\usebox{\plotpoint}}
%\put(553,617){\usebox{\plotpoint}}
%\put(555,638){\usebox{\plotpoint}}
%\put(558,659){\usebox{\plotpoint}}
%\put(561,679){\usebox{\plotpoint}}
%\put(565,700){\usebox{\plotpoint}}
%\put(569,720){\usebox{\plotpoint}}
%\put(587,724){\usebox{\plotpoint}}
%\put(607,724){\usebox{\plotpoint}}
%\put(628,724){\usebox{\plotpoint}}
%\put(649,724){\usebox{\plotpoint}}
%\put(670,724){\usebox{\plotpoint}}
%\put(690,724){\usebox{\plotpoint}}
%\put(711,724){\usebox{\plotpoint}}
%\put(732,724){\usebox{\plotpoint}}
%\put(753,724){\usebox{\plotpoint}}
%\put(773,724){\usebox{\plotpoint}}
%\put(794,724){\usebox{\plotpoint}}
%\put(806,710){\usebox{\plotpoint}}
%\put(811,690){\usebox{\plotpoint}}
%\put(816,669){\usebox{\plotpoint}}
%\put(821,649){\usebox{\plotpoint}}
%\put(826,629){\usebox{\plotpoint}}
%\put(831,609){\usebox{\plotpoint}}
%\put(836,589){\usebox{\plotpoint}}
%\put(841,569){\usebox{\plotpoint}}
%\put(846,548){\usebox{\plotpoint}}
%\put(851,528){\usebox{\plotpoint}}
%\put(856,508){\usebox{\plotpoint}}
%\put(861,488){\usebox{\plotpoint}}
%\put(866,468){\usebox{\plotpoint}}
%\put(871,448){\usebox{\plotpoint}}
%\put(877,428){\usebox{\plotpoint}}
%\put(881,407){\usebox{\plotpoint}}
%\put(886,387){\usebox{\plotpoint}}
%\put(891,367){\usebox{\plotpoint}}
%\put(896,347){\usebox{\plotpoint}}
%\put(901,327){\usebox{\plotpoint}}
%\put(906,307){\usebox{\plotpoint}}
%\put(911,287){\usebox{\plotpoint}}
%\put(916,267){\usebox{\plotpoint}}
%\put(927,252){\usebox{\plotpoint}}
%\put(948,252){\usebox{\plotpoint}}
%\put(969,252){\usebox{\plotpoint}}
%\put(990,252){\usebox{\plotpoint}}
%\put(1010,252){\usebox{\plotpoint}}
%\put(1031,252){\usebox{\plotpoint}}
%\put(1052,252){\usebox{\plotpoint}}
%\put(1073,252){\usebox{\plotpoint}}
%\put(1085,262){\usebox{\plotpoint}}
%\put(1089,283){\usebox{\plotpoint}}
%\put(1091,303){\usebox{\plotpoint}}
%\put(1094,324){\usebox{\plotpoint}}
%\put(1097,344){\usebox{\plotpoint}}
%\put(1100,365){\usebox{\plotpoint}}
%\put(1104,385){\usebox{\plotpoint}}
%\put(1107,406){\usebox{\plotpoint}}
%\put(1110,426){\usebox{\plotpoint}}
%\put(1113,447){\usebox{\plotpoint}}
%\put(1116,467){\usebox{\plotpoint}}
%\put(1119,488){\usebox{\plotpoint}}
%\put(1122,508){\usebox{\plotpoint}}
%\put(1125,529){\usebox{\plotpoint}}
%\put(1128,550){\usebox{\plotpoint}}
%\put(1132,570){\usebox{\plotpoint}}
%\put(1135,591){\usebox{\plotpoint}}
%\put(1138,611){\usebox{\plotpoint}}
%\put(1141,632){\usebox{\plotpoint}}
%\put(1144,652){\usebox{\plotpoint}}
%\put(1146,673){\usebox{\plotpoint}}
%\put(1149,693){\usebox{\plotpoint}}
%\put(1152,714){\usebox{\plotpoint}}
\put(1168,724){\usebox{\plotpoint}}
\put(1188,724){\usebox{\plotpoint}}
\put(1209,724){\usebox{\plotpoint}}
\put(1230,724){\usebox{\plotpoint}}
\put(1251,724){\usebox{\plotpoint}}
\put(1272,724){\usebox{\plotpoint}}
\put(1292,724){\usebox{\plotpoint}}
\put(1313,724){\usebox{\plotpoint}}
\put(1334,724){\usebox{\plotpoint}}
\put(1355,724){\usebox{\plotpoint}}
\put(1375,724){\usebox{\plotpoint}}
\put(1396,724){\usebox{\plotpoint}}
\put(1417,724){\usebox{\plotpoint}}
\put(1436,724){\usebox{\plotpoint}}
\put(100,480){$i$}
\put(90,700){(A)}
\put(820,40){$t$}
\put(1300,40){(s)}
\put(300,724){$I_2$}
\put(300,270){$I_1$}
\put(364,435){$T_{D}$}
\put(264,435){$\longleftarrow$}
\put(420,435){$\longrightarrow$}
\put(494,408){\rule{1pt}{20pt}}
\put(530,380){$T_{R}$}
\put(494,500){\rule{1pt}{20pt}}
\put(570,408){\rule{1pt}{20pt}}
\put(800,408){\rule{1pt}{20pt}}
\put(810,437){$T_{F}$}
\put(670,435){$W$}
\put(570,435){$\longleftarrow$}
\put(730,435){$\longrightarrow$}
\put(1083,500){\rule{1pt}{20pt}}
\put(924,408){\rule{1pt}{20pt}}
%\put(264,8){\rule{1pt}{20pt}}
\put(360,724){\rule[-0.175pt]{20pt}{0.683pt}}
\put(494,550){\rule[-0.175pt]{60pt}{0.683pt}}
\put(827,550){\rule[-0.175pt]{62pt}{0.683pt}}
\put(494,540){$\longleftarrow$}
\put(770,540){$T$}
\put(1000,540){$\longrightarrow$}
\end{picture}

\caption[Voltage source transient pulse ({\tt PULSE}) waveform]{Voltage source
transient pulse ({\tt PULSE}) waveform for\newline \hspace*{\fill}
{\tt PULSE(0.3 1.8 1 2.5 0.3 1 0.7)} \hspace*{\fill} \label{fig:vpulse}}
\end{figure}

\noindent{\underline{\bf Piece-Wise Linear:}}
\\[0.2in]
\form{{\tt PWL(} $T_1$ $V_1$\B $T_2$ $V_2$ ... $T_i$ $V_i$ ... $T_N$ $V_N$ \E )}
Each pair of values ($T_i$, $V_i$) specifies that  the  value
of  the  source  is $V_i$ at time = $T_i$. At times between $T_i$ and
$T_{i+1}$ the values are linearly interpolated.
If $T_1 >$ 0 then the voltage is constant at {\it DCValue}
(specified on the element line) until time $T_1$.
\begin{equation}
v = \left\{ \begin{array}{ll}
    {\it DCvalue}& t < T_1\\
    V_i         & t = T_i\\
    V_{i+1}     & t = T_{i+1}\\
    V_i + \left({{t-T_i} \over {T_{i+1} - T_i}}\right)(V_{i+1}-V_i)
                & T_i < t \le T_{i+1}\\
    V_N         & t > T_N\\
     \end{array} \right. %}
\end{equation}
\begin{figure}[hbp]
\centering
%set samples 200
%i(x) = (x < 0.5)? 0.25 :( (x < 1)? -0.5 + 1.50*x : ( (x<2)? 1.5-x/2: ( (x<3)? 0.5:\
%       ( (x<4)? -1+x/2 : x/4))))
%set term latex
%set output 'ipwl.tex'
%set size 1.0,1.0
%set border
%set nokey
%set xtics
%set ytics
%set xzeroaxis
%set yzeroaxis
%plot [0:5] i(x) with line 1, i(x) with line 4
%set term x11
%plot [0:5] i(x) with line 1, i(x) with line 4
% GNUPLOT: LaTeX picture
\setlength{\unitlength}{0.240900pt}
\ifx\plotpoint\undefined\newsavebox{\plotpoint}\fi
\sbox{\plotpoint}{\rule[-0.175pt]{0.350pt}{0.350pt}}%
\begin{picture}(1500,900)(0,0)
%\tenrm
\sbox{\plotpoint}{\rule[-0.175pt]{0.350pt}{0.350pt}}%
\put(264,158){\rule[-0.175pt]{282.335pt}{0.350pt}}
\put(264,158){\rule[-0.175pt]{0.350pt}{151.526pt}}
\put(264,158){\rule[-0.175pt]{4.818pt}{0.350pt}}
\put(242,158){\makebox(0,0)[r]{0}}
\put(1416,158){\rule[-0.175pt]{4.818pt}{0.350pt}}
\put(264,221){\rule[-0.175pt]{4.818pt}{0.350pt}}
\put(242,221){\makebox(0,0)[r]{0.2}}
\put(1416,221){\rule[-0.175pt]{4.818pt}{0.350pt}}
\put(264,284){\rule[-0.175pt]{4.818pt}{0.350pt}}
\put(242,284){\makebox(0,0)[r]{0.4}}
\put(1416,284){\rule[-0.175pt]{4.818pt}{0.350pt}}
\put(264,347){\rule[-0.175pt]{4.818pt}{0.350pt}}
\put(242,347){\makebox(0,0)[r]{0.6}}
\put(1416,347){\rule[-0.175pt]{4.818pt}{0.350pt}}
\put(264,410){\rule[-0.175pt]{4.818pt}{0.350pt}}
\put(242,410){\makebox(0,0)[r]{0.8}}
\put(1416,410){\rule[-0.175pt]{4.818pt}{0.350pt}}
\put(264,473){\rule[-0.175pt]{4.818pt}{0.350pt}}
\put(242,473){\makebox(0,0)[r]{1}}
\put(1416,473){\rule[-0.175pt]{4.818pt}{0.350pt}}
\put(264,535){\rule[-0.175pt]{4.818pt}{0.350pt}}
\put(242,535){\makebox(0,0)[r]{1.2}}
\put(1416,535){\rule[-0.175pt]{4.818pt}{0.350pt}}
\put(264,598){\rule[-0.175pt]{4.818pt}{0.350pt}}
\put(242,598){\makebox(0,0)[r]{1.4}}
\put(1416,598){\rule[-0.175pt]{4.818pt}{0.350pt}}
\put(264,661){\rule[-0.175pt]{4.818pt}{0.350pt}}
\put(242,661){\makebox(0,0)[r]{1.6}}
\put(1416,661){\rule[-0.175pt]{4.818pt}{0.350pt}}
\put(264,724){\rule[-0.175pt]{4.818pt}{0.350pt}}
\put(242,724){\makebox(0,0)[r]{1.8}}
\put(1416,724){\rule[-0.175pt]{4.818pt}{0.350pt}}
\put(264,787){\rule[-0.175pt]{4.818pt}{0.350pt}}
\put(242,787){\makebox(0,0)[r]{2}}
\put(1416,787){\rule[-0.175pt]{4.818pt}{0.350pt}}
\put(264,158){\rule[-0.175pt]{0.350pt}{4.818pt}}
\put(264,113){\makebox(0,0){0}}
\put(264,767){\rule[-0.175pt]{0.350pt}{4.818pt}}
\put(381,158){\rule[-0.175pt]{0.350pt}{4.818pt}}
\put(381,113){\makebox(0,0){0.5}}
\put(381,767){\rule[-0.175pt]{0.350pt}{4.818pt}}
\put(498,158){\rule[-0.175pt]{0.350pt}{4.818pt}}
\put(498,113){\makebox(0,0){1}}
\put(498,767){\rule[-0.175pt]{0.350pt}{4.818pt}}
\put(616,158){\rule[-0.175pt]{0.350pt}{4.818pt}}
\put(616,113){\makebox(0,0){1.5}}
\put(616,767){\rule[-0.175pt]{0.350pt}{4.818pt}}
\put(733,158){\rule[-0.175pt]{0.350pt}{4.818pt}}
\put(733,113){\makebox(0,0){2}}
\put(733,767){\rule[-0.175pt]{0.350pt}{4.818pt}}
\put(850,158){\rule[-0.175pt]{0.350pt}{4.818pt}}
\put(850,113){\makebox(0,0){2.5}}
\put(850,767){\rule[-0.175pt]{0.350pt}{4.818pt}}
\put(967,158){\rule[-0.175pt]{0.350pt}{4.818pt}}
\put(967,113){\makebox(0,0){3}}
\put(967,767){\rule[-0.175pt]{0.350pt}{4.818pt}}
\put(1084,158){\rule[-0.175pt]{0.350pt}{4.818pt}}
\put(1084,113){\makebox(0,0){3.5}}
\put(1084,767){\rule[-0.175pt]{0.350pt}{4.818pt}}
\put(1202,158){\rule[-0.175pt]{0.350pt}{4.818pt}}
\put(1202,113){\makebox(0,0){4}}
\put(1202,767){\rule[-0.175pt]{0.350pt}{4.818pt}}
\put(1319,158){\rule[-0.175pt]{0.350pt}{4.818pt}}
\put(1319,113){\makebox(0,0){4.5}}
\put(1319,767){\rule[-0.175pt]{0.350pt}{4.818pt}}
\put(1436,158){\rule[-0.175pt]{0.350pt}{4.818pt}}
\put(1436,113){\makebox(0,0){5}}
\put(1436,767){\rule[-0.175pt]{0.350pt}{4.818pt}}
\put(264,158){\rule[-0.175pt]{282.335pt}{0.350pt}}
\put(1436,158){\rule[-0.175pt]{0.350pt}{151.526pt}}
\put(264,787){\rule[-0.175pt]{282.335pt}{0.350pt}}
\put(264,158){\rule[-0.175pt]{0.350pt}{151.526pt}}
\put(264,237){\usebox{\plotpoint}}
\put(264,237){\rule[-0.175pt]{28.426pt}{0.350pt}}
\put(382,238){\rule[-0.175pt]{0.350pt}{0.482pt}}
\put(383,240){\rule[-0.175pt]{0.350pt}{0.482pt}}
\put(384,242){\rule[-0.175pt]{0.350pt}{0.482pt}}
\put(385,244){\rule[-0.175pt]{0.350pt}{0.482pt}}
\put(386,246){\rule[-0.175pt]{0.350pt}{0.482pt}}
\put(387,248){\rule[-0.175pt]{0.350pt}{0.482pt}}
\put(388,250){\rule[-0.175pt]{0.350pt}{0.482pt}}
\put(389,252){\rule[-0.175pt]{0.350pt}{0.482pt}}
\put(390,254){\rule[-0.175pt]{0.350pt}{0.482pt}}
\put(391,256){\rule[-0.175pt]{0.350pt}{0.482pt}}
\put(392,258){\rule[-0.175pt]{0.350pt}{0.482pt}}
\put(393,260){\rule[-0.175pt]{0.350pt}{0.482pt}}
\put(394,262){\rule[-0.175pt]{0.350pt}{0.530pt}}
\put(395,264){\rule[-0.175pt]{0.350pt}{0.530pt}}
\put(396,266){\rule[-0.175pt]{0.350pt}{0.530pt}}
\put(397,268){\rule[-0.175pt]{0.350pt}{0.530pt}}
\put(398,270){\rule[-0.175pt]{0.350pt}{0.530pt}}
\put(399,273){\rule[-0.175pt]{0.350pt}{0.482pt}}
\put(400,275){\rule[-0.175pt]{0.350pt}{0.482pt}}
\put(401,277){\rule[-0.175pt]{0.350pt}{0.482pt}}
\put(402,279){\rule[-0.175pt]{0.350pt}{0.482pt}}
\put(403,281){\rule[-0.175pt]{0.350pt}{0.482pt}}
\put(404,283){\rule[-0.175pt]{0.350pt}{0.482pt}}
\put(405,285){\rule[-0.175pt]{0.350pt}{0.482pt}}
\put(406,287){\rule[-0.175pt]{0.350pt}{0.482pt}}
\put(407,289){\rule[-0.175pt]{0.350pt}{0.482pt}}
\put(408,291){\rule[-0.175pt]{0.350pt}{0.482pt}}
\put(409,293){\rule[-0.175pt]{0.350pt}{0.482pt}}
\put(410,295){\rule[-0.175pt]{0.350pt}{0.482pt}}
\put(411,297){\rule[-0.175pt]{0.350pt}{0.482pt}}
\put(412,299){\rule[-0.175pt]{0.350pt}{0.482pt}}
\put(413,301){\rule[-0.175pt]{0.350pt}{0.482pt}}
\put(414,303){\rule[-0.175pt]{0.350pt}{0.482pt}}
\put(415,305){\rule[-0.175pt]{0.350pt}{0.482pt}}
\put(416,307){\rule[-0.175pt]{0.350pt}{0.482pt}}
\put(417,309){\rule[-0.175pt]{0.350pt}{0.482pt}}
\put(418,311){\rule[-0.175pt]{0.350pt}{0.482pt}}
\put(419,313){\rule[-0.175pt]{0.350pt}{0.482pt}}
\put(420,315){\rule[-0.175pt]{0.350pt}{0.482pt}}
\put(421,317){\rule[-0.175pt]{0.350pt}{0.482pt}}
\put(422,319){\rule[-0.175pt]{0.350pt}{0.482pt}}
\put(423,321){\rule[-0.175pt]{0.350pt}{0.482pt}}
\put(424,323){\rule[-0.175pt]{0.350pt}{0.482pt}}
\put(425,325){\rule[-0.175pt]{0.350pt}{0.482pt}}
\put(426,327){\rule[-0.175pt]{0.350pt}{0.482pt}}
\put(427,329){\rule[-0.175pt]{0.350pt}{0.482pt}}
\put(428,331){\rule[-0.175pt]{0.350pt}{0.482pt}}
\put(429,333){\rule[-0.175pt]{0.350pt}{0.442pt}}
\put(430,334){\rule[-0.175pt]{0.350pt}{0.442pt}}
\put(431,336){\rule[-0.175pt]{0.350pt}{0.442pt}}
\put(432,338){\rule[-0.175pt]{0.350pt}{0.442pt}}
\put(433,340){\rule[-0.175pt]{0.350pt}{0.442pt}}
\put(434,342){\rule[-0.175pt]{0.350pt}{0.442pt}}
\put(435,344){\rule[-0.175pt]{0.350pt}{0.482pt}}
\put(436,346){\rule[-0.175pt]{0.350pt}{0.482pt}}
\put(437,348){\rule[-0.175pt]{0.350pt}{0.482pt}}
\put(438,350){\rule[-0.175pt]{0.350pt}{0.482pt}}
\put(439,352){\rule[-0.175pt]{0.350pt}{0.482pt}}
\put(440,354){\rule[-0.175pt]{0.350pt}{0.482pt}}
\put(441,356){\rule[-0.175pt]{0.350pt}{0.482pt}}
\put(442,358){\rule[-0.175pt]{0.350pt}{0.482pt}}
\put(443,360){\rule[-0.175pt]{0.350pt}{0.482pt}}
\put(444,362){\rule[-0.175pt]{0.350pt}{0.482pt}}
\put(445,364){\rule[-0.175pt]{0.350pt}{0.482pt}}
\put(446,366){\rule[-0.175pt]{0.350pt}{0.482pt}}
\put(447,368){\rule[-0.175pt]{0.350pt}{0.578pt}}
\put(448,370){\rule[-0.175pt]{0.350pt}{0.578pt}}
\put(449,372){\rule[-0.175pt]{0.350pt}{0.578pt}}
\put(450,375){\rule[-0.175pt]{0.350pt}{0.578pt}}
\put(451,377){\rule[-0.175pt]{0.350pt}{0.578pt}}
\put(452,379){\rule[-0.175pt]{0.350pt}{0.482pt}}
\put(453,382){\rule[-0.175pt]{0.350pt}{0.482pt}}
\put(454,384){\rule[-0.175pt]{0.350pt}{0.482pt}}
\put(455,386){\rule[-0.175pt]{0.350pt}{0.482pt}}
\put(456,388){\rule[-0.175pt]{0.350pt}{0.482pt}}
\put(457,390){\rule[-0.175pt]{0.350pt}{0.482pt}}
\put(458,392){\rule[-0.175pt]{0.350pt}{0.482pt}}
\put(459,394){\rule[-0.175pt]{0.350pt}{0.482pt}}
\put(460,396){\rule[-0.175pt]{0.350pt}{0.482pt}}
\put(461,398){\rule[-0.175pt]{0.350pt}{0.482pt}}
\put(462,400){\rule[-0.175pt]{0.350pt}{0.482pt}}
\put(463,402){\rule[-0.175pt]{0.350pt}{0.482pt}}
\put(464,404){\rule[-0.175pt]{0.350pt}{0.482pt}}
\put(465,406){\rule[-0.175pt]{0.350pt}{0.482pt}}
\put(466,408){\rule[-0.175pt]{0.350pt}{0.482pt}}
\put(467,410){\rule[-0.175pt]{0.350pt}{0.482pt}}
\put(468,412){\rule[-0.175pt]{0.350pt}{0.482pt}}
\put(469,414){\rule[-0.175pt]{0.350pt}{0.482pt}}
\put(470,416){\rule[-0.175pt]{0.350pt}{0.442pt}}
\put(471,417){\rule[-0.175pt]{0.350pt}{0.442pt}}
\put(472,419){\rule[-0.175pt]{0.350pt}{0.442pt}}
\put(473,421){\rule[-0.175pt]{0.350pt}{0.442pt}}
\put(474,423){\rule[-0.175pt]{0.350pt}{0.442pt}}
\put(475,425){\rule[-0.175pt]{0.350pt}{0.442pt}}
\put(476,427){\rule[-0.175pt]{0.350pt}{0.482pt}}
\put(477,429){\rule[-0.175pt]{0.350pt}{0.482pt}}
\put(478,431){\rule[-0.175pt]{0.350pt}{0.482pt}}
\put(479,433){\rule[-0.175pt]{0.350pt}{0.482pt}}
\put(480,435){\rule[-0.175pt]{0.350pt}{0.482pt}}
\put(481,437){\rule[-0.175pt]{0.350pt}{0.482pt}}
\put(482,439){\rule[-0.175pt]{0.350pt}{0.482pt}}
\put(483,441){\rule[-0.175pt]{0.350pt}{0.482pt}}
\put(484,443){\rule[-0.175pt]{0.350pt}{0.482pt}}
\put(485,445){\rule[-0.175pt]{0.350pt}{0.482pt}}
\put(486,447){\rule[-0.175pt]{0.350pt}{0.482pt}}
\put(487,449){\rule[-0.175pt]{0.350pt}{0.482pt}}
\put(488,451){\rule[-0.175pt]{0.350pt}{0.482pt}}
\put(489,453){\rule[-0.175pt]{0.350pt}{0.482pt}}
\put(490,455){\rule[-0.175pt]{0.350pt}{0.482pt}}
\put(491,457){\rule[-0.175pt]{0.350pt}{0.482pt}}
\put(492,459){\rule[-0.175pt]{0.350pt}{0.482pt}}
\put(493,461){\rule[-0.175pt]{0.350pt}{0.482pt}}
\put(494,463){\rule[-0.175pt]{0.350pt}{0.361pt}}
\put(495,464){\rule[-0.175pt]{0.350pt}{0.361pt}}
\put(496,466){\rule[-0.175pt]{0.350pt}{0.361pt}}
\put(497,467){\rule[-0.175pt]{0.350pt}{0.361pt}}
\put(498,469){\rule[-0.175pt]{0.350pt}{0.361pt}}
\put(499,470){\rule[-0.175pt]{0.350pt}{0.361pt}}
\put(500,472){\usebox{\plotpoint}}
\put(501,471){\usebox{\plotpoint}}
\put(502,470){\usebox{\plotpoint}}
\put(503,469){\usebox{\plotpoint}}
\put(505,468){\rule[-0.175pt]{0.361pt}{0.350pt}}
\put(506,467){\rule[-0.175pt]{0.361pt}{0.350pt}}
\put(508,466){\rule[-0.175pt]{0.361pt}{0.350pt}}
\put(509,465){\rule[-0.175pt]{0.361pt}{0.350pt}}
\put(511,464){\rule[-0.175pt]{0.361pt}{0.350pt}}
\put(512,463){\rule[-0.175pt]{0.361pt}{0.350pt}}
\put(514,462){\rule[-0.175pt]{0.361pt}{0.350pt}}
\put(515,461){\rule[-0.175pt]{0.361pt}{0.350pt}}
\put(517,460){\rule[-0.175pt]{0.361pt}{0.350pt}}
\put(518,459){\rule[-0.175pt]{0.361pt}{0.350pt}}
\put(520,458){\rule[-0.175pt]{0.361pt}{0.350pt}}
\put(521,457){\rule[-0.175pt]{0.361pt}{0.350pt}}
\put(523,456){\rule[-0.175pt]{0.361pt}{0.350pt}}
\put(524,455){\rule[-0.175pt]{0.361pt}{0.350pt}}
\put(526,454){\rule[-0.175pt]{0.361pt}{0.350pt}}
\put(527,453){\rule[-0.175pt]{0.361pt}{0.350pt}}
\put(529,452){\rule[-0.175pt]{0.361pt}{0.350pt}}
\put(530,451){\rule[-0.175pt]{0.361pt}{0.350pt}}
\put(532,450){\rule[-0.175pt]{0.361pt}{0.350pt}}
\put(533,449){\rule[-0.175pt]{0.361pt}{0.350pt}}
\put(535,448){\rule[-0.175pt]{0.361pt}{0.350pt}}
\put(536,447){\rule[-0.175pt]{0.361pt}{0.350pt}}
\put(538,446){\rule[-0.175pt]{0.361pt}{0.350pt}}
\put(539,445){\rule[-0.175pt]{0.361pt}{0.350pt}}
\put(541,444){\rule[-0.175pt]{0.361pt}{0.350pt}}
\put(542,443){\rule[-0.175pt]{0.361pt}{0.350pt}}
\put(544,442){\rule[-0.175pt]{0.361pt}{0.350pt}}
\put(545,441){\rule[-0.175pt]{0.361pt}{0.350pt}}
\put(547,440){\rule[-0.175pt]{0.361pt}{0.350pt}}
\put(548,439){\rule[-0.175pt]{0.361pt}{0.350pt}}
\put(550,438){\rule[-0.175pt]{0.361pt}{0.350pt}}
\put(551,437){\rule[-0.175pt]{0.361pt}{0.350pt}}
\put(553,436){\usebox{\plotpoint}}
\put(554,435){\usebox{\plotpoint}}
\put(555,434){\usebox{\plotpoint}}
\put(556,433){\usebox{\plotpoint}}
\put(558,432){\rule[-0.175pt]{0.361pt}{0.350pt}}
\put(559,431){\rule[-0.175pt]{0.361pt}{0.350pt}}
\put(561,430){\rule[-0.175pt]{0.361pt}{0.350pt}}
\put(562,429){\rule[-0.175pt]{0.361pt}{0.350pt}}
\put(564,428){\rule[-0.175pt]{0.361pt}{0.350pt}}
\put(565,427){\rule[-0.175pt]{0.361pt}{0.350pt}}
\put(567,426){\rule[-0.175pt]{0.361pt}{0.350pt}}
\put(568,425){\rule[-0.175pt]{0.361pt}{0.350pt}}
\put(570,424){\rule[-0.175pt]{0.361pt}{0.350pt}}
\put(571,423){\rule[-0.175pt]{0.361pt}{0.350pt}}
\put(573,422){\rule[-0.175pt]{0.361pt}{0.350pt}}
\put(574,421){\rule[-0.175pt]{0.361pt}{0.350pt}}
\put(576,420){\rule[-0.175pt]{0.361pt}{0.350pt}}
\put(577,419){\rule[-0.175pt]{0.361pt}{0.350pt}}
\put(579,418){\rule[-0.175pt]{0.361pt}{0.350pt}}
\put(580,417){\rule[-0.175pt]{0.361pt}{0.350pt}}
\put(582,416){\rule[-0.175pt]{0.361pt}{0.350pt}}
\put(583,415){\rule[-0.175pt]{0.361pt}{0.350pt}}
\put(585,414){\rule[-0.175pt]{0.361pt}{0.350pt}}
\put(586,413){\rule[-0.175pt]{0.361pt}{0.350pt}}
\put(588,412){\rule[-0.175pt]{0.361pt}{0.350pt}}
\put(589,411){\rule[-0.175pt]{0.361pt}{0.350pt}}
\put(591,410){\rule[-0.175pt]{0.361pt}{0.350pt}}
\put(592,409){\rule[-0.175pt]{0.361pt}{0.350pt}}
\put(594,408){\rule[-0.175pt]{0.482pt}{0.350pt}}
\put(596,407){\rule[-0.175pt]{0.482pt}{0.350pt}}
\put(598,406){\rule[-0.175pt]{0.482pt}{0.350pt}}
\put(600,405){\rule[-0.175pt]{0.361pt}{0.350pt}}
\put(601,404){\rule[-0.175pt]{0.361pt}{0.350pt}}
\put(603,403){\rule[-0.175pt]{0.361pt}{0.350pt}}
\put(604,402){\rule[-0.175pt]{0.361pt}{0.350pt}}
\put(606,401){\usebox{\plotpoint}}
\put(607,400){\usebox{\plotpoint}}
\put(608,399){\usebox{\plotpoint}}
\put(609,398){\usebox{\plotpoint}}
\put(611,397){\rule[-0.175pt]{0.361pt}{0.350pt}}
\put(612,396){\rule[-0.175pt]{0.361pt}{0.350pt}}
\put(614,395){\rule[-0.175pt]{0.361pt}{0.350pt}}
\put(615,394){\rule[-0.175pt]{0.361pt}{0.350pt}}
\put(617,393){\rule[-0.175pt]{0.361pt}{0.350pt}}
\put(618,392){\rule[-0.175pt]{0.361pt}{0.350pt}}
\put(620,391){\rule[-0.175pt]{0.361pt}{0.350pt}}
\put(621,390){\rule[-0.175pt]{0.361pt}{0.350pt}}
\put(623,389){\rule[-0.175pt]{0.361pt}{0.350pt}}
\put(624,388){\rule[-0.175pt]{0.361pt}{0.350pt}}
\put(626,387){\rule[-0.175pt]{0.361pt}{0.350pt}}
\put(627,386){\rule[-0.175pt]{0.361pt}{0.350pt}}
\put(629,385){\rule[-0.175pt]{0.361pt}{0.350pt}}
\put(630,384){\rule[-0.175pt]{0.361pt}{0.350pt}}
\put(632,383){\rule[-0.175pt]{0.361pt}{0.350pt}}
\put(633,382){\rule[-0.175pt]{0.361pt}{0.350pt}}
\put(635,381){\rule[-0.175pt]{0.361pt}{0.350pt}}
\put(636,380){\rule[-0.175pt]{0.361pt}{0.350pt}}
\put(638,379){\rule[-0.175pt]{0.361pt}{0.350pt}}
\put(639,378){\rule[-0.175pt]{0.361pt}{0.350pt}}
\put(641,377){\rule[-0.175pt]{0.361pt}{0.350pt}}
\put(642,376){\rule[-0.175pt]{0.361pt}{0.350pt}}
\put(644,375){\rule[-0.175pt]{0.361pt}{0.350pt}}
\put(645,374){\rule[-0.175pt]{0.361pt}{0.350pt}}
\put(647,373){\rule[-0.175pt]{0.361pt}{0.350pt}}
\put(648,372){\rule[-0.175pt]{0.361pt}{0.350pt}}
\put(650,371){\rule[-0.175pt]{0.361pt}{0.350pt}}
\put(651,370){\rule[-0.175pt]{0.361pt}{0.350pt}}
\put(653,369){\rule[-0.175pt]{0.361pt}{0.350pt}}
\put(654,368){\rule[-0.175pt]{0.361pt}{0.350pt}}
\put(656,367){\rule[-0.175pt]{0.361pt}{0.350pt}}
\put(657,366){\rule[-0.175pt]{0.361pt}{0.350pt}}
\put(659,365){\usebox{\plotpoint}}
\put(660,364){\usebox{\plotpoint}}
\put(661,363){\usebox{\plotpoint}}
\put(662,362){\usebox{\plotpoint}}
\put(664,361){\rule[-0.175pt]{0.361pt}{0.350pt}}
\put(665,360){\rule[-0.175pt]{0.361pt}{0.350pt}}
\put(667,359){\rule[-0.175pt]{0.361pt}{0.350pt}}
\put(668,358){\rule[-0.175pt]{0.361pt}{0.350pt}}
\put(670,357){\rule[-0.175pt]{0.361pt}{0.350pt}}
\put(671,356){\rule[-0.175pt]{0.361pt}{0.350pt}}
\put(673,355){\rule[-0.175pt]{0.361pt}{0.350pt}}
\put(674,354){\rule[-0.175pt]{0.361pt}{0.350pt}}
\put(676,353){\rule[-0.175pt]{0.361pt}{0.350pt}}
\put(677,352){\rule[-0.175pt]{0.361pt}{0.350pt}}
\put(679,351){\rule[-0.175pt]{0.361pt}{0.350pt}}
\put(680,350){\rule[-0.175pt]{0.361pt}{0.350pt}}
\put(682,349){\rule[-0.175pt]{0.361pt}{0.350pt}}
\put(683,348){\rule[-0.175pt]{0.361pt}{0.350pt}}
\put(685,347){\rule[-0.175pt]{0.361pt}{0.350pt}}
\put(686,346){\rule[-0.175pt]{0.361pt}{0.350pt}}
\put(688,345){\rule[-0.175pt]{0.361pt}{0.350pt}}
\put(689,344){\rule[-0.175pt]{0.361pt}{0.350pt}}
\put(691,343){\rule[-0.175pt]{0.361pt}{0.350pt}}
\put(692,342){\rule[-0.175pt]{0.361pt}{0.350pt}}
\put(694,341){\rule[-0.175pt]{0.361pt}{0.350pt}}
\put(695,340){\rule[-0.175pt]{0.361pt}{0.350pt}}
\put(697,339){\rule[-0.175pt]{0.361pt}{0.350pt}}
\put(698,338){\rule[-0.175pt]{0.361pt}{0.350pt}}
\put(700,337){\rule[-0.175pt]{0.361pt}{0.350pt}}
\put(701,336){\rule[-0.175pt]{0.361pt}{0.350pt}}
\put(703,335){\rule[-0.175pt]{0.361pt}{0.350pt}}
\put(704,334){\rule[-0.175pt]{0.361pt}{0.350pt}}
\put(706,333){\rule[-0.175pt]{0.361pt}{0.350pt}}
\put(707,332){\rule[-0.175pt]{0.361pt}{0.350pt}}
\put(709,331){\rule[-0.175pt]{0.361pt}{0.350pt}}
\put(710,330){\rule[-0.175pt]{0.361pt}{0.350pt}}
\put(712,329){\rule[-0.175pt]{0.402pt}{0.350pt}}
\put(713,328){\rule[-0.175pt]{0.402pt}{0.350pt}}
\put(715,327){\rule[-0.175pt]{0.401pt}{0.350pt}}
\put(717,326){\rule[-0.175pt]{0.361pt}{0.350pt}}
\put(718,325){\rule[-0.175pt]{0.361pt}{0.350pt}}
\put(720,324){\rule[-0.175pt]{0.361pt}{0.350pt}}
\put(721,323){\rule[-0.175pt]{0.361pt}{0.350pt}}
\put(723,322){\rule[-0.175pt]{0.361pt}{0.350pt}}
\put(724,321){\rule[-0.175pt]{0.361pt}{0.350pt}}
\put(726,320){\rule[-0.175pt]{0.361pt}{0.350pt}}
\put(727,319){\rule[-0.175pt]{0.361pt}{0.350pt}}
\put(729,318){\rule[-0.175pt]{0.482pt}{0.350pt}}
\put(731,317){\rule[-0.175pt]{0.482pt}{0.350pt}}
\put(733,316){\rule[-0.175pt]{0.482pt}{0.350pt}}
%\put(735,315){\rule[-0.175pt]{55.889pt}{0.350pt}}
\put(967,316){\rule[-0.175pt]{0.482pt}{0.350pt}}
\put(969,317){\rule[-0.175pt]{0.482pt}{0.350pt}}
\put(971,318){\rule[-0.175pt]{0.361pt}{0.350pt}}
\put(972,319){\rule[-0.175pt]{0.361pt}{0.350pt}}
\put(974,320){\rule[-0.175pt]{0.361pt}{0.350pt}}
\put(975,321){\rule[-0.175pt]{0.361pt}{0.350pt}}
\put(977,322){\rule[-0.175pt]{0.361pt}{0.350pt}}
\put(978,323){\rule[-0.175pt]{0.361pt}{0.350pt}}
\put(980,324){\rule[-0.175pt]{0.361pt}{0.350pt}}
\put(981,325){\rule[-0.175pt]{0.361pt}{0.350pt}}
\put(983,326){\rule[-0.175pt]{0.402pt}{0.350pt}}
\put(984,327){\rule[-0.175pt]{0.402pt}{0.350pt}}
\put(986,328){\rule[-0.175pt]{0.401pt}{0.350pt}}
\put(988,329){\rule[-0.175pt]{0.361pt}{0.350pt}}
\put(989,330){\rule[-0.175pt]{0.361pt}{0.350pt}}
\put(991,331){\rule[-0.175pt]{0.361pt}{0.350pt}}
\put(992,332){\rule[-0.175pt]{0.361pt}{0.350pt}}
\put(994,333){\rule[-0.175pt]{0.361pt}{0.350pt}}
\put(995,334){\rule[-0.175pt]{0.361pt}{0.350pt}}
\put(997,335){\rule[-0.175pt]{0.361pt}{0.350pt}}
\put(998,336){\rule[-0.175pt]{0.361pt}{0.350pt}}
\put(1000,337){\rule[-0.175pt]{0.361pt}{0.350pt}}
\put(1001,338){\rule[-0.175pt]{0.361pt}{0.350pt}}
\put(1003,339){\rule[-0.175pt]{0.361pt}{0.350pt}}
\put(1004,340){\rule[-0.175pt]{0.361pt}{0.350pt}}
\put(1006,341){\rule[-0.175pt]{0.361pt}{0.350pt}}
\put(1007,342){\rule[-0.175pt]{0.361pt}{0.350pt}}
\put(1009,343){\rule[-0.175pt]{0.361pt}{0.350pt}}
\put(1010,344){\rule[-0.175pt]{0.361pt}{0.350pt}}
\put(1012,345){\rule[-0.175pt]{0.361pt}{0.350pt}}
\put(1013,346){\rule[-0.175pt]{0.361pt}{0.350pt}}
\put(1015,347){\rule[-0.175pt]{0.361pt}{0.350pt}}
\put(1016,348){\rule[-0.175pt]{0.361pt}{0.350pt}}
\put(1018,349){\rule[-0.175pt]{0.361pt}{0.350pt}}
\put(1019,350){\rule[-0.175pt]{0.361pt}{0.350pt}}
\put(1021,351){\rule[-0.175pt]{0.361pt}{0.350pt}}
\put(1022,352){\rule[-0.175pt]{0.361pt}{0.350pt}}
\put(1024,353){\rule[-0.175pt]{0.361pt}{0.350pt}}
\put(1025,354){\rule[-0.175pt]{0.361pt}{0.350pt}}
\put(1027,355){\rule[-0.175pt]{0.361pt}{0.350pt}}
\put(1028,356){\rule[-0.175pt]{0.361pt}{0.350pt}}
\put(1030,357){\rule[-0.175pt]{0.361pt}{0.350pt}}
\put(1031,358){\rule[-0.175pt]{0.361pt}{0.350pt}}
\put(1033,359){\rule[-0.175pt]{0.361pt}{0.350pt}}
\put(1034,360){\rule[-0.175pt]{0.361pt}{0.350pt}}
\put(1036,361){\usebox{\plotpoint}}
\put(1037,362){\usebox{\plotpoint}}
\put(1038,363){\usebox{\plotpoint}}
\put(1039,364){\usebox{\plotpoint}}
\put(1041,365){\rule[-0.175pt]{0.361pt}{0.350pt}}
\put(1042,366){\rule[-0.175pt]{0.361pt}{0.350pt}}
\put(1044,367){\rule[-0.175pt]{0.361pt}{0.350pt}}
\put(1045,368){\rule[-0.175pt]{0.361pt}{0.350pt}}
\put(1047,369){\rule[-0.175pt]{0.361pt}{0.350pt}}
\put(1048,370){\rule[-0.175pt]{0.361pt}{0.350pt}}
\put(1050,371){\rule[-0.175pt]{0.361pt}{0.350pt}}
\put(1051,372){\rule[-0.175pt]{0.361pt}{0.350pt}}
\put(1053,373){\rule[-0.175pt]{0.361pt}{0.350pt}}
\put(1054,374){\rule[-0.175pt]{0.361pt}{0.350pt}}
\put(1056,375){\rule[-0.175pt]{0.361pt}{0.350pt}}
\put(1057,376){\rule[-0.175pt]{0.361pt}{0.350pt}}
\put(1059,377){\rule[-0.175pt]{0.361pt}{0.350pt}}
\put(1060,378){\rule[-0.175pt]{0.361pt}{0.350pt}}
\put(1062,379){\rule[-0.175pt]{0.361pt}{0.350pt}}
\put(1063,380){\rule[-0.175pt]{0.361pt}{0.350pt}}
\put(1065,381){\rule[-0.175pt]{0.361pt}{0.350pt}}
\put(1066,382){\rule[-0.175pt]{0.361pt}{0.350pt}}
\put(1068,383){\rule[-0.175pt]{0.361pt}{0.350pt}}
\put(1069,384){\rule[-0.175pt]{0.361pt}{0.350pt}}
\put(1071,385){\rule[-0.175pt]{0.361pt}{0.350pt}}
\put(1072,386){\rule[-0.175pt]{0.361pt}{0.350pt}}
\put(1074,387){\rule[-0.175pt]{0.361pt}{0.350pt}}
\put(1075,388){\rule[-0.175pt]{0.361pt}{0.350pt}}
\put(1077,389){\rule[-0.175pt]{0.361pt}{0.350pt}}
\put(1078,390){\rule[-0.175pt]{0.361pt}{0.350pt}}
\put(1080,391){\rule[-0.175pt]{0.361pt}{0.350pt}}
\put(1081,392){\rule[-0.175pt]{0.361pt}{0.350pt}}
\put(1083,393){\rule[-0.175pt]{0.361pt}{0.350pt}}
\put(1084,394){\rule[-0.175pt]{0.361pt}{0.350pt}}
\put(1086,395){\rule[-0.175pt]{0.361pt}{0.350pt}}
\put(1087,396){\rule[-0.175pt]{0.361pt}{0.350pt}}
\put(1089,397){\usebox{\plotpoint}}
\put(1090,398){\usebox{\plotpoint}}
\put(1091,399){\usebox{\plotpoint}}
\put(1092,400){\usebox{\plotpoint}}
\put(1094,401){\rule[-0.175pt]{0.361pt}{0.350pt}}
\put(1095,402){\rule[-0.175pt]{0.361pt}{0.350pt}}
\put(1097,403){\rule[-0.175pt]{0.361pt}{0.350pt}}
\put(1098,404){\rule[-0.175pt]{0.361pt}{0.350pt}}
\put(1100,405){\rule[-0.175pt]{0.482pt}{0.350pt}}
\put(1102,406){\rule[-0.175pt]{0.482pt}{0.350pt}}
\put(1104,407){\rule[-0.175pt]{0.482pt}{0.350pt}}
\put(1106,408){\rule[-0.175pt]{0.361pt}{0.350pt}}
\put(1107,409){\rule[-0.175pt]{0.361pt}{0.350pt}}
\put(1109,410){\rule[-0.175pt]{0.361pt}{0.350pt}}
\put(1110,411){\rule[-0.175pt]{0.361pt}{0.350pt}}
\put(1112,412){\rule[-0.175pt]{0.361pt}{0.350pt}}
\put(1113,413){\rule[-0.175pt]{0.361pt}{0.350pt}}
\put(1115,414){\rule[-0.175pt]{0.361pt}{0.350pt}}
\put(1116,415){\rule[-0.175pt]{0.361pt}{0.350pt}}
\put(1118,416){\rule[-0.175pt]{0.361pt}{0.350pt}}
\put(1119,417){\rule[-0.175pt]{0.361pt}{0.350pt}}
\put(1121,418){\rule[-0.175pt]{0.361pt}{0.350pt}}
\put(1122,419){\rule[-0.175pt]{0.361pt}{0.350pt}}
\put(1124,420){\rule[-0.175pt]{0.361pt}{0.350pt}}
\put(1125,421){\rule[-0.175pt]{0.361pt}{0.350pt}}
\put(1127,422){\rule[-0.175pt]{0.361pt}{0.350pt}}
\put(1128,423){\rule[-0.175pt]{0.361pt}{0.350pt}}
\put(1130,424){\rule[-0.175pt]{0.361pt}{0.350pt}}
\put(1131,425){\rule[-0.175pt]{0.361pt}{0.350pt}}
\put(1133,426){\rule[-0.175pt]{0.361pt}{0.350pt}}
\put(1134,427){\rule[-0.175pt]{0.361pt}{0.350pt}}
\put(1136,428){\rule[-0.175pt]{0.361pt}{0.350pt}}
\put(1137,429){\rule[-0.175pt]{0.361pt}{0.350pt}}
\put(1139,430){\rule[-0.175pt]{0.361pt}{0.350pt}}
\put(1140,431){\rule[-0.175pt]{0.361pt}{0.350pt}}
\put(1142,432){\usebox{\plotpoint}}
\put(1143,433){\usebox{\plotpoint}}
\put(1144,434){\usebox{\plotpoint}}
\put(1145,435){\usebox{\plotpoint}}
\put(1147,436){\rule[-0.175pt]{0.361pt}{0.350pt}}
\put(1148,437){\rule[-0.175pt]{0.361pt}{0.350pt}}
\put(1150,438){\rule[-0.175pt]{0.361pt}{0.350pt}}
\put(1151,439){\rule[-0.175pt]{0.361pt}{0.350pt}}
\put(1153,440){\rule[-0.175pt]{0.361pt}{0.350pt}}
\put(1154,441){\rule[-0.175pt]{0.361pt}{0.350pt}}
\put(1156,442){\rule[-0.175pt]{0.361pt}{0.350pt}}
\put(1157,443){\rule[-0.175pt]{0.361pt}{0.350pt}}
\put(1159,444){\rule[-0.175pt]{0.361pt}{0.350pt}}
\put(1160,445){\rule[-0.175pt]{0.361pt}{0.350pt}}
\put(1162,446){\rule[-0.175pt]{0.361pt}{0.350pt}}
\put(1163,447){\rule[-0.175pt]{0.361pt}{0.350pt}}
\put(1165,448){\rule[-0.175pt]{0.361pt}{0.350pt}}
\put(1166,449){\rule[-0.175pt]{0.361pt}{0.350pt}}
\put(1168,450){\rule[-0.175pt]{0.361pt}{0.350pt}}
\put(1169,451){\rule[-0.175pt]{0.361pt}{0.350pt}}
\put(1171,452){\rule[-0.175pt]{0.361pt}{0.350pt}}
\put(1172,453){\rule[-0.175pt]{0.361pt}{0.350pt}}
\put(1174,454){\rule[-0.175pt]{0.361pt}{0.350pt}}
\put(1175,455){\rule[-0.175pt]{0.361pt}{0.350pt}}
\put(1177,456){\rule[-0.175pt]{0.361pt}{0.350pt}}
\put(1178,457){\rule[-0.175pt]{0.361pt}{0.350pt}}
\put(1180,458){\rule[-0.175pt]{0.361pt}{0.350pt}}
\put(1181,459){\rule[-0.175pt]{0.361pt}{0.350pt}}
\put(1183,460){\rule[-0.175pt]{0.361pt}{0.350pt}}
\put(1184,461){\rule[-0.175pt]{0.361pt}{0.350pt}}
\put(1186,462){\rule[-0.175pt]{0.361pt}{0.350pt}}
\put(1187,463){\rule[-0.175pt]{0.361pt}{0.350pt}}
\put(1189,464){\rule[-0.175pt]{0.361pt}{0.350pt}}
\put(1190,465){\rule[-0.175pt]{0.361pt}{0.350pt}}
\put(1192,466){\rule[-0.175pt]{0.361pt}{0.350pt}}
\put(1193,467){\rule[-0.175pt]{0.361pt}{0.350pt}}
\put(1195,468){\usebox{\plotpoint}}
\put(1196,469){\usebox{\plotpoint}}
\put(1197,470){\usebox{\plotpoint}}
\put(1198,471){\usebox{\plotpoint}}
%\put(1200,472){\rule[-0.175pt]{0.723pt}{0.350pt}}
%\put(1203,473){\rule[-0.175pt]{0.723pt}{0.350pt}}
%\put(1206,474){\rule[-0.175pt]{0.723pt}{0.350pt}}
%\put(1209,475){\rule[-0.175pt]{0.723pt}{0.350pt}}
%\put(1212,476){\rule[-0.175pt]{0.723pt}{0.350pt}}
%\put(1215,477){\rule[-0.175pt]{0.723pt}{0.350pt}}
%\put(1218,478){\rule[-0.175pt]{0.723pt}{0.350pt}}
%\put(1221,479){\rule[-0.175pt]{0.723pt}{0.350pt}}
%\put(1224,480){\rule[-0.175pt]{0.723pt}{0.350pt}}
%\put(1227,481){\rule[-0.175pt]{0.723pt}{0.350pt}}
%\put(1230,482){\rule[-0.175pt]{0.723pt}{0.350pt}}
%\put(1233,483){\rule[-0.175pt]{0.723pt}{0.350pt}}
%\put(1236,484){\rule[-0.175pt]{0.723pt}{0.350pt}}
%\put(1239,485){\rule[-0.175pt]{0.723pt}{0.350pt}}
%\put(1242,486){\rule[-0.175pt]{0.723pt}{0.350pt}}
%\put(1245,487){\rule[-0.175pt]{0.723pt}{0.350pt}}
%\put(1248,488){\rule[-0.175pt]{0.602pt}{0.350pt}}
%\put(1250,489){\rule[-0.175pt]{0.602pt}{0.350pt}}
%\put(1253,490){\rule[-0.175pt]{0.723pt}{0.350pt}}
%\put(1256,491){\rule[-0.175pt]{0.723pt}{0.350pt}}
%\put(1259,492){\rule[-0.175pt]{0.723pt}{0.350pt}}
%\put(1262,493){\rule[-0.175pt]{0.723pt}{0.350pt}}
%\put(1265,494){\rule[-0.175pt]{0.723pt}{0.350pt}}
%\put(1268,495){\rule[-0.175pt]{0.723pt}{0.350pt}}
%\put(1271,496){\rule[-0.175pt]{0.723pt}{0.350pt}}
%\put(1274,497){\rule[-0.175pt]{0.723pt}{0.350pt}}
%\put(1277,498){\rule[-0.175pt]{0.723pt}{0.350pt}}
%\put(1280,499){\rule[-0.175pt]{0.723pt}{0.350pt}}
%\put(1283,500){\rule[-0.175pt]{0.723pt}{0.350pt}}
%\put(1286,501){\rule[-0.175pt]{0.723pt}{0.350pt}}
%\put(1289,502){\rule[-0.175pt]{0.723pt}{0.350pt}}
%\put(1292,503){\rule[-0.175pt]{0.723pt}{0.350pt}}
%\put(1295,504){\rule[-0.175pt]{0.723pt}{0.350pt}}
%\put(1298,505){\rule[-0.175pt]{0.723pt}{0.350pt}}
%\put(1301,506){\rule[-0.175pt]{0.602pt}{0.350pt}}
%\put(1303,507){\rule[-0.175pt]{0.602pt}{0.350pt}}
%\put(1306,508){\rule[-0.175pt]{0.723pt}{0.350pt}}
%\put(1309,509){\rule[-0.175pt]{0.723pt}{0.350pt}}
%\put(1312,510){\rule[-0.175pt]{0.723pt}{0.350pt}}
%\put(1315,511){\rule[-0.175pt]{0.723pt}{0.350pt}}
%\put(1318,512){\rule[-0.175pt]{0.723pt}{0.350pt}}
%\put(1321,513){\rule[-0.175pt]{0.723pt}{0.350pt}}
%\put(1324,514){\rule[-0.175pt]{0.723pt}{0.350pt}}
%\put(1327,515){\rule[-0.175pt]{0.723pt}{0.350pt}}
%\put(1330,516){\rule[-0.175pt]{0.723pt}{0.350pt}}
%\put(1333,517){\rule[-0.175pt]{0.723pt}{0.350pt}}
%\put(1336,518){\rule[-0.175pt]{0.723pt}{0.350pt}}
%\put(1339,519){\rule[-0.175pt]{0.723pt}{0.350pt}}
%\put(1342,520){\rule[-0.175pt]{1.445pt}{0.350pt}}
%\put(1348,521){\rule[-0.175pt]{0.723pt}{0.350pt}}
%\put(1351,522){\rule[-0.175pt]{0.723pt}{0.350pt}}
%\put(1354,523){\rule[-0.175pt]{0.602pt}{0.350pt}}
%\put(1356,524){\rule[-0.175pt]{0.602pt}{0.350pt}}
%\put(1359,525){\rule[-0.175pt]{0.723pt}{0.350pt}}
%\put(1362,526){\rule[-0.175pt]{0.723pt}{0.350pt}}
%\put(1365,527){\rule[-0.175pt]{0.723pt}{0.350pt}}
%\put(1368,528){\rule[-0.175pt]{0.723pt}{0.350pt}}
%\put(1371,529){\rule[-0.175pt]{0.723pt}{0.350pt}}
%\put(1374,530){\rule[-0.175pt]{0.723pt}{0.350pt}}
%\put(1377,531){\rule[-0.175pt]{0.723pt}{0.350pt}}
%\put(1380,532){\rule[-0.175pt]{0.723pt}{0.350pt}}
%\put(1383,533){\rule[-0.175pt]{0.723pt}{0.350pt}}
%\put(1386,534){\rule[-0.175pt]{0.723pt}{0.350pt}}
%\put(1389,535){\rule[-0.175pt]{0.723pt}{0.350pt}}
%\put(1392,536){\rule[-0.175pt]{0.723pt}{0.350pt}}
%\put(1395,537){\rule[-0.175pt]{0.723pt}{0.350pt}}
%\put(1398,538){\rule[-0.175pt]{0.723pt}{0.350pt}}
%\put(1401,539){\rule[-0.175pt]{0.723pt}{0.350pt}}
%\put(1404,540){\rule[-0.175pt]{0.723pt}{0.350pt}}
%\put(1407,541){\rule[-0.175pt]{0.602pt}{0.350pt}}
%\put(1409,542){\rule[-0.175pt]{0.602pt}{0.350pt}}
%\put(1412,543){\rule[-0.175pt]{0.723pt}{0.350pt}}
%\put(1415,544){\rule[-0.175pt]{0.723pt}{0.350pt}}
%\put(1418,545){\rule[-0.175pt]{0.723pt}{0.350pt}}
%\put(1421,546){\rule[-0.175pt]{0.723pt}{0.350pt}}
%\put(1424,547){\rule[-0.175pt]{0.723pt}{0.350pt}}
%\put(1427,548){\rule[-0.175pt]{0.723pt}{0.350pt}}
%\put(1430,549){\rule[-0.175pt]{0.723pt}{0.350pt}}
%\put(1433,550){\rule[-0.175pt]{0.723pt}{0.350pt}}
\sbox{\plotpoint}{\rule[-0.250pt]{0.500pt}{0.500pt}}%
%\put(264,237){\usebox{\plotpoint}}
%\put(264,237){\usebox{\plotpoint}}
%\put(284,237){\usebox{\plotpoint}}
%\put(305,237){\usebox{\plotpoint}}
%\put(326,237){\usebox{\plotpoint}}
%\put(347,237){\usebox{\plotpoint}}
%\put(367,237){\usebox{\plotpoint}}
%\put(384,243){\usebox{\plotpoint}}
%\put(394,262){\usebox{\plotpoint}}
%\put(403,281){\usebox{\plotpoint}}
%\put(412,299){\usebox{\plotpoint}}
%\put(421,318){\usebox{\plotpoint}}
%\put(431,336){\usebox{\plotpoint}}
%\put(440,355){\usebox{\plotpoint}}
%\put(449,373){\usebox{\plotpoint}}
%\put(458,392){\usebox{\plotpoint}}
%\put(467,411){\usebox{\plotpoint}}
%\put(477,429){\usebox{\plotpoint}}
%\put(486,448){\usebox{\plotpoint}}
%\put(496,466){\usebox{\plotpoint}}
%\put(511,463){\usebox{\plotpoint}}
%\put(528,452){\usebox{\plotpoint}}
%\put(545,440){\usebox{\plotpoint}}
%\put(562,428){\usebox{\plotpoint}}
%\put(580,417){\usebox{\plotpoint}}
%\put(597,406){\usebox{\plotpoint}}
%\put(614,394){\usebox{\plotpoint}}
%\put(632,382){\usebox{\plotpoint}}
%\put(649,371){\usebox{\plotpoint}}
%\put(666,359){\usebox{\plotpoint}}
%\put(683,347){\usebox{\plotpoint}}
%\put(700,336){\usebox{\plotpoint}}
%\put(718,325){\usebox{\plotpoint}}
\put(736,315){\usebox{\plotpoint}}
\put(756,315){\usebox{\plotpoint}}
\put(777,315){\usebox{\plotpoint}}
\put(798,315){\usebox{\plotpoint}}
\put(819,315){\usebox{\plotpoint}}
\put(839,315){\usebox{\plotpoint}}
\put(860,315){\usebox{\plotpoint}}
\put(881,315){\usebox{\plotpoint}}
\put(902,315){\usebox{\plotpoint}}
\put(922,315){\usebox{\plotpoint}}
\put(943,315){\usebox{\plotpoint}}
\put(964,315){\usebox{\plotpoint}}
\put(982,325){\usebox{\plotpoint}}
\put(999,336){\usebox{\plotpoint}}
\put(1016,348){\usebox{\plotpoint}}
\put(1034,359){\usebox{\plotpoint}}
\put(1051,371){\usebox{\plotpoint}}
\put(1068,383){\usebox{\plotpoint}}
\put(1085,394){\usebox{\plotpoint}}
\put(1102,406){\usebox{\plotpoint}}
\put(1120,417){\usebox{\plotpoint}}
\put(1137,429){\usebox{\plotpoint}}
\put(1154,440){\usebox{\plotpoint}}
\put(1171,452){\usebox{\plotpoint}}
\put(1188,463){\usebox{\plotpoint}}
\put(1206,474){\usebox{\plotpoint}}
\put(1226,480){\usebox{\plotpoint}}
\put(1246,487){\usebox{\plotpoint}}
\put(1265,494){\usebox{\plotpoint}}
\put(1285,500){\usebox{\plotpoint}}
\put(1305,507){\usebox{\plotpoint}}
\put(1305,500){$\bullet$}
\put(1324,514){\usebox{\plotpoint}}
\put(1344,520){\usebox{\plotpoint}}
\put(1364,526){\usebox{\plotpoint}}
\put(1383,533){\usebox{\plotpoint}}
\put(1403,539){\usebox{\plotpoint}}
\put(1423,546){\usebox{\plotpoint}}
\put(1436,551){\usebox{\plotpoint}}
\put(290,250){$I_{DC}$}
\put(400,200){($T_1,I_1$)}
\put(100,480){$v$}
\put(90,700){(A)}
\put(820,40){$t$}
\put(1300,40){(s)}
\put(420,490){($T_2,I_2$)}
\put(660,250){($T_3,I_3$)}
\put(920,250){($T_i,I_i$)}
\put(960,480){($T_{i+1},I_{i+1}$)}
\put(1180,550){($T_N,I_N$)}
\end{picture}

\caption[Voltage source transient piece-wise linear ({\tt PWL}) waveform]{Voltage
source transient piece-wise linear ({\tt PWL}) waveform for\newline \hspace*{\fill}
{\tt PWL(1 0.25  1 1 2 0.5 $\ldots$ 3 0.5 4 1 $\ldots$ 4.5 1.25 $\ldots$)} with
{\it DCValue = 0.25}.  \hspace*{\fill} }
\end{figure}


\noindent{\underline{\bf Sinusoidal:}}
\\[0.2in]
\form{SIN( $V_O$ $V_A$ \B $F$ \E \B $T_D$ \E \B $\theta$ \E {\tt )}}

\pspiceform{
   SIN( $V_O$ $V_A$ \B $F$ \E \B $T_D$ \E \B $\theta$ $\phi$ \E {\tt )}}

\pspiceninetytwoform{
   SIN( $V_O$ $V_A$ \B $F$ \E \B $T_D$ \E \B $\theta$ $\phi$ \E {\tt )}}

\keywordtable{
$V_O     $&voltage offset   & A & \reqd    \X
$V_A     $&voltage amplitude& A & \reqd    \X
$F       $&frequency        & Hz&   1/{\tt TSTOP}\X
$T_D     $&time delay       & s &   0    \X
$\Theta  $&damping factor   &1/s&   0\X
$\phi    $&phase            &degree&   0\X
}
The sinusoidal transient waveform is defined by
\begin{equation}
v = \left\{ \begin{array}{ll}
V_0                         & t \le T_D\\
V_0 + V_1 e^{-[\textstyle (t -T_D)\Theta]} \sin{2\pi[F(t-T_D) + \phi/360]}
                            & t > T_D
     \end{array} \right. %}
\end{equation}

\begin{figure}[hbp]
\centering
%set samples 200
%set yrange [-1:1]
%i(x) = (x < 1)? 0.1 : 0.1 + 0.8*exp(-(x-1)*0.3)*sin(2*3.14159*2*(x-1))
%set term latex
%set output 'iexp.tex'
%set size 1.0,1.0
%set border
%set nokey
%set xtics
%set ytics
%set xzeroaxis
%set yzeroaxis
%plot [0:4] i(x) with line 1, 0 with line 4
%set term x11
%plot [0:4] i(x) with line 1, 0 with line 4
% GNUPLOT: LaTeX picture
\setlength{\unitlength}{0.240900pt}
\ifx\plotpoint\undefined\newsavebox{\plotpoint}\fi
\sbox{\plotpoint}{\rule[-0.175pt]{0.350pt}{0.350pt}}%
\begin{picture}(1500,900)(0,0)
%\tenrm
\sbox{\plotpoint}{\rule[-0.175pt]{0.350pt}{0.350pt}}%
\put(264,473){\rule[-0.175pt]{282.335pt}{0.350pt}}
\put(264,158){\rule[-0.175pt]{0.350pt}{151.526pt}}
\put(264,158){\rule[-0.175pt]{4.818pt}{0.350pt}}
\put(242,158){\makebox(0,0)[r]{-1}}
\put(1416,158){\rule[-0.175pt]{4.818pt}{0.350pt}}
\put(264,221){\rule[-0.175pt]{4.818pt}{0.350pt}}
\put(242,221){\makebox(0,0)[r]{-0.8}}
\put(1416,221){\rule[-0.175pt]{4.818pt}{0.350pt}}
\put(264,284){\rule[-0.175pt]{4.818pt}{0.350pt}}
\put(242,284){\makebox(0,0)[r]{-0.6}}
\put(1416,284){\rule[-0.175pt]{4.818pt}{0.350pt}}
\put(264,347){\rule[-0.175pt]{4.818pt}{0.350pt}}
\put(242,347){\makebox(0,0)[r]{-0.4}}
\put(1416,347){\rule[-0.175pt]{4.818pt}{0.350pt}}
\put(264,410){\rule[-0.175pt]{4.818pt}{0.350pt}}
\put(242,410){\makebox(0,0)[r]{-0.2}}
\put(1416,410){\rule[-0.175pt]{4.818pt}{0.350pt}}
\put(264,473){\rule[-0.175pt]{4.818pt}{0.350pt}}
\put(242,473){\makebox(0,0)[r]{0}}
\put(1416,473){\rule[-0.175pt]{4.818pt}{0.350pt}}
\put(264,535){\rule[-0.175pt]{4.818pt}{0.350pt}}
\put(242,535){\makebox(0,0)[r]{0.2}}
\put(1416,535){\rule[-0.175pt]{4.818pt}{0.350pt}}
\put(264,598){\rule[-0.175pt]{4.818pt}{0.350pt}}
\put(242,598){\makebox(0,0)[r]{0.4}}
\put(1416,598){\rule[-0.175pt]{4.818pt}{0.350pt}}
\put(264,661){\rule[-0.175pt]{4.818pt}{0.350pt}}
\put(242,661){\makebox(0,0)[r]{0.6}}
\put(1416,661){\rule[-0.175pt]{4.818pt}{0.350pt}}
\put(264,724){\rule[-0.175pt]{4.818pt}{0.350pt}}
\put(242,724){\makebox(0,0)[r]{0.8}}
\put(1416,724){\rule[-0.175pt]{4.818pt}{0.350pt}}
\put(264,787){\rule[-0.175pt]{4.818pt}{0.350pt}}
\put(242,787){\makebox(0,0)[r]{1}}
\put(1416,787){\rule[-0.175pt]{4.818pt}{0.350pt}}
\put(264,158){\rule[-0.175pt]{0.350pt}{4.818pt}}
\put(264,113){\makebox(0,0){0}}
\put(264,767){\rule[-0.175pt]{0.350pt}{4.818pt}}
\put(411,158){\rule[-0.175pt]{0.350pt}{4.818pt}}
\put(411,113){\makebox(0,0){0.5}}
\put(411,767){\rule[-0.175pt]{0.350pt}{4.818pt}}
\put(557,158){\rule[-0.175pt]{0.350pt}{4.818pt}}
\put(557,113){\makebox(0,0){1}}
\put(557,767){\rule[-0.175pt]{0.350pt}{4.818pt}}
\put(704,158){\rule[-0.175pt]{0.350pt}{4.818pt}}
\put(704,113){\makebox(0,0){1.5}}
\put(704,767){\rule[-0.175pt]{0.350pt}{4.818pt}}
\put(850,158){\rule[-0.175pt]{0.350pt}{4.818pt}}
\put(850,113){\makebox(0,0){2}}
\put(850,767){\rule[-0.175pt]{0.350pt}{4.818pt}}
\put(997,158){\rule[-0.175pt]{0.350pt}{4.818pt}}
\put(997,113){\makebox(0,0){2.5}}
\put(997,767){\rule[-0.175pt]{0.350pt}{4.818pt}}
\put(1143,158){\rule[-0.175pt]{0.350pt}{4.818pt}}
\put(1143,113){\makebox(0,0){3}}
\put(1143,767){\rule[-0.175pt]{0.350pt}{4.818pt}}
\put(1290,158){\rule[-0.175pt]{0.350pt}{4.818pt}}
\put(1290,113){\makebox(0,0){3.5}}
\put(1290,767){\rule[-0.175pt]{0.350pt}{4.818pt}}
\put(1436,158){\rule[-0.175pt]{0.350pt}{4.818pt}}
\put(1436,113){\makebox(0,0){4}}
\put(1436,767){\rule[-0.175pt]{0.350pt}{4.818pt}}
\put(264,158){\rule[-0.175pt]{282.335pt}{0.350pt}}
\put(1436,158){\rule[-0.175pt]{0.350pt}{151.526pt}}
\put(264,787){\rule[-0.175pt]{282.335pt}{0.350pt}}
\put(264,158){\rule[-0.175pt]{0.350pt}{151.526pt}}
\put(264,504){\usebox{\plotpoint}}
\put(264,504){\rule[-0.175pt]{69.620pt}{0.350pt}}
\put(553,504){\rule[-0.175pt]{0.350pt}{0.771pt}}
\put(554,507){\rule[-0.175pt]{0.350pt}{0.771pt}}
\put(555,510){\rule[-0.175pt]{0.350pt}{0.771pt}}
\put(556,513){\rule[-0.175pt]{0.350pt}{0.771pt}}
\put(557,516){\rule[-0.175pt]{0.350pt}{0.771pt}}
\put(558,520){\rule[-0.175pt]{0.350pt}{2.449pt}}
\put(559,530){\rule[-0.175pt]{0.350pt}{2.449pt}}
\put(560,540){\rule[-0.175pt]{0.350pt}{2.449pt}}
\put(561,550){\rule[-0.175pt]{0.350pt}{2.449pt}}
\put(562,560){\rule[-0.175pt]{0.350pt}{2.449pt}}
\put(563,570){\rule[-0.175pt]{0.350pt}{2.449pt}}
\put(564,581){\rule[-0.175pt]{0.350pt}{2.289pt}}
\put(565,590){\rule[-0.175pt]{0.350pt}{2.289pt}}
\put(566,600){\rule[-0.175pt]{0.350pt}{2.289pt}}
\put(567,609){\rule[-0.175pt]{0.350pt}{2.289pt}}
\put(568,619){\rule[-0.175pt]{0.350pt}{2.289pt}}
\put(569,628){\rule[-0.175pt]{0.350pt}{2.289pt}}
\put(570,638){\rule[-0.175pt]{0.350pt}{1.847pt}}
\put(571,645){\rule[-0.175pt]{0.350pt}{1.847pt}}
\put(572,653){\rule[-0.175pt]{0.350pt}{1.847pt}}
\put(573,661){\rule[-0.175pt]{0.350pt}{1.847pt}}
\put(574,668){\rule[-0.175pt]{0.350pt}{1.847pt}}
\put(575,676){\rule[-0.175pt]{0.350pt}{1.847pt}}
\put(576,684){\rule[-0.175pt]{0.350pt}{1.405pt}}
\put(577,689){\rule[-0.175pt]{0.350pt}{1.405pt}}
\put(578,695){\rule[-0.175pt]{0.350pt}{1.405pt}}
\put(579,701){\rule[-0.175pt]{0.350pt}{1.405pt}}
\put(580,707){\rule[-0.175pt]{0.350pt}{1.405pt}}
\put(581,713){\rule[-0.175pt]{0.350pt}{1.405pt}}
\put(582,718){\rule[-0.175pt]{0.350pt}{0.843pt}}
\put(583,722){\rule[-0.175pt]{0.350pt}{0.843pt}}
\put(584,726){\rule[-0.175pt]{0.350pt}{0.843pt}}
\put(585,729){\rule[-0.175pt]{0.350pt}{0.843pt}}
\put(586,733){\rule[-0.175pt]{0.350pt}{0.843pt}}
\put(587,736){\rule[-0.175pt]{0.350pt}{0.843pt}}
\put(588,740){\usebox{\plotpoint}}
\put(589,741){\usebox{\plotpoint}}
\put(590,742){\usebox{\plotpoint}}
\put(591,743){\usebox{\plotpoint}}
\put(592,744){\usebox{\plotpoint}}
\put(593,745){\usebox{\plotpoint}}
\put(594,744){\rule[-0.175pt]{0.350pt}{0.361pt}}
\put(595,743){\rule[-0.175pt]{0.350pt}{0.361pt}}
\put(596,741){\rule[-0.175pt]{0.350pt}{0.361pt}}
\put(597,740){\rule[-0.175pt]{0.350pt}{0.361pt}}
\put(598,738){\rule[-0.175pt]{0.350pt}{0.361pt}}
\put(599,737){\rule[-0.175pt]{0.350pt}{0.361pt}}
\put(600,733){\rule[-0.175pt]{0.350pt}{0.964pt}}
\put(601,729){\rule[-0.175pt]{0.350pt}{0.964pt}}
\put(602,725){\rule[-0.175pt]{0.350pt}{0.964pt}}
\put(603,721){\rule[-0.175pt]{0.350pt}{0.964pt}}
\put(604,717){\rule[-0.175pt]{0.350pt}{0.964pt}}
\put(605,713){\rule[-0.175pt]{0.350pt}{0.964pt}}
\put(606,705){\rule[-0.175pt]{0.350pt}{1.831pt}}
\put(607,697){\rule[-0.175pt]{0.350pt}{1.831pt}}
\put(608,690){\rule[-0.175pt]{0.350pt}{1.831pt}}
\put(609,682){\rule[-0.175pt]{0.350pt}{1.831pt}}
\put(610,675){\rule[-0.175pt]{0.350pt}{1.831pt}}
\put(611,667){\rule[-0.175pt]{0.350pt}{1.887pt}}
\put(612,659){\rule[-0.175pt]{0.350pt}{1.887pt}}
\put(613,651){\rule[-0.175pt]{0.350pt}{1.887pt}}
\put(614,643){\rule[-0.175pt]{0.350pt}{1.887pt}}
\put(615,635){\rule[-0.175pt]{0.350pt}{1.887pt}}
\put(616,628){\rule[-0.175pt]{0.350pt}{1.887pt}}
\put(617,618){\rule[-0.175pt]{0.350pt}{2.208pt}}
\put(618,609){\rule[-0.175pt]{0.350pt}{2.208pt}}
\put(619,600){\rule[-0.175pt]{0.350pt}{2.208pt}}
\put(620,591){\rule[-0.175pt]{0.350pt}{2.208pt}}
\put(621,582){\rule[-0.175pt]{0.350pt}{2.208pt}}
\put(622,573){\rule[-0.175pt]{0.350pt}{2.208pt}}
\put(623,563){\rule[-0.175pt]{0.350pt}{2.329pt}}
\put(624,553){\rule[-0.175pt]{0.350pt}{2.329pt}}
\put(625,543){\rule[-0.175pt]{0.350pt}{2.329pt}}
\put(626,534){\rule[-0.175pt]{0.350pt}{2.329pt}}
\put(627,524){\rule[-0.175pt]{0.350pt}{2.329pt}}
\put(628,515){\rule[-0.175pt]{0.350pt}{2.329pt}}
\put(629,505){\rule[-0.175pt]{0.350pt}{2.329pt}}
\put(630,495){\rule[-0.175pt]{0.350pt}{2.329pt}}
\put(631,486){\rule[-0.175pt]{0.350pt}{2.329pt}}
\put(632,476){\rule[-0.175pt]{0.350pt}{2.329pt}}
\put(633,466){\rule[-0.175pt]{0.350pt}{2.329pt}}
\put(634,457){\rule[-0.175pt]{0.350pt}{2.329pt}}
\put(635,447){\rule[-0.175pt]{0.350pt}{2.208pt}}
\put(636,438){\rule[-0.175pt]{0.350pt}{2.208pt}}
\put(637,429){\rule[-0.175pt]{0.350pt}{2.208pt}}
\put(638,420){\rule[-0.175pt]{0.350pt}{2.208pt}}
\put(639,411){\rule[-0.175pt]{0.350pt}{2.208pt}}
\put(640,402){\rule[-0.175pt]{0.350pt}{2.208pt}}
\put(641,394){\rule[-0.175pt]{0.350pt}{1.927pt}}
\put(642,386){\rule[-0.175pt]{0.350pt}{1.927pt}}
\put(643,378){\rule[-0.175pt]{0.350pt}{1.927pt}}
\put(644,370){\rule[-0.175pt]{0.350pt}{1.927pt}}
\put(645,362){\rule[-0.175pt]{0.350pt}{1.927pt}}
\put(646,354){\rule[-0.175pt]{0.350pt}{1.927pt}}
\put(647,347){\rule[-0.175pt]{0.350pt}{1.486pt}}
\put(648,341){\rule[-0.175pt]{0.350pt}{1.486pt}}
\put(649,335){\rule[-0.175pt]{0.350pt}{1.486pt}}
\put(650,329){\rule[-0.175pt]{0.350pt}{1.486pt}}
\put(651,323){\rule[-0.175pt]{0.350pt}{1.486pt}}
\put(652,317){\rule[-0.175pt]{0.350pt}{1.486pt}}
\put(653,312){\rule[-0.175pt]{0.350pt}{1.044pt}}
\put(654,308){\rule[-0.175pt]{0.350pt}{1.044pt}}
\put(655,303){\rule[-0.175pt]{0.350pt}{1.044pt}}
\put(656,299){\rule[-0.175pt]{0.350pt}{1.044pt}}
\put(657,295){\rule[-0.175pt]{0.350pt}{1.044pt}}
\put(658,291){\rule[-0.175pt]{0.350pt}{1.044pt}}
\put(659,288){\rule[-0.175pt]{0.350pt}{0.530pt}}
\put(660,286){\rule[-0.175pt]{0.350pt}{0.530pt}}
\put(661,284){\rule[-0.175pt]{0.350pt}{0.530pt}}
\put(662,282){\rule[-0.175pt]{0.350pt}{0.530pt}}
\put(663,280){\rule[-0.175pt]{0.350pt}{0.530pt}}
\put(664,280){\rule[-0.175pt]{0.723pt}{0.350pt}}
\put(667,281){\rule[-0.175pt]{0.723pt}{0.350pt}}
\put(670,282){\rule[-0.175pt]{0.350pt}{0.683pt}}
\put(671,284){\rule[-0.175pt]{0.350pt}{0.683pt}}
\put(672,287){\rule[-0.175pt]{0.350pt}{0.683pt}}
\put(673,290){\rule[-0.175pt]{0.350pt}{0.683pt}}
\put(674,293){\rule[-0.175pt]{0.350pt}{0.683pt}}
\put(675,296){\rule[-0.175pt]{0.350pt}{0.683pt}}
\put(676,299){\rule[-0.175pt]{0.350pt}{1.164pt}}
\put(677,303){\rule[-0.175pt]{0.350pt}{1.164pt}}
\put(678,308){\rule[-0.175pt]{0.350pt}{1.164pt}}
\put(679,313){\rule[-0.175pt]{0.350pt}{1.164pt}}
\put(680,318){\rule[-0.175pt]{0.350pt}{1.164pt}}
\put(681,323){\rule[-0.175pt]{0.350pt}{1.164pt}}
\put(682,328){\rule[-0.175pt]{0.350pt}{1.646pt}}
\put(683,334){\rule[-0.175pt]{0.350pt}{1.646pt}}
\put(684,341){\rule[-0.175pt]{0.350pt}{1.646pt}}
\put(685,348){\rule[-0.175pt]{0.350pt}{1.646pt}}
\put(686,355){\rule[-0.175pt]{0.350pt}{1.646pt}}
\put(687,362){\rule[-0.175pt]{0.350pt}{1.646pt}}
\put(688,369){\rule[-0.175pt]{0.350pt}{1.927pt}}
\put(689,377){\rule[-0.175pt]{0.350pt}{1.927pt}}
\put(690,385){\rule[-0.175pt]{0.350pt}{1.927pt}}
\put(691,393){\rule[-0.175pt]{0.350pt}{1.927pt}}
\put(692,401){\rule[-0.175pt]{0.350pt}{1.927pt}}
\put(693,409){\rule[-0.175pt]{0.350pt}{1.927pt}}
\put(694,417){\rule[-0.175pt]{0.350pt}{2.128pt}}
\put(695,425){\rule[-0.175pt]{0.350pt}{2.128pt}}
\put(696,434){\rule[-0.175pt]{0.350pt}{2.128pt}}
\put(697,443){\rule[-0.175pt]{0.350pt}{2.128pt}}
\put(698,452){\rule[-0.175pt]{0.350pt}{2.128pt}}
\put(699,461){\rule[-0.175pt]{0.350pt}{2.128pt}}
\put(700,470){\rule[-0.175pt]{0.350pt}{2.168pt}}
\put(701,479){\rule[-0.175pt]{0.350pt}{2.168pt}}
\put(702,488){\rule[-0.175pt]{0.350pt}{2.168pt}}
\put(703,497){\rule[-0.175pt]{0.350pt}{2.168pt}}
\put(704,506){\rule[-0.175pt]{0.350pt}{2.168pt}}
\put(705,515){\rule[-0.175pt]{0.350pt}{2.168pt}}
\put(706,524){\rule[-0.175pt]{0.350pt}{2.128pt}}
\put(707,532){\rule[-0.175pt]{0.350pt}{2.128pt}}
\put(708,541){\rule[-0.175pt]{0.350pt}{2.128pt}}
\put(709,550){\rule[-0.175pt]{0.350pt}{2.128pt}}
\put(710,559){\rule[-0.175pt]{0.350pt}{2.128pt}}
\put(711,568){\rule[-0.175pt]{0.350pt}{2.128pt}}
\put(712,576){\rule[-0.175pt]{0.350pt}{2.264pt}}
\put(713,586){\rule[-0.175pt]{0.350pt}{2.264pt}}
\put(714,595){\rule[-0.175pt]{0.350pt}{2.264pt}}
\put(715,605){\rule[-0.175pt]{0.350pt}{2.264pt}}
\put(716,614){\rule[-0.175pt]{0.350pt}{2.264pt}}
\put(717,624){\rule[-0.175pt]{0.350pt}{1.606pt}}
\put(718,630){\rule[-0.175pt]{0.350pt}{1.606pt}}
\put(719,637){\rule[-0.175pt]{0.350pt}{1.606pt}}
\put(720,644){\rule[-0.175pt]{0.350pt}{1.606pt}}
\put(721,650){\rule[-0.175pt]{0.350pt}{1.606pt}}
\put(722,657){\rule[-0.175pt]{0.350pt}{1.606pt}}
\put(723,664){\rule[-0.175pt]{0.350pt}{1.124pt}}
\put(724,668){\rule[-0.175pt]{0.350pt}{1.124pt}}
\put(725,673){\rule[-0.175pt]{0.350pt}{1.124pt}}
\put(726,678){\rule[-0.175pt]{0.350pt}{1.124pt}}
\put(727,682){\rule[-0.175pt]{0.350pt}{1.124pt}}
\put(728,687){\rule[-0.175pt]{0.350pt}{1.124pt}}
\put(729,692){\rule[-0.175pt]{0.350pt}{0.683pt}}
\put(730,694){\rule[-0.175pt]{0.350pt}{0.683pt}}
\put(731,697){\rule[-0.175pt]{0.350pt}{0.683pt}}
\put(732,700){\rule[-0.175pt]{0.350pt}{0.683pt}}
\put(733,703){\rule[-0.175pt]{0.350pt}{0.683pt}}
\put(734,706){\rule[-0.175pt]{0.350pt}{0.683pt}}
\put(735,708){\usebox{\plotpoint}}
\put(735,709){\rule[-0.175pt]{0.482pt}{0.350pt}}
\put(737,710){\rule[-0.175pt]{0.482pt}{0.350pt}}
\put(739,711){\rule[-0.175pt]{0.482pt}{0.350pt}}
\put(741,710){\rule[-0.175pt]{0.350pt}{0.402pt}}
\put(742,708){\rule[-0.175pt]{0.350pt}{0.402pt}}
\put(743,706){\rule[-0.175pt]{0.350pt}{0.402pt}}
\put(744,705){\rule[-0.175pt]{0.350pt}{0.402pt}}
\put(745,703){\rule[-0.175pt]{0.350pt}{0.402pt}}
\put(746,702){\rule[-0.175pt]{0.350pt}{0.401pt}}
\put(747,698){\rule[-0.175pt]{0.350pt}{0.883pt}}
\put(748,694){\rule[-0.175pt]{0.350pt}{0.883pt}}
\put(749,690){\rule[-0.175pt]{0.350pt}{0.883pt}}
\put(750,687){\rule[-0.175pt]{0.350pt}{0.883pt}}
\put(751,683){\rule[-0.175pt]{0.350pt}{0.883pt}}
\put(752,680){\rule[-0.175pt]{0.350pt}{0.883pt}}
\put(753,674){\rule[-0.175pt]{0.350pt}{1.325pt}}
\put(754,669){\rule[-0.175pt]{0.350pt}{1.325pt}}
\put(755,663){\rule[-0.175pt]{0.350pt}{1.325pt}}
\put(756,658){\rule[-0.175pt]{0.350pt}{1.325pt}}
\put(757,652){\rule[-0.175pt]{0.350pt}{1.325pt}}
\put(758,647){\rule[-0.175pt]{0.350pt}{1.325pt}}
\put(759,640){\rule[-0.175pt]{0.350pt}{1.686pt}}
\put(760,633){\rule[-0.175pt]{0.350pt}{1.686pt}}
\put(761,626){\rule[-0.175pt]{0.350pt}{1.686pt}}
\put(762,619){\rule[-0.175pt]{0.350pt}{1.686pt}}
\put(763,612){\rule[-0.175pt]{0.350pt}{1.686pt}}
\put(764,605){\rule[-0.175pt]{0.350pt}{1.686pt}}
\put(765,595){\rule[-0.175pt]{0.350pt}{2.264pt}}
\put(766,586){\rule[-0.175pt]{0.350pt}{2.264pt}}
\put(767,576){\rule[-0.175pt]{0.350pt}{2.264pt}}
\put(768,567){\rule[-0.175pt]{0.350pt}{2.264pt}}
\put(769,558){\rule[-0.175pt]{0.350pt}{2.264pt}}
\put(770,549){\rule[-0.175pt]{0.350pt}{2.048pt}}
\put(771,541){\rule[-0.175pt]{0.350pt}{2.048pt}}
\put(772,532){\rule[-0.175pt]{0.350pt}{2.048pt}}
\put(773,524){\rule[-0.175pt]{0.350pt}{2.048pt}}
\put(774,515){\rule[-0.175pt]{0.350pt}{2.048pt}}
\put(775,507){\rule[-0.175pt]{0.350pt}{2.048pt}}
\put(776,498){\rule[-0.175pt]{0.350pt}{2.008pt}}
\put(777,490){\rule[-0.175pt]{0.350pt}{2.008pt}}
\put(778,481){\rule[-0.175pt]{0.350pt}{2.008pt}}
\put(779,473){\rule[-0.175pt]{0.350pt}{2.008pt}}
\put(780,465){\rule[-0.175pt]{0.350pt}{2.008pt}}
\put(781,457){\rule[-0.175pt]{0.350pt}{2.007pt}}
\put(782,449){\rule[-0.175pt]{0.350pt}{1.847pt}}
\put(783,441){\rule[-0.175pt]{0.350pt}{1.847pt}}
\put(784,434){\rule[-0.175pt]{0.350pt}{1.847pt}}
\put(785,426){\rule[-0.175pt]{0.350pt}{1.847pt}}
\put(786,418){\rule[-0.175pt]{0.350pt}{1.847pt}}
\put(787,411){\rule[-0.175pt]{0.350pt}{1.847pt}}
\put(788,404){\rule[-0.175pt]{0.350pt}{1.606pt}}
\put(789,397){\rule[-0.175pt]{0.350pt}{1.606pt}}
\put(790,391){\rule[-0.175pt]{0.350pt}{1.606pt}}
\put(791,384){\rule[-0.175pt]{0.350pt}{1.606pt}}
\put(792,377){\rule[-0.175pt]{0.350pt}{1.606pt}}
\put(793,371){\rule[-0.175pt]{0.350pt}{1.606pt}}
\put(794,365){\rule[-0.175pt]{0.350pt}{1.285pt}}
\put(795,360){\rule[-0.175pt]{0.350pt}{1.285pt}}
\put(796,354){\rule[-0.175pt]{0.350pt}{1.285pt}}
\put(797,349){\rule[-0.175pt]{0.350pt}{1.285pt}}
\put(798,344){\rule[-0.175pt]{0.350pt}{1.285pt}}
\put(799,339){\rule[-0.175pt]{0.350pt}{1.285pt}}
\put(800,335){\rule[-0.175pt]{0.350pt}{0.803pt}}
\put(801,332){\rule[-0.175pt]{0.350pt}{0.803pt}}
\put(802,328){\rule[-0.175pt]{0.350pt}{0.803pt}}
\put(803,325){\rule[-0.175pt]{0.350pt}{0.803pt}}
\put(804,322){\rule[-0.175pt]{0.350pt}{0.803pt}}
\put(805,319){\rule[-0.175pt]{0.350pt}{0.803pt}}
\put(806,317){\usebox{\plotpoint}}
\put(807,316){\usebox{\plotpoint}}
\put(808,314){\usebox{\plotpoint}}
\put(809,313){\usebox{\plotpoint}}
\put(810,312){\usebox{\plotpoint}}
\put(811,311){\usebox{\plotpoint}}
\put(812,311){\rule[-0.175pt]{0.482pt}{0.350pt}}
\put(814,312){\rule[-0.175pt]{0.482pt}{0.350pt}}
\put(816,313){\rule[-0.175pt]{0.482pt}{0.350pt}}
\put(818,314){\rule[-0.175pt]{0.350pt}{0.771pt}}
\put(819,317){\rule[-0.175pt]{0.350pt}{0.771pt}}
\put(820,320){\rule[-0.175pt]{0.350pt}{0.771pt}}
\put(821,323){\rule[-0.175pt]{0.350pt}{0.771pt}}
\put(822,326){\rule[-0.175pt]{0.350pt}{0.771pt}}
\put(823,330){\rule[-0.175pt]{0.350pt}{1.084pt}}
\put(824,334){\rule[-0.175pt]{0.350pt}{1.084pt}}
\put(825,339){\rule[-0.175pt]{0.350pt}{1.084pt}}
\put(826,343){\rule[-0.175pt]{0.350pt}{1.084pt}}
\put(827,348){\rule[-0.175pt]{0.350pt}{1.084pt}}
\put(828,352){\rule[-0.175pt]{0.350pt}{1.084pt}}
\put(829,357){\rule[-0.175pt]{0.350pt}{1.405pt}}
\put(830,362){\rule[-0.175pt]{0.350pt}{1.405pt}}
\put(831,368){\rule[-0.175pt]{0.350pt}{1.405pt}}
\put(832,374){\rule[-0.175pt]{0.350pt}{1.405pt}}
\put(833,380){\rule[-0.175pt]{0.350pt}{1.405pt}}
\put(834,386){\rule[-0.175pt]{0.350pt}{1.405pt}}
\put(835,392){\rule[-0.175pt]{0.350pt}{1.686pt}}
\put(836,399){\rule[-0.175pt]{0.350pt}{1.686pt}}
\put(837,406){\rule[-0.175pt]{0.350pt}{1.686pt}}
\put(838,413){\rule[-0.175pt]{0.350pt}{1.686pt}}
\put(839,420){\rule[-0.175pt]{0.350pt}{1.686pt}}
\put(840,427){\rule[-0.175pt]{0.350pt}{1.686pt}}
\put(841,434){\rule[-0.175pt]{0.350pt}{1.847pt}}
\put(842,441){\rule[-0.175pt]{0.350pt}{1.847pt}}
\put(843,449){\rule[-0.175pt]{0.350pt}{1.847pt}}
\put(844,456){\rule[-0.175pt]{0.350pt}{1.847pt}}
\put(845,464){\rule[-0.175pt]{0.350pt}{1.847pt}}
\put(846,472){\rule[-0.175pt]{0.350pt}{1.847pt}}
\put(847,479){\rule[-0.175pt]{0.350pt}{1.887pt}}
\put(848,487){\rule[-0.175pt]{0.350pt}{1.887pt}}
\put(849,495){\rule[-0.175pt]{0.350pt}{1.887pt}}
\put(850,503){\rule[-0.175pt]{0.350pt}{1.887pt}}
\put(851,511){\rule[-0.175pt]{0.350pt}{1.887pt}}
\put(852,519){\rule[-0.175pt]{0.350pt}{1.887pt}}
\put(853,527){\rule[-0.175pt]{0.350pt}{1.807pt}}
\put(854,534){\rule[-0.175pt]{0.350pt}{1.807pt}}
\put(855,542){\rule[-0.175pt]{0.350pt}{1.807pt}}
\put(856,549){\rule[-0.175pt]{0.350pt}{1.807pt}}
\put(857,557){\rule[-0.175pt]{0.350pt}{1.807pt}}
\put(858,564){\rule[-0.175pt]{0.350pt}{1.807pt}}
\put(859,572){\rule[-0.175pt]{0.350pt}{1.606pt}}
\put(860,578){\rule[-0.175pt]{0.350pt}{1.606pt}}
\put(861,585){\rule[-0.175pt]{0.350pt}{1.606pt}}
\put(862,592){\rule[-0.175pt]{0.350pt}{1.606pt}}
\put(863,598){\rule[-0.175pt]{0.350pt}{1.606pt}}
\put(864,605){\rule[-0.175pt]{0.350pt}{1.606pt}}
\put(865,612){\rule[-0.175pt]{0.350pt}{1.325pt}}
\put(866,617){\rule[-0.175pt]{0.350pt}{1.325pt}}
\put(867,623){\rule[-0.175pt]{0.350pt}{1.325pt}}
\put(868,628){\rule[-0.175pt]{0.350pt}{1.325pt}}
\put(869,634){\rule[-0.175pt]{0.350pt}{1.325pt}}
\put(870,639){\rule[-0.175pt]{0.350pt}{1.325pt}}
\put(871,645){\rule[-0.175pt]{0.350pt}{0.964pt}}
\put(872,649){\rule[-0.175pt]{0.350pt}{0.964pt}}
\put(873,653){\rule[-0.175pt]{0.350pt}{0.964pt}}
\put(874,657){\rule[-0.175pt]{0.350pt}{0.964pt}}
\put(875,661){\rule[-0.175pt]{0.350pt}{0.964pt}}
\put(876,665){\rule[-0.175pt]{0.350pt}{0.964pt}}
\put(877,669){\rule[-0.175pt]{0.350pt}{0.578pt}}
\put(878,671){\rule[-0.175pt]{0.350pt}{0.578pt}}
\put(879,673){\rule[-0.175pt]{0.350pt}{0.578pt}}
\put(880,676){\rule[-0.175pt]{0.350pt}{0.578pt}}
\put(881,678){\rule[-0.175pt]{0.350pt}{0.578pt}}
\put(882,681){\rule[-0.175pt]{0.723pt}{0.350pt}}
\put(885,682){\rule[-0.175pt]{0.723pt}{0.350pt}}
\put(888,681){\rule[-0.175pt]{0.350pt}{0.402pt}}
\put(889,679){\rule[-0.175pt]{0.350pt}{0.402pt}}
\put(890,677){\rule[-0.175pt]{0.350pt}{0.402pt}}
\put(891,676){\rule[-0.175pt]{0.350pt}{0.402pt}}
\put(892,674){\rule[-0.175pt]{0.350pt}{0.402pt}}
\put(893,673){\rule[-0.175pt]{0.350pt}{0.401pt}}
\put(894,669){\rule[-0.175pt]{0.350pt}{0.843pt}}
\put(895,666){\rule[-0.175pt]{0.350pt}{0.843pt}}
\put(896,662){\rule[-0.175pt]{0.350pt}{0.843pt}}
\put(897,659){\rule[-0.175pt]{0.350pt}{0.843pt}}
\put(898,655){\rule[-0.175pt]{0.350pt}{0.843pt}}
\put(899,652){\rule[-0.175pt]{0.350pt}{0.843pt}}
\put(900,647){\rule[-0.175pt]{0.350pt}{1.164pt}}
\put(901,642){\rule[-0.175pt]{0.350pt}{1.164pt}}
\put(902,637){\rule[-0.175pt]{0.350pt}{1.164pt}}
\put(903,632){\rule[-0.175pt]{0.350pt}{1.164pt}}
\put(904,627){\rule[-0.175pt]{0.350pt}{1.164pt}}
\put(905,623){\rule[-0.175pt]{0.350pt}{1.164pt}}
\put(906,616){\rule[-0.175pt]{0.350pt}{1.486pt}}
\put(907,610){\rule[-0.175pt]{0.350pt}{1.486pt}}
\put(908,604){\rule[-0.175pt]{0.350pt}{1.486pt}}
\put(909,598){\rule[-0.175pt]{0.350pt}{1.486pt}}
\put(910,592){\rule[-0.175pt]{0.350pt}{1.486pt}}
\put(911,586){\rule[-0.175pt]{0.350pt}{1.486pt}}
\put(912,579){\rule[-0.175pt]{0.350pt}{1.646pt}}
\put(913,572){\rule[-0.175pt]{0.350pt}{1.646pt}}
\put(914,565){\rule[-0.175pt]{0.350pt}{1.646pt}}
\put(915,558){\rule[-0.175pt]{0.350pt}{1.646pt}}
\put(916,551){\rule[-0.175pt]{0.350pt}{1.646pt}}
\put(917,545){\rule[-0.175pt]{0.350pt}{1.646pt}}
\put(918,537){\rule[-0.175pt]{0.350pt}{1.767pt}}
\put(919,530){\rule[-0.175pt]{0.350pt}{1.767pt}}
\put(920,523){\rule[-0.175pt]{0.350pt}{1.767pt}}
\put(921,515){\rule[-0.175pt]{0.350pt}{1.767pt}}
\put(922,508){\rule[-0.175pt]{0.350pt}{1.767pt}}
\put(923,501){\rule[-0.175pt]{0.350pt}{1.767pt}}
\put(924,493){\rule[-0.175pt]{0.350pt}{1.726pt}}
\put(925,486){\rule[-0.175pt]{0.350pt}{1.726pt}}
\put(926,479){\rule[-0.175pt]{0.350pt}{1.726pt}}
\put(927,472){\rule[-0.175pt]{0.350pt}{1.726pt}}
\put(928,465){\rule[-0.175pt]{0.350pt}{1.726pt}}
\put(929,458){\rule[-0.175pt]{0.350pt}{1.726pt}}
\put(930,450){\rule[-0.175pt]{0.350pt}{1.879pt}}
\put(931,442){\rule[-0.175pt]{0.350pt}{1.879pt}}
\put(932,434){\rule[-0.175pt]{0.350pt}{1.879pt}}
\put(933,426){\rule[-0.175pt]{0.350pt}{1.879pt}}
\put(934,419){\rule[-0.175pt]{0.350pt}{1.879pt}}
\put(935,413){\rule[-0.175pt]{0.350pt}{1.365pt}}
\put(936,407){\rule[-0.175pt]{0.350pt}{1.365pt}}
\put(937,402){\rule[-0.175pt]{0.350pt}{1.365pt}}
\put(938,396){\rule[-0.175pt]{0.350pt}{1.365pt}}
\put(939,390){\rule[-0.175pt]{0.350pt}{1.365pt}}
\put(940,385){\rule[-0.175pt]{0.350pt}{1.365pt}}
\put(941,380){\rule[-0.175pt]{0.350pt}{1.004pt}}
\put(942,376){\rule[-0.175pt]{0.350pt}{1.004pt}}
\put(943,372){\rule[-0.175pt]{0.350pt}{1.004pt}}
\put(944,368){\rule[-0.175pt]{0.350pt}{1.004pt}}
\put(945,364){\rule[-0.175pt]{0.350pt}{1.004pt}}
\put(946,360){\rule[-0.175pt]{0.350pt}{1.004pt}}
\put(947,357){\rule[-0.175pt]{0.350pt}{0.683pt}}
\put(948,354){\rule[-0.175pt]{0.350pt}{0.683pt}}
\put(949,351){\rule[-0.175pt]{0.350pt}{0.683pt}}
\put(950,348){\rule[-0.175pt]{0.350pt}{0.683pt}}
\put(951,345){\rule[-0.175pt]{0.350pt}{0.683pt}}
\put(952,343){\rule[-0.175pt]{0.350pt}{0.683pt}}
\put(953,343){\usebox{\plotpoint}}
\put(954,342){\usebox{\plotpoint}}
\put(955,341){\usebox{\plotpoint}}
\put(956,340){\usebox{\plotpoint}}
\put(957,339){\usebox{\plotpoint}}
\put(958,338){\usebox{\plotpoint}}
\put(959,337){\usebox{\plotpoint}}
\put(960,338){\usebox{\plotpoint}}
\put(961,339){\usebox{\plotpoint}}
\put(962,340){\usebox{\plotpoint}}
\put(963,341){\usebox{\plotpoint}}
\put(965,342){\rule[-0.175pt]{0.350pt}{0.602pt}}
\put(966,344){\rule[-0.175pt]{0.350pt}{0.602pt}}
\put(967,347){\rule[-0.175pt]{0.350pt}{0.602pt}}
\put(968,349){\rule[-0.175pt]{0.350pt}{0.602pt}}
\put(969,352){\rule[-0.175pt]{0.350pt}{0.602pt}}
\put(970,354){\rule[-0.175pt]{0.350pt}{0.602pt}}
\put(971,357){\rule[-0.175pt]{0.350pt}{0.964pt}}
\put(972,361){\rule[-0.175pt]{0.350pt}{0.964pt}}
\put(973,365){\rule[-0.175pt]{0.350pt}{0.964pt}}
\put(974,369){\rule[-0.175pt]{0.350pt}{0.964pt}}
\put(975,373){\rule[-0.175pt]{0.350pt}{0.964pt}}
\put(976,377){\rule[-0.175pt]{0.350pt}{0.964pt}}
\put(977,381){\rule[-0.175pt]{0.350pt}{1.245pt}}
\put(978,386){\rule[-0.175pt]{0.350pt}{1.245pt}}
\put(979,391){\rule[-0.175pt]{0.350pt}{1.245pt}}
\put(980,396){\rule[-0.175pt]{0.350pt}{1.245pt}}
\put(981,401){\rule[-0.175pt]{0.350pt}{1.245pt}}
\put(982,406){\rule[-0.175pt]{0.350pt}{1.245pt}}
\put(983,411){\rule[-0.175pt]{0.350pt}{1.783pt}}
\put(984,419){\rule[-0.175pt]{0.350pt}{1.783pt}}
\put(985,426){\rule[-0.175pt]{0.350pt}{1.783pt}}
\put(986,434){\rule[-0.175pt]{0.350pt}{1.783pt}}
\put(987,441){\rule[-0.175pt]{0.350pt}{1.783pt}}
\put(988,448){\rule[-0.175pt]{0.350pt}{1.606pt}}
\put(989,455){\rule[-0.175pt]{0.350pt}{1.606pt}}
\put(990,462){\rule[-0.175pt]{0.350pt}{1.606pt}}
\put(991,468){\rule[-0.175pt]{0.350pt}{1.606pt}}
\put(992,475){\rule[-0.175pt]{0.350pt}{1.606pt}}
\put(993,482){\rule[-0.175pt]{0.350pt}{1.606pt}}
\put(994,488){\rule[-0.175pt]{0.350pt}{1.606pt}}
\put(995,495){\rule[-0.175pt]{0.350pt}{1.606pt}}
\put(996,502){\rule[-0.175pt]{0.350pt}{1.606pt}}
\put(997,508){\rule[-0.175pt]{0.350pt}{1.606pt}}
\put(998,515){\rule[-0.175pt]{0.350pt}{1.606pt}}
\put(999,522){\rule[-0.175pt]{0.350pt}{1.606pt}}
\put(1000,529){\rule[-0.175pt]{0.350pt}{1.526pt}}
\put(1001,535){\rule[-0.175pt]{0.350pt}{1.526pt}}
\put(1002,541){\rule[-0.175pt]{0.350pt}{1.526pt}}
\put(1003,547){\rule[-0.175pt]{0.350pt}{1.526pt}}
\put(1004,554){\rule[-0.175pt]{0.350pt}{1.526pt}}
\put(1005,560){\rule[-0.175pt]{0.350pt}{1.526pt}}
\put(1006,566){\rule[-0.175pt]{0.350pt}{1.365pt}}
\put(1007,572){\rule[-0.175pt]{0.350pt}{1.365pt}}
\put(1008,578){\rule[-0.175pt]{0.350pt}{1.365pt}}
\put(1009,584){\rule[-0.175pt]{0.350pt}{1.365pt}}
\put(1010,589){\rule[-0.175pt]{0.350pt}{1.365pt}}
\put(1011,595){\rule[-0.175pt]{0.350pt}{1.365pt}}
\put(1012,601){\rule[-0.175pt]{0.350pt}{1.084pt}}
\put(1013,605){\rule[-0.175pt]{0.350pt}{1.084pt}}
\put(1014,610){\rule[-0.175pt]{0.350pt}{1.084pt}}
\put(1015,614){\rule[-0.175pt]{0.350pt}{1.084pt}}
\put(1016,619){\rule[-0.175pt]{0.350pt}{1.084pt}}
\put(1017,623){\rule[-0.175pt]{0.350pt}{1.084pt}}
\put(1018,628){\rule[-0.175pt]{0.350pt}{0.763pt}}
\put(1019,631){\rule[-0.175pt]{0.350pt}{0.763pt}}
\put(1020,634){\rule[-0.175pt]{0.350pt}{0.763pt}}
\put(1021,637){\rule[-0.175pt]{0.350pt}{0.763pt}}
\put(1022,640){\rule[-0.175pt]{0.350pt}{0.763pt}}
\put(1023,643){\rule[-0.175pt]{0.350pt}{0.763pt}}
\put(1024,647){\rule[-0.175pt]{0.350pt}{0.402pt}}
\put(1025,648){\rule[-0.175pt]{0.350pt}{0.402pt}}
\put(1026,650){\rule[-0.175pt]{0.350pt}{0.402pt}}
\put(1027,652){\rule[-0.175pt]{0.350pt}{0.402pt}}
\put(1028,653){\rule[-0.175pt]{0.350pt}{0.402pt}}
\put(1029,655){\rule[-0.175pt]{0.350pt}{0.401pt}}
\put(1030,657){\rule[-0.175pt]{1.445pt}{0.350pt}}
\put(1036,655){\rule[-0.175pt]{0.350pt}{0.434pt}}
\put(1037,653){\rule[-0.175pt]{0.350pt}{0.434pt}}
\put(1038,651){\rule[-0.175pt]{0.350pt}{0.434pt}}
\put(1039,649){\rule[-0.175pt]{0.350pt}{0.434pt}}
\put(1040,648){\rule[-0.175pt]{0.350pt}{0.434pt}}
\put(1041,644){\rule[-0.175pt]{0.350pt}{0.763pt}}
\put(1042,641){\rule[-0.175pt]{0.350pt}{0.763pt}}
\put(1043,638){\rule[-0.175pt]{0.350pt}{0.763pt}}
\put(1044,635){\rule[-0.175pt]{0.350pt}{0.763pt}}
\put(1045,632){\rule[-0.175pt]{0.350pt}{0.763pt}}
\put(1046,629){\rule[-0.175pt]{0.350pt}{0.763pt}}
\put(1047,624){\rule[-0.175pt]{0.350pt}{1.044pt}}
\put(1048,620){\rule[-0.175pt]{0.350pt}{1.044pt}}
\put(1049,616){\rule[-0.175pt]{0.350pt}{1.044pt}}
\put(1050,611){\rule[-0.175pt]{0.350pt}{1.044pt}}
\put(1051,607){\rule[-0.175pt]{0.350pt}{1.044pt}}
\put(1052,603){\rule[-0.175pt]{0.350pt}{1.044pt}}
\put(1053,597){\rule[-0.175pt]{0.350pt}{1.325pt}}
\put(1054,592){\rule[-0.175pt]{0.350pt}{1.325pt}}
\put(1055,586){\rule[-0.175pt]{0.350pt}{1.325pt}}
\put(1056,581){\rule[-0.175pt]{0.350pt}{1.325pt}}
\put(1057,575){\rule[-0.175pt]{0.350pt}{1.325pt}}
\put(1058,570){\rule[-0.175pt]{0.350pt}{1.325pt}}
\put(1059,564){\rule[-0.175pt]{0.350pt}{1.445pt}}
\put(1060,558){\rule[-0.175pt]{0.350pt}{1.445pt}}
\put(1061,552){\rule[-0.175pt]{0.350pt}{1.445pt}}
\put(1062,546){\rule[-0.175pt]{0.350pt}{1.445pt}}
\put(1063,540){\rule[-0.175pt]{0.350pt}{1.445pt}}
\put(1064,534){\rule[-0.175pt]{0.350pt}{1.445pt}}
\put(1065,527){\rule[-0.175pt]{0.350pt}{1.486pt}}
\put(1066,521){\rule[-0.175pt]{0.350pt}{1.486pt}}
\put(1067,515){\rule[-0.175pt]{0.350pt}{1.486pt}}
\put(1068,509){\rule[-0.175pt]{0.350pt}{1.486pt}}
\put(1069,503){\rule[-0.175pt]{0.350pt}{1.486pt}}
\put(1070,497){\rule[-0.175pt]{0.350pt}{1.486pt}}
\put(1071,490){\rule[-0.175pt]{0.350pt}{1.486pt}}
\put(1072,484){\rule[-0.175pt]{0.350pt}{1.486pt}}
\put(1073,478){\rule[-0.175pt]{0.350pt}{1.486pt}}
\put(1074,472){\rule[-0.175pt]{0.350pt}{1.486pt}}
\put(1075,466){\rule[-0.175pt]{0.350pt}{1.486pt}}
\put(1076,460){\rule[-0.175pt]{0.350pt}{1.486pt}}
\put(1077,454){\rule[-0.175pt]{0.350pt}{1.325pt}}
\put(1078,449){\rule[-0.175pt]{0.350pt}{1.325pt}}
\put(1079,443){\rule[-0.175pt]{0.350pt}{1.325pt}}
\put(1080,438){\rule[-0.175pt]{0.350pt}{1.325pt}}
\put(1081,432){\rule[-0.175pt]{0.350pt}{1.325pt}}
\put(1082,427){\rule[-0.175pt]{0.350pt}{1.325pt}}
\put(1083,422){\rule[-0.175pt]{0.350pt}{1.124pt}}
\put(1084,417){\rule[-0.175pt]{0.350pt}{1.124pt}}
\put(1085,413){\rule[-0.175pt]{0.350pt}{1.124pt}}
\put(1086,408){\rule[-0.175pt]{0.350pt}{1.124pt}}
\put(1087,403){\rule[-0.175pt]{0.350pt}{1.124pt}}
\put(1088,399){\rule[-0.175pt]{0.350pt}{1.124pt}}
\put(1089,394){\rule[-0.175pt]{0.350pt}{1.012pt}}
\put(1090,390){\rule[-0.175pt]{0.350pt}{1.012pt}}
\put(1091,386){\rule[-0.175pt]{0.350pt}{1.012pt}}
\put(1092,382){\rule[-0.175pt]{0.350pt}{1.012pt}}
\put(1093,378){\rule[-0.175pt]{0.350pt}{1.012pt}}
\put(1094,375){\rule[-0.175pt]{0.350pt}{0.522pt}}
\put(1095,373){\rule[-0.175pt]{0.350pt}{0.522pt}}
\put(1096,371){\rule[-0.175pt]{0.350pt}{0.522pt}}
\put(1097,369){\rule[-0.175pt]{0.350pt}{0.522pt}}
\put(1098,367){\rule[-0.175pt]{0.350pt}{0.522pt}}
\put(1099,365){\rule[-0.175pt]{0.350pt}{0.522pt}}
\put(1100,365){\usebox{\plotpoint}}
\put(1100,365){\rule[-0.175pt]{0.361pt}{0.350pt}}
\put(1101,364){\rule[-0.175pt]{0.361pt}{0.350pt}}
\put(1103,363){\rule[-0.175pt]{0.361pt}{0.350pt}}
\put(1104,362){\rule[-0.175pt]{0.361pt}{0.350pt}}
\put(1106,361){\usebox{\plotpoint}}
\put(1107,362){\usebox{\plotpoint}}
\put(1108,363){\usebox{\plotpoint}}
\put(1109,364){\usebox{\plotpoint}}
\put(1110,365){\usebox{\plotpoint}}
\put(1111,366){\usebox{\plotpoint}}
\put(1112,366){\rule[-0.175pt]{0.350pt}{0.522pt}}
\put(1113,368){\rule[-0.175pt]{0.350pt}{0.522pt}}
\put(1114,370){\rule[-0.175pt]{0.350pt}{0.522pt}}
\put(1115,372){\rule[-0.175pt]{0.350pt}{0.522pt}}
\put(1116,374){\rule[-0.175pt]{0.350pt}{0.522pt}}
\put(1117,376){\rule[-0.175pt]{0.350pt}{0.522pt}}
\put(1118,378){\rule[-0.175pt]{0.350pt}{0.883pt}}
\put(1119,382){\rule[-0.175pt]{0.350pt}{0.883pt}}
\put(1120,386){\rule[-0.175pt]{0.350pt}{0.883pt}}
\put(1121,389){\rule[-0.175pt]{0.350pt}{0.883pt}}
\put(1122,393){\rule[-0.175pt]{0.350pt}{0.883pt}}
\put(1123,397){\rule[-0.175pt]{0.350pt}{0.883pt}}
\put(1124,400){\rule[-0.175pt]{0.350pt}{1.124pt}}
\put(1125,405){\rule[-0.175pt]{0.350pt}{1.124pt}}
\put(1126,410){\rule[-0.175pt]{0.350pt}{1.124pt}}
\put(1127,414){\rule[-0.175pt]{0.350pt}{1.124pt}}
\put(1128,419){\rule[-0.175pt]{0.350pt}{1.124pt}}
\put(1129,424){\rule[-0.175pt]{0.350pt}{1.124pt}}
\put(1130,428){\rule[-0.175pt]{0.350pt}{1.285pt}}
\put(1131,434){\rule[-0.175pt]{0.350pt}{1.285pt}}
\put(1132,439){\rule[-0.175pt]{0.350pt}{1.285pt}}
\put(1133,445){\rule[-0.175pt]{0.350pt}{1.285pt}}
\put(1134,450){\rule[-0.175pt]{0.350pt}{1.285pt}}
\put(1135,455){\rule[-0.175pt]{0.350pt}{1.285pt}}
\put(1136,461){\rule[-0.175pt]{0.350pt}{1.365pt}}
\put(1137,466){\rule[-0.175pt]{0.350pt}{1.365pt}}
\put(1138,472){\rule[-0.175pt]{0.350pt}{1.365pt}}
\put(1139,477){\rule[-0.175pt]{0.350pt}{1.365pt}}
\put(1140,483){\rule[-0.175pt]{0.350pt}{1.365pt}}
\put(1141,489){\rule[-0.175pt]{0.350pt}{1.365pt}}
\put(1142,494){\rule[-0.175pt]{0.350pt}{1.686pt}}
\put(1143,502){\rule[-0.175pt]{0.350pt}{1.686pt}}
\put(1144,509){\rule[-0.175pt]{0.350pt}{1.686pt}}
\put(1145,516){\rule[-0.175pt]{0.350pt}{1.686pt}}
\put(1146,523){\rule[-0.175pt]{0.350pt}{1.686pt}}
\put(1147,530){\rule[-0.175pt]{0.350pt}{1.285pt}}
\put(1148,535){\rule[-0.175pt]{0.350pt}{1.285pt}}
\put(1149,540){\rule[-0.175pt]{0.350pt}{1.285pt}}
\put(1150,545){\rule[-0.175pt]{0.350pt}{1.285pt}}
\put(1151,551){\rule[-0.175pt]{0.350pt}{1.285pt}}
\put(1152,556){\rule[-0.175pt]{0.350pt}{1.285pt}}
\put(1153,561){\rule[-0.175pt]{0.350pt}{1.164pt}}
\put(1154,566){\rule[-0.175pt]{0.350pt}{1.164pt}}
\put(1155,571){\rule[-0.175pt]{0.350pt}{1.164pt}}
\put(1156,576){\rule[-0.175pt]{0.350pt}{1.164pt}}
\put(1157,581){\rule[-0.175pt]{0.350pt}{1.164pt}}
\put(1158,586){\rule[-0.175pt]{0.350pt}{1.164pt}}
\put(1159,590){\rule[-0.175pt]{0.350pt}{0.923pt}}
\put(1160,594){\rule[-0.175pt]{0.350pt}{0.923pt}}
\put(1161,598){\rule[-0.175pt]{0.350pt}{0.923pt}}
\put(1162,602){\rule[-0.175pt]{0.350pt}{0.923pt}}
\put(1163,606){\rule[-0.175pt]{0.350pt}{0.923pt}}
\put(1164,610){\rule[-0.175pt]{0.350pt}{0.923pt}}
\put(1165,613){\rule[-0.175pt]{0.350pt}{0.602pt}}
\put(1166,616){\rule[-0.175pt]{0.350pt}{0.602pt}}
\put(1167,619){\rule[-0.175pt]{0.350pt}{0.602pt}}
\put(1168,621){\rule[-0.175pt]{0.350pt}{0.602pt}}
\put(1169,624){\rule[-0.175pt]{0.350pt}{0.602pt}}
\put(1170,626){\rule[-0.175pt]{0.350pt}{0.602pt}}
\put(1171,629){\usebox{\plotpoint}}
\put(1172,630){\usebox{\plotpoint}}
\put(1173,631){\usebox{\plotpoint}}
\put(1174,632){\usebox{\plotpoint}}
\put(1175,633){\usebox{\plotpoint}}
\put(1176,634){\usebox{\plotpoint}}
\put(1177,636){\rule[-0.175pt]{1.445pt}{0.350pt}}
\put(1183,633){\rule[-0.175pt]{0.350pt}{0.361pt}}
\put(1184,632){\rule[-0.175pt]{0.350pt}{0.361pt}}
\put(1185,630){\rule[-0.175pt]{0.350pt}{0.361pt}}
\put(1186,629){\rule[-0.175pt]{0.350pt}{0.361pt}}
\put(1187,627){\rule[-0.175pt]{0.350pt}{0.361pt}}
\put(1188,626){\rule[-0.175pt]{0.350pt}{0.361pt}}
\put(1189,623){\rule[-0.175pt]{0.350pt}{0.683pt}}
\put(1190,620){\rule[-0.175pt]{0.350pt}{0.683pt}}
\put(1191,617){\rule[-0.175pt]{0.350pt}{0.683pt}}
\put(1192,614){\rule[-0.175pt]{0.350pt}{0.683pt}}
\put(1193,611){\rule[-0.175pt]{0.350pt}{0.683pt}}
\put(1194,609){\rule[-0.175pt]{0.350pt}{0.683pt}}
\put(1195,604){\rule[-0.175pt]{0.350pt}{1.108pt}}
\put(1196,599){\rule[-0.175pt]{0.350pt}{1.108pt}}
\put(1197,595){\rule[-0.175pt]{0.350pt}{1.108pt}}
\put(1198,590){\rule[-0.175pt]{0.350pt}{1.108pt}}
\put(1199,586){\rule[-0.175pt]{0.350pt}{1.108pt}}
\put(1200,581){\rule[-0.175pt]{0.350pt}{1.164pt}}
\put(1201,576){\rule[-0.175pt]{0.350pt}{1.164pt}}
\put(1202,571){\rule[-0.175pt]{0.350pt}{1.164pt}}
\put(1203,566){\rule[-0.175pt]{0.350pt}{1.164pt}}
\put(1204,561){\rule[-0.175pt]{0.350pt}{1.164pt}}
\put(1205,557){\rule[-0.175pt]{0.350pt}{1.164pt}}
\put(1206,551){\rule[-0.175pt]{0.350pt}{1.245pt}}
\put(1207,546){\rule[-0.175pt]{0.350pt}{1.245pt}}
\put(1208,541){\rule[-0.175pt]{0.350pt}{1.245pt}}
\put(1209,536){\rule[-0.175pt]{0.350pt}{1.245pt}}
\put(1210,531){\rule[-0.175pt]{0.350pt}{1.245pt}}
\put(1211,526){\rule[-0.175pt]{0.350pt}{1.245pt}}
\put(1212,520){\rule[-0.175pt]{0.350pt}{1.285pt}}
\put(1213,515){\rule[-0.175pt]{0.350pt}{1.285pt}}
\put(1214,510){\rule[-0.175pt]{0.350pt}{1.285pt}}
\put(1215,504){\rule[-0.175pt]{0.350pt}{1.285pt}}
\put(1216,499){\rule[-0.175pt]{0.350pt}{1.285pt}}
\put(1217,494){\rule[-0.175pt]{0.350pt}{1.285pt}}
\put(1218,488){\rule[-0.175pt]{0.350pt}{1.245pt}}
\put(1219,483){\rule[-0.175pt]{0.350pt}{1.245pt}}
\put(1220,478){\rule[-0.175pt]{0.350pt}{1.245pt}}
\put(1221,473){\rule[-0.175pt]{0.350pt}{1.245pt}}
\put(1222,468){\rule[-0.175pt]{0.350pt}{1.245pt}}
\put(1223,463){\rule[-0.175pt]{0.350pt}{1.245pt}}
\put(1224,458){\rule[-0.175pt]{0.350pt}{1.164pt}}
\put(1225,453){\rule[-0.175pt]{0.350pt}{1.164pt}}
\put(1226,448){\rule[-0.175pt]{0.350pt}{1.164pt}}
\put(1227,443){\rule[-0.175pt]{0.350pt}{1.164pt}}
\put(1228,438){\rule[-0.175pt]{0.350pt}{1.164pt}}
\put(1229,434){\rule[-0.175pt]{0.350pt}{1.164pt}}
\put(1230,430){\rule[-0.175pt]{0.350pt}{0.923pt}}
\put(1231,426){\rule[-0.175pt]{0.350pt}{0.923pt}}
\put(1232,422){\rule[-0.175pt]{0.350pt}{0.923pt}}
\put(1233,418){\rule[-0.175pt]{0.350pt}{0.923pt}}
\put(1234,414){\rule[-0.175pt]{0.350pt}{0.923pt}}
\put(1235,411){\rule[-0.175pt]{0.350pt}{0.923pt}}
\put(1236,408){\rule[-0.175pt]{0.350pt}{0.723pt}}
\put(1237,405){\rule[-0.175pt]{0.350pt}{0.723pt}}
\put(1238,402){\rule[-0.175pt]{0.350pt}{0.723pt}}
\put(1239,399){\rule[-0.175pt]{0.350pt}{0.723pt}}
\put(1240,396){\rule[-0.175pt]{0.350pt}{0.723pt}}
\put(1241,393){\rule[-0.175pt]{0.350pt}{0.723pt}}
\put(1242,391){\rule[-0.175pt]{0.350pt}{0.401pt}}
\put(1243,389){\rule[-0.175pt]{0.350pt}{0.401pt}}
\put(1244,388){\rule[-0.175pt]{0.350pt}{0.401pt}}
\put(1245,386){\rule[-0.175pt]{0.350pt}{0.401pt}}
\put(1246,384){\rule[-0.175pt]{0.350pt}{0.401pt}}
\put(1247,383){\rule[-0.175pt]{0.350pt}{0.401pt}}
\put(1248,383){\usebox{\plotpoint}}
\put(1248,383){\rule[-0.175pt]{0.602pt}{0.350pt}}
\put(1250,382){\rule[-0.175pt]{0.602pt}{0.350pt}}
\put(1253,381){\usebox{\plotpoint}}
\put(1254,382){\usebox{\plotpoint}}
\put(1255,383){\usebox{\plotpoint}}
\put(1256,384){\usebox{\plotpoint}}
\put(1257,385){\usebox{\plotpoint}}
\put(1258,386){\usebox{\plotpoint}}
\put(1259,386){\rule[-0.175pt]{0.350pt}{0.522pt}}
\put(1260,388){\rule[-0.175pt]{0.350pt}{0.522pt}}
\put(1261,390){\rule[-0.175pt]{0.350pt}{0.522pt}}
\put(1262,392){\rule[-0.175pt]{0.350pt}{0.522pt}}
\put(1263,394){\rule[-0.175pt]{0.350pt}{0.522pt}}
\put(1264,396){\rule[-0.175pt]{0.350pt}{0.522pt}}
\put(1265,398){\rule[-0.175pt]{0.350pt}{0.763pt}}
\put(1266,402){\rule[-0.175pt]{0.350pt}{0.763pt}}
\put(1267,405){\rule[-0.175pt]{0.350pt}{0.763pt}}
\put(1268,408){\rule[-0.175pt]{0.350pt}{0.763pt}}
\put(1269,411){\rule[-0.175pt]{0.350pt}{0.763pt}}
\put(1270,414){\rule[-0.175pt]{0.350pt}{0.763pt}}
\put(1271,417){\rule[-0.175pt]{0.350pt}{0.964pt}}
\put(1272,422){\rule[-0.175pt]{0.350pt}{0.964pt}}
\put(1273,426){\rule[-0.175pt]{0.350pt}{0.964pt}}
\put(1274,430){\rule[-0.175pt]{0.350pt}{0.964pt}}
\put(1275,434){\rule[-0.175pt]{0.350pt}{0.964pt}}
\put(1276,438){\rule[-0.175pt]{0.350pt}{0.964pt}}
\put(1277,442){\rule[-0.175pt]{0.350pt}{1.124pt}}
\put(1278,446){\rule[-0.175pt]{0.350pt}{1.124pt}}
\put(1279,451){\rule[-0.175pt]{0.350pt}{1.124pt}}
\put(1280,455){\rule[-0.175pt]{0.350pt}{1.124pt}}
\put(1281,460){\rule[-0.175pt]{0.350pt}{1.124pt}}
\put(1282,465){\rule[-0.175pt]{0.350pt}{1.124pt}}
\put(1283,469){\rule[-0.175pt]{0.350pt}{1.205pt}}
\put(1284,475){\rule[-0.175pt]{0.350pt}{1.204pt}}
\put(1285,480){\rule[-0.175pt]{0.350pt}{1.204pt}}
\put(1286,485){\rule[-0.175pt]{0.350pt}{1.204pt}}
\put(1287,490){\rule[-0.175pt]{0.350pt}{1.204pt}}
\put(1288,495){\rule[-0.175pt]{0.350pt}{1.204pt}}
\put(1289,500){\rule[-0.175pt]{0.350pt}{1.204pt}}
\put(1290,505){\rule[-0.175pt]{0.350pt}{1.204pt}}
\put(1291,510){\rule[-0.175pt]{0.350pt}{1.204pt}}
\put(1292,515){\rule[-0.175pt]{0.350pt}{1.204pt}}
\put(1293,520){\rule[-0.175pt]{0.350pt}{1.204pt}}
\put(1294,525){\rule[-0.175pt]{0.350pt}{1.204pt}}
\put(1295,530){\rule[-0.175pt]{0.350pt}{1.124pt}}
\put(1296,534){\rule[-0.175pt]{0.350pt}{1.124pt}}
\put(1297,539){\rule[-0.175pt]{0.350pt}{1.124pt}}
\put(1298,544){\rule[-0.175pt]{0.350pt}{1.124pt}}
\put(1299,548){\rule[-0.175pt]{0.350pt}{1.124pt}}
\put(1300,553){\rule[-0.175pt]{0.350pt}{1.124pt}}
\put(1301,558){\rule[-0.175pt]{0.350pt}{1.156pt}}
\put(1302,562){\rule[-0.175pt]{0.350pt}{1.156pt}}
\put(1303,567){\rule[-0.175pt]{0.350pt}{1.156pt}}
\put(1304,572){\rule[-0.175pt]{0.350pt}{1.156pt}}
\put(1305,577){\rule[-0.175pt]{0.350pt}{1.156pt}}
\put(1306,581){\rule[-0.175pt]{0.350pt}{0.723pt}}
\put(1307,585){\rule[-0.175pt]{0.350pt}{0.723pt}}
\put(1308,588){\rule[-0.175pt]{0.350pt}{0.723pt}}
\put(1309,591){\rule[-0.175pt]{0.350pt}{0.723pt}}
\put(1310,594){\rule[-0.175pt]{0.350pt}{0.723pt}}
\put(1311,597){\rule[-0.175pt]{0.350pt}{0.723pt}}
\put(1312,600){\rule[-0.175pt]{0.350pt}{0.522pt}}
\put(1313,602){\rule[-0.175pt]{0.350pt}{0.522pt}}
\put(1314,604){\rule[-0.175pt]{0.350pt}{0.522pt}}
\put(1315,606){\rule[-0.175pt]{0.350pt}{0.522pt}}
\put(1316,608){\rule[-0.175pt]{0.350pt}{0.522pt}}
\put(1317,610){\rule[-0.175pt]{0.350pt}{0.522pt}}
\put(1318,613){\usebox{\plotpoint}}
\put(1319,614){\usebox{\plotpoint}}
\put(1320,615){\usebox{\plotpoint}}
\put(1321,616){\usebox{\plotpoint}}
\put(1322,617){\usebox{\plotpoint}}
\put(1323,618){\rule[-0.175pt]{0.723pt}{0.350pt}}
\put(1327,617){\rule[-0.175pt]{0.723pt}{0.350pt}}
\put(1330,614){\usebox{\plotpoint}}
\put(1331,613){\usebox{\plotpoint}}
\put(1332,612){\usebox{\plotpoint}}
\put(1333,610){\usebox{\plotpoint}}
\put(1334,609){\usebox{\plotpoint}}
\put(1335,608){\usebox{\plotpoint}}
\put(1336,605){\rule[-0.175pt]{0.350pt}{0.642pt}}
\put(1337,602){\rule[-0.175pt]{0.350pt}{0.642pt}}
\put(1338,599){\rule[-0.175pt]{0.350pt}{0.642pt}}
\put(1339,597){\rule[-0.175pt]{0.350pt}{0.642pt}}
\put(1340,594){\rule[-0.175pt]{0.350pt}{0.642pt}}
\put(1341,592){\rule[-0.175pt]{0.350pt}{0.642pt}}
\put(1342,588){\rule[-0.175pt]{0.350pt}{0.843pt}}
\put(1343,585){\rule[-0.175pt]{0.350pt}{0.843pt}}
\put(1344,581){\rule[-0.175pt]{0.350pt}{0.843pt}}
\put(1345,578){\rule[-0.175pt]{0.350pt}{0.843pt}}
\put(1346,574){\rule[-0.175pt]{0.350pt}{0.843pt}}
\put(1347,571){\rule[-0.175pt]{0.350pt}{0.843pt}}
\put(1348,567){\rule[-0.175pt]{0.350pt}{0.964pt}}
\put(1349,563){\rule[-0.175pt]{0.350pt}{0.964pt}}
\put(1350,559){\rule[-0.175pt]{0.350pt}{0.964pt}}
\put(1351,555){\rule[-0.175pt]{0.350pt}{0.964pt}}
\put(1352,551){\rule[-0.175pt]{0.350pt}{0.964pt}}
\put(1353,547){\rule[-0.175pt]{0.350pt}{0.964pt}}
\put(1354,541){\rule[-0.175pt]{0.350pt}{1.301pt}}
\put(1355,536){\rule[-0.175pt]{0.350pt}{1.301pt}}
\put(1356,530){\rule[-0.175pt]{0.350pt}{1.301pt}}
\put(1357,525){\rule[-0.175pt]{0.350pt}{1.301pt}}
\put(1358,520){\rule[-0.175pt]{0.350pt}{1.301pt}}
\put(1359,515){\rule[-0.175pt]{0.350pt}{1.124pt}}
\put(1360,510){\rule[-0.175pt]{0.350pt}{1.124pt}}
\put(1361,506){\rule[-0.175pt]{0.350pt}{1.124pt}}
\put(1362,501){\rule[-0.175pt]{0.350pt}{1.124pt}}
\put(1363,496){\rule[-0.175pt]{0.350pt}{1.124pt}}
\put(1364,492){\rule[-0.175pt]{0.350pt}{1.124pt}}
\put(1365,487){\rule[-0.175pt]{0.350pt}{1.084pt}}
\put(1366,483){\rule[-0.175pt]{0.350pt}{1.084pt}}
\put(1367,478){\rule[-0.175pt]{0.350pt}{1.084pt}}
\put(1368,474){\rule[-0.175pt]{0.350pt}{1.084pt}}
\put(1369,469){\rule[-0.175pt]{0.350pt}{1.084pt}}
\put(1370,465){\rule[-0.175pt]{0.350pt}{1.084pt}}
\put(1371,461){\rule[-0.175pt]{0.350pt}{0.964pt}}
\put(1372,457){\rule[-0.175pt]{0.350pt}{0.964pt}}
\put(1373,453){\rule[-0.175pt]{0.350pt}{0.964pt}}
\put(1374,449){\rule[-0.175pt]{0.350pt}{0.964pt}}
\put(1375,445){\rule[-0.175pt]{0.350pt}{0.964pt}}
\put(1376,441){\rule[-0.175pt]{0.350pt}{0.964pt}}
\put(1377,437){\rule[-0.175pt]{0.350pt}{0.763pt}}
\put(1378,434){\rule[-0.175pt]{0.350pt}{0.763pt}}
\put(1379,431){\rule[-0.175pt]{0.350pt}{0.763pt}}
\put(1380,428){\rule[-0.175pt]{0.350pt}{0.763pt}}
\put(1381,425){\rule[-0.175pt]{0.350pt}{0.763pt}}
\put(1382,422){\rule[-0.175pt]{0.350pt}{0.763pt}}
\put(1383,419){\rule[-0.175pt]{0.350pt}{0.602pt}}
\put(1384,417){\rule[-0.175pt]{0.350pt}{0.602pt}}
\put(1385,414){\rule[-0.175pt]{0.350pt}{0.602pt}}
\put(1386,412){\rule[-0.175pt]{0.350pt}{0.602pt}}
\put(1387,409){\rule[-0.175pt]{0.350pt}{0.602pt}}
\put(1388,407){\rule[-0.175pt]{0.350pt}{0.602pt}}
\put(1389,405){\usebox{\plotpoint}}
\put(1390,404){\usebox{\plotpoint}}
\put(1391,402){\usebox{\plotpoint}}
\put(1392,401){\usebox{\plotpoint}}
\put(1393,400){\usebox{\plotpoint}}
\put(1394,399){\usebox{\plotpoint}}
\put(1395,399){\rule[-0.175pt]{1.445pt}{0.350pt}}
\put(1401,398){\usebox{\plotpoint}}
\put(1402,399){\usebox{\plotpoint}}
\put(1403,400){\usebox{\plotpoint}}
\put(1404,401){\usebox{\plotpoint}}
\put(1405,402){\usebox{\plotpoint}}
\put(1406,403){\usebox{\plotpoint}}
\put(1407,403){\rule[-0.175pt]{0.350pt}{0.578pt}}
\put(1408,405){\rule[-0.175pt]{0.350pt}{0.578pt}}
\put(1409,407){\rule[-0.175pt]{0.350pt}{0.578pt}}
\put(1410,410){\rule[-0.175pt]{0.350pt}{0.578pt}}
\put(1411,412){\rule[-0.175pt]{0.350pt}{0.578pt}}
\put(1412,414){\rule[-0.175pt]{0.350pt}{0.683pt}}
\put(1413,417){\rule[-0.175pt]{0.350pt}{0.683pt}}
\put(1414,420){\rule[-0.175pt]{0.350pt}{0.683pt}}
\put(1415,423){\rule[-0.175pt]{0.350pt}{0.683pt}}
\put(1416,426){\rule[-0.175pt]{0.350pt}{0.683pt}}
\put(1417,429){\rule[-0.175pt]{0.350pt}{0.683pt}}
\put(1418,432){\rule[-0.175pt]{0.350pt}{0.883pt}}
\put(1419,435){\rule[-0.175pt]{0.350pt}{0.883pt}}
\put(1420,439){\rule[-0.175pt]{0.350pt}{0.883pt}}
\put(1421,442){\rule[-0.175pt]{0.350pt}{0.883pt}}
\put(1422,446){\rule[-0.175pt]{0.350pt}{0.883pt}}
\put(1423,450){\rule[-0.175pt]{0.350pt}{0.883pt}}
\put(1424,453){\rule[-0.175pt]{0.350pt}{0.964pt}}
\put(1425,458){\rule[-0.175pt]{0.350pt}{0.964pt}}
\put(1426,462){\rule[-0.175pt]{0.350pt}{0.964pt}}
\put(1427,466){\rule[-0.175pt]{0.350pt}{0.964pt}}
\put(1428,470){\rule[-0.175pt]{0.350pt}{0.964pt}}
\put(1429,474){\rule[-0.175pt]{0.350pt}{0.964pt}}
\put(1430,478){\rule[-0.175pt]{0.350pt}{1.044pt}}
\put(1431,482){\rule[-0.175pt]{0.350pt}{1.044pt}}
\put(1432,486){\rule[-0.175pt]{0.350pt}{1.044pt}}
\put(1433,491){\rule[-0.175pt]{0.350pt}{1.044pt}}
\put(1434,495){\rule[-0.175pt]{0.350pt}{1.044pt}}
\put(1435,499){\rule[-0.175pt]{0.350pt}{1.044pt}}
\sbox{\plotpoint}{\rule[-0.250pt]{0.500pt}{0.500pt}}%
%\put(264,473){\usebox{\plotpoint}}
%\put(264,473){\usebox{\plotpoint}}
%\put(284,473){\usebox{\plotpoint}}
%\put(305,473){\usebox{\plotpoint}}
%\put(326,473){\usebox{\plotpoint}}
%\put(347,473){\usebox{\plotpoint}}
%\put(367,473){\usebox{\plotpoint}}
%\put(388,473){\usebox{\plotpoint}}
%\put(409,473){\usebox{\plotpoint}}
%\put(430,473){\usebox{\plotpoint}}
%\put(450,473){\usebox{\plotpoint}}
%\put(471,473){\usebox{\plotpoint}}
%\put(492,473){\usebox{\plotpoint}}
%\put(513,473){\usebox{\plotpoint}}
%\put(533,473){\usebox{\plotpoint}}
%\put(554,473){\usebox{\plotpoint}}
%\put(575,473){\usebox{\plotpoint}}
%\put(596,473){\usebox{\plotpoint}}
%\put(616,473){\usebox{\plotpoint}}
%\put(637,473){\usebox{\plotpoint}}
%\put(658,473){\usebox{\plotpoint}}
%\put(679,473){\usebox{\plotpoint}}
%\put(699,473){\usebox{\plotpoint}}
%\put(720,473){\usebox{\plotpoint}}
%\put(741,473){\usebox{\plotpoint}}
%\put(762,473){\usebox{\plotpoint}}
%\put(782,473){\usebox{\plotpoint}}
%\put(803,473){\usebox{\plotpoint}}
%\put(824,473){\usebox{\plotpoint}}
%\put(845,473){\usebox{\plotpoint}}
%\put(865,473){\usebox{\plotpoint}}
%\put(886,473){\usebox{\plotpoint}}
%\put(907,473){\usebox{\plotpoint}}
%\put(928,473){\usebox{\plotpoint}}
%\put(948,473){\usebox{\plotpoint}}
%\put(969,473){\usebox{\plotpoint}}
%\put(990,473){\usebox{\plotpoint}}
%\put(1011,473){\usebox{\plotpoint}}
%\put(1031,473){\usebox{\plotpoint}}
%\put(1052,473){\usebox{\plotpoint}}
%\put(1073,473){\usebox{\plotpoint}}
%\put(1094,473){\usebox{\plotpoint}}
%\put(1114,473){\usebox{\plotpoint}}
%\put(1135,473){\usebox{\plotpoint}}
%\put(1156,473){\usebox{\plotpoint}}
%\put(1177,473){\usebox{\plotpoint}}
%\put(1197,473){\usebox{\plotpoint}}
%\put(1218,473){\usebox{\plotpoint}}
%\put(1239,473){\usebox{\plotpoint}}
%\put(1260,473){\usebox{\plotpoint}}
%\put(1281,473){\usebox{\plotpoint}}
%\put(1301,473){\usebox{\plotpoint}}
%\put(1322,473){\usebox{\plotpoint}}
%\put(1343,473){\usebox{\plotpoint}}
%\put(1364,473){\usebox{\plotpoint}}
%\put(1384,473){\usebox{\plotpoint}}
%\put(1405,473){\usebox{\plotpoint}}
%\put(1426,473){\usebox{\plotpoint}}
%\put(1436,473){\usebox{\plotpoint}}
\put(100,480){$v$}
\put(90,700){(A)}
\put(820,40){$t$}
\put(1300,40){(s)}
\end{picture}

\caption[Voltage source transient sine ({\tt SIN}) waveform]{Voltage source transient
sine ({\tt SIN}) waveform for\newline \hspace*{\fill}
{\tt SIN(0.1 0.8 2 1 0.3 )}.  \label{fig:vsin} \hspace*{\fill}}
\end{figure}
