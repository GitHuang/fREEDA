\element{X}{Subcircuit Call}
\begin{figure}[h]
\centering
\ \pfig{x_spice.ps}
\caption{X --- subcircuit call element.}
\end{figure}

\form{ {\tt X}name $N_1$ \B $N_2$ $N_3$ ... $N_N$\E  SubcircuitName}

\pspiceform{ {\tt X}name $N_1$ \B $N_2$ $N_3$ ... $N_N$\E  SubcircuitName
\B PARAMS: \B keyword = {\tt \{} Expression {\tt \}} $\ldots$ \E
\B Keyword = Value $\ldots$ \E \E}

\begin{widelist}
\item[$N_1$] is the first node of the subcircuit.
\item[$N_N$] is the $N$th node of the subcircuit.
\item[{\it SubcircuitName}] is the name of the subcircuit.
\item[{\tt PARAMS:}] indicates that parameters are to be passed to
the subcircuit.
\item[{\tt keyword:}] is keyword corresponding to the keywords defined in the
{\tt .SUBCKT} statement. (See page \pageref{.SUBCKTstatement}).
\item[{\tt value:}] is numeric value.
\item[{\tt Expression:}] is an algebraic expression which evaluates to a numeric
value.  {\tt .SUBCKT} statement. (See section \ref{section:algebraic:expression}
on page \pageref{section:algebraic:expression}).
\end{widelist}
\example{X1 2 4 17 3 1 MULTI}

Subcircuits are incorporated by using the ``{\tt X}'' element.
The number of nodes of the ``{\tt X}'' element must correspond to
the number of nodes in the definition of the subcircuit (i.e. is
on the {\tt .SUBCKT} statement (see page \pageref{.SUBCKTstatement}).
