\addcontentsline{toc}{part}{PART II  SPICE SYNTAX}

\chapter{Input File\label{chapter:input}}

\section{Introduction}

The operation of \spice\ is controlled by statements which are embedded
in an input file which includes as well descriptions of elements and
their topology.
The description of the elements and their topology is also known as a
netlist.\marginid{netlist}\index{netlist}
The output or results of a \spice\ run are logged in an output file
and in more modern versions of \spice , in a data file for
subsequent interactive plotting and analysis.  With \pspice\ the program
for subsequent analysis is probe and with \spicethree\ the comparable tool
is \nutmeg .

\section{Circuit Model}

The model used by \spice\  to represent circuits is as shown
in Figure \ref{spice:circuit}.
\begin{figure}
\begin{center}
\ \epsfxsize=5.5in\pfig{spcircuit.ps}
\end{center}
\caption{The \spice\ circuit representation.\label{spice:circuit}}
\end{figure}
\marginid{model}\marginid{subcircuits}
\index{model}\index{subcircuits}
\spice\ supports a hierarchical description of a circuit with subcircuits.
A large number of parameters are required for many
elements, especially for active devices, and for these the model
concept
is introduced where most of the parameters of active elements can be defined
separately from invocation of the element.  This permits a single model
description to be used by many elements.
A model is specific to a particular element type but not all elements have
models.
Mostly models are used to set the parameters describing a semiconductor
fabrication process and so are common to many elements.


When the input file is read subcircuits definitions and models
are stored internally separately from the main circuit. Subcircuit calls
are expanded if the subcricuits referenced are already defined and stored
internally.  If a referenced subcircuit is not defined then
the subcircuit call is flagged as not being expanded and only when the input
file has been completely read (up to the {\tt .END} statement)
is an attempt made to resolve incomplete
expansions.  In \pspice\ library files (described in the {\tt .LIB}
statement discussion on page \pageref{.LIBstatement}) are checked.
The first time a library file is to be searched, an internal table of
which subcircuits and models are available and where they can be found in
the library files is constructed.
Models are treated in a similar way, a model is used if it has been defined
otherwise resolved references to models are expanded once the input file
has been completely scanned (up to the {\tt .END} statement).
With \pspice\ evaluation of expressions in subcircuits and models is only
performed when on fully expanded subcircuits. Note that expressions and
libraries are not supported in \spicetwo or in \spicethree .

\section{Input Lines}
The input file of \spice\ is essentially unstructured.
It must begin with a {\tt TITLE}\marginid{title}\index{title} line
and should end in a
{\tt .END} statement\marginid{.end}\index{.end} although this is
automaticlly assumed if the end of the input file is read.
The string on the {\tt TITLE} line is
used as the banner in the output log file appearing at
the top of each page.  The {\tt .END} statement marks the end of
one circuit with the effect that several circuits can be specified in
the file (at least for
\spicethree\ and \pspice ).
In between the {\tt TITLE} line and the {\tt .END}
statement can be any mix of statements --- which control the operation of the
simulator and the analysis to be performed; optional comment lines --- for
documenting the input file; and element lines --- which specify
the circuit elements.
The input file can in fact contain no statements and the simulator will
then perform an operating point {\tt .OP} analysis.
\spice\ does not distinguish between upper and lower case characters.

Except for the {\tt TITLE} line which is the first line of the input file,
the type of input lines is distinguished by the first character on the line:
statements begin with a period ``{\tt .}''; element lines begin with
an alphabetic character ``{\tt A}--{\tt Z}'' --- with the letter identifying the
element type (e.g. {\tt R} for a resistor); and comment lines begin with
an asterisk ``{\tt *}''. In older terminology, based on the original use of
punched cards, statements are referred to as ``{\it dot cards}'' or
``{\it statement cards}''; element lines are referred to as ``{\it element
cards}'' or ``{\it device
cards}''; and comment lines as ``{\it comment cards}.''

\pspice\ allows for in line comments indicated by a semicolon ``{\tt ;}''.
The semicolon and everything following it on the same line are ignored
except for purpose of echoing the input file in the output log file.


\subsection{Analysis Statements}
An analysis statement identifies the type of analysis to be performed.
Any combination of analyses may be specified.
Reporting of the results of an analysis is controlled by the
{\tt .PRINT}, {\tt .PLOT} {and, in with \pspice
by the} {\tt .PROBE} control statements.
If  no analysis statement is included in the input file then an operating
point analysis ({\tt .OP}) is performed by default.  A brief description of
the analysis options and the page on which a complete description can be
found as follows.

\myinsline{.AC}{AC Analysis}
Obtains the small-signal circuit response as a function of frequency.
The {\tt .AC} analysis is one of several small signal \ac\ analyses.

\myinsline{.DC}{DC Analysis}
DC solutions over a range of input conditions ({\tt .DC}).

\myinsline{.DISTO}{Small-Signal Distortion Analysis
\spicetwo\ and \spicethree\ Only}
Analyze the circuit for harmonic and intermodulation distortion.
This analysis is available in \spicetwo\ and \spicethree\ but
is not available in \pspice\ as it has proved to be unreliable.
In \spicetwo\ the distortion analysis must be performed in
conjunction with a {\tt .AC} analysis.

\myinsline{.FOUR}{Fourier Analysis}
The {\tt .FOUR} control statement can be used to find the spectrum
of any time-varying signal in a ({\tt .TRAN}) transient analysis.
The {\tt .FOUR} statement is unlike other analysis statements as it does not
initiate a simulation but interprets the result of a simulation initiated by
the {\tt .TRAN} sattement.

\myinsline{.MC}{Monte Carlo Analysis {(\pspice\ only)}}
The Monte Carlo analysis is a statistical analysis of the circuit causing
the circuit to be analyzed many times with a random change of model parameters
(parameters in a {\tt .MODEL} statement).  The analyses specified in
the {\tt .DC}, {\tt .AC} or {\tt .TRAN} statements can be simulated multiple
times.


\myinsline{.NOISE}{Small-Signal Noise Analysis}
Conduct a small-signal noise analysis as a function of frequency.
{In \pspice\ this statement must be used in conjunction with
a {\tt .AC} statement.}

\myinsline{.OP}{Operating Point Analysis}
DC solution for a particular input voltage/current condition.
This is the default analysis if no analysis is specified in the input
file. This is the default analysis if not analysis type is specified in the
input file.

\myinsline{.PZ}{Pole-Zero Analysis, \spicethree\ Only}
In this analysis the poles and zeros of the small signal \ac
transfer function of a two-port is evaluated.

\myinsline{.SAVEBIAS}{Save Bias Conditions}

\myinsline{.SENS}{Sensitivity Analysis}
The sensitivity of the DC value of an output to some set of
parametric variations is calculated.

\myinsline{.STEP}{Parameteric Analysis \pspiceninetytwo\ Only}

\myinsline{.TF}{Transfer Function Specification {\pspice\ Only}}
Small signal DC transfer functions, including gain and input and output
resistance are computed

\myinsline{.TRAN}{Transient Analysis}
In the transient analysis response is observed
with one or more time-varying inputs.

{\myinsline{.WCASE}{Sensistivity and Worst Case Analysis
\pspiceninetytwo\ Only}}

\subsection{Control Statements}

\myinsline{.DISTRIBUTION}{Distribution Specification
         {(\pspice\ only)}}
This statement specifies the statistical tolerance
distribution used in a Monte Carlo analysis (see the {\tt .MC}
statement on page \pageref{.MCstatement}).

\myinsline{.END}{End Statement}
This statement indicates the end of the input of one circuit.

\myinsline{.ENDS}{End Subcircuit Statement}
The end subcircuit statement indicates the end of a subcircuit definition.

{ \myinsline{.FUNC}{Function Definition \pspiceninetytwo Only}
Enables commonly used expressions to be more conveniently defined as functions.}

\myinsline{.IC}{Initial Conditions}
This statement is used to set initial conditions for transient
analysis. It has no effect on other types of analyses.

\myinsline{.INC}{Include Statement{, \pspice\ only}}
Specifies the name of a file which is to be treated as part of the input
file.

\myinsline{.LIB}{Library Statement{, \pspice\ only}}
Specifies the name of a library file.

\myinsline{.MODEL}{Model Statement}
Specifies the parameters of elements that either are
too numerous to put on the element line or are common to many elements.

\myinsline{.NODESET}{Node Voltage Initialization}
Specifies the voltage at one or more nodes to be used as an initial guess.

\myinsline{.OPTIONS}{Option Specification}
The options specification provides the user control over the program
and sets defaults for certain elements and analyses.

\myinsline{.PARAM}{Parameter Definition{, \pspiceninetytwo\ Only}}
This statement defines parameters that can be used in subsequent statements
and element lines.

\myinsline{.PLOT}{Plot Specification}
The plot specification controls the information that is plotted in
the output file as a character plot. This is one way to view the result
of various analyses.

\myinsline{.PRINT}{Print Specification}
The print specification controls the information that is reported as the result
of various analyses.

\myinsline{.PROBE}{Data Output Specification{, \pspice\ Only}}
This statement saves the node voltages and device currents in a
file for subsequent interactive plotting

\myinsline{.SUBCKT}{Subcircuit Statement}
Indicates the start of a subcircuit description and describes int interface
to the subcircuit.

\myinsline{.TEMP}{Temperature Specification}
Specifies the temperature(s) to perform the analysis at.

{\myinsline{.TEXT}{Text Parameter Definition,
\pspiceninetytwo\ Only}}

{\myinsline{.WATCH}{Watch Analysis Statement \pspiceninetytwo\ Only}}

\myinsline{.WIDTH}{Width Specification}
Specifies the column width for the output file.

\subsection{Elements}

The general form for elements is device name, followed by
a list of nodes, followed by the numeric value of the element,
followed in some cases by the name of a model, and then by other keywords:\\
\hspace*{\fill}\offsetparbox{{\it Name Node1 Node2 $\ldots$ NodeN
NumericValue ModelName\\
{\tt +} keyword=NumericValue $\ldots$ InitialConditions.}}\\
For some elements
initial conditions ({\it InitialConditions}) can also be specified
which can be used to ensure that the desired initial state of astable circuits
is obtained and also to aid in convergence. The first letter of the
{\it Name} identifies the element.  For example, if {\it Name} is
{\tt RTEST} then the element is a resistor.

The general form above is not the form for every element.
The way in which
\spice\ evolved resulted in the syntax for element lines not being
fully consistent. Commercial extensions, as with \pspice,
allowing alphabetic names for nodes rather than just an integer designation
also result in syntaxical problems.
With this extension it is not possible to
use the fact that a field was alphabetic to distinguish between a node
and a parameter name.
However this change has necessitated no change to the the standard syntax
as defined by \spicetwo .
The problem appears in conjunction with other
extensions which allow for an arbitrary
number of nodes in some statements. The result is that the syntax can be
slightly different than would be expected.
For these reasons the description of the form
of a particular element or statement must be consulted to ensure that the
syntax is correct.

\subsubsection{Passive Elements}

The passive devices supported in
{\spicetwo , \spicethree\ and \pspice}
and where their descriptions can be found are as follows:

\myineline{C}{Capacitor}
\myineline{K}{Mutual Inductor}
\myineline{L}{Inductor}
\myineline{R}{Resistor}
\myineline{S}{Voltage Controlled Switch}
\myineline{W}{Current Controlled Switch}

\subsubsection{Active Elements}

\form{ {\tt Q}name  NCollector NBase NEmitter  \B NSubstrate\E  ModelName
 \B Area\E  \B {\tt OFF}\E\\{\tt +} \B {\tt IC=}Vbe,Vce\E }

Unlike passive devices active
%, or semiconductor,
devices can not be specified by one or a few parameter values.
Since many of the parameter values are the same for many devices
it is convenient to specify them in a {\tt .MODEL} statement that
can be reused many times. All active elements require a {\tt
.MODEL} statement and most allow an optional substrate node to be
used on the element line.

The active devices supported in
{ \spicetwo , \spicethree\ and \pspice}
and where their descriptions can be found are as follows:

\myineline{B}{GaAs MESFET {(\pspice\ only)}}
\mycontentsline{0.7in}{\ }{(See Z element for \spicethree
equivalent)}{Z}{element}
\myineline{D}{Diode}
\myineline{J}{Junction Field-Effect Transistor}
\myineline{M}{MOSFET}
{\myineline{Z}{MESFET}

\mycontentsline{0.7in}{}{(See B element for \pspice\ equivalent)}{B}{element}}

\subsection{Distributed Elements}

The distributed devices
and where they are described are as follows:

The convolution element enables a linear circuit described by a set of
frequency dependent complex $y$ parameters to be included in transient
analysis.

\myineline{T}{Transmission Line}
\mycontentsline{0.7in}{U}{Lossy RC Transmission Line}{U2}{element}

\subsection{Source Elements}

The sources supported and where they can be found are as follows:

\myineline{E}{Voltage-Controlled Voltage Source}
\myineline{F}{Current-Controlled Current Source}
\myineline{G}{Voltage-Controlled Current Source}
\myineline{H}{Current-Controlled Voltage Source}
\myineline{I}{Independent Current Source}
\myineline{V}{Independent Voltage Source}

The control of the E, F, G and H elements can be control by a polynomial
function of voltage or current.

\subsection{Interface Elements}

The interface elements supported and where they can be found
are as follows:

\myineline{N}{Digital Input Interface{, \pspice\ only}}
Interfaces digital analog simulation by providing a means for a state
transistion to control an analog response.

\myineline{O}{Digital Output Interface{, \pspice\ only}}
Determines the equivalent digital state of an analog signal.

\myineline{P}{Port Element}
Element enabling the scattering parameters of a circuit to be directly
calculated. (Available in only a few versions of \justspice .


{\mycontentsline{0.7in}{U}{Digital Device}{U1}{element}}

\myineline{X}{Subcircuit Call}
Interfaces a circuit (or subcircuit) to a subcircuit.

\section{Input Grammar}

Each input line contains fields which are delimited (separated) by one of a
number of characters.  The most obvious delimiter is simply a space
but \marginid{Delimiter}\index{Delimiter}other characters are also treated
as ``{\it white space}'' characters.
\marginid{White Space}\index{White Space}White space is defined as one or any
combination of the following characters:\\[0.1in]
\offset ``blank'' \hspace{0.3in}
``tab'' \hspace{0.3in}
( \hspace{0.3in}
) \hspace{0.3in}
, \hspace{0.3in}
=\\
While the above characters appear in the input file they are ignored except
that they are treated as a field delimiter. The
``('', ``)'', ``='' and ``,'' characters are often used in the input file
and are included in specifiying the syntax for elements and statements but
they serve only to add visual structure to the input.

Continuation lines\marginid{Line}\marginid{Continuation}\index{line
continuation}\index{continuation line} begin with a plus ``{\tt +}'' in the
first column. For example\\[0.05in]
\hspace*{\fill}\offsetparbox{\tt
    R1 1 2 1000\\[0.05in]
    {\rm and}
    R1 1 2\\
    +1000}\\[0.05in]
are equivalent.

Comment lines begin with an asterisk ``{\tt *}'' in the first position.
In line comments are supported in \pspice\ and these begin with a semicolon
``{\tt ;}'' and must be contained wholly on one line. For example
\hspace*{\fill}\offsetparbox{
    * This just shows off a comment\\
    R1 1 2 1000  ;This shows of an in line comment}\\[0.05in]

In nearly every situation where a numeric value is required in the
input an algebraic expression can be used instead.
\marginid{Expression}\index{Expression}
Everything between a ``\{'' character and the matching ``\}'' character
is treated as an algebraic expression. The evaluation of algebraic
expressions are discussed in the ALgebraic Expressions section on
on page \pageref{section:algebraic:expression}.

\subsection{Prefixes and Units}

Almost-standard metric prefixes are used
in\index{Prefix}\index{Units}
\spice, the prefix abbreviation, the full metric name, and the represented
scale factors being as follows:
\begin{center}
  \begin{tabular}{|l|l|l|}
\hline
\multicolumn{1}{|c}{\bf \spice}&
\multicolumn{1}{|c}{\bf Metric}&
\multicolumn{1}{|c|}{\bf Scale}\\
\multicolumn{1}{|c}{\bf Prefix}&
\multicolumn{1}{|c}{\bf Name}&
\multicolumn{1}{|c|}{\bf Factor}\\
\hline
\hline
  F & femto & $10^{-15}$ \\
  P & pico  & $10^{-12}$ \\
  N & nano  & $10^{-9}$ \\
  U & micro & $10^{-6}$ \\
  M & milli & $10^{-3}$ \\
  K & kilo  & $10^{+3}$ \\
  MEG & mega  & $10^{+6}$ \\
  G & giga  & $10^{+9}$ \\
  T & tera  & $10^{+12}$ \\
\hline
  \end{tabular}
\end{center}
As \spice\ does not differentiate between upper and lower case, `{\tt MEG}'
(or `{\tt meg}')
is used for `mega' instead of the standard metric upper case `M'.

The value of an element is specified in terms of the conventionally
accepted units,
e.g. resistance in Ohms, capacitance in Farads, and inductance in Henries.
If you wish you can spell it out more fully, e.g.
\begin{center}
\begin{tabular}{ll}
{\tt  C1  0  2  5fF} & or \\
{\tt C1  0  2  5fFarad}  &or even \\
{\tt C1 0 2 5fthingies} &
\end{tabular}
\end{center}
The last alternative is allowed as
\spice\ actually ignores whatever follows the `{\tt f}' and assumes Farads.

\section{Parameters \label{section:parameters}}
Parameters can be defined in two ways:
\begin{itemize}
\item In a parameter definition ({\tt .PARAM}).
\item As a subcircuit parameter in a {\tt .SUBCKT} statement.
\end{itemize}
Parameters defined in a {\tt .PARAM} statement
can be used in subsequent statements
and element lines by replacing a numeric value by an expression in which the
parameter is used. The general form of the {\tt .PARAM} statement is\\

\hspace*{\fill}\offsetparbox{{\tt .PARAM}
\B ParameterName = NumericValue $\ldots$ \E\\
{\tt +}\B ParameterName = {\tt \{} Expression {\tt \}} $\ldots$ \E}\\
Here the {\it ParameterName} is the name of a parameter with the first character
being alphabetic (a-zA-Z) and can be assigned a numeric value
{\it NumericValue} which may be followed immediately by a spice scale
factor. For example, SMALL=1.E-9, SMALL=+1N, SMALL=1NV and SMALL=1.E-9V are
equivalent and all establish a parameter SMALL with a value of $10^{-9}$.
If {\it ParameterName} is the name of a
previously defined parameter at the same level of subcircut expansion
then the parameter value is changed.
If the {\tt .PARAM} statement is
is in the top level circuit then the parameter value is global and is
available any where in the netlist.
If the {\tt .PARAM} statement is in a subcircuit then the parameter value
is local and can be used at the current subcircuit expansion level or lower
in the subcircuit expansion hierarchy.
The same idea applies to values of a parameter
changed in a subcircuit.  Value changes are local and are available in the
current subcircuit and lower nested
subcircuits.  {Libraries are searched for parameters not
defined in the
      circuit NETLIST or in included files. A {\tt .PARAM} statement does
      not have to be within a subcircuit in a library.}

Instead of a numeric value an algebraic expression can be used to establish
the value of the parameter.
The expression is evaluated in the standard
way for an algebraic expression replacing numeric values and is evaluated at
the time of expansion rather than as the netlist is read.
  This ensures the correct
hierarchical interpretation of the netlist.  The treatment of expression is
discussed in section \ref{section:expression} on page
\pageref{section:expression}.  Note that as always the expression must be
enclosed in matching braces ({\tt \{} $\ldots$ {\tt \}}).

Parameters can be used nearly anywhere a numeric value is expected
by including them in an expression evaluation even if the expression
contains a single parameter. For example
\hspace*{\fill}\offsetparbox{{\tt .PARAM rbig=10MEG\\ R1 1 2 \{RBIG\}}}\\
establishes a resistance {\tt R1} between nodes 1 and 2 with a value of
$10^6~\Omega$.

Several predefined parameters are supported and the user must avoid
defining these as unpredictable results may result.
The predefined parameters are

{\offset\begin{tabular}{|l|l|l|}
\hline
{\bf Name} & {\bf Value}   &  {\bf Description}  \\
\hline
\hline
TEMP       & not supported & Analysis temperature.\\
           & Reserved for future expansion&\\
       & & \\
VT         & not supported & Thermal voltage.\\
           & Reserved for future expansion&\\
\hline
\end{tabular}
}


\section{Expressions \label{section:expression}}

{In \pspice\ most places where a numeric value is normally used
an expression (within braces {\tt \{ $\ldots$ \})} can be used instead.
An expression can contain any supported mathematical operation, constant numeric
values or expressions. Exceptions are 
     \begin{itemize}
     \item Polynomial coefficients.
     \item The values of the transmission line device parameters {\tt NL} and
           {\tt F}.
     \item The values of the piece-wise linear characteristic in the {\tt PWL}
           form of the independent voltage ({\tt V}) and current ({\tt I})
           sources.
     \item The values of the resistor device parameter {\tt TC}.
     \item As node numbers.
     \item[and]
     \item Values of most statements (such as .TEMP, .AC, .TRAN etc.)
     \end{itemize}
Specifically included are
     \begin{itemize}
     \item The values of all other device parameters.
     \item The values in {\tt .IC} and {\tt .NODESET} statements.
     \item The values in {\tt .SUBCKT} statements.
     \item[and]
     \item The values of all model parameters.
           {\tt F}.
     \end{itemize}

Operators that can be used in expressions are listed in Table
\ref{table:expression:operators}.
\clearpage
\begin{table}
\caption{Expression operators.\label{table:expression:operators}}
\begin{center}
\begin{tabular}{|l|l|l|}
\hline
{\bf Operator }	& {\bf   Syntax } & {\bf Description}  \\
\hline
PLUS		& {\it x}{\tt +}{\it y}		& plus \\
MINUS		& {\it x}{\tt -}{\it y}		& minus \\
UNARY\_PLUS	& {\tt +}{\it x}            & unary plus \\
UNARY\_MINUS	& {\tt -}{\it x}            & unary minus \\
MULTIPLY	& {\it x}{\tt *}{\it y}		& multiply \\
DIVIDE		& {\it y}{\tt /}{\it x}	        & divide \\
POW		& {\it x}{\tt \^{ }}{\it y}  or
                  {\it x}{\tt **}{\it y} & raise to a power, $x^y$ \\
AND		& {\it x}{\tt \&}{\it y}           & AND \\
OR		& {\it x}{\tt |}{\it y}           & OR \\
NOT		& {\tt !}{\it x}		& NOT \\
XOR		& {\it x}{\tt ~}{\it y}		& XOR (exclusive or) \\
SIN		& {\tt sin(}{\it x}{\tt )}	& sine, argument in radians \\
COS		& {\tt cos(}{\it x}{\tt )}	& cosine, argument in radians \\
TAN		& {\tt tan(}{\it x}{\tt )}& tangent, argument in radians \\
ASIN		& {\tt asin(}{\it x}{\tt )}& arcsine, argument in radians \\
ACOS		& {\tt acos(}{\it x}{\tt )}& arccosine, argument in radians \\
ATAN		& {\tt atan(}{\it x}{\tt )}& arctangent, argument in radians \\
SINH		& {\tt sinh(}{\it x}{\tt )}	& hyperbolic sine \\
COSH		& {\tt cosh(}{\it x}{\tt )}	& hyperbolic cosine \\
TANH		& {\tt tanh(}{\it x}{\tt )}	& hyperbolic tangent \\
EXP		& {\tt exp(}{\it x}{\tt )}	& exponentiation, $e^x$ \\
ASINH		& {\tt asinh(}{\it x}{\tt )}& arc-hyberbolic sine \\
ACOSH		& {\tt acosh(}{\it x}{\tt )}& arc-hyberbolic cosine \\
ATANH		& {\tt atanh(}{\it x}{\tt )}& arc-hyberbolic tangent \\
ABS		& {\tt abs(}{\it x}{\tt )}	& absolute, $|x|$ \\
SQRT		& {\tt sqrt(}{\it x}{\tt )}	& square root, $\sqrt{x}$ \\
\hline
\end{tabular}
\end{center}
\end{table}
}

\subsection{Polynomials \label{section:poly}}
\index{polynomial}\index{POLY}

Polynomial expressions can be used with the controlled source
elements ({\tt E}, {\tt F}, {\tt G} and {\tt H}) to realize
nonlinear controlled sources. The specification of the polynomial
must be at the end of the input line and has two forms. The
polynomial format for a voltage-controlled current source (the
{\tt G} element)
or a voltage-controlled voltage source (the {\tt E} element) is\\[0.1in]
\hspace*{\fill}\offsetparbox{
{\tt POLY({\it N}) ($N_{C1+}$,{\it $N_{C1-}$})
$\ldots$ ($N_{CN+}$, $N_{CN-}$)
$C_0$ $C_1$ $C_2$ $C_3$ $\ldots$
}}
where
\begin{widelist}
\item[{\tt POLY}] is the keyword indicating that a polynomial description
follows.
\item[{\it N}] is the degree of the polynomial.
\item[$N_{C1+}$, $N_{C1-}$]The voltage at the node
$N_{C1+}$ with respect to the voltage at the node
$N_{C1-}$ is the controlling voltage $V_1$.
\item[$N_{CN+}$, $N_{CN-}$]The voltage at the node
$N_{CN+}$ with respect to the voltage at the node
$N_{CN-}$ is the controlling voltage $V_N$.
\item[$C_0$ $C_1$ $\ldots$] are the polynomial coefficients. Not all of the
coefficients need be specified as the trailing coefficients that are not
specified are treated as if they are zero.
\end{widelist}
Note that in spice parentheses, ``('' and ``)'', and commas, ``,'', are treated
as if they are spaces.  The use of
parentheses and commas serves only to make the netlist more easily read.
{\bf The exception to this is their use in expressions (see section
\ref{section:expression}).}

For voltage-controlled elements the output is calculated as
\begin{eqnarray}
{\rm OUTPUT} & = & C_0 \nonumber\\
         &   & + C_1V_1 + \ldots + C_NV_N\nonumber\\
         &   & + C_{N+1}V_1V_1 + C_{N+2}V_1V_2 + \ldots + C_{N+N}V_1V_N
               \nonumber\\
         &   & + C_{2N+1}V_2V_2 + C_{2N+2}V_2V_3 +\ldots + C_{2N+N-1}V_2V_N
               \nonumber\\
         &   & \vdots \nonumber\\
         &   & + C_{N!/(2(N-2)!)+2N}V_NV_N
               \nonumber\\
         &   & + C_{N!/(2(N-2)!)+2N+1}V_1V_1V_1 +
               C_{N!/(2(N-2)!)+2N+2}V_1V_1V_2\nonumber\\
         &   &\ \ \ \ \ \ \ \ \ + \ldots +
               C_{N!/(2(N-2)!)+2N+N-1}V_1V_1V_N
                     \nonumber\\
         &   & + C_{N!/(2(N-2)!)+3N}V_1V_2V_2 + \ldots +
               C_{N!/(2(N-2)!)+3N+N-2}V_1V_2V_N
                     \nonumber\\
         &   & \vdots\nonumber
\end{eqnarray}
A one dimensional polynomial (with only one pair of controlling nodes) is
evaluated as
\[
{\rm OUTPUT} = C_0 + C_1V_1 + C_2V_1^2 + C_3V_1^3 + \ldots C_NV_1^N
\]
An example of a voltage-controlled voltage source is\\[0.1in]
\hspace*{\fill}\offsetparbox{{\tt E1 2 3 POLY(2) (10,0) (12,2) 0.5 1 1 0.2 0.3
   0.2}} \\[0.1in]
and of a voltage-controlled current source is\\[0.1in]
\hspace*{\fill}\offsetparbox{{\tt G1 2 3 POLY(4) (10,0) (12,2) (11,0) (13,0)
   0.5 1 1 1 1 0.2 0.3 0.2}}\\[0.1in]

The format for a current-controlled current source (the {\tt F} element)
or a current-controlled voltage source (the {\tt H} element) is\\[0.1in]
\hspace*{\fill}\offsetparbox{
{\tt POLY({\it N}) {\it VoltageSourceName$_1$}
$\ldots$ {\it VoltageSourceName$_N$}
$C_0$ $C_1$ $C_2$ $C_3$ $\ldots$
}}
where
\begin{widelist}
\item[{\tt POLY}] is the keyword indicating that that a polynomial description
follows.
\item[{\it N}] is the degree of the polynomial.
\item[{\it VoltageSourceName$_1$}] is the name of the voltage source the
current through which is control current $I_1$.
\item[{\it VoltageSourceName$_N$}] is the name of the voltage source the
current through which is control current $I_N$.
\item[$C_0$ $C_1$ $\ldots$] are the polynomial coefficients.
\end{widelist}
For these elements the output is calculated as
\begin{eqnarray}
{\rm OUTPUT} & = & C_0 \nonumber\\
         &   & + C_1V_1 + \ldots + C_NV_N\nonumber\\
         &   & + C_{N+1}V_1V_1 + C_{N+2}V_1V_2 + \ldots + C_{N+N}V_1V_N
             \nonumber\\
         &   & + C_{2N+1}V_2V_2 + C_{2N+2}V_2V_3 +\ldots + C_{2N+N-1}V_2V_N
             \nonumber\\
         &   & \vdots
             \nonumber\\
         &   & + C_{N!/(2(N-2)!)+2N}V_NV_N
             \nonumber\\
         &   & + C_{N!/(2(N-2)!)+2N+1}V_1V_1V_1 +
               C_{N!/(2(N-2)!)+2N+2}V_1V_1V_2\\
         &   & \ \ \ \ \ \ \ + \ldots +
               C_{N!/(2(N-2)!)+2N+N-1}V_1V_1V_N
                   \nonumber\\
         &   & + C_{N!/(2(N-2)!)+3N}V_1V_2V_2 + \ldots +
               C_{N!/(2(N-2)!)+3N+N-2}V_1V_2V_N
                   \nonumber\\
         &   & \vdots
\end{eqnarray}
An example of a current-controlled voltage source is:\\[0.1in]
\hspace*{\fill}\offsetparbox{H1 2 3 POLY(2) VIN V2 0.5 1 1 0.2 0.3 0.2}
  \\[0.1in]
and of a current-controlled current source is:\\[0.1in]
\hspace*{\fill}\offsetparbox{F1 2 3 POLY(4) VIN V2 (11,0) (13,0) 0.5 1 1
1 1 0.2 0.3 0.2}

{\subsection{Laplace Expressions \label{section:laplaceexpressions}}
\index{laplace expressions}}
{\subsection{Chebyschev \label{section:chebyschev}}
\index{chebyschev expressions}}
\section{\label{section:function}
Function Definition .FUNC \pspiceninetytwo\ Only}
\index{function}\index{.FUNC}\index{FUNC}

The {\tt .FUNC} statement can be used to conveniently define
commonly used expressions.

\hspace*{\fill}\offsetparbox{{\tt .FUNC}
FunctionName{\tt (}\B Argument1, Argument2, $\ldots$ Argument10\E\E\E{\tt )}
= FunctionDeclaration
}

{\it FunctionName} is the name of the function being defined.  It
must begin with an alphabetic character (A-Z).

{\it Argument1} is a function argument.
There can be from 0 to 10 arguments.

{\it FunctionDeclaration} can be any regular algebraic expression (see section
\ref{section:algebraic:expression} on page \pageref{section:algebraic:expression})
and can use previously defined functions and the Laplace variable $s$.
The expression delimiters
{\tt \{} and {\tt \}} need not be used.
The {\it FunctionDeclaration} is automaticly enclosed within the expression
delimiters {\tt \{} and {\tt \}}. The function declaration plus the
two delimiters must be no more than 80 characters (one line) long.


The names of predefined functions msut be avoided.  The predefined
functions are listed in section
\ref{section:algebraic:expression} on page \pageref{section:algebraic:expression}.

Functions are treated as macros in the C programming language.
when user defined functions are invoked a textual expansion is
performed and the resultant expansion is evaluated as a regular
expression.  The FunctionDeclaration before and after expansion is
enclosed within expression delimiters {\tt \{} and {\tt \}}.  This
defines how nested functions are treated.

It is faster to use predefined functions if available. Predefined
functions also test the validity of the arguments and evaluate the correct
asymptotic behavior.

\section{Syntax Variations}
\index{Differences between versions, see Versions}
\index{Versions, Syntax}
Commercial versions have enhanced the syntax of Berkeley version of
\justspice .  In virtually all cases the syntax of \spicetwo\ and
\spicethree is a subset of the syntax of commercial versions of \justspice .
Here we list some exceptions.
\begin{itemize}
\item Units.  Spice does not allow units immediately following a quantity.
      For example, the following is acceptable in all versions.\\
      \indent {\tt VIN 1 0 DC 4}\\
      \indent {\tt VIN 1 0 DC 4UV}\\
      For example, the following is not acceptable in all \spicetwo\ and
      \spicethree\\
      \indent {\tt VIN 1 0 DC 4V}\\
      but is acceptable in \hspice , \pspice  and \sspice.
\end{itemize}