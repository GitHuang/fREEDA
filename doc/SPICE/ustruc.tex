%
% .MODEL STRUC
%
\eskipv{U\ STRUC}
\modelx{STRUC}
{\sspice\ Only}
{Structure of Multiple Coupled Line Element Model}

\begin{figure}[h]
\centering
\ \epsfxsize=2in\pfig{uplines.ps}\\
(a)\\
\ \epsfxsize=5.5in\pfig{mstruc.ps}\\
(b)\\
\vspace*{2in}
\caption{U element, STRUC model --- multiple lossy coupled line element:
(a) schematic; and (b) cross-section}
\end{figure}

\begin{enumerate}
\item
\form{ {\tt U}name  Nin1 \B Nin2 ... Ninn \E Nout1 \B Nout2 ... Noutn \E
       ModelName {\tt L} =  value }
\example{
    {Uoneline  1 2 ONELINEMODEL L = 10.0 } \\
    {.MODEL ONELINEMODEL STRUC }
        }
\item
\form{ {\tt U}name  Nin1 \B Nin2 ... Ninn \E Nout1 \B Nout2 ... Noutn \E
       ModelName {\tt PHI} =  value }
\example{
    {Uthreeline  1 2 3 4 5 6 THREELINEMODEL PHI = 90 } \\
    {.MODEL THREELINEMODEL STRUC }
        }
\end{enumerate}

\modelform{ {\tt .MODEL} ModelName {\tt STRUC(}
             \B  \B keyword = value\E  ... \E {\tt )}}

\begin{widelist}
\item[{\it Nin1}] is the first input node.
\item[{\it Nin2}] is the second input node.
\item[{\it Ninn}] is the {\it n}th input node of the {\it n} coupled
                  transmission line system.
\item[{\it Nout1}] is the first output node.
\item[{\it Nout2}] is the second output node.
\item[{\it Noutn}] is the {\it n}th output node of the {\it n} coupled
                  transmission line system.
\item[{\tt L}]	 is the electrical length.
                 (Units: m; Either {\tt L} or {\tt PHI} is required;
                 Symbol: $L$;
\item[{\tt PHI}] is the electrical length as an angle .
                (Units: degrees; Either {\tt L} or {\tt PHI} is required;
                Symbol: $\phi$;
\end{widelist}

\keyword{
{\tt EPS1,\newline EPS2, \newline ...,\newline EPS5}
	& Permittivity of dielectric layers	& -	&${\tt EPS}_i$=1 \X
{\tt H1, H2, \newline ..., H5}
	&{Height of dielectric layers	 \newline
	\hspace*{\fill}{\tt H1 = 0} means {\tt H1} $\Rightarrow \infty$ \newline
	\hspace*{\fill}{\tt H2 = 0} means {\tt H2} $\Rightarrow \infty$ }
	&  m	&${\tt H}_i$=0 \X
{\tt XVAS\newline XVBS\newline XVCS\newline XVDS}
	&{Left origin of line geometry in plane A,B,C,D \newline
	  {\tt XV\#S}=0 means: \newline
	  {\tt XV\#S} = {\tt A}-$\sum ({\tt W}\#_i + {\tt S}\#_i$)/2 \newline
	  i.e. a symmetrical arrangement}
	& m
	& {\tt XVAS}=0\newline {\tt XVBS}=0\newline {\tt XVDS}=0 \X
{\tt WA}
	& ${\tt WA}_i$ Line and gap widths for plane A		& m&
        ${\tt WA}_i$=0
	\X
{\tt SA}
	& ${\tt SA}_i$, gap widths for plane A 		        & m&$SA_i$=0
	\X
{\tt WB}
	& ${\tt WB}_i$, Line and gap widths for plane B		& m&
        ${\tt WB}_i$=0
	 \X
{\tt SB}
	& ${\tt SB}_i$, gap widths for plane B 		        & m&
        ${\tt SB}_i$=0
	\X
{\tt SC}
	& tuning septum gap. SC=0 \&h4=0 indicate no layer 4 & m & SC=0 \X
{\tt WD}
	& ${\tt WD}_i$, Line and gap widths for plane D		& m&
        ${\tt WD}_i$=0
	\X
{\tt SD}
	& ${\tt WA}_i$, gap widths for plane D 		        & m&
        ${\tt SD}_i$=0
	\X
{\tt IDICK} 	& Line thickness flag. {\tt IDICK}=1
          indicates finite metallization\newline
	  thickness t = {\tt H3}+{\tt H5}& - &1\X
}

\keywordtable{
\multicolumn{4}{|c|}{OHMIC LOSS PARAMETERS} \X
{\tt RFA}	& Ohmic conductor loss of metal lines in plane A normalized by
	  the value of copper
	& -	& 0	\X
{\tt RFB}	& Ohmic conductor loss of metal lines in plane B normalized by
	  the value of copper
	& -	& 0	\X
{\tt RFD}	& Ohmic conductor loss of metal lines in plane D normalized by
	  the value of copper
	& -	& 0	\X
{\tt HMET}	& Thickness of metallization (see note on HMET)
        & m & 0.005\X
\multicolumn{4}{|c|}{PARAMETERS FOR THE ADVANCED USER} \X
{\tt A}	& Breadth of housing multiple. Actual housing breadth =
	  {\tt A}$\cdot${\tt H1}
	& - & 10 \X
{\tt LAT}
	 & Lateral boundary condition\newline
	  {\tt LAT}=0 $\Rightarrow$ electrical-wall/electrical-wall\newline
	  {\tt LAT}=1 $\Rightarrow$ magnetic-wall/electrical-wall\newline
	  {\tt LAT}=2 $\Rightarrow$ electrical-wall/magnetic-wall\newline
	  {\tt LAT}=3 $\Rightarrow$ magnetic-wall/magnetic-wall\newline
	  {\tt LAT}=5 $\Rightarrow$ circular lines	&- & 0\\
	&						&&	\X
{\tt KAL}	& Density of the lines of discretization\newline
	See note on KAL for default.
	& - &\X
     }

\note{\item[{\tt KAL}] {\tt KAL} is defaulted so that on each strip there is
     a minimum of
     3 lines of discretization and in each slot is a minimum of two lines of
     discretization. In the case of nonequidistant lines (indicated by
     {\tt IPOL}=2)
     {\tt KAL} is the minimum number of lines on each strip.}


\begin{figure}[b]
\hspace*{\fill}\ \epsfxsize=2in\pfig{uequi.ps}
\hfill\hfill
\ \epsfxsize=2in\pfig{unequi.ps}\hspace*{\fill}\\
\hspace*{\fill} (a) \hfill \hfill (b) \hspace*{\fill}\\
\caption[Arrangement of lines of discretization used in the method of
lines]{Arrangement of lines of discretization for use in the method of lines:
(a) equidistant lines indicated by {\tt IPOL}=0 or {\tt IPOL} =1;
(b) nonequidistant method of lines indicated by {\tt IPOL}=2.}
\end{figure}

\note{\item[{\tt HMET}] {\tt HMET} is only necessary if {\tt RFA},
 {\tt RFB} or {\tt RFC} is specified.
The default value of {\tt HMET} is 20 $\mu$m. If {\tt IDICK}=0 then the
default value of
{\tt HMET} is H3 for plane A. 
In this case {\tt RFB} (and may be {\tt RFD}) must not be specified.}

\clearpage
\example{Microstrip:}
\begin{figure}[h]
\centering
         \ \epsfysize=2in\pfig{mstrucms.ps}
\caption{Microstrip example.}
\end{figure}

\clearpage
\example{Buried Microstrip:}
\begin{figure}[h]
\centering
         \ \epsfysize=2.5in\pfig{mstrucbm.ps}
\caption{Buried microstrip example.}
\end{figure}

\clearpage
\example{Stripline:}
\begin{figure}[h]
\centering
         \ \epsfysize=2.5in\pfig{mstrucst.ps}
\caption{Stripline example.}
\end{figure}
