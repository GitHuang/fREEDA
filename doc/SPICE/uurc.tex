\eskipv{U\ URC}
%
% URC Model
%
\modelx{URC}{\spicethree\ Only}{Lossy RC Transmission Line Model}

\vspace*{\fill}
\begin{figure}[h]
\centering
\ \epsfxsize=2in\pfig{uurc.ps}
\caption{U element, URC model --- RC
transmission line element.}
\end{figure}

\elementform{ {\tt U}name  N1 N2 N3  ModelName {\tt L} =  value
                    \B {\tt N} =  value\E }

\modelform{ {\tt .MODEL} ModelName {\tt URC(} \B  \B keyword = value\E  ... \E
{\tt )}}

\begin{widelist}
\item[{\it N1}] is the first node of the transmission line.
\item[{\it N2}] is the second node of the transmission line..
\item[{\it N3}] is the reference node of the transmission line..
\item[{\tt L}]	is the electrical length
                (Units: m; Required; Symbol: $L$;
\item[{\tt N}]	Number of RC lumped element segments used in modeling the line.
                (Units: m; Optional --- Infereed if not specified.
                See note 3; Symbol: $N$;
\end{widelist}



\example{
    Urcline  1 0 2 0 RCLINE N = 4 L=0.10\\
    .MODEL RCLINE URC(K=1 FMAX=1E9 CPERL=1E-12) }


\modelkeyword{
{\tt K}
      & Propagation constant               & -     & 2.0     \X
{\tt FMAX}
   & Maximum frequency of interest      & Hz    & 1.0G    \X
{\tt RPERL}
  & Resistance per unit length         & $\Omega$/m & 1000    \X
{\tt CPERL}
  & Capacitance per unit length        & F/m   & 1.0E-15 \X
{\tt ISPERL}
 & Saturation current per unit length & A/m & 0   \X
{\tt RSPERL}
 & Diode resistance per unit length   & $\Omega$/m & 0    \X
   }


\note{
\item[1]
The URC element models a transmission line as a number of cascaded resistor
and capacitor segments.
The URC model is derived from a model  proposed  by  L.
Gertzberrg  in 1974.
The segments model a section of the transmission line
and the length of each is in a geometric
progression increasing toward  the
middle  of  the  URC  line with {\tt K} as a proportionality constant.
The number of segments is given by the parameter {\tt N} which defaults to
\item[2]
The number of segments, {\tt N}, defaults to
\begin{equation}
N=
\end{equation}
\item[3]
The URC line will be made up strictly of  resistor  and
capacitor  segments  unless  the ISPERL parameter is given a
non-zero value,
If the
{\tt ISPERL} parameter is non-zero then
distributed diode loading on the line is indicated.
If {\tt ISPERL} is non-zero 
the  capacitors  are  replaced
with reverse biased diodes.  The zero-bias junction capacitance of these diodes
is equivalent to the capacitance  replaced.
The saturation  current of the diodes is {\tt ISPERL} multiplied by the length
of the transmission line segment.
The diodes have series resistance equal to  {\tt RSPERL}
multiplied by the length of the transmission line segment.}
