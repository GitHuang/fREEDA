\element{Z}{MESFET}
\label{ZSPICE3element}
\begin{figure}[h]
\centering
\ \pfig{zspice3.ps}
\caption{Z --- GASFET element.
\label{zspice3.ps}}
\end{figure}


\form{{\tt Z}name NDrain NGate NSource ModelName
\B AREA\E  \B OFF\E  \B IC=VDS,VGS\E }

\example{Z1 7 2 3 ZM1 OFF}

\begin{widelist}
\item[{\it NDrain}] is the drain node.
\item[{\it NGate}] is the gate node.
\item[{\it NSource}] is the source node.
\item[{\tt ModelName}]  is  the  model name.
\item[{\tt OFF}] indicates an (optional) initial condition on the device for
\dc\ analysis.  If specified the \dc\ operating point is calculated with the
terminal voltages set to zero.  Once convergence is obtained, the
program continues to iterate to obtain the exact  value of
the  terminal  voltages.  The OFF option is used to enforce the solution
to  correspond  to  a  desired  state if the circuit has more than one stable
state.
\item[{\tt IC}] is the optional  initial condition specification.
Using {\tt IC=}$V_{DS},V_{GS}, V_{BS}$
is intended for use with the {\tt UIC} option
on  the  {\tt .TRAN}  line,  when  a transient analysis is desired
starting from other than the quiescent operating point.
Specification of the transient initial conditions using the {\tt .IC}
statement (see page \pageref{.ICstatement}) is preferred and is more
convenient.
\end{widelist}

\modeltype{GASFET}

%
% GASFET
%
\modelx{GASFET}{}{GaAs MESFET Model}\label{ZGASFETmodel}
\label{GASFETmodelsp3}
\begin{figure}[h]
\centering
\ \epsfxsize=2.75in\pfig{sp3gaas.ps}
\caption[Schematic of the \spicethree\ GASFET model]{Schematic of the \spicethree
GASFET model. \label{zspice3gasfet} $V_{GS}$, $V_{DS}$, and $V_{GD}$ are intrinsic
gate-source, drain-source and gate-drain voltages
between the internal gate, drain, and source terminals designated
$G$, $D$, and $S$ respectively.  }
\end{figure}
\vfill
\form{{\tt .MODEL} ModelName {\tt GASFET(} \B\B keyword {\tt =} value\E  ... \E
{\tt )}}

\example{.MODEL GAAS12 GASFET() }

Raytheon model\newline
This model is also known as the Statz model and
model was developed at Raytheon for the modeling of GaAs
MESFETs used in digital circuits.  It is based on empirical
fits to measured data \cite{statz:87}.

The parameters of the {\tt GASFET} model for \pspice are given in table
\ref{zsp3table}.

It is assumed that the model parameters were determined or
measured at the nominal temperature $T_{\ms{NOM}}$ (default
$27^{\circ}$C) specified in the most recent {\tt .OPTIONS} statement
preceeding the {\tt .MODEL} statement.
\begin{table}[h]
\caption{\spicethree GASFET model keywords. \label{zsp3table}}
\keywordtwo{Area}{
\raggedright
{\tt VTO}   &pinch-off voltage $V_{T0}$\sym{T_{C,VT0}}& V      &-2.0   &\X
{\tt BETA}
  &transconductance parameter \sym{\beta}       & $\mbox{A/V}^2$&1.0E-4 &  \STAR\X
{\tt B}
     &\raggedright doping tail extending parameter\sym{B}
     & 1/V    &0.3    &  \STAR\X
{\tt ALPHA}
 &saturation voltage parameter\sym{alpha}& 1/V    &2      &  \STAR\X
{\tt LAMBDA} &\raggedright channel length modulation parameter
    \sym{\lambda}
    & 1/V    &0      &   \X
{\tt RD}
    &drain ohmic resistance \sym{R_D}   & $\Omega$&0      &  \STAR\X
{\tt RS}
    &source ohmic resistance \sym{R_S}  & $\Omega$&0      &  \STAR\X
{\tt CGS}
   &\raggedright zero-bias G-S junction capacitance  \sym{C'_{GS}}
   & F      &0      &  \STAR\X
{\tt CGD}
   &\raggedright zero-bias G-D junction capacitance \sym{C'_{GD}}
   & F      &0      &  \STAR\X
{\tt PB}    &gate junction potential  \sym{V_{\ms{BI}}}   & V      &1      &\X
{\tt KF}    &flicker noise coefficient\sym{K_F} & -      &0      &\X
{\tt AF}    &flicker noise exponent \sym{A_F}   & -      &1      &\X
{\tt FC}    &coefficient for forward-bias depletion capacitance formula
      &  -   &    0.5      &\X
}
\end{table}
The physical constants used in the model evaluation are
\begin{center}
\begin{tabular}{|l|l|l|}
\hline
$k$ & Boltzman's constant           &  $1.3806226\,10^{-23}$~J/K\\
$q$ & electronic charge             & $1.6021918\,10^{-19}$~C\\
\hline
\end{tabular}
\end{center}
\noindent\underline{\sl \large Standard Calculations}\\[0.1in]
Absolute temperatures (in kelvins, K) are used.
The thermal voltage
\begin{equation}
V_{\ms{TH}} = {{kT_{\ms{NOM}}} \over q} .
\end{equation}

\noindent and the band gap energy at the nominal temperature is
\begin{equation}
E_G(T_{\ms{NOM}})=E_G(0)-0.000702{{4T_{\ms{NOM}}^2}\over{T_{\ms{NOM}}+1108}}.
\end{equation}
Here $E_G(0)$ is the parameter {\tt EG} --- the band gap energy at 0~K.\\[0.2in]
\noindent\underline{\sl \large Temperature Dependence}
\index{GASFET, Temperature Dependence}
\index{Temperature Dependence, see GASFET}

Temperature effects are incorporated as follows where $T$ and $T_{\ms{NOM}}$
are absolute temperatures in Kelvins (K).
\begin{eqnarray}
\beta(T) & = & \beta 1.01^{\textstyle(T_{C,\beta}(T-T_{\ms{NOM}}}\\
I_S (T) & = & I_S e^{\left( \textstyle E_g(T) {T \over {T_{\ms{NOM}}}}
 - E_G(T) \right) /(nV_{\ms{TH}})}
    \left({{\textstyle T}\over{\textstyle T_{\ms{NOM}}}}
    \right)^{(\textstyle X_{TI}/n)}\\
C'_{GS} (T)&=&C_{GS}\left\{1 + M \left[0.0004(T-T_{\ms{NOM}})+\left(1-
   {{\textstyle V_{BI} (T)} \over {V_{BI}}}\right) \right]\right\} \\
C'_{GD} (T)&=&C_{GD}\left\{1 + M \left[0.0004(T-T_{\ms{NOM}})+\left(1-
   {{\textstyle V_{BI} (T)} \over {V_{BI}}}\right) \right]\right\} \\
E_G(T) & = & E_G(0) - 0.000702{{4T_{\ms{NOM}}^2} \over {T_{\ms{NOM}}+1108}}\\
V_{BI} (T) & = & V_{BI} {T \over {T_{\ms{NOM}}}}
 - 3V_{\ms{TH}} \ln{\left( {{T} \over {T_{\ms{NOM}}}}\right) }
              + E_G (T_{\ms{NOM}}) {T \over {T_{\ms{NOM}}}} -E_G(T)\\
V_{T0}(T)&=&V_{T0} + T_{C,VT0}(T- T_{NOM})\\
V_{\ms{TH}} & = & {{kT} \over q}
\end{eqnarray}
\noindent\underline{\sl \large Parasitic Resistances}\\[0.1in]
\index{Parasitic Resistances, see GASFET, \pspice}
\index{GASFET, \pspice\ Parasitic Resistance}
\index{GASFET, \pspice\ $R_S$}
\index{GASFET, \pspice\ $R_G$}
\index{GASFET, \pspice\ $R_D$}
\index{MESFET, see GASFET (\pspice)}

The resistive parasitics
$R_S$, $R_G$ and $R_D$ are
are calculated from the sheet resistivities
{\tt RS} (= $R'_S$), {\tt RG} (= $R'_G$) and {\tt RD} (= $R'_D$), and the
$Area$ specified on the element line.
\begin{eqnarray}
R_D & = & R'_D Area\\
R_G & = & R'_G Area\\
R_S & = & R'_S Area
\end{eqnarray}
\notforsspice{The parasitic resistance parameter dependencies are summarized in
figure \ref{zsp3para}.
%\\[0.2in]

\begin{figure}[h]
\parbox[t]{1.3in}{
\begin{tabular}[t]{|p{1in}|}
\hline
\multicolumn{1}{|c|}{PROCESS} \\
\multicolumn{1}{|c|}{PARAMETERS} \\
\hline
\hline
{\tt RD} \hfill $R'_D$\\
{\tt RG} \hfill $R'_G$\\
{\tt RS} \hfill $R'_S$\\
\hline
\end{tabular}
}
\hfill
\parbox{0.2in}{\ \vspace*{0.2in}\newline +}
\hfill
\begin{tabular}[t]{|p{1in}|}
\hline
\multicolumn{1}{|c|}{GEOMETRY} \\
\multicolumn{1}{|c|}{PARAMETERS} \\
\hline
$Area$\\
\hline
\end{tabular}
\hfill
\parbox{0.2in}{\ \vspace*{0.2in}\newline $\rightarrow$}
\hfill
\begin{tabular}[t]{|p{1.8in}|}
\hline
\multicolumn{1}{|c|}{DEVICE}\\
\multicolumn{1}{|c|}{PARAMETERS}\\
\hline
\hspace*{\fill} $R_D = f(Area, R'D)$\\
\hspace*{\fill} $R_G = f(Area, R'G)$\\
\hspace*{\fill} $R_S = f(Area, R'S)$\\
\hline
\end{tabular}
\caption{MOSFET parasitic resistance parameter
relationships. \label{zsp3para}}
\end{figure}
}

\noindent\underline{\bf Leakage Currents}\\[0.1in]

\index{Leakage Currents, see GASFET}
\index{GASFET, I/V Characteristics, Leakage Currents}
\index{GASFET, Leakage Currents}
Current flows across the normally reverse biased gate-source and gate-drain
junctions.
The gate-source leakage current
\begin{equation}
I_{GS} = Area\,I_{S}e^{(\textstyle V_{GS}/V_{\ms{TH}} -1)}
\end{equation}
and the gate-drain leakage current
\begin{equation}
I_{GD} = Area\,I_{S}e^{(\textstyle V_{GD}/V_{\ms{TH}} -1)}
\end{equation}
\notforsspice{The dependencies of the parameters describing the leakage current
are summarized in figure \ref{zsp3leakage}.\\[0.2in]
\begin{figure}[b]
\begin{tabular}[t]{|p{1in}|}
\hline
\multicolumn{1}{|c|}{PROCESS} \\
\multicolumn{1}{|c|}{PARAMETERS} \\
\hline
\hline
{\tt IS} \hfill $I_S$\\
\hline
\end{tabular}
\hfill
\parbox{0.2in}{\ \vspace*{0.2in}\newline +}
\hfill
\begin{tabular}[t]{|p{1in}|}
\hline
\multicolumn{1}{|c|}{GEOMETRY} \\
\multicolumn{1}{|c|}{PARAMETERS} \\
\hline
$Area$\\
\hline
\end{tabular}
\hfill
\parbox{0.2in}{\ \vspace*{0.2in}\newline $\rightarrow$}
\hfill
\begin{tabular}[t]{|p{1.8in}|}
\hline
\multicolumn{1}{|c|}{DEVICE} \\
\multicolumn{1}{|c|}{PARAMETERS} \\
\hline
\hspace*{\fill}$I_{GS} = f(I_S, Area)$\\
\hspace*{\fill}$I_{GD} = f(I_S, Area)$\\
\hline
\end{tabular}
\caption{GASFET leakage current parameter dependencies. \label{zsp3leakage}}
\end{figure}
}

\noindent\underline{\sl I/V Characteristics}\\[0.1in]

\index{GASFET, \spicethree (Raytheon) Model, I/V}
\index{GASFET, \spicethree  (Raytheon) Model, I/V}
\index{I/V Characteristics, see GASFET}
\index{GASFET, I/V Characteristics}
\index{I-V characteristics, see GASFET}
The current/voltage characteristics are evaluated after first
determining the mode (normal: $V_{DS} \ge 0$ or inverted:
$V_{DS} < 0$) and the region (cutoff,
linear or saturation) of the current
$(V_{DS}, V_{GS})$ operating point.\\[0.1in]

\noindent{\sl Normal Mode: ($V_{DS} \ge 0$)}\\[0.2in]
The regions are as follows:\\[0.1in]
\hspace*{\fill}\offsetparbox{
\begin{tabular}{ll}
cutoff region:&$V_{GS}(t-\tau) < V_{T0}$\\
linear region:&$V_{GS}(t-\tau) > V_{T0} \mbox{ and } V_{DS} \le 3/\alpha$\\
saturation region:&$V_{GS}(t-\tau) > V_{T0} \mbox{ and } V_{DS} > 3/\alpha$\\
\end{tabular}}\\[0.1in]
Then
\begin{equation}
I_{DS} = \left\{ \begin{array}{ll}
      0  & \mbox{cutoff region} \\ \\
      Area\,\beta
      \left(1 + \lambda V_{DS}\right)
      {{\textstyle\left[V_{GS}(t-\tau)-V_{T0}\right]^2}\over
      {\textstyle 1 + B[{V_{GS}(t-\tau)-V_{T0}]}}}
      \mbox{Ktanh}
         &\mbox{linear and saturation}\\
         &\mbox{regions} \end{array} \right. %}
      \label{zsp3id}
\end{equation}
where
\begin{equation}
\mbox{Ktanh} = \left\{ \begin{array}{ll}
       1 - \left(1 - V_{DS} \frac{\alpha}{3}\right)^3
        & \mbox{linear region} \\ \\
      1
         &\mbox{saturation regions} \end{array} \right. %}
\end{equation}
is a taylor series approximation to the tanh function.
\noindent{\sl Inverted Mode: ($V_{DS} < 0)$}\\[0.2in]
In the inverted mode the MOSFET I/V characteristics are evaluated as in the
normal mode (\ref{zsp3id}) but with the drain and source subscripts
exchanged.

\notforsspice{The relationships of the parameters describing the I/V
characteristics of the model are summarized in figure
\ref{zsp3i/v}.\\[0.1in]
\begin{figure}[b]
\begin{tabular}[t]{|p{1in}|}
\hline
\multicolumn{1}{|c|}{PROCESS} \\
\multicolumn{1}{|c|}{PARAMETERS} \\
\hline
\hline
{\tt ALPHA} \hfill $\alpha$\\
{\tt B} \hfill $B$\\
{\tt BETA} \hfill $\beta$\\
{\tt LAMBDA} \hfill $\lambda$\\
{\tt VTO} \hfill $V_{T0}$\\
\hline
\end{tabular}
\hfill
\parbox{0.2in}{\ \vspace*{0.2in}\newline +}
\hfill
\begin{tabular}[t]{|p{1in}|}
\hline
\multicolumn{1}{|c|}{GEOMETRY} \\
\multicolumn{1}{|c|}{PARAMETERS} \\
\hline
\hspace*{\fill}$Area$\\
\hline
\end{tabular}
\hfill
\parbox{0.2in}{\ \vspace*{0.2in}\newline $\rightarrow$}
\hfill
\begin{tabular}[t]{|p{1.8in}|}
\hline
\multicolumn{1}{|c|}{DEVICE} \\
\multicolumn{1}{|c|}{PARAMETERS} \\
\hline
\{$I_{DS} =$\newline\hspace*{\fill}$f(Area, \alpha, B, \beta, \lambda, V_{T0})$\}\\
\hline
\end{tabular}
\caption{LEVEL 2 (Raytheon model) I/V dependencies. \label{zsp3i/v}}
\end{figure}
}

\noindent\underline{\sl Capacitances}

\noindent
The drain-source capacitance
\begin{equation}
C_{DS} = Area\,C'_{DS}
\end{equation}
The gate-source capacitance
\begin{equation}
C_{GS} = Area\left[C'_{GS}F_1F_2\left(1 - {{\textstyle V_{\ms{new}}}
   \over{\textstyle V_{BI}}}\right)^{\textstyle - \frac{1}{2}}
   + C'_{GD}F_3\right]
\end{equation}
The gate-source capacitance
\begin{equation}
C_{GD} = Area\left[C'_{GS}F_1F_3\left(1 - {{\textstyle V_{\ms{new}}}
   \over{\textstyle V_{BI}}}\right)^{\textstyle - \frac{1}{2}}
   + C'{GD}F_2\right]
\end{equation}
where
\begin{eqnarray}
F_1 & = & {{\textstyle 1}\over{\textstyle 2}} \left\{ 1 +
    {{\textstyle V_{\ms{eff}}-V_{T0}}
    \over{\textstyle\sqrt{\left( V_e-V_{T0}\right)^2+\delta^2}}}\right\}\\
F_2 & = & {{\textstyle 1}\over{\textstyle 2}} \left\{ 1 +
    {{\textstyle V_{GS}-V_{GD}}
    \over{\textstyle\sqrt{\left( V_{GS}-V_{GD}\right)^2+\alpha^{-2}}}}\right\}\\
F_3 & = & {{\textstyle 1}\over{\textstyle 2}} \left\{ 1 -
    {{\textstyle V_{GS}-V_{GD}}
    \over{\textstyle\sqrt{\left( V_{GS}-V_{GD}\right)^2+\alpha^{-2}}}}\right\}\\
V_{\ms{eff}} & = & {{\textstyle 1}\over{\textstyle 2}} \left\{V_{GS}+V_{GD}+
    \sqrt{\left( V_{GS}-V_{GD}\right)^2+\alpha^{-2}}\right\}\\
\end{eqnarray}
\begin{equation}
V_{\ms{new}} = \left\{ \begin{array}{ll}
    A_1 & A_1 < V_{\ms{MAX}} \\ \\
    V_{\ms{MAX}} & A_1 \ge V_{\ms{MAX}} \\
      \end{array} \right. %}
\end{equation}
and
\begin{equation}
    A_1 = \frac{1}{2}\left[V_e + V_{T0}
          + \sqrt{(V_e+V_{T0})^2+\delta^2}\right]
\end{equation}
In the model $\delta$ and $V_{\ms{MAX}}$ are not
settable by the
user.  Empirically they were determined to be\\[0.1in]
\hspace*{\fill}
$V_{\ms{MAX}} = 0.5$
\hspace*{\fill}
$delta = 0.2$
\hspace*{\fill}\\[0.1in]
\notforsspice{
The capacitance parameter dependencies are summarized in figure
\ref{zsp3cap}.\\[0.2in]

\begin{figure}[b]
\begin{tabular}[t]{|p{1in}|}
\hline
\multicolumn{1}{|c|}{PROCESS} \\
\multicolumn{1}{|c|}{PARAMETERS} \\
\hline
\hline
{\tt ALPHA} \hfill $\alpha$\\
{\tt CGD} \hfill $C'_{GD}$\\
{\tt CGS} \hfill $C'_{GS}$\\
{\tt CDS} \hfill $C'_{DS}$\\
{\tt VBI} \hfill $V_{BI}$\\
{\tt VT0} \hfill $V_{T0}$\\
{\tt M} \hfill $B$\\
\hline
\end{tabular}
\hfill
\parbox{0.2in}{\ \vspace*{0.2in}\newline +}
\hfill
\begin{tabular}[t]{|p{1in}|}
\hline
\multicolumn{1}{|c|}{GEOMETRY} \\
\multicolumn{1}{|c|}{PARAMETERS} \\
\hline
\hspace*{\fill}$Area$\\
\hline
\end{tabular}
\hfill
\parbox{0.2in}{\ \vspace*{0.2in}\newline $\rightarrow$}
\hfill
\begin{tabular}[t]{|p{1.8in}|}
\hline
\multicolumn{1}{|c|}{DEVICE} \\
\multicolumn{1}{|c|}{PARAMETERS} \\
\hline
\{$C_{DS} = f(Area, C'_{DS})$\}\\
\{$C_{GD} = f(Area, C'_{GD}, \alpha,$\newline\hspace*{\fill}$ B, F_C,
V_{BI}, V_{T0})$\}\\
\{$C_{GS} = f(Area, C'_{GS}, \alpha,$\newline\hspace*{\fill}$ B, F_C,
V_{BI}, V_{T0})$\}\\
\hline
\end{tabular}
\caption{Capacitance dependencies. \label{zsp3cap}}
\end{figure}
}

\noindent\underline{\sl \large AC Analysis}\\[0.1in]
\index{GASFET, AC Analysis}

The AC analysis uses the model of figure  \ref{zsp3.ps} with the capacitor values
evaluated at the \dc\ operating point with
\begin{equation}
g_m = {{\textstyle\partial I_{DS}} \over {\textstyle\partial V_{GS}}}
\end{equation}
and
\begin{equation}
R_{DS} = {{\textstyle\partial I_{DS}} \over {\textstyle\partial V_{DS}}}
\end{equation}\\[0.1in]
\noindent\underline{\sl \large Noise Analysis}\\[0.1in]
\index{GASFET, Noise Model}
\index{GASFET, Noise Analysis}

The MOSFET noise model accounts for thermal noise generated in the
parasitic resistamces and shot and flicker noise generated in the
drain source current generator.  The rms (root-mean-square) values of
thermal noise current generators shunting the four parasitic resistance
$R_D$, $R_G$ and $R_S$ are

\begin{eqnarray}
I_{n,D} &=& \sqrt{4kT/R_D}~\mbox{A/}\sqrt{\mbox{Hz}}\\
I_{n,G} &=& \sqrt{4kT/R_G}~\mbox{A/}\sqrt{\mbox{Hz}}\\
I_{n,S} &=& \sqrt{4kT/R_S}~\mbox{A/}\sqrt{\mbox{Hz}}
\end{eqnarray}
Shot and flicker noise are modeled by
a noise current generator in series with the drain-source current generator.
The rms value of this noise generator is
\begin{equation}
I_{n,DS} = \sqrt{I_{\ms{SHOT},DS}^2 + I_{\ms{FLICKER},DS}^2}
\end{equation}
\begin{eqnarray}
I_{\ms{SHOT},DS} &=& \sqrt{4kTg_m\frac{2}{3}} ~~~~\mbox{A/}\sqrt{\mbox{Hz}}
~\mbox{A/}\sqrt{\mbox{Hz}}\\
I_{\ms{FLICKER},DS} &=& \sqrt{{{\textstyle\KF I_{DS}^{\AF}}
                         \over {\textstyle f }}}
~~~~\mbox{A/}\sqrt{\mbox{Hz}}
\end{eqnarray}
where the transconductance
\begin{equation}
g_m = {{\textstyle\partial I_{DS}} \over {\textstyle\partial V_{GS}}}
\end{equation}
is evaluated at the \dc\ operating point and $f$ is the analysis frequency.
