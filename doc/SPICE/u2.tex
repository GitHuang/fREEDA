\notforsspice{\elementx{U}{\spicethree Only}{Lossy RC Transmission Line}
\label{U2element}

\form{\tt U}name N1 N2 N3 ModelName L=LEN \B N=LUMPS\E 

\example{U1 1 2 0 URCMOD L=50U}
URC2 1 12 2 UMODL l=1MIL N=6


N1 and N2 are the two element nodes the  RC  line  connects,
while  N3  is the node to which the capacitances are
connected.  ModelName is the model name, LEN is  the  length  of
the  RC  line in meters.  LUMPS, if specified, is the number
of lumped segments to use in modeling the RC line  (see  the
model  description for the action taken if this parameter is
omitted).

URC Models

The URC model is derived from a model  proposed  by  L.
Gertzberrg  in 1974.  The model is accomplished by a subcircuit
 type expansion of the URC line into a network of lumped
RC  segments  with  internally generated nodes.  The RC segments
 are in a geometric progression, increasing toward  the
middle  of  the  URC  line, with K as a proportionality constant.
The number of lumped segments used, if not specified
on the URC line, is determined by the following formula:


N=

The URC line will be made up strictly of  resistor  and
capacitor  segments  unless  the ISPERL parameter is given a
non-zero value, in which case the  capacitors  are  replaced
with reverse biased diodes with a zero-bias junction capacitance
equivalent to the capacitance  replaced,  and  with  a
saturation  current of ISPERL amps per meter of transmission
line and an optional series resistance equivalent to  RSPERL
ohms per meter.

\keywordtwo{Area}{
{\tt K}
      & Propagation Constant               & -     & 2.0     &   \X
{\tt FMAX}
   & Maximum Frequency of interest      & Hz    & 1.0G    & \X
{\tt RPERL}
  & Resistance per unit length         & $\Omega$/m & 1000    & \X
{\tt CPERL}
  & Capacitance per unit length        & F/m   & 1.0E-15 & \X
{\tt ISPERL}
 & Saturation current per\newline unit length & A/m & 0       &      \X
{\tt RSPERL}
 & Diode Resistance per unit length   & $\Omega$/m & 0       &\X
   }
}
