\chapter{Error\label{chapter:error}}

\section{Introduction}

\spice\ errors can result from errors in syntax, errors in the wiring
of the original circuit resulting in the circuit not being solvable,
and convergence errors during analysis that prevent continuation of
analysis.

\noindent then we have not specified our circuit as drawn.  In this case, we also
leave one terminal of the resistor unconnected to anything else
and \spice\ detects the error and reports it in the output file:
\begin{verbatim}
0*ERROR*:  LESS THAN 2 CONNECTIONS AT NODE      2
\end{verbatim}
In a complex circuit it is always easy
to get one node number wrong on one element but leave all of the nodes connected
to two or more elements.  In this case \spice\ might detect no errors.  If the
output looks `wrong' for any reason, the first thing to do is 
to draw your circuit by looking at the \spice\ file as written and check
that against your intended circuit.

Another important thing to remember about error messages is that \justspice\
is not very good at drawing attention to them.

More expensive commercial versions of \justspice\ are much more user friendly but
still maintain full
\justspice\ upwards compatibility by reporting all errors in the traditional
\justspice\ way as well.
\justspice\ output files 
tend to be long and are cryptic looking.  Error and Warning messages
can be found almost anywhere within them.  If an error has occured it may be
necessary to examine the entire output file to determine exactly what caused
the error.
Mainstream commercial versions of \justspice\ have somewhat better error reporting
than less expensive commercial versions and the public domain versions.

In the following we list errors common to most versions of \spice as they derive
from \spicetwo.  Common errors for some of the commercial versions of
\justspice\ are also included. The errors are arranged in alphabetic order.
Each commercial version of \spice\ supports additional elements
that are particular to tha particular version but generally are selfexplanatory.
For errors messages indicating a particular element the syntax of the element
described in the element catalog chapter (chapter \ref{chapter:element}).


\section{List of Errors}

\error{at least two numeric values required}
Not enough numeric values are provided. See the element form for the element.

\error{cannot use LIST with DEC or OCT sweeps}
See statement for correct form.

\error{conflicting length}
With a transmission line element either the time delay {\tt TD}
and the reference frequency {\tt F} are both specified or the time delay {\tt TD}
and the normalized electrical length {\tt NL} are both specified.
if the parameter pairs are specified then there are two possible electrical
lengths.  See the input forms of the {\tt T} element on page \pageref{Telement}.

\error{conflicting specifications}
With an independent current source {\tt I} or independent 
voltage source{\tt V} element two or more transient types
are specified. See the allowable element line forms on
page \pageref{Ielement} for the {\tt I} element and on page
\pageref{Velement} for the {\tt V} element.

\error{contrary parameters}
See page \pageref{Yelement} for description of this element.

\error{coupling coefficient out of range.}
           The coupling coefficient for a K element must be between 0 amd 1.

\error{digital files option not present}
The digital files (registered) option must be purchased separately.
\justspice\ version dependent.

\error{ERROR -- $\ldots$ does not match nodes of $\ldots$}
Commonly this is because the number of nodes of a subcircuit does not match the
number of nodes of a subcircuit call ({\tt X} element).

\error{ERROR:  CPU Time limit exceeded}
The CPU time specified by the {\tt .OPTIONS} parameter {\tt CPTIME}, or its
default, has been exceeded.

\error{ERROR:  less than 2 connections at node nnnn}
Every node must have at least two connections or elese a node is left
floating and the voltage at the node can not be determined. This may be
either because the node is floating and the voltage at the node is
indeterminate or else because the numerical techniques used require it.

\error{ERROR: model $\ldots$ referenced by $\ldots$ is undefined}
A model was referenced but the actual model was never specified via a {\tt
.MODEL} statement.

\error{ERROR: Node is floating}
This is either because this is only one connection between this node or there is
no \dc\ path from this node to ground as required in determining the \dc\ voltage
at the node.

\error{ERROR: subcircuit $\ldots$ is undefined}
A subcircuit was referenced but the actual subcircuit was never specified via a
{\tt .SUBCKT} statement.

\error{ERROR: transient analysis iterations}
The number of transient iterations
specified by the {\tt .OPTIONS} parameters {\tt ITL4} or {\tt ITL5}, or their
defaults, has been exceeded.

\error{ERROR: voltage loop}
Voltage sources and/or elements such as inductors or transmission lines
that are modeled using controlled voltage sources, are arranged in a loop.
This results in a modified nodeal admittance matrix that can not be solved.

\error{ERROR: voltage source $\ldots$ which controls switch $\ldots$ is undefined}
A voltage source was referenced but the actual element was never specified.

\error{expand: parameter syntax error for $\ldots$}
Error occurred during subcircuit expansion in handling parameters.
With some version of \spice\ parameters are supported.
Parameters must be in the form {\it Keyword = Value}
where {\it Value} may be a numeric value or an
expression.  Either the Keyword is missing or is not an alphanumeric quantity,
or Value is missing or is neither a numeric quantity nor an expression that
evaluates to a numeric quantity.  The error is either on the {\tt X} element
line or on the {\tt .SUBCKT} statement.

\error{Expression evaluation error: function syntax error}
Error in expression evaluation or input prevents continuation.

\error{Expression evaluation error: syntax error}
Error in expression evaluation or input prevents continuation.

\error{Expression evaluation error: undefined parameter}
Error in expression evaluation or input prevents continuation.

\error{Error in expression.}
Error in expression evaluation or input prevents continuation.

\error{Expression syntax error.}
some versions of \spice\ support expressions.  The expression is syntactically
incorrect or other error that prevents evaluation of the expression.

\error{extra fields}
Extra quantities on element line or statement that were not used.

\error{function syntax error in expression.}
Error in expression evaluation or input prevents continuation.

\error{incomplete range}
A range was indicated but is incomplete.

\error{I(node) is not valid}
To specify currents a voltage source must be indicated.  Specifying the
current at a node is meaningless.

\error{incorrect number of parameters}
Not enough parameters specified for this element.

\error{inductor: mutual coupling requires two (or more) inductors}
A K element must comprise two or more inductors.
This element is \justspice\ version dependent.

\error{inductor part of another CORE device}
An inductor is specified as a component of two different K elements
An inductor can only be part of one K element.
This element is \justspice\ version dependent.

\error{inductor part of another K element}
An inductor is specified as a component of two different K elements
An inductor can only be part of one K element.
This element is \justspice\ version dependent.

\error{inductor part of another mutual coupling device}
An inductor is specified as a component of two different K elements
An inductor can only be part of one K element.
This element is \justspice\ version dependent.

\error{inductor part of this K device}
A K element contains two inductors of the same name.
An inductor can only be specified once in a K element inductor list.
This element is \justspice\ version dependent.

\error{invalid .WIDTH card}
The parameters on the {\tt .WIDTH} card are incorrectly specified
or a parameter is not supported by this version of \spice.

\error{invalid analysis type}
Analysis type specified on a .FOUR .TF .NOISE .SENS .MC .PLOT .PRINT .PROBE
statement is incorrect.  Supported analysis types include AC, DC, TRAN and NOISE
although this list is \justspice\ version dependent.
See the full description of the allowable analysis types for the
statement.

\error{invalid card}
The statement may be spelled incorrectly or
this version of \spice\ does not recognize this statement.

\error{invalid device}
The first letter of an element card indicates the particular device being
referred to.  Not all versions of \justspice\ support the same set of elements
and here an non-supported element is being invoked.

\error{invalid device in subcircuit}
Generally any element may be used with a subcircuit.
(between {\tt .SUBCKT} and {\tt .ENDS}. Some versions of
\justspice\ support special elements or forms of elements that can only be used
at the top level circuit.  See the description of the element.

\error{invalid dimension}
The degree (dimension or order) of a polynomial must be more
than 0 and less than the max polynomial order supported by the current
version of \spice .

\error{invalid function}
Some commercial versions of \justspice\ support function evaluations. An
unsupported function is being used.

\error{invalid increment}
Usually caused by specifying an increment of zero on a .DC statement

\error{invalid node number in .SUBCKT statement}
The invalid node number was specified in .SUBCKT statement. This is usually
because the ground node (either 0 or, in some versions of \justspice\,
``{\tt GND}'') was specified in the list of nodes in the {\tt .SUBCKT}
statement.

\error{invalid number}
Due to a number being of the wrong sign or too small or zero where this is not valid.
Possibly an integer was expected and a floating point number was supplied.

\error{invalid option}
The value of the parameter is invalid.

\error{invalid outside of .SUBCKT}
.ENDS is used out of context and does not terminate a subcircuit.

\error{invalid parameter}
Invalid parameter on element line. Probably
misspelled or not supported by this version of \justspice.

\error{invalid parameter in model}
Either the parameter is misspelled ot it is not supported by this version of
\justspice.

\error{invalid port name for transmission line}
Port name in output list is either missing or invalid.

\error{invalid print interval}
See the analysis statement for the requirements on specifying the output
reporting interval.

\error{invalid run number}
A negative number of runs was specified in a Monte Carlo analysis.

\error{invalid specification}
Error in specifying element. See the form for this element.

\error{invalid step size}
Step size must be positive.

\error{invalid sweep type}
Sweep type not supported by this version of \spice\ or sweep type
incorrectly specified.

\error{invalid value}
Due to a value being of the wrong sign or too small or zero where this is not valid.
Possibly an integer was expected and a floating point number was supplied.

\error{last PWL pair incomplete}
The piecewise linear characteristic of an {\tt I} or {\tt V}
element must be specified in time,value pairs.

\error{list of runs}
No runs were specified in a Monte Carlo analysis

\error{missing .ENDS in .SUBCKT}
A subcircuit must end with a {\tt .ENDS} statement. This was missing.

\error{missing analysis type}
Analysis type must be specified in a .FOUR .TF .NOISE .SENS .MC .PLOT
.PRINT .PROBE statement but is missing.

\error{missing component value}
Component value for R L or C element missing

\error{missing control node}
A control node is missing for an {\tt E} element.

\error{missing controlling source}
The name of a controlling voltage source is missing for a {\tt G} element.

\error{missing device or node}
device or node expected but missing.

\error{missing dimension}
The degree (dimension or order) of a polynomial not specified.
The degree of a polynomial should be specified in the form
``{\tt POLY({\it n})}'' or ``{\tt POLY~{\it n}~}''.

\error{missing file name}
File name expected in a {\tt .INCL} but missing.

\error{missing frequency}
Frequency missing on a .FOUR card.

\error{missing gain}
The gain must be specified for an {\tt E} or {\tt F} element.

\error{missing inductor}
In reading K element line expected to read in the name of
an  inductor but it was missing.

\error{missing INOISE or ONOISE}
.NOISE statement and either INOISE or ONOISE parameters required
or illegal parameter.

\error{missing model}
Model Name expected on element line but was missing

\error{missing model name}
Model Name expected on element line but was missing

\error{missing name}
Subcircuit name expected on {\tt .SUBCKT} (or similar) statement but not found.

\error{missing node}
A node number expected but not provided. Possibly not enough nodes specified.
See the form for this element in the element catalog (chapter \ref{chapter:element}.

\error{missing node list}
No nodes are specified.

\error{missing or invalid model name or type}
Either the model name is missing or it is not a valid type for this element.

\error{missing or invalid value}
Due to a value not being supplied where it is expected, being of the wrong sign
or too small or zero where this is not valid.
Possibly an integer was expected and a floating point number was supplied.

\error{missing output variables}
No output variables specified for a
.FOUR .TF .NOISE .SENS .MC .PLOT .PRINT .PROBE statement.

\error{missing parameter}
Parameter expected but not found.

\error{missing pnr}
In a .AS statement the port number is missing.

\error{missing polynomial}
The {\tt POLY} keyword was specified but no polynomial coefficients were found.

\error{missing run count}
The number of Monte Carlo runs is incorrectly specified.

\error{missing second port in S(pnr1,pnr2)}
There must be a second port in a .AS output specification.

\error{missing second node in V(node1,node2)}
In this syntax two nodes must be specified.
%If the voltage at a single node is required use the syntax ???

\error{missing seed}
A random number seed was expected for this element but it was missing

\error{missing source}
Name of source expected but it was not supplied.

\error{missing subcircuit name}
Subcircuit name expected on {\tt .SUBCKT} statement but not found.

\error{missing sweep type}
Sweep type not specified for {\tt .AC analysis}.

\error{missing transconductance}
The transconductance must be specified for a {\tt G} element.

\error{missing transresistance}
The transresistance must be specified for an {\tt H} element.

\error{missing value}
Value (or expression) expected but not found. Either a non-numeric quantity (i.e.
not a number) is in a location where a numeric value value is expected or a
quantity is missing.  If an expression was specified it was either incorrectly
delimited or its evaluation is not supported by this version of \spice.

\error{must be >= 1}
An positive integer value was expected upon input but the value was not 1 or more.

\error{must be a two terminal device}
To specify a current a two terminal device must be indicated.
More elements with more than three terminals the edge current cannot be
uniquely identified by specifying the nodes.

\error{must be a voltage source name}
Name of voltage source expected on element line but not supplied
A voltage source name must begin with V.

\error{must be an inductor}
In reading K element line expected to read in the name of
an  inductor but the name did not begin with 'L'

\error{must be I or V}
In a .AC, .FOUR .TF .SENS .MC .PLOT .PRINT .PROBE statement.
Only I or V can be specified. Something else is in output list.
Generally this is because only {\tt I} or {\tt V} elements can be swept.

\error{must be independent source (I or V)}
An element was specified but it was not an I or V element.

\error{must be monotonically increasing or decreasing}
Quantities in list must be monotonically increasing or decreasing.

\error{must be S}
The keyword ``S'' expected.

\error{must be V}
A list of voltage sources is required but either a numeric value was provided
an element other than a voltage source specified.
(The voltage sources in list of must begin with {\tt V}.)
In the case of a polynomial specification
the number of controlling voltage sources must be equal to the polynomial
degree previously specified on the element line.  Either not enough
voltage sources are specified or a non-voltage source is specified.

\error{name on .ENDS does not match .SUBCKT}
An optional name may be included on a {\tt .ENDS} statement. If specified it must
match the name specified on the matching {\tt .SUBCKT} statement.

\error{nesting level exceeded}
The number of files that can be included  using {\tt .INCL} is limited.
The limit is \spice\ version specific but is typically around 5.

\error{node's voltage already set}
Two attempts have been made to set the initial voltage at a node.
The initial value may be specified using either a {\tt .NODESET}
statement or using the initial condition {\tt IC} parameter on some elements.

\error{not a valid parameter for model type}
parameter not recognized for this model.

\error{not unique}
Some versions of \spice allow abbreviated forms of statements.  The minimum
allowable abbreviation must be unique.  The abbreviation used in the netlist
is not unique and is an abbreviation of two or more statements.

\error{only .AC .DC and .TRAN valid}
The type of analysis in a Monte Carlo run must be either
\error{.DC}, {\tt .AC} or {\tt .TRAN}.

\error{only .MODEL valid in subcircuit}
{\tt .MODEL} is the only statement allowed within a
subcircuit description (between {\tt /SUBCKT} and {\tt .ENDS}.

\error{only one .TEMP and .DC TEMP allowed}
One .TEMP statement and one .DC TEMP statement allowed but not both.
This is \justspice\ version dependent.

\error{Parameter syntax error}
With some version of \spice\ parameters are supported.
The ``{\tt PARAMS:}'' keyword indicates that parameters are to be specified in the
form {\it Keyword = Value} where {\it Value} may be a numeric value or an
expression.  Either the Keyword is missing or is not an alphanumeric quantity,
or Value is missing or is neither a numeric quantity nor an expression that
evaluates to a numeric quantity..

\error{PNR already defined}
The port number can only be defined once.

\error{PNR missing or invalid}
The port number was either missing or not correctly specified.

\error{run count}
The number of Monte Carlo runs is incorrectly specified.

\error{run count must be > 1}
The number of Monte Carlo runs is incorrectly specified.

\error{syntax error.}
Input is in error. Often due to an  non-numeric value in input where a
non-numeric value expected or vice-versa.

\error{syntax error in expression.}
Error in expression evaluation or input prevents continuation.

\error{TD or F must be specified}
With a transmission line element either the time delay {\tt TD}
or reference frequency {\tt F} must be specified.  See the
{\tt T} element on page \pageref{Telement}.

\error{temperature}
A 0~K (Kelvin) temperature is not valid.
The usual problem is that a 0 Celsius temperature was specified but
a temperature specified by a {\tt .temp} statement must be an
absolute temperature (in Kelvin)Q

\error{time must be increasing}
In specifying the transient behavior of an {\tt I} or {\tt V}
element times must be increasing.

\error{time must not be negative}
In specifying the transient behavior of an {\tt I} or {\tt V}
element a negative time was specified.

\error{too many coefficients}
The number of polynomial coefficients specified exceeds that
supported in this version of \spice.

\error{too many inductors}
There is a limit on the number of inductors per K element.
This limit has been exceeded.

\error{too many tolerances}
The number of tolerances that may be specified is limited.  This
limit is \spice\ version dependent.

\error{TooMany}
Too many parameters, values or nodes on element line or {\tt .SUBCKT} statement.

\error{unable to open file}
File specified in a {\tt .INCL} or {\tt .LIB} does not exist in the current
directory or default directories.

\error{undefined parameter}
An unsupported parameter keyword specified. Either this version of
\spice\ does not support this parameter for this element or statement
or the parameter is misspelled.

\error{unknown parameter}
A parameter was used on an element line or in a {\tt .MODEL}
statement which was not recognized. Either this version of
\spice\ does not support this parameter or the parameter is misspelled.

\error{value may not be 0}
A non-zero value expected.

\error{voltage source name}
Name of voltage source expected on element line but not supplied

\error{WIDTH must be 80 or 132}
A \spice\ supports two output log formats that are either 80 columns
or 132 columns wide.  A width other than 80 or 132 was specified.

\error{Z0 must be specified}
With a transmission line element the {\tt Z0} parameter must
always be input using the syntax {\tt Z0=}{\it CharacteristicImpedance}.

%\error{number of nodes must be even}
%The number of nodes of a coupled line {\tt Y} element must be even.
%See page \pageref{Yelement} for description of this element.
%
%\error{type must be EL or PHI}
%The type of coupled line {\tt Y} element model must be {\tt EL} or {\tt PHI}.
%See page \pageref{Yelement} for description of this element.
%
%\error{value is invalid}
%invalid numeric value See page \pageref{Yelement} for description of this element.
%
%\error{H1 or H2 must be nonzero if lat=3}
%See page \pageref{Yelement} for description of this element.
%
%\error{incorrect number of lines or slots}
%See page \pageref{Yelement} for description of this element.
%
%\error{unknown parameter}
%See page \pageref{Yelement} for description of this element.
