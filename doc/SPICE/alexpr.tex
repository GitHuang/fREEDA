In \pspice\ most places where a numeric value is normally used
an expression (within braces {\tt \{ $\ldots$ \})} can be used instead.
An expression can contain any supported mathematical operation, constant numeric
values or expressions. Exceptions are 
     \begin{itemize}
     \item Polynomial coefficients.
     \item The values of the transmission line device parameters {\tt NL} and
           {\tt F}.
     \item The values of the piece-wise linear characteristic in the {\tt PWL}
           form of the independent voltage ({\tt V}) and current ({\tt I})
           sources.
     \item The values of the resistor device parameter {\tt TC}.
     \item As node numbers.
     \item[and]
     \item Values of most statements (such as .TEMP, .AC, .TRAN etc.)
     \end{itemize}
Specifically included are
     \begin{itemize}
     \item The values of all other device parameters.
     \item The values in {\tt .IC} and {\tt .NODESET} statements.
     \item The values in {\tt .SUBCKT} statements.
     \item[and]
     \item The values of all model parameters.
           {\tt F}.
     \end{itemize}

Operators that can be used in expressions are listed in Table
\ref{table:expression:operators}.
\clearpage
\begin{table}
\caption{Expression operators.\label{table:expression:operators}}
\begin{center}
\begin{tabular}{|l|l|l|}
\hline
{\bf Operator }	& {\bf   Syntax } & {\bf Description}  \\
\hline
PLUS		& {\it x}{\tt +}{\it y}		& plus \\
MINUS		& {\it x}{\tt -}{\it y}		& minus \\
UNARY\_PLUS	& {\tt +}{\it x}            & unary plus \\
UNARY\_MINUS	& {\tt -}{\it x}            & unary minus \\
MULTIPLY	& {\it x}{\tt *}{\it y}		& multiply \\
DIVIDE		& {\it y}{\tt /}{\it x}	        & divide \\
POW		& {\it x}{\tt \^{ }}{\it y}  or
                  {\it x}{\tt **}{\it y} & raise to a power, $x^y$ \\
AND		& {\it x}{\tt \&}{\it y}           & AND \\
OR		& {\it x}{\tt |}{\it y}           & OR \\
NOT		& {\tt !}{\it x}		& NOT \\
XOR		& {\it x}{\tt ~}{\it y}		& XOR (exclusive or) \\
SIN		& {\tt sin(}{\it x}{\tt )}	& sine, argument in radians \\
COS		& {\tt cos(}{\it x}{\tt )}	& cosine, argument in radians \\
TAN		& {\tt tan(}{\it x}{\tt )}& tangent, argument in radians \\
ASIN		& {\tt asin(}{\it x}{\tt )}& arcsine, argument in radians \\
ACOS		& {\tt acos(}{\it x}{\tt )}& arccosine, argument in radians \\
ATAN		& {\tt atan(}{\it x}{\tt )}& arctangent, argument in radians \\
SINH		& {\tt sinh(}{\it x}{\tt )}	& hyperbolic sine \\
COSH		& {\tt cosh(}{\it x}{\tt )}	& hyperbolic cosine \\
TANH		& {\tt tanh(}{\it x}{\tt )}	& hyperbolic tangent \\
EXP		& {\tt exp(}{\it x}{\tt )}	& exponentiation, $e^x$ \\
ASINH		& {\tt asinh(}{\it x}{\tt )}& arc-hyberbolic sine \\
ACOSH		& {\tt acosh(}{\it x}{\tt )}& arc-hyberbolic cosine \\
ATANH		& {\tt atanh(}{\it x}{\tt )}& arc-hyberbolic tangent \\
ABS		& {\tt abs(}{\it x}{\tt )}	& absolute, $|x|$ \\
SQRT		& {\tt sqrt(}{\it x}{\tt )}	& square root, $\sqrt{x}$ \\
\hline
\end{tabular}
\end{center}
\end{table}
