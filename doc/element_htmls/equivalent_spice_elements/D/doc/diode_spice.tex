%\documentclass{article}
%\usepackage{epsf,html}
%\newcommand{\fig}[1]{J:/eos.ncsu.edu/users/m/mbs/mbs_group/figures/#1}
%\newcommand{\fig}[1]{../figures/#1}
%\newcommand{\pfig}[1]{\epsfbox{\fig{#1}}}

\oddsidemargin 10mm \topmargin 0.0in \textwidth 5.5in \textheight
7.375in \evensidemargin 1.0in \headheight 0.18in \footskip 0.16in
%%%%%%%%%%%%%%%%%%%%%%%%%%%%%%%%%%%%%%%% The title
%\begin{document}
\section[D \- Diode]{\noindent{\LARGE \textbf{Diode}} \hspace{120mm}\huge\textbf{D}}
\linethickness{1mm}
\line(1,0){425}
\normalsize
%%%%%%%%%%%%%%%%%%%%%%%%%%%%%%%%%%%%%%%% the resistor figure
\begin{figure}[h]
\centerline{\epsfxsize=1in\pfig{diode_transim.ps}} \caption{D ---
Diode Element.}
\end{figure}
%\newline
%%%%%%%%%%%%%%%%%%%%%%%%%%%%%%%%%%%%%%%%%%% SPICE form
%\vspace{2mm}
\newline
\linethickness{0.5mm} \line(1,0){425}
\newline
\texttt{SPICE} \textit{Form:}
\newline
\texttt{D}\textit{name} $n_1$ $n_2$
\texttt{[}\textit{ModelName}\texttt{]} \ [\textit{Area}] \
[\texttt{OFF}] \ [\texttt{IC}=$V_D$]
\newline
%\vspace{2mm}
%%%%%%%%%%%%%%%%%%%%%%%%%%%%%%%%%%%%%%%%%%%%%%% explanation of terms in the SPICE form
\newline
\begin{tabular}{r l}
$n_1$ & is the positive element node, \\
$n_2$ & is the negative element node, \\
\textit{ModelName} & is the optional model name, (\textit{ModelType} is \texttt{diode}.)\\
\textit {Area} & is an optional relative area factor,  \\
                & (Units: none; Optional; Default: 1, Symbol:
                \textit{Area})\\
\texttt{OFF} & Indicates an optional starting condition on the device for DC operating point analysis.\\
             & If specified, the DC operating point is calculated with the terminal voltages set to zero.\\
             & Once convergence is obtained, the program continues to iterate to obtain the exact value\\
             & of terminal voltages. The OFF option, is used to enforce the solution to correspond to a \\
             & desired state if the circuit has more than one stable state.\\
\texttt{IC}  & is the optional initial condition specification. Using \texttt{IC}=$V_D$ is used with \\
             & the \textit{UIC} option on the .\textit{TRAN} line when a transient analysis is desired  \\
             & with initial current $V_D$ across the diode rather than the quiescent  \\
             & operating point. Specification of the transient initial condition using the  \\
             & .\texttt{IC} is preferred and is more convenient.
\end{tabular}
%\newline
%%%%%%%%%%%%%%%%%%%%%%%%%%%%%%%%%%%%%%%%%%%%%%% Parameter table
%\vspace{4mm}
\newline
\linethickness{0.5mm} \line(1,0){425}
\newline
\textit{Diode Model:}\\
\textit{Form:}\\
.MODEL \ \textit{ModelName} \ \texttt{D}([[$\mathit{keyword} = \mathit{value}$]...])\\
\linethickness{0.5mm} \line(1,0){425}
\newline
\textit{Example:}
\newline
\texttt{DCLMP \ 3 \ 7 DMOD \ 3.0 IC=0.2}
\newline
\texttt{.MODEL DMOD D(\texttt{IS}=100pA \ \texttt{N}=1.68 \
\texttt{BV}=10V \ \texttt{IBV}=1nA)}
\newline
\linethickness{0.5mm} \line(1,0){425}
\newline
\textit{Model Parameters:}
\newline
%%%%%%%%%%%%%%%%%%%%%%%%%%%%%%%%%%%%%%%%%%%%%% Parameters
\begin{tabular}{|r|l|c|c|}
\hline
\textbf{Name} & \textbf{Description} & \textbf{Units} & \textbf{Default} \\
\hline
\texttt{AF} & Flicker noise exponent ($AF$) & & 1 \\
\hline
\texttt{AFAC} & Temperature related coefficient ($AFAC$) & & 1 \\
\hline
\texttt{ALFA} & Slope factor of conduction current ($ALFA$) & /volt & 38.696 \\
\hline
\texttt{AR0} & R0 Linear temperature coefficient ($AR_0$) & K & 0 \\
\hline
\texttt{AREA} & Area multiplier ($AREA$) & & 1.0 \\
\hline
\texttt{BR0} & R0 Quadratic temperature coefficient ($BR_0$) & K$^2$ & 0 \\
\hline
\texttt{BV} & Magnitude of current at breakdown voltage ($V_B$) & & $10^-14$ \\
\hline
\texttt{CD0} & Zero bias diffusion capacitance ($C_{D0}$) & /volt & 0.0 \\
\hline
\texttt{CJ0} & Zero bias depletion capacitance ($C_{J0}$) & farad & 0 \\
\hline
\texttt{E} & Power law parameters of breakdown current ($E$) & & 10.0 \\
\hline
\texttt{EG} & Barrier height at 0K ($E_G$) & eV & 0.8 \\
\hline
\texttt{IS} & Saturation current ($I_S$) & amps & 0 \\
\hline
\texttt{KF} & Flicker noise coefficient ($K_F$) & & 1.0 \\
\hline
\texttt{M} & Grading coefficient ($M$) & & 0.5 \\
\hline
\texttt{N} & Emission coefficient ($n$) & & 1.0 \\
\hline
\texttt{R0} & Bias dependent part of series resistance in the forward bias ($R_0$) & ohms & 0\\
\hline
\texttt{TT} & Intrinsic time constant of depletion layer for abrupt junction ($\tau _t$) & secs & 0 \\
\hline
\texttt{VB} & Breakdown voltage ($V_B$)& volts & $\infty$ \\
\hline
\texttt{VJ} & Junction potential ($\phi$) & volts & 1.0 \\
\hline
\texttt{XTI} & IS temperature coefficient ($XTI$) & & 1 \\
\hline
\end{tabular}
%\vspace{4mm}
%%%%%%%%%%%%%%%%%%%%%%%%%%%%%%%%%%%%%%%%%%%%%%%%%%%%%%%%%%%%%%%%%%%%% example in SPICE
\newline
\linethickness{0.5mm} \line(1,0){425}
\newline
\textit{Example:}
\newline
\texttt{DCLMP \ 3 \ 7 DMOD \ 3.0 IC=0.2}
\newline
\texttt{.MODEL DMOD D(\texttt{IS}=100pA \ \texttt{N}=1.68 \
\texttt{BV}=10V \ \texttt{IBV}=1nA)}
\newline
\linethickness{0.5mm} \line(1,0){425}
\newline
\textit{Description:}\\
%\centerline{Table 1}
%\newpage
%\linethickness{0.5mm}\line(1,0){425}
%\vspace{2mm}
%\newline
DIODE MODEL:\\
%\newline
\begin{figure}[h]
\centerline{\epsfxsize=1.5in\pfig{diode_fig2.ps}}
\caption{Schematic of the Diode Element Model}
\end{figure}
\newline
%\vspace{5mm}
%%%%%%%%%%%%%%%%%%%%%%%%%%%%%%%%%%%%%%%%%%%%%%%%%%% Device equations start
The physical constants used in the model evaluation are:
\newline
\begin{tabular}{c c c}
$k$ & Boltzmann's constant & 1.3806226 $10^{-23} J/K$ \\
$q$ & electron charge & 1.6021918 $10^{-19} C$
\end{tabular}
%\vspace{4mm}
\newline
\line(1,0){425}
\newline
DEVICE EQUATIONS:
\newline
%\vspace{4mm}
%%%%%%%%%%%%%%%%%%%%%%%%%%%%%%%%%%%%%%%%%%%% Current characteristics
\underline{Current Characteristics}
\newline
\begin{equation}
I_D = I_S(e^{ALFA \: V_D} - 1) - I_B
\end{equation}
\begin{equation}
I_B = \left\{ \begin{array}{ll}
                0 & V_D \geq (1+V_B) \\
                I_{BV}{(1+V_B+V_D)}^E & V_D < (1+V_B)
                \end{array}
        \right.
\end{equation}
where $ALFA$ = 1/($n$ $V_{TH}$)
\newline
$V_{TH}$=$kT/q$
\newline
\newline
%%%%%%%%%%%%%%%%%%%%%%%%%%%%%%%%%%%%% Capacitance equations
\underline{Capacitance}
\newline
\begin{equation}
C_J= \left\{ \begin{array}{ll}
             C_{J0}(1-V_D/\phi)^{-M} + C_D & V_D \leq 0.8\phi \\
             C_{J0} 0.2^{-M} + C_D & V_D > 0.8\phi
             \end{array}
    \right.
\end{equation}
where $C_D$ is the diffusion capacitance.
\newline
$C_D$ = $C_{D0}e^{AFAC V_D}$
\newline
%%%%%%%%%%%%%%%%%%%%%%%%%%%%%%%% Resistance equations
%\newpage
\underline{Parasitic Resistance}
\newline
\begin{equation}
R_S= \left\{ \begin{array}{ll}
             R_0 - \tau/C_J & R_0 > \tau/C_J \\
             0 & R_0 \leq \tau/C_J
             \end{array}
    \right.
\end{equation}
%%%%%%%%%%%%%%%%%%%%%%%%%%%%%%%%%%%%%% Temperature related equations
\underline{Temperature Dependence}
\newline
\newline
$T$ is the analysis temperature
\newline
$T_NOM$ is the reference temperature (298 K)
\newline
$V_TH=kT/q$
\newline
\begin{equation}
\begin{array}{l}
I_S(T)=I_S[e^{(T/T_{NOM-1})E_G/V_{TH}}](T/T_{NOM})^{XTI} \\
\\
\phi(T) = \phi(T/T_NOM) -3V_{TH}\ln(T/T_{NOM}) - E_G(T_{NOM})(T/T_{NOM}) + E_G(T) \\
\\
E_G(T)=E_G-0.000702(T^2/T+1108) \\
\\
C_{J0}(T)=C_{J0}(1+M(0.0004(T-T_{NOM})+(1-\phi(T)/\phi))) \\
\\
R_0(T)=R_0(1+AR_0(T-T_{NOM})+BR_0(T-T_{NOM})^2) \\
\\
V_B(T)=V_B(1+AV_B(T-T_{NOM})+BV_B(T-T_{NOM})^2)
\end{array}
\end{equation}
%%%%%%%%%%%%%%%%%%%%%%%%%%%%%%%%%%%%%%%%%%%%% Credits
%\newline
%\vspace{4mm}
\newline
\linethickness{0.5mm} \line(1,0){425}
\newline
\textit{Notes:}\\
The actual element in \FDA is the \texttt{D} element.
%See \FDA element \texttt{diode} for full documentation.\\
See
\htmladdnormallink{diode}{J:/eos.ncsu.edu/users/m/mbs/mbs_group/transim_manual/transim_elements/diode/documentation/diode.tex}
for full documentation.\\
%%%%%%%%%%%%%%%%%%%%%%%%%%%%%%%%%%%%%%%%%%%%% Credits
%\newline
%\vspace{4mm}
%\newline
\linethickness{0.5mm} \line(1,0){425}
\newline
\textit{Credits:}
\newline
\begin{tabular}{l l l l}
Name & Affiliation & Date & Links \\
Carlos E. Christofferson & NC State University & Sept 2000 & \epsfxsize=1in\pfig{logo.eps} \\
cechrist@ieee.org & & & www.ncsu.edu    \\
\end{tabular}
%\end{document}
