%\documentclass{article}
%\usepackage{epsf}
%\newcommand{\fig}[1]{J:/eos.ncsu.edu/users/m/mbs/mbs_group/figures/#1}
%\newcommand{\fig}[1]{../figures/#1}
%\newcommand{\pfig}[1]{\epsfbox{\fig{#1}}}
%\newcommand{\ms}[1]{\mbox{\scriptsize #1}}
%\newcommand{\B}{{ \rm [}}     % begin optional parameter in \form{}
%\newcommand{\E}{{\ \rm\hspace{-0.04in}] }}   % end optional parameter in \form{}
\oddsidemargin 10mm \topmargin 0.0in \textwidth 5.5in \textheight
7.375in \evensidemargin 1.0in \headheight 0.18in \footskip 0.16in
%%%%%%%%%%%%%%%%%%%%%%%%%%%%%%%%%%%%%%%% The title
%\begin{document}
\section[M \- MOSFET]{\noindent{\LARGE \textbf{MOSFET} \hspace{90mm}\huge\textbf{M}}}
%\newline
\linethickness{1mm}
\line(1,0){425}
\normalsize
%%%%%%%%%%%%%%%%%%%%%%%%%%%%%%%%%%%%%%%% the resistor figure
\begin{figure}[h]
\centering \epsfxsize=4in\pfig{m_spice.ps} \caption[M --- MOSFET
element]{M --- MOSFET element: (a) $n$-channel enhancement-mode
MOSFET; (b) $p$-channel enhancement-mode MOSFET; (c) $n$-channel
depletion-mode MOSFET; and (d) $p$-channel depletion-mode MOSFET.
\label{m.ps}}
\end{figure}
%%%%%%%%%%%%%%%%%%%%%%%%%%%%%%%%%%%%%%%% form for \FDA
\linethickness{0.5mm} \line(1,0){425}
\newline
\textit{Form:}
\newline
{\tt M}{\it name} {\it NDrain NGate NSsource NBulk ModelName}
              \B {\tt L}={\it Length}\E  \B {\tt W}={\it Width}\E  \\
      {\tt +} \B {\tt AD}={\it DrainDiffusionArea}\E
              \B {\tt AS}={\it SourceDiffusionArea}\E  \\
      {\tt +} \B {\tt PD}={\it DrainPerimeter}\E  \B {\tt PS}={\it SourcePerimeter}\E \\
      {\tt +} \B {\tt NRD}={\it RelativeDrainResistivity}\E
              \B {\tt NRS}={\it RelativeSourceResistivity}\E \\
      {\tt +} \B {\tt OFF}\E  \B {\tt IC=}$V_{DS},V_{GS},V_{BS}$\E
\newline
where\\
%%%%%%%%%%%%%%%%%%%%%%%%%%%%%%%%%%%%%%%%
\begin{widelist}
\item[{\it NDrain}] is the drain node.
\item[{\it NGate}] is the gate node.
\item[{\it NSource}] is the source node.
\item[{\it NBulk}] is the bulk or substrate node.
\item[{\it ModelName}]  is  the  model name.
\end{widelist}

\begin{widelist}
\item[{\it L}] is the channel lateral diffusion  {\it Length}.\\
               (Units: m; Optional; Symbol: $\Length$;
\notforsspice{ The default is version
           dependent.\\
               \spicetwo\ and \spicethree\ Default: the
               length {\tt DEFL} most recently specified in a {\tt .OPTION}
               statement which in-turn defaults to 100~$\mu$m (100U);
           \pspice\ }Default: the length
           {\tt L} -- length specified in
               model {\it ModelName} which in turn defaults to the default
               length {\tt DEFL} most recently specified in a {\tt .OPTION}
               statement which in-turn defaults to 100~$\mu$m (100U).)

\item[{\it W}] is the channel lateral diffusion {\it Width}.\\
               (Units: m; Optional; Symbol: $\W$;
\notforsspice{The default is version dependent.\\
           \spicetwo\ and \spicethree\ Default: the
               width {\tt DEFW} most recently specified in a {\tt .OPTION}
               statement which in-turn defaults to 100~$\mu$m (100U);\\
           \pspice\ }Default: the width
           {\tt W} -- width specified in
               model {\it ModelName} which in turn defaults to the default
               width {\tt DEFW} most recently specified in a {\tt .OPTION}
               statement which in-turn defaults to 100~$\mu$m (100U).)

\item[{{\tt AD}}]  is the area of the drain diffusion
               ({\it DrainDiffusionArea}). The default is {\tt DEFAD}
           most recently specified in a {\tt .OPTIONS} statement.\\
               (Units: $\mbox{m}^2$; Optional; Default: {\tt DEFAD};
           Symbol: $A_D$)

\item[{{\tt AS}}]  is the area of the source diffusion
               ({\it SourceDiffusionArea}).  The default is {\tt DEFAS}
           most recently specified in a {\tt .OPTIONS} statement.\\
               (Units: $\mbox{m}^2$; Optional; Default: {\tt DEFAS};
           Symbol: $A_S$)

\item[{\tt PD}]  is the perimeter of the drain junction
               ({\it DrainPerimeter}).\\
               (Units: m; Optional; Default: 0; Symbol: $P_D$)

\item[{\tt PS}]  is the perimeter of the source junction
               ({\it SourcePerimeter}).\\
               (Units: m; Optional; Default: 0; symbol: $P_S$)

\item[{\tt NRD}]  is  the  relative resistivity in squares of the drain region
               ({\it RelativeDrainResistivity}).
               The sheet resistance {\tt RSH} specified in the model {\it
           ModelName} is divided  by this factor to obtain the
           parasitic drain resistance.\\
           (Units: squares; Optional; Default: 0; Symbol: $\NRD$)

\item[{\tt NRS}]  is  the  relative resistivity in squares of the source region
               ({\it RelativeSourceResistivity}).
               The sheet resistance {\tt RSH} specified in the model {\it
           ModelName} is divided  by this factor to obtain the
           parasitic source resistance.\\
           (Units: squares; Optional; Default: 0; Symbol: $\NRS$)
\pspiceonly{
\item[{\tt NRG}]  is  the  relative resistivity in squares of the gate region
               ({\it RelativeGateResistivity}).
               The sheet resistance {\tt RSH} specified in the model {\it
           ModelName} is divided  by this factor to obtain the
           parasitic gate resistance.\\
           (Units: squares; Optional; Default: 0; Symbol: $\NRG$)
           }
\end{widelist}

\notforsspice{
\begin{widelist}
\pspiceonly{
\item[{\tt NRB}] is the relative resistivity in squares of the bulk
               (substrate) region
               ({\it RelativeBulkResistivity}).
               The sheet resistance {\tt RSH} specified in the model {\it
           ModelName} is divided  by this factor to obtain the
           parasitic bulk resistance.\\
           (Units: squares; Optional; Default: 0; Symbol: $\NRB$)
           }
\item[{\tt OFF}] indicates an (optional) initial condition on the device for
\dc\ analysis. If specified the \dc\ operating point is calculated
with the terminal voltages set to zero.  Once convergence is
obtained, the program continues to iterate to obtain the exact
value of the  terminal  voltages.  The OFF option is used to
enforce the solution to  correspond  to  a  desired  state if the
circuit has more than one stable state.
\item[{\tt IC}] is the optional  initial condition specification.
Using {\tt IC=}$V_{DS},V_{GS}, V_{BS}$ is intended for use with
the {\tt UIC} option on  the  {\tt .TRAN}  line,  when  a
transient analysis is desired starting from other than the
quiescent operating point. Specification of the transient initial
conditions using the {\tt .IC} statement is preferred and is more
convenient.
\end{widelist}
}
%%%%%%%%%%%%%%%%%%%%%%%%%%%%%%%%%%%%%%% Parameter list
%%%%%%%%%%%%%%%%%%%%%%%%%%%%%%%%%%%%%%% example in \FDA
%\newline
\linethickness{0.5mm} \line(1,0){425}
\newline
\textit{Example:}
\newline
\texttt{M1 24 2 0 20 TYPE1 \\
         M31 2 17 6 10 MODM L=5U W=2U \\
         M1 2 9 3 0 MOD1 L=10U W=5U AD=100P AS=100P PD=40U PS=40U}
\newline
\linethickness{0.5mm} \line(1,0){425}
\newline
\textit{Description:}\\
The parameters of a MOSFET can be completely specified in the
model {\it ModelName}. This facilitates the use of standard
transistors by using absolute quantities in the model.
Alternatively scalable process parameters can be specified in the
model {\it ModelName} and these scaled by geometric parameters on
the MOSFET element line. \notforsspice{In \spicetwo\ and
\spicethree\ the width {\tt W} can not be specified in the model
statement. For these simulators absolute device parameters must be
specified in the model statement if parameters are not input on
the element line.}

\modeltype{NMOS\\PMOS}

\example{
M1 5 5 1 1 PCH L=2.0U W=20U AD=136P AS=136P\\
$\cdot$\\
$\cdot$\\
$\cdot$\\
.MODEL PCH PMOS LEVEL=2 VTO=-0.76 GAMMA=0.6 CGSO=3.35E-10\\
+ CGDO=3.35E-10 CJ=4.75E-4 MJ=0.4 TOX=225E-10 NSUB=1.6E16\\
+ XJ=0.2E-6 LD=0 UO=139 UEXP=0 KF=5E-30 LAMBDA=0.02\\
\\
\\
%M3 5 3 4 0 NCH L=2.4U W=80U AD=136P AS=136P\\
%$\cdot$\\
%$\cdot$\\
%$\cdot$\\
%.MODEL NCH NMOS LEVEL=2 VTO=0.71 GAMMA=0.29 CGSO=2.89E-10\\
%+ CGDO=2.89E-10 CJ=3.27E-4 MJ=0.4 TOX=225E-10 NSUB=3.5E16\\
%+ XJ=0.2E-6 LD=0 UO=411 UEXP=0 KF=6.5E-28 LAMBDA=0.02 \\
}

%
% NMOS/PMOS
%
\model{NMOS}{N-CHANNEL MOSFET MODEL} \model{PMOS}{P-CHANNEL MOSFET
MODEL}


Two groups of model parameters define the linear and nonlinear
elements of the MOSFET models. One group defines absolute
quantities and another group defines quantities that are
multiplied by scaling parameters related to area and dimension
which are specified on the element line. This enables the MOSFET
element to be used in two ways.  Using the absolute quantities the
characteristics of a device can be defined independent of the
parameters on the element line. Thus the model of a standard
transistor, perhaps resident in a library, can be used without
user-knowledge required. Using the scalable quantities the
parameters of a fabrication process can be defined in the model
statement and scaling parameters such as the lateral diffusion
length (specified by {\tt L}) and the lateral diffusion width
(specified by {\tt W}), and the drain and source diffusion areas
(specfified by {\tt AD} and {\tt AS} specified on the element
line). An example is the specification of the drain-bulk
saturation current $I_{D,\ms{SAT}}$.  This parameter can be
specified by the absolute parameter $\IS$ specified by the {\tt
IS} model keyword.  It can also be determined as $\IS$ = $\JS\cdot
A_D$ using the scalable parameter $\JS$ specified by the {\tt JS}
model keyword and $A_D$ specified by the {\tt AD} element keyword.

\spice\ provides four MOSFET device models.  The first three
models, known as {\tt LEVEL}s 1, 2 and 3 differ in  the
formulation of the I-V characteristic.  The fourth model, known as
the BSIM model, uses a completely different formulation utilizing
extensive semiconductor parameters. The parameter {\tt LEVEL}
specifies the model to be used:
\\[0.2in]
\noindent\begin{tabular}{p{0.9in}p{0.2in}p{4in}}
         LEVEL = 1& $\rightarrow$ &  ``Shichman-Hodges'', MOS1\newline
                    This model was the first SPICE MOSFET model and
              was developed
              in 1968 \cite{shichman:hodges:68}.
              It is an elementary model and has a limited
              scaling capability.  It is applicable to fairly large
              devices with gate lengths greater than 4~$\mu$m.
              Its main attribute is that only a few parameters need
              be specified and so it is good for preliminary analyses.\\
              \\
         LEVEL = 2& $\rightarrow$ &  MOS2\newline
             This is an analytical model which uses a combination of
         processing parameters and geometry. The major development over
         the {\tt LEVEL} 1 model is improved treatment of the
         capacitances due to the channel charge.
         \cite{meyer:71,ward:dutton:78,oh:ward:80}.  The model dates
         from 1980 and is applicable for chanel lengths of 2~$\mu$m
         and higher \cite{vladimirescu:liu:80}.

             The {\tt LEVEL} 2 model has convergence problems and is slower and
         less accurate than the {\tt LEVEL} 3 model.
\end{tabular}
\vspace*{\fill}

\noindent\begin{tabular}{p{0.9in}p{0.2in}p{4in}}
         LEVEL = 3& $\rightarrow$ &  MOS3\newline
             This is a semi-empirical model developed in 1980
             \cite{vladimirescu:liu:80}.  It is also
         used for gate lengths of 2~$\mu$m and more.
         The parameters of this model are determined by experimental
         characterization and so it is more accurate than the {\tt LEVEL}
             1 and 2 models that use the more indirect process parameters.
\notforsspice{
         \\
         \\
         LEVEL = 4& $\rightarrow$ &  BSIM or BSIM1\newline
             The BSIM model is an advanced empirical model which uses process
         information and a larger number of parameters (more than 60)
         to describe the operation of devices with gate lengths as short
         as 1~$\mu$m.  It was developed in 1985 \cite{sheu:scharfetter:87}.
         }
         \end{tabular}\\[0.1in]
\notforsspice{ Other MOSFET models or {\tt LEVEL}s are available
in various versions of SPICE. These {\tt LEVEL}s are optimized for
MOSFETs fabricated in a particular foundary or provide a
proprietary edge for the advanced commercial SPICE programs. The
reader interested in more advanced MOSFET models is refered to
\cite{lee:shur:93}.\\\vfill} \noindent{\large\bf LEVEL 1, 2 and 3
MOSFET models.} \noindent\myline
\begin{figure}[h]
\epsfxsize=2.75in\centerline{\pfig{level123.ps}}
\caption[Schematic of {\tt LEVEL} 1, 2 and 3 MOSFET
models]{Schematic of {\tt LEVEL} 1, 2 and 3 MOSFET models.
\label{mlevel123} $V_{GS}$, $V_{DS}$, $V_{GD}$, $V_{GB}$, $V_{DB}$
and $V_{BS}$ are voltages between the internal gate, drain, bulk
and source terminals designated $G$, $D$, $B$ and $C$
respectively. }
\end{figure}

\noindent The {LEVEL} 1, 2 and 3 models have much in common. These
models evaluate the junction depletion capacitances and parasitic
resistances of a transistor in the same way. They differ in the
procedure used to evaluate the overlap capacitances ($C_{GD}$,
$C_{GS}$ and $C_{GB}$) and that used to determine the
current-voltage characteristics of the active region of a
transistor. The overlap capacitances model charge storage as
nonlinear thin-oxide capacitance distributed among the gate,
source drain and bulk regions. These capacitances are important in
describing the operation of MOSFETs. The {\tt LEVEL} 1, 2 and 3
models are intimately intertwined as combinations of parameters
can result in using equations from more than one model. The {\tt
LEVEL} parameter resolves conflicts when there is more than one
way to calculate the transistor characteristics with the
parameters specified by the user. Antognetti and Massobrio provide
a comprehensive discussion of the development of the {\tt LEVEL}
1, 2 and 3 models \cite{antognetti:massobrio:88}.

The parameters of the {\tt LEVEL} 1, 2 and 3 models are given in
table \ref{mtable123}. Parameter extensions for \pspice\ are given
in table \ref{mtable123pspice}. It is assumed that the model
parameters were determined or measured at the nominal temperature
$T_{\ms{NOM}}$ (default $27^{\circ}C$) specified in the most
recent {\tt .OPTIONS} statement preceeding the {\tt .MODEL}
statement. In \pspice\ this is overwritten by the {\tt
T\_MEASURED} parameter. Most of the parameters have default
values. Those parameters that have INFERRED defaults are
calculated from other parameters. \vfill

\begin{table}[h]
\caption{MOSFET model keywords for {\tt LEVEL}s 1, 2, 3.
\label{mtable123}} \keywordtable{ {\tt AF}      &flicker noise
exponent       \sym{\AF}& -     &   1      \X {\tt CBD} &zero-bias
B-D junction capacitance \sym{\CBD}
     & F     &   0    \X
{\tt CBS} &zero-bias B-S junction capacitance \sym{\CBS}
     & F     &   0    \X
{\tt CGBO}    &gate-bulk overlap capacitance
         per meter of channel length \para\sym{\CGBO}&F/m   &   0    \X
{\tt CGDO}    &gate-drain overlap capacitance
         per meter of channel width\para\sym{\CGDO}& F/m   &   0    \X
{\tt CGSO}    &gate-source overlap capacitance
         per meter of channel width \para\sym{\CGSO}&F/m   &   0    \X
{\tt CJ}      &zero-bias bulk junction bottom capcitance
         per square meter of junction area \para \sym{\CJ}&$\mbox{F/m}^2$&0\X
{\tt CJSW}    &zero-bias bulk junction sidewall capacitance
         per meter of junction perimeter \para\sym{\CJSW}&F/m&0\X
{\tt DELTA}   &width effect on threshold voltage
         ({\tt LEVEL}=2 and {\tt LEVEL}=3)   \sym{\DELTA}  & -     &   0      \X
{\tt ETA}     &static feedback ({\tt LEVEL}=3 only)  \sym{\ETA} &
-& \inferred\X }
\end{table}

\begin{table}[h]
\caption{MOSFET model keywords for {\tt LEVEL}s 1, 2, 3 continued.
\label{mtable1234}} \keywordtable{
%\kw{Table \ref{mtable123}continued: MOSFET Model Keywords for
            %{\tt LEVEL}s 1, 2, 3.\vshift}{
{\tt FC}      &coefficient for forward-bias
         depletion capacitance\newline formula \para \sym{\FC}&- & 0.5\X
{\tt GAMMA}
   &bulk threshold parameter\sym{\GAMMA}& $\mbox{V}^{\frac{1}{2}}$&   \inferred\X
{\tt IS}
      &bulk junction saturation current \para \sym{\IS}  & A     &   $10^{-14}$\X
{\tt JS}      &bulk junction saturation current
         per sq-meter of junction area \para\sym{\JS}&$\mbox{A/m}^2$ &0\X
{\tt KAPPA}   &saturation field factor ({\tt LEVEL}=3 only)
\sym{\KAPPA}&
          -     &   0.2\X
{\tt KF}      &flicker noise coefficient\sym{\KF}& -     &   0
\X {\tt KP}
      &transconductance parameter \sym{\KP}   & $\mbox{A/V}^2$&   2.$10^{-5}$  \X
{\tt LAMBDA}  &channel-length modulation
                  \newline ({\tt LEVEL}=1, 2 only) \sym{\LAMBDA}
    &   1/V      &  0\X
{\tt LD}      &lateral diffusion   \sym{\LD}&m       &   0      \X
{\tt LEVEL}   &model index                       & -     &   1 \X
{\tt MJ}      &bulk junction bottom grading coefficient \para
 \sym{\MJ}& -     &   0.5  \X
{\tt MJSW}
    &bulk junction sidewall grading coefficient
    \para\sym{\MJSW}&-&0.33  \X
{\tt NSUB}
    &substrate doping  \sym{\NSUB}     &$\mbox{cm}^{-3}$&   \inferred    \X
{\tt NSS}
     &surface state density \sym{\NSS}
     &$\mbox{cm}^{-2}$&   \inferred    \X
{\tt NFS}
     &fast surface state density\sym{\NFS}   &$\mbox{cm}^{-2}$&   0      \X
{\tt NEFF}    &total channel charge (fixed and
         mobile) coefficient ({\tt LEVEL}=2 only) \sym{\NEFF}&- &1 \X
{\tt PB}
      &bulk junction potential \sym{\PB}\newline
       (This is the interface potential in the channel relative to the source
       at threshold.)
& V     &   0.8       \X {\tt PHI}
     &surface inversion potential   \sym{\PHI}   & V     &   0.6       \X
{\tt RD}      &drain ohmic resistance \para\sym{\RD}&$\Omega$&   0
\X {\tt RS}      &source ohmic resistance \para\sym{\RS}  &
$\Omega$&   0\X {\tt RSH}     &drain and source diffusion
         sheet resistance \para\sym{\RSH}     & $\Omega$/square&   0      \X
{\tt THETA}   &mobility modulation ({\tt LEVEL}=3 only)
\sym{\THETA} &
                 1/V   &   0         \X
{\tt TOX}
     &oxide thickness\sym{\TOX}\newline
     Default for {\tt LEVEL} 2 and 3 is 0.1~$\mu$m.\newline
     If {\tt LEVEL} 1 and {\tt TOX} is omitted then the process oriented model
     is not used.
     & m   &   -  \X
}
\end{table}

\begin{table}[h]
\caption{MOSFET model keywords for {\tt LEVEL}s 1, 2, 3 continued.
\label{mtable1234}} \keywordtable{
%\kw{Table \ref{mtable123}continued: MOSFET Model Keywords for
%{\tt LEVEL}s 1, 2, 3.\\[-0.05in]}{
{\tt TPG}     &type of gate material: \sym{\TPG} \newline
\hspace*{\fill} 1 $\rightarrow$ polysilicon, opposite type to
substrate \newline \hspace*{\fill} $-1$ $\rightarrow$ polysilicon,
same type as substrate \newline
        \hspace*{\fill} 0 $\rightarrow$  aluminum gate           & -       & 1\X
{\tt UCRIT}   &critical field for mobility
         degradation ({\tt LEVEL}=2 only) \sym{\UCRIT}& V/cm  &   $10^4$ \X
{\tt UEXP}    &critical field exponent in
         mobility\newline degradation ({\tt LEVEL}=2 only) \sym{\UEXP}& -&0\X
{\tt UO}      & surface mobility (U-oh) \sym{\UO}
        &$\mbox{cm}^2/\mbox{V-s}$&600\X
{\tt UTRA}    &transverse field coefficient (mobility)
         ({\tt LEVEL} = 1 and 3 only)\sym{\UTRA}& -     &   0        \X
{\tt VMAX}    &maximum drift velocity of carriers\sym{\VMAX}
        & m/s   &   0    \X
{\tt VTO}     &zero-bias threshold voltage\newline
                 N-channel devices: positive for enhancement mode
         \newline \hspace*{\fill} and negative
         for depletion mode devices.\newline
                 P-channel devices: negative for enhancement mode
         \newline \hspace*{\fill} and positive
         for depletion mode devices.\newline
                 (VT-oh) \sym{\VTZERO}
         &V&0\X
{\tt XJ}      &metallurgical junction depth\sym{\XJ}    & m
&   0\X } \vspace{-0.3in} \vfill
\end{table}

\begin{table}[h]
\caption{MOSFET model keywords for {\tt LEVEL}s 1, 2, 3;
\pspice\notforsspice extensions. \label{mtable123pspice}}
\vspace*{-0.15in} \keywordtable{ {\tt JSSW}    & bulk $p$-$n$
junction sidewall current per unit length
                   \para\sym{\JSSW} & A/m   &   0 \X
{\tt L}    & channel length \sym{\Length} & m     &   {\tt DEFL}
\X {\tt N}    & bulk $p$-$n$ emission coefficient \para\sym{\N} &
- & 0      \X {\tt PBSW} & bulk $p$-$n$ sidewall potential
                \para\sym{\PBSW}&V&{\tt PB} \X
{\tt RB}   & bulk ohmic resistance \para\sym{\RB} & $\Omega$&   0
\X {\tt RG}   & gate ohmic resistance \para\sym{\RB} & $\Omega$&
0   \X {\tt RDS}  & drain-source shunt
resistance\sym{\RDS}&$\Omega$  & $\infty$   \X {\tt T\_ABS}&
\sym{T_{\ms{ABS}}}
       & $^{\circ}$C & current\newline temp.\X
\multicolumn{2}{|l|}{{\tt T\_MEASURED}}
       & $^{\circ}$C & TNOM\\
       &\sym{T_{\ms{MEASURED}}}&&\X
\multicolumn{2}{|l|}{{\tt T\_REL\_GLOBAL}}
       & $^{\circ}$C & 0\\
       &\sym{T_{\ms{REL\_GLOBAL}}}&&\X
\multicolumn{2}{|l|}{{\tt T\_REL\_LOCAL}}
       & $^{\circ}$C & 0\\
       &\sym{T_{\ms{REL\_LOCAL}}}&&\X
{\tt TT}   & bulk $p$-$n$ transit time\sym{\TT} & s     &   0
\X {\tt W}    & channel width \sym{\W} & m     &   {\tt DEFW}  \X
{\tt WD}   & lateral diffusion width\sym{\WD} & m     &   0
\X {\tt XQC}  & fraction of channel charge attributable to drain
in
             saturation region \sym{\XQC}\newline
         If $\XQC > 0.5$ the Meyer Capacitance Model is used.\newline
         If $\XQC \le 0.5$ the Ward-Dutton Capacitance Model is used.\newline
         & -&   1\X
}
\end{table}

\vfill The MOSFET {\tt LEVEL} 1,2 and 3 parameters fall into three
categories: absolute device parameters, scalable and process
parameters and geometric parameters. In most cases the absolute
device parameters can be derived from the scalable and process
parameters and the geometry parameters.   However, if specified,
the values of the device parameters are used.

The physical constants used in the model evaluation are
\begin{center}
\begin{tabular}{|l|l|l|}
\hline
$k$ & Boltzmann's constant           &  $1.3806226\,10^{-23}$~J/K\\
$q$ & electronic charge             & $1.6021918\,10^{-19}$~C\\
$\epsilon_0$& free space permittivity &  $8.854214871\,10^{-12}$~F/m\\
$\epsilon_{\ms{Si}}$& permittivity of silicon &  $11.7 \epsilon_0$\\
$\epsilon_{\ms{OX}}$& permittivity of silicon dioxide &  $3.9 \epsilon_0$\\
$n_i$&intrinsic concentration of silicon @ 300~K& $1.45\,10^{16}~\mbox{m}^{-3}$\\
\hline
\end{tabular}
\end{center}
%vbi is $F_{\ms{FB}}$\\
%wkfng is $\phi_{\ms{GATE}}$ ???\\
%wkfngs is $\phi_{\ms{MS}}$\\
%egfet is $E_G$ ??? \\
\vfill
\noindent\underline{\sl \large Standard Calculations}\\[0.1in]
Absolute temperatures (in kelvins, K) are used. The thermal
voltage
\begin{equation}
V_{\ms{TH}}(T_{\ms{NOM}}) = {{kT_{\ms{NOM}}} \over q} .
\end{equation}
\noindent The silicon bandgap energy
\begin{equation}
E_G(T_{\ms{NOM}})=1.16 - 0.000702{{4T_{\ms{NOM}}^2} \over
{T_{\ms{NOM}}+1108}} .
\end{equation}
The difference of the gate and bulk contact potentials
\begin{equation}
\phi_{\ms{MS}} = \phi_{\ms{GATE}} - \phi_{\ms{BULK}} .
\end{equation}
The gate contact potential
\begin{equation}
\phi_{\ms{GATE}} = \left\{ \begin{array}{ll}
        3.2 & \mbox{$\TPG = 0$}\\
        3.25& \mbox{NMOS \& $\TPG = 1$}\\
        3.25 + E_G     & \mbox{NMOS \& $\TPG = -1$}\\
        3.25 + E_G     & \mbox{PMOS \& $\TPG = 1$} \\
        3.25& \mbox{PMOS \& $\TPG = -1$}
        \end{array} \right. .
    %}
\end{equation}
The potential drop across the oxide
\begin{equation}
\phi_{\ms{OX}} = - {{\textstyle Q'_0 } \over { C'_{\ms{OX}}}} .
\end{equation}

\noindent The contact potential of the bulk material
\begin{equation}
\phi_{\ms{BULK}} = \left\{ \begin{array}{ll}
                          3.25 + E_G & \mbox{if NMOS}\\
                          3.25       & \mbox{if PMOS}
                          \end{array} \right. .
              %}
\end{equation}
The equivalent gate oxide interface charge per unit area
\begin{equation}
Q'_0 = q\NSS .
\end{equation}
The capacitance per unit area of the oxide is
\begin{equation}
C'_{OX} = {{\epsilon_{OX}} \over { \TOX}} .
\end{equation}
The effective length $L_{\ms{EFF}}$ of the channel is reduced by
the amount $\LD$ (= {\tt LD}) of the lateral diffusion at the
source and drain regions:
\begin{equation}
L_{\ms{EFF}} = \Length - 2\LD
\end{equation}
Similarly the effective length $W_{\ms{EFF}}$ of the channel is
reduced by the amount $\WD$ (= {\tt WD}) of the lateral diffusion
at the edges of the channel.
\begin{equation}
W_{\ms{EFF}} = \W - 2\WD
\end{equation}
$\KAPPA$ is limited: if the specified value of $\KAPPA$ is less
than or equal to zero the following parameters are set:
\begin{eqnarray}
\KAPPA &=& 0.2\\
\LAMBDA &=& 0\\
\UCRIT &=& 0\\
\UEXP &=& 0\\
\UTRA &=& 0
\end{eqnarray}
\vfill \noindent\underline{\sl \large Process Oriented Model}
\index{MOSFET, Process Oriented Model}
\\[0.1in]
If omitted, device parameters are computed from process parameters
using defaults if necessary provided that both {\tt TOX} = $\TOX$
and {\tt NSUB} = $\NSUB$ are specified.  If either {\tt TOX} or
{\tt NSUB} is not specified then the critical device parameters
must be specified.  Which parameters are critical depends on the
model {\tt LEVEL}.

If {\tt VTO} is not specified in the model statement then it is
evaluated as
\begin{equation}
{\tt VTO} = \VTO = \left\{ \begin{array}{ll}
       V_{\ms{FB}} + \GAMMA \sqrt{\PHI} + \PHI & \mbox{if NMOS} \\
       V_{\ms{FB}} - \GAMMA \sqrt{\PHI} + \PHI & \mbox{if PMOS}
      \end{array} \right. %}
\end{equation}
where
\begin{equation}
V_{\ms{FB}} = \phi_{\ms{MS}} - \phi_{\ms{OX}} \label{mFB1}
\end{equation}
is the flat-band voltage. Otherwise if {\tt VTO} is specified in
the model statement
\begin{equation}
V_{\ms{FB}} = \left\{ \begin{array}{ll}
      \VTO - \GAMMA\sqrt{\PHI} +\PHI & \mbox{if NMOS}\\
      \VTO + \GAMMA\sqrt{\PHI} +\PHI & \mbox{if PMOS} \end{array}\right. %}
\label{mFB2}
\end{equation}

\noindent If {\tt GAMMA} is not specified in the model statement
then
\begin{equation}
{\tt GAMMA} = \GAMMA = {{\sqrt{2\epsilon_{\ms{Si}} q N_B}} \over
{C'_{OX}}} \label{gammaomitted}
\end{equation}
If {\tt PHI} is not specified in the model statement then
\begin{equation}
{\tt PHI} = \PHI = 2 V_{\ms{TH}} \ln{{N_B}\over{n_i}}
\end{equation}
and is limited to 0.1 if calculated. $\NSUB$ = {\tt NSUB} as
supplied in the model statement and $n_i$ at 300~K are used. If
{\tt KP} is not specified in the model statement then
\begin{equation}
{\tt KP} = \KP = \UO C'_{OX}
\end{equation}
If {\tt UCRIT} is not specified in the model statement then
\begin{equation}
{\tt UCRIT} = \UCRIT = {{\epsilon_{Si}}\over{\textstyle\TOX}}
\end{equation}
\begin{equation}
X_d = \sqrt{{{\textstyle2\epsilon_{\ms{Si}}} \over {\textstyle q
\NSUB}}}
\end{equation}
is proportional to the depletion layer widths at the source and
rdain regions. \vshift \vfill \noindent\underline{\sl \large
Temperature Dependence} \index{MOSFET, Temperature Dependence}
\index{Temperature Dependence, see MOSFET}
\\[0.1in]
Temperature effects are incorporated as follows where $T$ and
$T_{\ms{NOM}}$ are absolute temperatures in Kelvins (K).
\begin{eqnarray}
V_{\ms{TH}} & = & {{kT} \over q}\\
\IS (T) & = & \IS e^{\left( \textstyle E_g(T) {T \over
{T_{\ms{NOM}}}}
 - E_G(T) \right) /V_{\ms{TH}}}\\
\JS (T) & = & \JS e^{\left(\textstyle E_g(T) {T \over
{T_{\ms{NOM}}}}
 - E_G(T) \right) /V_{\ms{TH}}}\\
\JSSW (T) & = & \JSSW e^{\left(\textstyle E_g(T) {T \over
{T_{\ms{NOM}}}}
 - E_G(T) \right) /V_{\ms{TH}}}\\
\PB (T) & = & \PB {T \over {T_{\ms{NOM}}}} - 3V_{\ms{TH}} \ln{
                {T \over {T_{\ms{NOM}}}} } + E_G (T_{\ms{NOM}})
                {T \over {T_{\ms{NOM}}}} -E_G(T)\\
\PBSW (T) & = & \PBSW {T \over {T_{\ms{NOM}}}}
 - 3V_{\ms{TH}} \ln{ {T \over {T_{\ms{NOM}}}} }
              + E_G (T_{\ms{NOM}}) {T \over {T_{\ms{NOM}}}} -E_G(T)\\
\PHI (T) & = & \PHI
                {T \over {T_{\ms{NOM}}}} - 3V_{\ms{TH}}
                \ln{ {T \over {T_{\ms{NOM}}}} } + E_G (T_{\ms{NOM}} -E_G(T)\\
\CBD (T) & = & \CBD \{1 + \MJ [0.0004(T-T_{\ms{NOM}})+(1-\PB (T)/\PB )]\} \\
\CBS (T) & = & \CBS \{1 + \MJ [0.0004(T-T_{\ms{NOM}})+(1-\PB (T)/\PB )]\} \\
\CJ (T) & = & \CJ \{1 + \MJ [0.0004(T-T_{\ms{NOM}})+(1-\PBSW
(T)/\PBSW )]\}
\end{eqnarray}

\begin{eqnarray}
\CJSW (T) & = & \CJSW \{1 + \MJSW [0.0004(T-T_{\ms{NOM}})+(1-\PB (T)/\PB )]\} \\
\KP (T) & = & \KP  (T_{\ms{NOM}}/T)^{3/2} \\
\UO (T) & = & \UO  (T_{\ms{NOM}}/T)^{3/2} \\
E_g(T) & = & 1.16 - 0.000702{{T^2} \over {T+1108}}\\
\sspiceonly{ \mbox{NOT IMPLEMENTED:}\nonumber\\
n_i(T) & = & 1.45\,10^{16} \left( {{T}\over
{300}}\right)^{\frac{3}{2}}
           e^{ \textstyle\left(
           {{1.16}\over {300}} - {{E_G} \over {T}} \right)}}
\end{eqnarray}\\[0.1in]
\noindent\underline{\sl \large Parasitic Resistances}\\[0.1in]
\index{Parasitic Resistances, see MOSFET} \index{MOSFET, Parasitic
Resistance} \index{Parasitic Resistance, see MOSFET}
\index{MOSFET, $R_S$} \index{MOSFET, $R_G$} \index{MOSFET, $R_D$}
\index{MOSFET, $R_B$} \index{PMOS, $R_S$} \index{PMOS, $R_G$}
\index{PMOS, $R_D$} \index{PMOS, $R_B$} \index{NMOS, $R_S$}
\index{NMOS, $R_G$} \index{NMOS, $R_D$} \index{NMOS, $R_B$} The
resistive parasitics $R_S$, $R_G$, $R_D$ and $R_B$ are treated in
the same way for the {\tt LEVEL} 1, 2 and 3 models.  They may be
specified as the absolute device parameters {\tt RS}, {\tt RG},
{\tt RD}, and {\tt RB} or calculated from the sheet resistivity
$\RSH$ (= {\tt RSH}) and area parameters $\NRS$ (= {\tt NRS}),
$\NRG$ (= {\tt NRG}), $\NRD$ (= {\tt NRD}) and $\NRB$ (= {\tt
NRB}). As always the absolute device parameters take precedence if
they are specified. Otherwise
\begin{eqnarray}
R_S & = & \NRS\RSH\\
R_G & = & \NRG\RSH\\
R_D & = & \NRD\RSH\\
R_B & = & \NRB\RSH
\end{eqnarray}
Note that neither geometry parameters nor process parameters are
required if the absolute device resistances are specified.
\notforsspice{The parasitic resistance parameter dependencies are
summarized in figure \ref{m123para}.
\begin{figure}[b]
\parbox[t]{1.3in}{
\begin{tabular}[t]{|p{1in}|}
\hline
\multicolumn{1}{|c|}{PROCESS} \\
\multicolumn{1}{|c|}{PARAMETERS} \\
\hline \hline
{\tt RSH} \hfill $\RSH$\\
\hline
\end{tabular}
} \hfill
\parbox{0.1in}{\ \vspace*{0.2in}\newline +}
\hfill
\begin{tabular}[t]{|p{1in}|}
\hline
\multicolumn{1}{|c|}{GEOMETRY} \\
\multicolumn{1}{|c|}{PARAMETERS} \\
\hline
{\tt NRS} \hfill $\NRS$\\
{\tt NRD} \hfill $\NRD$\\
{\tt NRG} \hfill $\NRG$\\
{\tt NRB} \hfill $\NRB$\\
\hline
\end{tabular}
\hfill
\parbox{0.1in}{\ \vspace*{0.2in}\newline $\rightarrow$}
\hfill
\begin{tabular}[t]{|p{1.8in}|}
\hline
\multicolumn{1}{|c|}{DEVICE} \\
\multicolumn{1}{|c|}{PARAMETERS} \\
\hline
{\tt RD} \hfill $\RD = f(\RSH, \NRD)$ \\
{\tt RS} \hfill $\RS = f(\RSH, \NRS)$ \\
{\tt RG} \hfill $\RG = f(\RSH, \NRG)$ \\
{\tt RB} \hfill $\RB = f(\RSH, \NRB)$ \\
\hline
\end{tabular}
\caption{MOSFET {\tt LEVEL} 1, 2 and 3 parasitic resistance
parameter relationships. \label{m123para}}
\end{figure}}
\noindent\underline{\sl Leakage Currents}\\[0.1in]
\index{Leakage Currents, see MOSFET} \index{MOSFET, I/V
Characteristics, Leakage Currents} \index{MOSFET, Leakage
Currents} Current flows across the normally reverse biased
source-bulk and drain-bulk junctions. The bulk-source leakage
current
\begin{equation}
I_{BS} = I_{BSS}\left(e^{(\textstyle V_{BS}/V_{\ms{TH}})}
-1\right) \label{eqn:m:ibs}
\end{equation}
where
\begin{equation}
I_{BSS} = \left\{ \begin{array}{ll}
         I_S       & \mbox{if {\tt IS} specified} \\
         A_S J_S + \PS\JSSW   & \mbox{if {\tt IS} not specified}
     \end{array} \right. %}
\end{equation}

The bulk-drain leakage current
\begin{equation}
I_{BD} = I_{BDS}\left( e^{(\textstyle V_{BD}/V_{\ms{TH}})}
-1\right) \label{eqn:m:ibd}
\end{equation}
where
\begin{equation}
I_{BDS} = \left\{ \begin{array}{ll}
         I_S       & \mbox{if {\tt IS} specified} \\
         A_D J_S   + \PS\JSSW & \mbox{if {\tt IS} not specified}
         \end{array} \right. %}
\end{equation}
\notforsspice{The parameter dependencies of the parameters
describing the leakage current in the {\tt LEVEL} 1, 2 and 3
MOSFET models are summarized in
figure \ref{mlevel123leakage}.\\[0.2in]
\begin{figure}[b]
\begin{tabular}[t]{|p{1in}|}
\hline
\multicolumn{1}{|c|}{PROCESS} \\
\multicolumn{1}{|c|}{PARAMETERS} \\
\hline \hline
{\tt JS} \hfill $\JS$\\
\hline
\end{tabular}
\hfill
\parbox{0.1in}{\ \vspace*{0.2in}\newline +}
\hfill
\begin{tabular}[t]{|p{1in}|}
\hline
\multicolumn{1}{|c|}{GEOMETRY} \\
\multicolumn{1}{|c|}{PARAMETERS} \\
\hline
{\tt AD} \hfill $A_D$\\
{\tt AS} \hfill $A_S$\\
{\tt PD} \hfill $P_D$\\
{\tt PS} \hfill $P_S$\\
\hline
\end{tabular}
\hfill
\parbox{0.1in}{\ \vspace*{0.2in}\newline $\rightarrow$}
\hfill
\begin{tabular}[t]{|p{1.8in}|}
\hline
\multicolumn{1}{|c|}{DEVICE} \\
\multicolumn{1}{|c|}{PARAMETERS} \\
\hline
{\tt IS} \hfill $\IS = f(\JS, \JSSW, A_D, A_S, P_D, P_S)$\\
\hline
\end{tabular}
\caption{MOSFET leakage current parameter dependecies.
\label{mlevel123leakage}}
\end{figure}
}

\noindent\underline{\sl \large Depletion Capacitances}\\[0.1in]
\index{Depletion capacitance, see MOSFET} \index{MOSFET, Depletion
capacitance} $C_{BS}$ and $C_{BD}$ are the depletion capacitances
at the bulk-source and bulk-drain depletion regions respectively.
These depletion capacitances are calculated and used in the same
way in all three ({\tt LEVEL} = 1, 2 and 3) models. Although they
may be specified as absolute device parameters they are strong
functions of the voltages across the junction and are complex
functions of geometry and of semiconductor doping. As such they
are usually calculated from process parameters. They are the sum
of component capacitances \index{$C_{BS}$}
\begin{equation}
C_{BS} = C_{BS,\ms{JA}} + C_{BS,\ms{SW}} + C_{BS,\ms{TT}}
\end{equation}
where the sidewall capacitance
\begin{eqnarray}
C_{BS,\ms{SW}} &=& \PS\CJSW C_{BSS} \\
C_{BSS} &=& \left\{ \begin{array}{l}
       \left( 1 - {{\textstyle V_{BS}}
       \over {\textstyle\PBSW}} \right)^{\textstyle -\MJSW} \hfill
       \mbox{for $V_{BS}  \le \FC\PB$} \\ \\
       (1 - \FC )^{\textstyle -(1+\MJSW )}
       \left(1 - \FC (1+\MJSW )
       + {{\textstyle \MJSW V_{BS}}
       \over { \textstyle \PBSW}} \right)
       \\ \hspace*{\fill} \mbox{for $V_{BS}  > \FC\PB$}
       \end{array} \right. \\ %}
\end{eqnarray}
the area capacitance
\begin{eqnarray}
C_{BS,\ms{JA}} &=&
  \left\{ \begin{array}{ll}
  \CBS C_{BSJ} &  \mbox{if {\tt CBS} (= $\CBS$)
                    is specified in the model} \\ \\
  A_S\CJ C_{BSJ} & \mbox{otherwise} \end{array} \right. \\ %}
C_{BSJ} &\hspace{-0.1in}=&\hspace{-0.1in} \left\{
\begin{array}{ll}
       \left(1 - {\textstyle {V_{BS}} \over
       {\textstyle \PB}} \right)^{\textstyle -\MJ}
       & \mbox{for $V_{BS}  \le \FC\PB$} \\ \\
       (1 -\FC)^{\textstyle -(1+\MJ)}
       \left(1 - \FC (1+\MJ)
       + {{\textstyle\MJ V_{BS}}
       \over {\textstyle \PB}} \right)
       & \mbox{for $V_{BS}  > \FC\PB$}
       \end{array} \right. %}
\end{eqnarray}
and the transit time capacitance
\begin{equation}
C_{BS,\ms{TT}} = \TT G_{BS}
\end{equation}
where the bulk-source conductance $G_{BS} = \partial I_{BS} /
\partial V_{BS}$ and $I_{BS}$ is defined in (\ref{eqn:m:ibs}).
\index{$C_{BD}$}
\begin{equation}
C_{BD} = C_{BD,\ms{JA}} + C_{BD,\ms{SW}}
\end{equation}
where the sidewall capacitance
\begin{eqnarray}
C_{BD,\ms{SW}} &=& \PD\CJSW C_{BDS} \\
C_{BDS} &=& \left\{ \begin{array}{l}
       \left( 1 - {{\textstyle V_{BD}}
       \over {\textstyle\PBSW}} \right)^{\textstyle -\MJSW}
       \hfill \mbox{for $V_{BD}  \le \FC\PB$} \\ \\
       (1 - \FC )^{\textstyle -(1+\MJSW )}
       \left(1 - \FC (1+\MJSW )
       + {{\textstyle \MJSW V_{BD}}
       \over { \textstyle \PBSW}} \right)
       \\ \hspace*{\fill} \mbox{for $V_{BD}  > \FC\PB$}
       \end{array} \right. \\ %}
\end{eqnarray}

the area capacitance
\begin{eqnarray}
C_{BD,\ms{JA}} &=&
  \left\{ \begin{array}{ll}
  \CBD C_{BDJ} &  \mbox{if {\tt CBD} (= $\CBD$)
                    is specified in the model} \\ \\
  A_D\CJ C_{BDJ} & \mbox{otherwise} \end{array} \right. \\ %}
C_{BDJ} &=& \left\{ \begin{array}{ll}
       \left(1 - {\textstyle {V_{BD}} \over
       {\textstyle \PB}} \right)^{\textstyle -\MJ}
       & \mbox{for $V_{BS}  \le \FC\PB$} \\ \\
       (1 -\FC)^{\textstyle -(1+\MJ)}
       \left(1 - \FC (1+\MJ)
       + {{\textstyle\MJ V_{BD}} \over {\textstyle \PB}} \right)
       & \mbox{for $V_{BD}  > \FC\PB$}
       \end{array} \right. %}
\end{eqnarray}
and the transit time capacitance
\begin{equation}
C_{BS,\ms{TT}} = \TT G_{BS}
\end{equation}
where the bulk-source conductance $G_{BD} = \partial I_{BD} /
\partial V_{BD}$ and $I_{BD}$ is defined in (\ref{eqn:m:ibd}).

In the {\tt LEVEL} 1 MOSFET model the depletion capacitances are
piecewise linear.  They are calculated at the current operating
point and then treated as linear. In the {\tt LEVEL} 2 and 3
models they are treated as nonlinear. \notforsspice{The depletion
capacitance parameter dependencies are summarized in
figure \ref{level123depletionc}.\\[0.1in]
\begin{figure}[b]
\parbox[t]{1.3in}{
\begin{tabular}[t]{|p{1in}|}
\hline
\multicolumn{1}{|c|}{PROCESS} \\
\multicolumn{1}{|c|}{PARAMETERS} \\
\hline \hline
{\tt CJ} \hfill $\CJ$\\
{\tt CJSW} \hfill $\CJSW$\\
{\tt MJ} \hfill $\MJ$\\
{\tt MJSW} \hfill $\MJSW$\\
{\tt PB} \hfill $\PB$\\
{\tt PBSW} \hfill $\PBSW$\\
{\tt FC} \hfill $\FC$\\
\hline
\end{tabular}
} \hfill
\parbox{0.1in}{\ \vspace*{0.2in}\newline +}
\hfill
\begin{tabular}[t]{|p{1in}|}
\hline
\multicolumn{1}{|c|}{GEOMETRY} \\
\multicolumn{1}{|c|}{PARAMETERS} \\
\hline
{\tt AD} \hfill $A_D$\\
{\tt AS} \hfill $A_S$\\
{\tt PD} \hfill $P_D$\\
{\tt PS} \hfill $P_S$\\
\hline
\end{tabular}
\hfill
\parbox{0.1in}{\ \vspace*{0.2in}\newline $\rightarrow$}
\hfill
\begin{tabular}[t]{|p{1.8in}|}
\hline
\multicolumn{1}{|c|}{DEVICE} \\
\multicolumn{1}{|c|}{PARAMETERS} \\
\hline
{\tt CBD} \hfill $\CBD = f(\CJ, A_D)$\\
{\tt CBS} \hfill $\CBS = f(\CJ, A_S)$\\
\{$C_{BS} = f( \PS, \CBS, \CJSW, \TT$\hspace*{\fill}\\
\hspace*{\fill}$ \MJ, \MJSW,
\PB, \PBSW, \FC)$\}\\
\{$C_{BD} = f( \PD, \CBD, \CJSW, \TT$\hspace*{\fill}\\
\hspace*{\fill}$ \MJ, \MJSW,
\PB, \PBSW, \FC)$\}\\
\hline
\end{tabular}
\caption[MOSFET {\tt LEVEL} 1, 2 and 3 depletion capacitance
parameter relationships]{MOSFET {\tt LEVEL} 1, 2 and 3 junction
depletion capacitance parameter relationships.
\label{level123depletionc}}
\end{figure}
}
\noindent\underline{\sl {\tt LEVEL 1} I/V Characteristics}\\[0.1in]
\index{I/V Characteristics, see MOSFET} \index{MOSFET, I/V
Characteristics} \index{NMOS, I/V Characteristics} \index{PMOS,
I/V Characteristics} \index{I-V characteristics, see MOSFET}

For the {\tt LEVEL} 1 model the device parameters (other than
capacitances and resistances) are evaluated using $\TOX$ ({\tt
TOX}), $\UO$ ({\tt UO}), $\NSS$ ({\tt NSS} and $\NSUB$ ({\tt
NSUB}) if they are not specified in the {\tt .MODEL} statement.

The {\tt LEVEL 1} current/voltage characteristics are evaluated
after first determining the mode (normal: $V_{DS} \ge 0$ or
inverted: $V_{DS} < 0$) and the region (cutoff, linear or
saturation) of the current
$(V_{DS}, V_{GS})$ operating point.\\[0.1in]

\noindent{\sl Normal Mode: ($V_{DS} \ge 0$)}\\[0.2in]
The regions are as follows:\\[0.1in]
\hspace*{\fill}\offsetparbox{
\begin{tabular}{ll}
cutoff region:&$V_{GS} < V_T$\\
linear region:&$V_{GS} > V_T$ and $V_{DS} < V_{GS}-V_{T}$\\
saturation region:&$V_{GS} > V_T$ and $V_{DS} > V_{GS}-V_{T}$\\
\end{tabular}}\\[0.1in]
where the threshold voltage
\begin{equation}
V_{T} = \left\{ \begin{array}{ll}
          V_{\ms{FB}} + \PHI + \GAMMA \sqrt{\PHI - V_{BS}}
               & V_{BS} \ge \PHI \\
          V_{\ms{FB}} + \PHI & V_{BS} < \PHI
          \end{array} \right. %}
\end{equation}
%\sspiceonly{
%This expression for $V_T$ is not that used in superspice.
%Instead
%\[
%V_T = V_{\ms{FB}} + \GAMMA\left(\PHI - V_{BS}/(2\sqrt{\PHI}\right)
%\]
%see mosfet.c .
%}

Then
\begin{equation}
I_{D} = \left\{ \begin{array}{ll}
      0  & \mbox{cutoff region} \\ \\
      { {\textstyle W_{\ms{EFF}}} \over {\textstyle L_{\ms{EFF}}}}
      {{\textstyle \KP} \over {\textstyle 2}}
      ( 1+\LAMBDA V_{DS})
      V_{DS} \left[2(V_{GS}-V_{T})-V_{DS}\right]
         &\mbox{linear region} \\ \\
      { {\textstyle W_{\ms{EFF}}} \over {\textstyle L_{\ms{EFF}}}}
      {{\textstyle \KP} \over {\textstyle 2}}
      (1+\LAMBDA V_{DS})
      \left[V_{GS}-V_{T}\right]^2
         &\mbox{saturation region} \end{array} \right. %}
      \label{mlevel1ids}
\end{equation}
\noindent{\sl Inverted Mode: ($V_{DS} < 0)$}\\[0.2in]
In the inverted mode the MOSFET I/V characteristics are evaluated
as in the normal mode (\ref{mlevel1ids}) but with the drain and
source subscripts interchanged. \notforsspice{The relationships of
the parameters describing the I/V characteristics for the {\tt
LEVEL} 1 model are summarized in figure \ref{mlevel1i/v}.\\[0.1in]
\begin{figure}[b]
\begin{tabular}[t]{|p{1in}|}
\hline
\multicolumn{1}{|c|}{PROCESS} \\
\multicolumn{1}{|c|}{PARAMETERS} \\
\hline \hline
{\tt NSUB} \hfill $\NSUB$\\
{\tt TOX} \hfill $\TOX$\\
{\tt NSS} \hfill $\NSS$\\
{\tt UO} \hfill $\UO$\\
\hline
\end{tabular}
\hfill
\parbox{0.1in}{\ \vspace*{0.2in}\newline +}
\hfill
\begin{tabular}[t]{|p{1in}|}
\hline
\multicolumn{1}{|c|}{GEOMETRY} \\
\multicolumn{1}{|c|}{PARAMETERS} \\
\hline
\hspace*{\fill} -- \hspace*{\fill} \\
\hline \hline
\multicolumn{1}{|c|}{Required} \\
\hline
{\tt L} \hfill $L$\\
{\tt W} \hfill $W$ \\
\hline \hline
\multicolumn{1}{|c|}{Optional} \\
\hline
{\tt LD} \hfill $\LD$\\
{\tt WD} \hfill $\WD$ \\
\hline
\end{tabular}
\hfill
\parbox{0.1in}{\ \vspace*{0.2in}\newline $\rightarrow$}
\hfill
\begin{tabular}[t]{|p{1.8in}|}
\hline
\multicolumn{1}{|c|}{DEVICE} \\
\multicolumn{1}{|c|}{PARAMETERS} \\
\hline
{\tt VTO} \newline \hspace*{\fill} $\VTZERO = f(\PHI, \NSS, \TOX, \GAMMA)$\\
{\tt KP} \hfill $\KP = f(\UO, \TOX)$\\
{\tt LAMBDA} \hfill $\LAMBDA$\\
{\tt PHI} \hfill $\PHI = f(\NSUB)$\\
{\tt GAMMA} \hfill $\GAMMA = f(\TOX, \NSUB)$\\
\{$I_{D} = f(\W, \Length, \WD,
\LD$\hspace*{\fill}\newline\hspace*{\fill}
$\VTO, \KP, \LAMBDA, \PHI, \GAMMA)$\}\\
\hline
\end{tabular}
\caption{LEVEL 1 I/V dependencies. \label{mlevel1i/v}}
\end{figure}
}
\noindent\underline{\sl \large {\tt LEVEL} 1 Overlap Capacitances}\\[0.1in]
\index{Overlap Capacitance, see MOSFET} \index{MOSFET, Overlap
Capacitance LEVEL 1} \index{MOSFET, $C_{GB}$} \index{MOSFET,
$C_{GD}$} \index{MOSFET, $C_{GS}$}. In the {\tt LEVEL} 1 model the
gate overlap capacitances $C_{GS}$, $C_{GD}$ and $C_{GB}$ are
constant and are calculated using the per unit width overlap
capacitances $\CGSO$ ({\tt CGSO}), $\CGDO$ ({\tt CGDO}) and
$\CGBO$ ({\tt CGBO}):
\begin{eqnarray}
C_{GS}&=&\CGSO\W \\
C_{GD}&=&\CGDO\W \\
C_{GB}&=&\CGBO\W
\end{eqnarray}
\notforsspice{The overlap capacitance parameter dependencies are
summarized in figure \ref{mlevel1overlap}.

\begin{figure}[hb]
\parbox[t]{1.3in}{
\begin{tabular}[t]{|p{1in}|}
\hline
\multicolumn{1}{|c|}{PROCESS} \\
\multicolumn{1}{|c|}{PARAMETERS} \\
\hline \hline
\hspace*{\fill} -- \hspace*{\fill}\\
\hline
\end{tabular}
} \hfill
\parbox{0.1in}{\ \vspace*{0.2in}\newline +}
\hfill
\begin{tabular}[t]{|p{1in}|}
\hline
\multicolumn{1}{|c|}{GEOMETRY} \\
\multicolumn{1}{|c|}{PARAMETERS} \\
\hline \hline
\multicolumn{1}{|c|}{Required} \\
\hline
{\tt W} \hfill $W$ \\
\hline \hline
\multicolumn{1}{|c|}{Optional} \\
\hline
{\tt WD} \hfill $\WD$ \\
\hline
\end{tabular}
\hfill
\parbox{0.1in}{\ \vspace*{0.2in}\newline $\rightarrow$}
\hfill
\begin{tabular}[t]{|p{1.8in}|}
\hline
\multicolumn{1}{|c|}{DEVICE} \\
\multicolumn{1}{|c|}{PARAMETERS} \\
\hline \hline
\multicolumn{1}{|c|}{Overlap Capacitances}\\
\hline
{\tt CGSO} \hfill $\CGSO$\\
{\tt CGDO} \hfill $\CGDO$\\
{\tt CGBO} \hfill $\CGBO$\\
\hspace*{\fill}\{$C_{GS} = f(\CGSO, \W, \WD)\}$\\
\hspace*{\fill}\{$C_{GD} = f(\CGDO, \W, \WD)\}$\\
\hspace*{\fill}\{$C_{GB} = f(\CGBO, \W, \WD)\}$\\
\hline
\end{tabular}
\caption{ MOSFET {\tt LEVEL} 1 overlap capacitance parameter
relationships. \label{mlevel1overlap}}
\end{figure}}
\vshift \notforsspice{

\noindent\underline{\sl {\tt LEVEL 2} I/V Characteristics}
\index{I/V Characteristics, see MOSFET MOSFET} \index{MOSFET, I/V
Characteristics LEVEL 2} \index{NMOS, I/V Characteristics LEVEL 2}
\index{PMOS, I/V Characteristics LEVEL 2} \index{MOSFET, LEVEL 2
I/V Characteristics} \index{NMOS, LEVEL 2 I/V Characteristics}
\index{PMOS, LEVEL 2 I/V Characteristics}} \sspiceonly{
\noindent\underline{\sl {\tt LEVEL 2} Characteristics}}
\\[0.1in]
The {\tt LEVEL} 2 I/V characteristics are based on empirical fits
resulting in a more accurate description of the I/V response than
obtained with the {\tt LEVEL} 1 model. \sspiceonly{The {\tt LEVEL}
2 model has largely been superceeded by the {\tt LEVEL} 3 model.}
\notforsspice{ The {\tt LEVEL 2} current/voltage characteristics
are evaluated after first determining the mode (normal: $V_{DS}
\ge 0$ or inverted: $V_{DS} < 0$) and the region (cutoff, linear
or saturation) of the current $(V_{DS}, V_{GS})$ operating point.

\noindent{\sl Normal Mode: ($V_{DS} \ge 0$)}\\[0.2in]
The regions are as follows:\\[0.1in]

\hspace*{\fill}\offsetparbox{
\begin{tabular}{ll}
cutoff region:&$V_{GS}<V_{T}$\\
weak inversion region:&$V_T < V_{GS} \le V_{\ms{ON}}$ \\
linear region (strong inversion):&$V_{GS} > V_{\ms{ON}}$
                  and $V_{DS} < V_{DS,\ms{SAT}}$ \\
saturation region (strong inversion):&$V_{GS} > V_{\ms{ON}}$
                  and $V_{DS} > V_{DS,\ms{SAT}}$
\end{tabular}}\\
where
\begin{equation}
V_T = V'_{\ms{FB}} + \gamma_{\ms{EFF}}X_S
\end{equation}
\begin{equation}
V_{\ms{ON}} = \left\{ \begin{array}{ll}
     V_T             & \NFS = 0\\
     V_T + V_{\ms{TH}}x_n & \NFS \ne 0
     \end{array} \right. %}
\end{equation}
\begin{equation}
%sarg=
X_S = \left\{ \begin{array}{ll}
     {{\textstyle\sqrt{\PHI}}\over{\textstyle[1 + \frac{1}{2}V_{BS}/(\PHI)]}}
         & V_{BS} > 0\\
     \\
     \sqrt{\PHI-V_{BS}} & V_{BS} \le 0\\
     \end{array} \right. %}
     \label{xs:eqn}
\end{equation}
where
\begin{equation}
x_n      = 1 + F_N -\gamma_{\ms{EFF}}X_1 -X_2X_S
   + W_{\ms{EFF}}L_{\ms{EFF}} {{q \NFS} \over {C'_{OX}}}
\end{equation}
\begin{equation}
\eta = 1 + F_N
\end{equation}
the effect of channel width on threshold voltage is modeled by
\begin{equation}
V'_{\ms{FB}}= V_{\ms{FB}} + F_N(\PHI -V_{BS}) \label{eqn:vdashfb}
\end{equation}
and the flat band voltage, $V_{\ms{FB}}$, is calculated using
(\ref{mFB1}) or (\ref{mFB2}).
\begin{equation}
V_{GST} = V_{GS} - V_{ON}
\end{equation}
The factor describing the effect of channel width on threshold is
\begin{equation}
F_N={{\textstyle\epsilon_s\delta\pi} \over {\textstyle 4 C'_{OX}
\W_{\ms{EFF}} }}
\end{equation}
The effective bulk threshold parameter is affected by charge in
the drain and source depletion regions.  This is important for
short channels. The factor describing short channel effects is
\begin{equation}
\gamma_{\ms{EFF}} = \left\{ \begin{array}{ll}
     \GAMMA               & \GAMMA \le 0 \mbox{ or } \NSUB \le 0\\
     \GAMMA(1-F_{DD}-F_{SD}) & \GAMMA > 0 \mbox{ and } \NSUB > 0
     \end{array} \right. %}
\end{equation}
where the effect of depletion charge at the drain is described by
\begin{equation}
F_{DD} = \left\{ \begin{array}{ll}
         \frac{1}{2}\left( \sqrt{1+2X_DX_B}-1\right)
         & V_{DS} \le V_{DS,\ms{SAT}} \\ \\
         \frac{1}{2}\left( \sqrt{1+2X_DX_{B,\ms{SAT}}}-1\right)
         {{\textstyle\XJ}\over{L_{\ms{EFF}}}}
         & V_{DS} > V_{DS,\ms{SAT}} \\
         \end{array} \right. %}
\end{equation}

the effect of depletion charge at the source is described by
\begin{equation}
F_{SD} = \frac{1}{2}\left( \sqrt{1+2X_DX_S}-1\right)
         {{\textstyle\XJ}\over{L_{\ms{EFF}}}}
\end{equation}
and
\begin{equation}
%barg=
X_B = \left\{ \begin{array}{ll}
{{\textstyle\sqrt{\PHI}}\over{\textstyle[1+
\frac{1}{2}(V_{BS}-V_{DS})/(\PHI)]}}
         & V_{DS} <V_{BS}\\
     \\
     \sqrt{\PHI+V_{DS}-V_{BS}} & V_{DS} \le V_{BS}\\
     \end{array} \right. %}
\end{equation}
and for saturation
\begin{equation}
X_{B,\ms{SAT}} = \left\{ \begin{array}{ll}
{{\textstyle\sqrt{\PHI}}\over{\textstyle[1+
         \frac{1}{2}(V_{BS}-V_{DS,\ms{SAT}})/(\PHI)]}}
         & V_{DS,\ms{SAT}} <V_{BS}\\
     \\
     \sqrt{\PHI+V_{DS,\ms{SAT}}-V_{BS}} & V_{DS,\ms{SAT}} \ge  V_{BS}\\
     \end{array} \right. %}
\end{equation}
$X_S$ is evaluated using (\ref{xs:eqn}) and $X_D$ using
(\ref{m3:xd:eqn}).
\begin{equation}
%dsarg=
X_1 = \left\{ \begin{array}{ll}
           {{\textstyle -X_S^2}\over{\textstyle 2(\PHI)^{(3/2)} }}
         & V_{BS} > 0 \\
     \\
           -{{\textstyle 1}\over{\textstyle 2X_S}}
         & V_{BS} \le 0
     \end{array} \right. %}
\end{equation}
and
\begin{equation}
X_2 = \left\{ \begin{array}{ll}
           -\GAMMA {{\textstyle 1}\over{\textstyle 2}}
           {{\textstyle X_DX_1}\over{\textstyle L_{\ms{EFF}} X_S}}
         & \XJ  > 0 \\
     \\
           0 & \XJ \le 0
     \end{array} \right. %}
\end{equation}
The effective mobility due to modulation by the gate
\begin{equation}
\mu_{\ms{EFF}} = \left\{ \begin{array}{ll}
               \UO\left( {{\textstyle\UCRIT}\over
                {\textstyle V_{GS}-V_{\ms{ON}}}} \right)^{\UEXP}
               & C_{\ms{OX}} \ne 0\mbox{ and } (V_{GS}-V_{ON}) > \UCRIT\\
                 \\
                 \UO
               & C_{\ms{OX}} = 0\mbox{ or } (V_{GS}-V_{ON}) \le \UCRIT
               \end{array} \right. %}
\end{equation}
and the factor describing this effect
\begin{equation}
    F_G = {{\mu_{\ms{EFF}}}\over{\UO}}
\end{equation}
The channel shortening factor
\begin{equation}
    F_D = {{L_{\ms{EFF}}}\over{L'_{EFF}}}
\end{equation}
where the effective channel length due to channel shortening is
\begin{equation}
L'_{\ms{EFF}} = \left\{ \begin{array}{ll}
    {{\textstyle X_{WB}}\over
    {\textstyle 1 +(\Delta_L - L_{\ms{EFF}} + X_{WB})/X_{WB}}}
    & (1-\lambda V_{DS})L_{\ms{EFF}} < X_{WB} \\ \\
    (1-\lambda V_{DS})L_{\ms{EFF}}
    & (1-\lambda V_{DS})L_{\ms{EFF}} \ge X_{WB}
    \end{array} \right. %}
\label{m2leff}
\end{equation}

The expression for $L'_{EFF}$ when $(1-\lambda
V_{DS})L_{\ms{EFF}}$ $<$ $X_{WB}$ limits chanel shortening at
punch-through. In (\ref{m2leff})
\begin{equation}
\Delta_L = \lambda V_{DS} L_{\ms{EFF}}
\end{equation}
and the distance that the depletion region at the drain extends
into the channel is
\begin{equation}
X_{WB} = \left\{ \begin{array}{ll}
         X_D\sqrt{\PB}         &   \NSUB \ne 0\\
         0.25\,10^{-6}         &   \NSUB  =  0
    \end{array} \right. %}
\end{equation}
The pinch-off voltage
\begin{equation}
V_P = \left\{ \begin{array}{ll}
      MAX((V_{GSX}-V'_{\ms{FB}}),0)    &  {{\textstyle \gamma_{\ms{EFF}}}
                                          \over{\textstyle \eta}} \le 0\\
  \\
      V_T                              &  {{\textstyle \gamma_{\ms{EFF}}}
                                          \over{\textstyle \eta}} > 0
                                          \mbox{ and } X_V \le 0 \\
  \\
      V_T                              &  {{\textstyle \gamma_{\ms{EFF}}}
                                          \over{\textstyle \eta}} > 0
                                          \mbox{ and } X_3 < 0 \\
  \\
     V'_{\ms{FB}}+\eta V_{DS}+\sqrt{X_3}&  {{\textstyle \gamma_{\ms{EFF}}}
                                          \over{\textstyle \eta}} > 0
                                          \mbox{ and } X_3 \ge 0
                                          \mbox{ and } X_V > 0
      \end{array} \right. %}
\end{equation}
where
\begin{equation}
X_3 = (V_{DS} + \PHI - V_{BS})
      \left({{\textstyle\gamma_{\ms{EFF}}}\over{\textstyle\eta}}\right)^2
\end{equation}
The drain-source saturation voltage
\begin{equation}
V_{DS,\ms{SAT}} = \left\{ \begin{array}{ll}
       V_P \mbox{ (Grove Frohman approximation)}  & \VMAX \le 0\\
       \\
       V_{DS,\ms{SAT}} \mbox{ from Baum's theory of}      & \VMAX > 0\\
       \mbox{~~~~~~~~~velocity saturation}&
       \end{array} \right. %}
\label{eqn:vdssat:l2}
\end{equation}
In Baum's theory of velocity saturation the saturation voltage,
$V_{DS,\ms{SAT}}$, is the solution of the quartic equation
\begin{equation}
aV^4_{DS,\ms{SAT}} + bV^3_{DS,\ms{SAT}} + cV^2_{DS,\ms{SAT}} +
dV_{DS,\ms{SAT}} = 0
\end{equation}
where
\begin{eqnarray}
a & = & {{\textstyle 3}\over {\textstyle 4}} {{\gamma_{\ms{EFF}}}\over{\eta}}\\
b & = & -2(V_1+V_X)\\
c & = & -2{{\gamma_{\ms{EFF}}}\over{\eta}}V_X\\
d & = & 2V_1(V_2+V_X) - V_2^2 -
        {{\textstyle 3}\over {\textstyle 4}} {{\gamma_{\ms{EFF}}}\over{\eta}}
        X_S^3\\
V_1 & = & {{(V_{GSX}-V'_{FB})}\over{\eta}} + \PHI - V_{BS}\\
V_2 & = & \PHI - V_{BS}\\
V_X & = & {{\VMAX  L_{\ms{EFF}}}\over{\mu_{\ms{EFF}}}}
\end{eqnarray}

The body effect factor is
\begin{equation}
F_B = X^3_B - X^3_S
\end{equation}
and the body effect factor in saturation is
\begin{equation}
F_{B,\ms{EFF}} = X^3_{B,\ms{SAT}} - X^3_S
\end{equation}
\underline{cutoff region}
\begin{equation}
I_{D} = 0
\end{equation}\\[0.2in]
%
%
\noindent\underline{linear region} \index{MOSFET, LEVEL 2 linear
region}
\begin{equation}
 I_{D} =  \KP {{W_{\ms{EFF}}}\over{ L_{\ms{EFF}}}} F_G F_D
           \left[ \left( V_{GS} - V'_{\ms{FB}}
           - {{\textstyle \eta}\over{\textstyle 2}} V_{DS}
       \right) V_{DS}
           - {{\textstyle 3}\over{\textstyle 2}}
           {{\textstyle\gamma_{\ms{EFF}}}\over{\textstyle \eta}} F_B
           \right]
      \label{mos2linearids}
\end{equation}
%
%
\underline{weak inversion region}
\\[0.1in]
When $V_{GS}$ is slightly above $V_T$, $I_{D}$ increases slowly
over a few thermal voltages $V_{\ms{TH}}$ in exponential manner
becoming $I_{D}$ calculated for strong inversion. This effect is
handled empirically by defining two exponential which, as well as
ensuring an exponential increase in $I_{D}$, also ensure that the
transconductance $G_M$ (= $\partial I_{D}/\partial V_{GS}$) is
continuous at $V_{GS}$ = $V_{\ms{ON}}$.
\begin{equation}
I_{D} = \left\{ \begin{array}{ll}
         I_{D,\ms{ON}} \left[ {{10}\over{11}} e^{\textstyle
                     (V_{GS}-V_{\ms{ON}})/(x_nV_{TH})}
       + {{1}\over{11}} e^{\textstyle \alpha (V_{GS}-V_{\ms{ON}})} \right]
          &  \alpha > 0 \\
         I_{D,\ms{ON}} e^{\textstyle (V_{GS}-V_{\ms{ON}})/(x_nV_{TH})}
         &  \alpha \le 0
         \end{array} \right. %}
\end{equation}
where
\begin{equation}
\alpha=11\left({{\textstyle G_{M,\ms{ON}}}\over {\textstyle
I_{D,\ms{ON}}}}
       -{{\textstyle 1}\over {\textstyle x_nV_{\ms{TH}}}}\right)
\end{equation}
\begin{equation}
G_{M,\ms{ON}} = {{\partial I_{D,\ms{ON}}} \over {\partial V_{GS}}}
\end{equation}
\begin{equation}
I_{D,\ms{ON}} =  \left\{ \begin{array}{ll}
                  \parbox{3in}{$I_{D}$ in (\ref{mos2linearids})
                               with $V_{GS}$ = $V_{ON}$}
                  & V_{DS} \le V_{DS,\ms{SAT}} \\ \\
                  \parbox{3in}{$I_{D}$ in (\ref{mos2satids})
                               with $V_{GS}$ = $V_{ON}$ and
                               $V_{DS}$ = $V_{DS,\ms{SAT}}$}
                  & V_{DS} \le V_{DS,\ms{SAT}}
                  \end{array} \right. %}
\end{equation}
%
%
\underline{saturation region} \index{MOSFET, LEVEL 2 saturation
region}
\begin{equation}
I_{D} =   {{L_{\ms{EFF}}}\over{\textstyle L_{\ms{EFF}-\Delta_L}}}
           I_{D,\ms{SAT}}
      \label{mos2satids}
\end{equation}
\begin{equation}
I_{D,\ms{SAT}}  =   \KP {{\W_{\ms{EFF}}}\over{ L_{\ms{EFF}}}} F_G
F_D
           \left[ \left( V_{GS} - V'_{\ms{FB}}
            -{{\eta}\over 2} V_{DS,\ms{SAT}}\right)
       V_{DS,\ms{SAT}}
           -{{\textstyle 3}\over{\textstyle 2}}
           {{\textstyle \gamma_{\ms{EFF}}}\over{\textstyle \eta }}
           F_{B,\ms{SAT}}
           \right]
\end{equation}
\notforsspice{ The {\tt LEVEL} 3 current-voltage parameter
dependencies are summarized in
figure \ref{mlevel2iv}.\\[0.2in]
\begin{figure}[hb]
\begin{tabular}[t]{|p{1in}|}
\hline
\multicolumn{1}{|c|}{PROCESS} \\
\multicolumn{1}{|c|}{PARAMETERS} \\
\hline \hline
{\tt DELTA} \hfill $\DELTA$\\
{\tt LAMBDA} \hfill $\LAMBDA$\\
{\tt NSS} \hfill $\NSS$\\
{\tt NSUB} \hfill $\NSUB$\\
{\tt PB} \hfill $\PB$\\
{\tt TOX} \hfill $\TOX$\\
{\tt UCRIT} \hfill $\UCRIT$\\
{\tt UEXP} \hfill $\UEXP$\\
{\tt VMAX} \hfill $\VMAX$\\
{\tt UO} \hfill $\UO$\\
{\tt XJ} \hfill $\XJ$\\
\hline
\end{tabular}
\hfill
\parbox{0.1in}{\ \vspace*{0.2in}\newline +}
\hfill
\begin{tabular}[t]{|p{1in}|}
\hline
\multicolumn{1}{|c|}{GEOMETRY} \\
\multicolumn{1}{|c|}{PARAMETERS} \\
\hline \hline
\multicolumn{1}{|c|}{Required} \\
\hline
{\tt L} \hfill $L$\\
{\tt W} \hfill $W$ \\
\hline \hline
\multicolumn{1}{|c|}{Optional} \\
\hline
{\tt LD} \hfill $\LD$\\
{\tt WD} \hfill $\WD$ \\
\hline
\end{tabular}
\hfill
\parbox{0.1in}{\ \vspace*{0.2in}\newline $\rightarrow$}
\hfill
\begin{tabular}[t]{|p{1.8in}|}
\hline
\multicolumn{1}{|c|}{DEVICE} \\
\multicolumn{1}{|c|}{PARAMETERS} \\
\hline \hline
\multicolumn{1}{|c|}{I/V}\\
\hline
{\tt VTO} \hfill $\VTZERO = f(\PHI, \NSS, \TOX)$\\
{\tt KP} \hfill $\KP = f(\UO, \TOX)$\\
{\tt PHI} \hfill $\PHI = f(\NSUB)$\\
{\tt GAMMA} \hfill $\GAMMA = f(\UO, \TOX, \NSUB)$\\
\hspace*{\fill}\{$I_{D} = f(\W, \Length, \WD, \LD,$\\
\hspace*{\fill}$\KP, \PHI, \UO, \GAMMA, \VTO, \PB$\\
\hspace*{\fill}$\VMAX, \XJ, \LAMBDA, \UEXP, \UCRIT, \DELTA)$\}\\
\hline
\end{tabular}
\caption{ MOSFET {\tt LEVEL} 2 I/V parameter relationships.
\label{mlevel2iv}}
\end{figure}}
\vshift
\noindent\underline{\sl \large {\tt LEVEL} 2 Overlap Capacitances}\\[0.1in]
\index{Overlap Capacitances, see MOSFET} \index{MOSFET, Overlap
Capacitances LEVEL 2} \index{MOSFET, Overlap Capacitances}
\index{MOSFET, $C_{GS}$ $C_{GD}$ $C_{GB}$} In the {\tt LEVEL 2}
model the gate overlap capacitances asre strong functions of
voltage. Two overlap capacitance models are available in \pspice
the Meyer model based on the model originally proposed by Meyer
\cite{meyer:71} and the Ward-Dutton model
\cite{ward:dutton:78,oh:ward:80}. \spicetwo and \spicethree use
just the Meyer model. The Meyer and Ward-Dutton models differ in
the derivation of the channel charge.
\\[0.2in]
\noindent\underline{\sl \large {\tt LEVEL} 2 Meyer Model}\\[0.1in]
\index{Meyer model, see MOSFET} \index{MOSFET, overlap
capacitances Meyer model} \index{MOSFET, Meyer model}
\index{MOSFET, Meyer model, LEVEL 2} \index{MOSFET, $C_{GS}$}
\index{MOSFET, $C_{GD}$} \index{MOSFET, $C_{GB}$} This model is
selected when the parameter {\tt XQC} = $\XQC$ is not specified or
$\XQC < 0.5$.

The voltage dependent thin-oxide capacitances are used only if
$\TOX$ is specified in the model statement.

Four operating regions are defined in the Meyer model:\\[0.1in]
\hspace*{\fill}\offsetparbox{
\begin{tabular}{ll}
accumulation region:&$V_{GS} < V_{\ms{ON}} - \PHI$\\
depletion region:   &$V_{\ms{ON}} - \PHI < V_{GS} < V_{\ms{ON}}$\\
saturation region:  &$V_{\ms{ON}} < V_{GS} <V_{\ms{ON}} + V_{DS}$\\
linear region:&$V_{GS} > V_{\ms{ON}}+V_{DS}$\\
\end{tabular}}\\[0.1in]
where
\begin{eqnarray}
V_{\ms{ON}}  & = & \left\{ \begin{array}{ll}
            V_{T} +  x_nV_{\ms{TH}} &
              \mbox{if } \NFS = {\tt NFS} \mbox{ specified}\\
            V_{T}                   & \mbox{if } \NFS = {\tt NFS}
                                       \mbox{ not specified}
            \end{array} \right.\\ %}
V_{T} & = & V_{T0} + \GAMMA
        \left[\sqrt{\PHI - V_{BS}}-\sqrt{\PHI} \right]\\
x_n      & = & 1 + {{q \NFS} \over {C'_{OX}}} + {{C_D} \over {C'_{OX}}}\\
C'_{OX}&=&{{\epsilon_{OX}} \over { \TOX}}\\
C_{OX}&=& C'_{OX}\W_{\ms{EFF}} L_{\ms{EFF}}\\
C_D & = & {{\textstyle\GAMMA} \over {\textstyle 2 \sqrt{\PHI -
V_{BS}}}}
\end{eqnarray}
\begin{equation}
C_{GS} =  \left\{ \begin{array}{ll}
            \CGSO\W                     & \mbox{accumulation region}\\
            \frac{2}{3} C_{OX}
             \left( 1+ {{\textstyle V_{\ms{ON}}
             - V_{GS}} \over {\textstyle\PHI}}\right)
            + \CGSO\W_{\ms{EFF}}                   & \mbox{depletion region}\\
           \frac{2}{3} C_{OX} + \CGSO\W_{\ms{EFF}}  & \mbox{saturation region}\\
            C_{OX} \left\{ 1 - \left[
              {{\textstyle V_{GS} - V_{DS} - V_{\ms{ON}}} \over
               {\textstyle 2(V_{GS}-V_{\ms{ON}}) - V_{DS}}} \right]^2\right\}
            + \CGSO\W_{\ms{EFF}}  & \mbox{linear region}

            \end{array}
            \right. %}
\end{equation}
\begin{equation}
C_{GD} =  \left\{ \begin{array}{ll}
            \CGDO\W_{\ms{EFF}}          & \mbox{accumulation region}\\
            \CGDO\W_{\ms{EFF}}          & \mbox{depletion region}\\
            \CGDO\W_{\ms{EFF}}  & \mbox{saturation region}\\
            C_{OX} \left\{ 1 - \left[
              {{\textstyle V_{GS} - V_{\ms{ON}}} \over
               {\textstyle 2(V_{GS}-V_{\ms{ON}}) - V_{DS}}} \right]^2\right\}
              + \CGDO\W_{\ms{EFF}}  & \mbox{linear region}
            \end{array}
            \right. %}
\end{equation}
\begin{equation}
C_{GB} =  \left\{ \begin{array}{ll}
            C_{OX} + \CGBO L_{\ms{EFF}}        & \mbox{accumulation region}\\
            C_{OX} \left({{\textstyle V_{\ms{ON}}-V_{GS}}\over {\textstyle\PHI}}
                   \right) + \CGBO L_{\ms{EFF}} & \mbox{depletion region}\\
            \CGBO L_{\ms{EFF}} & \mbox{saturation region}\\
            \CGBO L_{\ms{EFF}} & \mbox{linear region}
            \end{array}
            \right.%}
\end{equation}
\vshift

\noindent\underline{\sl \large {\tt LEVEL} 2 Ward-Dutton Model
\notforsspice{\pspice\ only}}\\[0.1in]
\index{Ward-Dutton model, see MOSFET} \index{MOSFET, overlap
capacitances Ward-Dutton Model} \index{MOSFET, Ward-Dutton model}
\index{MOSFET, Ward-Dutton model, LEVEL 2} \index{MOSFET,
$C_{GS}$} \index{MOSFET, $C_{GD}$} \index{MOSFET, $C_{GB}$}

This model is selected when the parameter {\tt XQC} is specified
and less than 0.5. The charge in the gate $Q_G$ and the substrate
$Q_B$ is calculated and the difference of these is taken as the
channel charge $Q_{\ms{CHANNEL}}$. This charge is then partitioned
and allocated between the source as $Q_S$ and the drain $Q_D$ as
follows:
\begin{eqnarray}
Q_{\ms{CHANNEL}} & = & Q_D + Q_S \\
Q_D         & = & \XQC Q_{\ms{CHANNEL}}\\
\end{eqnarray}
so that $Q_S          =  (1 - \XQC) Q_{\ms{CHANNEL}}$. This
partitioning is somewhat arbitrary but produces transient results
that more closely match measurements than does the Meyer
capacitance model. However this is at the price of poorer
convergence properties and sometimes error. This is particularly
so when $V_{DS}$ is changing sign.

Two operating regions are defined in the Ward-Dutton model:\\[0.1in]
\hspace*{\fill}\offsetparbox{
\begin{tabular}{ll}
off region:&$V_{GS} \le V_T$\\
on  region:&$V_{GS} > V_T$\\
\end{tabular}}\\[0.1in]
\noindent where
\begin{eqnarray}
V_{T} & = & V_{T0} + \GAMMA
        \left[\sqrt{\PHI - V_{BS}}-\sqrt{\PHI} \right]\\
C_{OX}&=& C'_{OX}\W_{\ms{EFF}} L_{\ms{EFF}}\\
C'_{OX}&=&{{\epsilon_{OX}} \over { \TOX}}\\
\end{eqnarray}
\noindent In the charge evaluations the following terms are used:
\begin{eqnarray}
  v_{G}  &=& v_{GB} - V'_{FB} + \PHI\\
    v_D  &=& \left\{ \begin{array}{ll}
             \PHI - v_{BD} & v_{BD} > \PHI \\
             0 &v_{BD} \le \PHI
             \end{array} \right. %}
             \\
    v_S  &=& \left\{ \begin{array}{ll}
             \PHI - v_{BS} & v_{BS} > \PHI \\
             0 &v_{BS} \le \PHI
             \end{array} \right. %}
             \\
    v_E  &=& \left\{ \begin{array}{ll}
              v_D     & v_D < v_{DS,\ms{SAT}}\\
              v_{DS,\ms{SAT}} & v_D \ge v_{DS,\ms{SAT}}
              \end{array}
              \right. %}
              \\
    x_5  &=& (v_E + vs)( \sqrt{v_E} + \sqrt{v_S})\\
    x_6  &=& ((v_E^2+v_S^2) + v_Ev_S) + \sqrt{v_E}\sqrt{v_S} (v_E+v_S)\\
    D   &=& v_G (\sqrt{v_E}+\sqrt{v_S})
          - \gamma_{\ms{EFF}} ( (v_E+v_S) + \sqrt{v_E}\sqrt{v_S} )/1.5
          \nonumber \\
          && - .5 (\sqrt{v_E}+\sqrt{v_S}) (v_E+v_S)
\end{eqnarray}
where $V'_{FB}$ is defined in (\ref{eqn:vdashfb}) and
$V_{DS,\ms{SAT}}$ in (\ref{eqn:vdssat:l2}).

\noindent \underline{off region}

\begin{eqnarray}
Q_G &=& \left\{ \begin{array}{ll}
       \gamma_{\ms{EFF}} C_{OX}
       ( \sqrt{\frac{1}{4}\gamma_{\ms{EFF}}^2 + v_G} - \gamma_{\ms{EFF}}/2 )
                                      & v_G > 0 \\
       C_{OX} v_G                     & v_G \le 0
       \end{array} \right. %}
       \\
  Q_B &=& -Q_G\\
  Q_{\ms{CHANNEL}} &=& -( Q_G + Q_B )
\end{eqnarray}


\noindent \underline{on region}


\begin{eqnarray}
    Q_B &=& -\gamma_{\ms{EFF}} C_{OX}
          \left( {{\textstyle
             vg \frac{2}{3} ( (v_E+v_S) + \sqrt{v_E}\sqrt{v_S} )
             - \frac{1}{2} \gamma_{\ms{EFF}} x_5
             - .4 x_6 } \over {\textstyle D}} \right) \\
    Q_G &=& C_{OX} \nonumber \\
        &&   \left( vg - {{\textstyle .5 vg x_5 - .4 \gamma_{\ms{EFF}} x_6
             - ((v_E^2+v_S^2) + v_Ev_S)(\sqrt{v_E}+\sqrt{v_S})/3
 } \over{\textstyle D}} \right)
\end{eqnarray}

where $V'_{FB}$ is defined in (\ref{eqn:vdashfb}). \noindent The
overlap capacitances are then evaluated as
\begin{eqnarray}
    C_{GDB}  &=& \partial Q_G / \partial v_D\\
    C_{GSB}  &=& \partial Q_G / \partial v_S\\
    C_{GGB}  &=& \partial Q_G / \partial v_G\\
    C_{BDB}  &=& \partial Q_B / \partial v_D\\
    C_{BSB}  &=& \partial Q_B / \partial v_S\\
    C_{BGB}  &=& \partial Q_B / \partial v_G
\end{eqnarray}

\notforsspice{ The {\tt LEVEL} 2 overlap capacitance parameter
dependencies are summarized in figure \ref{mlevel2overlap}.
\begin{figure}[hb]
\parbox[t]{1.3in}{
\begin{tabular}[t]{|p{1in}|}
\hline
\multicolumn{1}{|c|}{PROCESS} \\
\multicolumn{1}{|c|}{PARAMETERS} \\
\hline \hline
{\tt CJ} \hfill $\CJ$\\
{\tt CJSW} \hfill $\CJSW$\\
{\tt MJ} \hfill $\MJ$\\
{\tt MJSW} \hfill $\MJSW$\\
{\tt PB} \hfill $\PB$\\
{\tt PBSW} \hfill $\PB$\\
{\tt FC} \hfill $\PB$\\
\hline
\end{tabular}
} \hfill
\parbox{0.1in}{\ \vspace*{0.2in}\newline +}
\hfill
\begin{tabular}[t]{|p{1in}|}
\hline
\multicolumn{1}{|c|}{GEOMETRY} \\
\multicolumn{1}{|c|}{PARAMETERS} \\
\hline
{\tt AD} \hfill $A_D$\\
{\tt AS} \hfill $A_S$\\
{\tt PD} \hfill $A_D$\\
{\tt PS} \hfill $A_S$\\
\hline
\end{tabular}
\hfill
\parbox{0.1in}{\ \vspace*{0.2in}\newline $\rightarrow$}
\hfill
\begin{tabular}[t]{|p{1.8in}|}
\hline
\multicolumn{1}{|c|}{DEVICE} \\
\multicolumn{1}{|c|}{PARAMETERS} \\
\hline \hline
\multicolumn{1}{|c|}{Constant Overlap}\\
\multicolumn{1}{|c|}{Capacitances}\\
\hline
{\tt CGSO} \hfill $\CGSO$\\
{\tt CGDO} \hfill $\CGDO$\\
{\tt CGBO} \hfill $\CGBO$\\
\hline
\end{tabular}
\caption{ MOSFET {\tt LEVEL} 2 overlap capacitance parameter
relationships. \label{mlevel2overlap}}
\end{figure}}
\\[0.2in]}

\noindent\underline{\sl {\tt LEVEL 3} I/V Characteristics}
\index{I/V Characteristics, see MOSFET} \index{I/V
Characteristics, NMOS} \index{I/V Characteristics, PMOS}
\index{MOSFET, I/V Characteristics} \index{NMOS, I/V
Characteristics} \index{PMOS, I/V Characteristics} \index{MOSFET,
LEVEL 3 I/V Characteristics} \index{NMOS, LEVEL 3 I/V
Characteristics} \index{PMOS, LEVEL 3 I/V Characteristics}

The {\tt LEVEL} 3 I/V characteristics are based on empirical fits
resulting in a more accurate description of the I/V response than
obtained with the {\tt LEVEL} 2 model. The {\tt LEVEL 3}
current/voltage characteristics are evaluated after first
determining the mode (normal: $V_{DS} \ge 0$ or inverted: $V_{DS}
< 0$) and the region (cutoff, linear or saturation) of the current
$(V_{DS}, V_{GS})$ operating point.

\noindent{\sl Normal Mode: ($V_{DS} \ge 0$)}\\[0.2in]
The regions are as follows:\\[0.1in]
\hspace*{\fill}\offsetparbox{
\begin{tabular}{ll}
cutoff region:&$V_{GS}<V_{T}$\\
weak inversion region:&$V_T < V_{GS} \le V_{\ms{ON}}$ \\
linear region (strong inversion):&$V_{GS} > V_{\ms{ON}}$
                  and $V_{DS} < V_{DS,\ms{SAT}}$ \\
saturation region (strong inversion):&$V_{GS} > V_{\ms{ON}}$
                  and $V_{DS} > V_{DS,\ms{SAT}}$
\end{tabular}}\\
where
\begin{equation}
V_T = \VTO - \sigma V_{DS} + F_C
\end{equation}
the effect of short and narrow channel on threshold voltage
\begin{equation}
F_C = \GAMMA F_S\sqrt{\PHI- V_{BS}}
         + F_N(\PHI - V_{BS})
\label{eqn:fc:level3}
\end{equation}
and
\begin{equation}
V_{\ms{ON}} = \left\{ \begin{array}{ll}
     V_T             & NFS = 0\\
     V_T + V_{TH}x_n & NFS \ne 0
     \end{array} \right. %}
\end{equation}

The effect of the short channel is described by
\begin{equation}
F_S = 1 - {\XJ\over{L_{\ms{EFF}}}} \left(
      {{X_{JL} + W_C}\over{\XJ}}
      \sqrt{1 - {{W_P\over{\XJ + W_P}}}} - {{X_{JL}}\over\XJ}\right)
\end{equation}
where
\begin{equation}
W_P = X_D\sqrt{\PB-V_{BS}}
\end{equation}
\begin{equation}
X_D = \sqrt{{2 \epsilon_s}\over q \NSUB} \label{m3:xd:eqn}
\end{equation}
\begin{equation}
W_C = \XJ \left[ 0.0831353 + 0.8013929{{W_P}\over{\XJ}}
      +0.0111077{{W_P}\over{\XJ}} \right]
\end{equation}
and
\begin{equation}
\sigma = \ETA{{8.15 \time 10^{-22}}\over{C'_{OX}L_{\ms{EFF}}^3}}
\label{eqn:m:sigma:l3}
\end{equation}
The effect of channel width on threshold is
\begin{equation}
F_N = {{\textstyle\epsilon_s\delta\pi} \over {\textstyle 4
C'_{OX}\W_{\ms{EFF}}}}
\end{equation}

The effective mobility due to modulation by the gate
\begin{equation}
\mu_S = \UO F_G
\end{equation}
and the factor describing mobility modulation by the gate is
\begin{equation}
    F_G = {1 \over {1 + \THETA (V_{GSX}-V_T)}}
\end{equation}
where
\begin{equation}
V_{GSX} = \left\{ \begin{array}{ll}
          V_{GS}      &  V_{GS} < V_{\ms{ON}}\\
          V_{\ms{ON}} &  V_{GS} \ge V_{\ms{ON}}
          \end{array} \right. %}
\label{eqn:vgsx:l3}
\end{equation}
The drain-source saturation voltage
\begin{equation}
V_{DS,\ms{SAT}} = \left\{ \begin{array}{ll}
           V_A + V_B - \sqrt{V_A^2 + V_B^2} & \VMAX > 0\\
           V_P                  & \VMAX \le 0
           \end{array} \right. %}
\end{equation}
where
\begin{equation}
V_A = {{V_{GSX} - V_T} \over {1+F_B}}
\end{equation}
\begin{equation}
V_B = {{v_{\ms{MAX}}L_{\ms{EFF}}}\over{\mu_S}}
\end{equation}
\begin{equation}
V_P = V_{GSX} - V_T
\end{equation}
The body effect factor
\begin{equation}
F_B = {{\GAMMA F_S} \over {4 \sqrt{\phi_{BS}}}} + F_N
\label{eqn:fb:l3}
\end{equation}
where
\begin{equation}
\phi_{BS} = \left\{ \begin{array}{ll}
            \PHI - V_{BS}    &  V_{BS} \le 0\\
            {{\textstyle \PHI}\over{\textstyle\sqrt{1+\frac{1}{2}V_{BS}/
                 (\PHI)}}}    & V_{BS} > 0
           \end{array} \right. %}
\end{equation}
The velocity saturation factor is
\begin{equation}
    F_D = \left\{ \begin{array}{ll}
          {{\textstyle 1} \over {\textstyle 1 + V_{DS}/V_B}}
                      & \mbox{for} \VMAX \ne 0 \\
           1                              & \mbox{for} \VMAX = 0
       \end{array} \right. %}
\end{equation}
\underline{cutoff region}
\begin{equation}
I_{D} = 0
\end{equation}\\[0.2in]
%
%
\underline{linear region}
\begin{equation}
 I_{D} =  \KP {{W_{\ms{EFF}}}\over{ L_{\ms{EFF}}}} F_G F_D
           \left[ V_{GSX} - V_T - {{1 + F_{\ms{B}}} \over 2} V_{DS}
       \right] V_{DS}
      \label{mos3linearids}
\end{equation}
%
%
\underline{weak inversion region}\\[0.1in]
When $V_{GS}$ is slightly above $V_T$, $I_{D}$ increases slowly
over a few thermal voltages $V_{\ms{TH}}$ in exponential manner
becoming $I_{D}$ calculated for strong inversion. This effect is
handled empirically by defining two exponential which, as well as
ensuring an exponential increase in $I_{D}$, also ensure that the
transconductance $G_M$ (= $\partial I_{D}/\partial V_{GS}$) is
continuous at $V_{GS}$ = $V_{\ms{ON}}$.
\begin{equation}
I_{D} = \left\{ \begin{array}{ll}
         I_{D,\ms{ON}}
         \left[ {{\textstyle 10}\over{\textstyle 11}}
         e^{\textstyle (V_{GS}-V_{\ms{ON}})/(x_nV_{TH})}
       + {{\textstyle 1}\over{\textstyle 11}}
         e^{\textstyle \alpha (V_{GS}-V_{\ms{ON}})} \right]  &  \alpha > 0 \\
         \\
         I_{D,\ms{ON}}
         e^{\textstyle (V_{GS}-V_{\ms{ON}})/(x_nV_{TH})}
         &  \alpha \le 0
         \end{array} \right. %}
\end{equation}
where
\begin{equation}
\alpha = 11 \left( {{\textstyle G_{M,\ms{ON}}}
         \over {\textstyle I_{D,\ms{ON}}}}
         - {{\textstyle 1} \over {\textstyle
         x_n V_{\ms{TH}}}}
         \right)
\end{equation}
\begin{equation}
G_{M,\ms{ON}} = {{\partial I_{D,\ms{ON}}} \over {\partial V_{GS}}}
\end{equation}
and
\begin{equation}
I_{D,\ms{ON}} =  \left\{ \begin{array}{ll}
                  \parbox{3in}{$I_{D}$ in (\ref{mos3linearids})
                               with $V_{GS}$ = $V_{ON}$}
                  & V_{DS} \le V_{DS,\ms{SAT}} \\ \\
                  \parbox{3in}{$I_{D}$ in (\ref{mos3satids})
                               with $V_{GS}$ = $V_{ON}$ and
                               $V_{DS}$ = $V_{DS,\ms{SAT}}$}
                  & V_{DS} \le V_{DS,\ms{SAT}}
                  \end{array} \right. %}
\end{equation}
\\[0.2in]
%
%
\underline{saturation region}
\begin{equation}
I_{D} = {{L_{\ms{EFF}}}\over{\textstyle L_{\ms{EFF}-\Delta_L}}}
I_{D,\ms{SAT}}
      \label{mos3satids}
\end{equation}
\begin{equation}
I_{D,\ms{SAT}}  =   \KP {{\W_{\ms{EFF}}}\over{ L_{\ms{EFF}}}} F_G
F_D
           \left[ V_{GSX} - V_T - {{1 + F_{\ms{B}}}\over 2} V_{DS,\ms{SAT}}
       \right] V_{DS,\ms{SAT}}
\end{equation}
The reduction in the channel length due to $V_{DS}$ modulation is
\begin{equation}
\Delta_L = \left\{ \begin{array}{ll}
            \Delta'_L    & \Delta'_L < \frac{1}{2} L_{\ms{EFF}}\\
            L_{\ms{EFF}} - {{\textstyle L_{\ms{EFF}}} \over
            {\textstyle \Delta'_L}}
             & \Delta'_L \ge \frac{1}{2} L_{\ms{EFF}}
        \end{array} \right. %}
            \label{m3deltal}
\end{equation}
where the punch through approximation is used for $\Delta'_L \ge
\frac{1}{2} L_{\ms{EFF}}$> In (\ref{m3deltal}) the distance that
the depletion region at the drain extends into the channel is
\begin{equation}
\Delta'_L = \sqrt{ \left( {{E_PX_D^2} \over 2}\right)^2
                           + \KAPPA X_D^2(V_{DS} - V_{DS,\ms{SAT}})}
            - {{E_PX_D^2}\over2}
\end{equation}
and
\begin{equation}
E_P = {{I_{D,\ms{SAT}}}\over{G_{DS,\ms{SAT}}L_{\ms{EFF}}}}
\end{equation}
Here
\begin{equation}
G_{DS,\ms{SAT}} = {{\partial I_{D,\ms{SAT}}}\over{\partial
V_{DS,\ms{SAT}}}}
\end{equation}
\sspiceonly{This equation does not appear to be calculated
correctly in
superspice.\\[0.1in]}
\notforsspice{The {\tt LEVEL} 3 current-voltage parameter
dependencies
are summarized in figure \ref{mlevel3iv}.\\[0.2in]
\begin{figure}[hb]
\begin{tabular}[t]{|p{1in}|}
\hline
\multicolumn{1}{|c|}{PROCESS} \\
\multicolumn{1}{|c|}{PARAMETERS} \\
\hline \hline
{\tt KAPPA} \hfill $\KAPPA$\\
{\tt NSS} \hfill $\NSS$\\
{\tt NSUB} \hfill $\NSUB$\\
{\tt PB} \hfill $\PB$\\
{\tt THETA} \hfill $\THETA$\\
{\tt TOX} \hfill $\TOX$\\
{\tt UO} \hfill $\UO$\\
{\tt VMAX} \hfill $\VMAX$\\
{\tt XJ} \hfill $\XJ$\\
\hline
\end{tabular}
\hfill
\parbox{0.1in}{\ \vspace*{0.2in}\newline +}
\hfill
\begin{tabular}[t]{|p{1in}|}
\hline
\multicolumn{1}{|c|}{GEOMETRY} \\
\multicolumn{1}{|c|}{PARAMETERS} \\
\hline \hline
\multicolumn{1}{|c|}{Required} \\
\hline
{\tt L} \hfill $L$\\
{\tt W} \hfill $W$ \\
\hline \hline
\multicolumn{1}{|c|}{Optional} \\
\hline
{\tt LD} \hfill $\LD$\\
{\tt WD} \hfill $\WD$ \\
\hline
\end{tabular}
\hfill
\parbox{0.1in}{\ \vspace*{0.2in}\newline $\rightarrow$}
\hfill
\begin{tabular}[t]{|p{1.8in}|}
\hline
\multicolumn{1}{|c|}{DEVICE} \\
\multicolumn{1}{|c|}{PARAMETERS} \\
\hline \hline
\multicolumn{1}{|c|}{I/V}\\
\hline
{\tt VTO} \hfill $\VTZERO = f(\PHI, \NSS, \TOX)$\\
{\tt KP} \hfill $\KP = f(\UO, \TOX)$\\
{\tt PHI} \hfill $\PHI = f(\NSUB)$\\
{\tt GAMMA} \hfill $\GAMMA = f(\UO, \TOX, \NSUB)$\\
\{$I_{D} = f(\W, \Length, \WD, \LD$\newline \hspace*{\fill}$\KP,
\PHI, \NSUB, \TOX, \UO, \THETA$\newline
\hspace*{\fill}$\GAMMA, \VTO, \PB, \VMAX, \XJ, \KAPPA, \ETA)$\}\\
\hline
\end{tabular}
\caption{ MOSFET {\tt LEVEL} 3 I/V parameter relationships.
\label{mlevel3iv}}
\end{figure}}
\noindent\underline{\sl \large {\tt LEVEL} 3 Overlap Capacitances}\\[0.1in]
\index{MOSFET, overlap capacitances LEVEL 3} \index{MOSFET, LEVEL
3, overlap capacitances LEVEL 3} \index{MOSFET, $C_{GS}$}
\index{MOSFET, $C_{GD}$} \index{MOSFET, $C_{GB}$}
\\[0.1in]
\index{Overlap Capacitances, see MOSFET} \index{MOSFET, Overlap
Capacitances LEVEL 3} \index{MOSFET, Overlap Capacitances} In the
{\tt LEVEL 3} model the gate overlap capacitances asre strong
functions of voltage. Two overlap capacitance models are available
the Meyer model based on the model originally proposed by Meyer
\cite{meyer:71} and the Ward-Dutton model
\cite{ward:dutton:78,oh:ward:80}. These models differ in the
derivation of the
channel charge.\\[0.2in]

\noindent\underline{\sl \large {\tt LEVEL} 3 Meyer Model}\\[0.1in]
\index{Meyer model, see MOSFET} \index{MOSFET, overlap
capacitances Meyer model} \index{MOSFET, Meyer model}
\index{MOSFET, Meyer model, LEVEL 3} \index{MOSFET, $C_{GS}$}
\index{MOSFET, $C_{GD}$} \index{MOSFET, $C_{GB}$}. This model is
selected when the parameter {\tt XQC} = $\XQC$ is not specified or
$\XQC < 0.5$.

The voltage dependent thin-oxide capacitances are used only if
$\TOX$ is specified in the model statement.

Four operating regions are defined in the Meyer model:\\[0.1in]
\hspace*{\fill}\offsetparbox{
\begin{tabular}{ll}
accumulation region:&$V_{GS} < V_{\ms{ON}} - \PHI$\\
depletion region:   &$V_{\ms{ON}} - \PHI < V_{GS} < V_{\ms{ON}}$\\
saturation region:  &$V_{\ms{ON}} < V_{GS} <V_{\ms{ON}} + V_{DS}$\\
linear region:&$V_{GS} > V_{\ms{ON}}+V_{DS}$\\
\end{tabular}}\\[0.1in]
where
\begin{eqnarray}
V_{\ms{ON}}  & = & \left\{ \begin{array}{ll}
            V_{T} +  x_nV_{\ms{TH}} &
              \mbox{if } \NFS = {\tt NFS} \mbox{ specified}\\
            V_{T}                   & \mbox{if } \NFS = {\tt NFS}
                                       \mbox{ not specified}
            \end{array} \right.\\ %}
V_{T} & = & V_{T0} + \GAMMA
        \left[\sqrt{\PHI - V_{BS}}-\sqrt{\PHI} \right]\\
x_n      & = & 1 + {{q \NFS} \over {C'_{OX}}} + {{C_D} \over {C'_{OX}}}\\
C'_{OX}&=&{{\epsilon_{OX}} \over { \TOX}}\\
C_{OX}&=& C'_{OX}\W_{\ms{EFF}} L_{\ms{EFF}}\\
C_D & = & {{\textstyle\GAMMA} \over {\textstyle 2 \sqrt{\PHI -
V_{BS}}}}
\end{eqnarray}
\begin{equation}
C_{GS} =  \left\{ \begin{array}{ll}
            \CGSO\W                     & \mbox{accumulation region}\\
            \frac{2}{3} C_{OX}
             \left( 1+ {{\textstyle V_{\ms{ON}}
             - V_{GS}} \over {\textstyle\PHI}}\right)
            + \CGSO\W_{\ms{EFF}}                   & \mbox{depletion region}\\
           \frac{2}{3} C_{OX} + \CGSO\W_{\ms{EFF}}  & \mbox{saturation region}\\
            C_{OX} \left\{ 1 - \left[
              {{\textstyle V_{GS} - V_{DS} - V_{\ms{ON}}} \over
               {\textstyle 2(V_{GS}-V_{\ms{ON}}) - V_{DS}}} \right]^2\right\}
            + \CGSO\W_{\ms{EFF}}  & \mbox{linear region}

            \end{array}
            \right. %}
\end{equation}
\begin{equation}
C_{GD} =  \left\{ \begin{array}{ll}
            \CGDO\W_{\ms{EFF}}          & \mbox{accumulation region}\\
            \CGDO\W_{\ms{EFF}}          & \mbox{depletion region}\\
            \CGDO\W_{\ms{EFF}}  & \mbox{saturation region}\\
            C_{OX} \left\{ 1 - \left[
              {{\textstyle V_{GS} - V_{\ms{ON}}} \over
               {\textstyle 2(V_{GS}-V_{\ms{ON}}) - V_{DS}}} \right]^2\right\}
              + \CGDO\W_{\ms{EFF}}  & \mbox{linear region}
            \end{array}
            \right. %}
\end{equation}
\begin{equation}
C_{GB} =  \left\{ \begin{array}{ll}
            C_{OX} + \CGBO L_{\ms{EFF}}        & \mbox{accumulation region}\\
            C_{OX} \left({{\textstyle V_{\ms{ON}}-V_{GS}}\over {\textstyle\PHI}}
                   \right) + \CGBO L_{\ms{EFF}} & \mbox{depletion region}\\
            \CGBO L_{\ms{EFF}} & \mbox{saturation region}\\
            \CGBO L_{\ms{EFF}} & \mbox{linear region}
            \end{array}
            \right.%}
\end{equation}
\vshift

\noindent\underline{\sl \large {\tt LEVEL} 3 Ward-Dutton Model}\\[0.1in]
\index{Ward-Dutton model, see MOSFET} \index{MOSFET, overlap
capacitances Ward-Dutton Model} \index{MOSFET, Ward-Dutton model}
\index{MOSFET, Ward-Dutton model, LEVEL 3} \index{MOSFET,
$C_{GS}$} \index{MOSFET, $C_{GD}$} \index{MOSFET, $C_{GB}$}

This model is selected when the parameter {\tt XQC} is specified
and less than 0.5. The charge in the gate $Q_G$ and the substrate
$Q_B$ is calculated and the difference of these is taken as the
channel charge $Q_{\ms{CHANNEL}}$. This charge is then partitioned
and allocated between the source as $Q_S$ and the drain $Q_D$ as
follows:
\begin{eqnarray}
Q_{\ms{CHANNEL}} & = & Q_D + Q_S \\
Q_D         & = & \XQC Q_{\ms{CHANNEL}}\\
\end{eqnarray}
so that $Q_S          =  (1 - \XQC) Q_{\ms{CHANNEL}}$. This
partitioning is somewhat arbitrary but produces transient results
that more closely match measurements than does the Meyer
capacitance model. However this is at the price of poorer
convergence properties and sometimes error. This is particularly
so when $V_{DS}$ is changing sign.
Two operating regions are defined in the Ward-Dutton model:\\[0.1in]
\hspace*{\fill}\offsetparbox{
\begin{tabular}{ll}
off region:&$V_{GS} \le V'_T$\\
on  region:&$V_{GS} > V'_T$\\
\end{tabular}}\\[0.1in]
\noindent where
 \begin{eqnarray}
  V'_T &=& v_{BIX} + F_C\\
  v_{BIX} &=& V_{FB} - \sigma V_{DS}
\end{eqnarray}
$F_C$ is defined in (\ref{eqn:fc:level3}), $\sigma$ in
(\ref{eqn:m:sigma:l3}) and $V_{FB}$ is defined in (\ref{mFB1}) or
(\ref{mFB1}). \underline{off region}

\begin{eqnarray}
       Q_G &=&
      \left\{ \begin{array}{l}
       \gamma F_S C_{\ms{OX}} \left[
         \sqrt{ \left({{\textstyle \gamma F_S}\over{\textstyle 2}}\right)^2
         + (v_{GB}-V_{FB}+\PHI) } -
         {{\textstyle\gamma F_S}\over{\textstyle 2}} \right]\\
      \hspace*{\fill} v_{GB} > (V_{FB}-\PHI)\\
      \\
      C_{\ms{OX}} (v_{GB}-V_{FB}+\PHI)
      \hspace*{\fill} v_{GB} \le (V_{FB}-\PHI)
      \end{array} \right. %}
      \\
    Q_B &=& -Q_G
\end{eqnarray}
\underline{on region}

\begin{eqnarray}
      Q_G &=& \left\{ \begin{array}{ll}
            C_{\ms{OX}} (V_{GS}-V_{FB}) &V_{DSX} = 0 \\
             C_{\ms{OX}} ( V_{GS} - v_{BIX} - \frac{1}{2} V_{DSX} + x_a )
            &V_{DSX} \ne 0
            \end{array} \right. %}
            \\
       Q_B &=& \left\{ \begin{array}{ll}
            -C_{\ms{OX}} F_C &V_{DSX} = 0 \\
             -C_{\ms{OX}} ( F_C + \frac{1}{2} F_B V_{DSX}
             - x_a F_B ) &V_{DSX} \ne 0
            \end{array} \right. %}
\end{eqnarray}
where
\begin{equation}
      x_a  = {{\textstyle (1+F_B) V_{DSX}^2}\over
              {12 V_{GSX} - V'_T - \frac{1}{2} (1+F_B) V_{DSX} }}
\end{equation}


\begin{equation}
Q_{\ms{CHANNEL}} = -( Q_G + Q_B )
\end{equation}
where $F_B$ is defined in (\ref{eqn:fb:l3}), $V_{GSX}$ in
(\ref{eqn:vgsx:l3}) and
\begin{equation}
      V_{DSX}   = \left\{ \begin{array}{ll}
                    V_{DS,\ms{SAT}} & V_{DS} > V_{D,\ms{SAT}}\\
                    V_{DS} & V_{DS} \le V_{D,\ms{SAT}}
                    \end{array} \right. %}
\end{equation}
The overlap capacitances are then evaluated as
\begin{eqnarray}
    C_{GDB}  &=& \partial Q_G / \partial v_D\\
    C_{GSB}  &=& \partial Q_G / \partial v_S\\
    C_{GGB}  &=& \partial Q_G / \partial v_G\\
    C_{BDB}  &=& \partial Q_B / \partial v_D\\
    C_{BSB}  &=& \partial Q_B / \partial v_S\\
    C_{BGB}  &=& \partial Q_B / \partial v_G
\end{eqnarray}

\notforsspice{The overlap capacitance parameter dependencies are
summarized in figure \ref{m3overlap:fig}.
\begin{figure}[hb]
\parbox[t]{1.3in}{
\begin{tabular}[t]{|p{1in}|}
\hline
\multicolumn{1}{|c|}{PROCESS} \\
\multicolumn{1}{|c|}{PARAMETERS} \\
\hline \hline
{\tt NSUB} \hfill $\NSUB$\\
{\tt CJ} \hfill $\CJ$\\
{\tt CJSW} \hfill $\CJSW$\\
{\tt MJ} \hfill $\MJ$\\
{\tt MJSW} \hfill $\MJSW$\\
{\tt PB} \hfill $\PB$\\
{\tt PBSW} \hfill $\PBSW$\\
{\tt FC} \hfill $\FC$\\
\hline
\end{tabular}
} \hfill
\parbox{0.1in}{\ \vspace*{0.2in}\newline +}
\hfill
\begin{tabular}[t]{|p{1in}|}
\hline
\multicolumn{1}{|c|}{GEOMETRY} \\
\multicolumn{1}{|c|}{PARAMETERS} \\
\hline
{\tt AD} \hfill $A_D$\\
{\tt AS} \hfill $A_S$\\
{\tt PD} \hfill $P_D$\\
{\tt PS} \hfill $P_S$\\
\hline
\end{tabular}
\hfill
\parbox{0.1in}{\ \vspace*{0.2in}\newline $\rightarrow$}
\hfill
\begin{tabular}[t]{|p{1.8in}|}
\hline
\multicolumn{1}{|c|}{DEVICE} \\
\multicolumn{1}{|c|}{PARAMETERS} \\
\hline \hline
\multicolumn{1}{|c|}{Constant Overlap}\\
\multicolumn{1}{|c|}{Capacitances}\\
\hline
{\tt CGSO} \hfill $\CGSO$\\
{\tt CGDO} \hfill $\CGDO$\\
{\tt CGBO} \hfill $\CGBO$\\
\hline
\end{tabular}
\caption{ MOSFET {\tt LEVEL} 3 overlap capacitance parameter
relationships. \label{m3overlap:fig}}
\end{figure}
\vspace*{\fill}

\noindent {\large \bf BSIM1 ({\tt LEVEL} 4) MOSFET models.}\\
\noindent\rule{\textwidth}{0.5mm}\\[0.2in]

The parameters of the BSIM1 ({\tt LEVEL 4} model are  all  values
obtained  from process  characterization,  and  can  be generated
automatically. J. Pierret \cite{pierret:84} describes a  means  of
generating  a `process'  file,  and  the  program  Proc2Mod
provided in the UC Berkeley standard SPICE3 distribution converts
this file  into  a  sequence  of  .MODEL lines  suitable  for
inclusion  in  a  SPICE  circuit file. Parameters marked below
with an * in  the  L/W  column  also have corresponding parameters
with a length and width dependency.


Unlike most other models the  BSIM1 model  is  designed  for use
with a process characterization system that provides all the
parameters, thus there  are  no defaults  for  the  parameters,
and leaving one out is considered an error.
\begin{table}[hbt]
\caption{SPICE BSIM1 (level 4) parameters. \label{bsim1table}}
\keywordtwotable{L/W}{ {\tt CGBO} &gate-bulk overlap capacitance
per meter channel length
        \para\sym{\CGBO}    & F/m  & \reqd  &\X
{\tt CGDO} &gate-drain overlap capacitance per meter channel
        width \para\sym{\CGDO}    & F/m     & \reqd &\X
{\tt CGSO} &gate-source overlap capacitance per meter channel
width
        \para\sym{\CGSO}   & F/m  & \reqd   &\X
{\tt CJ}   &source-drain junction capacitance per unit
area\para\sym{\CJ}
        & F/$\mbox{m}^2$ & \reqd    & \X
{\tt CJSW} &source-drain junction sidewall capacitance per unit
length
        \para\sym{\CJSW}& F/m  & \reqd  & \X
{\tt DL}   &shortening of channel \sym{\DL} & $\mu$m
    & \reqd & \X
{\tt DW}   &narrowing of channel\sym{\DW}  & $\mu$m
    & \reqd & \X
{\tt DELL} &source-drain junction length\newline reduction
\sym{DELL}          & m
    & \reqd & \X
{\tt ETA}  &zero-bias drain-induced barrier lowering coefficient
        \sym{\ETA}& -
    & \reqd &\STAR\X
{\tt JS}   &source-drain junction current\newline density
        \sym{\JS}& A/$\mbox{m}^2$
    & \reqd & \X
{\tt K1}   &body effect coefficient
\sym{\KONE}&$\mbox{V}^{\frac{1}{2}}$
    & \reqd &\STAR\X
}
\end{table}

\kwtwo{Table \ref{bsim1table} continued: BSIM ({\tt LEVEL}) 4
model keywords.\vshift}{L/W}{ {\tt K2}   &drain/source depletion
charge sharing\newline coefficient
       \sym{\KTWO} & -
    & \reqd &\STAR\X
{\tt MJ}   &grading coefficient of source-drain junction
        \sym{\MJ}& - & \reqd    & \X
{\tt MJSW} &grading coefficient of source-drain junction sidewall
        \sym{\MJSW}     & - & \reqd &\X
{\tt MUS}  &mobility at zero substrate bias and at $V_{DS}$ =
$V_{DD}$
        \sym{\MUS}& \hspace*{-0.08in} $\mbox{cm}^2/\mbox{V}^2$s & \reqd&\STAR\X
{\tt MUZ}  &zero-bias mobility\sym{\MUZ}& $\mbox{cm}^2/\mbox{Vs}$
    & \reqd & \X
{\tt N0}   &zero-bias subthreshold slope coefficient (N-zero)
\sym{\NZERO}&-
    & \reqd & \STAR\X
{\tt NB}   &sensitivity of subthreshold slope to substrate bias
        \para\sym{\NB}& - & \reqd   & \STAR\X
{\tt ND}   &sensitivity~of~subthreshold slope to drain bias
\sym{\ND} & -
    & \reqd & \STAR \X
{\tt PB}   &built in potential of source/\newline drain
junction\sym{\PB}& V
    & \reqd & \X
{\tt PBSW} &built in potential of source/\newline drain juntion
sidewall
        \para\sym{\PBSW}& V & \reqd & \X
{\tt PHI}  &surface inversion potential \sym{\PHI} & V &
\reqd&\STAR\X {\tt RSH}  &drain and source diffusion sheet
resistance\para
        \sym{\RSH}& $\Omega$-square & \reqd & \X
{\tt TEMP} &temperature at which parameters were measured
\sym{\TEMP}& C
    & \reqd &\X
{\tt TOX}  &gate oxide thickness \sym{\TOX}   & $\mu$m
    & \reqd &\X
{\tt U0}   &zero-bias transverse-field mobility degradation
coefficient
        (U-zero \sym{\UZERO}& $\mbox{V}^{-1}$   & \reqd &\STAR\X
{\tt U1}   &zero-bias velocity saturation coefficient
\sym{\UONE}&$\mu$m/V
    & \reqd & \STAR\X
{\tt VDD}  &measurement bias range  \sym{\VDD}
& V
    & \reqd &\X
{\tt WDF}  &source-drain junction default width\newline\sym{\WDF}
& m
    & \reqd &\X
{\tt VFB}  &flat-band voltage \sym{\VFB}                  & V
    & \reqd &\STAR\X
{\tt X2E} &sensitivity of drain-induced barrier lowering\newline
effect
       to substrate bias
    \sym{\XTWOE}& $\mbox{V}^{-1}$ & \reqd   &\STAR \X
{\tt X2MS} &sensitivity of mobility to substrate bias at $V_{DS}$
= $V_{DD}$
        \sym{\XTWOMS}& \hspace*{-0.08in} $\mbox{cm}^2/\mbox{V}^2\mbox{s}$
        & \reqd&\STAR\X
}

\kwtwo{Table \ref{bsim1table} continued: BSIM ({\tt LEVEL}) 4
model keywords.\vshift}{L/W}{ {\tt X2MZ} &sensitivity of mobility
to substrate bias at $V_{DS}$ = 0
        \sym{\XTWOMZ}& \hspace*{-0.08in} $\mbox{cm}^2/\mbox{V}^2\mbox{s}$
        & \reqd&\STAR\X
{\tt X2U0} &sensitivity of transverse field mobility\newline
degradation
        effect to substrate bias\newline (X2U-zero)
        \sym{\XTWOUZERO} & $\mbox{V}^{-2}$ & \reqd&\STAR\X
{\tt X2U1} &sensitivity of velocity saturation effect to substrate
bias
        \sym{\XTWOUONE}&$\mu\mbox{mV}^{-2}$ & \reqd &\STAR \X
{\tt X3E}  &sensitivity of drain-induced barrier lowering effect
to drain
        bias at $V_{DS}$ = $V_{DD}$\sym{\XTHREEE}& $\mbox{V}^{-1}$
    & \reqd &\STAR \X
{\tt X3MS} &sensitivity of mobility to drain bias at $V_{DS}$=
$V_{DD}$
        \sym{\XTHREEMS}
        & \hspace*{-0.08in} $\mbox{cm}^2/\mbox{V}^2\mbox{s}$ & \reqd&\STAR\X
{\tt X3U1} &sensitivity of velocity saturation effect on drain
bias at
        $V_{DS}$ = $V_{DD}$ \sym{\XTHREEUONE}
        & $\mu\mbox{V}^{-2}$ & \reqd    & \STAR \X
{\tt XPART}&gate oxide capacitance charge partition model flag
        \sym{\XPART}\newline
        XPART~=~0: selects a 40:60
      drain:source\newline\hspace*{\fill}charge partition\newline
        XPART~=~1: selects a
      0:100 drain:source\newline\hspace*{\fill}charge partition
    & - & \reqd &\X
} \vspace{-0.1in}
\begin{table}[h]
\caption{SPICE BSIM1 (level 4) parameters, \pspice extensions.
\label{bsim1pspicetable}} \keywordtable{ {\tt AF}      &flicker
noise exponent  \sym{\AF}    & -     &   1      \X {\tt CBD}
&zero-bias B-D junction capacitance \sym{\CBD}
     & F     &   0    \X
{\tt CBS} &zero-bias B-S junction capacitance \sym{\CBS}
     & F     &   0    \X
{\tt FC}      &coefficient for forward-bias
         depletion capacitance\newline formula \sym{\FC}  & -     &   0.5      \X
{\tt IS} &bulk junction saturation current\sym{\IS}  & A     &
1E-14\X {\tt KF}      &flicker noise coefficient \sym{\KF}      &
-     &   0      \X {\tt L}    & channel length \sym{\Length}& m
&   {\tt DEFL}  \X {\tt N}    & bulk $p$-$n$ emission coefficient
\para\sym{\N} & - & 0      \X {\tt PBSW} & bulk $p$-$n$ sidewall
potential\para\sym{\PBSW}& V& {\tt PB}\X}
\end{table}

\kw{Table \ref{bsim1pspicetable} continued: BSIM ({\tt LEVEL}) 4
model keywords, \pspice extensions.\vshift}{ {\tt JSSW}    & bulk
junction sidewall current per unit length
                \para\sym{\JSSW}& A/m   &   0 \X
{\tt RB}   & bulk ohmic resistance\para\sym{\RB}& $\Omega$&   0
\X {\tt RD}   & drain ohmic resistance\para\sym{\RD}& $\Omega$&
0   \X {\tt RDS}  & drain-source shunt
resistance\sym{\RDS}&$\Omega$  & 0   \X {\tt RG}   & gate ohmic
resistance\para\sym{\RG}& $\Omega$&   0   \X {\tt RS}   & source
ohmic resistance\para\sym{\RS}& $\Omega$&   0   \X {\tt TT}   &
bulk $p$-$n$ transit time\sym{\TT}& s     &   0      \X {\tt W}
& channel width \sym{\W}& m     &   {\tt DEFL}\X } }

\noindent\underline{\sl \large AC Analysis}\\[0.1in]
\index{MOSFET, AC Analysis}. The AC analysis uses the model of
figure  \ref{m.ps} with the capacitor values evaluated at the \dc\
operating point with
\begin{equation}
g_m = {{\textstyle\partial I_{DS}} \over {\textstyle\partial
V_{GS}}}
\end{equation}
and
\begin{equation}
R_{DS} = {{\textstyle\partial I_{DS}} \over {\textstyle\partial
V_{DS}}}
\end{equation}
\vshift
\noindent\underline{\sl \large Noise Analysis}\\[0.1in]
\index{MOSFET, Noise Model} \index{MOSFET, Noise Analysis}. The
MOSFET noise model accounts for thermal noise generated in the
parasitic resistances and shot and flicker noise generated in the
drain source current generator. The rms (root-mean-square) values
of thermal noise current generators shunting the four parasitic
resistance $R_B$, $R_D$, $R_G$ and $R_S$ are
\begin{eqnarray}
I_{n,B} &=& \sqrt{4kT/R_B}~\mbox{A/}\sqrt{\mbox{Hz}}\\
I_{n,D} &=& \sqrt{4kT/R_D}~\mbox{A/}\sqrt{\mbox{Hz}}\\
I_{n,G} &=& \sqrt{4kT/R_G}~\mbox{A/}\sqrt{\mbox{Hz}}\\
I_{n,S} &=& \sqrt{4kT/R_S}~\mbox{A/}\sqrt{\mbox{Hz}}
\end{eqnarray}
The rms value of noise current generators in series with the
drain-source current generator
\begin{equation}
I_{n,DS} = \left( I_{\ms{SHOT},DS}^2 + I_{\ms{FLICKER},DS}^2
\right)^{1/2}
\end{equation}
\begin{eqnarray}
I_{\ms{SHOT},DS} &=& \sqrt{4kTg_m\frac{2}{3}}
~~~~\mbox{A/}\sqrt{\mbox{Hz}}
~\mbox{A/}\sqrt{\mbox{Hz}}\\
I_{\ms{FLICKER},DS} &=& \sqrt{{{\textstyle\KF I_{D}^{\AF}}
                         \over {\textstyle f K_{\ms{CHANNEL}}}}}
~~~~\mbox{A/}\sqrt{\mbox{Hz}}
\end{eqnarray}
where the transconductance
\begin{equation}
g_m = {{\textstyle\partial I_{D}} \over {\textstyle\partial
V_{GS}}}
\end{equation}
is evaluated at the \dc\ operating point, and
\begin{equation}
K_{\ms{CHANNEL}} = {{\textstyle\partial
L_{\ms{EFF}}^2\epsilon_{\ms{Si}}} \over
                   {\textstyle\partial \TOX}}
\end{equation}
\newline
\linethickness{0.5mm} \line(1,0){425}
\newline
\textit{Notes:}\\
The actual element in \FDA are the \texttt{nmos3}, \texttt{pmos3}
and \texttt{mos14} elements. See \texttt{nmos3}, \texttt{pmos3}
and \texttt{mos14} for full documentation.\\
%%%%%%%%%%%%%%%%%%%%%%%%%%%%%%%%%%%%%%%%%%%%% Credits
\linethickness{0.5mm} \line(1,0){425}
\newline
\textit{Credits:}\\
\begin{tabular}{l l l l}
Name & Affiliation & Date & Links \\
Nikhil Kriplani & NC State University & Sept 2000 & \epsfxsize=1in\pfig{logo.eps} \\
nmkripla@eos.ncsu.edu & & & www.ncsu.edu    \\
\end{tabular}
%\end{document}
