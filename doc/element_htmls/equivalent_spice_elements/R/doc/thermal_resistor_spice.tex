\documentclass{article}
\usepackage{epsf}

\newcommand{\fig}[1]{J:/eos.ncsu.edu/users/m/mbs/mbs_group/figures/#1}
%\newcommand{\fig}[1]{../figures/#1}
\newcommand{\pfig}[1]{\pfig{\fig{#1}}}

\oddsidemargin 10mm \topmargin 0.0in \textwidth 5.5in \textheight 9.5in
\evensidemargin 1.0in \headheight 0.18in  \footskip 0.4in
%%%%%%%%%%%%%%%%%%%%%%%%%%%%%%%%%%%%%%%% The title
\linethickness{1mm} \line(1,0){425} \normalsize
\newline
\begin{document}
\LARGE \textbf{Resistor} \hspace{50mm} \Huge \textbf{R}
\newline
%%%%%%%%%%%%%%%%%%%%%%%%%%%%%%%%%%%%%%%% the resistor figure
\begin{figure}[h]
\centerline{\epsfxsize=2in\pfig{thermal_resistor.eps}} \caption{R
--- Resistor element.}
\end{figure}
\newline
%%%%%%%%%%%%%%%%%%%%%%%%%%%%%%%%%%%%%%%%%%% SPICE form
\linethickness{0.5mm} \line(1,0){425}
\newline
\texttt{SPICE} \textit{Form}:
\newline
\texttt{R}\textit{name} $n_1$ $n_2$ \textit{ResistorValue} \
[\texttt{IC}=$V_D$]
\newline
%%%%%%%%%%%%%%%%%%%%%%%%%%%%%%%%%%%%%%%%%%%%%%% explanation of terms in the SPICE form
\begin{tabular}{r l}
$n_1$ & is the positive element node  \\
& \\
$n_2$ & is the negative element node \\
&  \\
\textit{ResistorValue} & is the resistance \\
&  \\
\texttt{IC}  & is the optional initial condition specification. Using \texttt{IC}=$V_D$ is used with the \textit{UIC} option\\
             & rather than with the quiescent operating point.  Specification of  the transient initial \\
             & condition using the .\texttt{IC} is preferred and is more convenient.\\
\end{tabular}
%%%%%%%%%%%%%%%%%%%%%%%%%%%%%%%%%%%%%%%%%%%%%%%%%%%%%%%%%%%%%%%%%%%%% example in SPICE
\textit{Example}:
\newline
\linethickness{0.5mm}
\line(1,0){425}
\newline
\texttt{R1 1 2 3.5MEG}
\newline
%%%%%%%%%%%%%%%%%%%%%%%%%%%%%%%%%%%%%%%%%%%%%%% Parameter table
\textit{Model Parameters}:
\newline
%%%%%%%%%%%%%%%%%%%%%%%%%%%%%%%%%%%%%%%%%%%
\begin{tabular}{|r|l|c|c|}
\hline
\textbf{Name} & \textbf{Description} & \textbf{Units} & \textbf{Default} \\
\hline
\texttt{R0} & Resistance Value  & Ohm  &  \\
\hline
\texttt{L} & Length & Meters &  \\
\hline
\texttt{W} & Width & Meters & 1e-6 \\
\hline
\texttt{T} & System Temperature & Celsius &  \\
\hline
\texttt{RSH} & Sheet Resistance & Ohms/sq &  \\
\hline
\texttt{DEFW} & Default device width & meters & 1e-6 \\
\hline
\texttt{NARROW} & Narrowing due to side etching  & meters & 0.0 \\
\hline
\texttt{TNOM} & Nominal temperature & Celsius & 27 \\
\hline
\texttt{TC1}  & Temperature coefficient & $1/Celsius$ & 0 \\
\hline
\texttt{TC2} & Temperature coefficient  & $1/Celsius$ & 0 \\
\hline
\texttt{PDR} & Power dependent resistor & & false \\
\hline
\end{tabular}
\hspace{1.5in} Table 1
\newline
%%%%%%%%%%%%%%%%%%%%%%%%%%%%%%%%%%%%%%%%%%%%%%%%%%% Device equations start
\textit{DEVICE EQUATIONS}:
\newline
\line(1,0){425}
\newline
%%%%%%%%%%%%%%%%%%%%%%%%%%%%%%%%%%%%%%%%%%%% Current characteristics
\underline{\textit{Current characteristics}}:
\newline
\begin{equation}
V_R = R * I_R
\end{equation}
\begin{equation}
P = V_R * I_R
\end{equation}
where $P$ is the power dissipated by the resistor.
\newline
%%%%%%%%%%%%%%%%%%%%%%%%%%%%%%%%%%%%% Resistance equations
\underline{\textit{Resistance value}}:
\newline
The resistor model consists of process-related device data that
allow the resistance to be calculated from geometric information
and to be corrected for temperature.

The sheet resistance is used with the narrowing parameter and L
and W from the resistor line to determine the nominal resistance
by the formula.
\begin{equation}
R= \left\{ \begin{array}{ll}
            R_0(1+T_{C1}(T-T_{NOM})+T_{C2}(T-T_{NOM})^{2}) & if~R_0~is~specified\\
& \\
            R_{MULTIPLIER}(1+T_{C1}(T-T_{NOM})+T_{C2}(T-T_{NOM})^{2}) & if R_0~is~not~specified \\
             \end{array}
    \right.
\end{equation}
where ~~~R_{MULTIPLIER} = R_{SH}(L-NARROW)/(W-NARROW)
\newline
%%%%%%%%%%%%%%%%%%%%%%%%%%%%%%%%%%%%%%
\underline{\textit{NOTE}}: \\
It is important to note that the resistor will behave just as a
normal resistor if the parameter ``pdr'' is 0 i.e. the system
temperature will be fixed during the simulation. However, if
``pdr'' is 1, the resistor will function as an electro-thermal
resistor.
%%%%%%%%%%%%%%%%%%%%%%%%%%%%%%%%%%%%%%%%%%%%% Credits
\newline
\linethickness{0.5mm} \line(1,0){425}
\newline
\underline{\textit{Credits}}:
\newline
\begin{tabular}{l l l l}
Name & Affiliation & Date & Logo \\
Houssam S. Kanj & NC State University & June 2001 & \epsfxsize=1in\pfig{logo.eps} \\
houssam@ieee.org & & & www.ncsu.edu \\
\end{tabular}
\end{document}
