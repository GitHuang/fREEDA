%\documentclass{article}
%\usepackage{epsf}
%\newcommand{\fig}[1]{J:/eos.ncsu.edu/users/m/mbs/mbs_group/figures/#1}
%\newcommand{\fig}[1]{../figures/#1}
%\newcommand{\pfig}[1]{\epsfbox{\fig{#1}}}
%\newcommand{\ms}[1]{\mbox{\scriptsize #1}}
%\newcommand{\B}{{ \rm [}}     % begin optional parameter in \form{}
%\newcommand{\E}{{\ \rm\hspace{-0.04in}] }}   % end optional parameter in \form{}

\oddsidemargin 10mm \topmargin 0.0in \textwidth 5.5in \textheight 7.375in
\evensidemargin 1.0in \headheight 0.18in \footskip 0.16in
%%%%%%%%%%%%%%%%%%%%%%%%%%%%%%%%%%%%%%%% The title
%\begin{document}
\section[V \- Independent Voltage Source(Exponential)]{\noindent{\LARGE \textbf{Independent Voltage Source} \hspace{55mm}\huge \textbf{V}}}
%\newline
\linethickness{1mm}
\line(1,0){425}
\normalsize
%%%%%%%%%%%%%%%%%%%%%%%%%%%%%%%%%%%%%%%% the resistor figure
\begin{figure}[h]
\centerline{\epsfxsize=0.75in\pfig{v_spice.eps}}
\caption{Independent Voltage Source Element.}
\end{figure}
%%%%%%%%%%%%%%%%%%%%%%%%%%%%%%%%%%%%%%%% form for \FDA
\linethickness{0.5mm} \line(1,0){425}
\newline
\textit{Form:}
\newline
\texttt{V}\textit{name}$N_{+}$$N_{-}$[[\textit{DC}] \
[\textit{DCvalue}] \
[\texttt{AC}[\textit{ACmagnitude}[\textit{ACphase}]]] \
[\texttt{DISTOF1}[\textit{F1magnitude}[\textit{F1phase}]]] \
[\texttt{DISTOF2}[\textit{F2magnitude}[\textit{F2phase}]]]]
\newline
%%%%%%%%%%%%%%%%%%%%%%%%%%%%%%%%%%%%%%%%
\begin{tabular}{r l}
$N_{+}$ & is the positive voltage source node.\\
$N_{-}$ & is the negative voltage source node.\\
\texttt{DC} & is the optional keyword for the \texttt{dc} value of
the source.\\
\textit{DCvalue} & is the \texttt{dc} voltage value of the source.(Units: V; Optional; Default: 0; Symbol: $V_{DC}$)\\
\texttt{AC} & is the keyword for the \texttt{ac} value of the source.\\
\textit{ACmagnitude} & is the \texttt{ac} magnitude of the source
used during \texttt{ac} analysis. That is, it is the peak\\
& \texttt{ac} voltage so that the \texttt{ac} signal is $\textit{ACmagnitude}\,\mbox{sin}(\omega t + \mbox{ACphase})$. \textit{ACmagnitude} is\\
& ignored for other types of analyses. (Units: V; Optional; Default: 1; Symbol:$V_{AC}$)\\
\textit{ACphase} & is the ac phase of the source. It is used only in \texttt{ac} analysis.\\
& (Units: Degrees; Optional; Default: 0; Symbol:$\phi_{\ms{AC}}$)\\
\texttt{DISTOF1} & is the distortion keyword for distortion component 1 which has frequency \texttt{F1}.\\
\textit{F1magnitude} & is the magnitude of the distortion component at \texttt{F1}. See \texttt{.DISTOF1} keyword above.\\
& (Units: V; Optional; Default: 1; Symbol: $V_{F1}$)\\
\textit{F1phase} & is the phase of the distortion component at \texttt{F1}. See \texttt{.DISTOF1} keyword above.\\
& (Units: Degrees; Optional; Default: 0; Symbol: $\phi_{F1}$)\\
\texttt{DISTOF2} & is the distortion keyword for distortion component 2 which has frequency {\tt F2}.\\
\textit{F2magnitude} & is the magnitude of the distortion component at \texttt{F2}. See \texttt{.DISTOF2} keyword above.\\
& (Units: V; Optional; Default: 1; Symbol: $V_{F2}$)\\
\textit{F2phase} & is the phase of the distortion component at \texttt{F2}. See \texttt{.DISTOF2} keyword above.\\
& (Units: Degrees; Optional; Default: 0; Symbol: $\phi_{F2}$)\\
\end{tabular}
%%%%%%%%%%%%%%%%%%%%%%%%%%%%%%%%%%%%%%% Parameter list
\newline
\underline{\bf{Exponential}}:\\
%%%%%%%%%%%%%%%%%%%%%%put the SIN form here%%%%%%%%%%%%%%%%%%%%
\texttt{EXP($V_1$ $V_2$ \B $T_{D1}$ \E \B $\tau_1$ \E
       \B $T_{D2}$ \E \B$\tau_2$ \E)}\\
\textit{Parameters:}
\begin{table}[tbph]
\begin{tabular}{|c|c|c|c|}
\hline
Name&Description&Units&Default\\
\hline
$V_1$& initial voltage & V & \scriptsize{REQUIRED}\\
\hline
$V_2$ & pulsed voltage & V & \scriptsize{REQUIRED}\\
\hline
$T_{D1}$ & rise delay time & s & 0.0\\
\hline
$\tau_1$ & rise time constant & s & {\tt TSTEP}\\
\hline
$T_{D2}$ & fall delay time & s & {\tt TSTEP}\\
\hline
$\tau_2$ & fall time constant & s &  {\tt TSTEP}\\
\par
\hline
\end{tabular}
\end{table}
%%%%%%%%%%%%%%%%%%%%%%%%%%%%%%%%%%%%%%% example in \FDA
\newline
\linethickness{0.5mm} \line(1,0){425}
\newline
\textit{Example:}
\newline
\texttt{VSIGNAL \ 2\ 0\ EXP(0.1 0.8 1 0.35 2 1)}
\newline
\linethickness{0.5mm} \line(1,0){425}
\newline
\textit{Description:}\\
The exponential transient is a single-shot event specifying two
exponentials. The voltage is $V_1$ for the first $T_{D1}$ seconds
at which it begins increasing exponentially towards $V_2$ with a
time constant of $\tau_1$ seconds.  At time $T_{D2}$ the voltage
exponentially decays towards $V_1$ with a time constant of
$\tau_2$. That is,
\begin{equation}
v = \left\{ \begin{array}{ll}
     V_1                                           & t \le T_{D1}\\
     V_1+(V_2-V_1)(1-e^{\textstyle (-(t-T_{D1})/\tau_1)})  & T_{D1} < t \le T_{D2}\\
     V_1+(V_2-V_1)(1-e^{\textstyle (-(t-T_{D1})/\tau_1)})
        +(V_1-V_2)(1-e^{\textstyle (-(t-T_{D2})/\tau_2)})  &  t > T_{D2}
     \end{array} \right. %}
\end{equation}
\begin{figure}[h]
\centerline{\epsfxsize=3in\pfig{vexp.eps}} \caption{Voltage source
exponential ({\tt EXP}) waveform]{Voltage source exponential ({\tt
EXP}) waveform for {\tt EXP(0.1 0.8 1 0.35 2 1)}
\label{fig:vexp:spice} }}
\end{figure}
\newline
\linethickness{0.5mm} \line(1,0){425}
\newline
\textit{Notes:}\\
The actual element in \FDA is the \texttt{vexp} element.
See \texttt{vexp} for full documentation.\\
%%%%%%%%%%%%%%%%%%%%%%%%%%%%%%%%%%%%%%%%%%%%% Credits
\linethickness{0.5mm} \line(1,0){425}
\newline
\textit{Credits:}\\
\begin{tabular}{l l l l}
Name & Affiliation & Date & Links \\
Satish Uppathil & NC State University & Sept 2000 & \epsfxsize=1in\pfig{logo.eps} \\
svuppath@eos.ncsu.edu & & & www.ncsu.edu    \\
\end{tabular}
%\end{document}
