%\documentclass{article}
%\usepackage{epsf}
%\newcommand{\fig}[1]{J:/eos.ncsu.edu/users/m/mbs/mbs_group/figures/#1}
%\newcommand{\fig}[1]{../figures/#1}
%\newcommand{\pfig}[1]{\epsfbox{\fig{#1}}}

\oddsidemargin 10mm \topmargin 0.0in \textwidth 5.5in \textheight
7.375in \evensidemargin 1.0in \headheight 0.18in \footskip 0.16in
%%%%%%%%%%%%%%%%%%%%%%%%%%%%%%%%%%%%%%%% The title
%\begin{document}
\section[C \- Capacitor]{\noindent{\LARGE \textbf{Capacitor}} \hspace{107mm}\huge \textbf{C}}
%\newline
\linethickness{1mm} \line(1,0){425} \normalsize
%\newline
%%%%%%%%%%%%%%%%%%%%%%%%%%%%%%%%%%%%%%%% the resistor figure
\begin{figure}[h]
\centerline{\epsfxsize=1in\pfig{c_spice.ps}} \caption{C ---
Capacitor Element.}
\end{figure}
%\newline
%%%%%%%%%%%%%%%%%%%%%%%%%%%%%%%%%%%%%%%%%%% SPICE form
%\vspace{2mm}
\newline
\linethickness{0.5mm} \line(1,0){425}
\newline
\texttt{SPICE} \textit{Form:}
\newline
\texttt{C}\textit{name} $n_1$ $n_2$
\texttt{[}\textit{ModelName}\texttt{]} \textit{Capacitor Value}
[\texttt{IC}=$V_C$] [\texttt{L}=Length] [\texttt{W}=Width]
\newline
%\vspace{2mm}
%%%%%%%%%%%%%%%%%%%%%%%%%%%%%%%%%%%%%%%%%%%%%%% explanation of terms in the SPICE form
\newline
\begin{tabular}{r l}
$n_1$ & is the positive element node, \\
$n_2$ & is the negative element node, \\
\textit{ModelName} & is the optional model name,(\textit{ModelType} is \texttt{CAP}.) \\
\textit{CapacitorValue} & is the capacitance, (Units: Farads; Required) \\
\texttt{L}   & length of the integrated capacitor, (Units: m; Required; Symbol \textit{L}) \\
\texttt{W}   & is the width of the integrated capacitor, (Units: m; Optional, with the \\
             & default width \texttt{DEFW} specified in the device model; Symbol {L}), \\
\texttt{IC}  & is the optional initial condition specification. Using \texttt{IC}=$V_C$ is used with \\
             & the \textit{UIC} option on the .\textit{TRAN} line when a transient analysis is desired  \\
             & with initial voltage $V_C$ across the capacitor rather than the quiescent  \\
             & operating point. Specification of  the transient initial condition using the  \\
             & .\texttt{IC} is preferred and is more convenient.
\end{tabular}
%\newline
%%%%%%%%%%%%%%%%%%%%%%%%%%%%%%%%%%%%%%%%%%%%%%% Parameter table
%\vspace{4mm}
\newline
\textit{Model Parameters:}
\newline

%%%%%%%%%%%%%%%%%%%%%%%%%%%%%%%%%%%%%%%%%%%%%% Parameters
\begin{tabular}{|r|l|c|c|}
\hline
\textbf{Name} & \textbf{Description} & \textbf{Units} & \textbf{Default} \\
\hline
\texttt{C} & Capacitor Value & farads & - \\
\hline
\texttt{intg} & Internal conductance value & siemens & - \\
\hline
\texttt{timed} & Flag: if true then calculate in the time domain & - & - \\
\hline
\end{tabular}
%\vspace{4mm}
%%%%%%%%%%%%%%%%%%%%%%%%%%%%%%%%%%%%%%%%%%%%%%%%%%%%%%%%%%%%%%%%%%%%% example in SPICE
\newline
\linethickness{0.5mm} \line(1,0){425}
\newline
\textit{Example:}
\newline
\texttt{C \ 1 \ 3 C1 \ 5pF}
\newline
\linethickness{0.5mm} \line(1,0){425}
\newline
\textit{Notes:}\\
The actual element in \FDA is the \texttt{C} element. See
\texttt{C} for full documentation.\\
%%%%%%%%%%%%%%%%%%%%%%%%%%%%%%%%%%%%%%%%%%%%% Credits
%\newline
%\vspace{4mm}
%\newline
\linethickness{0.5mm} \line(1,0){425}
\newline
\textit{Credits:}
\newline
\begin{tabular}{l l l l}
Name & Affiliation & Date & Links \\
Carlos E. Christofferson & NC State University & Sept 2000 & \epsfxsize=1in\pfig{logo.eps} \\
cechrist@ieee.org & & & www.ncsu.edu    \\
\end{tabular}
%\end{document}
