\documentclass{article}
\usepackage{epsf}
\usepackage{graphics}
\newcommand{\fig}[1]{/#1}
\newcommand{\pfig}[1]{\epsfbox{\fig{#1}}}
\newcommand{\TPG}{T_{PG}}
\newcommand{\PHI}{2\phi_{B}}
\newcommand{\GAMMA}{\gamma}
\newcommand{\FDA}{FreeDa}
\newcommand{\LAMBDA}{\lambda}

\oddsidemargin 10mm \topmargin 0.0in \textwidth 5.5in \textheight 7.375in
\evensidemargin 1.0in \headheight 0.18in \footskip 0.16in
%%%%%%%%%%%%%%%%%%%%%%%%%%%%%%%%%%%%%%%% The title
\begin{document}
{\noindent{\LARGE \textbf{N Channel MOSFET level 1} \hspace{25mm}
\textbf{mosn1}}}
%\newline
\linethickness{1mm}
\line(1,0){425}
\normalsize
%\newline
%%%%%%%%%%%%%%%%%%%%%%%%%%%%%%%%%%%%%%%% the resistor figure
\begin{figure}[h]
\centerline{\epsfxsize=4in\epsfbox{figures/m.ps}} \caption{N Channel MOSFET
Level 1 model}
\end{figure}
%%%%%%%%%%%%%%%%%%%%%%%%%%%%%%%%%%%%%%% form for \FDA
\linethickness{0.5mm} \line(1,0){425}
\newline
\textit{Form:}
\newline
$\tt mosn1$:$\langle \tt{instance\ name}\rangle$ $n_1\ n_2\ n_3\
n_4\ $ $\langle \tt{parameter\ list}\rangle$
\newline
%%%%%%%%%%%%%%%%%%%%%%%%%%%%%%%%%%%%%%%%
\begin{tabular}{r l}
instance name & is the model name, \\
$n_1$ & is the drain node, \\
$n_2$ & is the gate node, \\
$n_3$ & is the source node, \\
$n_4$ & is the bulk node. \\
\end{tabular}
%%%%%%%%%%%%%%%%%%%%%%%%%%%%%%%%%%%%%%% Parameter list
\newline
\textit{Parameters:}
\begin{table}[tbph]
\begin{tabular}{|l|c|c|c|}
\hline
Parameter&Type&Default value&Required?\\
\hline
vt0: Zero bias threshold voltage (V)& DOUBLE & 0 & no\\
\hline
kp: Transconductance parameter ($\textrm{A}/\textrm{V}^2$) & DOUBLE & $2\times10^{-5}$ & no\\
\hline
gamma: Bulk threshold parameter ($\textrm{V}^{0.5}$) & DOUBLE & 0 & no\\
\hline
phi: Surface inversion potential (V) & DOUBLE & 0.6 & no\\
\hline
lambda: Channel-length modulation ($\textrm{1/V}$) & DOUBLE & 0 & no\\
\hline
pb: Bulk junction potential (V) & DOUBLE & 0.8 & no\\
\hline
tox: Oxide thickness (m)& DOUBLE & $1\times10^{-7}$ & no\\
\hline
ld: Lateral diffusion length (m) & DOUBLE & 0 & no\\
\hline
u0: Surface mobility ($\textrm{cm}^2/\textrm{V-s}$)& DOUBLE & 600 & no\\
\hline
fc: Forward bias junction fit parameter & DOUBLE &  0.5 & no\\
\hline
nsub: Substrate doping ($\textrm{cm}^{-3}$) & DOUBLE & $1\times10^{15}$ & no\\
\hline
tpg: Gate material type & DOUBLE & 1 & no\\
\hline
nss: Surface state density ($\textrm{cm}^{-2}$) & DOUBLE & 0 & no\\
\hline
tnom: Nominal temperature (C)& DOUBLE & 27 & no\\
\hline
t: Device temperature (C)& DOUBLE & 27 & no\\
\hline
l: Device length (m)& DOUBLE & $2\times10^{-6}$ & no\\
\hline
w: Device width (m)& DOUBLE & $50\times10^{-6}$ & no\\
\hline
alpha: Impact ionization current coefficient & DOUBLE & 0 & no\\
\hline

\end{tabular}
\end{table}

%%%%%%%%%%%%%%%%%%%%%%%%%%%%%%%%%%%%%%% example in \FDA
%\newline
\linethickness{0.5mm} \line(1,0){425}
\newline
\textit{Example:}
\newline
\texttt{mosn1:m1\ 2\ 3\ 0\ 0\ l=1.2u w=20u}
\newline
\linethickness{0.5mm} \line(1,0){425}
\newline
\textit{Description:}\\
\FDA{} has the NMOS level 1 model based on the MOS level 1,
Schichman-Hodges model in SPICE. The model uses the charge
conservative Yang-Chatterjee model for modeling charge and capacitance. \\
%%%%%%%%%%%%% Equations follow below

The physical constants used in the model evaluation are
\begin{center}
\begin{tabular}{|l|l|l|}
\hline
$k$ & Boltzmann's constant           &  $1.3806226\,10^{-23}$~J/K\\
$q$ & electronic charge             & $1.6021918\,10^{-19}$~C\\
$\epsilon_0$& free space permittivity &  $8.854214871\,10^{-12}$~F/m\\
$\epsilon_{Si}$& permittivity of silicon &  $11.7 \epsilon_0$\\
$\epsilon_{OX}$& permittivity of silicon dioxide &  $3.9 \epsilon_0$\\
$n_i$&intrinsic concentration of silicon @ 300~K& $1.45\,10^{16}~\mbox{m}^{-3}$\\
\hline
\end{tabular}
\end{center}

%vbi is $F_{\ms{FB}}$\\
%wkfng is $\phi_{\ms{GATE}}$ ???\\
%wkfngs is $\phi_{\ms{MS}}$\\
%egfet is $E_G$ ??? \\
\vfill
\noindent\underline{\sl \large Standard Calculations}\\[0.1in]
Absolute temperatures (in kelvins, K) are used. The thermal
voltage
\begin{equation}
V_{TH}(T_{NOM}) = {{kT_{NOM}} \over q} .
\end{equation}
\noindent The silicon bandgap energy
\begin{equation}
E_G(T_{NOM})=1.16 - 0.000702{{4T_{NOM}^2} \over {T_{NOM}+1108}} .
\end{equation}
The difference of the gate and bulk contact potentials
\begin{equation}
\phi_{\tt{MS}} = \phi_{GATE} - \phi_{BULK} .
\end{equation}
The gate contact potential
\begin{equation}
\phi_{GATE} = \left\{ \begin{array}{ll}
        3.2 & \mbox{$\TPG = 0$}\\
        3.25& \mbox{NMOS \& $\TPG = 1$}\\
        3.25 + E_G     & \mbox{NMOS \& $\TPG = -1$}\\
        \end{array} \right. .
    %}
\end{equation}

\noindent The contact potential of the bulk material
\begin{equation}
\phi_{BULK} = 3.25 + E_G \mbox{ \ for  NMOS}\\
\end{equation}

\noindent The capacitance per unit area of the oxide is
\begin{equation}
C_{OX} = {{\epsilon_{OX}} \over {\tt TOX}} .
\end{equation}
The effective length $L_{EFF}$ of the channel is reduced by the
amount {\tt LD} of the lateral diffusion at the source and drain
regions:
\begin{equation}
L_{EFF} = {\tt L} - 2{\tt LD}
\end{equation}
Similarly the effective length $W_{EFF}$ of the channel is reduced
by the amount {\tt WD} of the lateral diffusion at the edges of
the channel.
\begin{equation}
W_{EFF} = {\tt W}  - 2{\tt WD}
\end{equation}

\vfill \noindent\underline{\sl \large Process Oriented Model}
\index{MOSFET, Process Oriented Model}
\\[0.1in]
If omitted, device parameters are computed from process parameters
using defaults if necessary provided that both {\tt TOX} and {\tt
NSUB} are specified.  If either {\tt TOX} or {\tt NSUB} is not
specified then the critical device parameters must be specified.\\

\noindent If {\tt KP} is not specified in the model statement then
\begin{equation}
{\tt KP} = \mu_{0} / C_{OX}
\end{equation}

\noindent If {\tt PHI} is not specified, then it is evaluated as
\begin{equation}
{\tt PHI} = \PHI = 2V_{TH}(T_{NOM})\ln{{\tt NSUB} \over {n_{i}}}
\end{equation}

\noindent If {\tt GAMMA} is not specified in the model statement
then
\begin{equation}
{\tt GAMMA} = \GAMMA = {{\sqrt{2\epsilon_{Si} q NSUB}} \over
{C_{OX}}} \label{gammaomitted}
\end{equation}

\noindent If {\tt VTO} is not specified in the model statement
then it is evaluated as
\begin{equation}
{\tt VTO} = V_{FB} + {\GAMMA} \sqrt{\PHI} + \PHI
\end{equation}

where $V_{FB}$ is
\begin{equation}
V_{FB} = \phi_{MS} - { q \ \tt {NSS} \over {C_{OX}}}
\end{equation}

\vfill \noindent\underline{\sl \large Temperature Dependence}
\index{MOSFET, Temperature Dependence} \index{Temperature
Dependence, see MOSFET}
\\[0.1in]
Temperature effects are incorporated as follows where $T$ and
$T_{\tt{NOM}}$ are absolute temperatures in Kelvins (K).
\begin{eqnarray}
V_{TH} & = & {{kT} \over q}\\
KP(T) & = & {\tt KP}  (T_{NOM}/T)^{3/2} \\
\mu (T) & = & \mu_{0}  (T_{NOM}/T)^{3/2} \\
\PHI (T) & = & \PHI {T \over {T_{NOM}}} - 3V_{TH} \ln{{T \over
{T_{NOM}}}
 } + E_G (T_{NOM} -E_G(T)
\end{eqnarray}

\vfill \noindent\underline{\sl \large DC Current
Calculations}\\[0.1in]

\noindent The Schichman-Hodges model computes the current of the
MOS device in only three regions, cutoff, linear and saturation.
The regions are defined as:\\[0.1in]
\hspace*{\fill}\parbox{5in}{
\begin{tabular}{ll}
cutoff region:&$V_{GS} < V_T$\\
linear region:&$V_{GS} > V_T$ and $V_{DS} < V_{GS}-V_{T}$\\
saturation region:&$V_{GS} > V_T$ and $V_{DS} > V_{GS}-V_{T}$\\
\end{tabular}}\\[0.1in]
where the threshold voltage is
\begin{equation}
V_{T} = \left\{ \begin{array}{ll}
          V_{FB} + \PHI + \GAMMA \sqrt{\PHI - V_{BS}}
               & V_{BS} \ge \PHI \\
          V_{FB} + \PHI & V_{BS} < \PHI
          \end{array} \right. %}
\end{equation}

\noindent Then
\begin{equation}
I_{D} = \left\{ \begin{array}{ll}
      0  & \mbox{cutoff region} \\ \\
      { {\textstyle W_{EFF}} \over {\textstyle L_{EFF}}}
      {{\textstyle KP} \over {\textstyle 2}}
      ( 1+\LAMBDA V_{DS})
      V_{DS} \left[2(V_{GS}-V_{T})-V_{DS}\right]
         &\mbox{linear region} \\ \\
      { {\textstyle W_{EFF}} \over {\textstyle L_{EFF}}}
      {{\textstyle KP} \over {\textstyle 2}}
      (1+\LAMBDA V_{DS})
      \left[V_{GS}-V_{T}\right]^2
         &\mbox{saturation region} \end{array} \right. %}
      \label{mlevel1ids}
\end{equation}


\vfill \noindent\underline{\sl \large Yang-Chatterjee
ChargeModel}\\[0.1in]

\noindent Once the DC channel current has been calculated, the
charge at each terminal of the MOS device is computed.  The charge
values are used to compute the AC current flow through each
terminal according to the Yang-Chatterjee model.  This model
ensures continuity of the charges and capacitances throughout
different regions of operation. The intermediate quantities are:

\begin{equation}
V_{FB} = V_{to} - \GAMMA \, \sqrt{\PHI} - \PHI
\end{equation}
and
\begin{equation}
C_o = C_{OX} \, W_{eff} \, L_{eff}
\end{equation}

\noindent The charge is computed for for each of the four regions
of operation, accumulation, cut-off, saturation and linear.\\

\textit{Accumulation region} $V_{GS} \leq V_{FB} + V_{BS}$
\begin{eqnarray}
Q_d & = & 0\\
Q_s & = & 0\\
Q_b & = & -C_o \, (V_{GS} - V_{FB} - V_{BS})
\end{eqnarray}

\textit{Cut-off region} $V_{FB} + V_{BS} < V_{GS} \leq V_{th}$
\begin{eqnarray}
Q_d & = & 0\\
Q_s & = & 0\\
Q_b & = & -C_o \, {\texttt{GAMMA}^2 \over 2} \, \{-1 + \sqrt{1 +
{4 \, (V_{GS} - V_{FB} - V_{BS}) \over \texttt{GAMMA}^2}}\}
\end{eqnarray}

\textit{Saturation region} $V_{th} < V_{GS} \leq V_{DS} + V_{th}$
\begin{eqnarray}
Q_d & = & 0\\
Q_s & = & -{2 \over 3} \, C_o \, (V_{GS} - V_{th}) \\
Q_b & = & C_o \, (V_{FB}\, \texttt{PHI} - V_{th})
\end{eqnarray}

\textit{Linear region} $V_{GS} > V_{DS} + V_{th}$
\begin{eqnarray}
Q_d & = & -C_o \, [{V_{DS}^2 \over 8\,(V_{GS} - V_{th} - {V_{DS} \over 2})} + {V_{GS} - V_{th} \over 2} - {3 \over 4}\,V_{DS}] \\
Q_s & = & -C_o \, [{V_{DS}^2 \over 24\,(V_{GS} - V_{th} - {V_{DS} \over 2})} + {V_{GS} - V_{th} \over 2} + {1 \over 4}\,V_{DS}] \\
Q_b & = & C_o \, (V_{FB}\, \texttt{PHI} - V_{th})
\end{eqnarray}

\noindent The final currents at the transistor nodes are given by
\begin{eqnarray}
I_d & = & I_{D} + {dQ_d \over dt} \\
I_g & = & {dQ_g \over dt} \\
I_s & = & -I_{D} + {dQ_s \over dt}
\end{eqnarray}

%\newpage
%\begin{figure}[h] \leavevmode \hspace*{\fill}
%\epsfxsize=2.5in\pfig{nmos_dc.ps} \hspace{\fill} \hspace{\fill}
%\epsfxsize=2.5in\pfig{nmos_tran.ps}
%\hspace*{\fill}\\
%\hspace*{\fill} (a) \hspace{\fill} \hspace{\fill} (b)
%\hspace*{\fill} \caption {\label{NMOS Analysis results} Analysis
%results.}
%\end{figure}

%\begin{figure}[h]
%  \begin{center}
%    \begin{tabular}{c c}
%      \resizebox{60mm}{!}{\includegraphics{nmos_dc.ps}} & \resizebox{60mm}{!}{\includegraphics{nmos_tran.ps}} \\
%    \end{tabular}
%    \caption{Analysis results}
%    \label{NMOS Analysis results}
%  \end{center}
%\end{figure}
%\noindent\linethickness{0.5mm} \line(1,0){425}

\noindent\textit{Notes:}\\
This is the \texttt{M} element in the SPICE compatible netlist.
The unmodified Yang-Chatterjee charge model has a charge partition
scheme in the saturation region that sets the drain charge to
zero.  This results in
a loss of the high-frequency current roll-off at the drain node in saturation. \\
%%%%%%%%%%%%%%%%%%%%%%%%%%%%%%%%%%%%%%%%%%%%% Credits
\linethickness{0.5mm} \line(1,0){425}
\newline
\textit{Credits:}\\
\begin{tabular}{l l l l}
Name & Affiliation & Date & Links \\
Aaron Walker & NC State University & August 2002 & www.ncsu.edu \\
nmkripla@unity.ncsu.edu & & & www.ncsu.edu
\end{tabular}
\end{document}
