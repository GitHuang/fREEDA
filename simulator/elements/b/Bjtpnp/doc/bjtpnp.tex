\documentclass{article}
\usepackage{epsf}\usepackage{here}
% SUMMARY OF USEFUL MACROS
%
% 1. \marginpar[LeftText]{RightText} Standard Latex \marginpar
%
% 2. \mymarginpar{MarginText}{BodyText} Places MarginText in margin and
%     BodyText in main text opposite margin. Places lineas above and below
%     text.  Used in element and model catalogs and elsewhere.
%
% 3. \marginlabel{text} Places large text in margin and underlines. Used
%    to put element name in margin at top of page for continued
%    descriptions.  Does not automaticly put it at the top of a page
%    and so a \clearpage is required.
%
% 4. \marginid{text} Used to place a short identifier in the margin. Just one
%    word and justified to the outside edge.
%
% 5. \offset a standard indent.
%
% 6. \offsetparbox{} Places text in an offset parbox.  Use
%    \hspace*{\fill} \offsetparbox{text}        to insert an offset text.
%
% 7. \boxed{} similar to above but starts a new line and right justifies
%    offsetparbox.
%
% 8. \form
%
% 9. \example
%
%10. \begin{widelist}  .... \end{widelist} 
%    A widelist that has 1 inch wide labels that has additional left hand
%    margin of 0.6 inch.
%
%11. \spicethreeonly{} Text is inserted only if output is for spice3.
%
%12. \pspiceonly{} 
%    Text is inserted only if output is for pspice (superspice for the most
%    part is upwards compatible to pspice).
%
%13. \sspiceonly{} 
%    Text is inserted only if output is for superspice but not for pspice.

%14. \sym{}
%    Right justifies a symbol in a keyword table.
%
%15. Use the following wherever as then we can have a standard way to
%    report them.  Note that "\dc\ " is required to get a space after dc.
%    \dc
%    \ac
%    \SPICE
%
%16. \kwnote{}  This is a convenient way to include notes in a keyword table.
%    It right justifies a note in the description column and starts it on a
%    newline.
\newcommand{\kwnote}[1]{\newline\hspace*{\fill} #1}

%
%17. \kwversion{} This is a convenient way to indicate versions in a keyword
%    table. It does a \kwnote .  It should not printed when outputing for
%     specific versions.
% typical usage is \kwversion{\sspice ; \hspice} to produce the output
%            VERSIONS: SUPERSPICE; HSPICE
% typical usage is \kwversionNote{\sspice ; \hspice} to produce the output
%                 |                    VERSIONS: SUPERSPICE; HSPICE |
%for unifomity the recommended version order is:
%    \hspice \pspice \sspice \spicetwo \spicethree
\newcommand{\version}[1]{({\sc version: #1})}
\newcommand{\kwversion}[1]{\newline({\sc version: #1})}
\newcommand{\kwversionNote}[1]{\kwnote{({\sc version: #1})}}
\newcommand{\versions}[1]{({\sc versions: #1})}
\newcommand{\kwversions}[1]{\newline({\sc versions: #1})}
\newcommand{\kwversionsNote}[1]{\kwnote{({\sc versions: #1})}}

%


\newcommand{\mymargin}{1in}   % Use to set width of margin notes
\newcommand{\mymarginparsep}{0.1in}
\newcommand{\mymarginparsepplustext}{5.6in}
\newcommand{\mymarginplus}{1.1in}
\newcommand{\mymarginplustext}{6.6in}
\newcommand{\underlinehead}{\\[-0.1in] \rule{\mymarginplustext}{0.01in}}
\newcommand{\overlinefoot}{\rule{\mymarginplustext}{0.01in}\\}
% WIDEPARBOX uses margin space as well.
\newcommand{\wideparbox}[1]{\parbox[t]{\mymarginplustext}{#1}}
%
% HEADER AND FOOT
% Standard header and foot, but includes copyright notice.a
% To omit copyrite notice do not include copyrite.sty
%
\newcommand{\mycopyrite}{}
\def\ps@headings{\let\@mkboth\markboth
\def\@oddfoot{\wideparbox{\overlinefoot\rm\today \hfill \mycopyrite}}
\def\@evenfoot{\hspace*{-\mymarginplus}\wideparbox{\overlinefoot
\mycopyrite\hfill \today}}
\def\@evenhead{\hspace*{-\mymarginplus}\wideparbox{\rm
\thepage\hfill \sl \leftmark \underlinehead }}
\def\@oddhead{\wideparbox{\hbox{}\sl \rightmark \hfill
\rm\thepage \underlinehead}}\def\chaptermark##1{\markboth {\uppercase{\ifnum
\c@secnumdepth
>\m@ne
 \@chapapp\ \thechapter. \ \fi ##1}}{}}\def\sectionmark##1{\markright
 {\uppercase{\ifnum \c@secnumdepth >\z@
  \thesection. \ \fi ##1}}}}

%\oddsidemargin 0.25in \topmargin 0.0in \textwidth 6.5in \textheight 9in
%\evensidemargin 0.25in \headheight 0.18in \footskip 0.16in

%%   EPSF.TEX macro file:
%   Written by Tomas Rokicki of Radical Eye Software, 29 Mar 1989.
%   Revised by Don Knuth, 3 Jan 1990.
%   Revised by Tomas Rokicki to accept bounding boxes with no
%      space after the colon, 18 Jul 1990.
%
%   TeX macros to include an Encapsulated PostScript graphic.
%   Works by finding the bounding box comment,
%   calculating the correct scale values, and inserting a vbox
%   of the appropriate size at the current position in the TeX document.
%
%   To use with the center environment of LaTeX, preface the \epsffile
%   call with a \leavevmode.  (LaTeX should probably supply this itself
%   for the center environment.)
%
%   To use, simply say
%   \input epsf           % somewhere early on in your TeX file
%   \epsfbox{filename.ps} % where you want to insert a vbox for a figure
%
%   Alternatively, you can type
%
%   \epsfbox[0 0 30 50]{filename.ps} % to supply your own BB
%
%   which will not read in the file, and will instead use the bounding
%   box you specify.
%
%   The effect will be to typeset the figure as a TeX box, at the
%   point of your \epsfbox command. By default, the graphic will have its
%   `natural' width (namely the width of its bounding box, as described
%   in filename.ps). The TeX box will have depth zero.
%
%   You can enlarge or reduce the figure by saying
%     \epsfxsize=<dimen> \epsfbox{filename.ps}
%   (or
%     \epsfysize=<dimen> \epsfbox{filename.ps})
%   instead. Then the width of the TeX box will be \epsfxsize and its
%   height will be scaled proportionately (or the height will be
%   \epsfysize and its width will be scaled proportiontally).  The
%   width (and height) is restored to zero after each use.
%
%   A more general facility for sizing is available by defining the
%   \epsfsize macro.    Normally you can redefine this macro
%   to do almost anything.  The first parameter is the natural x size of
%   the PostScript graphic, the second parameter is the natural y size
%   of the PostScript graphic.  It must return the xsize to use, or 0 if
%   natural scaling is to be used.  Common uses include:
%
%      \epsfxsize  % just leave the old value alone
%      0pt         % use the natural sizes
%      #1          % use the natural sizes
%      \hsize      % scale to full width
%      0.5#1       % scale to 50% of natural size
%      \ifnum#1>\hsize\hsize\else#1\fi  % smaller of natural, hsize
%
%   If you want TeX to report the size of the figure (as a message
%   on your terminal when it processes each figure), say `\epsfverbosetrue'.
%
\newread\epsffilein    % file to \read
\newif\ifepsffileok    % continue looking for the bounding box?
\newif\ifepsfbbfound   % success?
\newif\ifepsfverbose   % report what you're making?
\newdimen\epsfxsize    % horizontal size after scaling
\newdimen\epsfysize    % vertical size after scaling
\newdimen\epsftsize    % horizontal size before scaling
\newdimen\epsfrsize    % vertical size before scaling
\newdimen\epsftmp      % register for arithmetic manipulation
\newdimen\pspoints     % conversion factor
%
\pspoints=1bp          % Adobe points are `big'
\epsfxsize=0pt         % Default value, means `use natural size'
\epsfysize=0pt         % ditto
%
\def\epsfbox#1{\global\def\epsfllx{72}\global\def\epsflly{72}%
   \global\def\epsfurx{540}\global\def\epsfury{720}%
   \def\lbracket{[}\def\testit{#1}\ifx\testit\lbracket
   \let\next=\epsfgetlitbb\else\let\next=\epsfnormal\fi\next{#1}}%
%
\def\epsfgetlitbb#1#2 #3 #4 #5]#6{\epsfgrab #2 #3 #4 #5 .\\%
   \epsfsetgraph{#6}}%
%
\def\epsfnormal#1{\epsfgetbb{#1}\epsfsetgraph{#1}}%
%
\def\epsfgetbb#1{%
%
%   The first thing we need to do is to open the
%   PostScript file, if possible.
%
\openin\epsffilein=#1
\ifeof\epsffilein\errmessage{I couldn't open #1, will ignore it}\else
%
%   Okay, we got it. Now we'll scan lines until we find one that doesn't
%   start with %. We're looking for the bounding box comment.
%
   {\epsffileoktrue \chardef\other=12
    \def\do##1{\catcode`##1=\other}\dospecials \catcode`\ =10
    \loop
       \read\epsffilein to \epsffileline
       \ifeof\epsffilein\epsffileokfalse\else
%
%   We check to see if the first character is a % sign;
%   if not, we stop reading (unless the line was entirely blank);
%   if so, we look further and stop only if the line begins with
%   `%%BoundingBox:'.
%
          \expandafter\epsfaux\epsffileline:. \\%
       \fi
   \ifepsffileok\repeat
   \ifepsfbbfound\else
    \ifepsfverbose\message{No bounding box comment in #1; using defaults}\fi\fi
   }\closein\epsffilein\fi}%
%
%   Now we have to calculate the scale and offset values to use.
%   First we compute the natural sizes.
%
\def\epsfclipstring{}% do we clip or not?  If so,
\def\epsfclipon{\def\epsfclipstring{ clip}}%
\def\epsfclipoff{\def\epsfclipstring{}}%
%
\def\epsfsetgraph#1{%
   \epsfrsize=\epsfury\pspoints
   \advance\epsfrsize by-\epsflly\pspoints
   \epsftsize=\epsfurx\pspoints
   \advance\epsftsize by-\epsfllx\pspoints
%
%   If `epsfxsize' is 0, we default to the natural size of the picture.
%   Otherwise we scale the graph to be \epsfxsize wide.
%
   \epsfxsize\epsfsize\epsftsize\epsfrsize
   \ifnum\epsfxsize=0 \ifnum\epsfysize=0
      \epsfxsize=\epsftsize \epsfysize=\epsfrsize
      \epsfrsize=0pt
%
%   We have a sticky problem here:  TeX doesn't do floating point arithmetic!
%   Our goal is to compute y = rx/t. The following loop does this reasonably
%   fast, with an error of at most about 16 sp (about 1/4000 pt).
% 
     \else\epsftmp=\epsftsize \divide\epsftmp\epsfrsize
       \epsfxsize=\epsfysize \multiply\epsfxsize\epsftmp
       \multiply\epsftmp\epsfrsize \advance\epsftsize-\epsftmp
       \epsftmp=\epsfysize
       \loop \advance\epsftsize\epsftsize \divide\epsftmp 2
       \ifnum\epsftmp>0
          \ifnum\epsftsize<\epsfrsize\else
             \advance\epsftsize-\epsfrsize \advance\epsfxsize\epsftmp \fi
       \repeat
       \epsfrsize=0pt
     \fi
   \else \ifnum\epsfysize=0
     \epsftmp=\epsfrsize \divide\epsftmp\epsftsize
     \epsfysize=\epsfxsize \multiply\epsfysize\epsftmp   
     \multiply\epsftmp\epsftsize \advance\epsfrsize-\epsftmp
     \epsftmp=\epsfxsize
     \loop \advance\epsfrsize\epsfrsize \divide\epsftmp 2
     \ifnum\epsftmp>0
        \ifnum\epsfrsize<\epsftsize\else
           \advance\epsfrsize-\epsftsize \advance\epsfysize\epsftmp \fi
     \repeat
     \epsfrsize=0pt
    \else
     \epsfrsize=\epsfysize
    \fi
   \fi
%
%  Finally, we make the vbox and stick in a \special that dvips can parse.
%
   \ifepsfverbose\message{#1: width=\the\epsfxsize, height=\the\epsfysize}\fi
   \epsftmp=10\epsfxsize \divide\epsftmp\pspoints
   \vbox to\epsfysize{\vfil\hbox to\epsfxsize{%
      \ifnum\epsfrsize=0\relax
        \special{PSfile=#1 llx=\epsfllx\space lly=\epsflly\space
            urx=\epsfurx\space ury=\epsfury\space rwi=\number\epsftmp
            \epsfclipstring}%
      \else
        \epsfrsize=10\epsfysize \divide\epsfrsize\pspoints
        \special{PSfile=#1 llx=\epsfllx\space lly=\epsflly\space
            urx=\epsfurx\space ury=\epsfury\space rwi=\number\epsftmp\space
            rhi=\number\epsfrsize \epsfclipstring}%
      \fi
      \hfil}}%
\global\epsfxsize=0pt\global\epsfysize=0pt}%
%
%   We still need to define the tricky \epsfaux macro. This requires
%   a couple of magic constants for comparison purposes.
%
{\catcode`\%=12 \global\let\epsfpercent=%\global\def\epsfbblit{%BoundingBox}}%
%
%   So we're ready to check for `%BoundingBox:' and to grab the
%   values if they are found.
%
\long\def\epsfaux#1#2:#3\\{\ifx#1\epsfpercent
   \def\testit{#2}\ifx\testit\epsfbblit
      \epsfgrab #3 . . . \\%
      \epsffileokfalse
      \global\epsfbbfoundtrue
   \fi\else\ifx#1\par\else\epsffileokfalse\fi\fi}%
%
%   Here we grab the values and stuff them in the appropriate definitions.
%
\def\epsfempty{}%
\def\epsfgrab #1 #2 #3 #4 #5\\{%
\global\def\epsfllx{#1}\ifx\epsfllx\epsfempty
      \epsfgrab #2 #3 #4 #5 .\\\else
   \global\def\epsflly{#2}%
   \global\def\epsfurx{#3}\global\def\epsfury{#4}\fi}%
%
%   We default the epsfsize macro.
%
\def\epsfsize#1#2{\epsfxsize}
%
%   Finally, another definition for compatibility with older macros.
%
\let\epsffile=\epsfbox

\usepackage{epsf}
\oddsidemargin 0.25in \evensidemargin 0.25in
\topmargin 0.0in
\textwidth 6.5in \textheight 9in
\headheight 0.18in \footskip 0.16in
\leftmargin -0.5in \rightmargin -0.5in

\newcommand{\fig}[1]{figures/#1}
\newcommand{\pfig}[1]{\epsfbox{\fig{#1}}}
\newcommand{\newfig}[1]{\epsffile{\fig{#1}}}

\newcommand{\fdaelement}[1]{elements/#1}
\newcommand{\spiceelement}[1]{equivalent_spice_elements/#1}

\newcommand{\FREEDA}{{\Huge{\textsl{\textsf{f}}}${\mathsf{REEDA}}^{{\tiny{\textsf{TM}}}}$}}
\newcommand{\FDA}{{\textsl{\textsf{f}}}${\mathsf{REEDA}}^{{\tiny{\textsf{TM}}}}$}
\newcommand{\notforsspice}[1]{#1}
\newcommand{\spicetwoonly}[1]{#1}
\newcommand{\spicethreeonly}[1]{#1}
\newcommand{\pspiceonly}[1]{#1}
\newcommand{\pspiceninetytwoonly}[1]{#1}
\newcommand{\sspiceonly}[1]{}
\newcommand{\fornutmeg}[1]{}
\newcommand{\sspicetwoonly}[1]{}

%%%%%%%%%%%%%%%%%%%%%%%%%%%%%%%%%%%%%%%%%%%%%%%%%%%%%%%%%%%%%%%%%%%%%%%%%%%%%%%%
\marginparwidth=\mymargin
\marginparsep=\mymarginparsep
\newcommand{\textplusmarginparwidth}{\textwidth+\marginparsep+\mymargin}
\newcommand{\X}{\\ \hline}    % line terminataion in keyword environment
\newcommand{\B}{{ \rm [}}     % begin optional parameter in \form{}
\newcommand{\E}{{\ \rm\hspace{-0.04in}] }}   % end optional parameter in \form{}
\newcommand{\expr}{{\sc Expressions supported}}
\newcommand{\reqd}{{\scriptsize REQUIRED}}
\newcommand{\omitted}{{\scriptsize OMITTED}}
\newcommand{\inferred}{{\scriptsize INFERRED}}
\newcommand{\para}{\newline{\scriptsize (PARASITIC)}}
\newcommand{\opt}{{\tiny  OPTIONAL}}

\newcommand{\Spice}{{\sc Spice}}
\newcommand{\spice}{{\sc Spice}}
\newcommand{\justspice}{{\sc Spice}}
\newcommand{\nutmeg}{{\sc NUTMEG}}
\newcommand{\probe}{{\sc Probe}}

\newcommand{\accusim}{{\sc AccuSim}}
\newcommand{\contec}{{\sc ContecSpice}}
\newcommand{\cdsspice}{{\sc CDS Spice}}
\newcommand{\hpimpulse}{{\sc HP Impulse}}
\newcommand{\hspice}{{\sc HSpice}}
\newcommand{\igspice}{{\sc IG\_SPICE}}
\newcommand{\isspice}{{\sc IsSpice}}
\newcommand{\mspice}{{\sc Microwave Spice}}
\newcommand{\pspice}{{\sc PSpice}}
\newcommand{\justpspice}{{\sc PSpice}}
\newcommand{\spectre}{{\sc Spectre}}
\newcommand{\spicetwo}{{\sc Spice2g6}}
\newcommand{\spicethree}{{\sc Spice3}}
\newcommand{\spiceplus}{{\sc SpicePlus}}
\newcommand{\sspice}{{\sc SuperSpice}}

\newcommand{\modelversion}[1]{& #1}
%%%%%%%%%%%%%%%%%%%%%%%%%%%%%%%%%%%%%%%%%%%%%%%%%%%%%%%%%%%%%%%%%%%%%%%%%%%%%%%%
% set up a counter for all occasions
%
\newcounter{count}

% set up new commands
%
\newcommand{\vshift}{\vspace{0.2in}}
% OFFSET
\newcommand{\offset}{\hspace*{0.45in} }
% OFFSETPARBOXWIDTH argument should be \textwith - offset
\newcommand{\offsetparbox}[1]{\parbox[t]{5in}{#1}}

% note: no labeling
\newcommand{\elementx}[3]{\clearpage\rm\markright{#3:#2}
\addcontentsline{toc}{section}{#1, #2: #3}
\mymarginparx{#1}{#2}{#3}\index{#1:#2}\index{#3:#2}}

% macros for sub elements
\newcommand{\subelement}[2]{\clearpage 
   \noindent\rule{\textwidth /2}{.5mm} \\[0.1in]
   {\large \bf #1} \hspace{0.2in} {\bf #2} \\
   \noindent\rule{\textwidth /2}{.5mm} \index{#1} \index{#2}}
% macros for models
\newcommand{\model}[2]{{
   \noindent\vspace{0.2in}\parbox{\textwidth}{
   \noindent\rule{\textwidth}{.5mm} \\[0.1in]
   {\large \bf #1 Model} \label{#1model} \hfill {\large #2} \\
   \noindent\rule{\textwidth}{.5mm} \index{#1} \index{#2}}}}
\newcommand{\modelx}[3]{{
   \noindent\vspace{0.2in}\parbox{\textwidth}{
   \noindent\rule{\textwidth}{.5mm} \\[0.1in]
   {\large \bf #1 Model} \hfill {\large #2} \hfill {\large #3} \\
   \noindent\rule{\textwidth}{.5mm} \index{#1} \index{#2}}}}



% BOXED
\newcommand{\boxed}[1]{\noindent
\newline \vshift \hspace*{\fill} {\tt \offsetparbox{\tt #1}}
 \vshift}


%
% KEYWORD
%
\newcommand{\keywordtable}[1]{
        \sloppy
        \hyphenation{ca-pac-i-t-an-ce} 
        \begin{center}
    \sf
        \begin{tabular}[t]
        {|p{0.58in}|p{3.07in}|p{0.55in}|p{0.60in}|}
        \hline
        \multicolumn{1}{|c}{\bf Name} &
        \multicolumn{1}{|c}{\parbox{2.77in}{\bf Description}}  &
        \multicolumn{1}{|c}{\bf Units} &
        \multicolumn{1}{|c|}{\bf Default} \X
        #1
        \end{tabular}
        \end{center}
    }

\newcommand{\keywordtwotable}[2]{
        \sloppy
        \hyphenation{ca-pac-i-t-an-ce} 
        \begin{center}
    \sf
        \begin{tabular}[t]
        {|p{0.58in}|p{2.38in}|p{0.55in}|p{0.60in}|p{0.53in}|}
        \hline
        \multicolumn{1}{|c}{\bf Name} &
        \multicolumn{1}{|c}{\parbox{2.20in}{\bf Description}}  &
        \multicolumn{1}{|c}{\bf Units} &
        \multicolumn{1}{|c}{\bf Default} &
        \multicolumn{1}{|c|}{\bf #1} \X
        #2
        \end{tabular}
        \end{center}
    }

\newcommand{\kw}[2]{
     \samepage{
     \noindent {\sl #1} \vspace{-0.5in} \\
     \keywordtable{#2} }}

\newcommand{\kwtwo}[3]{
     \samepage{
     \noindent {\sl #1} \vspace{-0.4in} \\
     \keywordtwotable{#2}{#3} }}

\newcommand{\keyword}[1]{\kw{Keywords:}{#1}}
\newcommand{\keywordtwo}[2]{\kwtwo{Keywords:}{#1}{#2}}
\newcommand{\modelkeyword}[1]{\kw{Model Keywords}{#1}}
\newcommand{\modelkeywordtwo}[2]{\kwtwo{Model Keywords}{#1}{#2}}

\newcommand{\myline}{\\[-0.1in]
\noindent \rule{\textwidth}{0.01in} \newline}

\newcommand{\myThickLine}{\\[-0.1in]
\noindent \rule{\textwidth}{0.02in} \newline}


% SPICE 2G6 FORM
%\newcommand{\spicetwoform}[1]{
%\spicetwoonly{\samepage{\noindent{\sl\spicetwo\form{#1}}}}}

% PSPICE88 FORM
%\newcommand{\pspiceform}[1]{
%\pspiceonly{\samepage{\noindent{\sl\pspice}\form{#1}}}}

% PSPICE92 FORM
%\newcommand{\pspiceninetytwoform}[1]{
%\pspiceninetytwoonly{\samepage{\noindent{\sl\pspice92\form{#1}}}}}


% SPICE3E2 FORM
%\newcommand{\spicethreeform}[1]{
%\spicethreeonly{\samepage{\noindent{\sl\spicethree\form{#1}}}}}

% FORM
\newcommand{\form}[1]{\samepage{\noindent
 {\sl Form} \myline
% \hspace*{\fill} % For some reason \fill = 0 when \pspiceform{} is used?
\offset
\it  \offsetparbox{#1}}
\\[0.1in]}

% ELEMENT FORM
\newcommand{\elementform}[1]{\samepage{\noindent
 {\sl Element Form} \myline
% \hspace*{\fill} % For some reason \fill = 0 when \pspiceform{} is used?
\offset
\it  \offsetparbox{#1}}
\\[0.1in]}

% MODEL FORM
\newcommand{\modelform}[1]{\samepage{\noindent
 {\sl Model Form} \myline
% \hspace*{\fill} % For some reason \fill = 0 when \pspiceform{} is used?
\offset
\it  \offsetparbox{#1}}
\\[0.1in]}

% LIMITS
\newcommand{\mylimits}[1]{\samepage{\noindent
 {\sl Limits} \myline
 \hspace*{\fill} \it  \offsetparbox{#1}}
 \vshift}

% EXAMPLE
\newcommand{\example}[1]{\samepage{\noindent
{\sl Example} \myline
\offset \tt  \offsetparbox{#1}}
 \vshift}

% PSPICE88 EXAMPLE
\newcommand{\pspiceexample}[1]{\samepage{\noindent
{\sl \pspice\ Example} \myline
\offset \tt  \offsetparbox{#1}}
 \vshift}

% MODEL TYPES
\newcommand{\modeltype}[1]{\samepage{\noindent
{\sl Model Type} \myline
 \hspace*{\fill} \tt \offsetparbox{#1}}
 \\[0.1in]}

% MODEL TYPES
\newcommand{\modeltypes}[1]{\samepage{\noindent
{\sl Model Types:} \myline
 \hspace*{\fill} \tt \offsetparbox{\tt #1}}
 \vshift}

% OFFSET ENUMERATE
\newcommand{\offsetenumerate}[1]{
     \offset \hspace*{-0.1in} {\begin{enumerate} #1 \end{enumerate}}}

% NOTE
\newcommand{\note}[1]{
\vshift\samepage{\noindent {\sl Note}\myline\vspace{-0.24in}}
 \offsetenumerate{#1} }

% SPECIAL NOTE
\newcommand{\specialnote}[2]{
\vshift\samepage{\noindent {\sl #1}\myline\vspace{-0.24in}}\\#2}

\newcommand{\dc}{\mbox{\tt DC}}
\newcommand{\ac}{\mbox{\tt AC}}
\newcommand{\SPICE}{\mbox{\tt SPICE}}
\newcommand{\m}[1]{{\bf #1}}                           % matrix command  \m{}

% ////// Changing nodes to terminals///////
% print terminals in \tt and enclose in a circle use outside
\newcommand{\terminal}[1]{\: \mbox{\tt #1} \!\!\!\! \bigcirc }
%
% set up environment for example
%
\newcounter{excount}
\newcounter{dummy}
\newenvironment{eg}{\vspace{0.1in}\noindent\rule{\textwidth}{.5mm}
   \begin{list}
   {{\addtocounter{excount}{1}
   \em Example\/ \arabic{chapter}.\arabic{excount}\/}:}
   {\usecounter{dummy}
   \setlength{\rightmargin}{\leftmargin}}
   }{\end{list} \rule{\textwidth}{.5mm}\vspace{0.1in}}
%
% set up environment for block
% currently this draws a horizontal line at the start of block and another
% at the end of block.
%
\newenvironment{block}{\vspace{0.1in}\noindent\rule{\textwidth}{.5mm}
   }{\rule{\textwidth}{.5mm}\vspace{0.1in}}
%

%
% Macros for element summaries
%
% macros for element summary
%\newcommand{\summaryelement}[2]{
%   \vspace{0.4in}
%   \mymarginpar{#1}{#2} 
%   \addcontentsline{toc}{section}{#1, #2}
%   \vspace{-0.6in} \\
%   \noindent Full description on page \pageref{#1element}. \vshift\\
%   }

% Summary MODEL TYPE
%\newcommand{\summarymodeltype}[1]{\samepage{\noindent
%{\sl Model Type} \myline
% \hspace*{\fill} \tt \offsetparbox{\tt #1
% \hfill Summary on \pageref{#1summary} \index{#1}}}
% }

% Summary MODEL TYPES
%\newcommand{\summarymodeltypes}[1]{\samepage{\noindent
%{\sl Model Types} \myline
% \hspace*{\fill} \tt \offsetparbox{\tt #1
% \hfill Summary on \pageref{#1summary} \index{#1}}}
% }

%
% macros for model summary
%\newcommand{\summarymodel}[2]{\clearpage
%   \addcontentsline{toc}{section}{#1, #2}
%   \mymarginpar{#1}{#2} \label{#1summary}
%   \index{#1} \index{#2}
%   \noindent Full description on page \pageref{#1model} \\[0.1in]
%   }



%
% set up wide descriptive list
%
\newenvironment{widelist}
    {\begin{list}{}{\setlength{\rightmargin}{0in} \setlength{\itemsep}{0.1in}
    \setlength{\labelwidth}{0.95in} \setlength{\labelsep}{0.1in}
\setlength{\listparindent}{0in} \setlength{\parsep}{0in}
    \setlength{\leftmargin}{1.0in}}
    }{\end{list}}

\newcommand{\STAR}{\hspace*{\fill} * \hspace*{\fill}}

\newcommand{\sym}[1]{\hspace*{\fill} ($#1$)}

%
% The thebibliography environment is redefined so the the word References is
% not output
%\def\thebibliography#1{\list
% {[\arabic{enumi}]}{\settowidth\labelwidth{[#1]}\leftmargin\labelwidth
% \advance\leftmargin\labelsep
% \usecounter{enumi}}
% \def\newblock{\hskip .11em plus .33em minus -.07em}
% \sloppy\clubpenalty4000\widowpenalty4000
% \sfcode`\.=1000\relax}
%\let\endthebibliography=\endlist
% END thebibliography environment redefinition



%\newcommand{\eskipv}[1]{\clearpage\marginlabel{#1}}
%\newcommand{\eskip}[1]{\vspace*{\fill}\clearpage\marginlabel{#1}}
%\newcommand{\eskipnv}[1]{\newpage\marginlabel{#1}}
%\newcommand{\eskipn}[1]{\vspace*{\fill}\newpage\marginlabel{#1}}

%marginlabel is very wide
%\newcommand{\eskipfullv}[1]{\clearpage\marginlabelfull{#1}}
%\newcommand{\eskipfull}[1]{\vspace*{\fill}\clearpage\marginlabelfull{#1}}
%\newcommand{\eskipfullnv}[1]{\newpage\marginlabelfull{#1}}
%\newcommand{\eskipfulln}[1]{\vspace*{\fill}\newpage\marginlabelfull{#1}}

%\newcommand{\mycontentsline}[5]{\parbox{#1}{#2}#3
%\hspace{0.1in}\dotfill\parbox{0.3in}{\hfill\pageref{#4#5}}\\[0.1in]}
%\newcommand{\mysline}[2]{\mycontentsline{1.2in}{#1}{#2}{#1}{statement}}
%\newcommand{\mymline}[2]{\mycontentsline{0.7in}{#1}{#2}{#1}{model}}
%\newcommand{\myeline}[2]{\mycontentsline{0.7in}{#1}{#2}{#1}{element}}

%\newcommand{\myincontentsline}[5]{\vspace{0.05in}\noindent\parbox{#1}{#2}#3
%\hspace{0.1in}
%\dotfill\parbox{0.7in}{\hfill Page \pageref{#4#5}}\\[0.05in]\noindent}
%\newcommand{\myinsline}[2]{\myincontentsline{1.2in}{#1}{#2}{#1}{statement}}
%\newcommand{\myinmline}[2]{\myincontentsline{0.5in}{#1}{#2}{#1}{model}}
%\newcommand{\myineline}[2]{\myincontentsline{0.5in}{#1}{#2}{#1}{element}}

%
%
% The following a symbols that could used alot.
\newcommand{\ms}[1]{\mbox{\scriptsize #1}}
\newcommand{\AF}{A_F}
\newcommand{\CBD}{C'_{BD}}
\newcommand{\CBS}{C'_{BS}}
\newcommand{\CGBO}{C_{GBO}}
\newcommand{\CGDO}{C_{GDO}}
\newcommand{\CGSO}{C_{GSO}}
\newcommand{\CJ}{C_J}
\newcommand{\CJSW}{C_{J,\ms{SW}}}
\newcommand{\DELTA}{\delta}
\newcommand{\ETA}{\eta}
\newcommand{\FC}{F_C}
\newcommand{\GAMMA}{\gamma}
\newcommand{\IS}{I_S}
\newcommand{\JS}{J_S}
\newcommand{\KAPPA}{\kappa}
\newcommand{\KF}{K_F}
\newcommand{\KP}{K_P}
\newcommand{\LAMBDA}{\lambda}
\newcommand{\LD}{X_{JL}}
\newcommand{\LEVEL}{M_J}
\newcommand{\MJ}{M_J}
\newcommand{\MJSW}{M_{J,\ms{SW}}}
\newcommand{\NSUB}{N_B}
\newcommand{\NSS}{N_{\ms{SS}}}
\newcommand{\NFS}{N_{\ms{FS}}}
\newcommand{\NEFF}{N_{\ms{EFF}}}
\newcommand{\PB}{\phi_J}
\newcommand{\PHI}{2\phi_B}
\newcommand{\RD}{R_D}
\newcommand{\RS}{R_S}
\newcommand{\RSH}{R_{\ms{SH}}}
\newcommand{\THETA}{\theta}
\newcommand{\TOX}{T_{OX}}
\newcommand{\TPG}{T_{\ms{PG}}}
\newcommand{\UCRIT}{U_C}
\newcommand{\UEXP}{U_{\ms{EXP}}}
\newcommand{\UO}{\mu_0}
\newcommand{\UTRA}{U_{\ms{TRA}}}
\newcommand{\VMAX}{V_{\ms{MAX}}}
\newcommand{\VTZERO}{V_{T0}}
\newcommand{\VTO}{V_{T0}}
\newcommand{\XJ}{X_J}
\newcommand{\Length}{L} %  \L already used
\newcommand{\N}{N}
\newcommand{\PBSW}{\phi_{J,{\ms{SW}}}}
\newcommand{\RB}{R_B}
\newcommand{\RG}{R_B}
\newcommand{\RDS}{R_{DS}}
\newcommand{\TT}{\tau_T}
\newcommand{\W}{W}
\newcommand{\WD}{W_D}
\newcommand{\XQC}{X_{QC}}
\newcommand{\JSSW}{J_{S,{\ms{SW}}}}
\newcommand{\DL}{\Delta_L}
\newcommand{\DW}{\Delta_W}
\newcommand{\DELL}{\Delta_{L,\ms{SW}}}
\newcommand{\KONE}{K_1}
\newcommand{\KTWO}{K_2}
\newcommand{\MUS}{\mu_S}
\newcommand{\MUZ}{\mu_Z}
\newcommand{\NZERO}{N_0}
\newcommand{\NB}{N_B}
\newcommand{\ND}{N_D}
\newcommand{\TEMP}{T}
\newcommand{\VDD}{V_{DD}}
\newcommand{\WDF}{W_{\ms{DF}}}
\newcommand{\VFB}{V_{\ms{FB}}}
\newcommand{\UZERO}{U_0}
\newcommand{\UONE}{U_1}
\newcommand{\XTWOE}{X_{2E}}
\newcommand{\XTWOMS}{X_{2\ms{MS}}}
\newcommand{\XTWOMZ}{X_{2\ms{MZ}}}
\newcommand{\XTWOUZERO}{X_{2\ms{U0}}}
\newcommand{\XTWOUONE}{X_{2\ms{U1}}}
\newcommand{\XTHREEE}{X_{3E}}
\newcommand{\XTHREEMS}{X_{3\ms{MS}}}
\newcommand{\XTHREEMZ}{X_{3\ms{MZ}}}
\newcommand{\XTHREEUZERO}{X_{3\ms{U0}}}
\newcommand{\XTHREEUONE}{X_{3\ms{U1}}}
\newcommand{\XPART}{X_{\ms{PART}}}
\newcommand{\PS}{P_S}
\newcommand{\PD}{P_D}
\newcommand{\NRS}{N_{RS}}
\newcommand{\NRG}{N_{RG}}
\newcommand{\NRB}{N_{RB}}
\newcommand{\NRD}{N_{RD}}


\newcommand{\Net}{{${\cal N}$}}                          % network \N
\newcommand{\Nprime}{{${\cal N}^{\prime}$}}            % another network \Nprime
\newcommand{\Nold}{{${\cal N}^{\mbox{old}}$}}          % old network  \Nold
\newcommand{\Nnew}{{${\cal N}^{\mbox{new}}$}}          % new network  \Nnew

\newcommand{\GMIN}{{G_{\ms{MIN}}}}

\newcommand{\optionitem}[2]{
\item[{\tt #1}{#2}]\label{.OPTION#1}\index{.OPTIONS, #1}\index{#1}}

\newcommand{\error}[1]{\vspace{0.1in}\noindent{\tt #1}\\}


%For numbering an equation which is incoorporated
%with text.
\newcommand{\inlineeq}{\hspace*{\fill}\refstepcounter{equation}{\rm
(\theequation)}\\}

\begin{document}
\noindent{\LARGE \textbf{Bipolar Junction Transistor}
\hspace{\fill}\textbf{bjtpnp}}
\hrulefill\linethickness{0.5mm}\line(1,0){425} \normalsize
\newline
% the bjt figure
\begin{figure}[h]
\centerline{\epsfxsize=4in\pfig{q1.eps}} \caption[Q --- bipolar
junction transistor element]{Q --- Bipolar Polar Junction
Transistor: (a) NPN transistor; (b) PNP transistor.}
\end{figure}
\newline
% form for Transim
\linethickness{0.5mm}\line(1,0){425}
\newline
\textit{\FDA Form:}
%\newline
$\tt bjtpnp$:$\langle \tt{instance\ name}\rangle$ $n_1\ n_2\ n_3\
$ $\langle \tt{parameter\ list}\rangle$
\newline
% explanation of the parameters
\begin{tabular}{r l}
$n_1$ & is the base node \\
&  \\
$n_2$ & is the collector node \\
&  \\
$n_3$ & is the emitter node \\
& \\
%parameter list & see table 1 for parameter list
\end{tabular}
% SPICE form
\newline
\noindent\texttt{SPICE} \textit{Form:}
%\newline
%\linethickness{0.5mm}
%\line(1,0){425}
\newline
{\tt Q}name  NCollector NBase NEmitter  [NSubstrate]  ModelName
[Area] [{\tt OFF}] {\tt} [{\tt IC=}Vbe,Vce] \\
where \\
\begin{tabular}{r l}
{\it NCollector} & is the collector node. \\
& \\
{\it NBase} & is the base node. \\
& \\
{\it NEmitter} & is the emitter node. \\
& \\
{\it NSubstrate} & is the optional substrate node. If not specified, then the ground is used as the \\
& substrate node. If {\it NSubstrate} is a name as allowed in it must be enclosed in \\
& square brackets, e.g. {\tt [{\it NSubstrate}]}, to distinguish it from {\it ModelName}. \\
& \\
{\it ModelName} & is  the  model  name. \\
& \\
{\it Area} & is  the  area  factor. If the area  factor  is  omitted,  a  value of 1.0 is assumed. \\
& (Units: none; Optional; Default: 1; Symbol: $Area$) \\
& \\
{\tt OFF} & indicates an (optional) initial condition on the device for the dc\ analysis. If \\
& specified the dc\ operating point is calculated with the terminal voltages set to \\
& zero. Once convergence is obtained, the program continues to iterate to obtain \\
& the exact value of the  terminal  voltages.  The OFF option is used to enforce \\
& the solution to  correspond to  a  desired state if the circuit has \\
& more than one stable state. \\
& \\
{\tt IC} & is the optional initial condition specification using  {\tt IC=}$V_{BE},V_{CE}$ is  intended \\
& for use with the {\tt UIC} option on the {\tt .TRAN} line, when a transient analysis is \\
& desired  starting  from  other than  the  quiescent  operating  point. See  the {\it .IC} \\
& line description for a better way to set transient initial conditions.
\end{tabular}
\newpage
%\vspace{4mm}
% example in SPICE
%\newline
\noindent\linethickness{0.5mm}\line(1,0){425}
\newline
\noindent\textit{Example:}
\newline
%\newline
bjtpnp:Q20 10 50 0\\
bjtpnp:QFAST IC=0.65,15.0 \\
bjtpnp:Q5PUSH 10 29 14 200 MODEL1
% Parameter table
%\newpage
\newline
\linethickness{0.5mm}\line(1,0){425}

\noindent\textit{Model Parameters:}\\
\newline
\begin{tabular}{|r|l|c|c|} \hline
\textbf{Name} & \textbf{Description} & \textbf{Units} & \textbf{Default} \\
\hline
\texttt{AREA} & Current multiplier & & 1.0 \\
\hline
\texttt{BF} & Ideal maximum forward beta ($B_F$) & & 100.0 \\
\hline
\texttt{BR} & Ideal maximum reverse beta ($B_R$) & & 1.0  \\
\hline
\texttt{C2} & Base-emitter leakage saturation coefficient & & $I_{SE} / I_S$ \\
\hline
\texttt{C4} & Base-collector leakage saturation coefficient & & ($I_{SC} / I_S$) \\
\hline
\texttt{CJC} & Base collector zero bias p-n capacitance ($C_{JC}$) & F &  0.0\\
\hline
\texttt{CJE} & Base emitter zero bias p-n capacitance ($C_{JE}$) & F & 0.0\\
\hline
\texttt{EG} & Bandgap voltage ($E_G$) & eV & 1.11 \\
\hline
\texttt{FC} & Forward bias depletion capacitor coefficient ($F_C$) & & 0.5 \\
\hline
\texttt{IKF} & Corner of forward beta high-current roll-off ($I_{KF}$) & A & $10^{-10}$ \\
\hline
\texttt{IKR} & Corner for reverse-beta high current roll off ($I_{KR}$) & & $10^{-10}$ \\
\hline
\texttt{IS} & Transport saturation current ($I_S$) & A & $10^{-16}$  \\
\hline
\texttt{ISC} & Base collector leakage saturation current ($I_{SC}$) & A & 0.0 \\
\hline
\texttt{ISE} & Base-emitter leakage saturation current ($I_{SE}$) & A & 0.0\\
\hline
\texttt{IRB} & Current at which \texttt{RB} falls to half of \texttt{$R_{BM}$} ($I_{RB}$) & A & $10^{-10}$ \\
\hline
\texttt{ITF} & Transit time dependency on \texttt{IC} ($I_{TF}$) & A & 0.0 \\
\hline
\texttt{MJC} & Base collector p-n grading factor ($M_{JC}$) & & 0.33 \\
\hline
\texttt{MJE} & Base emitter p-n grading factor ($M_{JE}$) & & 0.33 \\
\hline
\texttt{NC} & Base-collector leakage emission coefficient ($N_C$) & & 2.0\\
\hline
\texttt{NE} & Base-emitter leakage emission coefficient ($N_E$) &  & 1.5 \\
\hline
\texttt{NF} & Forward current emission coefficient ($N_F$) & & 1.0 \\
\hline
\texttt{NR} & Reverse current emission coefficient ($N_R$) & & 1.0 \\
\hline
\texttt{RB} & Zero bias base resistance ($R_B$) & $\Omega$ & 0.0 \\
\hline
\texttt{RBM} & Minimum base resistance ($R_{BM}$) & $\Omega$ & $R_B$ \\
\hline
\texttt{RE} & Emitter ohmic resistance ($R_E$) & $\Omega$ & 0.0 \\
\hline
\texttt{RC} & Collector ohmic resistance ($R_C$) & $\Omega$ &  0.0 \\
\hline
\texttt{T} & Operating Temperature $T$& K &  300 \\
\hline
\texttt{TF} & Ideal forward transit time ($T_S$) & secs & 0.0 \\
\hline
\texttt{TNOM} & Nominal temperature ($T_{NOM}$) & K & 300 \\
\hline
\texttt{TR} & Ideal reverse transit time ($T_R$) & S & 0.0 \\
\hline
\texttt{TRB1} & \texttt{RB} temperature coefficient (linear) ($T_{RB1}$) & & 0.0 \\
\hline
\texttt{TRB2} & \texttt{RB} temperature coefficient  (quadratic) ($T_{RB2}$) & & 0.0 \\
\hline
\texttt{TRC1} & \texttt{RC} temperature coefficient (linear) ($T_{RC1}$) & & 0.0 \\
\hline
\texttt{TRC2} & \texttt{RC} temperature coefficient (linear) ($T_{RC2}$) & & 0.0 \\
\hline
\texttt{TRE1} & \texttt{RE} temperature coefficient (linear) ($T_{RE1}$) & & 0.0 \\
\hline
\texttt{TRE2} & \texttt{RE} temperature coefficient (quadratic) ($T_{RE2}$) & & 0.0 \\
\hline
\texttt{TRM1} & \texttt{RBM} temperature coefficient (linear) ($T_{RM1}$) & & 0.0 \\
\hline
\texttt{TRM2} & \texttt{RBM} temperature coefficient (quadratic) ($T_{RM2}$) & & 0.0 \\
\hline
\end{tabular}

\begin{tabular}{|r|l|c|c|}
\hline
\textbf{Name} & \textbf{Description} & \textbf{Units} & \textbf{Default} \\
\hline
\texttt{VA} & alternative keyword for \texttt{VAF} ($V_A$) & V & $10^{-10}$ \\
\hline
\texttt{VAF} & Forward early voltage ($V_{AF}$) & V & $10^{-10}$ \\
\hline
\texttt{VAR} & Reverse early voltage ($V_{AR}$) & & $10^{-10}$ \\
\hline
\texttt{VB} & alternative keyword for \texttt{VAR} ($V_B$) & & $10^{-10}$ \\
\hline
\texttt{VJC} & Base collector built in potential ($V_{JC}$) & V & 0.75\\
\hline
\texttt{VJE} & Base emitter built in potential ($V_{JE}$) & V & 0.75 \\
\hline
\texttt{VTF} & Transit time dependency on \texttt{VBC} ($V_{TF}$) & V & $10^{-10}$ \\
\hline
\texttt{XCJC} & Fraction of \texttt{CBC} connected internal to \texttt{RB} ($X_{CJC}$) & & 1.0 \\
\hline
\texttt{XTB} & Forward and reverse beta temperature coefficient ($X_{TB}$) & & 0.0\\
\hline
\texttt{XTF} & Transit time bias dependence coefficient ($X_{TF}$) & & 0.0 \\
\hline
\texttt{XTI} & \texttt{IS} temperature effect exponent ($X_{TI}$) & & 3.0 \\
\hline
\end{tabular}
\newline
%\vspace{6mm}
\newpage
%\linethickness{0.5mm}
\textbf{ELEMENT Model}
\newline
%\line(1,0){425}
%\vspace{2mm}
%\newline
\begin{figure}[h]
\centerline{\epsfxsize=4in\pfig{bipolar1.eps}}
\caption{Schematic
of the BJT Model}
\end{figure}
\newline
%\vspace{5mm}
% Device equations start
\newpage
\noindent\underline{\large Standard Calculations}\\

The physical constants used in the model evaluation are
\begin{center}
\begin{tabular}{|l|l|l|}
\hline
$k$ & Boltzman's constant &  $1.3806226\,10^{-23}$~J/K\\
$q$ & electronic charge & $1.6021918\,10^{-19}$~C\\
\hline
\end{tabular}
\end{center}
Absolute temperatures (in kelvins, K) are used.
The thermal voltage
\begin{equation}
V_{TH}(T_{NOM}) = {{k\ T_{NOM}} / q} .
\end{equation}
\newline
%\vspace{6mm}
% Current charactersitics
\noindent\underline{Current Characteristics}\\

\noindent{The base-emitter current, }
\begin{equation}
I_{BE} = {{\textstyle I_{BF} } / {\textstyle \beta_F }} + I_{LE}
\end{equation}

\noindent{the base-collector current, }
\begin{equation}
I_{BC} = {{\textstyle I_{BR} } / {\textstyle \beta_R }} + I_{LC}
\end{equation}

\noindent{and the collector-emitter current,}
\begin{equation}
I_{CE} = {{\textstyle I_{BF} - I_{BR} } / {\textstyle K_{QB} }}
\end{equation}

\noindent{where the forward diffusion current,}
\begin{equation}
I_{BF} = I_S\left(e^{\textstyle V_{BE}/(N_F V_{TH}) - 1} \right)
\end{equation}

\noindent{the nonideal base-emitter current,}
\begin{equation}
I_{LE}=I_{SE}\left(e^{\textstyle V_{BE}/(N_E V_{TH}) - 1} \right)
\end{equation}

\noindent{the reverse diffusion current,}
\begin{equation}
I_{BR} = I_S\left(e^{\textstyle V_{BC}/(N_R V_{TH}) - 1} \right)
\end{equation}

\noindent{the non-ideal base-collector current,}
\begin{equation}
I_{LC}=I_{SC}\left(e^{\textstyle V_{BC}/(N_C V_{TH}) - 1} \right)
\end{equation}

\noindent{and the base charge factor,}
\begin{equation}
K_{QB} = {{\textstyle 1}/{2}} \left[1 -
{{\textstyle V_{BC}}/{\textstyle V_{AF}}}- {{\textstyle V_{BE}}/{\textstyle V_{AB}}}
    \right]^{-1} \left(1 + \sqrt{1 + 4\left(
        {{\textstyle I_{BF}}/{\textstyle I_{KF}}}+
         {{\textstyle I_{BR}}/{\textstyle I_{KR}}}
        \right)}\right)\\
\end{equation}

\noindent{Thus the conductive current flowing into the base,}
\begin{equation}
I_B = I_{BE}+I_{BC}
\end{equation}

\noindent{the conductive current flowing into the collector,}
\begin{equation}
I_C = I_{CE}-I_{BC}
\end{equation}

\noindent{and the conductive current flowing into the emitter,}
\begin{equation}
I_C = I_{BE}+I_{CE}
\end{equation}

%\vspace{0.1in}
% Capacitances
\noindent\underline{\large Capacitances}\\

$C_{BE} = Area( C_{BE\tau} + C_{BEJ})$
where the base-emitter transit time or diffusion capacitance

\begin{equation}
C_{BE\tau} = \tau_{F,EFF} {{\textstyle\partial I_{BF}} /
       {\textstyle\partial V_{BE}}}
\end{equation}

the effective base transit time is empirically modified to account for base
puchout, space-charge limited current flow, quasi-saturation and lateral
spreading which tend to increase $\tau_F$

\begin{equation}
\tau_{F,EFF} =\tau_F\left[ 1+X_{TF}(3x^2-2x^3)
     e^{\textstyle (V_{BC}/(1.44V_{TF})}\right]
\end{equation}
and $x = {I_{BF}}/(I_{BF} + Area I_{TF})$.


The base-emitter junction (depletion) capacitance
\begin{equation}
C_{BEJ} = \left\{ \!\! \begin{array}{ll}
C_{JE} \left(1-{{\textstyle V_{BE}} / {\textstyle V_{JE}}}\right)^{\textstyle -M_{JE}} & \! V_{BE} \le F_C V_{JE}\\
C_{JE} \left(1-F_C\right)^{\textstyle -(1+M_{JE})} \left(1-F_C(1+M_{JE})+M_{JE}{{\textstyle V_{BE}} / {\textstyle V_{JE}}} \right) & \! V_{BE} > F_C V_{JE}
                         \end{array}
         \right.
\end{equation}


The base-collector capacitance,
$C_{BC} = Area(C_{BC\tau} +X_{CJC} C_{BCJ})$
where the base-collector transit time or diffusion capacitance
\begin{equation}
C_{BC\tau} = \tau_R {{\textstyle\partial I_{BR}} /
       {\textstyle\partial V_{BC}}}
\end{equation}


The base-collector junction (depletion) capacitance
\begin{equation}
C_{BCJ} = \left\{ \! \! \begin{array}{ll}
C_{JC} \left(1-{{\textstyle V_{BC}}/{\textstyle V_{JC}}}\right)^{\textstyle -M_{JC}} & \! V_{BC} \le F_C V_{JC}\\
C_{JC} \left(1-F_C\right)^{\textstyle -(1+M_{JC})}\left(1-F_C(1+M_{JC})+M_{JC}{{\textstyle V_{BC}}/ {\textstyle V_{JC}}} \right) & \! V_{BC} > F_C V_{JC}
                       \end{array}
         \right.
\end{equation}



The capacitance between the extrinsic base and the intrinsic collector
\begin{equation}
C_{BX} = \left\{ \begin{array}{ll}
Area(1-X_{CJC}) C_{JC} \left(1-{{\textstyle V_{BX}} / {\textstyle V_{JC}}}\right)^{\textstyle -M_{JC}} & V_{BX} \le F_C V_{JC}\\ \\
(1-X_{CJC}) C_{JC} \left(1-F_C\right)^{\textstyle -(1+M_{JC})} & V_{BX} > F_C V_{JC}\\
\;\;\;\;\;\times\; \left(1-F_C(1+M_{JC})+M_{JC}{{\textstyle V_{BX}} / {\textstyle V_{JC}}} \right)  &
                \end{array}
        \right.
\end{equation}\\
%\vspace{6mm}
%The substrate junction capacitance
%\begin{equation}
%C_{JS} = \left\{ \begin{array}{ll}
%Area C_{JS} \left(1-{{\textstyle V_{CJS}} / {\textstyle V_{JS}}}\right)^{\textstyle -M_{JS}} & V_{CJS} \le 0\\
%Area C_{JS} \left(1+M_{JS}{{\textstyle V_{CJS}} / {\textstyle V_{JS}}} \right) & V_{CJS} > 0
%                   \end{array}
%       \right.
%\end{equation}

% Bugs
%\newline
\noindent \linethickness{0.5mm}\line(1,0){425}
\newline
\noindent\textit{Bugs:}\\
\newline
Parameters: OFF and IC are not functional.\\
\linethickness{0.5mm} \line(1,0){425}
\newline
\textit{Version:}\\
2002.09.01 \\
% Credits
\newline
\linethickness{0.5mm} \line(1,0){425}
\newline
\noindent\textit{Credits:}
\newline
\newline
\begin{tabular}{l l l l}
Name & Affiliation & Date &  \\
Senthil Velu & North Carolina State University & Sept 2002 & \epsfxsize=1in\pfig{logo.eps} \\
\end{tabular}
%Publications
\newline
\noindent \linethickness{0.5mm} \line(1,0){425}
\newline
\textit{Publications:}
\begin{enumerate}
\item C. Christoffersen, S. Velu and M. B. Steer, `` A Universal Parameterized Nonlinear
Device Model Formulation for Microwave Circuit Simulation,'' 2002
IEEE Int. Microwave Symp. Digest, June 2002, pp 2189-2192.
\end{enumerate}
\end{document}
